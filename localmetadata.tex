%% -*- coding:utf-8 -*-
\title{语法理论}
\subtitle{从转换语法到基于约束的理论}
%\title{Grammatical theory}
%\subtitle{From transformational grammar to constraint-based approaches}
\author{Stefan Müller}
%\author{斯特凡 $\cdot$ 米勒$\quad$著}
\typesetter{Stefan Müller}
%\typesetter{斯特凡 $\cdot$ 米勒}
\translator{王璐璐、孙薇薇、黄思思}
%\translator{Andrew Murphy, Stefan Müller}
\proofreader{%
陈榕、
霍安頔(Andreas Hölzl)、
林巧莉、
刘畅、
李姝姝、
练斐、
牛若晨、
史红改、
佟和龙、
屠爱萍、 
万姝君、
王小溪、
杨丰榕、
曾巧仪
}

\openreviewer{%
曹晓玉、
刘海涛、
刘晓、
卢达威、
王佳骏、
夏军、
詹卫东
}

\BackTitle{语法理论}
%\BackTitle{Grammatical theory}
\BackBody{
本书全面系统地探讨了在当代语言学发展进程中起到重要作用的形式语法理论的理论贡献,并深入地比较了这些理论假说在分析语言学的普遍问题时提出的观点与方法。全书共分为两个部分。第一部分系统地介绍了在理论语言学界具有重要地位的语法理论,包括短语结构语法、转换语法—管辖与约束理论、转换语法—最简方案、广义短语结构语法、词汇功能语法、范畴语法、中心语驱动的短语结构语法、构式语法、依存语法和树邻接语法。作者对重要的理论学说进行了深入浅出的讲解,并详细说明了每种理论是如何分析论元、主被动变换、局部重新排序、动词替换以及跨长距离依存的成分前置等问题的。
%  This book introduces formal grammar theories that play a role in current linguistic theorizing (Phrase Structure Grammar,
%  Transformational Grammar/Government \& Binding, Generalized Phrase Structure Grammar, Lexical
%  Functional Grammar, Categorial Grammar, Head-​Driven Phrase Structure Grammar, Construction
%  Grammar, Tree Adjoining Grammar). The key assumptions are explained and it is shown how the
%  respective theory treats arguments and adjuncts, the active/passive alternation, local
%  reorderings, verb placement, and fronting of constituents over long distances. The analyses are
%  explained with German as the object language.

第二部分重点评论了这些理论方法针对语言习得和心理语言学的可证伪性所做的理论假说。例如,天赋论假说假设了人类具有基因上决定的内在的具体语言的知识。本书对此观点进行了深度地评判,并对语言习得的其他模型也进行了讨论。该部分还探讨了理论构建中的备受争议的问题,如平铺结构还是二叉结构更为合适?构式是短语层还是词汇层上的问题?以及抽象的、不可见的实体在句法分析中是否起到了重要的作用?实际上,不同的理论框架对这些问题的解释是互通的。作者在最后一章说明了如何对所有语言都具有的共性特征或某类语言具有的共同特征进行描写。
%The second part of the book compares these approaches with respect to their predictions regarding language
%acquisition and psycholinguistic plausibility. The nativism hypothesis, which assumes that humans
%posses genetically determined innate language-specific knowledge, is critically examined and
%alternative models of language acquisition are discussed. The second part then addresses controversial issues
%of current theory building such as the question of flat or binary branching structures being more appropriate, the question whether constructions %should be treated on the phrasal or the lexical level, and the question whether abstract, non-visible entities should play a role in syntactic analyses. %It is shown that the analyses suggested
%in the respective frameworks are often translatable into each other. The book closes with a chapter
%showing how properties common to all languages or to certain classes of languages can be
%captured.

\bigskip

%\vfill

~\bigskip

\noindent
Müller对各种语法理论进行了严格且公允的评价,这项工作填补了已有文献的空白。
\begin{flushright}
——\, \href{http://dx.doi.org/10.1515/zrs-2012-0040}{卡伦 $\cdot$ 莱曼,《德语语言学评论杂志》,2012年}
\end{flushright}
%``With this critical yet fair reflection on various grammatical theories, Müller fills what has been a major gap in the literature.'' \href{http://dx.doi.org/%10.1515/zrs-2012-0040}{Karen Lehmann, \textit{Zeitschrift für Rezensionen zur germanistischen Sprachwissenschaft}, 2012}


\smallskip

\noindent
Stefan Müller近期发表的《语法理论》这本导论性教科书,是一本针对句法理论现状概览的令人叹为观止的既全面又深入的入门教材。
 \begin{flushright}
——\, \href{http://dx.doi.org/10.1515/zfs-2012-0010}{沃尔夫冈 $\cdot$ 施特恩费尔德和弗兰克 $\cdot$ 里希特,《语言学杂志》,2012年}
\end{flushright}
%``Stefan Müller’s recent introductory textbook, ``Grammatiktheorie'', is an astonishingly
%comprehensive and insightful survey of the present state of syntactic
%theory for beginning students.'' \href{http://dx.doi.org/10.1515/zfs-2012-0010}{Wolfgang Sternefeld und Frank Richter, \textit{Zeitschrift für %Sprachwissenschaft}, 2012}


\smallskip

\noindent
本书所做的工作就是那种会被广为推崇的研究……作者在文中客观公允的论述尤为让人感到耳目一新。
 \begin{flushright}
——\, \href{http://dx.doi.org/10.1515/germ-2011-537}{维尔纳 $\cdot$ 亚伯拉罕,《日耳曼语言文学》,2012年}
\end{flushright}
%``This is the kind of work that has been sought after for a while. [\dots] The impartial and
%objective discussion offered by the author is particularly %refreshing.''
%\href{http://dx.doi.org/10.1515/germ-2011-537}{Werner Abraham, \textit{Germanistik}, 2012}

\smallskip

\noindent
这部两卷本著作代表了自Sells(1985)的开创性研究以来的一种努力。那就是为理论语言学家以及与他们紧密合作的人提供当代形式框架的普遍表征机制以及那些区分各形式框架的关键问题。总体来讲,文中对实证语料以及理论概念的呈现对于专家和学生都很好理解。这些材料的使用最适合想要更好地了解在这些框架中决定了一般表示方法的核心概念的那些人。尽管这里所涉及的许多特征都是相同的,但也有一些固有的根本差异。本书至少能让那些对关键问题有不同看法的人参与到讨论中,或许能更好地理解和欣赏彼此的研究进展。最后,跟Sternefeld和Richter(2012)针对这部作品的早期版本和生成语法的先验状态所提出的悲观观点不同的是,我从积极的角度看待Müller在这里所做的工作,并将之视为一个通道,它有可能把形式语言学家们聚集到一起,以期获得在他们自己的学术团体之外的针对理论研究的更为全面的评价。
\begin{flushright}
——\, \href{http://doi.org/10.5334/gjgl.414}{迈克尔 $\cdot$ T $\cdot$  帕特南,《Glossa:普通语言学杂志》,2017年}
\end{flushright}
%\href{http://doi.org/10.5334/gjgl.414}{Michael T. Putnam, Glossa, 2017}

%These two volumes represent one of the first attempts since Sells’ (1985) seminal work to provide theoretical linguists and those who work closely with them with an overview of the general representational machinery of contemporary frameworks and the key issues that separate those who prefer one over another. In general, the presentation of empirical data and theoretical concepts is highly accessible to scholar and student alike. The best use of these materials is for those seeking to gain a better understanding of the core concepts that motivate the general representations present in these frameworks. Although there are traits that are shared across many of these covered here, there are also fundamental differences that persist. These volumes at the very least enable those with different per- spectives on key issues to engage in discussions and perhaps gain a better understanding and appreciation of each other’s research moving forward. In closing, contra Sternefeld and Richter’s (2012) somewhat pessimistic statements directed at an earlier version of this work and toward the state of generative grammar a priori, I view Müller’s work here in a positive light as a conduit that has the potential to bring formal linguists together to gain a fuller appreciation of theoretical work beyond their own immediate communities.

}
\dedication{献给Max}
%\dedication{For Max}
\renewcommand{\lsISBNdigital}{978-3-96110-256-3}
\renewcommand{\lsISBNhardcover}{??}
\renewcommand{\lsISBNsoftcover}{??}
\renewcommand{\lsSeries}{tbls} % use lowercase acronym, e.g. sidl, eotms, tgdi
\renewcommand{\lsSeriesNumber}{??} % wahrscheinlich 6 will be assigned when the book enters the proofreading stage
\renewcommand{\lsURL}{http://langsci-press.org/catalog/book/177} % contact the coordinator for the right number
\renewcommand{\lsID}{177} % github, paperhive



%\AdditionalFontImprint{SimSun(宋体)、SimKai(楷体)}
\AdditionalFontImprint{SourceHanSerifSC(宋体)、FZKTJW(楷体)}

%      <!-- Local IspellDict: en_US-w_accents -->
