转换语法--最简方案跟上一章介绍的管辖与约束理论一样,最简方案MinimalistProgram(MP)(是由乔姆斯基在波士顿的麻省理工学院提出来的。Chomsky93b-u,Chomsky95a-u提出,我们应该认真地对待语言进化的问题,而且我们应该能够回答出语言知识是如何成为我们天赋的一部分的问题。为了回答这些问题,他认为我们应该重新关注那些语言学分析机制所需的最精简假设的理论发展,并最终得到那些更少地针对个别语言的内在知识的模型。与GB理论一样,最简方案也广为流传:全世界的理论学家都在这一框架下工作,所以说下面列出的研究者与研究机构一定是不够全面的。LinguisticInquiry(《语言探索》)与Syntax(《句法》)这两本期刊几乎只刊登最简方案方面的工作,而且读者可以从这些期刊中获知在该领域中较为活跃的人物。德语方面最为著名的学者有ArtemisAlexiadou(柏林洪堡大学)、GüntherGrewendorf2002a(缅因的法兰克福)、JosephBayer(康斯坦斯)、和GereonMüller(莱比锡)。尽管理论具有极大的创新性,而且GB理论下对小句的分析也具有非常重要的影响,我们可以在本书中讨论的大部分学说中找到这些内容,但是最简方案下技术层面的工作却非常少。无论如何,技术层面的工作是十分有用的,因为在最简方案这个框架下人们已经做了很多研究,而且理解该理论的基本机制可以有效地获知该理论下对语言事实的一些有趣的研究成果。尽管1980年代和1990年代的GB文献有很多相同的假设,最简方案则引发了许多不同的方法,这些方法使得我们难以跟上理论的发展。下面的内容主要参考了DavidAdger编著的教材之中的内容Adger2003a。表示形式的一般说明最简方案框架下的理论假说是以GB框架下的已经完成的工作为基础的。所以,在上一章解释的很多内容可以在这一章继续讨论。但是,在基础假设方面有一些改动。该理论摒弃了最为普遍的参数原则,而是将相关区别放在特征之中。语言间的差异表现在某些特征的值上面,而且,这些特征可以是强特征,也可以是弱特征,而且特征强度也是区分不同语言的一个属性。强特征使得句法对象移向更高的位置上。读者应该对这种特征驱动的移位方法较为熟悉了,因为它是第sec-passive-gb节中介绍的基于移位的被动分析的一部分。在被动的GB分析中,宾语为了获得格,必须要移到IP的限定语位置上。这类由于特征值的缺失而导致的移位是最简方案中的核心部分。基本框架乔姆斯基提出,只有两条操作(规则)用来整合语言对象:外部合并与内部合并。外部合并是指将the和book这两类要素组合起来,从而得到一个复杂的短语。内部合并被用来指称组成成分的移位。它应用于一个语言对象,并且从这个语言对象中取出一部分,并将之链接到对应对象的左边。外部合并与内部合并的应用可以按照任意顺序来实现。比如说,两个对象可以按照外部合并组合起来,然后一个合并项通过应用内部合并而移动到左边。得到的对象可以再与其他对象进行外部的合并,并且继续合并。如例(1)所示:themanwhoweknowDET男人CONJ我们认识`我们认识的那个男人'为了得到这个NP,动词“know”与它的宾语“who”通过外部合并到一起。在下面讨论的几个中间过程的合并之后,“knowwho”将与“we”合并,最后“who”通过内部合并移到左边,得到了“whoweknow”。这个关系小句还可以通过外部合并与“man”组合在一起,并且继续进行下去。所以,最简方案与GB理论是不同的,它没有假设具有一个根据某个语法生成的深层结构以及一个通过从深层结构生成的表层结构。相反,它假定短语是由外部合并和内部合并(组合与移位)按照任意顺序生成一个结构,然后就被定型了。一般认为,该结构被发送到表层:一方面是发音-感知系统(AP),另一方面是概念-意念系统(CI)。AP对应于GB理论中的语音形式(PF)层,CI对应于逻辑形式(LF)层。图fig-architecture-minimalism描述了这种新的架构。在短语模型之前的最简方案的架构显性句法表示通常具有可见效应的句法操作。在显性句法之后,句法对象被发送到表层,这里在这个定型点之后有一些转换操作。因为这类转换并不会影响发音,这部分句法被叫做隐性句法。就像GB理论中的LF,隐性句法被用来推导出由辖域决定的意义。这一架构后来被修改成在推导过程中的很多点都可以作定型点。现在是这样假定的,在推导过程中包括很多阶段,一旦一个完整的短语被用在与中心语的组合中Chomsky2008a,它就被定型了。比如说,像例(1)中的从句“thatPetercomes”是一个短语,它在整个句子完成之前被送到表层。AndreasPankau(p.c.2015)跟我指出,这种分段的概念有一个根本性的问题,因为如果是这样的话,那么只有与中心语有关系的要素被送到表层,然后推导中的最顶层短语就再也不能被送到表层了,因为它没有任何可供依存的中心语。HebelievesthatPetercomes.他认为CONJPeter来`他认为Peter会来。'对于截止到哪个范畴构成完整的短语这一问题有许多不同的看法。因为阶段这个概念对于下面的内容并不重要,我会在下面忽略这一概念。请看第sec-dtc节中有关阶段的心理语言学的可能性与普遍意义上的最简结构的讨论。配价、特征核对与一致关系最简方案的基本机制是特征核对。feature!checking例如,名词letters(信)有一个P特征,这个特征是指它要与一个PP相结合以构成一个完整的短语。letterstoPeter信PREPPeter`给Peter的信'一般认为,既有可预测的特征,也有不可预测的特征。不可预测的特征的例子是名词的数的特征。单数和复数的区别在语义上是具有相关性的。词类的范畴特征纯粹是句法特征,所以它不能在语义上被解读。最简方案认为所有不可预测的特征需要在复杂的语言对象的推导过程中被用到。这种对特征的穷尽过程被叫做核对(checking)。比如说,我们再来看名词letters的例子。例(0)的分析如图fig-letters-to-peter-minimalism所示。不可预测特征的配价表征letters的P特征是不可预测的,这点由P前面的小u表示。letters的P特征的不可预测性可以通过“toPeter”的P特征来进行核对。所有核对过的特征就被自动删除了feature!deletion。在图中,删除通过将特征突显而得到表示。像例(1)的字符串被作为完整的推导被规则排除了,因为P的N特征没有被核查。这一情况如图fig-letters-to-minimalism所示。lettersto信PREP基于不可预测特征的不合法的句法对象如果这个结构能够用在拼读出的更大的结构中,那么这个推导就会崩溃了,因为概念系统不允许N特征还出现在P节点上。选择性特征是原子式的,这就是说,在GB和本书中介绍的其他理论中,介词不能选择NP[acc],除非这个NP[acc]是原子式的。这样,一个附加的机制就在选择特征之外来核对其他特征。这个机制叫做一致(Agree)Agree(。letterstohe信PREP他letterstohim信PREP他`给他的信'例(0b)的分析如图fig-letters-to-him-minimalism所示。一致的特征核对在选择性特征核对与一致核对之间有着有趣的差别。一致所核查的特征不需要是与中心语相组合的对象的最高节点。这在后面被动和局部重新排序的分析中发挥着一定的作用。Agree)短语结构与理论在第Abb-GB-Min-Max页的图Abb-GB-Min-Max中给出了结构的投射。按照理论的早期版本,可以有任意多的补足语与组合以构成。也可以有任意多的附加语附加到上,然后至多一个限定语可以组合到中,以构成一个XP。最简方案采用二分叉结构,所以至多只有一个补足语,它是第一个合并的项目。进而,它并没有指定一个特殊的限定语位置。Chomsky倾向于这样的认识,所有不是补足语的项目是限定语。也就是说,他将首次合并(补足语)与后合并的项目(限定语)区分开来。图fig-head-comp-spec显示了带有两个限定语的例子。最简方案下的补足语与限定语也可以只有一个补足语且没有限定语,或者有一个或三个限定语。最终允准哪种结构在于参与到合并操作中的项目的特征。短语投射是还是XP取决于这个短语是否被用作另一个中心语的补足语或是限定语,或者它是否用作进一步合并操作的中心语。如果一个短语被用为限定语或补足语,它的状态就固定为一个短语(XP),否则得到的短语的投射状态被看作是未被限定的。在合并操作中的词汇中心语的子节点具有范畴X,而合并操作中的复杂中心语的子节点们具有范畴。这就解决了标准的理论方法具有代词和专有名词时的问题:必须假定出许多的一元结构(看Abb-GB-Min-Max中的左图)。这在最简方案中就不是问题了。有关这一方面的问题请看[Chapter2.1]Brosziewski2003a-u。小v在第节,我用结构来表示双及物动词,该动词带上宾格宾语构成一个,然后跟与格宾格构成一个。这种动词带宾语构成的二元结构以及平铺结构的分析被大部分采用GB理论和最简方案的学者所反对,因为这些分支并不是反身代词与否定极项的约束现象所需的分支结构。在例(1a)中,Benjamin与himself具有约束关系是不可能的:EmilyshowedhimselfBenjamininthemirror.Emily展示他自己BenjaminPREPDET镜子PetershowedhimselfBenjamininthemirror.Peter展示他自己BenjaminPREPDET镜子`Peter在镜子中给他自己看Benjamin。'约束关系所需要的分析,以及按照树的构造分析这些现象的这些理论中的NPI现象是,反身代词在树中比专有名词Benjamin的位置高。更准确地说,反身代词himself必须c-统制Benjamin。c-统制的定义如下所示[117]Adger2003a:c-统制在GB理论中也发挥着重要的作用。实际上,管辖与约束理论的一部分是约束理论,我们在前面的章节中没有讨论这一现象,因为本书中并不涉及管辖现象。节点Ac-统制B,当且仅当A的子节点满足以下条件之一:是B,或者包括B在图左边和中部的树中,并没有所期待的c-统制:在最左边的树中,所有的NPs互相c-统制,在中部的树中,“Benjamin”c-统制“himself”,而不是其他成分。双及物动词的三种可能分析所以说,在左边和中间的结构是不合适的,而且还有一些附加的结构包括范畴v,它被叫做小v[Section4.4]Adger2003a。“himself”的子节点是,而且包括“Benjamin”,所以“himself”c-统制“Benjamin”。由于“Benjamin”的子节点是V,而且V既不是“himself”,也不包括“himself”,“Benjamin”并没有c-统制“himself”。关于双及物动词,早期的分析认为包括一个附加的动词性中心语Larson88a。[70]HK93a-u认为,这个动词性中心语具有致使的语义。图中结构的生成被认为是动词show始于V位置,然后移动到v的位置上。show被认为具有see的意义,而且在小v的位置上,它具有致使含义,这样就得到了致使-看见的含义[133]Adger2003a。移动到小v的双及物分析尽管带有空动词中心语的动词壳分析最初是由Larson88a提出来分析双及物动词的,现在它还被用来分析严格的及物动词,甚至是不及物动词。Adger2003a认为,在具体的树的配置中,语义角色的配置是不一致的。的NP子节点被分析为施事VP的NP子节点被分析为主事的PP子节点被分析为目标Adger认为,这种语义角色指派的不一致有助于语言认知languageacquisition的过程,而且从此,按照这一点,小v在严格的及物和不及物动词的分析中也发挥了重要的作用。图fig-transitives-little-v和图fig-intransitives-little-v显示了分别包括动词burn和laugh的句子的分析。如果这一类型的所有不及物动词都被认为是具有作主语的施事,那么就需要对施事做出更为宽泛的界定,以包括sleep这类动词的主语。通常,sleeping不是一个有意为之的活动。包括小v的严格及物动词的分析包括小v的不及物动词分析Adger2003a认为,不及物和及物动词从V移动到小v的位置上。这点在下面的图中有所显示。CP、TP、和VP第节分析GB理论中的CP/IP系统。在最简方案的发展过程中,屈折短语被分成几个功能性投射Chomsky89a-u,其中只有时态category!functional!Tense短语在目前的最简方案的分析中有所涉及。所以,最简方案的TP对应于GB分析中的IP。除了这一变化,CP/IP分析的核心思想被转化为英语的最简方案分析。这一小节将先讨论触发移位的特殊特征(第sec-epp-features小节),然后是格指派(第sec-case-mp小节)。特征作为移位的触发语:T的EPP特征在GB理论中,情态动词和助词被分析为范畴I的成员,主语是IP的限定语。在上一节,我说明了主语是如何被分析为的限定语的。现在,如果我们假设有一个情态动词,包括这样一个,且主语在情态动词后面,这一语序与观察到的英语的语序要求是不一致的。要解决这一问题,可以假定在T上有一个强势的不可解释的D特征。因为特征很强,一个合适的D必须要移动到T的限定语的位置上,然后在域内核查D。图fig-Anna-will-read-the-book-minimalism显示了在例(1)的分析中TP所发挥的作用:Annawillreadthebook.Anna将读DET书`Anna将要读书。'包括情态词和从v到T的主语移位的“Annawillreadthebook”的句子分析figureDP“thebook”是read的宾语,然后核查read的D特征。小v选择了主语Anna。因为T有着强势的D特征(由星号`*'*标记),Anna一定不能在内部,但是会移动到TP的限定语位置上。完整的句子是CPs。针对例(0)的分析,空C的中心语被认为是与TP组合在一起。空C贡献了从句的类型特征Decl。例(0)的完整分析如图fig-Anna-will-read-the-book-minimalism-CP所示。带有空C和小句类型特征Decl的CP的“Annawillreadthebook”的句子分析例(1)中问句的分析包括对于问句(question)的句子类型的未指派值的T的小句类型特征。WhatwillAnnaread?什么将Anna读`Anna要读什么?'空补足语C具有Q特征,它可以给T的小句类型特征赋值。由于T的小句类型特征具有强势的Q值,T元素必须要移到C来局部核查。另外,wh元素被移位了。这个移位是由C上的强wh特征决定的。例(0)的分析如图fig-What-will-Anna-read-minimalism所示。带有空C和强wh特征的“WhatwillAnnaread?”的句子分析格指派在第chap-gb章介绍的GB分析中,主格是由(定式)I所指派的,而其他格是由动词指派的(请看第sec-case-assignment节)。主格的指派由最简方案接管,所以一般认为主格由(定式)T所指派。但是,在我们考虑的最简方案中,没有一个单一的动词投射,但是有两个动词性投射:和VP。现在,我们可以认为V指派给它的补足语宾格,或者v将宾格指派给它所统制的动词的补足语。[Section6.3.2,Section6.4]Adger2003a认同后一种方法,因为它与所谓的非宾格动词和被动的分析是一致的。图fig-Anna-reads-the-book-minimalism-TP显示了例(1)中的TP:Annareadsthebook.Anna读DET书`Anna读这本书。'的格指派和“Annareadsthebook”这句的TP中的v“Anna”和“thebook”这两个NP开始是没有赋值的不可解读的格特征:[u格:]。这些特征被赋值为T和v。一般认为,只有一个特征通过合并得到核查,所以这可以是T上的D特征,并为其他可能的核对机制留下格特征:一致。一致可以用来核对子节点的特征,也可是树上较远距离的特征。第一个节点需要c-统制跟它具有一致关系的节点。c-统制大概是指:一个节点在上,然后任意多节点在下。所以vc-统制VP、V和DP“thebook”,以及所有该DP内部的节点。由于一致可以给c-统制的节点赋值,v上的宾格可以给DP“thebook”的格特征赋值。一致内部的非局部性带来一个问题:为什么例(1)是不合乎语法的?Himlikesshe.他喜欢她v的宾格可以通过它的主语得到核查,而T的主格可以通过likes的宾语得到核查。所有的DP都与T和v具有必需的c-关系。这一问题就通过要求所有的一致关系都包括最近的可能的要素而得到解决。[218]Adger2003a构建了如下的限制条件:principle-locality-of-matching匹配的局部性locality!ofmatching:在X上的特征F与在Y上匹配的特征F具有一致关系,当且仅当没有介于中间的Z[F]。这种介于关系在例(1)中被界定为:def-intervention介于关系intervention:在结构[XZY]中,Z介于X与Y中间,当且仅当Xc-统制Y。所以说,因为T可能与Anna具有一致关系,它一定不能与“thebook”具有一致关系。由此,(ex-him-likes-she)中指派给she的主格是不可能的,而且(ex-him-likes-she)也被准确地排除了。说明语[Section4.2.3]Adger2003a认为说明语附加在XP上,并构成了一个新的XP。他把这一操作叫做邻接(Adjoin)。由于这一操作并不消耗任何特征,它与外部合并是不同的,所以说这是引入到理论中的一个新操作,这与Chomsky所主张的人类语言只使用合并作为结构构建操作是相互矛盾的。也有人提议将说明语看作是带有空中心语的特殊的副词性短语(请看第sec-functional-projections-minimalism节),并将其看作是功能性投射层级中的一部分。我个人更倾向于Adger在许多其他的框架下提出的解决办法:我们用一条特殊的规则和操作来解决说明语和中心语的组合问题(请看第sec-adjuncts-hpsg节有关HPSG框架下针对中心语说明语的组合问题)。动词位置根据前一节介绍的机制,德语动词位于首位的句子的分析是比较直接的。基本观点与GB理论中的是一致的:定式动词从V移到v,再移到T,然后到C。移到T的移位是由T上的强势时态特征所控制的,由T复杂式到C的移位由T上的小句类型特征得到加强,该特征通过C被赋值为强势的Decl。例(1)的分析由图fig-kennt-jeder-diesen-mann-minimalism所示。KenntjederdiesenMann?认识每人这男人`每个人都认识这个男人吗?'在Adger2003a的分析下有关“KenntjederdiesenMann?”(每个人都认识这个男人吗?)的分析长距离依存在解释完动词位于句首的句子之后,动词位于第二位的句子的分析就不稀奇了:[331]Adger2003a认为有一个特征触发了成分向C的限定语的位置上的移位。Adger称这个特征为向上,但是这个术语是不恰当的,因为德语陈述句的首位并不限制为话题成分。图fig-diesen-mann-kennt-jeder显示了例(1)的分析:DiesenMannkenntjeder.这男人认识每人`每个人都认识这个男人。'Adger2003a的分析下有关“DiesenMannkenntjeder.”(这个男人,每个人都认识。)的分析被动Adger2003apassive(针对英语被动式提出了相关的分析,这里我将其应用于德语。就像第sec-passive-gb节讨论的GB分析中,一般认为动词并没有将宾格指派给shalagen(打)的宾语。在最简方案的术语中,这意味着小v不具有需要被核查的宾格特征。小v的这一特殊版本在所谓的非宾格动词的句子的分析中发挥了重要作用Perlmutter78。非宾格动词verb!unaccusative是具有许多有趣特征的不及物动词的小类。比如说,他们可以被用在形容词分词participle!adjectival中,尽管这在不及物动词中并不常见。dergetanzteMannDET跳舞人dergestorbeneMannDET死人`这个死人'对于这一区别的解释是形容词分词说明了主动句中的宾语:dassderManndasBuchgelesenhatCONJDET人DET书读AUX`这个人读这本书'dasgeleseneBuchDET读书现在的设想是gestorben(死)的论元看上去像宾语,而getanzt(跳舞)的论元像主语。如果形容词性被动式可以说明宾语,这就解释了为什么例(-1b)是可能的,而例(-1a)是不可能的。Adger2003a提出了图fig-little-v-unaccusative中带非宾格动词的的结构。在Adger2003a下,带fall、collapse和wilt这类非宾格动词的结构一般认为,这个小v的非宾格变量在被动的分析中起到了重要的作用。非宾格动词与被动动词相似,因为他们都有一个主语,这些主语在某种程度上也有宾语的特征。小v的特别版本由被动式中心语werden选择,这就构成了一个被动短语category!functional!Passive(缩写为PassP)。请看图fig-passive-schlagen-mp中针对(1)中的例子的分析:dassergeschlagenwurdeCONJ他打AUX`他被打了'基于一致的带有非域内格指派的没有移位的被动的最简方案分析Pass中心语要求小v的Infl特征具有Pass的值,这就导致了输出层面的分词形态变化。所以采用的形式是geschlagen(打)。助词移动到T,来核对T的Infl的强势特征,并且由于Infl特征是过去式,werden的过去式形式是wurde,该形式用于输出表层。T有一个主格特征尚需被核查。有趣的是,最简方案并不要求schlagen的宾语移动到T的限定语位置上来指派格,因为格指派是通过一致来达成的。所以说原则上,schlagen的凸显论元可以在它的宾语位置上,并且无论如何都会从T上得到主格。这就可以解决[Section4.4.3]Lenerz77指出的GB分析中的问题。请看第ex-passive-German-no-movement页有关Lenerz的例子和问题的讨论。但是,[332]Adger2003a认为德语在T上具有强EPP特征。如果这一假设得到支持,那么GB理论下所有的问题都会延伸到最简方案的分析中:所有的宾语都需要移到T上,即使没有发生重新排序。进而,例(1)这类人称被动式就会有问题,因为没有名词短语能够为了核查EPP特征而移到T上:weilgetanztwurde因为跳舞AUX`因为那儿有人跳舞'局部重新排序Adger2003a并没有分析局部重新排序。但是文献中有一些其他方面的建议。因为最简方案中所有的重新排序都是特征驱动的,所以就必须有一个特征可以触发例(1b)中的重新排序:[weil]jederdiesenMannkennt	因为每人这人认识`因为每个人都认识这个人'[weil]diesenMannjederkennt	因为这人每人认识像话题短语[222]Laenzlinger2004a或提供可移动位置的AgrS和AgrO[Chapter4]Meinunger2000a这类功能性投射有着许多不同的看法。G.[Section3.5]GMueller2014a-u给出了一个简洁的解决方案。在他的方法中,宾语简单地移到小v的第二个限定语位置上。这一分析在图fig-scrambling-minimalism中有所描述。G.Müller提出了v的可选特征和触发局部重新排序的V(第48页)。这些在图中没有显示。dassdiesenMannjederkennt”(每个人都认识这个人)句中宾语移到v的限定语位置上的分析Laenzlinger2004a提出了一个观点来假定宾语的几个宾语短语可以按照任意次序排列。宾语移到这些投射的限定语位置上,而且因为宾语短语的语序没有限制,例(1)中的所有顺序都是可分析的:dassHansdiesenBriefmeinemOnkelgibtCONJHans这信我的舅舅给`Hans把这封信给我的舅舅'dassHansmeinemOnkeldiesenBriefgibtCONJHans我的舅舅这信给`Hans给我的舅舅这封信'新的发展与理论变体在90年代初期,乔姆斯基对GB理论的基本假设进行了重新思考,并只保留了那些绝对必要的部分。在最简方案(MinimalistProgram)中,乔姆斯基说明了对GB理论进行修正的核心动因Chomsky93b-u,Chomsky95a-u。直到90年代早期,格理论CaseTheory、Theta"=标准theta-criterion@Theta"=Criterion、理论、邻接理论Subjacency、约束理论BindingTheory、控制理论ControlTheory等都属于语言的内在机制[804]Richards2015a。当然,这就涉及到了非常具体的语言知识是如何进入我们的基因组的问题。最简方案沿着这一思路,并且试图解释更为普遍的认知原则下的语言属性,以及减少具体的内在语言知识的数量。比如说,深层结构和表层结构D"=structureS"=structure之间的区别就被取消了。移位仍是一种操作,但是只直接用来构建子结构,而不是在一个完整的D"=结构之后完成。语言之间的差别在于这种移位是可见的还是不可见的。尽管乔姆斯基的最简方案应该被看作是GB理论的后续理论,最简方案的支持者经常强调这样一个事实,即最简方案并不是一种理论,而是一个研究项目(乔姆斯基[4]Chomsky2007a、[6]Chomsky2013a)。在Chomsky95a-u介绍这一研究项目时,乔姆斯基提出的实际分析被理论家们热烈地评论,并且有时会引发严重的质疑*Kolb97a,JL97a-u-platte,JL99a-u-gekauft,LLJ2000b,LLJ2000a,LLJ2001a,Seuren2004a,PJ2005a。不过,我们应该承认有些评论偏离了问题的关键。最简方案有很多分支。在下面的内容中,我将讨论一些核心的观点,并解释哪些部分被认为是有问题的。移位、合并、特征驱动的移位与功能投射Johnson、Lappin和Kolb曾质疑过乔姆斯基系统的计算方面。乔姆斯基将经济原则引入了理论。在某些条件下,语法系统可以创造出一个任意数量的结构,但是只能是最经济的,也就是说,那些需要最少力气来产生的结构被认为是合乎语法的,也叫做转移派生式经济原则(transderivationaleconomyeconomy!transderivational)。这一假设并不需要被过分重视,实际上,它在最简方案框架下的很多研究中没有发挥重要的作用(尽管Richards2015a在最新的有关生成的方法中用经济性进行了比较)。无论如何,乔姆斯基的理论的其他方面可以在很多近期的研究中有所发现。比如说,乔姆斯基提出将基本结构的构建允准规则的数量减少到两个:移位Move和合并Merge(即内部Merge!Internal和外部合并Merge!External)。移位对应于操作,这点已经在第chap-gb章有所讨论,合并是将(两个)语言对象进行组合。普遍认为的是,两个对象可以被组合起来[226]Chomsky95a-u。对于移位来说,一个给定的移位操作一定会有一个动因。这个移位的原因被认为是可以核查它要移到的位置上的某些特征featurechecking。这一观点早在第节有关被动的分析中就有所说明:宾格宾语在被动句中不能带有格信息,进而必须移到能够接收格的位置上。这类方法也可以用在一系列其他现象中。例如,有的短语的中心语是焦点focus和话题topic范畴。德语和英语中相应的功能中心语永远是空的。尽管如此,提出这些中心语是受到别的语言中有表示话题和焦点的形态变化的启发。这一论断是合理的,只是建立在所有其他的语言中都有同样的范畴的假设上。不过,这一具有普遍性的共享部分(普遍语法,UniversalGrammar,UG)跟具体的语言知识这种假设是冲突的,并且只被少数在乔姆斯基传统外的学者们所认可。即使在乔姆斯基式语言学下工作的学者而言,仍有很多像是否这样讨论问题是合适的问题被提出来,因为这只是创造循环结构的能力,这一能力负责人类应用语言的具体能力(狭义的语言的功用)--正如*HCF2002a所认为的--那么个别的句法范畴不属于普遍语法,其他语言的数据也不能用来解释另一种语言不可见的范畴。功能投射与语言知识的模式化移位必须由特征核查所允准这一思想导致了(静态)功能中心语数量的膨胀category!functional。这类中心语的假设并不是必要的,因为特征可以被集中,然后他们能被一起核查。在这一点,HPSG理论在本质上是类似的,请看[SectionII.3.3.4,SectionII.4.2]Sternefeld2006a-u。在所谓的制图cartography方法中,每个形态句法特征都对应于一个独立的句法中心语[54,61]CR2010a。对于一个明显形式化的方法来说,在一个组合操作中有一个特征被利用了(请看[335]Stabler2001a)。Stabler的最简方案语法(MinimalistGrammars)在第节中有更为详细的讨论。[297]Rizzi97a-u提出了图Abbildung-Rizzi(也可以参考Grewendorf[85,240]Grewendorf2002a、Grewendorf2009a)中的结构。在[297]Rizzi97a-u下句子的句法结构功能范畴Forcecategory!functional!Force、TopTopcategory!functional!Top、Foccategory!functional!Foc和Fincategory!functional!Fin对应于小句类型、话题、焦点和定式。一般认为移位总是锚定限定语位置。话题和焦点元素总是移动到相应短语的限定语位置上。话题可以在焦点元素的前面或后面,这就是为什么会有两个话题投射:一个在FocP上面、一个在FocP下面。话题短语是可循环的,也就是说,任意数目的TopP可以出现在图中的TopP位置上。根据[70]Grewendorf2002a,话题和焦点短语只因具体的信息结构的需要而得到实现,比如说移位。对于功能投射是不是可选的具有不同的看法。有些作者认为功能投射的完整层级总是存在的,但是功能中心语可以是空的(如[106]Cinque99a-u和[55]CR2010a)。[147]Chomsky95a-uSeite-AgrO采纳了Pollock89a-u的观点,并认为所有语言都有主宾一致关系isagreement!object和否定(AgrScategory!functional!AgrS、AgrOcategory!functional!AgrO、Negcategory!functional!Neg)的功能投射。	请看[Section4.10.1]Chomsky95a-u。[78]Sternefeld95a、[103]Stechow96a和[100--101,124]Meinunger2000a区分了直接宾语和间接宾语(AgrOcategory!functional!AgrO、AgrIOcategory!functional!AgrIO)的两个一致关系的位置。同样对于AgrScategory!functional!AgrS、AgrOcategory!functional!AgrO和Negcategory!functional!Neg,BS97a-u认为功能性中心语为了解释英语在LF层特征驱动的辖域现象来进行分享category!functional!Share和分发category!functional!Dist的操作。对于没有空元素或移动的辖域现象的分析,请看第节。[13]BG2005a提出了PolPcategory!functional!Pol、PolPcategory!functional!+Pol和PolPcategory!functional!Pol范畴来讨论极性问题。[76]Webelhuth95a对功能性投射进行了概述,并在1995年提出了以下范畴的定义,包括AgrAcategory!functional!AgrAAgrNcategory!functional!AgrN、AgrVcategory!functional!AgrV、Auxcategory!functional!Aux、Clitic态category!functional!CliticVoices、Gendercategory!functional!Gender、Honorificcategory!functional!Honorific、category!functional!、Numbercategory!functional!Number、Personcategory!functional!Person、Predicatecategory!functional!Predicate、Tensecategory!functional!Tense和Zcategory!functional!Z。除了AdvPcategory!functional!Adverb、NegPcategory!functional!Neg、AgrP、FinP、TopP和ForceP,*WHBH2007a-u提出了外围TopPcategory!functional!OuterTop。[31]Poletto2000a-u针对意大利语中附着语的位置,提出了HearerPcategory!functional!Hearer和SpeakerPcategory!functional!Speaker。[106]Cinque99a-u在他的研究中采用了32个功能性中心语,如图Tabelle-Cinque所示。tabletabular[t]@r@l@r@l@r@l@r@l@1.&MoodSpeechAct&2.&MoodEvaluative&3.&MoodEvidential&4.&MoodEpistemic5.&T(Past)&6.&T(Future)&7.&MoodIrrealis&8.&ModNecessity9.&ModPossibility&10.&ModVolitional&11.&ModObligation&12.&ModAbility/permission13.&AspHabitual&14.&AspRepetitive(I)&15.&AspFrequentative(I)&16.&AspCelerative(I)17.&T(Anterior)&18.&AspTerminative&19.&AspContinuative&20.&AspPerfect(?)21.&AspRetrospective&22.&AspProximative&23.&AspDurative&24.&AspGeneric/progressive25.&AspProspective&26.&AspSgCompletive(I)&27.&AspPlCompletive&28.&AspVoicecategory!functional!Voice29.&AspCelerative(II)&30.&AspSgCompletive(II)&31.&AspRepetitive(II)&32.&AspFrequentative(II)tabularcategory!functional!Moodcategory!functional!Tcategory!functional!Modcategory!functional!Aspcategory!functional!Perfect(?)Tabelle-Cinque[106]Cinque99a-u的功能性中心语table他认为所有的句子都包括一个具有所有这些功能中心语的结构。这些中心语的限定语位置可以由副词充当或者保持空位。Cinque认为这些功能中心语和相应的结构构成了普遍语法中的一部分,即这些结构的知识是内在的(第107页)。图Tabelle-Cinque显示了小句域内的功能中心语。[96,99]Cinque94a-u也说明了带有投射的形容词的顺序:质量category!functional!Quality、大小category!functional!Size、形状category!functional!Shape、颜色category!functional!Color和国籍category!functional!Nationality。这些范畴和他们的语序被认为与普遍语法是有关系的(第100页)。[96]Cinque94a-u认为最有可能有七个属性形容词,并解释了这样的事实,在名词域中有着有限数量的功能投射。如第Beispiel-Iteration-Adjektive页所示的,在合适的语境中,有可能有这七种形容词,这就是为什么Cinque的功能性投射需要重复的原因。Laenzlinger2004a在Cinque的思想下,提出德语的功能中心语语序。他还采用了Kayne94a-u的思想,他认为所有的句法结构都有跨语言的具体语序的中心语补足语,即使表层语序看起来是与之矛盾的。组成成分序列最终是可见的是由左向移位推导而成的。fn-Kayne-Extraposition这也适用于外置extraposition,即德语中将成分移到后场。通常这会被分析为右向移位,[Chapter9]Kayne94a-u将它分析为把所有其他成分向左移动。Kayne认为(i.b)是通过移动一部分的NP来从(i.a)推导而来的:justwalkedintotheroom[NPsomeonewhowedon'tknow].刚走进DET房间NP某人谁我们不认识`刚走进房间[NP我们不认识的某人]'Someonejustwalkedintotheroom[NPwhowedon'tknow].某人刚走进DET房间NP谁我们不认识`某人刚走进房间[NP我们不认识的]'(i.a)一定是某种推导出的中间表示,否则英语就不会是SV(O),而是V(O)S。由此,(i.a)是通过将VP“justwalkedintotheroom”前置由(ii)推导出来的。Someonewhowedon'tknowjustwalkedintotheroom某人谁我们不认识刚走进DET房间“我们不认知的某个人刚走进房间'需要承认的是,这类分析不能轻易地与运用模型相组合(请看第章)。图Abbildung-Remnant-Movement-Satzstruktur显示了动词位于末位的小句的分析,这里功能副词中心语被省略了。这些结构并不对应于第sec-xbar节介绍的理论。在某些情况下,中心语与补足语相组合以构成XP而不是X。更多有关最简方案中理论的内容,请看第节。在[224]Laenzlinger2004a下的左向剩余移位和功能中心语的句子结构分析主语和宾语分别作为vP和VP内部的论元而生成。主语移到了主语短语category!functional!Subj的限定语位置,而宾语移到了宾语短语category!functional!Obj的限定语位置。动词性投射(VP))移到了助词前,并移到了包括助词的短语的限定语位置上。SubjP和ObjP的唯一功能在于给各自移位提供落地点。对于宾语在主语前的句子而言,Laenzlinger认为宾语移到了话题短语的限定语位置上。图Abbildung-Remnant-Movement-Satzstruktur只包含一个ModP和一个AspP,尽管Laenzlinger认为Cinque提出的所有中心语在所有德语小句的结构中都是存在的。Zwart1994a-u针对荷兰语提出了一个类似的分析,宾语和主语从动词位于末位的VP移到了Agr位置上。针对Kayne模型的争论,请看Haider2000a。Haider指出,Kayne类的理论对德语作出了不正确的判断(比如说对于选择的副词和第二谓词的位置,以及动词的复杂构成),由此就无法证明该理论可以解释所有的语言的论断。[Section4]Haider97a指出,由Pollock89a-u、Haegeman95a-u和其他人提出的空否定中心语的论断是有问题的。请看Bobaljik99a关于Cinque的副词投射的论断的讨论。进而,还需要指出SubjP和ObjP,TraPcategory!functional!Tra(及物短语)和IntraPcategory!functional!Intra(不及物短语)([1745]Karimi-Doostan2005a)和TopPcategory!functional!Top(话题短语),DistPcategory!functional!Dist(数量短语),AspP\is{category!functional!Asp}(情态短语)(\citealp[\page2]EKiss2003a-u、[35]Karimi2005a),PathPcategory!functional!PathP和PlacePcategory!functional!PlaceP[246]Svenonius2004a-u这些范畴符号都包含了语法功能、配价、信息结构和语义的信息。更多例子和参考资料请看Newmeyer([194]Newmeyer2004b、[82]Newmeyer2005a)Newmeyer也研究了规定每个语义角色(如施事、受益格、工具、致事、随伴格和逆行格短语)投射的条件。某种程度上,这是对范畴符号的误用,但是这种对信息结构和语义范畴的误用是必要的,因为句法、语义和信息结构是紧密相连的,而且,语义解读句法信息,即语义在句法之后进行(看图Abb-T-Modell和图fig-architecture-minimalism)。通过在句法中运用语义和语用相关的范畴,在形态、句法、语义和语用层面上已经没有清晰的界限了:每个元素都被“句法化”了。FelixBildhauerFelixBildhauer(p.c.2012)向我指出,意义的单个方面由节点表示的一系列功能投射的假设实际上与构式语法中的短语方法非常接近了([470]Adger2013a也持这一观点)。我们只是简单地列出这些句法配置,并且这些配置被赋予某种意义(或在句法后被解读的特征,如[62]CR2010a有关TopP的分析)。限定语位置的特征核对如果有人在从限定语"=中心语关系到逻辑结论中采用了特征核对理论,那么就可以得到诸如[452]Radford97a-u提出的分析结果。Radford认为,介词嵌套在附加于例(1)中的结构的一致关系短语中,通常认为,介词邻接于一致关系短语的中心语,并且介词的论元移到了一致关系短语的限定语位置上。minimal-pp-structure[PPPDP]这里的问题是宾语在介词之前。为了校正这一点,Radford提出了带有空中心语的功能投射p(读作小p),这一中心语是介词连接的对象。这一分析如图Abbildung-Radfords-PP所示。理论下有关限定语位置和小p的格指派分析这一机制只在保证特征核查发生在限定语"=中心语关系这一假设中是必需的。如果想允许介词决定它所带的宾语的格,那么所有这些理论上的演算就没有必要了,而且可以保证(minimal-pp-structure)中的结构是完好的。[549--550]Sternefeld2006a-u对这一分析提出质疑,并将之比喻为瑞士奶酪(充满了空洞)。这个瑞士奶酪的比喻也许太贴切了,因为,与瑞士奶酪不同的时是,这一分析中空洞的比率非常大(两个词与五个空元素)。我们已经看过第Abbildung-NP-ohne-Det页有关名词短语的分析了,这里NP的结构,只包括一个形容词klugen(聪明),它所包含的空元素比显性的词还多。与这里所讨论的PP的分析不同的是,空成分只出现在显性的限定词和名词实际出现的位置上。另一方面,小p投射是全部由理论内部推导的。对于图Abbildung-Radfords-PP的分析没有任何一条附加的假设是由理论外推导而成的(请看[549--550]Sternefeld2006a-u)。这一分析的变体由*[124]HNG2005a提出。这些作者没有利用小p,而是不那么复杂的结构。他们提出了例(1)中的结构,对应于图Abbildung-Radfords-PP中的AgrOP"=子树。[AgrPDP[AgrP+Agr[PPtt]]]作者认为DP向SpecAgrP的移动是不可见的,即隐性的movement!covert。这就解决了Radford的难题,并使得pP的假设多余了。这些作者通过匈牙利语中的一致现象来推动这一分析:匈牙利语的名词短语后的后介词与其在人称和数上保持一致关系。也就是说,他们认为英语的前介词与匈牙利语的后介词短语具有同样由移位生成的结果,尽管英语中这一移位是隐性的。按照这一思路,我们有可能减少基本操作的数量,并降低其复杂度,这一分析可以说是最简的。但是,这些结构仍然是非常复杂的。本书中所介绍的其他理论没有一种是需要用如此膨胀的结构来分析介词和名词短语的组合的。例(0)中的结果不能由英语数据来说明,由此无法从语言输入而获得。提出这类结构的理论需要提出一个特征只能在(特定的)限定语位置上来核查的普遍语法(看第chap-innateness章和第chap-acquisition章有关普遍语法和语言习得的内容)。更多有关(隐性)移位的讨论请看[Section2.3]Haider2016a。局部选择与功能投射应用功能性中心语来表示语序会带来另一个问题。在经典的CP/IP"=系统和这里讨论的所有其他理论中,一个范畴代表带有同样分布的一类宾语,即NP(或DP)表示代词和复杂名词短语。中心语选择哪些带有一定范畴的短语。在CP/IP"=系统中,I选择了一个VP和一个NP,而C选择了一个IP。在较新的分析中,这类选择机制并不能简单地奏效。因为例(1b)中已经有移位了,我们需要在“dasBuchdemMannzugeben”(这个人要给的这本书)这句话中处理TopP或FocP。由此,um不能简单地选择非定式IP,而是分别选择TopP、FocP或IP。需要确定的是,TopPs和FocPs标记了包含其中的动词的形式,因为um只能与zu"=不定式相搭配。umdemManndasBuchzugebenPREPDET人DET书INF给`给那个人这本书'umdasBuchdemMannzugebenPREPDET书DET人INF给`把这本书给那个人'由此,范畴系统、选择机制和特征投射就需要被处理得更为复杂,尤其是将他们与简单基础生成的系统或从IP移出以构成一个新的IP的系统相比。支持Cinque99a-u理论的观点是有问题的,原因是:Cinque认为范畴AdverbPcategory!functional!Adverb用来组合副词和VP。这是一个空的功能中心语,它将动词性投射作为它的补足语,而且,副词表面上位于该投射的限定语位置。在这些系统中,副词短语需要传递动词的屈折属性,因为具有特定的屈折属性的动词(定式的、带zu的不定式、不带zu的不定式、助词)需要由更高位置的中心语来选择(请看第Beispiel-GPSG-Kopfeigenschaften页和第节)。当然也有应用一致关系来表示的,但是这样的话,所有的选择关系就是非局部的,而且所有的选择关系都不是一致的。进而,这个分析的更为严重的问题有由副词修饰副词的结构,这与英语中的部分前置和位于动词前的副词性的非短语结构有关,请看[Section5]Haider97a。副词问题的一个特殊情况是否定问题:Ernst92a更为细致地研究了否定的句法,并指出否定不能附加在几个不同的动词投射(1a,b)、形容词(1c)和副词(1d)上。Kencouldnothaveheardthenews.KenAUX不AUX听DET新闻`Ken不可能听过这个新闻。'Kencouldhavenotheardthenews.KenAUXAUX不听DET新闻Ken可能没听过这个新闻。`'a[notunapproachable]ART不不可接触人物`一位不可接触的人物'[Notalways]hassheseasonedthemeat.AUX总是AUX她DET肉`她并不总是把肉调味。'如果所有这些投射只是没有更多属性(有关动词形式、形容词词性、副词词性)的NegP,这就无法说明他们不同的句法分布。否定很明显只是更多普遍性问题的一个特殊的情况,因为副词在传统意义上可以附加在形容词上构成形容词短语,而不是Chinque意义上的副词短语。比如说,例(1)中的副词oft(经常)修饰lachender(笑)构成了形容词短语oftlachender,这就像未被修饰的形容词助词lachender:它修饰Mann(人),并位于它前面。einlachenderMannART笑人`一位大笑的人'einoftlachenderMannART经常大笑人`一位经常大笑的人'当然我们可以想出针对上述三个问题的解决方案,并且应用一致Agree关系来加强非局部的选择限制,但是这样就会破坏选择的局部限制(请看[110]Ernst92a和本书第sec-locality节的讨论),而且这比依存关系的直接选择来说要复杂得多。与前述我们讨论的局部问题相关的是安放附着词的特殊功能投射的假设:如果用SpeakerPcategory!functional!Speaker,那么第一人称单数的附着词就可以移到正确的限定语位置上,而对于HearerPcategory!functional!Hearer来说,第二人称附着词就可以移到正确的位置上[31]Poletto2000a-u,然后我们需要的用来编码所有与小句有关的特征的特殊投射(另外,我们当然可以假定有非局部的一致关系来表示分配事实)。在这些特征之外,范畴标签包括允许更高的中心语选择包括附着词的小句的信息。在其他方法和转换语法早期变体中,选择被认为是严格的局部限制locality,这样更高的中心语只能接触到所嵌套范畴的属性,这些范畴直接与选择相关(citealp[223]Abraham2005a、Sag2007a),而不是有关小句内中心语的论元是否是说话者还是听话者,或者小句中的一些论元是否被实现为附着词。局部限制将在第节讨论。特征驱动的移位最后,对于特征驱动的移位来说有一个概念上的问题,这一问题由GisbertFanselow指出:[27]Frey2004atopic(focus(informationstructure(提出了KontrPcategory!functional!Kontr(对比短语),并且Frey2004b-u提出了TopPcategory!functional!Top(话题短语)(请看Rizzi97a-u提出意大利语中的TopP和FocPcategory!functional!Foc(焦点短语)、Haftka95a、Grewendorf[85,240]Grewendorf2002a、Grewendorf2009a、[19]Abraham2003a、[224]Laenzlinger2004a以及[18]Hinterhoelzl2004a有关德语中带TopP和/""或FocP的分析)。构成成分根据信息结构的状态移到这些功能中心语的限定语位置上。Fanselow2003b指出这种基于移位的理论对于中场的元素排序来说与最简方案的假设是不一致的。这一现象的原因是,有时移动的发生是为了给其他元素创造出位置(利他移位)movement!altruistic。如果句子的信息结构要求离动词最近的宾语既不是焦点,也不是焦点的一部分,那么离动词最近的宾语不应在小句处得到最重音。这可以通过去重音化而得到,即通过将重音移到另一个成分上,或者进而,如例(1b)所示,通过将宾语从接受结构重音的位置上移到不同的位置上。dassdiePolizeigesternLinguistenverhaftete	CONJDET经常昨天语言学家逮捕`昨天警察逮捕了语言学家'dassdiePolizeiLinguistengesternverhaftete	CONJDET警察语言学家昨天逮捕`昨天警察逮捕了语言学家'在西班牙语中,部分焦点可以不通过特殊的语调就能实现,但是只能通过利他移位来将宾语移出焦点位置。请看德语中有关利他的多重前置fronting!apparentmultiple的讨论。为了核对由信息结构的属性驱动的一些特征,就不可能认为成分移到了树中的特殊位置上。因为特征核对是现今最简方案中移位的前提条件,我们就不得不假定有一个特殊的属性,该属性只有触发利他移位的功能。Fanselowadjunct(([Section4]Fanselow2003b、[8]Fanselow2006a)还指出,话题、焦点和句子副词的语序限制可以被一种理论来合理地描述,其中,论元首先跟他们的中心语一个接一个地组合(最简方案中的术语是合并)Merge,然后状语可以再附加在任意投射层上。在句中焦点位置前的句子副词有这样的语义解释:由于句子副词像焦点敏感算子,他们不得不直接位于他们所指的元素之前。这样,话语(话题)中不属于焦点的元素需要在句子副词之前。这样就必须假定一个话题位置来解释中场的局部重新排序。这一分析在LFG和HPSG中也有所涉及。相应的分析在后面的章节中有更为详细的讨论。标记在最简方案中,乔姆斯基label(尝试将组合操作及其基础简化到越简单越好。为了支撑这一假设,他认为带有较少具体语言知识的普遍语法比带有较多具体语言支持的普遍语法而言在演化的角度来看更为可行[135]Chomsky2008a。基于此,他取消了理论中的投射层、语迹、指标和“相似的描述技术”[138]Chomsky2008a。剩下只有合并和移位,即内部和外部合并。内部和外部合并将两个句法对象和合并进一个表示为集合,的更大的句法对象中。和要么是词汇项,要么是内部复杂的句法对象。内部合并将宾语的一部分移到了它的边缘。更为具体来说,句法对象的一部分被复制,然后该复制成分被放到整个对象的边缘位置。该复制的原型不再与发音有关(移位的复制理论)。CopyTheoryofMovement内部合并元素的结果是集合,,其中,是的一部分。外部合并也得到有两个元素的集合。但是,两个独立的对象被合并了。通过合并创造出的对象有一定的范畴(一套特征)。例如,如果我们将元素和相组合,那么我们就会得到l,,,其中,l是结果对象的范畴。这一范畴也叫做标记(label)。因为我们认为所有成分都有中心语,指派到,的范畴要么是范畴,要么是范畴。[145]Chomsky2008a讨论了下面两条决定集合标记的规则。Label-BerechnungLabel1在H,中,H和LI,H是标记。Label2如果在内部合并到上,构成了,,然后的标记是,的标记。正如Chomsky指出的,这些规则不是没有问题的,因为在所有的情况中,标记并不是唯一确定的。比如说有两个词汇元素的组合,如果例(0a)中的H和都是词汇项(LI),那么H和可以是结果结构的标记。Chomsky意识到,这可以得到派生结构,但是他指出这一点是没有问题的,并且将之忽视了。Chomsky在他的Chomsky2013a这篇论文中给出了两个词汇项到组合的分析。解决这一问题的方法是界定所有的词汇项的组合都包括一个功能元素和一个根元素Marantz97a,Borer2005a-u。根在定义上不是标记另一个通过定义被排除为标记的范畴Conj,它表示并列conjunction[45--46]Chomsky2013a。这是使得并列可以完成的规定。请看下面。而且功能元素范畴决定了组合的范畴[47]Chomsky2013a。这样的分析只能遭到反对:最简方案的目标是简化理论设想,使得语言习得和语言演化的模型变成可行的,但是为了简化基本概念,它就要设计名词不仅仅是一个名词,而是需要一个功能元素来说明名词所在的范畴。Chomsky的光杆短语结构Chomsky95b-u的所有设想是在理论中去除一元分支结构,我们不清楚他们为什么在现在通过借壳而被重新介绍出来,带有一个额外的空元素就更为复杂了。(i.a)中旧的规则对应于(i.b)中的二元组合。NNNN-funcroot在(i.a)中,一个词汇名词投射到N中,而且在(i.b)中,一个根元素与功能名词中心语相组合为一个名词范畴。诸如范畴语法和HPSG的理论可以将词汇项直接组合,无须假定任何助词投射或空元素。请看Rauh2013a有关早期的转换语法、HPSG、构式语法、角色参数语法RoleandReferenceGrammar和Chomsky2013a提出的基于根的新构建主义理论中句法范畴的处置的比较分析。Rauh总结道,句法和语义信息的直接连接是需要的,而且Marantz和Borer的新构建主义必须遭到反对。更多有关新构建主义的评论请看Wechsler2008a和[Sections6.1and7]MWArgSt。代词与动词投射相组合带来的问题与上面讲的问题有关。在“Heleft”的分析中,代词he是一个词汇元素,所以需要为“Heleft”的标记负责,因为left是最简方案中的一个内在复杂动词投射。结果可以是一个名词标记,而不是动词的。为了解决这一问题,[46]Chomsky2013a指出he有一个复杂的内部结构:“也许是D-pro”,即he(可能)是在一个可见的限定词和一个名词之外构成的。两个非词汇项外部合并的情况(如一个名词短语和动词短语)并没有在Chomsky2008a中有所讨论。[43--44]Chomsky2013a指出,短语XP与XP,YP的标记无关,如果XP进一步被移位的话(或者是在移位的复制理论中被复制CopyTheoryofMovement)。Chomsky认为在XP,YP组合中的两个短语中的一个要被移位,因为不然的话就无法标记了(第12页)。fn-labeling-gleiche-Kategorie他的解释是自相矛盾的:在第11页,Chomsky认为两个具有相同范畴的实体的组合的标记就是这个范畴。但是在他对并列的操作中,他认为其中一个并列项需要被移动,否则完整的结构就不能得到标记了。我们用下面并列的例子来进行说明。Chomsky认为,“ZandW”这个表达式可以这样进行分析:首先,Z和W合并。这一表达式再与Conj相组合ex-coord-a,然后在下一步Z被提升ex-coord-b。Chomsky-problems-of-projection-coordinationex-coord-a[Conj[ZW]]ex-coord-b[Z[Conj[ZW]]由于中的Z只是一个复制,它并不能表示标记,而且可以获得W这个标记。并且要求Z和相组合,Conj不能是标记,所以完整结构的标记是Z。正如BobBorsley(p.c.2013)跟我指出的,这种方法对于两个用and连接的单一名词短语的并列来说会得出错误的结论,因为并列的结果是一个复数NP,而不是像第一个并列成分的单一个体。HPSG之类的理论可以通过将特征集中起来表示这一特点,这些特征在并列结构中共享(句法特征和非域内特征,请看[202]ps2)。进而,整个分析并不能解释为什么(i.b)被规则排除出去了。bothKimandLee都Kim和Lee`Kim和Lee两个人'bothKimorLee都Kim或Lee并列的信息必须是“orLee”这个表达式的一部分,这样可以用来与“andLee”进行对比。另一个问题是这个标记应该是W的标记,因为Conj并不能决定是哪个标记。这就导致,我们需要在Z和W之间进行选择,来决定的标记。按照Chomsky的逻辑,要么是Z,要么是W必须要移位来使得标记成立。Chomsky2013a在脚注40中提到了这一问题,但是并没有给出解决办法。Chomsky讨论的另一特殊情况是LI与非LI的内部合并。按照规则Label1,标记应该是。按照Label2,标记应为(也请看Donati2006a-u)Chomsky讨论了代词“what”与“youwrote”相组合的例子。ex-what-you-wrotewhat[C[youwrotet]]什么C你写t`你写了什么'如果标记是由Label2决定的,那么在GB框架下就有一个被叫做CP的句法对象;由于这个CP是疑问句,它可以在1中被用作wonder的补足语。如果标记是由Label1来决定的,我们就可以在1中得到被用为read的宾格宾语的宾语,即在GB术语中与DP相关的成分。Iwonderwhatyouwrote.我想知道什么你写`我想知道你写了什么'ex-i-read-what-you-wroteIreadwhatyouwrote.我读什么你写`我读了你写的东西。'例(0b)中的“whatyouwrote”被叫做自由关系从句relativeclause!free(。Chomsky关于自由关系从句的方法是有趣的,但是不能在全维度上来描述这一现象。问题是包括关系代词的短语可以是复杂的(与Donati的观点相比,请看[930--932]Citko2008a)。[47]Chomsky2013a承认,目前关于自由关系小句中的标记还有很多问题,所以他承认这类标记相关的问题还有很多。1是从[333]BG78中选出的英语例子。[155]Bausewein90和[78]Mueller99b中的德语例子在例2中。I'llread[whicheverbook]yougiveme.我-将读任何一个书你给我`我会读你给我的任何一本书。'Ihrkönntbeginnen,[mitwem]ihrwollt.你可以开始[PREP谁你想要`你想跟你想要的任何一个人开始。'[155]Bausewein90.[WessenBirne]nochhalbwegsinderFassungsteckt,pflegtsolcherleiErloschenezumeiden;[谁的灯泡/头也半路PREPDET插座AUX使用这样消逝INF避免`那些还有一半智慧的人想要避免这类空字符;'ThomasGsella,taz,1997/2/12,第20页。[WessenSchuhe]"`danach"'besprenkeltsind,hatkeinenBaumgefundenundwarnichtzueinemBogeninderLage.[谁的鞋此后有斑点的AUXAUX不树找到和AUX不INFART弓PREPDET位置`那些鞋上有斑点的人找不到树也不能按照弧形来小便。'taz,tazmag,1998年8月8日或9日,第XII页。由于“wessenSchuhe”(谁的鞋)并不是一个词汇项,我们就需要应用规则Label2,而且对于这种情况没有额外的规则。这就意味着整个自由关系从句“wessenSchuhedanachbesprenkeltsind”被标记为CP。对于标记为CP的-1和0中的自由关系从句并不是期望得到的结果,因为他们作为主谓词的主语或宾语,并由此被标记为DP。但是,因为“wessenSchuhe”是一个复杂短语,而不是一个词汇项,Label1并不适用,所以就没有作为DP的自由关系小句的分析。所以说,看起来我们必须回到GR81提出的GB分析,至少是针对德语的例子。Gross和vanRiemsdijk认为自由关系从句包括一个空名词,它像正常名词一样被关系小句来修饰。在这一方法中,关系短语的复杂度是无关的。只有空中心语是跟整个短语的标记是有关的。假设空中心语是有问题的,因为它只能在那些由状语修饰的情况下用作论元,叫做关系小句[97]Mueller99b。对于后期重新发现这一问题的也可以看[187]Ott2011a。这一问题可以在HPSG理论下得到解决,通过假定一个一元投射,投射到关系小句的合适的范畴。我也应用一元投射来分析所谓的非匹配的自由关系小句Mueller99b。在非匹配的自由关系小句的结构中,关系小句填充了论元槽的位置,该位置并不对应于关系短语的属性Bausewein90。Bausewein讨论了下面的例子,这里关系短语是一个PP,但是自由关系代词填充了kocht(做饭)的宾格槽。exe(i)Siekocht,woraufsieAppetithat.她做饭是什么她胃口AUX`她想吃的时候就做饭。'请看[60--62]Mueller99b关于语料库的例子。最简方案不能应用一元投射。Ott2011a提出了关系短语的范畴被投射的分析,但是他没有为非匹配自由关系小句提供解决方案(第187页)。这跟Citko的分析是一样的,其中内部合并的XP可以提供标记。还有许多其他有标记,或者更多没有标记的研究。比如说,一些最简方案研究者想要标记删除,进一步提出没有标记的句法。如*OPG2011a所指出的,这类分析使得最简方案更像依存语法了。不清楚的是这些模型中的哪些能够处理非匹配的自由关系小句。[Section5.3.3]GO2009a在他们的依存语法中给出了自由关系小句的分析,但是否认了非匹配部分的存在(第78页)。他们提出,关系短语是自由关系小句的根或标记,所以他们具有与最简方案中非匹配自由小句相同的问题。正如[73]GO2009a和[327]OPG2011a所声称的:在(他们所谓的)依存语法中通常没有所谓的空中心语。也没有一元分支投射。这看起来使得带有关系短语YP的自由关系小句可以作为XP,使得XP是一个在KC77a的旁格层级中更高的范畴,Bausewein90发现的一般特征(也可以看[60--62]Mueller99b和[4]Vogel2001a)。为了能够表示相关事实,需要有一个与(i)中的worauf标记不同的元素或者标记。但是,一旦空中心语在分析中被批准,Label1到ex-what-you-wrote的应用就是不需要的了,因为应用可以导致针对ex-i-read-what-you-wrote的两个分析:一个是空名词性中心语,一个是被直接标记为NP的ex-what-you-wrote。有人可能会指出,在许多可能的推导式中,最为精简的会胜出,但是转移派生限制transderivationalconstraint的假设会得到不想要的结果[Section5]Pullum2013a。relativeclause!free)Chomsky2013a放弃了Label2中的标记条件,而是将它替换为普遍的标记规则,该规则约束两个短语的内部和外部合并。他区分了这两种情况。在第一种情况中,标记是可能的,因为集合XP,YP中的两个短语中的一个被移走了。这一情况早在上面就有所讨论。Chomsky还写了另一种情况:“X和Y在相关方面是相同的,并具有相同的标记,可以被看作是SO的标记”(第11页)。他草拟了第13页的疑问小句的分析,这里疑问短语具有一个Q特征,提取出Q短语的剩余的句子也具有Q特征。这类标记也许也可以被用作正常句子的标记,它们包括主语和动词短语在人称和数上的一致关系。这些特征可以对句子的标记负责。具体的细节还没有制定出来,但是可以肯定的是一定比Label2更为复杂。在Chomsky2005a和Chomsky2013a中较为一致的属性是由合并对象之一专门确定的标记。正如BobBorsley跟我指出的,这对1这类疑问/关系短语来说是有问题的。withwhomPREP谁`跟谁'0中的短语既是一个介词短语(因为第一词是介词),也是一个疑问/关系短语(因为第二个词是一个疑问/关系词)。所以,我们需要的是PPs的正确标记,像0中那种是覆盖了从子节点到母节点的不同属性的界定良好的形式。HPSG通过区分包括言语信息的中心语特征和包括提取和疑问/关系元素的信息的非局部特征来解决这一问题。中心语特征从中心语投射,母节点的非局部特征是子节点的非局部特征减去那些特定中心语或者在特定句法配置中没有约束的特征。[926]Citko2008a提出了一个分析,其中所有的子节点都可以为母节点贡献力量。结果是像P,D,N的复杂标记。这是一个高度复杂的数据结构,而且Citko并没有给出任何信息,它所包含的相关信息是如何被接触到的。带有标记P,D,N的宾语是P、D,还是N呢?我们可以说P有优先性,因为它在嵌套最少的集合中,但是D和N在一个集合中。冲突的特征怎么办呢?介词如何选择DP决定了D,N是D,还是N?任何情况下,很清楚的是形式化会包括循环的关系,它们为了接触到他们的特征能够发掘出子集的元素。这增加了整体的复杂度,而且在HPSG中不受推崇,因为它对每个语言对象都使用了一种词性。更多最近有关合并的构成的标记和大量的过度分析请看FSP2016a。总之,我们可以说,用来简化理论和减少具体语言内在知识所提出的标记,只能用在一定数量的规定中。例如,需要空功能词假设的词汇元素的组合,他们的唯一目的在于决定特定词汇元素的句法范畴。如果这与语言实体相关,有关标记的知识、各自的功能范畴,以及需要忽视标记的范畴可以被称为具体语言内在知识的一部分,而且什么都得不到。剩下的只是没有在最简方案的方向下取得进展的特别复杂的怪异分析。进而,还有很多以观察和实验为依据的问题以及大量未能解决的问题。结论是,二元组合的标记不应该由Chomsky2008a,Chomsky2013a提出的方式来决定。另一种用来计算标记的选择是用功能论元结构中的功能函子作为标记[145]BE95a。这是范畴语法Ajdukiewicz35a-u,Steedman2000a-u和Stabler的最简方案Stabler2010b中所采用的方法。为了使范畴语法的方法奏效,有必要将范畴x/x指派给状语,这里x表示状语所附加的中心语的范畴。例如,形容词与名词对象组合成一个名词对象。这里它的范畴是n/n,而不是adj。同样,Stabler的方法并不适用于状语,除非他想将名词范畴指派给定语形容词。解决这一问题的一个方法是假定对状语和他们的中心语有一个特殊的组合操作(请看[Section3.2]FG2002a)。这一组合操作与HPSG理论中的中心语"=状语范式是等效的。Stabler有关合并的形式化的内容将在第节进行讨论。限定语、补足语和理论的剩余部分[146]Chomsky2008a认为,每个中心语都只有一个补足语,但是有任意数量的限定语。在标准理论中,Xtheory@theory(specifier(complement(限制是在普遍的模式中最多只有一个补足语,而且假定这样的结构至多是二元分支的branching!binary:在标准理论中,一个词汇中心语与所有它的补足语相组合以构成一个X。如果在一个短语中最多有两个子节点,只能有一个补足语(双及物动词句子被分析为一个空中心语emptyelement允准一个额外的论元;请看Larson88a有关空动词中心语的研究以及[Sections6.1and7]MWArgSt对包括小v方法的批判性评价)。在标准的理论中,只有一个限定语。这一限制现在被废黜了。Chomsky写道,限定语和补足语的区别现在可以从哪些元素与他们的中心语相合并的顺序推导出来:首先合并的元素是补足语,其他所有的(即后期合并的)是限定语。这一方法对单价动词的句子来说是有问题的:根据Chomsky的观点,单价动词的主语不是限定语,而是补足语。PaulineJacobson(p.c.2013)指出,不及物动词的问题可以这样来解决,即假定最后合并的元素是限定语,而所有非最后合并的元素是补足语。这就可以解决非及物动词的问题以及ex-he-knows-and-loves-this-record-MP中动词的并列问题。但是,它不能解决coordination-head-final中中心语后置语言的并列问题。进而,现在的最简方案允许多重限定语,这就与Jacobson式的思想不一致了,除非希望对非首先合并元素的状态施以更为复杂的限制。有关这一问题的更多细节将在第节详细讨论。除此之外,认为与词组相合并的句法对象是限定语的分析不允许两个动词直接并列,如1所示:fn-Chomsky-on-Specifiers[46]Chomsky2013a在Chomsky-problems-of-projection-coordination中提出了并列分析:按照这一分析,动词可以直接合并,而且其中一个动词可以在生成的较后步骤中在连词前后移动。正如在上一节所指出的,这样的分析并不有利于达到关于具体语言的内在知识进行最小化的假设的目的,因为它完全无法清楚地解释这样的并列分析是如何由语言学习者习得的。所以说,我在这里不考虑并列分析。Chomsky2013年的这篇文章的另一个创新之处在于,他取消了限定语的概念。他在第43页的脚注27写道:“关于限定语的问题有非常多有启发性的文献,但是如果这里的推理是正确的,它们就不存在了,而且这些问题不能用公式来表示。”这么说是对的,但是这就意味着在最简方案框架下关于限定语概念的解释到目前为止不再有意义了。如果我们按照Chomsky的说法,过去几十年一大部分的语言学研究都没有价值了,而且需要重新做。Chomsky在他早期的研究中没有严格遵守线性化的思路,但是某种程度上我们要确认那些叫做限定语的实体在被叫做限定语的成分所在的位置上实现的。这就意味着下面的评论即使在Chomsky式的假设中也是相关的。ex-he-knows-and-loves-this-record-MPHe[knowsandloves]thisrecord.他知道和喜爱这专辑`他知道并喜爱这张专辑。'例如,在[264]Steedman91a提出的分析中,and(作为中心语)首先与love合并,然后得到的结果与knows合并。这一组合的结果是具有所组合部分相同句法属性的复杂对象:结果是一个复杂动词,该动词需要一个主语和一个宾语。在与两个动词并列的组合之后,结果需要与“thisrecord”和“he”组合。“thisrecord”在所有相关方面都像补足语。但是,按照Chomsky的定义,它应该是一个限定语,因为它与合并的第三次应用所组合。得到的结果是不清楚的。Chomsky认为合并不能说明成分序列。按照他的观点,线性化发生在语音形式(PhonologicalForm,PF)层。其中的限制在他近期的文章中并没有提及。但是,如果作为补足语或限定语的范畴在Kayne的工作中对线性化起到了作用,而且在Stabler的观点中(请看第Abschnitt-MG节),“thisrecord”需要在“knowsandloves”之前线性化,这是与事实相反的。这就意味着类似于范畴语法的有关并列的分析是不可行的,而且剩下的唯一选择看起来就是假定knows与宾语相组合,然后两个VPs并列在一起。[61,67]Kayne94a-u追随[303]WC80a-u的观点,提出这一观点,并认为第一个VP中的宾语被删除了。但是,[471]Borsley2005a指出这一分析作出了错误的预测,因为ex-whistled-a可以从ex-whistled-b推导出来,尽管这些句子在意义上是不同的。也可以看[102]BV72、[192--193]Jackendoff77、[143]Dowty79a、[104--105]denBesten83a、[8--9]Klein85和Eisenberg94a关于转换语法的早期版本中相似观察与相似观点的评论。ex-whistled-aHobbswhistledandhummedthesametune.Hobbs吹口哨和哼DET相同调子`Hobbs吹着口哨,并哼着相同的曲调。'ex-whistled-bHobbswhistledthesametuneandhummedthesametune.Hobbs吹口哨DET相同调子和哼DET相同调子`Hobbs吹着相同的调子,并且哼着相同的调子。'因为语义解读不能在语音形式层面看到像删除这样的处理[Chapter3]Chomsky95a-u,语义之间的区别不能通过删除材料的分析而得到解释。在另一个VP并列结构的分析中,有一个语迹与“thisrecord”相关。这可以是一个右节点提升(Right-Node-Raising)分析。Borsley2005a指出这样的分析是有问题的。下面的例子选自他讨论的几个有问题的例子(也可以看[615]Bresnan74a-u)。ex-tried-persuade-and-convince-himHetriedtopersuadeandconvincehim.他尝试INF劝说和说服他`他试着劝说并说服他。'Hetriedtopersuade,butcouldn'tconvince,him.他尝试INF劝说但是AUX-NEG说服他如果him没有重读的话,第二个例子是不合乎语法的。相较而言,ex-tried-persuade-and-convince-him是合格的,即使带有不重读的him。所以,如果ex-tried-persuade-and-convince-him是右节点提升的一个例子,对比就是出乎意料的。由此,Borsley排除了右节点提升分析。针对ex-he-knows-and-loves-this-record-MP这类句子分析的第三个可能性提出了非连续的成分概念,并应用了两次材料:两个VPsknowsthisrecord和lovesthisrecord与第一个VP的并列是非连续的。(请看Crysmann2000a和BS2004a在HPSG理论框架下提出的分析。)但是,非连续的成分通常不在最简方案框架下进行应用(比如说[67]Kayne94a-u)。而且,Abeille2006a指出,有证据表明词汇元素直接并列的结构。这就意味着我们需要上面讨论的CG分析,这就会导致刚刚讨论的限定语/补足语状态的问题。进而,Abeillé指出,在中心语后置的语言中,如韩语和日语中,NP并列在基于合并的分析中是有困难的。1是一个日语的例子。coordination-head-finalRobin-toKimRobin-和Kim`Kim和Robin'在第一步中,Robin与to合并。在第二步,Kim被合并。因为Kim是限定语,我们可以认为Kim位于中心语之前,因为在中心语后置的语言中,限定语就是这样排列的。Chomsky试图去除标准理论的一元分支结构,该结构用来将代词和限定词等词汇项投射到完整的短语中,相关工作由Muysken82a完成。Muysken应用二元特征min和max将句法对象区分为最小(词或者类词的复杂对象)或最大(代表完整短语的句法对象)。这样的特征系统可以用来将代词和限定词描述为[+min,max]。但是,像give一类的动词被归类为[+min,max]。它们必须为了达到[+max]-层而投射。如果限定语和补足语需要是[+max]的,那么限定词和代词满足了这一要求,而不需要从通过X投射到XP"=层。在Chomsky的系统里,min/max的区别表现在中心语的完整性(完整=短语),以及作为词汇项的属性。但是,在Muysken和Chomsky的观点中有一个小但是很重要的区别:关于上面讨论的并列数据的认识问题。在理论的范畴系统中,可以将两个s相组合以得到一个新的、复杂的。这一新对象基本上具有简单的s具有的相同的句法属性(请看[51]Jackendoff77和*GKPS85a)。在Muysken的系统里,可以构成这样的并列规则(或连词的词汇项)来将两个min并列为一个min项。在Chomsky的系统里,无法定义一个相似的规则,因为两个词汇项的并列不再是一个词汇项了。正如Chomsky最近提出的最简方案,范畴语法Ajdukiewicz35a-u和HPSG理论(PollardSagps、[39--40]ps2)并不(严格)遵循理论。所有的理论都将符号NP指派给代词(有关CG请看[p.615]SB2006a-u,有关为了量化来加入词汇类型的提升请看[Section4.4]Steedman2000a-u)。短语“likesMary”和词“sleeps”在范畴语法(snp)中具有相同的范畴。在所有的理论中,没有必要为了将像tree的名词与限定词或状语相组合而从投射到。在受限不定式中的限定词和单价动词在许多HPSG理论的分析中没有从层投射到XP层,因为每个语言对象(一个空的或)的价的属性足够决定他们的组合潜力以及句法分布(Mueller96a、Mueller99a)。如果最简的属性对于现象的描述是必要的,那么在HPSG理论中就应用了二元特征lex(PollardSag[172]ps、[22]ps2)。但是,这个特征对于区分限定语和补足语来说不是必要的。这一区分由将论元结构列表()的元素匹配到限定语与补足语的特征(分别缩写为和)的配价列表的原则所决定。一些作者提出了主语、限定语和补足语的三分法。严格来说,动词投射中的限定语是英语类这类层次结构化语言中动词的最间接论元。由于论元结构列表是由KC77a提出的间接层级所决定的,该列表中的第一个元素是动词的最间接论元,而且该论元匹配到上。中的元素实现为英语等SVO型语言的动词左边的成分。中的元素实现为他们的中心语右边的成分。[34,364]GSag2000a-u提出的方法认为中心语-补足语短语将动词及其论元组合起来,这与ex-he-knows-and-loves-this-record-MP这类并列具有相同的问题,因为VP的中心语不是一个词。正如上面提到的,对于非连续成分的多领域方法对于ex-he-knows-and-loves-this-record-MP的分析而言是可行的(请看Crysmann2000a和BS2004a)。但是,正如Abeille2006a强调的,词汇项的合并需要在原则上是可行的。还需要注意的是,并列的HPSG方法不能代替MP。原因是HPSG方法包括并列的特殊的语法规则,而MP则只有合并。所以说,组合型规则的额外介入并不是MP的一个选项。但是,中心语的这一限制可以由指称lex的特征所替代,而不是词或词项的属性。PollardSag和SagGinzburg提出了英语的平铺结构。由于其中一个子节点被标记为词汇,它自然得到的规则就不会将中心语与它的补足语的子集相结合,然后第二次将结果与其他补足语组合起来。所以说,像ex-gave-john-a-book-a的结构就被排除了,因为“gaveJohn”不是一个词,所以不能用作规则中的中心语节点。ex-gave-john-a-book-a[[gaveJohn]abook]给JohnART书`给John一本书'ex-gave-john-a-book-b[gaveJohnabook]给JohnART书`给John一本书'与ex-gave-john-a-book-a不同的是,只有像ex-gave-john-a-book-b的分析是被允许的;也就是说,中心语与它所有的论元相组合。另一方面是假定二元分支结构(MuellerHPSGHandbook、[Section1.2.2]MOeDanish)。在这一方法中,中心语补足语模式不能限制中心语节点的词或短语的状态。HPSG理论中的二元分支结构对应于MP的外部合并。在前面两节中,我们讨论了Chomsky有关标记的界定问题以及词汇项的并列问题。在下面一节中,我将探讨Stabler对最简方案的合并的界定,他的界定是与标记相关的,而且其中一个版本没有涉及到上面讨论的问题。我将指出他的形式化方法与HPSG的表示非常一致。Xtheory@theory)最简方案、范畴语法和HPSG理论在MinimalistGrammar(MG)(CategorialGrammar(CG)(这一小节,我将最简方案、范畴语法和HPSG理论联系起来。对范畴语法和HPSG理论不太熟悉的读者可以略过这一节,或者在看完第chap-feature-descriptions章、第Kapitel-CG章和第Kapitel-HPSG章之后再回来。在第Abschnitt-Labeling节,我们指出Chomsky的论文在未确定的标记上留下了许多重要的细节尚未讨论。Stabler的工作相对而言接近于近期的最简方案,但是在实现方面更为准确(也请看[397,399,400]Stabler2010a关于后GB方法的形式化)。Stabler2001a说明了Kayne的剩余移位理论是如何形式化和实现的。Stabler将他对最简方案理论的形式化的方式叫做最简语法(MinimalistGrammars,MG)。关于最简方案和其变体Michaelis2001a-u的能力不足capacity!generative有许多有趣的结论。比如说,可以用MGs创造的语言也可以用树邻接语法(请看第Kapitel-TAG章)来创造。这就意味着,可以将更多的词串指派到MGs的结构中,但是,由MGs推导出来的结构并不一定必须与TAGs创造的结构相同。更多有关语法的生成能力的内容,请看第sec-generative-capacity章。尽管Stabler的工作可以看作是Chomsky的最简方案思想的形式化,Stabler采用的方法与Chomsky的方法在某些细节方面是不同的。Stabler认为两个合并操作的结果不是集合,而是偶对。偶对的中心语由指针(`'或`')来标记。像(第Abschnitt-Labeling节讨论的)括号表达式,,替换为1中的树的表达形式。在(tree-stabler-mg)中,1是中心语,2是补足语,3是限定语。指针指向包括中心语的结构的那个部分。树中的子节点是按照顺序排列的,即3在1之前,1在2之前。[402]Stabler2010a对外部合并的界定如下所示:=f是一个选择特征,f是对应的范畴。当t[=f]和t[f]组合时,结果是一棵树,其中t的选择特征和t的个别范畴特征被删除了。0中的上层树表示一个(词汇)中心语与其补足语相组合。t位于t之前。只能有一个节点的条件与Chomsky的设想是一致的,即第一个合并是与补足语complement的合并,其他合并操作是与限定语specifier的合并[146]Chomsky2008a。Stabler将内部合并界定为:Fn-SMC除了1所示的情况,Stabler的定义包括最短移位限制(ShortestMoveConstraint,SMC)的变体,这与目前所讨论的问题是无关的,所以我们在这里就不说明了。t是一棵带有t子树的树,它的特征f的值是`'。这一子树被删除了(t[f]),然后没有特征f(t)的被删除子树的复制位于限定语的位置。在限定语位置上的元素必须是最大投射。这一要求由提升的`'来表示。Stabler为1中的句子提供了一个推导的例子。whoMariepraises谁Marie表扬`Marie表扬的人'praises是一个带有两个=D特征的二元动词。这就意味着有两个限定短语的选择。who和Marie是两个Ds,而且它们填充到动词的宾语和主语位置上。得到的动词投射“Mariepraiseswho”嵌套在空的补足语下面,它被限定为wh,这样就为who的移位提供了位置,它被放在内部合并的操作下CP的限定语位置上。who的wh特征被删除了,应用内部合并得到的结果是“Mariepraiseswho”。这Seite-leeres-Objekt一分析的问题由Stabler他自己在未发表的著作[124]Veenstra98a中提出来:它对单价动词verb!monovalent(的处理是不正确的。如果动词与NP组合,按照Definition-EM中有关外部合并的定义,就会将NP看作是补足语再与第Abschnitt-Spezfikatoren-MP节中Chomsky关于限定语和补足语的定义相比较。,并将其排到中心语的右边。与ex-max-sleeps-a中的句子的分析相反,我们会得到1中的字符串的分析。Max睡觉`Max睡觉。'睡觉Max为了解决这一问题,Stabler提出,单价动词与非显性宾语emptyelement相组合(请看[61,124]Veenstra98a,他引用了Stabler尚未发表的工作,而且也采用了这一解决方法)。有了这一空宾语,得到的结构和Max是限定语,并由此被排列在1中动词的左边。Maxsleeps.Max睡觉`Max睡觉。'当然,任何一种这类分析都是规定的,而且完全是特设的。而且,它恰巧可以佐证[Section2.1.2]CJ2005a讨论的转换生成语法在方法论上的缺陷:过度地追求统一。另一种方法是,假定有一个空的动词中心语,它带有补足语sleeps和主语Max。这种分析经常被用来说明最简方案中Larson式的动词壳Larson88a理论下的双及物动词。Larson式的分析一般认为有一个叫做小v的空的动词中心语,该中心语具有致使的含义。正如我们在第sec-little-v节讨论的,Adger2003a采用了基于小v的分析来讨论不及物动词。不考虑TP投射,图fig-little-v-intransitive显示了他的分析。Maxsleeps”(Max睡觉)的基于小v的分析Adger认为带有非作格动词的句子的分析包括选择了施事的小v,而带有非宾格动词的分析包括小v,但是它并不选择N中心语。对于非宾格动词而言,他认为动词选择了主题。他证明了小v不必具有致使的含义,但是要能引入施事。不过需要注意的是手头这个例子中,sleep的主语既没有引发事件,也不必故意做某事。所以,它实际上是遭受者,而不是施事。这意味着关于空v中心语的假设纯粹是基于理论的假设,而不是针对不及物动词的语义所驱动的。如果假定了双及物构式中小v的致使贡献,这就意味着我们需要两个小vs,一个具有致使含义,另一个不具有致使含义。小v除了缺乏理论外的驱动力,这类分析仍有一些实证方面的问题(比如说并列)。更多细节可以参考citew[Sections6.1and7]MWArgSt。verb!monovalent)除了Definition-EM和Definition-IM中界定的两个操作,MG中没有其他操作了。扩展内容请看[Section3.2]FG2002a除了coordination(单价动词的问题,它还会导致第Abschnitt-Spezfikatoren-MP节讨论的问题:对于例ex-he-knows-and-loves-this-record-MP--这里重复显示为1--的动词,还没有对其直接组合的分析。ex-he-knows-and-loves-this-record-MP-zweiHe[knowsandloves]thisrecord.他知道和喜爱这专辑`他知道并喜爱这张专辑。'原因是“know、and和loves”的组合包括三个节点,而且“knowsandloves”与“thisrecord”的合并会让“thisrecord”变成结构的限定语。由此“thisrecord”就会排在“knowsandloves”的前面,这是与事实相反的。由于可以由MGs生成的语言包括那些可以由TAGs和组合性范畴语法Michaelis2001a-u生成的语言,范畴语法分析的存在暗示了并列例子可以由MGs推导出来。但是对于语言学家而言,事实是无法生成不太重要的所有字符串(语法的能力较弱)。由语法所允准的实际结构大多是重要的(强势能力)。导向性的最简语法与范畴语法除了重新界定范畴,并列问题还可以通过改变合并的定义来解决,改变后的定义允许中心语确定与之组合的论元的方向:[p.635]Stabler2010b指出,决定论元的位置与选择特征一起重新定义了外部合并。em(t,t[x])=等号的位置决定了论元需要实现在中心语的哪一边。这对应于范畴语法的前向forwardapplication与后向应用backwardapplication(请看第sec-forward-backward-application节)。Stabler将这一形式叫做导向性的MG(DMG)。这一MG的变体避免了单价动词的问题,而且如果我们假设连词是带有变量范畴的中心语,这些变量范畴选择与它左边或右边具有同样范畴的元素,这样并列数据也没有问题了。know和love都会将宾语选在右边,主语选在左边,而且这一要求可以被传递到“knowsandloves”。不过,需要注意的是,这一传递使得选择复杂范畴变成了可能,这是我在MuellerUnifying忽略的事实。简单特征还是复杂特征的选择问题将在第sec-selection-features-vs-categories节讨论。coordination)Stabler针对DMGs提出的有关CG的更多细节请看[264]Steedman91a。Pollard88a最简语法与中心语驱动的短语结构语法用“”和“”标记结构的中心语直接对应于中心语的HPSG表示。由于HPSG是一个基于符号的理论,所有相关的语言学层面都表示为描写(音位、形态、句法、语义、信息结构)。1给出了一个例子:词grammar的词汇项。[word]phon&'gramərsynsemloc&[loc]cat&[cat]head&nounspr&DETcont&[grammar]inst&Xgrammar的词性是名词。为了构成一个完整的短语,它需要一个限定词。这由值为DET的特征来表示。语义信息列于下。更多细节请看第chap-HPSG章。由于我们穷尽地处理句法方面,只有使用过的特征的子集是相关的:配价信息和词类信息以及与短语的外在分布有关的某些形态句法属性表现为synsem""loc""cat路径下的特征描述。这里特别有趣的特征是所谓的中心语特征。在词汇中心语和它的最大投射之间共享中心语特征。复杂的层级结构也被模拟为特征值偶对。复杂语言对象的组成成分通常表现为复杂对象的表示的部分。比如说,特征head-daughter的值是一个特征结构,该特征结构模拟了包括短语的中心语的语言学对象。中心语特征原则1指向这个子节点,并且保证中心语与完整对象的中心语特征是相同的。headed"=phraseheaded"=phrasesynsemloccathead1head-dtrsynsemloccathead1指称是由具有相同数字的盒子表示的。[30]GSag2000a-u在给出了子节点属性的值的列表中表示了语言对象的所有子节点。特征head-daughter的值与子节点列表中的一个元素是同指的:gs-ahead-dtr&1dtrs&1,gs-bhead-dtr&1dtrs&,1[2]和是对语言学对象的描写的简称。0中的两个描写的要点在于中心语子节点与两个子节点的一个是相同的,这分别由和前的1决定。在第一个特征描述中,第一个子节点是中心语,而在第二个描述中,第二个子节点是中心语。由于中心语特征原则,整个短语的句法属性由中心语子节点决定。也就是说,中心语子节点的句法属性对应于Chomsky界定的标注label。这一说法直接对应于Stabler使用的术语:gs-a等同于1,而且gs-b等同于1。tabular[t]@l@l@l@l@a.&forestbaseline[<]forest&b.&stabler-bforestbaseline[>]foresttabular由[Chapter9]ps2讨论的,基本信息的另一种结构应用了head-daughter和non-head-daughters这两个特征,而不是head-daughter和daughters。这给出了像1的特征描述,它直接对应于Chomsky的基于集合的表示,并在第Abschnitt-Labeling节进行了讨论,而且这里重复为1。head-dtr&non-head-dtrs&,,例(0a)中的表示并不包括和的线性排列顺序。组成成分的线性排列顺序受到排列规则linearizationrule的限制,这些规则是受到(直接)统制之外的独立的限制。在Definition-IM中有关内部合并的定义对应于HPSG中的中心语"=填充语范式[164]ps2。Stabler的派生规则删除了t[f]子树。HPSG是单调的,即在语法允准的结构中是没有什么成分被删除的。与在更大的结构中删除t相反的是,包括空元素emptyelement(NB,不是树)的结构是被直接允准的。请看*BMS2001a在HPSG模式中有关无语迹的研究,以及本书中的第[Chapter7]MuellerGS章和第chap-empty章有关空元素的讨论。在Stabler的定义和HPSG范式中,t被实现为结构中的填充语。在Stabler关于内部合并的界定中,没有提及中心语子节点的范畴,但是[164]ps2将中心语子节点限制为一个定式的动词投射。[17]Chomsky2007a认为,所有的操作,除了外部合并操作,都是短语层面的。Chomsky认为CP和v*P是短语。如果这一限制被合并到Definition-IM的定义中,标记t的限制也会相应地得到扩展。在HPSG中,像1的句子被看作是VPs,而不是CPs。所以PollardSag要求中心语-填充语范式中的中心语子节点应该是动词性的,以对应于Chomsky提出的限制。Bagels,Ilike.百吉饼我喜欢`百吉饼,我喜欢。'所以说,除了较小的表现上的差异,我们可以得到这样的结论,内部合并与中心语-填充语模式的形式化是非常相似的。在HPSG和Stabler的定义之间一个重要的区别是移位在HPSG中不是特征驱动的。这是一个重要的优势,因为特征驱动的移位不能处理所谓的排他移位Fanselow2003b的例子,也就是说,组成成分移位的发生是为某个位置上的其他成分让地方(请看第sec-feature-driven-movement节)。在一般的理论和Stabler在内部合并上的形式化一方面与HPSG中的另一方面的差别是,后者并没有有关填充语子节点的完整性(或配价饱和度)的限制。填充语子节点是否是最大投射或者不是,当语迹是与它的中心语相组合的时候,这一问题受到局部强加的限制。这就使得我们有可能分析1这类没有剩余移位的句子。也可以参考MOe2013b有关丹麦语中宾语转换的分析,这种现象与没有剩余移位的动词前置是有关的。这一分析没有剩余分析所具有的任何问题。GelesenhatdasBuchkeiner.读AUXDET书没有人相反,Stabler被迫要假定1中这样的分析(请看G.GMueller98a有关剩余移位的分析)。第一步,“dasBuch”被移出了VP1,而且在第二步,空置的VP被提前,如1所示。Hat[dasBuch][keiner[VPgelesen]].[VPGelesen]hat[dasBuch][keiner].[281]Haider93a,[Section2]dKM2001a和Fanselow2002a指出,这类剩余移位分析对于德语是有问题的。Fanselow所指的需要剩余移位分析的唯一现象的问题是多重前置fronting!apparentmultiple(请看Mueller2003b针对更多相关数据的讨论)。Mueller2005c,Mueller2005d,MuellerGS针对这些多重前置提出了另一种分析方法,他提出了前场中的空动词中心语概念,但是并没有假定例(0b)中“dasBuch”这类附加语和论元是从前场成分中提取出来的。与剩余移位分析相反的是,范畴语法Geach70a,HN94a中的论元组成的机制被用来确保句中论元的合理实现。[20]Chomsky2007a已经将论元组成作为他的TPs和CPs的一部分。所以说,这两个剩余移位和论元组合被看作是最近的最简方案的理论。但是,HPSG看起来不需要太多的理论演算,并由此在简约的推理方面更为可取。最后,需要指出的是,所有的转换方法在跨域提取AcrosstheBoardExtraction方面都是有问题的,就像ex-bagels-i-like-and-ellison-hates和1中,一个元素对应于几个空位。ex-atb-minimalismex-bagels-i-like-and-ellison-hatesBagels,IlikeandEllisonhates.[205]ps2.Themanwho[Maryloves]and[Sallyhates]computedmytax.DET男人CONJMary爱和Sally恨计算我的税`Mary喜爱但是Sally憎恨的这个男人计算了我的税。'[2]Gazdar81针对GPSG解决了这一问题,而且该解决方案也被用在了HPSG中。最简方案的圈子试着通过引进侧边移位sidewardmovement来解决这些问题Nunes2004a-u,其中组成成分可以被插入子树中。所以(0a)的例子中,Bagels从hates的宾语位置拷贝到like的宾语位置,然后两个拷贝都与前置的元素相关。Kobele批评这些方法,因为他们过度生成,而且需要复杂的过滤。相反,他提出将GPSG式的机制引入到最简方案的理论中Kobele2008a。进而,移位矛盾[Chapter2]Bresnan2001a可以通过不在填充语和空位之间分享所有的信息来避免,这一方法在转换框架下是不可能的,它通常会假定填充语和空位具有同一性--正如在移位的复制理论下CopyTheoryofMovement--认为一个推导过程包括一个宾语的多重复制,其中只有一个是被拼写出来的。也可以参考Borsley2012a跟多有关基于移位的方法的问题与困惑。MG和HPSG更深一层的区分在于中心语"=填充语不是分析长距离议论的唯一模式。正如在第fn-Kayne-Extraposition页的脚注fn-Kayne-Extraposition所指出的,右侧(外置extraposition)和前置都有错位的情况。尽管这些问题一定会分析为长距离依存,他们与其他长距离依存在很多方面都有不同之处(请看第Abschnitt-Fernabhängigkeiten节)。在HPSG框架下的有关外置的分析,请看Keller95b、Bouma96和Mueller99a.除了长距离依存,在HPSG中当然还有其他不在MG和最简方案下的现象。这些模式包括描述那些没有中心语的结构或者需要捕捉结构内部的分布属性,这些不能轻易地在词汇分析中有所表示(比如说wh-和关系代词的分布情况)。请看第Abschnitt-Phrasale-Konstruktionen节。Chomsky2010a比较过助词转换的基于合并的分析与HPSG式的分析,并且批判说HPSG应用了十个模式,而不是一个(合并)。GSag2000a-u区分了三类带有移位的助词的结构:带有前置副词和wh"=问句(1a,b)的倒置句、倒置的感叹句(1c),以及极性疑问句(1d):Undernocircumstancesdidshethinktheywoulddothat.PREP没有条件AUX她想他们AUX做那`在没有任何条件下,她想他们会那样做。'Whosebookareyoureading?谁的书AUX你读`你在读谁的书?'AmItired!AUX我累`我累了!'DidKimleave?AUXKim离开`Kim离开了吗?'Fillmore99a在他的构式语法框架下分析助词倒装捕捉了几个不同的应用语境,并指出在不同的语境下有语义和语用的不同。每个理论都需要说明这些。进而,我们不需要十个模式。正如范畴语法那样,我们有可能在词汇项中来决定助词或空的中心语(请看第Abschnitt-Phrasal-Lexikalisch章更多有关词汇和短语分析的讨论)。除此之外,每个理论都要在某种程度上解释这十种区别。如果有人认为这与句法无关,那么这就需要在语义组成上进行模拟。这就意味着在这一点上没有理由倾向于一种理论。MinimalistGrammar(MG))CategorialGrammar(CG))原子特征的选择与复杂范畴的选择BE95a指出最简方案与范畴语法非常相似,而且我已经在MuellerUnifying和前面几节中讨论了最简方案和HPSG的相似之处。但是,我忽略了最简方案和范畴语法、依存语法、LFG、HPSG、TAG和构式语法中关于选择的一个重要区别:在理论的前一版本中选择的是一个单一特征,而后面的理论选择的是特征束。这一区别看起来很小,但是后果很严重。在第Definition-EM页给出的Stabler关于外部合并的定义移除了选择特征(=f),以及被选择的元素(f)的相应特征。在一些文献和本书的导言部分,选择特征作为不可预测的特征被标记为u。不可预测的特征需要被核对,然后从Stabler界定的语言对象中移除。他们被核对的事实通过选择被表示出来。所有不可预测的特征都需要在句法对象被送到表层(语义和语音)之前核对。如果不可预测的特征没有被核对,推导就失败了。[Section3.6]Adger2003a明确地讨论了这些假说的后果:一个选择的中心语核对被选择宾语的特征。我们不可能去检查与中心语相组合的包含在宾语中的元素的特征。只有在最高点的特征,所谓的根节点,可以由外在合并核对。在复杂对象内部的特征被核对的唯一途径是通过移位。这就意味着中心语可以不与部分饱和的语言对象相组合,即带有一个未核对选择特征的语言对象。我会讨论[95]Adger2003a提供的例子所指的决策设计。名词letters选择了一个P,而且Ps选了一个N。例(1a)的分析如图fig-letters-to-peter-adger所示。letterstoPeter信PREPPeter`给Peter的信'lettersto信PREPAdger2003a对“letterstoPeter”的分析例(0b)中的字符串被规则排除了,因为介词to的不可预测的N特征没有被核查。所以这就整合了这样的限制,所有的依存元素必须在核心机制中是最大的。这就使得(1)这类例子在最直接的方式上来分析是不可能的,即包括一个复杂的介词和缺少限定词的名词:vomBusPREP.DET公交车在复杂描述可以用来描写依存关系的理论中,被依存的对象可以部分满足。所以比如说HPSG中,像vom(from.the)的复合介词可以选择一个,它是一个缺少限定语的名词性投射:N[Det]例(0)中的描述是一个内部饱和了的特征值偶对的缩写(请看第Abschnitt-Spr节)。这里的例子只针对于差异的解释,因为有很多方法用来解释单一特征合并系统的情况。例如,我们可以假设一个DP分析,并且让复杂介词选择一个完整的NP(没有不可预测特征的范畴N)。另外,我们可以假设确实有一个完整的PP,带有通常所假定的所有结构,而且介词和限定词的组合在发音阶段形成。第一种方法省略了假定NP分析的选项,正如Bruening2009a在最简方案框架下所提出的。除了符合介词这个例子,还有其他想要将未饱和的语言对象相组合的例子。我已经讨论了上面的并列例子。另一个例子是像德语、荷兰语和日语这样的语言中的动词复杂形式。当然,还有一些看法并不认同动词复杂式(G.GMueller98a,Wurmbrand2003b),但是他们也不是没有问题的。有些问题在前一节中也有所讨论。简单总结这一小节的内容,需要说明的是构建进合并概念的特征核对机制比范畴语法、词汇功能语法、HPSG、构式语法和TAG中运用的选择来说更为严格。依我看,它有些太严格了。小结总之,我们可以说最简方案约束的计算机制与特征驱动的移位理论是有问题的(如转移派生限制与标记),而且空功能范畴的假设有时是特异的。如果我们不希望假定这些范畴在所有的语言中都存在的话,那么提出两种机制(合并与移位)并不能表示出语法的简化,因为每个必须约定的功能范畴都构成了整个系统的复杂度的一部分。标记机制在细节上也没有完全实现,它并不能描述它所生成可以描述的现象,由此它应该被替换为范畴语法和HPSG中基于中心语/函项的标记。MinimalistProgram(MP))总结这一节与第sec-summary-gb节类似。我首先评论语言习得,然后是形式化。关于语言习得的解释[135]Chomsky2008a将MP中的理论归纳为原则参数分析,并将MP参数确定在词汇中。也可以参考[396]Hornstein2013a。UG被界定为可能包括非具体语言的成分,他们是由基因决定的[7]Chomsky2007a。UG包括非绑定的合并,而且由语法生成的表达式的条件必须满足由语音和概念"=意愿接口赋予的限制。另外,特征的具体集合被看作是UG的一部分[6--7]Chomsky2007a。这些特征的具体本质没有被详细地解释,而且结果是,UG的力量有时是模糊的。但是,在不同的语言学阵营里有达成一致的地方,就像Chomsky并不认为我们在第Abschnitt-merkmalsgetriebene-Bewegung节介绍的功能投射的部分也构成UG的一部分(但是,像CR2010a这类作者认为功能投射的层级是UG的一部分)。由于参数的存在,第sec-acquisition-gb节用来反对GB理论关于语言习得的论断与最简方案中语言习得的论断具有相关性。请看第chap-acquisition章有关语言习得和原则参数模型,以及基于输入的方法的深入讨论。Chomsky在最简方案中的主要目标是简化语言的形式属性的理论假说,以及用来使得他们和相关部分是我们的基因天赋的一部分的计算机制。但是,如果我们总结一下本章的主要内容,我们很难相信最简方案可以达到这一目标。为了得到一个不及物动词的简单句,我们需要几个空的中心语和移位。特征可以是强的或者弱的,一致操作跨越了几个短语界限的句法树。而且为了得到正确的结果,必须要确保一致只能看到最近的元素(principle-locality-of-matching)--(def-intervention)。这与范畴语法相比是一个巨大的工程,范畴语法只组合临近的成分。范畴语法可以从输入获得(请看第Abschnitt-UDOP节),而它很难想象当输入很强时,特征驱动移位,而在输入很弱时就没有驱动这样的事实应该单独从数据获得。形式化第sec-formalization-gb节评论了转换语法直到上世纪九十年代都缺少形式化的研究内容。在最简方案中对形式化的态度没有变化,所以只有很少的最简方案理论的形式化与应用实现。Stabler2001a说明了如何对Kayne的剩余移位理论进行形式化与实现。在Stabler的实现中他的系统可以在以下网址找到:http://www.linguistics.ucla.edu/people/stabler/coding.html.2016/03/05.,并没有转移派生限制transderivationalconstraint,也没有计数numeration在[Chapter9]Veenstra98a中有一个计数的词汇项。这个词汇项包括一个计数的集合,它包括功能中心语,可以用在某类句子中。比如说,Veenstra认为计数可以用在带有二价动词的句子和主语位于首位的句子、带有单价动词的嵌套句、带有单价动词的wh"=疑问句,以及带有单价动词的极性疑问句。这一计数集合的一个元素对应于一个特别的句法配置以及在构式语法思想下的短语构式。Veenstra的分析并不是在最简方案下找到的计数的概念的形式化。通常,它被认为是包含所有句子生成所需词汇项的计数。如(i)所示,复杂句可以包括带有不同句子类型的句子的组合:DerMann,derbehauptethat,dassMariagelachthat,stehtnebenderPalme,dieimletztenJahrgepflanztwurde.DET男人CONJ声称AUXCONJMaria大笑AUX站PREPDET棕榈树CONJPREP上次年种植AUX`声称Maria大笑的男人站在去年种下的棕榈树旁。'在(i)中,有两个带有不同价的动词的关系小句,一个带有单价动词的嵌套句和主句。在传统上对计数的理解中,Veenstra需要假定有无限数量的词汇包含所有可能的句子类型的组合。,他不提倡一致Agree等关系(请看see[132]Fong2014a)。下述也是Stabler对于最简方案和GB理论的应用:没有大量的语法。Stabler的语法是小型的,作为概念和句法的证据。没有形态变化morphology测试句具有(i)中的形式。thekingwill-seatthekinghave-seat-enthekingbe-seat-ingtheking-swill-shavebeeneat-ingthepie,没有对多重一致关系的处理agreement[Section27.4.3]Stabler2010b,尤其是没有语义semantics。PF和LF的过程也没有被模拟。	请看SE2002a有关PF和LF"=移位以及复制部分删除的建议(第285页)。对于这一点的应用并不是微不足道的。由SandiwayFong开发的语法和计算系统与理论的大小和相似度基本一致FG2012a,Fong2014a:语法片段较小,句法编码,如直接在短语结构中标注[Section4]FG2012a,而且远远位于理论的后面。进而,他们不包括形态变化。没有实现拼写输出,所以最终无法剖析和生成任何话语。*[1221]BPYC2011a关于Fong的研究的评论是错误的:“但是因为我们有时转而考虑计算机处理的问题,正如可以‘核对’中心语或标签的特征的能力一样,这就指出了一个合理的疑问,我们的框架是否是可计算的。所以我们值得指出移位的拷贝理论,与上述描述的‘搜索与标注’过程的域内导向,可以作为有效的剖析器来计算应用;更多细节内容请看Fong(2011)。”如果一个软件不能剖析一个简单句,那么我们就不能说它是有效的,因为我们不知道程序所缺少的部分会不会使得它失效。进而,我们将这个软件与其他程序进行比较。正如上面已经讨论的,Fong没有用Chomsky所谓的标签,但是相反他应用了第Kapitel-PSG章所描述的短语结构语法。这里的标准由基于约束的理论中语法的应用所决定;比如说,在上世纪九十年代开发的德语、英语和日语的HPSG语法作为Wahlster2000a-ed-not-crossreferenced的一部分,用来对口语和大范围的LFG或CCG系统的分析。这些语法可以分析口语(涉及到日程安排和旅行计划领域)或书面语中83的话语。语言知识被用来生成和分析语言结构。在一个方向上,我们会得到词串的语义表示,而另一个方向上,可以创造出给定语义表征的词串。形态分析在分析有精细的形态标记的语言中自然发生的数据上是不可缺少的。在本书的其他部分,我们会在每章的开头讨论其他理论开发的语法和计算系统。在GB/MP内部缺少大规模的语法片段的原因有可能跟最简方案团体的基本假设的变化相对较快有关系:quoteZitat-Stabler在最简方案中,驱动的中心语通常被叫做探针(probe),移动的元素叫做目标(goal),而且对于激活句法效应的特征之间的关系有许多不同的看法。[p.229]Chomsky95a-u首先提出,当要求被满足,特征表示核对与删除的特征。第一个设想几乎立即被修改,这样只有特征的一部分子集,即“形式化”、“不可预测”的特征在成功的生成过程中被核对操作所删除(Collins,1997、[§4.5]Chomsky95a-u)。另一个想法是一些特征,特别是某些功能范畴的特征,也许受限是未被估值的,通过与其他元素进入合适的句法配置而被估值(Chomsky2008a、Hiraiwa,2005)。而且近期的工作采取了这样的观点,特征从未被删除[p.11]Chomsky2007a。这些问题都没有得到解决。[397]Stabler2010aquote为了开发全部的语法片段,一般需要至少三年时间,我们可以计算一下Barriers(1986)发表的文章和Stabler的应用(1992)发表的时间间隔。特别是大量的语法需要几个研究者在几年甚至是几十年间开展国际合作。如果基本假设在短期内重复变化,这一过程就会被打断。延伸阅读这一章主要参考了Adger2003a的内容。关于最简方案的其他教科书有Radford97a-u、Grewendorf2002a和*HNG2005a。Kuhn2007a对比了基于约束的LFG和HPSG的现代生成分析。Borsley2012a对比了HPSG中的长距离依存分析和GB/最简方案中的基于移位的分析。Borsley讨论了基于移位的方法的四种数据:没有填充语的抽取、带有多个空位的抽取、填充语和空位不匹配的抽取,以及没有空位的抽取。有关标注、理论的弃用,以及第Abschnitt-Labeling--Abschnitt-MG节Stabler的最简语法和HPSG的比较内容可以在MuellerUnifying找到。BG2005a所著的《韵律短语、非连续及否定的辖域》(IntonationalPhrasing,Discontinuity,andtheScopeofNegation)推荐给想深入了解的读者。作者将否定量词的分析与最简方案(在Kayne的影响下)和范畴语法(在Steedman的影响下)中的宽域进行了比较。Sternefeld2006a-u对开发了德语转换语法分析的句法(839页)进行了详细的介绍,这些语法(以转换为模)几乎与HPSG理论下的语法完全一致(按照层级体系在配价列表中的论元的特征描述)。Sternefeld的结构是最小的,因为他没有假定任何功能投射,如果他们不能在所研究的语言中被激发的话。Sternefeld对于其他研究中理所当然的某些方面特别严格。Sternefeld把他的书看作是教科书,通过这本书,学生可以学到在提出理论时如何做到论述的一致性。基于此,这本书不仅仅推荐给学生和博士生。SR2012a讨论了理论语言学中尤其注重描述这一章和前面章节中的内容。我非常能够理解作者们在面对分析的模糊性、论述的方式、研究的实证基础、修辞的套路、免疫尝试以及科学标准的一般方面时感到的挫折:现成的例子就是Chomsky2013a的文章《投射的问题》(ProblemsofProjection)。模糊性:在这篇文章中,perhaps出现了19次,may出现了17次,还有许多if。一致性:结论并不一致。请看本书第fn-labeling-gleiche-Kategorie页的脚注fn-labeling-gleiche-Kategorie。论证方式:术语限定语被弃用了,并声称与该术语有关的问题不存在了。由此,他们不属于这个世界。请看本书第fn-Chomsky-on-Specifiers页的脚注fn-Chomsky-on-Specifiers。免疫性:Chomsky这样论述空范畴原则:“明显的例外不能解释放弃一般性所到达的程度,而是探求能够解释在哪儿和为什么这样的深层原因”(第9页)。这一说法显然不是正确的,但是人们会想为了驳斥一个现有的分析需要多少证据。特别是在《投射的问题》(ProblemsofProjection)这篇文章中,我们必须要想知道为什么这篇文章在《关于短语》(Onphases)发表之后的五年就发表了。反对最初设想的证据是强有力的,而且好几个点都是Chomsky2013a自己提出来的。如果Chomsky想要应用他自己的标准(1957年他的引用,请看第quote-Chomsky-Formalisierung页),以及普遍的科学方法(奥卡姆的剪刀),结果自然是回到基于中心语的标注的分析中了。有关这篇文章的更多评论,请看第Abschnitt-Labeling节和第Abschnitt-Spezfikatoren-MP节。但是,我并不认同这篇文章的普遍的、悲观的观点。依我之见,病人的情况是很危险,但是他还没有死。Sternefeld和Richter所著的一篇评论指出,语言学领域的变化太快了,以后拥有MIT毕业的头衔已经不能保证能找到工作了(脚注16)。可见,一些科学家在某些实证方面的问题上在重新寻找方向,以合适的方式来处理数据并在不同理论阵营中提高交流水平是走出这场危机的一个途径。自从上世纪九十年代以来,可以看见更多的注意力放在了基于经验的研究上(特别是在德国),比如说,在语言合作研究中心(inguisticCollaborativeResearchCenters,SFBs)或年度语言事实(LinguisticEvidence)大会上进行的工作就是如此。正如上面引用的评论者所说的,在未来,只关注Chomsky决定的句法范畴已经是不够的了,如“Heleft”这个句子(请看第Abschnitt-Labeling节)。语言学博士论文需要有经验的事实,这些事实能够证明作者真正理解语言。进而,博士论文以及其他出版作品应该能够说明作者不仅仅了解一种框架,而是了解较大范围的相关描写性与理论性文献。正如我在第Abschnitt-MG节和MuellerUnifying中所讨论的,我也将在下面的章节和特别讨论的内容中来说明,在不同分析之间有很多相似性,而且他们在某些方面是一致的。走出当前的危机的方式就在于基于实证的原则、多理论的背景教育以及对于后代的培养。简言之:所有的老师和学生都应该阅读Sternefeld和Richter所著的医疗报告。我恳求学生们不要在读完之后直接放弃阅读他们的研究,而是至少在阅读完本书的其他章节之后再做出这一决定。
