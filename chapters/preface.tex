\chapter{前言}

本书是我的《语法理论》(\emph{Grammatiktheorie}) \citep{MuellerGTBuch2}这本德语书的扩展版与修订版。\footnote{
本译著是对2016年版《语法理论——从转换语法到基于约束的理论》一书的翻译,我们修改了第\ref{chapter-minimalism}章中的错误,并处理了其他一些小问题。——译者}
书中介绍了在当代理论学界发挥重要作用,或在过去做出重要贡献的如今仍占据主流地位的各种语法理论。我将这些理论的基本观点进行了解释,并将这些理论应用到德语的“核心语法”中。本书的目标语言仍然采用其德语版中所使用的目标语言,因为很多需要分析的现象不能用英语当作目标语言来解释。而且,很多理论都是在英语的基础上发展起来的,因为这些研究者的母语是英语,将这些理论应用到其他语言的研究中会很有启发性。我将说明这些理论是如何处理论元、附加语、主动/被动转换、局部重新排序(所谓的杂列)、动词位置以及长距离的短语前置(日耳曼语族的语言(除了英语)中动词位于第二位的属性)等现象。
%This book is an extended and revised version of my German book \emph{Grammatiktheorie}
%\citep{MuellerGTBuch2}. It introduces various grammatical theories that play a role in current
%theorizing or have made contributions in the past which are still relevant today. I explain some foundational
%assumptions and then apply the respective theories to what can be called the ``core grammar'' of
%German. I have decided to stick to the object language that I used in the German version of this
%book since many of the phenomena that will be dealt with cannot be explained with English as the object
%language. Furthermore, many theories have been developed by researchers with English as their native
%language and it is illuminative to see these theories applied to another language.
%I show how the theories under consideration deal with arguments and adjuncts, active/passive
%alternations, local reorderings (so-called scrambling), verb position, and fronting of phrases over
%larger distances (the verb second property of the Germanic languages without English).

第二部分探讨对理论发展很重要的根本性问题。这包括我们是否具有语言的内在知识的讨论、人类处理语言的心理语言学的证据的讨论、空语类的地位的讨论,以及我们是整体性地还是组合性地构造和获取话语的问题,也就是,我们是使用短语结构还是词汇结构。
%The second part deals with foundational questions that are important for developing theories.
%This includes a discussion of the question of whether we have innate domain specific knowledge of
%language (UG), the discussion of psycholinguistic evidence concerning the processing of language by
%humans, a discussion of the status of empty elements and of the question whether we construct and perceive utterances 
%holistically or rather compositionally, that is, whether we use phrasal or lexical constructions.

考虑到语言学这一科学领域中有大量的术语是混乱不清的,我在导言部分专门介绍了本书后面章节中将会运用到的术语。第二章介绍短语结构语法,该语法在本书介绍的许多理论中都发挥了重要的作用。我在德语专业本科生的导论课中讲解这两章的内容(除了\ref{sec-PSG-Semantik}有关短语结构语法和语义之间的关系)。高级读者可以略过这些导论性质的章节。后续的章节安排也适用于没有前期知识的读者来理解这些理论的基本内容。有关最新的理论发展的内容更有挑战性:这些内容参见后续将要介绍的章节,以及在现今的理论讨论中相关的其它文献,我们不会在本书中重复表示这些文献,或者对其进行总结。这部分内容可供高水平的学生与学者参考。我将这本书作为高年级本科生研讨课的教材,用它来讲解各种理论的句法方面。这些课件可在我的网页上下载。本书的第二部分更有挑战性,它包括对难点问题和当下研究文献的讨论。
%Unfortunately, linguistics is a scientific field 
%with a considerable amount of terminological chaos. I therefore wrote an introductory
%chapter that introduces terminology in the way it is used later on in the book. The second chapter
%introduces phrase structure grammars, which plays a role for many of the theories that are covered
%in this book. I use these two chapters (excluding the Section~\ref{sec-PSG-Semantik} on interleaving
%phrase structure grammars and semantics) in introductory courses of our BA curriculum for German
%studies. Advanced readers may skip these introductory chapters. The following chapters are
%structured in a way that should make it possible to understand the introduction of the theories
%without any prior knowledge. The sections regarding new developments and classification are more
%ambitious: they refer to chapters still to come and also point to other publications that are
%relevant in the current theoretical discussion but cannot be repeated or summarized in this
%book. These parts of the book address advanced students and researchers. I use this book for teaching
%the syntactic aspects of the theories in a seminar for advanced students in our BA. The slides are
%available on my web page. The second part of the book, the general discussion, is more ambitious and contains the discussion
%of advanced topics and current research literature.

本书只介绍相对较近的理论发展。对于历史文献的回顾,可参见,比如说 \citet{Robins97a-u}和\citet{JL2006a-u}。本书并不包括整
合语言学\isce{整合语言学}{Integrational Linguistics} \citep{Lieb83a-u,Eisenberg2004a,Nolda2007a-u}、
优选论\indexotc(\citealp{PS93a-u};\citealp{Grimshaw97a-u};G.\ \citealp{GMueller2000a-u})、角色与参照语法\isce{角色与参照语法}{Role and Reference Grammar} \citep{vanValin93a-ed}以及关系语法
\isce{关系语法}{Relational Grammar} \citep{Perlmutter83a-ed,Perlmutter84b-ed}的内容。我将这些内容留到以后的版本中。
%This book only deals with relatively recent developments. For a historical overview, see for instance
% \citew{Robins97a-u,JL2006a-u}. I am aware of the fact that chapters on
%Integrational Linguistics\is{Integrational Linguistics}
%\citep{Lieb83a-u,Eisenberg2004a,Nolda2007a-u}, Optimality Theory\indexot (\citealp{PS93a-u};
%\citealp{Grimshaw97a-u}; G.\ \citealp{GMueller2000a-u}), Role and Reference Grammar\is{Role and
%  Reference Grammar} \citep{vanValin93a-ed} and Relational Grammar\is{Relational Grammar}
%\citep{Perlmutter83a-ed,Perlmutter84b-ed} are missing. I will leave these theories for later editions.

德语书的最初版本只计划写400页,但是最后的成书规模超出了这一计划:德语教材的第一版有529页,第二版有564页。我在英语版中加入了依存语法和英语的最简语法这两章内容,现在本书有\pageref{LastPage}页。我尽最大努力将所选的理论表述清楚,并列出所有重要的文献。尽管参考文献的列表超过了85页,我也有可能没有列出全部的文献。我对此和其他问题表示歉意。
%The original German book was planned to have 400 pages, but it finally was much bigger: the first
%German edition has 525 pages and the second German edition has 564 pages. I
%added a chapter on Dependency Grammar and one on Minimalism to the English version and now the
%book has \pageref{LastPage} pages. I tried to represent the chosen theories appropriately and to cite all important work. Although the list of
%references is over 85 pages long, I was probably not successful.
%I apologize for this and any other shortcomings.

%% -*- coding:utf-8 -*-

\section*{本书的版本}
%\section*{Available versions of this book}

本书的官方版本是PDF文档,可在语言科学出版社的网页上直接下载\footnote{%
\url{http://langsci-press.org/catalog/book/177}
}。该网页还有打印版的链接。由于本书的内容非常多,我们决定将其分成两卷。第一卷包括所有的理论描述,第二卷是讨论部分。这两卷都包括参考文献和索引的完整列表。第二卷从第\pageref{part-discussion}页开始。由此,打印版与PDF文档中的部分是相同的。
%The canonical version of this book is the PDF document available from the Language Science Press
%webpage of this book\footnote{%
%\url{\lsURL}
%}. This page also links to a Print on Demand version. Since the book is very long, we decided to
%split the book into two volumes. The first volume contains the description of all theories and the
%second volume contains the general discussion. Both volumes contain the complete list of references
%and the indices. The second volume starts with page~\pageref{part-discussion}. The printed volumes
%are therefore identical to the parts of the PDF document.







%      <!-- Local IspellDict: en_US-w_accents -->


\section*{致谢}
%\section*{Acknowledgments}

%\addlines
我要感谢跟我讨论本书早期德语版的
%I would like to thank 
David Adger\ia{Adger, David}、Jason Baldridge\ia{Baldridge, Jason}、 Felix Bildhauer\ia{Bildhauer, Felix}、
Emily M.\ Bender\ia{Bender, Emily M.}、Stefan Evert\ia{Evert, Stefan}、Gisbert Fanselow\ia{Fanselow, Gisbert}, 
Sandiway Fong\ia{Fong, Sandiway}、Hans-Martin Gärtner\ia{Gärtner, Hans-Martin}、Kim Gerdes\ia{Gerdes, Kim}、Adele Goldberg\ia{Goldberg, Adele E.}、Bob Levine\ia{Levine, Robert D.}、Paul Kay\ia{Kay, Paul}、
Jakob Maché\ia{Maché, Jakob}、Guido Mensching\ia{Mensching, Guido}、Laura Michaelis\ia{Michaelis, Laura A.}、Geoffrey Pullum\ia{Pullum, Geoffrey K.}、Uli Sauerland\ia{Sauerland, Uli}、Roland Schäfer\ia{Schäfer, Roland},
Jan Strunk\ia{Strunk, Jan}、Remi van Trijp\ia{van Trijp, Remi}、Shravan Vasishth\ia{Vasishth, Shravan}、Tom Wasow\ia{Wasow, Thomas}和
%and
Stephen Wechsler\ia{Wechsler, Stephen}
%for discussion and 
%
以及对本书早期德语版本提出评论的
Monika Budde\ia{Budde, Monika}、Philippa Cook\ia{Cook, Philippa}、Laura Kallmeyer\ia{Kallmeyer, Laura}、Tibor Kiss\ia{Kiss, Tibor}、Gisela Klann-Delius\ia{Klann-Delius, Gisela}、 Jonas Kuhn\ia{Kuhn, Jonas},
Timm Lichte\ia{Lichte, Timm}、% für Kommentare zum TAG-Kapitel 
Anke Lüdeling\ia{Lüdeling, Anke}、Jens Michaelis\ia{Michaelis, Jens}、Bjarne Ørsnes\ia{Ørsnes, Bjarne}、Andreas Pankau\ia{Pankau, Andreas}、    % Chomsky 2013
Christian Pietsch\ia{Pietsch, Christian}、Frank Richter\ia{Richter, Frank}、Ivan Sag\ia{Sag, Ivan A.}
和
%and
Eva Wittenberg\ia{Wittenberg, Eva}。
%for comments on earlier versions of the German edition of this book and
%
%
我还要感谢在本书的早期版本中提出评论的Thomas Groß\ia{Groß, Thomas M.}、Dick Hudson\ia{Hudson, Richard}、
Sylvain Kahane\ia{Kahane, Sylvain}、Paul Kay\ia{Kay, Paul}、Haitao Liu (刘 海涛)\ia{Liu, Haitao}、Andrew McIntyre\ia{McIntyre, Andrew}、Sebastian Nordhoff\ia{Nordhoff, Sebastian}、Tim Osborne\ia{Osborne, Timothy}、
Andreas Pankau\ia{Pankau, Andreas}和Christoph Schwarze\ia{Schwarze, Christoph}。
%for comments on earlier versions of this book. 
感谢Leonardo Boiko和Sven Verdoolaege挑出了错别字。特别感谢Martin Haspelmath\ia{Haspelmath, Martin}对本书英文版的早期版本提出的详细评论。
%Thanks to Leonardo Boiko and Sven Verdoolaege for pointing out typos.
%Special thanks go to Martin Haspelmath\ia{Martin Haspelmath} for very detailed comments on an
%earlier version of the English book. 

%我还要感谢在此书的中文译本中给予我帮助的学者与学生们,他们是曹晓玉、刘海涛、刘晓、卢达威、詹卫东。
%I would also like to thank to 
%刘海涛、刘晓
%for comments on the Chinese version of this book.

本书是语言科学出版社出版的通过公开评审的第一本书(参见下文)。我感谢Dick Hudson、Paul Kay、Antonio Machicao y Priemer、Andrew McIntyre、Sebastian Nordhoff和一位匿名评论者对本书提出的评论。这些评论记录在\href{\lsURL}{本书的下载页面}中。除此之外,本书还经过了公开校对的阶段(也请参见下文)。有些校对者不仅做了校对的工作,还提出了具有高度价值的评论。我决定将这些评论作为附加的公开评论发布出来。在这里需要特别感谢的有Armin Buch、Leonel de Alencar、Andreas Hölzl、Gianina Iordăchioaia、Timm Lichte、Antonio Machicao y Priemer和Neal Whitman。
%This book was the first Language Science Press book that had an open review phase (see below). I
%thank Dick Hudson, Paul Kay, Antonio Machicao y Priemer, Andrew McIntyre, Sebastian Nordhoff, and one anonymous open
%reviewer for their comments. Theses comments are documented at the \href{\lsURL}{download page of
%  this book}. In addition the book went through a stage of community proofreading (see also
%below). Some of the proofreaders did much more than proofreading, their comments are highly
%appreciated and I decided to publish these comments as additional open reviews.
%%Armin Buch, 
%Leonel de Alencar,
%Andreas Hölzl,
%Gianina Iordăchioaia,
%Timm Lichte,
%Antonio Machicao y Priemer, and
%Neal Whitman
%deserve special mention here.

我感谢Wolfgang Sternefeld和Frank Richter,他们对本书的德语版做了详尽的评论。他们指出了一些错误和疏漏之处,我们在德语的第二版中进行了改正,英语版中自然就没有没有这些错误了。
%I thank Wolfgang Sternefeld and Frank Richter, who wrote a detailed review of the German version of
%this book \citep{SR2012a}. They pointed out some mistakes and omissions that were corrected in the second edition
%of the German book and which are of course not present in the English version.

感谢所有对本书进行评论和提出改进意见的学生们。特别是Lisa Deringer、Aleksandra Gabryszak、Simon Lohmiller、Theresa Kallenbach、Steffen Neu\-schulz、Reka Meszaros-Segner、Lena Terhart和Elodie Winckel。
%Thanks to all the students who commented on the book and whose questions lead to improvements. 
%Lisa Deringer,
%Aleksandra Gabryszak, % Student SS 2010 gute Fragen, GB-Verbbewegung und LMT
%Simon Lohmiller, %Student, Typos und allgemeine Anregung zu Einführungskapitel
%Theresa Kallenbach, %Studentin, GPSG
%Steffen Neu\-schulz,  % Student SS 2010 gute Fragen
%Reka Meszaros-Segner,
%Lena Terhart and
%Elodie Winckel deserve special mention.

由于本书是基于我在语言理论领域中的所有经验写成的,我想感谢那些在会议、工作坊、暑期学校期间以及通过邮件跟我讨论过语言学的学者们。
%Since this book is built upon all my experience in the area of grammatical theory, I want to thank
%all those with whom I ever discussed linguistics during and after talks at conferences, workshops,
%summer schools or via email.
特别值得列出的有Werner Abraham、John Bateman、
Dorothee Beermann、
Rens Bod、
Miriam Butt、
Manfred Bierwisch、
Ann Copestake、
Holger Diessel、
Kerstin Fischer、
Dan Flickinger、
Peter Gallmann、
%Adele Goldberg,  included above
Petter Haugereid、
Lars Hellan、
% Paul Kay, included above
Tibor Kiss、
Wolfgang Klein、 
Hans-Ulrich Krieger、
%Emily M. Bender, included above
Andrew McIntyre、
Detmar Meurers、
%Laura Michaelis,  included above
Gereon Müller、
Martin Neef、
Manfred Sailer、 
Anatol Stefanowitsch、
Peter Svenonius、
Michael Tomasello、 
Hans Uszkoreit、
Gert Webelhuth、
% Stephen Wechsler included above
Daniel Wiechmann 
和
%and 
Arne Zeschel。
%deserve special mention.

我感谢Sebastian Nordhoff针对递归(recursion\isce{递归}{recursion})这一术语的评论。
%I thank Sebastian Nordhoff for a comment regarding the completion of the subject index entry for \emph{recursion}\is{recursion}.

Andrew Murphy翻译了英文版第一章到第三章,第五章到第十章,以及第十二章到第二十三章的内容。特别感谢!
%Andrew Murphy translated part of Chapter~1 and the Chapters~2--3, 5--10, and 12--23. Many thanks for this!

我还要感谢28位校对者(
Armin Buch、 
Andreea Calude、
Rong Chen、
Matthew Czuba、
Leonel de Alencar、 
Christian Döhler、
Joseph T. Farquharson、
Andreas Hölzl、 
Gianina Iordăchioaia、 
Paul Kay、 
Anne Kilgus、 
Sandra Kübler、
Timm Lichte、 
Antonio Machicao y Priemer、
Michelle Natolo、
Stephanie Natolo、
Sebastian Nordhoff、
Elizabeth Pankratz、
Parviz Parsafar、 
Conor Pyle、
Daniela Schröder、
Eva Schultze-Berndt、
Alec Shaw、
Benedikt Singpiel、 
Anelia Stefanova、
Neal Whitman、
Viola Auermann、
Viola Wiegand),他们的工作对本书的改进提供了极大的帮助。我从他们每个人那里获得的意见比从出版商那里获得的意见还要多。有些意见是针对内容的,而不是错别字和格式的。没有一位受雇于出版商的校对人员能够发现这些错误与不一致的地方,因为出版商的雇员中没有人会懂得本书所囊括的所有语法理论。
%I also want to thank the 27 community proofreaders (\makeatletter\@proofreader\makeatother) that each worked on one or more chapters and
%really improved this book. I got more comments from every one of them than I ever got for a book
%done with a commercial publisher. Some comments were on content rather than on typos and layout
%issues. No proofreader employed by a commercial publisher would have spotted these mistakes and
%inconsistencies since commercial publishers do not have staff that knows all the grammatical
%theories that are covered in this book. 

过去的几年中,学界举办了几场理论比较的工作坊。我受邀参加了其中的三个工作坊。感谢Helge Dyvik\ia{Dyvik, Helge}和Torbjørn Nordgård\ia{Nordgård, Torbjørn}邀请我参加2005年在卑尔根举办的挪威博士生秋季学校“对比中的语言与理论”(\emph{Languages and Theories in Contrast})。Guido Mensching\ia{Mensching, Guido}和Elisabeth
Stark\ia{Stark, Elisabeth}邀请我参加了2007年在柏林自由大学举办的“比较语言与比较理论:生成语法与构式语法”(\emph{Comparing Languages and Comparing Theories:
  Generative Grammar and Construction Grammar})工作坊。Andreas Pankau\ia{Pankau, Andreas} 邀请我参加2009年在乌得勒支举办的“比较框架”(\emph{Comparing
  Frameworks})工作坊。我在跟参加这些活动的学者们的讨论中受益良多,本书也受益于这些交流。
%During the past years, a number of workshops on theory comparison have taken place. I was invited to three of them.
%I thank Helge Dyvik\ia{Helge Dyvik} and Torbjørn Nordgård\ia{Torbj{\o}rn
%  Nordg{\r{a}}rd}\todostefan{Indexeinträge für Torbjorn and Bjarne do not work} for inviting me to the fall school for Norwegian PhD
%students  \emph{Languages and Theories in Contrast}, which took place 2005 in Bergen. Guido Mensching\ia{Guido Mensching} and Elisabeth
%Stark\ia{Elisabeth Stark} invited me to the workshop \emph{Comparing Languages and Comparing Theories:
%  Generative Grammar and Construction Grammar}, which took place in 2007 at the Freie Universität
%Berlin and Andreas Pankau\ia{Andreas Pankau} invited me to the workshop \emph{Comparing
%  Frameworks} in 2009 in Utrecht. I really enjoyed the discussion with all participants of these
%events and this book benefited enormously from the interchange.

感谢Peter Gallmann\ia{Gallmann, Peter},我在耶拿期间跟他讨论了他课件中\gb 理论的内容。本书\ref{Abschnitt-T-Modell}--\ref{Abschnitt-GB-Passiv}节的内容与他的版本相似,并参考了其中很多内容。感谢David Reitte提供的组合性范畴语法的\LaTeX{}宏包,Mary Dalrymple和Jonas Kuhn提供的LFG宏包和示例结构,以及Laura Kallmeyer提供的大部分TAG分析中的\LaTeX{}资源。由于与\XeLaTeX 的兼容性问题,大部分树都调整为\texttt{forest}包的格式,但是原始的树和文本都给予了我很多灵感,没有他们,相应章节中的图绝不会像现在这样好看。
%I thank Peter Gallmann\ia{Peter Gallmann} for the discussion of his lecture notes on \gb
%during my time in Jena. The Sections~\ref{Abschnitt-T-Modell}--\ref{Abschnitt-GB-Passiv} have a
%structure that is similar to the one of his script and take over a lot. Thanks to David Reitter for
%the \LaTeX{} macros for Combinatorial Categorial Grammar, to Mary Dalrymple and Jonas Kuhn for the LFG
%macros and example structures, and to Laura Kallmeyer for the \LaTeX{} sources of most of the TAG
%analyses. Most of the trees have been adapted to the \texttt{forest} package because of compatibility issues
%with \XeLaTeX, but the original trees and texts were a great source of inspiration and without them
%the figures in the respective chapters would not be half as pretty as they are now.

我感谢Sašo Živanović实现了\LaTeX{}的宏包\texttt{forest}。这个宏包简化了树、依存图和类型层级的格式。
我还要感谢他在邮件和 \href{http://www.stackexchange.com}{stackexchange} 上给予我的具体帮助。当然,对于那些
在stackexchange上活跃的人所提供的帮助仅仅表示感谢是不够的:大部分有关本书格式的细节问题或者现在由
语言科学出版社使用的\LaTeX{}类型的应用都在几分钟内得到解答。感谢你们!因为本书是一本CC-BY版权下的公开图书,它
也是一本公开资源的著作。感兴趣的读者可以在 \url{https://github.com/langsci/25} 上拷贝这些资源。通过将本书的资源公开,我将\LaTeX{}大师们提供的资源传递下去,并希望其他人能够从中获益,并且学会按照更好看和更高效的方式来编写他们的语言学论文。
%I thank Sašo Živanović for implementing the \LaTeX{} package \texttt{forest}. It really simplifies
%typesetting of trees, dependency graphs, and type hierarchies. I also thank him for individual help
%via email and on \href{http://www.stackexchange.com}{stackexchange}. In general, those active on stackexchange could not be thanked
%enough: most of my questions regarding specific details of the typesetting of this book or the
%implementation of the \LaTeX{} classes that are used by \lsp now have been answered within several
%minutes. Thank you! Since this book is a true open access book under the CC-BY license, it can also
%be an open source book. The interested reader finds a \textsc{cop}y of the source code at \url{https://github.com/langsci/25}. By making the book open source I pass on the knowledge provided by the \LaTeX{} gurus and
%hope that others benefit from this and learn to typeset their linguistics papers in nicer and/or
%more efficient ways.

\largerpage
我还要感谢Viola Auermann、Antje Bahlke、Sarah Dietzfelbinger、Lea Helmers和Chiara Jancke所做的大量复印工作。Viola还在英译本定稿前的最后阶段帮我校对。我还要感谢我的(前)实验室成员们Felix Bildhauer、Philippa Cook、Janna Lipenkova、Jakob Maché、Bjarne Ørsnes和Roland Schäfer\ia{Schäfer, Roland}。他们在教学以及其他方面都给予我很多帮助。从2007年直到本书第一版德语版教材出版的这些年中,德语语言学系的三个终身教职中有两个职位都是空缺的,如果没有他们的帮助,我是无法完成教学任务并完成这本书的。
%Viola Auermann and Antje Bahlke, Sarah Dietzfelbinger, Lea Helmers, and Chiara Jancke cannot be thanked enough for their work at the \textsc{cop}y machines. Viola
%also helped a lot with proof reading prefinal stages of the translation.
%I also want to thank my (former) lab members Felix Bildhauer, Philippa Cook, Janna Lipenkova, Jakob Maché,
%Bjarne Ørsnes and Roland Schäfer\ia{Roland Sch{\"a}fer}, which were mentioned above already
%for other reasons, for their help with teaching. During the years from 2007 until the publication of
%the first German edition of this book two of the three tenured positions in German Linguistics were
%unfilled and I would have not been able to maintain the teaching requirements without their help and
%would have never finished the \emph{Grammatiktheorie} book.

%我要感谢Tibor Kiss针对提问技巧的建议。他的外交式的辞令给我树立了很好的榜样,我希望这点也体现在本书中。
%I thank Tibor Kiss for advice in questions of style. His diplomatic way always was a shining
%example for me and I hope that this is also reflected in this book.

%      <!-- Local IspellDict: en_US-w_accents -->

\section*{本书的出版过程}
%\section*{On the way this book is published}

我从1994年开始写我的毕业论文,并在1997年成功通过答辩。这一阶段的手稿可以在我的网页上获取。在答辩之后,我必须要找到出版商。我很高兴收到了Niemeyer的“语言学研究”系列丛书的邀请,但是同时我对价格感到震惊不已,当时每本书需要186德国马克,这还是在我没有出版商的任何帮助的情况下自己写书和排版(这个价格是纸版小说的二十倍)。\footnote{%
与此同时,Niemeyer被de Gruyter收购,并停止营业了。这本书的价格现在是139.95欧元 / 196.00美元。欧元的价格相当于273.72德国马克。
}这基本上意味着我的书是没有出版的:直到1998年,才能在我的网站上看到这本书,并随后在图书馆可以查询到。我的教授转正著作由CSLI出版社出版,价格相对来说合理多了。在我开始写教科书的时候,我就寻找不同的出版渠道,并跟无名印刷需求的出版社协商。Brigitte Narr负责Stauffenburg出版集团,她说服我在他们的出版社出版HPSG的教材。这本书的德语版属于我,这样我就可以在我的主页上出版。这一合作是成功的,由此我还可以跟Stauffenburg出版我的第二本关于语法理论的教科书。我想这本书具有更为广泛的相关性,并且可以供非德语的读者阅读。由此,我决定将之翻译为英语。不过,Stauffenburg重点出版德语书籍,我必须找到另一家出版社。幸运的是,出版界的情况与1997年相比发生了戏剧性的翻天覆地的变化:我们现在有高水平的出版社,不仅有严格的同行评审,还有着完全公开的途径。我很高兴Brigitte Narr将本书的版权卖回给我,我现在就可以在CC-BY版权下由语言科学出版社出版这本英文版教材了。
%I started to work on my dissertation in 1994 and defended it in 1997. During the whole time the
%manuscript was available on my web page. After the defense, I had to look for a publisher. I was
%quite happy to be accepted to the series \emph{Linguistische Arbeiten} by Niemeyer, but at the same time I
%was shocked about the price, which was 186.00 DM for a paperback book that was written and typeset
%by me without any help by the publisher (twenty times the price of a paperback novel).\footnote{%
 % As a side remark: in the meantime Niemeyer was bought by de Gruyter and closed down. The price of the book is now
 % 139.95 \euro / \$ 196.00. The price in Euro corresponds to 273.72 DM. 
%%This is a price increase of 47\,\%.
%} This
%basically meant that my book was depublished: until 1998 it was available from my web page and after
%%this it was available in libraries only. My Habilitationsschrift was published by CSLI Publications
%for a much more reasonable price. When I started writing textbooks, I was looking for alternative
%distribution channels and started to negotiate with no-name print on demand publishers. Brigitte Narr,
%who runs the Stauffenburg publishing house, convinced me to publish my HPSG textbook with her. The
%\textsc{cop}yrights for the German version of the book remained with me so that I could publish it on my web page. The collaboration was successful so that I also published my second textbook about
%grammatical theory with Stauffenburg. I think that this book has a broader relevance and should be
%accessible for non-German-speaking readers as well. I therefore decided to have it translated into
%English. Since Stauffenburg is focused on books in German, I had to look for another publisher. Fortunately the situation in the publishing sector changed quite dramatically in comparison
%to 1997: we now have high profile publishers with strict peer review that are entirely open access. I am very
%glad about the fact that Brigitte Narr sold the rights of my book back to me and that I can now 
%publish the English version with Language Science Press under a CC-BY license.

%      <!-- Local IspellDict: en_US-w_accents -->

\section*{语言科学出版社:归学者所有的高质量语言学出版物}
%\section*{Language Science Press: scholar-owned high quality linguistic books}

在2012年,有一群人发现出版界的情况令人难以容忍,他们一致认为有必要在公开平台上出版语言学书籍。也就是说,需要一个针对所有读者和作者公开的平台。我建立了一个网页,并征集了支持者,他们是来自全世界各地的著名语言学家,Martin Haspelmath和我随后就成立了语言科学出版社。几乎同时,DFG公布了一项公开专著的项目,我们申请\citep{MH2013a}并获得了资助(18个申请中只有两家获得了资助)。这笔钱支付给一位主任(Dr.\ Sebastian Nordhoff)、一位经济学家(Debora Siller)和两位程序员(Carola Fanselow和Dr.\ Mathias Schenner)。他们在公开专著出版社(OMP)出版平台工作,并应用转换软件来从我们的\LaTeX{}编码中生成不同的格式(ePub、XML、HTML)。Svantje Lilienthal负责OMP的文档,制作屏幕录像,并为作者、读者和编辑提供用户支持。
%In 2012 a group of people found the situation in the publishing business so unbearable that they
%agreed that it would be worthwhile to start a bigger initiative for publishing linguistics books in
%platinum open access, that is, free for both readers and authors. I set up a web page and collected
%supporters, very prominent linguists from all over the world and all subdisciplines and Martin
%Haspelmath and I then founded Language Science Press. At about the same time the DFG had announced
%a program for open access monographs and we applied \citep{MH2013a} and got funded (two out of 18 applications got
%funding). The money is used for a coordinator (Dr.\ Sebastian Nordhoff) and an economist (Debora
%Siller), two programmers (Carola Fanselow and Dr.\ Mathias Schenner), who work on the publishing
%plattform Open Monograph Press (OMP) and on conversion software that produces various formats (ePub, XML,
%HTML) from our \LaTeX{} code. Svantje Lilienthal works on the documentation of OMP, produces
%screencasts and does user support for authors, readers and series editors.

OMP在公开评论方面和社区建设的游戏化工具方面进行了扩展。所有语言科学出版社出版的图书都至少由两位外部审稿人审稿。审稿人和作者可同意出版这些审稿意见,并使得整个过程更为透明(也可以看 \citew{Pullum84a}关于期刊文章的公开评论的建议)。另外,还有可选的第二轮评审过程:公开评审。这一阶段对所有人都是公开的。整个社团都可以评论语言科学出版社出版的书籍。在第二轮评审阶段后,这通常需要持续两个月的时间,作者会进行修订,进而出版出改进的版本。这本书是经历了这个公开评审阶段的第一本书。标注了公开评审意见的版本可以通过\url{http://langsci-press.org/catalog/book/177}获得。距离本书第一版发表的两年时间中,该书大约有15000次的下载,并在全世界范围内用于教学与研究。这是每一位作者,也是每一位教师的愿景:将知识传播给每一个人。
%OMP is extended by open review facilities and community-building gamification tools
%\citep{MuellerOA,MH2013a}. All Language Science Press books are reviewed by at least two external
%reviewers. Reviewers and authors may agree to publish these reviews and thereby make the whole
%process more transparent (see also  \citew{Pullum84a} for the suggestion of open reviewing of journal
%articles). In addition there is an optional second review phase: the open
%review. This review is completely open to everybody. The whole community may comment on the document
%that is published by Language Science Press. After this second review phase, which usually lasts for
%two months, authors may revise their publication and an improved version will be published. This
%book was the first book to go through this open review phase. The annotated open review version of this book is still available via
%the \href{\lsURL}{web page of this book}. 

%距离本书第一版发表的两年时间中,该书大约有15000次的下载,并在全世界范围内用于教学与研究。%这是每一位作者,也是每一位教师的愿景:将知识传播给每一个人。语言科学出版社也有了长足的发%展,最新情况是\footnote{详细信息和图表请参考http://userblogs.fu-berlin.de/langsci-press/2018/01/18/%achievements-2017/
%}:我们有324本书的作者对出版社表示了极大的兴趣,并且有58本书已经出版。这些书按照20个系列%陆续出版,编委由二百六十三名来自六大洲四十四个国家的学者组成。这些书共有175000次的下载。来%自全世界的一百三十八位学者参与了审校工作。目前,共有296名审校者在语言科学出版社的网站上%注册。%语言科学出版社是一个基于社会团体的出版社,但是仍有专人管理,他是Sebatian %Nordhoff。他的职位是有偿的。
%我们已经成功得到了将近一百家学术组织的资助,包括哈佛大学、麻省理工学院和伯克利\footnote{资%助信息列表详见http://langsci-press.org/knowledgeunlatched
%}。如果你想支持我们,可以通过注册、出版、审校或者说服图书馆等组织来资助等方式帮助我们,详%情请见\url{http://langsci-press.org/supportUs}。
%The  first edition of this book was published almost exactly two years ago.  e book has app. 15,000 %downloads and is used for teaching and in research all over the world.  is is what every author and %every teacher dreams of: distribution of knowledge and accessibility for everybody.  e foreword of the  %rst edition ends with a description of Language Science Press in 2016.  is is the situation now:3 We %have 324 expressions of interest and 58 published books. Books are published in 20 book series with %263 members of editorial boards from 44 di erent countries from six continents. We have a total of %175,000 downloads. 138 linguists from all over the world have participated in proofreading.  ere are %currently 296 proofreaders registered with Language Science Press. Language Science Press is a %community-based publisher, but there is one person who manages everything: Sebastian Nordho . His %position has to be paid. We were successful in acquiring  nancial support by almost 100 academic %institutions including Harvard, the MIT, and Berkeley.4 If you want to support us by just signing the list %of supporters, by publishing with us, by helping as proofreader or by convincing your librarian/%institution to support Language Science Press  nancially, please refer to http: //langsci-press.org/%supportUs.

如今,语言科学出版社拥有20个语言学不同领域的系列书籍,这些高水平的编委成员由263名来自六大洲四十四个国家的学者组成。我们有58本已经出版的书籍,还有324本书的作者对出版社表示出了极大的兴趣。这些书共有175000次的下载。系列书籍的编委和作者主要用\LaTeX{}编辑的手稿,但是他们也有由语言科学出版社建立的基于网络的格式模版以及社区里的志愿者的支持。审校也是基于社群的。来自全世界的138位学者参与了审校工作。目前,共有296名审校者在语言科学出版社的网站上注册。他们的工作被记录在名人堂中:\url{http://langsci-press.org/about/hallOfFame}。

%Currently, Language Science Press has 17 series on various subfields of linguistics with high
%profile series editors from all continents. We have 18 published and 17 forthcoming books and 146
%expressions of interest. Series editors  and authors are responsible for
%delivering manuscripts that are typeset in \LaTeX{}, but they are supported by a web-based typesetting
%infrastructure that was set up by Language Science Press and by volunteer typesetters from the
%community. Proofreading is also community-based. Until now 53 people helped improving our
%books. Their work is documented in the Hall of Fame: \url{http://langsci-press.org/about/hallOfFame}.

如果你认为想阅读这类教科书的人都应得以免费拥有这些书,而且科学研究的出版不应落入利益驱动的出版社手中,那么你就应该加入语言科学出版社的群体,并在以下几个方面支持我们:可以在语言科学出版社上注册,将你的名字列在其他将近600名热心学者之中,还可以帮助校对或者修改格式,或者可以给某本书或者向语言科学出版社捐钱。我们也在寻求机构的支持,像基金会、社团、语言学系或大学图书馆。有关资助的详细信息请参见网页:\url{http://langsci-press.org/about/support}。
%我们已经成功得到了将近一百家学术组织的资助,包括哈佛大学、麻省理工学院和加州大学伯克利分校\footnote{资助信息列表详见http://langsci-press.org/knowledgeunlatched
%}。如果你想支持我们,可以通过注册、出版、审校或者说服图书馆等机构来资助等方式帮助我们,详情请见\url{http://langsci-press.org/supportUs}。
如有问题,请联系我或者语言科学出版社的主任 \href{mailto:contact@langsci-press.org}{contact@langsci-press.org}。
%If you think that textbooks like this one should be freely available to whoever wants to read them
%and that publishing scientific results should not be left to profit-oriented publishers, then you
%can join the Language Science Press community and support us in various ways: you can register with Language Science Press and have your name
%listed on our supporter page with almost 600 other enthusiasts, you may devote your time and help
%with proofreading and/or typesetting, or you may donate money for specific books or for Language
%Science Press in general. We are also looking for institutional supporters like foundations,
%societies, linguistics departments or university libraries. Detailed information on how to support
%us is provided at the following webpage: \url{http://langsci-press.org/about/support}.
%In case of questions, please contact me or the Language Science Press coordinator at \href{mailto:contact@langsci-press.org}{contact@langsci-press.org}.


~\medskip

%\noindent
\begin{flushright}
\begin{tabular}{c}
斯特凡 $\cdot$ 米勒\\
%Stefan Müller
柏林\\
2018年3月28日\\
\end{tabular}
\end{flushright}
%Berlin, \today\hfill Stefan Müller

%      <!-- Local IspellDict: en_US-w_accents -->


%\section*{Preface to the Chinese version}
\section*{中译本前言}

我非常高兴看到这本语法理论教材翻译成中文。我希望它对很多语言学专业的学生有所帮助。我衷心感谢王璐璐为这本书所付出的努力。我不知道她在建议翻译本书的时候是否知道需要做多少工作。我是不知道的。不管怎样,我很欣慰本书已翻译完成,我们也完成了这个项目。我还要感谢孙薇薇和黄思思,他们分别负责翻译了第\ref{Kapitel-LFG}章、第\ref{Kapitel-CG}章、第\ref{Kapitel-TAG}章和第\ref{Kapitel-CxG}章、第\ref{Abschnitt-Generativ-Modelltheoretisch}章、第\ref{Abschnitt-Diskussion-Performanz}章、第\ref{chap-acquisition}章、第\ref{sec-generative-capacity}章、第\ref{Kapitel-Binarybranching-locality-recursion}章、第\ref{Abschnitt-Diskussion-leere-Elemente}章、第\ref{chap-scrambling-extraction-passive}章、第\ref{Abschnitt-Phrasal-Lexikalisch}章、第\ref{Abschnitt-UG-mit-Hierarchie}章。
%Seeing this Chinese version of the grammar theory textbook makes me very happy. I hope it will be useful
%for many students of linguistics. I wholeheartedly want to thank Wang Lulu (王璐璐) for all the efforts she
%put into the book. I am not sure that she knew how much work this was when she suggested to
%translate the book. I did not. In any case I am really grateful that this book is translated and
%that we finished this project. I want to also thank XX (孙薇薇) and XX (黄思思) for translating Chapter~ and Chapter~,
%respectively.

我还要感谢14位校对者(\makeatletter\@proofreader\makeatother),他们对本译本的一章或几个章节提出了校对意见,他们的工作切实地提高了本译本的质量。我从他们每一个人那里得到的评论都比我从商业出版社获得的评论多得多。我并没有阅读所有的评论,因为它们大部分都是中文,而我并不会汉语。但是王璐璐告诉我,这些评论跟我在英语版所得到的评论一样:很多评论是关于内容的,而不是仅限于错别字和排版的问题。没有一家商业出版机构的校对员能够发现这些小的错误和缺陷,因为商业出版机构的雇员并不懂得本书所涵盖的所有理论知识。
%I also want to thank the XX proofreaders (\makeatletter\@proofreader\makeatother) that each worked on one or more chapters and
%really improved this book. We got more comments from every one of them than I ever got for a book
%done with a commercial publisher. I could not read all of the comments since most of them were in
%Mandarin Chinese, which I do not speak. But Wang Lulu told me that the comments were comparable in
%kind to the comments I got for the English version: some comments were on content rather than on typos and layout
%issues. No proofreader employed by a commercial publisher would have spotted these little mistakes and
%shortcomings since commercial publishers do not have staff that knows all the grammatical
%theories that are covered in this book. 

~\medskip

\noindent
\begin{flushright}
\begin{tabular}{c}
%斯特凡 $\cdot$ 米勒\\
Stefan Müller\\
柏林\\
2019年5月9日\\
\end{tabular}
\end{flushright}
%Berlin, \today\hfill Stefan Müller

% lulu/wsun/sisi: DONE
%      <!-- Local IspellDict: en_US-w_accents -->
