%% -*- coding:utf-8 -*-

\chapter[普遍语法与没有普遍语法的比较语言学]{普遍语法与不以(强)普遍语法为先验假设的比较语言学研究}
%\chapter[Universal Grammar and comparative linguistics without UG]{Universal Grammar and doing comparative linguistics without an a priori assumption of a (strong) UG}
\label{Abschnitt-UG-mit-Hierarchie}

下面两节讨论我认为获取概括所需的方法以及可以推导出这种概括的方式。
%The following two sections deal with the tools that I believe to be necessary to capture
%generalizations and the way one can derive such generalizations.

\section{获取概括的形式化方法}
%\section{Formal tools for capturing generalizations}

在第\ref{chap-innateness}章,我们看到以前提出的所有支持天赋语言学知识的证据实际上都是存在争议的。有些情况下,事实与讨论无关;其他情况下,事实可以用其他方式解释。有时,论证过程不符合逻辑或者大前提得不到支持。在其他情况下,论证绕圈子无解。所以,是否有天赋的知识至今仍没有答案。所有以天赋知识存在为前提的理论假设性都非常强。正如 \citet{Kayne94a-u}指出,如果假设所有的语言都有底层结构[限定语[中心语 补语]],并且移位全部向左,那么这两个基本假设必须是天赋语言知识的一部分。因为没有证据证明所有自然语言中的表达都有Kayne所说的结构。例如,读者可以检查Laenzlinger\citeyearpar[\page 224]{Laenzlinger2004a}给德语提出的方案,这一方案见第\pageref{Abbildung-Remnant-Movement-Satzstruktur}页的图\ref{Abbildung-Remnant-Movement-Satzstruktur}。按照Laenzlinger,(\mex{1}a)从(\mex{1}b)所示的底层结构推导而来:
%In Chapter~\ref{chap-innateness}, it was shown that all the evidence that has previously been
%brought forward in favor of innate linguistic knowledge is in fact controversial. In some cases, the
%facts are irrelevant to the discussion and in other cases, they could be explained in other
%ways. Sometimes, the chains of argumentation are not logically sound or the premises are not
%supported. In other cases, the argumentation is circular. As a result, the question of whether there
%is innate linguistic knowledge still remains unanswered. All theories that presuppose the existence
%of this kind of knowledge are making very strong assumptions. If one assumes, as  \citet{Kayne94a-u}
%for example, that all languages have the underlying structure [specifier [head complement]] and that
%movement is exclusively to the left, then
%, while it is possible to develop a very elegant system,
%these two basic assumptions must be part of innate linguistic knowledge since there is no evidence for the
%assumption that utterances in all natural languages have the structure that Kayne suggests. As an
%example, the reader may check Laenzlinger's proposal for German \citeyearpar[\page 224]{Laenzlinger2004a}, which is depicted in
%Figure~\ref{Abbildung-Remnant-Movement-Satzstruktur} on
%page~\pageref{Abbildung-Remnant-Movement-Satzstruktur}. According to Laenzlinger, (\mex{1}a) is derived from the underlying
%structure in (\mex{1}b):
\eal
\ex[]{
\gll weil der Mann wahrscheinlich diese Sonate nicht oft gut gespielt hat\hspace{-2pt}\\
     因为 \defart{} 男人 可能 这 奏鸣曲 \textsc{neg} 经常 好 演奏 \textsc{aux}\hspace{-2pt}\\
\mytrans{因为这个男人可能并没有经常演奏这首奏鸣曲}
%\gll weil der Mann wahrscheinlich diese Sonate nicht oft gut gespielt hat\hspace{-2pt}\\
%     because the man probably this sonata not often well played has\hspace{-2pt}\\
%\mytrans{because the man probably had not played this sonata well often}
}
\ex[*]{
\gll weil der Mann wahrscheinlich nicht oft gut hat gespielt diese Sonate\\
     因为 \defart{} 男人 可能 \textsc{neg} 经常 好 \textsc{aux} 演奏 这 奏鸣曲\\ 
%\gll weil der Mann wahrscheinlich nicht oft gut hat gespielt diese Sonate\\
%     because the man probably not often well has played this sonata\\ 
}
\zl

\noindent
(\mex{0}b) 完全不能解释,所以对应结构不能从输入中获得,所以允准该现象的原则和规则必须是天赋的。
%(\mex{0}b) is entirely unacceptable, so the respective structure cannot be acquired from input and hence the
%principles and rules that license it would have to be innate.

正如我们所见,与转换语法的大多数变体相比,有很多其他理论更趋向于表层导向的。就我们在前面章节讨论过的特定假设方面,这些其他理论间彼此也存在差异。例如,范畴语法中,在处理导致词项数量增加的长距离依存方面存在差异(见\ref{sec-pied-piping-cg})。正如 \citet{Jacobs2008a}、 \citet{Jackendoff2008a}和其他学者所展示的那样,像范畴语法这种假设每一个短语必须有一个功能符/中心语的方法不能以合适的方式来解释特定构式。有些基于承继的短语分析只是在词库中列出带有核心意义的中心语,并且让中心语出现的构式决定一个复杂表达的意义,这些方法在处理派生形态和解释论元实现的不同方式时会出现问题(见\ref{sec-val-morph}、\ref{inheritance-sec}和\ref{sec-mapping-between-levels})。因此,我们需要一种理论能够在词库中处理论元结构改变过程,并且还需要一些短语结构和相关模式。有些GB/MP的变体以及LFG、HPSG、TAG和CxG的一些变体都属于这种理论。当然这些理论中,只有HPSG和CxG的一些变体使用相同的描述方案⸺即(类型化)特征描述⸺来描述词根、词干、词、词汇规则和短语。通过使用统一的描述体系来描述所有这些对象,可以获得所有这些对象的概括。因此就可以描述特定词语与词汇规则或短语存在共同点。例如,-bar(“可……的”)的推导\isce{形态}{morphology}对应于一个带有情态词的复杂被动\isce{被动}{passive}构式。见(\mex{1}):
%As we have seen, there are a number of alternative theories that are much more surface-oriented than
%most variants of Transformational Grammar. These alternative theories often differ with
%regard to particular assumptions that have been discussed in the preceding sections. For example, there are differences in the treatment of long-distance dependencies that
%have led to a proliferation of lexical items in Categorial Grammar (see Section~\ref{sec-pied-piping-cg}). As has been shown by  \citet{Jacobs2008a},  \citet{Jackendoff2008a}
%and others, approaches such as Categorial Grammar that assume that every phrase must have a functor/""head cannot explain certain constructions in a plausible way.
%Inheritance-based phrasal analyses that only list heads with a core meaning in the lexicon and have the constructions in which the heads occur determine the meaning
%of a complex expression turn out to have difficulties with derivational morphology and with
%accounting for alternative ways of argument realization (see Section~\ref{sec-val-morph},
%\ref{inheritance-sec}, and~\ref{sec-mapping-between-levels}).
%We therefore need a theory that handles argument structure changing processes in the lexicon and still has some kind of phrase structure or relevant schemata. Some variants
%of GB/MP as well as LFG, HPSG, TAG and variants of CxG are examples of this kind of theory. Of these
%theories, only HPSG and some variants of CxG make use of the same descriptive tools ((typed) feature
%descriptions) for roots, stems, words, lexical rules and phrases. By using a uniform description for
%all these objects, it is possible to formulate generalizations over the relevant objects. It is therefore
%possible to capture what particular words have in common with lexical rules or phrases.
%For example, the \bard\is{morphology} corresponds to a complex passive\is{passive} construction with
%a modal verb. (\mex{1}) illustrates.
\eal
\ex 
\gll Das Rätsel ist lösbar.\\
    \defart{} 谜语 \textsc{cop} 可以解决的\\
    \mytrans{这个谜题是可以解开的。}
%\gll Das Rätsel ist lösbar.\\
%     the puzzle is solvable\\
\ex 
\gll Das Rätsel kann gelöst werden.\\
     \defart{} 谜语 可以 解决 \passive\\
\mytrans{这个谜题可以被解开。}
%\gll Das Rätsel kann gelöst werden.\\
%     the puzzle can solved be\\
%\mytrans{The puzzle can be solved.}
\zl
通过使用相同的描述体系来描述句法和形态,就有可能获得跨语言的概括:因为在一种语言中的屈折/派生形式,在另一种语言中可能是句法。
%By using the same descriptive inventory for syntax and morphology, it is possible to capture
%cross-linguistic generalizations: something that is inflection/derivation in one language can be syntax in another.

建构对词语和短语都相通的原则是可能的,此外,跨语言的概括或者某些语言群组的概括,都是可能的。例如,语言可以分为具有固定语序的语言以及具有更加灵活语序或者语序完全自由的语言。对应的类型可以通过在一个类型层级中的约束来表征。不同的语言可以用层级的一部分并且可以刻画每一种类型的不同约束(见\citet[\S~9.2]{AW98a})。HPSG\indexhpsg 与LFG\indexlfg 和TAG\indextag 等其他理论不同,因为在本体论上短语与单词没有区别。这意味着不存在特殊的c-结构或者树结构。对于复杂短语的描述只是简单地包含附加的特征来说明它们的子结点的信息。以这种方式,就有可能形成关于统制模式的跨语言学概括。在LFG中,c-结构规则通常针对不同的语言而不同。用统一描述体系的另外一个优势还在于可以获得词和短语规则以及词和短语之间的相似点。例如,dass (\textsc{comp})这样的标补词跟简单动词或者处在前面的并列动词有很多相同的属性:
%It is possible to formulate principles that hold for both words and phrases and furthermore, it is possible to capture
%cross-linguistic generalizations or generalizations that hold for certain groups of languages. For example, languages can be divided
%into those with fixed constituent order and those with more flexible or completely free constituent order. The corresponding types can be represented
%with their constraints in a type hierarchy. Different languages can use a particular part of the hierarchy and also formulate
%different constraints for each of the types (see \citealp[Section~9.2]{AW98a}).
%HPSG\indexhpsg differs from theories such as LFG\indexlfg and TAG\indextag in that phrases are not
%ontologically different from words. This means that there are no special c-structures or tree structures. Descriptions of complex phrases simply have additional
%features that say something about their daughters. In this way, it is possible to formulate cross-linguistic generalizations
%about dominance schemata. In LFG, the c-structure rules are normally specified separately for each language.
% Allerdings
% lässt sich auch hier eine Konversion der Ansätze feststellen:  \citet*{ADT2008a} zeigen, wie man
% Vererbungshierarchien für c-Strukturannotationen verwenden kann.
%Another advantage of consistent description is that one can capture similarities between words and
%lexical rules, as well as between words and phrases. For example, a complementizer such as \emph{dass} `that' shares a number of
%properties with a simple verb or with coordinated verbs in initial position:
\eal
\ex
\gll {}[dass] Maria die Platte kennt und liebt\\
	 {}\spacebr{}\textsc{comp} Maria \defart{} 唱片 知道 并且 喜欢\\
\mytrans{Maria知道并且喜欢这张唱片}
%\gll {}[dass] Maria die Platte kennt und liebt\\
%	 {}\spacebr{}that Maria the record knows and loves\\
%\mytrans{that Maria knows and loves the record}
\ex 
\gll {}[Kennt und liebt] Maria die Platte?\\
	 {}\spacebr{}知道 并且 喜欢 Maria \defart{} 唱片\\
\mytrans{Maria知道并且喜欢这张唱片吗?}
%\gll {}[Kennt und liebt] Maria die Platte?\\
%	 {}\spacebr{}knows and loves Maria the record\\
%\mytrans{Does Mary know and love the record?}
\zl
这两种语言学对象的差异主要在于它们能选择的短语类型的差异:标补词需要一个带有可见定式动词的句子,但是处于句首的动词需要一个没有可见定式动词的句子。
%The difference between the two linguistic objects mainly lies in the kind of phrase they select: the complementizer requires a sentence
%with a visible finite verb, whereas the verb in initial position requires a sentence without a visible finite verb.

\ref{Abschnitt-Vererbung-HPSG}呈现了一小部分承继层级。该部分包含可能在所有自然语言中都能发挥作用的类型:在每种语言中都有中心语-论元组合。如果没有这类组合操作,就不可能在两个概念之间建立联系。但是,建立联系是语言的一个基本属性。
%In Section~\ref{Abschnitt-Vererbung-HPSG}, a small part of an inheritance hierarchy was
%presented. This part contains types
%that probably play a role in the grammars of all natural languages: there are head-argument combinations in every language. Without this
%kind of combinatorial operation, it would not be possible to establish a relation between two concepts. The ability to create relations, however, is
%one of the basic properties of language.

除了更加概括的类型,一种特定语言的类型层级包含语言特定的类型或者那些针对特定语言类别的类型。所有语言都可以假设有一阶和二阶谓词,并且对于大多数语言(如果不是全部的话)来说,谈论动词\isce{动词}{verb}都是有意义的,之后就可以讨论一阶和二阶动词。这些动词再根据语言区分为不及物和及物动词。为各种类型制定的约束条件可以针对一般的语言或者针对特定语言。在英语\ilce{英语}{English}中,动词必须出现在补足语之前,因此是\initialv $+$ ;但是德语中的动词是\initialv $-$ ,并且是起始位置的词汇规则允准了具有\initialv $+$ 的动词。
%In addition to more general types, the type hierarchy of a particular language contains language-specific types or those specific to a particular class
%of languages. All languages presumably have one and two-place predicates and for most languages (if not all), it makes sense to talk about verbs\is{verb}.
%It is then possible to talk about one and two-place verbs. Depending on the language, these can then be subdivided into intransitive and transitive.
%Constraints are formulated for the various types that can either hold generally or be language-specific.
%In English\il{English}, verbs have to occur before their complements and therefore have the
%\initialv $+$, whereas verbs in German have the \initialv $-$ and it is the lexical rule for initial
%position that licenses a verb with an \initialv $+$.

对于德语和英语\initialvc 的不同设定让人想起了GB理论中的参数\isce{参数}{parameter}\isce[|(]{习得}{acquisition}。但是,这两者之间存在一个很大的差异:并没有假设一种语言的学习者一次性地为所有的中心语设置\initialvc 取值。使用一个\initialvc 取值与假设学习者利用词的位置信息来学习单个词的习得模型是兼容的。对于相应词语来说,可能针对一个特定特征展示出不同的取值。全部词类位置信息的概括只有在习得过程的后面才会习得。\isce[|)]{习得}{acquisition}
%The differing settings of the \initialv for German and English is reminiscent of
%parameters\is{parameter}\is{acquisition|(} from GB-Theory. There is one crucial difference,
%however: it is not assumed that a language learner sets the \initialv for all
%heads once and for all. The use of an \initialv is compatible with models of acquisition that assume that learners
%learn individual words with their positional properties. It is certainly possible for the respective
%words to exhibit different values for a particular feature. Generalizations about the position of
%entire word classes are only learned at a later point in the acquisition process.\is{acquisition|)}

图~\vref{Abbildung-Hierarchie}给出了类似于Croft(见\ref{Abschnitt-Croft})提出的层级。
%A hierarchy analogous to the one proposed by Croft (see Section~\ref{Abschnitt-Croft}) is given in Figure~\vref{Abbildung-Hierarchie}.
\begin{figure}
\centerfit{%
\begin{forest}
type hierarchy
[sign
  [stem
    [root
      [noun-root]
      [verb-root
        [intransitive-verb
          [strict-intransitive-verb
            [schlaf-\\睡觉, instance]]]
        [transitive-verb
          [strict-transitive-verb
            [lieb-\\爱, instance]]
          [ditransitive-verb
            [geb-\\给, instance]]]]]
    [complex-stem]]
  [word]
  [phrase
    [headed-phrase
      [head-argument-phrase]]]] 
\end{forest}
}
\caption{\label{Abbildung-Hierarchie}反映词汇项与支配模式承继层级的部分}
%\caption{\label{Abbildung-Hierarchie}Section of an inheritance hierarchy with lexical entries and dominance schemata}
\end{figure}%
对于屈折词,相关词根\isce{词根}{root}在词库中。例如,schlaf-(睡觉)、lieb-(爱)和geb-(给)。在图~\ref{Abbildung-Hierarchie}中,有\type{root}的不同次类型,词根的概括类型:例如,\type{intrans-verb}代表不及物动词,\type{trans-verb} 代表及物动词。及物动词可以进一步划分为严格及物动词(带有主格和宾格论元)和双及物动词(带有主格和宾格和与格论元)。上述层级当然必须需要大幅改进,因为及物和不及物动词都会有进一步的分类。例如,不及物动词可以分为非宾格\iscesub{动词}{verb}{非宾格}{unaccusative}和非作格\iscesub{动词}{verb}{非作格}{unergative}动词,并且即便是严格的及物动词也可以进一步区分出小类(见\citet[\S~2]{Welke2009a})。
%For inflected words, the relevant roots\is{root} are in the lexicon. Examples of this are \stem{schlaf} `sleep', \stem{lieb} `love' and \stem{geb}
%`give'. In Figure~\ref{Abbildung-Hierarchie}, there are different subtypes of \type{root}, the
%general type for roots: \eg \type{intrans-verb} for intransitive
%verbs and \type{trans-verb} for transitive verbs. Transitive verbs can be further subdivided into strictly transitive verbs (those with nominative and accusative
%arguments) and ditransitive verbs (those with nominative and both accusative and dative arguments). The hierarchy above would of course have to be refined considerably
%as there are even further sub-classes for both transitive and intransitive verbs. For example, one can divide intransitive verbs into unaccusative\is{verb!unaccusative}
%and unergative\is{verb!unergative} verbs and even strictly transitive verbs would have to be divided into further sub-classes (see \citealp[Section~2]{Welke2009a}).\pagebreak

除了词根类型之外,上图还包含词干\isce{词干}{stem}和词语\isce{词语}{word}的类型。复杂词根派生自简单词根,但是仍然必须是屈折的(\stem{lesbar}(可读的)、\stem{besing}(唱关于))。词语是不会屈折的对象。词语的例子是代词\isce{代词}{pronoun}er(他)、sie(她)等,还有介词。一个屈折\isce{屈折}{inflection}形式可以来自于一个动词词干,如geliebt(被爱)、besingt(唱关于)。屈折词和(复杂)词干可以再次通过派生\isce{派生}{derivation}规则来建立联系。以这种方式,geliebt(被爱)可以重新范畴化为形容词词干,该词干必须与形容词词尾组合(geliebt-e)。复杂词干/词的相关描述是\type{complex-stem}或者\type{word}的子类型。这些子类型描述了像geliebte这种复杂词必须有的形式。关于这一技术的使用,见 \citew[\S~3.2.7]{Mueller2002b}。使用统制模式,所有的词都可以组合成短语。这里给出的层级肯定不是完整的。有很多其他的价类,并且可以假设更加概括的类型来简单地描述一阶、二阶或三阶谓词。这些类型可能对于描述其他语言可行。这里,我们只是处理了类型层级的一小部分以便能与Croft的层级进行对比:在图\ref{Abbildung-Hierarchie}中,没有为形式为[Sbj IntrVerb]的句模提供类型,但是有为了一个特定价的词汇对象的类型(V[\subcat \sliste{ NP[\str] }])。然后,词汇规则就可以用于相关词汇对象,该词汇对象允准带有其他价或引入关于屈折信息的对象。完整的词可以在句法上使用相对概括的规则以完成组合,例如在中心语-论元结构中。纯粹短语方法遇到的问题因此就可以被避免了。但是,词位类型的概括和可以组合的表达就可以在层级中表示。
%In addition to a type for roots, the above figure contains types for stems\is{stem} and words\is{word}. Complex stems are complex objects that are
%derived from simple roots but still have to be inflected (\stem{lesbar} `readable', \stem{besing}
%`to sing about'). Words are objects that do not inflect. Examples of these
%are the pronouns\is{pronoun} \emph{er} `he', \emph{sie} `she' etc.\ as well as prepositions. An inflected\is{inflection} form can be formed from a verbal stem
%(\emph{geliebt} `loved', \emph{besingt} `sings about'). Relations between inflected words and (complex) stems can be formed again using derivation\is{derivation} rules. 
%In this way, \emph{geliebt} `loved' can be recategorized as an adjective stem that must then be combined with adjectival endings (\emph{geliebt-e}).
%The relevant descriptions of complex stems/words are subtypes of \type{complex-stem}
%or \type{word}. These subtypes describe the form that complex words such as \emph{geliebte} must have. For a technical implementation of this, see
% \citew[Section~3.2.7]{Mueller2002b}. Using dominance schemata, all words can be combined to form phrases. The hierarchy given here is of course by no means complete.
%There are a number of additional valence classes and one could also assume more general types that simply describe one, two and three-place predicates. 
%Such types are probably plausible for the description of other languages. Here, we are only dealing with a small part of the type hierarchy in order to have
%a comparison to the Croftian hierarchy: in Figure~\ref{Abbildung-Hierarchie}, there are no types for
%sentence patterns with the form
%[Sbj IntrVerb], but rather types for lexical objects with a particular valence
%(V[\subcat \sliste{ NP[\str] }]). Lexical rules can then be applied to the relevant lexical objects that license objects with another valence or introduce
%information about inflection. Complete words can be combined in the syntax with relatively general rules, for example in head-argument structures. The problems
%from which purely phrasal approaches suffer are thereby avoided. Nevertheless generalizations about lexeme classes and the utterances that can be formed can be
%captured in the hierarchy.

除了承继层级之外,还有其他原则:\ref{Abschnitt-HPSG-Semantik}展示的语义原则\iscesub{原则}{principle}{语义}{Semantics}适用于所有语言。我们也看到格原则\iscesub{原则}{principle}{格}{Case}只能用于特定类别的语言,即主格-宾格\iscesub{格}{case}{主格}{nominative}\iscesub{格}{case}{宾格}{accusative}语言。其他语言有一个作格-通格\iscesub{格}{case}{作格}{ergative}\iscesub{格}{case}{通格}{absolutive}系统。
%There are also principles in addition to inheritance hierarchies: the Semantics Principle\is{principle!Semantics} presented in Section~\ref{Abschnitt-HPSG-Semantik} holds
%for all languages. The Case Principle\is{principle!case} that we also saw is a constraint that only applies to a particular
%class of languages, namely nominative-accusative languages\is{case!nominative}\is{case!accusative}.
%Other languages have an ergative-absolutive system\is{case!ergative}\is{case!absolutive}.

这里所说的语言学理论不需要假设天赋的语言知识。正如第\ref{chap-innateness}章所述,是否存在这种知识还没有定论。即便是这类语言真的存在,什么是天赋,仍是个问题。我们可以合理地假设,与所有语言都相关的承继层级部分和相关原则都是天赋的(例如,中心语-论元结构和语义原则)。然而,情况也有可能是,有效类型和原则的某部分才是天赋的,因为,有一些天赋知识的存在,并不意味它们一定在所有语言中普遍存在(也可以参见\ref{Abschnitt-Universalien-Zusammenfassung})。
%The assumption of innate linguistic knowledge is not necessary for the theory of language sketched
%here. As the discussion in Section~\ref{chap-innateness} has shown, the question of whether this kind of knowledge
%exists has still not been answered conclusively. Should it turn out that this knowledge actually exists, the question arises of what exactly is innate. It would be a plausible assumption
%that the part of the inheritance hierarchy that is relevant for all languages is innate together
%with the relevant principles (\eg the constraints on Head-Argument structures and the Semantics Principle). It could, however, also be the case that only a part of the more generally
%valid types and principles is innate since something being innate does not follow from the fact that
%it is present in all languages (see also Section~\ref{Abschnitt-Universalien-Zusammenfassung}).\todostefan{Hier vielleicht noch etwas zur Variation sagen. Warum können benachbarte Dialekte nicht wild variieren?}

总之,最适用于比较语言之间相似处的方法,是使用统一的描述体系,并且使用承继层级获取概括。另外,就天赋知识是否存在的问题来说,无论答案是肯定的还是否定的,这种理论都是相容的。
%In sum, one can say that theories that describe linguistic objects using a consistent descriptive inventory and make use of inheritance hierarchies to capture
%generalizations are the ones best suited to represent similarities between languages. Furthermore,
%this kind of theory is compatible with both a positive and a negative
%answer to the question of whether there is innate linguistic knowledge.\il{German|)}

\section{如何发展获得跨语言概括的语言学理论}
%\section{How to develop linguistic theories that capture cross-linguistic generalizations}
\label{sec-develop-theories-coregram}

在上一节中,我建议,表征语言概括的好方法,是先统一所有描述层面的语言学知识与层级。本节将探索,如何由多种语言内的事实,来驱动语法的发展。
%In the previous section I argued for a uniform representation of linguistic knowledge at all
%descriptive levels and for type hierarchies as a good tool for representing generalizations. This
%section explores a way to develop grammars that are motivated by facts from several languages.

若我们看一下多种语言学流派的研究方式,可以发现,当今语言研究有两种极端方式。一方面,有主流的生成语法(Mainstream Generative Grammar,简称MGG)阵营并且,在另一方面,有构式语法/认知语法阵营。我在此很快地解释一下,这里所声明的只适用于极端案例,并非适用于以上这些流派里的所有语法。我们可以幽默的举例:一位MGG语言学家的目标是寻找底层结构。而既然所有语言的底层结构都相同(缺乏语言刺激),那他只需要研究一种语言,例如英语就够了。这种研究策略的结果是得到了一个最有影响力的英语语言学家提出的模型,其他学者就尽力去调整其他语言。因为英语有NP VP结构,所以所有语言都要有这个结构。因为英语在被动句中重新调整成分的顺序,被动是一种移位现象,所以所有语言必须这样运作。我在\ref{sec-case-assignment}和第\ref{chap-scrambling-extraction-passive}章都特别仔细地讨论了用该方法分析德语的效果,并且指出被动是移位这个假设对德语会产生错误的预测,因为德语被动句的主语仍然在宾语位置上。另外,这个分析需要假设不可见的虚位,即既看不见也没有任何意义的实体。
%If one looks at the current practice in various linguistic schools one finds two extreme ways of
%approaching language. On the one hand, we have the Mainstream Generative Grammar (MGG) camp and, on the
%other hand, we have the Construction Grammar/Cognitive Grammar camp. I hasten to say that what I state
%here does not hold for all members of these groups, but for the extreme cases. The caricature of the
%MGG scientist is that he is looking for underlying structures. Since these have to be the same for
%all languages (poverty of the stimulus), it is sufficient to look at one language, say English. The
%result of this research strategy is that one ends up with models that were suggested by the most
%influential linguist for English and that others then try to find ways to accommodate other
%languages. Since English has an NP VP structure, all languages have to have it. Since English reorders
%constituents in passive sentences, passive is movement and all languages have to work this way. I
%discussed the respective analyses of German in more detail in Section~\ref{sec-case-assignment}
%and in Chapter~\ref{chap-scrambling-extraction-passive} and showed that the assumption that passive is movement
%makes unwanted predictions for German, since the subject of passives stays in the object
%position in German. Furthermore, this analysis requires the assumption of invisible expletives, that is,
%entities that cannot be seen and do not have any meaning.

另一个极端是在构式语法或没有理论框架下进行工作的学者(相关讨论参见第~\ref{fn-ffs}页的脚注~\ref{fn-ffs})。他们认为所有语言是那么不同,以至于我们不能用相同的方式去分析它们。另外,在语言当中,有太多的对象以至于不可能(或太早)去表述任何概括。我这里描述的是极端的观点和陈旧的看法。
%On the other extreme of the spectrum we find people working in Construction Grammar or without any
%framework at all (see footnote~\ref{fn-ffs} on page~\ref{fn-ffs} for discussion) who claim that all languages are so different that we cannot even use the same
%vocabulary to analyze them. Moreover, within languages, we have so many different objects that it is impossible (or too early) to state any
%generalizations. Again, what I describe here are extreme positions and clichés.

\addlines
在下面,我们描述一下我们在核心语法工程\footnote{%
\url{https://hpsg.hu-berlin.de/Projects/CoreGram.html},\mytodayc。
}\citep{MuellerCoreGramBrief,MuellerCoreGram}中使用的程序。在核心语法项目中我们平行处理了类型不同的语言:
%In what follows, I sketch the procedure that we apply in the CoreGram project\footnote{%
%\url{https://hpsg.hu-berlin.de/Projects/CoreGram.html}, \today.
%} \citep{MuellerCoreGramBrief,MuellerCoreGram}. In the CoreGram project we work on a set of
%typologically diverse languages in parallel:
\begin{itemize}
\item 德语\ilce{德语}{German}  \citep{MuellerLehrbuch1,MuellerPredication,MuellerCopula,MOe2011a,MOe2013a,MuellerArten,MuellerGS}
%\item German\il{German}  \citep{MuellerLehrbuch1,MuellerPredication,MuellerCopula,MOe2011a,MOe2013a,MuellerArten,MuellerGS}
\item 丹麦语  \citep{Oersnes2009a,MuellerPredication,MuellerCopula,MOe2011a,MOe2013a,MOe2013b,MOeDanish}
%\item Danish  \citep{Oersnes2009a,MuellerPredication,MuellerCopula,MOe2011a,MOe2013a,MOe2013b,MOeDanish}
\item 波斯语\ilce{波斯语}{Persian} \citep*{MuellerPersian,MG2010a}
%\item Persian\il{Persian} \citep*{MuellerPersian,MG2010a}
\item 马耳他语\ilce{马耳他语}{Maltese} \citep{MuellerMalteseSketch}
%\item Maltese\il{Maltese} \citep{MuellerMalteseSketch}
\item 现代汉语\ilce{现代汉语}{Mandarin Chinese} \citep{Lipenkova2009a,ML2009a,ML2013a,MLChinese}
%\item Mandarin Chinese\il{Mandarin Chinese} \citep{Lipenkova2009a,ML2009a,ML2013a,MLChinese}
\item 依地语\ilce{依地语}{Yiddish} \citep{MOe2011a}
%\item Yiddish\il{Yiddish} \citep{MOe2011a}
\item 英语\ilce{英语}{English} \citep{MuellerPredication,MuellerCopula,MOe2013a}
%\item English\il{English} \citep{MuellerPredication,MuellerCopula,MOe2013a}
\item 印地语
%\item Hindi
\item 西班牙语\ilce{西班牙语}{Spanish} \citep{Machicao-y-Priemer2015a}
%\item Spanish\il{Spanish} \citep{Machicao-y-Priemer2015a}
\item 法语\ilce{法语}{French}
%\item French\il{French}
\end{itemize}

\noindent
	这些语言属于不同的语系(印欧语系、亚非语系、汉藏语系),并且印欧语系中的语言属于不同的语族(日耳曼、罗曼、印度--伊朗语族)。图~\ref{fig-lang-fams}提供了一个概要。我们在带有语义内容的HPSG\indexhpsg 理论框架内写出了完全形式化的、计算机可处理的语法模块。在这里不会讨论细节,但是有兴趣的读者可以参考 \citew{MuellerCoreGram}。
%These languages belong to diverse language families 
%(Indo-European, % Germanic: German, Danish, Yiddish, English, Romance: Spanish, French,
                % Indo-Iranian: Hindi, Persian
                % Slavic: Czech
% Afro-Asiatic,  % Semitic: Maltese, Hebrew
% Sino-Tibetan) % Sinitic: Mandarin Chinese, 
%and among the Indo-European languages the languages belong to different groups (Germanic, Romance,
%Indo-Iranian). Figure~\ref{fig-lang-fams} provides an overview.
%
% moved above the figure
%We work out fully formalized, computer-processable grammar fragments in the framework of
%HPSG\indexhpsg that have a semantics component. The details will not be discussed here, but the
%interested reader is referred to  \citew{MuellerCoreGram}. 
\begin{figure}
\centerfit{
\begin{forest}
[语言
        [印欧语系
          [日耳曼语族 [丹麦语] [英语] [德语] [依地语] ]
          [罗曼语族 [法语] [西班牙语] ] 
          [印度伊朗语族 [印地语] [波斯语] ] ]
        [亚非语系 
          [闪语族 [马耳他语] ] ]
        [汉藏语系 
          [汉语族 [现代汉语] ] ] ]
%[Languages
%        [Indo-European
%          [Germanic [Danish] [English] [German] [Yiddish] ]
%          [Romance [French] [Spanish] ] 
%          [Indo-Iranian [Hindi] [Persian] ] ]
%        [Afro-Asiatic 
%          [Semitic [Maltese] ] ]
%        [Sino-Tibetan 
%          [Sinitic [Mandarin Chinese] ] ] ]
\end{forest}
}
\caption{核心语法项目涵盖的语系和语支}\label{fig-lang-fams}
%\caption{Language families and groups of the languages covered in the CoreGram project}\label{fig-lang-fams}
\end{figure}%

正如前面章节所述,关于天赋的具体语言知识的假设应该减到最少。Chomsky在其最简方案中也持这个观点。甚至,特定语言的天赋知识,可能根本不存在。构式语法/认知语法都持这个观点。所以,不应该将来自于一种语言的约束强加于其他语言,一种自底向上的方法似乎更加合适:研究个别语言的语法,应发自此语言的内部。具有特定相同属性的语言应该被归为一类。这可以获得不同组语言或者自然语言本身的概括。让我们以一些语言为例:德语、荷兰语、丹麦语、英语和法语。如果开始为德语和荷兰语制定语法,我们就会发现这两种语言有很多相同的属性:例如,都是SOV和V2语言,并且都有动词性复杂体。最大的差异在于动词复杂体中成分的顺序。这种情况可以表示为图\vref{fig-german-dutch}。
%As was argued in previous sections, the assumption of innate language-specific knowledge should be
%kept to a minimum. This is also what Chomsky suggested in his Minimalist Program. There may even be no language-specific innate knowledge at all, a view taken in Construction
%Grammar/Cognitive Grammar. So, instead of imposing constraints from one language onto other languages, a bottom-up approach seems
%to be more appropriate: grammars for individual languages should be motivated language-internally. Grammars that share certain properties can be grouped in classes. This makes it possible
%to capture generalizations about groups of languages and natural language as such. Let us consider a
%few example languages: German, Dutch, Danish, English and French. If we start developing grammars for German and
%Dutch, we find that they share a lot of properties: for instance, both are SOV and V2 languages and both have a
%verbal complex. One main difference is the order of elements in the verbal complex. The situation
%can be depicted as in Figure~\vref{fig-german-dutch}.
\begin{figure}
\centering
\begin{tikzpicture}
    \tikzset{level 1+/.style={level distance=5\baselineskip}}%
    \tikzset{sibling distance=18pt}
%    \tikzset{frontier/.style={distance from root=10\baselineskip}}%
    \tikzset{every tree node/.style={
                          %  The shape:
                          rectangle,minimum size=6mm,rounded corners=3mm,
                          %  The rest
                          very thick,draw=black!50,
                          top color=white,bottom color=black!20,
                          font=\ttfamily},node distance=2mm}
    \Tree[.\node (Set3) { ~集合 3~ };
                  \node (Set1) { ~集合 1~ }; \node (Set2) { ~集合 2~ }; ]  

    \node [below=of Set1] {德语}; \node [below=of Set2] {荷兰语}; 
%    \node [below=of Set1] {German}; \node [below=of Set2] {Dutch}; 
%    \node [left=of Set5] {V2};
     \node [left=of Set3] {\begin{tabular}{@{}c@{}}Arg St\\V2\\SOV\\VC\end{tabular}};
%    \node [right=of Set11] {SVO};

     \end{tikzpicture}
\caption{\label{fig-german-dutch}德语和荷兰语共有的属性}
%\caption{\label{fig-german-dutch}Shared properties of German and Dutch}
\end{figure}%
有很多属性是德语和丹麦语所共有的(集合3)。例如,集合3包括词项的论元结构,一个包含论元句法和语义属性描述以及这些论元与词项意义联系的列表。除了SOV型语言的约束,集合3还包括动词的位置和V2小句中成分的前置。相应的约束为两种语法所共有。虽然这些集合在图\ref{fig-german-dutch}中也安排在一个层级当中,但是该图与前面章节所论述的类型层级没有关系。这些类型层级是我们语言学理论的一部分,并且这些层级的很多部分可以在不同集合中:那些与更加宽泛的方面有关的类型层级部分可见于图\ref{fig-german-dutch}中的集合3,荷兰语或德语特有的部分可以在相应的其他集合中。当我们增加其他语言时,例如丹麦语,我们又有了新的差异。虽然德语和荷兰语是SOV型,但是丹麦语是SVO型语言。图\vref{fig-german-dutch-danish}展示了增加语言之后的情况:最高的结点表征了对所有语言都适用的约束(例如,论元结构约束、联系和V2),该结点下面的结点(集合4)包含了只适用于德语和丹麦语的约束。 \footnote{%
原则上,会有适用于荷兰语和丹麦语的约束,但是不适用于德语或者适用于德语和丹麦语但是不适用于荷兰语。这些约束应该分别从集合1和集合2移出并插入更高的约束集合。这些集合没有出现在该图上,我仍然用来自于图\ref{fig-german-dutch}的集合1和集合2来称说德语和荷兰语的约束集合。  
}例如,集合4包括关于动词复杂体和SOV语序的约束。集合4和集合5的合并是图\ref{fig-german-dutch}的集合3。
%There are some properties that are shared between German and Dutch (Set 3). For instance, the
%argument structure of lexical items, a list containing descriptions of syntactic and semantic properties of
%arguments and the linking of these arguments to the meaning of the lexical items, is contained in Set 3. In
%addition to the constraints for SOV languages, the verb position and the fronting of a
%constituent in V2 clauses are contained in Set 3. The respective constraints are shared between the
%two grammars. Although these sets are arranged in a hierarchy in Figure~\ref{fig-german-dutch} and
%the following figures this has nothing to do with the type hierarchies that have been discussed in the previous subsection. These type
%hierarchies are part of our linguistic theories and various parts of such hierarchies can be in different
%sets: those parts of the type hierarchy that concern more general aspects can be in Set~3 in
%Figure~\ref{fig-german-dutch} and those that are specific to Dutch or German are in the respective
%other sets. When we add another language, say Danish, we get further differences. While German and Dutch are SOV, Danish
%is an SVO language. Figure~\vref{fig-german-dutch-danish} shows the resulting situation: the
%topmost node represents constraints that hold for all the languages considered so far (for instance the argument
%structure constraints, linking and V2) and the node below it (Set~4) contains
%constraints that hold for German and Dutch only.\footnote{%
%  In principle, there could be constraints that hold for Dutch and Danish, but not for German or for
%  German and Danish, but not for Dutch. These constraints would be removed from Set 1 and Set 2,
%  respectively, and inserted into another constraint set higher up in the hierarchy. These sets are not
%  illustrated in the figure and I keep the names Set~1 and Set~2 from Figure~\ref{fig-german-dutch} for the constraint sets for German
%  and Dutch.
%} For instance, Set~4 contains constraints regarding verbal complexes and SOV order.
% moved on top of the figure
%The union of Set 4 and Set 5 is Set 3 of Figure~\ref{fig-german-dutch}.
\begin{figure}
\centering
\begin{tikzpicture}
    \tikzset{level 1+/.style={level distance=5\baselineskip}}%
    \tikzset{sibling distance=18pt}
    \tikzset{frontier/.style={distance from root=10\baselineskip}}%
    \tikzset{every tree node/.style={
                          %  The shape:
                          rectangle,minimum size=6mm,rounded corners=3mm,
                          %  The rest
                          very thick,draw=black!50,
                          top color=white,bottom color=black!20,
                          font=\ttfamily},node distance=2mm}
    \Tree[.\node (Set5) { ~集合 5~ };
               [.\node (Set4) { ~集合 4~ };
                  \node (Set1) { ~集合 1~ }; \node (Set2) { ~集合 2~ }; ] \node (Set6) { ~集合 6~ }; ] 

    \node [below=of Set1] {德语}; \node [below=of Set2] {荷兰语};  \node [below=of Set6] {丹麦语};
%    \node [below=of Set1] {German}; \node [below=of Set2] {Dutch};  \node [below=of Set6] {Danish};
    \node [left=of Set5] {\begin{tabular}{@{}c@{}}Arg Str\\V2\end{tabular}};
    \node [left=of Set4] {\begin{tabular}{@{}c@{}}SOV\\VC\end{tabular}};
%    \node [right=of Set11] {SVO};

     \end{tikzpicture}
\caption{\label{fig-german-dutch-danish}德语、荷兰语和丹麦语共有的属性}
%\caption{\label{fig-german-dutch-danish}Shared properties of German, Dutch, and Danish}
\end{figure}%

如果我们增加更多的语言,就需要区分更多的约束集合。图\vref{fig-german-dutch-danish-english-french}展示了如果增加英语和法语会得到的结果。
%If we add further languages, further constraint sets will be
%distinguished. Figure~\vref{fig-german-dutch-danish-english-french} shows the situation that results
%when we add English and French.
\begin{figure}
\centering
\begin{tikzpicture}
    \tikzset{level 1+/.style={level distance=5\baselineskip}}%
    \tikzset{sibling distance=18pt}
    \tikzset{frontier/.style={distance from root=15\baselineskip}}%
    \tikzset{every tree node/.style={
                          %  The shape:
                          rectangle,minimum size=6mm,rounded corners=3mm,
                          %  The rest
                          very thick,draw=black!50,
                          top color=white,bottom color=black!20,
                          font=\ttfamily},node distance=2mm}
    \Tree[.\node (Set8) { ~集合 8~ };
            [.\node (Set7) { ~集合 7~ };
               [.\node (Set4) { ~集合 4~ };
                  \node (Set1) { ~集合 1~ }; \node (Set2) { ~集合 2~ }; ] \node (Set6) { ~集合 6~ }; ] 
               [.\node (Set11) { ~集合 11~ }; \node (Set12) { ~集合 12~ };  \node (Set13) { ~集合 13~ }; ] 
    ]
    \node [below=of Set1] {德语}; \node [below=of Set2] {荷兰语};  \node [below=of Set6] {丹麦语};
    \node [below=of Set12] {英语}; \node [below=of Set13] {法语}; 
%    \node [below=of Set1] {German}; \node [below=of Set2] {Dutch};  \node [below=of Set6] {Danish};
%    \node [below=of Set12] {English}; \node [below=of Set13] {French}; 
    \node [left=of Set8] {Arg Str};
    \node [left=of Set7] {V2};
    \node [left=of Set4] {\begin{tabular}{@{}c@{}}SOV\\VC\end{tabular}};
    \node [right=of Set11] {SVO};

    \draw (Set11.south) -- (Set6.north);
    \end{tikzpicture}

\caption{\label{fig-german-dutch-danish-english-french}语言和语言分组}
%\caption{\label{fig-german-dutch-danish-english-french}Languages and language classes}
\end{figure}%
当然,图是不全的,因为有约束为丹麦语和英语所共享,但是法语没有。但是总体观点应该是清楚的:按照这种系统性方式,我们可以得到一套约束集合,对应于类型学文献的研究结果。
%Again, the picture is not complete since there are constraints that are shared by Danish and English
%but not by French, but the general idea should be clear: by systematically working this way, we should
%arrive at constraint sets that directly correspond to those that have been established in the typological
%literature.

有趣的问题是如果我们考察足够多的语言,最高点集合应该是什么呢。初看起来,可能会认为所有的语言都有价表征以及价与词项语义之间的联系(即在HPSG理论中的论元结构列表)。但是, \citet{KM2012a}提出了一种针对奥奈达语\ilce{奥奈达语}{Oneida}(一种南部易洛族语言)的分析,这种分析不包括句法价的表征。如果这个分析是正确的,那么句法论元结构就不是普遍的。当然,这可能是大量语言的特点,但并不是最高点集合的一部分。那么唯一一个可能成为句法域中最高点集合的是:允准两种或多种语言学对象组合的约束。这基本上就是Chomsky的不带二叉\iscesub{分支}{branching}{二叉}{binary}约束的外部合并\iscesub{合并}{Merge}{外部}{External}原则。\footnote{%
注意二叉结构比平铺结构更加严格:有一个额外的约束让某一成分必须有两个子结点。正如\ref{Abschnitt-NPN-Konstruktion}所论述的,需要多于两个成分的短语构式。  
}另外,最高点集合当然应该包括基础的装置来表征音系和语义。
%The interesting question is what will be the topmost set if we consider enough languages. At
%first glance, one would expect that all languages have valence representations and linkings between
%these and the semantics of lexical items (argument structure lists in the HPSG framework). However,
% \citet{KM2012a} argue for an analysis of Oneida\il{Oneida} (a Northern Iroquoian language) that does not
%include a representation of syntactic valence. If this analysis is correct, syntactic argument
%structure would not be universal. It would, of course, be characteristic of a large number of
%languages, but it would not be part of the topmost set. So this leaves us with just one candidate
%for the topmost set from the area of syntax: the constraints that license the combination of two or more linguistic
%objects. This is basically Chomsky's External Merge\is{Merge!External} without the binarity restriction\is{branching!binary}\footnote{%
%  Note that binarity is more restrictive than flat structures: there is an additional constraint
%  that there have to be exactly two daughters. As was argued in Section~\ref{Abschnitt-NPN-Konstruktion} one needs phrasal
%  constructions with more than two constituents.
%}. In addition, the topmost set would, of course, contain the basic machinery for representing phonology and semantics.  

前述内容也让我们清楚,用这种方式做研究的学者,目标都是发现概括,并使用这些既有理论结构来描述新的语言。但是,正如上面所解释的,研究个别语言的语法,应该着重此语言内的现象,而非发自其他语言。若语言X的语言现象分析,得出一种以上的分析结果,这时,用同种架构分析语言Y能得到的结果就很有用,因为这种情况下,语言Y的分析结果可助评断语言X的多种结果。我将这种方法称为作弊的自底向上的方法:除非存在矛盾的证据,我们可以重新使用为其他语言发展出的分析。
%It should be clear from what has been said so far that the goal of every scientist who works this
%way is to find generalizations and to describe a new language in a way that reuses theoretical constructs
%that have been found useful for a language that is already covered. However, as was explained above,
%the resulting grammars should be motivated by data of the respective languages and not by facts from
%other languages. In situations where more than one analysis would be compatible with a given dataset
%for language X, the evidence from language Y with similar constructs is most welcome and can be used
%as evidence in favor of one of the two analyses for language X. I call this approach the
%\emph{bottom-up approach with cheating}: unless there is contradicting evidence, we can reuse
%analyses that have been developed for other languages.  

注意这一方法与 \citet{Haspelmath2010a}、 \citet{Dryer97a-u}、 \citet[\S~1.4.2--1.4.3]{Croft2001a}等其他人所支持的不可知论是兼容的,他们认为描述范畴应该是针对特定语言的,例如他加禄语的主语概念应该与英语的主语概念是不同的,英语中的名词概念与波斯语的名词概念是不同的。即便是遵循这样极端的观点,我们仍可以就成分结构、中心语-论元关系等等,来推导出概括。但是,我相信有些范畴是可以跨语言广泛使用的;如果不具有普遍性,也至少能对一类语言有用。正如 \citet[\page 692]{Newmeyer2010a}就主语概念所指出的那样:在同一种语言中,叫两个成分为主语并不意味着这二者有相同的属性。对于来自于不同语言的两种语言成分也是一样:称一个波斯语成分叫主语并不意味着它跟英语中称为主语的语言成分有相同的属性。当然,这一点对于其他范畴和关系也是对的,例如词类:波斯语中的名词并不一定跟英语中的名词有完全相同的属性。 \footnote{%
注意使用波斯语名词和英语名词这种标签(例如,参见\citet[\S~2]{Haspelmath2010a}关于格的这个建议,例如俄语与格,韩语与格\ldots)或多或少有些奇怪,因为这意味着波斯语名词和英语名词都或多或少是名词。与使用波斯语名词这个范畴不同,可以将相应类别的对象归入名词并且增加一个\textsc{language}特征,该特征的取值是\type{persian}。这一简单的技巧可以使得波斯语名词的类型和英语名词的类型的对象都属于名词类并且说明了两者存在的差异。当然,没有理论语言学家会引用\textsc{language}特征来区分波斯语名词和英语名词,但是在各自语言中的名词有别的特征将两者区分开来。所以像名词这种词类分类是多种语言中名词的一种概括,并且波斯语名词和英语名词范畴是包含进一步、语言特定信息的特征束。  
} \citet[\page 697]{Haspelmath2010b}:生成语法学家在描述单个语言时尽力使用尽可能多的跨语言范畴,这通常会导致不可克服的问题。如果假设一个范畴导致了问题,这些问题必须被解决。如果无法通过既定集合的范畴/特征来解决,就必须假设新的范畴/特征。这并不是方法论的劣势,反过来才是正确的:如果我们发现一些现象不能很好地融入我们已经有的理论框架中,这意味着我们发现了新的和令人兴奋的现象。如果我们坚持语言特定的范畴和特征,会更难注意到其中有特殊现象,因为所有的范畴和特征都是针对特定语言的。注意并非一种语言社团的所有语言都有完全相同的范畴。如果将语言特有范畴坚持到极致的话,就会得到说话人特有的范畴,例如Klaus--英语--名词。
%Note that this approach is compatible with the rather agnostic view advocated by
% \citet{Haspelmath2010a},  \citet{Dryer97a-u},  \citet[Section~1.4.2--1.4.3]{Croft2001a}, and others, who argue that descriptive categories should be
%language-specific, that is, the notion of \emph{subject} for Tagalog is different from the one for English,
%the category \emph{noun} in English is different from the category \emph{noun} in Persian and so on. Even if one
%follows such extreme positions, one can still derive generalizations regarding constituent structure,
%head-argument relations and so on. However, I believe that some categories can fruitfully be used
%cross-linguistically; if not universally, then at least for language classes. As  \citet[\page
%  692]{Newmeyer2010a} notes with regard to the notion of \emph{subject}: calling two items \emph{subject}
%in one language does not entail that they have identical properties. The same is true for two
%linguistic items from different languages: calling a Persian linguistic item \emph{subject} does not entail
%that it has exactly the same properties as an English linguistic item that is called
%\emph{subject}. The same is, of course, true for all other categories and relations, for instance, parts of speech:
%Persian nouns do not share all properties with English nouns.\footnote{%
%  Note that using labels like \emph{Persian Noun} and \emph{English Noun} (see for instance
%  \citealp[Section~2]{Haspelmath2010a} for such a suggestion regarding case, \eg Russian Dative,
%  Korean Dative, \ldots) is somehow strange since
%  it implies that both Persian nouns and English nouns are somehow nouns. Instead of using the
%  category \emph{Persian Noun} one could assign objects of the respective class to the class
%  \emph{noun} and add a feature \textsc{language} with the value \type{persian}. This simple trick
%  allows one to assign both objects of the type \emph{Persian Noun} and objects of the type
%  \emph{English Noun} to the class \emph{noun} and still maintain the fact that there are
%  differences. Of course, no theoretical linguist would introduce the \textsc{language} feature to
%  differentiate between Persian and English nouns, but nouns in the respective languages have other features that
%  make them differ. So the part of speech classification as noun is a generalization over nouns in
%  various languages and the categories \emph{Persian Noun} and \emph{English Noun} are feature
%  bundles that contain further, language-specific information.

  %%  \citet[\page 31]{Croft2001a} points out that it depends on the criteria that are chosen by the linguist
  %% whether languages like Makah have a Noun-Verb distinction or not. This is true, but as a result of
  %% the choice
%}
% \citet[\page 697]{Haspelmath2010b} writes: ``Generative linguists try to use as many crosslinguistic
%  categories in the description of individual languages as possible, and this often leads to
%  insurmountable problems.'' If the assumption of a category results in problems, they have to be
%solved. If this is not possible with the given set of categories/features, new ones have to be
%assumed. This is not a drawback of the methodology, quite the opposite is true: if we have found
%something that does not integrate nicely into what we already have, this is a sign that we have discovered
%something new and exciting. %For instance, 
%If we stick to language-particular categories and features, it is much
%harder to notice that a special phenomenon is involved, since all categories and features are
%specific to one language anyway. Note also that not all speakers of a language community have
%exactly the same categories. If one were to take the idea of language-particular category symbols to
%an extreme, one would end up with person specific category symbols like \emph{Klaus-English-noun}.

2013年我在MIT做完报告之后,语言学系的成员反对我们在核心语法工程中使用的方法,并且声称就可能/不可能的语言而言,这种方法不能做出任何预测。关于预测需要说明两点:一是,预测是基于特定语言的。我们以 \citet{Netter91}中的句子作为例子:
%After my talk at the MIT in 2013, members of the linguistics department objected to the
%approach taken in the CoreGram project and claimed that it would not make any predictions as far as possible/impossible languages
%are concerned. Regarding predictions two things must be said: firstly, predictions are being made on a
%language particular basis. As an example consider the following sentences from  \citet{Netter91}:

\eal
\ex 
\gll {}[Versucht, zu lesen], hat er das Buch nicht.\\
       \spacebr{}尝试 \textsc{inf} 读 \textsc{aux} 他.\nom{} \defart.\acc{} 书 \textsc{neg}\\
\mytrans{他并不想尝试读这本书。}
%\gll {}[Versucht, zu lesen], hat er das Buch nicht.\\
%       \spacebr{}tried to read has he.\nom{} the.\acc{} book not\\
%\mytrans{He did not try to read the book.}
\ex 
\gll {}[Versucht, einen Freund vorzustellen], hat er ihr noch nie.\\
       \spacebr{}尝试 一.\acc{} 朋友 介绍.\textsc{inf} \textsc{aux} 他.\nom{} 她.\dat{} 尚未 从不\\
\mytrans{他从未尝试介绍一个朋友给她。}
%\gll {}[Versucht, einen Freund vorzustellen], hat er ihr noch nie.\\
%       \spacebr{}tried a.\acc{} friend to.introduce has he.\nom{} her.\dat{} yet never\\
%\mytrans{He never before tried to introduce a friend to her.}
\zl
当我第一次读到这些句子时,我不知道它们的结构。我打开电脑并将它们输入,在很短时间内便获得了这些句子的分析,并且通过检查这些结构,我意识到这些句子是部分动词短语前置\isce{部分动词短语前置}{partial verb phrase fronting}和所谓的第三构式\isce{第三构式}{third construction}的组合\citep[\page 439]{Mueller99a}。我以前曾经实现过这两种现象的分析,但是从来没想过两种现象的交互。语法预测像(\mex{0}) 这种例子是合法的。相似的,语法的约束可相互作用以排除某些结构。所以也可以对不合法/不可能结构做出预测。
%When I first read these sentences I had no idea about their structure. I switched on my computer and typed them
%in and within milliseconds I got an analysis of the sentences and by inspecting the result I realized
%that these sentences are combinations of partial verb phrase fronting\is{partial verb phrase fronting} and the so-called third
%construction\is{third construction} \citep[\page 439]{Mueller99a}. I had previously implemented analyses of both phenomena
%but had never thought about the interaction of
%the two. The grammar predicted that examples like (\mex{0}) are grammatical. Similarly the
%constraints of the grammar can interact to rule out certain structures. So predictions about
%ungrammaticality/impossible structures are in fact made as well.

第二,最高约束集合对于至今考察的所有语言都适用。存在所有语言都有的属性可以看做是一个假设。这个约束集合包括句法和信息结构之间关系的约束,以及允许V2语言但是排除动词处在倒数第二位置的语言(参见\citet[\page 50]{Kayne94a-u},看其说明,这种语言是不存在的。Kayne发展出了一种复杂的句法系统来预测这个事实。)当然,如果发现一种语言将动词放在倒数第二个位置来编码句子类型或者其他交际效应,就需要定义一个更加概括的最高点集合。但是这一点是与最简理论平行的:如果语言发现与基本假设不兼容,基本假设就应该修改。与特定语言约束相同,最高点结合的约束预测了什么可能存在,什么不可能存在。
%Secondly, the topmost constraint set holds for all languages seen so far. It can be regarded as a
%hypothesis about properties that are shared by all languages. This constraint set contains
%constraints about the connection between syntax and information structure and such constraints allow
%for V2 languages but rule out languages with the verb in penultimate position (see \citealp[\page
%  50]{Kayne94a-u} for the claim that such languages do not exist. Kayne develops a complicated
%syntactic system that predicts this). Of course, if a language is found that places the verb in penultimate
%position for the encoding of sentence types or some other communicative effect, a more
%general topmost set has to be defined. But this is parallel for Minimalist theories: if languages
%are found that are incompatible with basic assumptions, the basic assumptions have to be revised. As
%with the language particular constraints, the constraints in the topmost set make certain
%predictions about what can be and what cannot be found in languages.

常被拿来讨论的例子有,有些语言通过颠倒词语在词串中的顺序来构成疑问句\citep{MMGRRBW2003a},这种情况不用语法来排除,因为这种现象被语言的外部限制排除了:我们缺乏做这种复杂计算的工作记忆\isce{语言运用}{performance}。
%Frequently discussed examples such as those languages that form questions by reversing the order of the words in a
%string \citep{MMGRRBW2003a} need not be ruled out by the grammar, since they are ruled out by
%language external constraints: we simply lack the working memory to do such complex computations\is{performance}.

这个观点的另一个版本来自David Pesetsky\aimention{David Pesetsky},这是他在Facebook上讨论Paul Ibbotson\aimention{Paul Ibbotson}和Michael Tomasello\aimention{Michael Tomasello}发表在卫报上一篇文章时提出的\footnote{%
《语言的根源:是什么让我们异于其他动物?》(\emph{The roots of language: What makes us different from other animals?}),发表于\zhdate{2015/11/05}。
\url{http://www.theguardian.com/science/head-quarters/2015/nov/05/roots-language-what-makes-us-different-animals}, \zhdate{2018/04/25}。
}。Pesetsky指出,Tomasello关于语言习得的理论不能解释为什么我们发现了V2语言却没有发现V3语言。首先,我不知道在现在的最简理论中有什么约束可以阻碍V3语言的产生。所以本质上不存在V3语言这个事实不能用于支持任何方法。当然,我们可以问以下问题:是不是V3模式对于我们达到交际目的更加有用,或者是否它更加容易习得。现在,使用V2作为一种模式,很清楚我们有一个位置可以用于V2句(话题或焦点)特殊的目的。对于一价和二价动词,我们有一种论元可以放在句首位置。这个情况与假想的V3语言不同:如果我们有像sleep这样的一价动词,那么第二个位置就什么也没有了。正如Pesetsky在回应我对一个博客发布的观点所表示的那样,语言通过虚位来解决这种问题。例如,有些语言插入虚位来标记在嵌套问句中的主语提取,因为不然的话,对于听话者来说,主语提取这一点不容易识别。所以,虚位有助于让结构更加明晰。如果说话者想要避免某成分出现在一个特殊的、指定的位置,那么V2语言就会使用虚位去填充句首位置。
%A variant of this argument comes from David Pesetsky\aimention{David Pesetsky} and was raised in Facebook discussions of an
%article by Paul Ibbotson\aimention{Paul Ibbotson} and Michael Tomasello\aimention{Michael Tomasello}
%published in The Guardian\footnote{%
%\emph{The roots of language: What makes us different from other animals?} Published 05.11.2015. \url{http://www.theguardian.com/science/head-quarters/2015/nov/05/roots-language-what-makes-us-different-animals}
%}. 
%% how [Tomasello's] `prefabricated phrases' proposal could scale up so these kids end up speaking a verb-second language (when and how do they learn a structure-dependent rule that moves a particular verb and something else to designated positions) — and how he [Tomasello] would explain the frequent appearance of verb-second vs. the non-existence of obligatory verb-third crosslinguistically.
%Pesetsky claimed that Tomasello's theory of language
%acquisition could not explain why we find V2 languages but no V3 languages. First, I do not know of
%anything that blocks V3 languages in current Minimalist theories. So per se the fact that V3
%languages may not exist cannot be used to support any of the competing approaches. Of course, the
%question could be asked whether the V3 pattern would be useful for reaching our communicative goals
%and whether it can be easily acquired. Now, with V2 as a pattern it is clear that we have exactly
%one position that can be used for special purposes in the V2 sentence (topic or focus). For monovalent and bivalent verbs we
%have an argument that can be placed in initial position. The situation is different for the
%hypothetical V3 languages, though: If we have monovalent verbs like \emph{sleep}, there is nothing
%for the second position. As Pesetsky pointed out in the answer to my comment on a blog post, languages solve such
%problems by using expletives. For instance some languages insert an expletive to mark subject
%extraction in embedded interrogative sentences, since otherwise the fact that the subject is
%extracted would not be recognizable by the hearer. So the expletive helps to make the structure
%transparent. V2 languages also use expletives to fill the initial position if speakers want to avoid
%something in the special, designated position:
\ea
\gll Es kamen drei Männer zum Tor hinein.\\
     \expl{} 来 三 男人 \textsc{prep}.\defart{} 门 进入\\
\mytrans{三个男人通过门进来了。}
%\gll Es kamen drei Männer zum Tor hinein.\\
%     \expl{} came three man to.the gate in\\
%\mytrans{Three man came through the gate.}
\z
为了达到在V2语言中相同的效果,V3语言必须在动词之前放置两个虚位成分。所以看起来V3有很多V2所没有的劣势,所以V3语言更加难以出现。如果它们真的存在,它们也会在时间发展过程中发生变化;例如,省略带有不及物动词的虚位,带有及物动词的V2可有可无,最后大致是V2。随着新的语言习得模拟技术和基于智能体的社区仿真技术的发展,有可能真的可以模拟这种过程,并且我猜想在未来的几年,我们会看到这一领域会有令人兴奋的进展。
%In order to do the same in V3 languages one would have to put two expletives in front of the
%verb. So there seem to be many disadvantages of a V3 system that V2 systems do not have and hence
%one would expect that V3 systems are less likely to come into existence. If they existed, they would
%be expected to be subject to change in the course of time; \eg omission of the expletive with
%intransitives, optional V2 with transitives and finally V2 in general. With the new modeling
%techniques for language acquisition and agent-based community simulation one can actually simulate
%such processes and I guess in the years to come, we will see exciting work in this area. 

\largerpage
\citew[\page 106]{Cinque99a-u}提出了一系列功能投射来解释世界语言中重复出现的语序问题。他假设了详细的树结构,该结构对所有语言的所有句子的分析都有作用,即便是没有证据来证明特定语言中相应的形态句法差异(也可以参见\citet[\page 55]{CR2010a})。在后一种情况中,Cinque假设相应的树结点是空的。Cinque的结果可以融入到这里支持的模型当中。我们会在最高点集合定义词类范畴和形态句法特征,并且说明线性化约束来要求Cinque在其树结构中直接编码的顺序。在这些范畴没有通过词汇材料来展示的语言中,这些约束永远不会起作用。空成分和详细树结构都不需要。所以,Cinque的数据可以在一个带有丰富UG的HPSG中得到更好地概括。但是,我还是避免向所有的语言理论中引入400个范畴(或者特征),并且,我再次指出,从基因的\isce{基因}{gene}观点来看,这样一个丰富的且针对特定语言的UG是不可行的。所以,我在等待其他跟Cinque的数据有关的解释(可能是功能的)。
% \citew[\page 106]{Cinque99a-u} suggested a cascade of functional projections to account for
%reoccurring orderings in the languages of the world. He assumes elaborate tree structures to play a
%role in the analysis of all sentences in all languages even if there is no evidence for respective
%morphosyntactic distinctions in a particular language (see also \citealp[\page 55]{CR2010a}). In the
%latter case, Cinque assumes that the respective tree nodes are empty. Cinque's results could be
%incorporated in the model advocated here. We would define part of speech categories and
%morpho-syntactic features in the topmost set and state linearization constraints that enforce the
%order that Cinque encoded directly in his tree structure. In languages in which such categories are
%not manifested by lexical material, the constraints would never apply. Neither empty elements nor
%elaborate tree structures would be needed. Thus Cinque's data could be
%covered in a better way in an HPSG with a rich UG but I, nevertheless, refrain from introducing 400 categories (or
%features) into the theories of all languages and, again, I point out that such a rich and
%language-specific UG is implausible from a genetic\is{gene} point of view. Therefore, I wait for other, probably functional, explanations of the Cinque data.

注意蕴含的普遍性\iscesub{普遍性}{universal}{蕴含的普遍性}{implicational}可以从正如这里提出的层级安排的约束集合中推导出来。例如,可以从图\ref{fig-german-dutch-danish-english-french}中推导出蕴涵关系,所有的SVO语言都是V2语言,因为不存在一种语言包含集合4中的约束,但是不包含集合7中的约束。当然,这一蕴涵表述是错误的,因为有很多SOV语言,但是只有很少的V2语言。所以,只要我们增加其他语言,例如波斯语或日语,这张图就变了。
%Note also that implicational universals\is{universal!implicative} can be derived from hierarchically organized constraint sets
%as the ones proposed here. For instance, one
%can derive from Figure~\ref{fig-german-dutch-danish-english-french} the implicational statement that
%all SVO languages are V2 languages, since there is no language that has constraints from Set~4 that
%does not also have the constraints of Set~7. Of course, this implicational statement is wrong, since there are
%lots and lots of SOV languages and just exceptionally few V2 languages. So, as soon as we add other
%languages as for instance Persian or Japanese, the picture will change.

这里所提出的方法论与MGG提出的方法论不同,因为MGG认为概括的约束是基于对所有语言都适用的概括假设。最好的情况是,这些普遍假设经过不同语言和语法经验的检验,最差的情况是它们来源于从一种或更多印欧语总结出来的观点。通常情况下,一些表面观察出的现象,常被用来激发深远的基本设计决策\citep{Fanselow2009a,SR2012a,Haider2016a}。值得注意的是,这正是MGG阵营为什么反对类型学家的原因。 \citet{EL2009a}指出,很多可以验证的普遍性都可以找到反例。对于这一点经常的回应是,未经分析的现象不能推翻语法假设(例如,可以参见\citet[\page 454]{Freidin2009a})。与此相似,有人说未加分析的现象不应用于建立理论(Fanselow2009)。在核心语法工程中,我们尝试为多种语言提出覆盖范围广的语法,所以充当最高点的那些约束不是基于对语言不明确的知识得出的。
%The methodology suggested here differs from what is done in MGG, since MGG stipulates the general
%constraints that are supposed to hold for all languages on the basis of general specualtions about
%language. In the best case, these general assumptions are fed by a lot of experience with different
%languages and grammars, in the worst case they are derived from insights gathered from one or more
%Indo-European languages. Quite often impressionistic data is used to motivate rather far-reaching
%fundamental design decisions \citep{Fanselow2009a,SR2012a,Haider2016a}. It is interesting to note
%that this is exactly what members of the MGG camp reproach typologists for.  \citet{EL2009a} pointed out
%that counterexamples can be found for many alleged universals. A frequent response to this is that
%unanalyzed data cannot refute grammatical hypotheses (see, for instance, \citealp[\page
%  454]{Freidin2009a}). In the very same way it has to be said that unanalyzed data should not be
%used to build theories on \citep{Fanselow2009a}. In the CoreGram project, we aim to develop
%broad-coverage grammars of several languages, so those constraints that make it to the top node are
%motivated and not stipulated on the basis of intuitive implicit knowledge about language.

因为此研究方法是着重于语言现象而不是预先假设天赋语言知识的存在,该研究策略与构式语法(\citet[\page 481]{Goldberg2013b}有详细的论述)兼容,并且在某种程度上与最简方案兼容。
%Since it is data-oriented and does not presuppose innate language-specific knowledge, this research strategy is compatible with work carried out in Construction Grammar (see
%\citealp[\page 481]{Goldberg2013b} for an explicit statement to this end) and in any case it should also be compatible with the Minimalist
%world.


%      <!-- Local IspellDict: en_US-w_accents -->
