%% -*- coding:utf-8 -*-

\section{二叉}
%\section{Binary branching}
\label{sec-branching}

我们\isc{分支!二叉|(}\is{branching!binary|(}已经看到这类分支问题在不同理论中的处理方式不同。经典\xbartc 认为一个动词可以与其所有补足语组合。在GB后来的变体中,所有结构都被严格限制为二叉的。其余理论框架在处理分支这一问题时采取的方式有所不同:有的理论坚持二叉结构,而其他理论框架选择平铺结构。
%We\is{branching!binary|(} have seen that the question of the kind of branching structures assumed has received differing treatments in various theories.
%Classical \xbart assumes that a verb is combined with all its complements. In later variants of GB, all structures are strictly binary branching.
%Other frameworks do not treat the question of branching in a uniform way: there are proposals that assume binary branching structures and others
%that opt for flat structures.

\citet[\S~2.5]{Haegeman94a-u}以学习力作为论据\isc{语言习得}\is{language acquisition} (习得等级,见\ref{Abschnitt-Geschwindigkeit-Spracherwerb}对于这一问题的论述)。她讨论了例(\mex{1})中的句子并且表示如果自然语言中允许平铺结构的话,语言学习者必须从八种结构中选择其中一种。另一方面,如果语言中只有二叉结构,那么首先例(\mex{1})中的句子就不会有图\vref{Abbildung-Haegmann-flach}中的结构,所以学习者就不必排除对应的假设。
% \citet[Section~2.5]{Haegeman94a-u} uses learnability arguments\is{language acquisition} (rate of acquisition, see Section~\ref{Abschnitt-Geschwindigkeit-Spracherwerb}
%on this point).
%She discusses the example in (\mex{1}) and claims that language learners have to choose one of eight structures if flat-branching structures can occur in natural
%language. If, on the other hand, there are only binary-branching structures, then the sentence in (\mex{1}) cannot have the structures in
%Figure~\vref{Abbildung-Haegmann-flach} to start with, and therefore a learner would not have to rule out the corresponding hypotheses.
\ea 
\gll Mummy must leave now.\\
     妈妈 必须 离开 现在\\
\mytrans{妈妈必须现在离开。}
\z
\begin{figure}
\begin{forest}
sm edges, empty nodes
[{},phantom    
[{},tier=flat
 [{} [Mummy;妈妈]]
 [{} [must;必须]]
 [{} [leave;离开]]
 [{} [now;现在]]]
%%%%%%%%%%%%%%%%%%%%%%%%%%%%%%
[{}
 [{},tier=flat 
     [{} [Mummy;妈妈]]
     [{} [must;必须]]
     [{} [leave;离开]]]
 [{} [now;现在]]]
%%%%%%%%%%%%%%%%%%%%%%%%%%%%%%%
[{} 
 [{} [Mummy;妈妈]]
 [{},tier=flat 
     [{} [must;必须]]
     [{} [leave;离开]]
     [{} [now;现在]]]]
]
\end{forest}
%\caption{\label{Abbildung-Haegmann-flach}Structures with partial flat-branching}
\caption{\label{Abbildung-Haegmann-flach}部分平铺结构}
\end{figure}%

\noindent
但是, \citet[\page 88]{Haegeman94a-u}提供了证据证明例(\mex{0})的结构如(\mex{1})所示:
%However,  \citet[\page 88]{Haegeman94a-u} provides evidence for the fact that (\mex{0}) has the structure in (\mex{1}):
\ea
\gll {}[Mummy [must [leave now]]]\\
     \spacebr{}妈妈 \spacebr{}必须 \spacebr{}离开 现在\\
     \mytrans{妈妈必须现在离开}
\z
证明这一点的相关测试包括省略构式\isc{省略}\is{ellipsis},换句话说,可以用代词指称(\mex{0})中的成分。这意味着确实有证据支持语言学家假设的例(\mex{-1})的结构,因此不必假设在我们大脑中,只有二叉结构是被允准的。 \citet[\page 143]{Haegeman94a-u}提到了二叉假说的后果:如果所有的结果都是二叉的,那么在\xbartc 中不可能直接解释包含双及物动词的句子。在\xbartc 中,假设一个中心语与其所有补足语同时组合(见\ref{sec-xbar})。所以,为在\xbartc 中解释双及物动词,就必须假设一个空成分\isc{空成分}\is{empty element}(\littlevc)(见\ref{sec-little-v})。
%The relevant tests showing this include elliptical constructions\is{ellipsis}, that is, the fact that it is possible to
%refer to the constituents in (\mex{0}) with pronouns. This means that there is actually evidence for
%the structure of (\mex{-1}) that is assumed by linguists and we therefore do not have to assume that
%it is just hard-wired in our brains that only binary-branching structures are allowed.  \citet[\page
%  143]{Haegeman94a-u} mentions a consequence of the binary branching hypothesis: if all structures are
%binary-branching, then it is not possible to straightforwardly account for sentences with
%ditransitive verbs in \xbart. In \xbart, it is assumed that a head is combined with all its
%complements at once (see Section~\ref{sec-xbar}). So in order to account for ditransitive verbs in
%\xbart, an empty element\is{empty element} (\littlev) has to be assumed (see Section~\ref{sec-little-v}).

在\ref{Abschnitt-PSA}讨论刺激贫乏论的过程中我们就应该清楚,只允许二叉结构是我们天赋语言知识的一部分的假设只是一种猜想。Haegeman没有为这一假设提供任何证据。在我们所见的各种理论的讨论中,可以用平铺结构来描述数据。例如,可以假设,在英语中动词与其论元用一个平铺结构来组合\citep[\page 39]{ps2}。有时候有一些理论内部的原因使得选择其中一种分支或另外一种,但是对于其他理论并非总是可行。例如,\gbtc 中的约束理论\isc{约束理论}\is{Binding Theory}是通过句法树中的统制关系来实现的 \citep[\page 188]{Chomsky81a}。如果假设句法结构对于代词约束有重要作用的话(见第~\pageref{Seite-Bindungstheorie}页),那么就可以根据可见的约束关系来就句法结构做出假设(也可以参见\ref{sec-little-v})。但是,约束现象在不同理论中受到了不同对待。在LFG\indexlfgc 中,对于f"=结构\isc{f-结构}\is{f"=structure}的限制用于约束理论\citep{Dalrymple93a},但是在HPSG\indexhpsgc 理论中约束理论用论元结构列表\isfeat{arg-st}(以一种特定顺序排列的价信息,见\ref{Abschnitt-Arg-St})来操作。
%It should have become clear in the discussion of the arguments for the Poverty of the Stimulus in Section~\ref{Abschnitt-PSA} that
%the assumption that only binary-branching structures are possible is part of our innate linguistic knowledge is nothing more than pure
%speculation. Haegeman offers no kind of evidence for this assumption. As shown in the discussions of the various theories we have seen,  
%it is possible to capture the data with flat structures. For example, it is possible to assume that, in English, the verb
%is combined with its complements in a flat structure \citep[\page 39]{ps2}. There are sometimes theory-internal reasons for
%deciding for one kind of branching or another, but these are not always applicable to other theories. For example, Binding Theory\is{Binding Theory}
%in \gbt is formulated with reference to dominance relations in trees \citep[\page 188]{Chomsky81a}. If one assumes that syntactic structure plays
%a crucial role for the binding of pronouns (see page~\pageref{Seite-Bindungstheorie}), then it is possible to make assumptions about syntactic
%structure based on the observable binding relations  (so also Section~\ref{sec-little-v}). Binding data have, however, received a very different treatment in various theories.
%In LFG\indexlfg, constraints on f"=structure\is{f"=structure} are used for Binding Theory \citep{Dalrymple93a}, whereas Binding Theory
%in HPSG\indexhpsg operates on argument structure lists\isfeat{arg-st} (valence information that are ordered in a particular way,
%see Section~\ref{Abschnitt-Arg-St}).
 
与Haegeman观点相反的是\citet[\S~1.6.2]{Croft2001a}提出了支持平铺结构。在其\indexcxgc 激进构式语法(Radical Construction Grammar)FAQ中,Croft注意到像(\mex{1}a) 中所示的短语构式可以被转换成(\mex{1}b)所指的范畴语法的词项\indexcgc。
%The opposite of Haegeman's position is the argumentation for flat structures put forward by Croft
%\citeyearpar[Section~1.6.2]{Croft2001a}. In his\indexcxg Radical Construction Grammar FAQ, Croft observes that
%a phrasal construction such as the one in (\mex{1}a) can be translated into a Categorial Grammar
%lexical entry\indexcg like (\mex{1}b).
\eal
\ex {}[\sub{VP} V NP ]
\ex VP/NP
\zl
他认为范畴语法的一个劣势在于它只允许二叉结构,而确实存在包含多于两个部分的构式(第49页)。但是他没有揭示这个问题的准确原因。他甚至自己也承认在范畴语法中可以用多于两个论元的方式来表示构式。对于一个双及物动词,英语范畴语法中的词项应该如 (\mex{1})所示:
%He claims that a disadvantage of Categorial Grammar is that it only allows for binary-branching structures and yet there exist constructions
%with more than two parts (p.\,49). The exact reason why this is a problem is not explained, however. He even acknowledges himself that
%it is possible to represent constructions with more than two arguments in Categorial Grammar. For a ditransitive verb, the entry in Categorial
%Grammar of English would take the form of (\mex{1}):
\ea
((s\bs np)/np)/np
\z
如果我们考察图\vref{Abbildung-TAG-flach-binaer}所示的TAG初级树,就清楚向平铺树和二叉树中融入语义信息都是可行的。二叉树对应范畴语法中的派生树。
%If we consider the elementary trees for TAG in Figure~\vref{Abbildung-TAG-flach-binaer}, it becomes clear that it is equally possible
%to incorporate semantic information into a flat tree and a binary-branching tree.
\begin{figure}
\hfill
\adjustbox{valign=c}{%
\begin{forest}
tag
[S
	[NP$\downarrow$]
	[VP
		[V
			[gives;给]]
		[NP$\downarrow$]
		[NP$\downarrow$]]]
\end{forest}
}
\hfill
\adjustbox{valign=c}{%
\begin{forest}
tag
[S
	[NP$\downarrow$]
	[VP
		[V$'$
			[V
				[gives;给]]
			[NP$\downarrow$]]
		[NP$\downarrow$]]]
\end{forest}}
\hfill\mbox{}
\caption{\label{Abbildung-TAG-flach-binaer}平铺和二叉的基本树}
%	\caption{\label{Abbildung-TAG-flach-binaer}Flat and binary-branching elementary trees}
\end{figure}%
在图\ref{Abbildung-TAG-flach-binaer}的两种分析中,都要赋予带有多个论元的中心语一个意义。归根结底,所需的确切结构取决于人们希望构成的结构的各种限制。本书没有论及这类限制,但是正如上面所论述的,有些理论借助树结构来建立约束关系\isc{约束理论}\is{Binding Theory}的模型。反身代词\isc{代词!反身}\is{pronoun!reflexive}必须限制在树的一个特定局域中。在LFG\indexlfgc 和HPSG\indexhpsgc 等理论中,这些约束限制没有借助树来刻画。这意味着来自于图\ref{Abbildung-TAG-flach-binaer}某一结构(或其他树结构)的约束现象的证据只是一种理论内部的证据。
%The binary-branching tree corresponds to a Categorial Grammar derivation. In both analyses in 
%Figure~\ref{Abbildung-TAG-flach-binaer}, a meaning is assigned to a head that occurs with a certain
%number of arguments. Ultimately, the exact structure required depends on the kinds of restrictions on structures
%that one wishes to formulate.
%In this book, such restrictions are not discussed, but as explained above some theories model binding relations\is{Binding Theory}
%with reference to tree structures. Reflexive pronouns\is{pronoun!reflexive} must be bound within a particular local domain inside the
%tree. In theories such as LFG\indexlfg and HPSG\indexhpsg, these binding restrictions are formulated
%without any reference to trees.
%\todostefan{Das stand irgendwie schon oben. Vielleicht ist aber ein
%  bisschen Redundanz auch OK.}
%This means that evidence from binding data for one of the structures in Figure~\ref{Abbildung-TAG-flach-binaer} (or for
%other tree structures) constitutes nothing more than theory-internal evidence.

假设句法树有多种结构的另一个动因是可以在任意结点插入附加语\isc{附加语}\is{adjunct}。在第\ref{Kapitel-HPSG}章中,给出了一个基于HPSG的假设双分支结构的分析。有了这一分析,就可能将一个附加语附加到任意结点,并借此解释附加语在中间区域的自由排列:
%Another reason to assume trees with more structure is the possibility to insert adjuncts\is{adjunct} on any node.
%In Chapter~\ref{Kapitel-HPSG}, an HPSG analysis for German that assumes binary-branching structures was proposed.
%With this analysis, it is possible to attach an adjunct to any node and thereby explain the free ordering of adjuncts
%in the middle field:
\eal
\ex 
\gll {}[weil] der Mann der Frau das Buch \emph{gestern} gab\\
	 {}\spacebr{}因为 \textsc{art}.\textsc{def} 男人 \textsc{art}.\textsc{def} 女人 \textsc{art}.\textsc{def} 书 昨天 给\\
\mytrans{因为这个男人昨天给这个女人这本书}	 
%	 {}\spacebr{}because the man the woman the book yesterday gave\\
%\mytrans{because the man gave the woman the book yesterday}
\ex 
\gll {}[weil] der Mann der Frau \emph{gestern} das Buch gab\\
	 {}\spacebr{}因为 \textsc{art}.\textsc{def} 男人 \textsc{art}.\textsc{def} 女人 昨天 \textsc{art}.\textsc{def} 书 给\\
%	 {}\spacebr{}because the man the woman yesterday the book gave\\
\ex 
\gll {}[weil] der Mann \emph{gestern} der Frau das Buch gab\\
	 {}\spacebr{}因为 \textsc{art}.\textsc{def} 男人 昨天 \textsc{art}.\textsc{def} 女人 \textsc{art}.\textsc{def} 书 给\\
%	 {}\spacebr{}because the man yesterday the woman the book gave\\
\ex 
\gll {}[weil] \emph{gestern} der Mann der Frau das Buch gab\\
	 {}\spacebr{}因为 昨天 \textsc{art}.\textsc{def} 男人 \textsc{art}.\textsc{def} 女人 \textsc{art}.\textsc{def} 书 给\\
%	 {}\spacebr{}because yesterday the man the woman the book gave\\
\zl
但是这个分析并不是唯一的可能。还可以假设一个完全平铺的结构,在这一结构中论元和附加语由一个结点统制。 \citet{Kasper94a}在\hpsgc 理论框架中给出了这样一种分析(也可以参见\ref{Abschnitt-Adjunkte-GPSG}中GPSG使用元规则\isc{元规则}\is{metarule}来引入附加语的分析)。Kapser需要复杂的关系约束来产生句法树中元素之间的句法关系,并且使用动词和附加语来计算整个成分的语义贡献。使用二叉结构的分析比使用复杂的关系约束的方法更加简单并且——鉴于平铺结构缺少理论外部的证据——应该选用平铺结构的分析。关于这一点,有人可能会反对说英语的附加语不能出现在论元之间的所有位置上,所以借助二叉的范畴语法分析和图~\ref{Abbildung-TAG-flach-binaer}中的TAG分析都是错误的。但是,这是不对的,因为指定附加语的附加位置在范畴语法中是非常重要的。一个副词有范畴 (s\bs np)\bs (s\bs np) 或 (s\bs np)/(s\bs np) ,所以只能与图~\ref{Abbildung-TAG-flach-binaer}所示的VP结点对应的成分组合。以同样的方式,在TAG中一个副词的初级树也只能附加到VP结点上(见第~\pageref{abb-Adjunktion}页的图~\ref{abb-Adjunktion})。所以,就英语附加语的处理而言,二叉结构因此不会做出任何错误的预测。
\isc{分支!二叉|)}\is{branching!binary|)}
%This analysis is not the only one possible, however. One could also assume an entirely flat structure where arguments
%and adjuncts are dominated by one node.  \citet{Kasper94a} suggests this kind of analysis in
%\hpsg (see also Section~\ref{Abschnitt-Adjunkte-GPSG} for GPSG analyses that make use of metarules\is{metarule} for the introduction of adjuncts). Kasper requires complex relational constraints\is{relation} that
%create syntactic relations between elements in the tree and also compute the semantic contribution of the entire constituent using the meaning
%of both the verb and the adjuncts. The analysis with binary-branching structures is simpler than those with complex relational constraints and --
%in the absence of theory-external evidence for flat structures -- should be preferred to the analysis with flat structures.
%At this point, one could object that adjuncts in English cannot occur in all positions between arguments and therefore the binary-branching
%Categorial Grammar analysis and the TAG analysis in Figure~\ref{Abbildung-TAG-flach-binaer} are wrong. This is not correct, however, as it is
%the specification of adjuncts with regard to the adjunction site that is crucial in Categorial Grammar.
%An adverb has the category (s\bs np)\bs (s\bs np) or (s\bs np)/(s\bs np) and can therefore only be combined with constituents that correspond to the VP node in
%Figure~\ref{Abbildung-TAG-flach-binaer}. In the same way, an elementary tree for an adverb in TAG
%can only attach to the VP node (see Figure~\ref{abb-Adjunktion} on
%page~\pageref{abb-Adjunktion}). For the treatment of adjuncts in English, binary-branching
%structures therefore do not make any incorrect predictions.
%\is{branching!binary|)}


%      <!-- Local IspellDict: en_US-w_accents -->
