\section*{语言科学出版社:属于学者的高质量语言学书籍}
%\section*{Language Science Press: scholar-owned high quality linguistic books}

在2012年,有一群人发现出版界的情况令人难以容忍,他们一致认为有必要在公开平台上出版语言学书籍。也就是说,需要一个针对所有读者和作者公开的平台。我建立了一个网页,并征集了支持者,他们是来自全世界各地的著名语言学家,Martin Haspelmath和我随后就成立了语言科学出版社。几乎同时,DFG公布了一项公开专著的项目,我们申请\citep{MH2013a}并获得了资助(18个申请中只有两家获得了资助)。这笔钱支付给一位主任(Dr.\ Sebastian Nordhoff)、一位经济学家(Debora Siller)和两位程序员(Carola Fanselow和Dr.\ Mathias Schenner)。他们在公开专著出版社(OMP)出版平台工作,并应用转换软件来从我们的\LaTeX{}编码中生成不同的格式(ePub、XML、HTML)。Svantje Lilienthal负责OMP的文档,制作屏幕录像,并为作者、读者和编辑提供用户支持。
%In 2012 a group of people found the situation in the publishing business so unbearable that they
%agreed that it would be worthwhile to start a bigger initiative for publishing linguistics books in
%platinum open access, that is, free for both readers and authors. I set up a web page and collected
%supporters, very prominent linguists from all over the world and all subdisciplines and Martin
%Haspelmath and I then founded Language Science Press. At about the same time the DFG had announced
%a program for open access monographs and we applied \citep{MH2013a} and got funded (two out of 18 applications got
%funding). The money is used for a coordinator (Dr.\ Sebastian Nordhoff) and an economist (Debora
%Siller), two programmers (Carola Fanselow and Dr.\ Mathias Schenner), who work on the publishing
%plattform Open Monograph Press (OMP) and on conversion software that produces various formats (ePub, XML,
%HTML) from our \LaTeX{} code. Svantje Lilienthal works on the documentation of OMP, produces
%screencasts and does user support for authors, readers and series editors.

OMP在公开评论方面和社区建设的游戏化工具方面进行了扩展。所有语言科学出版社出版的图书都至少由两位外部审稿人审稿。审稿人和作者可以同意出版这些审稿意见,并使得整个过程更为透明(也可以看\citew{Pullum84a}关于期刊文章的公开评论的建议)。另外,还有可选的第二轮评审过程:公开评审。这一阶段对所有人都是公开的。整个社团都可以评论语言科学出版社出版的书籍。在第二轮评审阶段后,这通常需要持续两个月的时间,作者会进行修订,进而出版出改进的版本。这本书是经历了这个公开评审阶段的第一本书。标注了公开评审意见的版本可以通过\href{\lsURL}{web page of this book}获得。
%OMP is extended by open review facilities and community-building gamification tools
%\citep{MuellerOA,MH2013a}. All Language Science Press books are reviewed by at least two external
%reviewers. Reviewers and authors may agree to publish these reviews and thereby make the whole
%process more transparent (see also \citew{Pullum84a} for the suggestion of open reviewing of journal
%articles). In addition there is an optional second review phase: the open
%review. This review is completely open to everybody. The whole community may comment on the document
%that is published by Language Science Press. After this second review phase, which usually lasts for
%two months, authors may revise their publication and an improved version will be published. This
%book was the first book to go through this open review phase. The annotated open review version of this book is still available via
%the \href{\lsURL}{web page of this book}. 

如今,语言科学出版社拥有17个语言学不同领域的系列书籍,这些高水平的编辑来自各个大陆。我们有18本已经出版的书籍,还有17本即将出版的书籍,还有146本书籍的作者对出版社表示出了极大的兴趣。系列编辑和作者主要负责用\LaTeX{}编辑的手稿,但是他们也有由语言科学出版社建立的基于网络的格式模版以及社区里的志愿者的支持。校对也是基于社区的。截至目前有53位学者帮助了我们出版的图书。他们的工作被记录在名人堂中: \url{http://langsci-press.org/about/hallOfFame}。
%Currently, Language Science Press has 17 series on various subfields of linguistics with high
%profile series editors from all continents. We have 18 published and 17 forthcoming books and 146
%expressions of interest. Series editors  and authors are responsible for
%delivering manuscripts that are typeset in \LaTeX{}, but they are supported by a web-based typesetting
%infrastructure that was set up by Language Science Press and by volunteer typesetters from the
%community. Proofreading is also community-based. Until now 53 people helped improving our
%books. Their work is documented in the Hall of Fame: \url{http://langsci-press.org/about/hallOfFame}.

如果你认为这类教科书应该对那些想阅读这些书籍的人免费获得,而且出版社不应该变成利益驱动的出版社,那么你就应该加入语言科学出版社的社区,并且在以下几个方面支持我们:你可以在语言科学出版社上注册,并将你的名字列在其他将近600名热心学者之中,你可以用你的时间帮助校对或者修改格式,或者你可以为某本书或者语言科学出版社捐钱。我们也在寻找基金会、社团、语言学系和大学图书馆等机构的支持。如何帮助我们的详细信息列在下面的网页中: \url{http://langsci-press.org/about/support}。如有问题,请联系我或者语言科学出版社的主任\href{mailto:contact@langsci-press.org}{contact@langsci-press.org}。
%If you think that textbooks like this one should be freely available to whoever wants to read them
%and that publishing scientific results should not be left to profit-oriented publishers, then you
%can join the Language Science Press community and support us in various ways: you can register with Language Science Press and have your name
%listed on our supporter page with almost 600 other enthusiasts, you may devote your time and help
%with proofreading and/or typesetting, or you may donate money for specific books or for Language
%Science Press in general. We are also looking for institutional supporters like foundations,
%societies, linguistics departments or university libraries. Detailed information on how to support
%us is provided at the following webpage: \url{http://langsci-press.org/about/support}.
%In case of questions, please contact me or the Language Science Press coordinator at \href{mailto:contact@langsci-press.org}{contact@langsci-press.org}.


~\medskip

%\noindent
\begin{flushright}
\begin{tabular}{c}
斯特凡 $\cdot$ 穆勒\\
2016年3月11日\\
柏林\\
\end{tabular}
\end{flushright}
%Berlin, \today\hfill Stefan Müller

%      <!-- Local IspellDict: en_US-w_accents -->
