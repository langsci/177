%% -*- coding:utf-8 -*-
\chapter{语言习得}
%\chapter{Language acquisition}
\label{chap-acquisition}

语言学家\isce[|(]{习得}{acquisition}和哲学家皆热衷于了解人类语言习得的能力。假设在儿童时期有相关输入的话,语言习得通常都是毫不费力就能获得的能力。 \citet[\page  24--25]{Chomsky65a}提出一个条件,也就是语法理论一定要能提供语言习得的模型,否则,语法理论最多只停留在描述的阶段。这一章里,我们将从一些不同的理论角度来讨论语言习得。
%Linguists\is{acquisition|(} and philosophers are fascinated by the human ability to acquire language. Assuming the relevant input during childhood,
%language acquisition normally takes place completely effortlessly.
% \citet[\page  24--25]{Chomsky65a} put forward the requirement that a grammatical theory must provide a plausible model of language acquisition.
%Only then could it actually explain anything and would otherwise remain descriptive at best. In this section, we will discuss
%theories of acquisition from a number of theoretical standpoints.

\section{原则 {\normalfont \&} 参数}
%\section{Principles \& Parameters}
\label{Abschnitt-PP}\label{sec-pro-drop}

\largerpage
关于语言习得的一个非常具有影响力的解释是Chomsky\citeyearpar{Chomsky81a}的原则与参数模型\isce[|(]{参数}{parameter}\isce[|(]{原则 \& 参数}{Principles \& Parameters}。Chomsky假设有一个内在的普遍语法,它涵盖跟所有语言都相关的知识。语言以特定的方式存在差异。在核心语法\isce{核心语法}{core grammar}领域,语言之间的每个差异都存在一个具有特定值的特征。参数的取值通常是二元的,也就是说,取值或者是“+”或者是“$-$”。依赖于参数的设定,一种语言会有一些特定的属性,即设置一个属性会决定一种语言是否属于语言的一个特定类型。假设参数可以同时影响语法的多种属性\citep[\page 6]{Chomsky81a}。例如, \citet{Rizzi86a}指出,pro-脱落参数\isce{参数}{parameter}{pro-脱落}{pro-drop}影响指称主语是否可以省略,虚位是否可以不出现,主语\iscesub{提取}{extraction}{主语}{subject}是否可以从带有标补词(that-t环境\isce{that-t环境}{that-t})的从句或疑问句中提取以及是否可以在VO语言中将主语置于动词之后(见\citealp[第4.3节]{Chomsky81a};\citealp[\page 12]{Meisel95a})。我们已经发现所有假设的关联都存在反例。\dotfootnote{%
\label{fn-Expletiva-Pro-Drop}%
见 \citew{Haider94c-u}和 \citew[第2.2节]{Haider2001a}的综述。Haider认为虚位主语不出现与pro-脱落之间至少存在一种关联。但是,加利西亚语\ilce{加利西亚语}{Galician}是一个带有虚位主语代词的pro-脱落语言\citep[第2.5节]{RU90a-u}。 \citet[\page 314]{Franks95a-u}指出上索布语\il{索布语 Sorbian!上索布语 Upper}和下索布语\il{索布语 Sorbian!下索布语 Lower}是带有虚位主语的pro-脱落语言。 \citet[\page 218]{SP2002b}指出在现代意大利语\ilce{意大利语}{Italian}中有一个虚位代词ci,虽然意大利语\ilce{意大利语}{Italian}是一种pro-脱落语。  
} 另一个参数的例子是\ref{Abschnitt-Kopfstellungsparameter}中讨论的中心语指向(Head Directionality)参数。正如所示,有的语言其中心语具有不同的支配方向。在其综述文章中, \citet{Haider2001a}仍然提到了参数化邻接原则\iscesub{参数}{parameter}{邻接}{subjacency},但是指出邻接在较新的理论版本中不再是一个原则(关于邻接更多的信息,见\ref{Abschnitt-Subjazenz-Extraktion})。
%A\is{parameter|(}\is{Principles \& Parameters|(} very influential explanation of language acquisition is Chomsky's Principles \& Parameters model
%\citeyearpar{Chomsky81a}. Chomsky assumes that there is an innate Universal Grammar that contains knowledge that is equally relevant for all languages.
%Languages can then vary in particular ways. For every difference between languages in the area of core grammar\is{core grammar}, there is a feature with a specific
%value. Normally, the value of a parameter is binary, that is, the value is either `+' or `$-$'.
%Depending on the setting of a parameter, a language will have certain properties, that is, setting a parameter determines whether
%a language belongs to a particular class of languages.
%Parameters are assumed to influence multiple properties of a grammar simultaneously \citep[\page 6]{Chomsky81a}. For example,  \citet{Rizzi86a} claims
%that the pro-drop parameter\is{parameter!pro-drop} affects whether referential subjects can be omitted, the absence of expletives, subject
%extraction\is{extraction!subject} from clauses with complementizers (\emph{that}-t contexts\is{that-t}) and interrogatives and
%finally the possibility of realizing the subject postverbally in VO-languages (see
%\citealp[Section~4.3]{Chomsky81a}; \citealp[\page 12]{Meisel95a}). It has been noted that there are counter-examples to all the correlations assumed.\footnote{%
%\label{fn-Expletiva-Pro-Drop}%
%  See  \citew{Haider94c-u} and  \citew[Section~2.2]{Haider2001a} for an overview. Haider assumes that there is at least a correlation
%  between the absence of expletive subjects and pro-drop. However, Galician\il{Galician} is a pro-drop language with expletive subject
%  pronouns \citep[Section~2.5]{RU90a-u}.  \citet[\page 314]{Franks95a-u} cites Upper\il{Sorbian!Upper} and Lower Sorbian\il{Sorbian!Lower} as pro-drop
%  languages with expletive subjects.  \citet[\page 218]{SP2002b} point out that there is an expletive pronoun \emph{ci} in modern Italian\il{Italian}
%  although Italian\il{Italian} is classed as a pro-drop language.}
%Another example of a parameter is the Head Directionality Parameter discussed in Section~\ref{Abschnitt-Kopfstellungsparameter}.
%As was shown, there are languages where heads govern in different directions. In his overview article,  \citet{Haider2001a} still mentions
%the parametrized Subjacency Principle\is{parameter!subjacency} but notes that subjacency\is{subjacency} is no longer assumed as a principle
%in newer versions of the theory (see Section~\ref{Abschnitt-Subjazenz-Extraktion} for more on subjacency).

\citet{Snyder2001a}发现了很多现象与能产的词根式复合构词具有相关性,正如两个名词复合的表现。他认为复杂谓词结构的习得与复合结构的习得相关联,并且有一个参数对这一类复合起作用,同时对以下现象也起作用:
% \citet{Snyder2001a} discovered a correlation of various phenomena with productive root compounding
%as it is manifested for instance in compounding of two nouns. He argues that the acquisition of
%complex predicate formation is connected to the acquisition of compound structures and that there is
%a parameter that is responsible for this type of compounding and simultaneously for the following set of
%phenomena:
\eal\settowidth\jamwidth{(double-object dative)}
\ex 
\gll John painted the house red. \\
     John 粉刷 \defart{} 房子 红\\              \jambox{(动结构式)\isce{动结构式}{resultative construction}}
\mytrans{John把房子粉刷成红色。} 
%\ex John painted the house red.                 \jambox{(resultative\is{resultative construction})}
\ex 
\gll Mary picked the book up/picked up the book. \\           
     Mary 拾 \defart{} 书 \textsc{part}/拾 \textsc{part} \defart{} 书\\ \jambox{(动词—小品词)\iscesub{动词}{verb}{小品词}{particle}}
\mytrans{Mary把书拾起来。} 
%\ex Mary picked the book up/picked up the book. \jambox{(verb-particle\is{verb!particle})}
\ex 
\gll Fred made Jeff leave. \\
    Fred 强迫 Jeff 离开\\              \jambox{(make—使役构式)}
\mytrans{Fred让Jeff离开。}  
%\ex Fred made Jeff leave.                       \jambox{(\emph{make}-causative)}
\ex 
\gll Fred saw Jeff leave.\\ 
    Fred 看见 Jeff 离开\\             \jambox{(感知表达构式)}
\mytrans{Fred看见Jeff离开了。}  
%\ex Fred saw Jeff leave.                        \jambox{(perceptual report)}
\ex 
\gll Bob put the book on the table. \\
    Bob 放置 \defart{} 书 \textsc{prep} \defart{} 桌子\\               \jambox{(put—处所构式)}
\mytrans{Bob把书放在桌子上了。}  
%\ex Bob put the book on the table.              \jambox{(\emph{put}-locative)}
\ex 
\gll Alice sent the letter to Sue.  \\
    Alice 寄 \defart{} 信 \textsc{prep} Sue\\             \jambox{(to—与格构式)}
\mytrans{Alice寄信给Sue。}  
%\ex Alice sent the letter to Sue.               \jambox{(\emph{to}-dative)}
\ex 
\gll Alice sent Sue the letter.\\ 
    Alice 寄 Sue \defart{} 信\\           \jambox{(双宾与格构式)\iscesub{动词}{verb}{双及物}{ditransitive}}
\mytrans{Alice寄给Sue信。}  
%\ex Alice sent Sue the letter.                  \jambox{(double-object dative\is{verb!ditransitive})}
\zl 
\addlines
Snyder检验了来自多个语系的语言:亚非语系、奥斯特罗-亚细亚语系、奥斯特罗西亚语系、芬兰--乌戈尔语族、印欧语系(德语,罗曼语系,斯拉夫语系),日语-韩语,尼日尔-科尔多凡语系(班图语系)和汉藏语系以及美国手语\ilcesub{手语}{sign language}{美国手语}{American (ASL)}和孤立语巴斯克语。这些被检验的语言或者具有所有这些现象或者一个也没有。这都是经过各自语言的母语者验证过的。另外,对于英语来说,只要名名复合词被大量使用,这些现象就会被习得,这一论断也用CHILDES数据验证过了。这一结果是积极的,只有双宾构式是一个例外,对此也找到了解释。(\mex{0})中现象之间的关联非常有趣,并曾被视为存在某种参数的证据,该参数可以关联几种语言现象。但是, \citet{Son2007a}和 \citet{SonS2008a}表明Synder对于日语\ilce{日语}{Japanese}的论断是错误的,还有其他语言,如韩语\ilce{韩语}{Korean}、希伯来语\ilce{希伯来语}{Hebrew}、捷克语\ilce{捷克语}{Czech}、马拉雅拉姆语\ilce{马拉雅拉姆语}{Malayalam}和\ilce{爪哇语}{Javanese},在这些语言中有些现象并没有联系。
%Snyder examined languages from various language groups: Afroasiatic, Austroasiatic, Austronesian,
%Finno-Ugric, Indo-European (Germanic, Romance, Slavic), Japanese-Korean, Niger-Kordofanian (Bantu),
%and Sino-Tibetan, as well as American Sign\il{sign language!American (ASL)} Language and the language isolate Basque\il{Basque}. The languages
%that were examined either had all of these phenomena or none. This was tested with native speakers
%of the respective languages. In addition the claim that these phenomena are acquired once noun-noun
%compounds are used productively was tested for English using CHILDES data. The result was positive
%with the exception of the double object construction, for which an explanation was provided. The
%correlation of the phenomena in (\mex{0}) is interesting and was interpreted as proof of the
%existence of a parameter that correlates several phenomena in a language. However,  \citet{Son2007a}
%and  \citet{SonS2008a} showed that Snyder's claims for Japanese\il{Japanese} were wrong and that there
%are further languages like Korean\il{Korean}, Hebrew\il{Hebrew}, Czech\il{Czech},
%Malayalam\il{Malayalam}, Javanese\il{Javanese}  in which some of the phenomena show no correlations. 
%%  \citet[\page 395]{SonS2008a} conclude that language-wide parameters of the type
%% discussed here do never partition the world languages into two sets and that all parameters that
%% were suggested in the past had to be subdivided into smaller ones. They therefore suggest that parametric variation
%% is confined to lexical items.

\citet{GW94a}讨论了成分序列的习得,并且假设了涉及动词相对主语位置(SV vs.\ VS)\iscesub{参数}{parameter}{SV}{SV}、相对宾语位置(VO vs.\ OV)\iscesub{参数}{parameter}{V2}{V2}以及V2属性的三个参数。至于哪些参数决定了语言的结构,文献中还没有达成一致(有关综述和批判性的讨论见\citealp[第3.2节]{Newmeyer2005a}和\citealp{Haspelmath2008a})。 \citet[\page 346--347]{Fodor98a}假设有20到30个参数, \citet[\page 408]{GW94a}认为是40个, \citet[\page 349]{Baker2003b}认为是10到20个,  \citet[\page 541]{RH2005a}认为是50到100个。在文献中还没有达成共识,应该假设哪些参数,它们怎么相互作用以及它们能预测什么。但是,可以仔细考虑怎么样通过设置需要的参数从UG中派生出一个特定语言的语法。Chomsky最初的想法\citeyearpar[第3.5.1节]{Chomsky86a}是,一旦从语言输入中获取相关数据,儿童就会基于语言输入为参数设置取值(也可以参见\citealp*{GW94a,NKN2001a})。在一个特定的给定事件中,学习者就有一个带有相关参数设定的语法,这一语法对应于当前的输入。为了完全习得一个语法,所有参数都必须被设置一个取值。理论上,30个语句就足以获得一个带有30个参数的语法,如果这些句子为一个特定参数值提供没有歧义的证据的话。
% \citet{GW94a} discuss the acquisition of constituent order and assume three parameters that concern the
%position of the verb relative to the subject (SV vs.\ VS)\is{parameter!SV} and relative to the
%object (VO vs.\ OV)\is{parameter!V2} as well as the V2-property\is{parameter!V2}. There is no consensus in the literature about which
%parameters determine the make-up of languages (see \citealp[Section~3.2]{Newmeyer2005a} and \citealp{Haspelmath2008a}
%for an overview and critical discussion).
% \citet[\page 346--347]{Fodor98a} assumes that there are 20 to 30 parameters,  \citet[\page 408]{GW94a}
%mention the number 40,  \citet[\page 349]{Baker2003b} talks of 10 to 20 and  \citet[\page 541]{RH2005a}
%of 50 to 100. There is no consensus in the literature as to which parameters one should assume, how they interact
%and what they predict. However, it is nevertheless possible to contemplate how a grammar of an individual language
%could be derived from a UG with parameters that need to be set.
%Chomsky's original idea \citeyearpar[Section~3.5.1]{Chomsky86a} was that the child sets the value of
%a parameter based on the language input as soon as the relevant evidence is present from the input
%(see also \citealp*{GW94a,NKN2001a}). At a given point in time, the learner has a grammar with certain parameter
% settings that correspond to the input seen so far. In order to fully acquire a grammar, all parameters must be assigned a value. In theory,
% thirty utterances should be enough to acquire a grammar with thirty parameters if these utterances
% provide unambiguous evidence for a particular parameter value.

这一方法经常遭受批评。如果设定一个参数就会导致学习者使用一个不同的语法,那么可以预见语言学行为上会发生突变。但是,事实并非如此 (\citealp[\page 731]{Bloom93a})。 \citet[\page 343--344]{Fodor98a} 也注意到了下面三个问题:1)参数可以影响无法从可见的成分序列中观察到的东西。2)许多句子在设定了一个特定参数的情况下都是有歧义的,也就是说,有时候存在与一个语句兼容的参数的多重组合\citep{BN96a,Fodor98b}。3)参数间的互动也存在一个问题。通常情况下,在一个语句中有多种参数起作用,所以很难确定某个参数有何作用,以及怎样决定参数的取值\nocite{Pullum83a}。
%This approach has often been criticized. If setting a parameter leads to a learner using a
%different grammar, one would expect sudden changes in linguistic behavior. This is, however,
%not the case (\citealp[\page 731]{Bloom93a}).  \citet[\page 343--344]{Fodor98a} also notes
%the following three problems: 1) Parameters can affect things that are not visible from the
%perceptible constituent order. 2) Many sentences are ambiguous with regard to the setting of a particular
%parameter, that is, there are sometimes multiple combinations of parameters compatible with one
%utterance. Therefore, the respective utterances cannot be used to set any parameters \citep{BN96a,Fodor98b}. 3) There is a problem with the interaction of parameters.
%Normally multiple parameters play a role in an utterance such that it can be difficult to determine
%which parameter contributes what and thus how the values should be determined.\nocite{Pullum83a}

1)和2)可以运用Gibson \& Wexler提出的成分序列参数进行解释:假设一个儿童听到了(\mex{1})中所示的英语\ilce{英语}{English}和德语的例子:
%Points 1) and 2) can be explained using the constituent order parameters of Gibson \& Wexler:\il{English}
%imagine a child hears sentences such as the English and the German examples in (\mex{1}):
\eal
\ex 
\gll Daddy drinks juice.\\  
    爸爸 喝 果汁\\
\mytrans{爸爸喝果汁。}  
%\ex Daddy drinks juice.
\ex 
\gll Papa trinkt Saft.\\
     爸爸 喝 果汁\\
\mytrans{爸爸喝果汁。}  
%\gll Papa trinkt Saft.\\
%     daddy drinks juice\\
\zl
即便这两个句子结构迥异,但是两者看起来完全一样。按照当前讨论的理论,英语句子的结构见第\pageref{Abb-GB-englischer-Satz-ohne-Hilfsverb}页的图\ref{Abb-GB-englischer-Satz-ohne-Hilfsverb},(\mex{1}a)是其简化形式。而德语句子的结构见第\pageref{Abb-GB-Vorfeldbesetzung}页的图\ref{Abb-GB-Vorfeldbesetzung} ,对应着这里的 (\mex{1}b):
%These sentences look exactly the same, even though radically different structures are assumed for each.
%According to the theories under discussion, the English sentence has the structure shown in Figure~\ref{Abb-GB-englischer-Satz-ohne-Hilfsverb} on
%page~\pageref{Abb-GB-englischer-Satz-ohne-Hilfsverb} given in abbreviated form in (\mex{1}a).
%The German sentence, on the other hand, has the structure in Figure~\ref{Abb-GB-Vorfeldbesetzung} 
%on page~\pageref{Abb-GB-Vorfeldbesetzung} corresponding to (\mex{1}b):
\eal
\ex 
\gll {}[\sub{IP} [Daddy [\sub{I$'$} \_$_k$ [\sub{VP} drinks$_k$ juice]]].\\  
       {}        \spacebr{}爸爸 \spacebr{} {} {} 喝$_k$ 果汁\\
%\ex {}[\sub{IP} [Daddy [\sub{I$'$} \_$_k$ [\sub{VP} drinks$_k$ juice]]].
\mytrans{爸爸喝果汁。}
\ex 
\gll {}[\sub{CP} Papa$_i$ [\sub{C$'$} trinkt$_k$ [\sub{IP} \_$_i$ [\sub{I$'$} [\sub{VP} Saft \_$_k$] \_$_k$]]]].\\  
   {} 爸爸 {} 喝 {} {} {} {} 果汁\\
   \mytrans{爸爸喝果汁。}
%\ex {}[\sub{CP} Papa$_i$ [\sub{C$'$} trinkt$_k$ [\sub{IP} \_$_i$ [\sub{I$'$} [\sub{VP} Saft \_$_k$] \_$_k$]]]].
\zl
英语有基本的SVO\isce{SVO}{SVO}语序。动词与宾语组合成一个组成成分(VP)然后再跟主语进行组合。所以参数设定一定是SV,VO和$-$V2。与此不同的是,德语是一个动词居末和动词二位的语言,所以参数取值应该是SV,OV和$+$V2。如果我们分析 (\mex{-1})中的句子,我们看到两个句子就动词及其论元次序而言,二者是一致的。
%English has the basic constituent order SVO\is{SVO}. The verb forms a constituent with the object (VP) and this
%is combined with the subject. The parameter setting must therefore be SV, VO and $-$V2. German,
%on the other had, is analyzed as a verb-final and verb-second language and the parameter values
%would therefore have to be SV, OV and $+$V2. If we consider the sentences in (\mex{-1}), we see that
%both sentences do not differ from one another with regard to the order of the verb and its arguments.

 \citet{Fodor98a,Fodor98b}由此得出结论,为了搞清楚语法允准的结构到底属于哪个语法类首先需要建立结构,因为首先需要 (\mex{0}b)中的结构才能知道部分成分中的动词在其VP(Saft \_$_k$)中的论元之后。现在的问题是怎样获得结构。具有30个参数的UG对应着2$^{30}$=1,073,741,824种完全实现的语法。假设儿童要同时或依次尝试这么多语法并不现实。
% \citet{Fodor98a,Fodor98b} concludes from this that one first has to build a structure in order to see
%what grammatical class the grammar licensing the structure belongs to since one first needs the structure
%in (\mex{0}b) in order to be able to see
%that the verb in the partial constituent occurs after its argument in the VP (Saft \_$_k$). The question is now how one achieves
%this structure. A UG with 30 parameters corresponds to 2$^{30}$ = 1,073,741,824 fully instantiated grammars.
%It is an unrealistic assumption that children try out these grammars successively or simultaneously.

\citet{GW94a}为这个问题提出了多种解决方式:参数有一个缺省取值\iscesub{参数}{parameter}{缺省值}{default value},学习者只需要改变一个参数就可以分析先前无法分析的句子(贪心性约束\isce{贪心性约束}{Greediness Constraint})。按照这一程序,一次只能改变一个参数(单一取值限制\isce{单一取值限制}{Single Value Constraint}),这一限制是为了排除大的跳跃,这种大的跳跃会导致十分不同的语法(见\citet[\page 612--613]{BN96a})。这就降低了处理要求,虽然有40个参数,但是面对40个参数,最坏的情况仍是学习者一一测试这些参数,即尝试使用40种不同的语法来分析一个句子。这一处理任务仍然是不现实的,所以 \citet[\page 442]{GW94a}还猜想一个输入句只检验一个假设,并假设特定参数只有在儿童成熟\isce{成熟}{maturation}之后才会起作用。在一个特定的时间点,只需要设置少数可及的参数。在设置完这些参数之后,新的参数也就开始起作用了。
% \citet{GW94a} discuss a number of solutions for this problem: parameters have a default value\is{parameter!default value}
%and the learner can only change a parameter value if a sentence that could previously not be analyzed
%can then be analyzed with the new parameter setting (\emph{Greediness Constraint}\is{Greediness Constraint}).
%In this kind of procedure, only one parameter can be changed at a time
%(\emph{Single Value Constraint}\is{Single Value Constraint}),
%which aims at ruling out great leaps leading to extremely different grammars (see \citealp[\page 612--613]{BN96a}, however).
%This reduces the processing demands, however with 40 parameters, the worst case could still be that
%one has to test 40 parameter values separately, that is, try to parse a sentence with 40 different
%grammars. This processing feat is still unrealistic, which is why  \citet[\page 442]{GW94a} additionally
%assume that one hypothesis is tested per input sentence.
%A further modification of the model is the assumption that certain parameters only begin to play a role
%during the maturation\is{maturation} of the child. At a given point in time, there could be only a few accessible parameters
%that also need to be set. After setting these parameters, new parameters could become available.

Gibson \& Wexler的文章论证了输入与参数设定之间的互动绝对是非常重要的。在他们给出的有三个参数的例子中,会出现以下情况:学习者为了分析一个新的句子而设定一个参数,但是设定这一参数会导致目标语法无法习得,因为一次只能修改一个取值并且只有在能够分析更多句子的情况下才能改变取值。学习者在这些问题案例中,达到了一种所谓的局部最大值(local maximum)。\dotfootnote{%
如果假设习得过程是爬山,那么贪心性约束\isce{贪心性约束}{Greediness Constraint}就确保了只能向上爬。但是,可能存在以下情况:某人开始爬错了山并且不能再爬回山下了。 
} Gibson \& Wexler又提出要为特定参数设置缺省值,缺省值可以让学习者避免有问题的情况。对于V2参数\iscesub{参数}{parameter}{V2}{V2},他们假设缺省值\iscesub{参数}{parameter}{缺省值}{default value}是“$-$”。
%In their article, Gibson \& Wexler show that the interaction between input and parameter setting is in no way trivial.
%In their example scenario with three parameters, a situation can arise in which a learner sets a parameter in order to analyze
%a new sentence, however setting this parameter leads to the fact that the target grammar cannot be acquired because only one value can
%be changed at a time and changes can only be made if more sentences can be analyzed than before. The learner reaches a so-called
%local maximum\is{local maximum} in these problematic cases.\footnote{%
%	If one imagines the acquisition process as climbing a hill, then the Greediness Constraint\is{Greediness Constraint} ensures
%	that one can only go uphill. It could be the case, however, that one begins to climb the wrong hill and can no longer get back down.}
%Gibson \& Wexler then suggest assigning a default value to particular parameters, whereby the default value is the one that will cause the learner
%to avoid problematic situations. For the V2 parameter\is{parameter!V2}, they assume `$-$' as the default value\is{parameter!default value}.

\citet{BN96a}认为Gibson \& Wexler错误地计算了有问题的条件,并且如果有人同意他们的假设就更有可能形成一种参数组合,而从这种参数组合中并不能通过改变单个参数取值来获得目标语法。他们指出Gibson \& Wexler没有解决的一个问题是$-$V2(第609页),并且给参数赋予缺省值的假设没有解决这一问题,因为“+”和“-”都有可能导致参数的错误组合。\dotfootnote{%
   \citet{Kohl99a,Kohl2000a}曾经用12个参数来分析这一习得模型。如果为参数设置最佳初始值的话,4096种可能的语法中,将有2336种(57\%)不可能被学会。
} Berwick和Niyogi在他们的文章中指出在上述例子(带有三个参数)中,如果放弃贪心性约束\isce{贪心性约束}{Greediness Constraint}或者单一取值\isce{单一取值限制}{Single Value Constraint}限制,学习者可以更快地学会目标语法。他们假设了一个过程,当句子无法分析时,只需随机地改变一个参数(随机步骤\isce{随机步骤}{Random Step},第615--616页)。两位作者指出,这一方法没有遇到Gibson \& Wexler所遇到的局部最大值\isce{局部最大值}{local maximum}问题,且能比他们的方法更快地达到目标。但是随机步骤(Ramdom Step)能够更快收敛这一事实与参数空间的质量有关(第618页)。因为学界对于参数尚未达成一致意见,所以不可能评估整个系统是如何工作的。
% \citet{BN96a} show that Gibson \& Wexler calculated the problematic conditions incorrectly and
%that, if one shares their assumptions, it is even more frequently possible
%to arrive at parameter combinations from which it is not possible to reach the target grammar by changing individual parameter values.
%They show that one of the problematic cases not addressed by Gibson \& Wexler is $-$V2 (p.\,609) and that the assumption of a default value
%for a parameter does not solve the problem as both `+' and `--' can lead to problematic combinations of parameters.\footnote{%
%   \citet{Kohl99a,Kohl2000a} has investigated this acquisition model in a case with twelve parameters. Of the 4096 possible grammars,
%  2336 (57\%) are unlearnable if one assumes the best initial values for the parameters.
%}
%In their article, Berwick and Niyogi show that learners in the example scenario above (with three
%parameters) learn the target grammar faster if one abandons the Greediness\is{Greediness Constraint} 
%or else the Single Value Constraint\is{Single Value Constraint}.
%They suggest a process that simply randomly changes one parameter if a sentence cannot be analyzed (\emph{Random Step}\is{Random Step}, p.\,615--616).
%The authors note that this approach does not share the problems with the local maxima\is{local maximum} that Gibson \& Wexler had in their example and that it also reaches its goal faster
%than theirs. However, the fact that \emph{Random Step} converges more quickly has to do with the quality of the parameter space (p.\,618).
%Since there is no consensus about parameters in the literature, it is not possible to assess how the entire system works.

 \citet[\page 453]{Yang2004a}批评了经典的原则 \& 参数模型,因为无法观察到设定参数后语法之间的急剧变化。相反,他提出了以下学习机制:
% \citet[\page 453]{Yang2004a} has criticized the classic Principles \& Parameters model since abrupt switching between grammars after setting
%a parameter cannot be observed. Instead, he proposes the following learning mechanism:
\ea
对于输入句子$s$,儿童:(i)选择语法G$_i$的概率为P$_i$ ,(ii)用G$_i$来分析句子$s$,(iii)如果成功,通过提升P$_i$来奖励语法G$_i$,否则通过降低P$_i$来惩罚G$_i$。
%For an input sentence, $s$, the child:
%(i) with probability P$_i$ selects a grammar G$_i$, (ii) analyzes $s$ with G$_i$, (iii) if successful, reward G$_i$ by increasing P$_i$,
%otherwise punish G$_i$ by decreasing P$_i$.
\z
Yang讨论了pro-脱落\iscesub[|(]{参数}{parameter}{pro-脱落}{pro-drop}和话题省略参数\iscesub{参数}{parameter}{话题省略}{topic drop}。在pro-脱落语言中(如意大利语\ilce[|(]{意大利语}{Italian}),可以省略主语;在话题省略语言(\egc 现代汉语\ilce{现代汉语}{Mandarin Chinese})中,充当话题的主语和宾语都可以省略。Yang对比了说英语\ilce[|(]{英语}{English}和说汉语\ilce{现代汉语}{Mandarin Chinese}的儿童,发现说英语的儿童在很早的语言阶段就省略了主语和宾语。他认为造成这一现象的原因是说英语的儿童从使用汉语语法开始。
%Yang discusses the example of the pro-drop\is{parameter!pro-drop|(} and topic drop parameters\is{parameter!topic drop}. In pro-drop languages (\eg Italian\il{Italian|}),
%it is possible to omit the subject and in topic drop languages (\eg Mandarin Chinese\il{Mandarin Chinese}), it possible to omit both the subject and the object if it is
%a topic. Yang compares English-speaking\il{English|(} and Chinese-speaking\il{Mandarin Chinese} children noting that English children omit both subjects and objects
%in an early linguistic stage. He claims that the reason for this is that English-speaking children start off using the Chinese grammar.

pro-脱落参数是原则 \& 参数理论讨论最广泛的参数之一,所以这里将进行更细致地讨论。假设说英语的人必须学会英语中所有的句子都需要一个主语,而说意大利语的人必须学会主语是可以省略的。可以观察到,既学习英语\ilce{英语}{English}又学习意大利语\ilce{意大利语}{Italian}的儿童也会省略主语(实际上德国儿童也是如此)。省略宾语的情况比省略主语的情况更常见。对于这种现象有两种可能的解释:一是基于语言能力\isce{语言能力}{competence}的,一是基于语言运用\isce{语言运用}{performance}的。在基于语言能力的方法中,假设儿童使用一种语法,允许他们省略主语并且后来才习得正确的语法(通过设置参数或者增加规则装置)。相反,在基于语言运用的方法中,主语的省略可被归因于这样的事实,由于大脑容量有限,儿童还不能规划和产出长句。因为在话段之初,认知需求是最大的,这就导致主语越来越多地被省略。  \citet{Valian91a}调查了这些不同的假设,并证明学习英语和学习意大利语的孩子省略主语的概率是不同的。主语比宾语更经常被省略。她因此得出结论:基于语言能力的解释是不充分的。主语的省略更多地应该看做是一种语言运用现象(也可以参见 \citealp{Bloom93a})。支持语言运用因素的影响的另一个证据是:主语的冠词比宾语的冠词更经常被省略(31\% vs.\ 18\%,参见 \citealp[\page  440]{Gerken91a})。正如 Bloom指出的,至今还没有提出主语冠词脱落参数\iscesub{参数}{parameter}{主语冠词脱落}{subject article drop}。如果我们将这种现象解释为一种语言运用现象,那么也可以假设完整主语的省略也是由语言运用因素造成的。\ilce[|)]{意大利语}{Italian}\ilce[|)]{英语}{English}
%The pro-drop parameter is one of the most widely discussed parameters in the context of Principles \& Parameters theory and it will
%therefore be discussed in more detail here. It is assumed that speakers of English have to learn that all sentences in English require
%a subject, whereas speakers of Italian learn that subjects can be omitted.
%One can observe that children learning both English\il{English} and Italian\il{Italian} omit subjects (German children too in fact).
%Objects are also omitted notably more often than subjects.
%There are two possible explanations for this: a competence-based\is{competence} one and a performance-based\is{performance} one.
%In competence-based approaches, it is assumed that children use a grammar that allows them to omit subjects and then only later acquire
%the correct grammar (by setting parameters or increasing the rule apparatus). In performance-based approaches, by contrast, the omission of subjects
%is traced back to the fact that children are not yet capable of planning and producing long utterances due to their limited brain capacity.
%Since the cognitive demands are greatest at the beginning of an utterance, this leads to subjects beings increasingly left out.
%  \citet{Valian91a} investigated these various hypotheses and showed that the frequency with which children learning English and Italian respectively omit subjects
% is not the same. Subjects are omitted more often than objects. She therefore concludes that competence-based explanations are
% not empirically adequate.\todostefan{hier fehlt irgendwie was bei den Details} The omission of
% subjects should then be viewed more as a performance phenomenon (see also \citealp{Bloom93a}). 
% Another argument for the influence of performance factors is the fact that articles of subjects are
% left out more often than articles of objects (31\% vs.\ 18\%, see \citealp[\page
%   440]{Gerken91a}). As Bloom notes, no subject article-drop parameter has been proposed so
% far\is{parameter!subject article drop}. If we explain this phenomenon as a performance phenomenon, then it is also plausible to assume that
%the omittance of complete subjects is due to performance issues.\il{Italian|)}\il{English|)}

\citet{Gerken91a}展示了话段的节律属性也起到一定作用:在一个儿童必须重复句子的实验中,这些儿童省略主语/冠词比省略宾语/冠词更为常见。这里,语调模式是抑扬的\isce{抑扬}{iambus}(弱-强)或者扬抑的\isce{扬抑}{trochee}(强-弱)。甚至可以发现以轻音节开头的单词比轻音节在词尾的单词更容易被省略。因此,比较起来,giRAFFE更容易省略为RAFFE,而MONkey不容易省略为MON。 Gerken为话段假设了以下节律\isce{节律}{metrics}结构:
% \citet{Gerken91a} shows that the metrical properties of utterances also play a role: in experiments where children had to repeat sentences,
%they omitted the subject/article of the subject more often  than the object/article of the object. Here, it made a difference whether the intonation pattern
%was iambic\is{iambus} (weak-strong) or trochaic\is{trochee} (strong-weak). It can even be observed with individual words that children leave out
%weak syllables at the beginning of words more often than at the end of the word. Thus, it is more probable that ``giRAFFE'' is reduced to ``RAFFE'' than
%``MONkey'' to ``MON''. Gerken assumes the following for the metrical\is{metrics} structure of utterances:
\begin{enumerate}
\item 一个音步且只包含一个重读音节。
%\item A metrical foot contains one and only one strong syllable.
\item 最大限度地构造出二元的左--到--右音步。
%\item Create maximally binary left-to-right feet.
\item 韵律结构独立于句法结构。
%\item Metrical structure is independent of syntactic structure.
\end{enumerate}
在英语中,主语代词在句首与重读的动词组成了抑扬格,如 (\mex{1}a)所示。但是,宾语代词充当扬抑格中的弱音节,如(\mex{1}b)所示。
%Subject pronouns in English are sentence-initial and form a iambic foot with the following strongly emphasized verb as in (\mex{1}a).
%Object pronouns, however, can form the weak syllable of a trochaic foot as in (\mex{1}b). 
\eal
\ex 
\gll she KISSED $+$ the DOG\\  
     她 亲吻 {} \defart{} 狗\\
\mytrans{她亲吻了这只狗。}  
%\ex she KISSED $+$ the DOG
\ex 
\gll the DOG $+$ KISSED her\\  
    \defart{} 狗 {} 亲吻 她\\
\mytrans{这只狗亲吻了她。}  
%\ex the DOG $+$ KISSED her
\ex 
\gll PETE $+$ KISSED the $+$ DOG\\  
    Pete {} 亲吻 \defart{} {} 狗\\
\mytrans{Pete亲吻了这只狗。}  
%\ex PETE $+$ KISSED the $+$ DOG
\zl
另外,(\mex{0}a)中处在扬抑音步中的冠词和(\mex{0}b)中的主语比(\mex{0}c)中处在抑扬音步中的宾语更加容易省略。
%Furthermore, articles in iambic feet as in the object of (\mex{0}a) and the subject of (\mex{0}b) are omitted more often
%than in trochaic feet such as with the object of (\mex{0}c).

由此看出,有多重因素影响着成分的省略并且不能仅仅以儿童的行为作为在两种语法之间进行转换的证据。
%It follows from this that there are multiple factors that influence the omission of elements and that one cannot simply take the behavior
%of children as evidence for switching between two grammars.

除了上述讨论之外,pro-脱落参数之所以有趣还有另外的原因:当涉及设置该参数时有一个问题。标准的解释是学习者认识到在所有英语句子中都要有主语,这一点可以从输入中会出现虚位代词\iscesub{代词}{pronoun}{虚位}{expletive}看出。
%Apart from what has been discussed so far, the pro-drop parameter is of interest for another reason: there is a problem when it comes to setting parameters. The standard explanation is
%that learners identify that a subject must occur in all English sentences, which is suggested by the appearance of expletive pronouns\is{pronoun!expletive} in 
%the input.

正如第\pageref{fn-Expletiva-Pro-Drop}页所述,pro-脱落属性与语言中是否出现虚位代词没有关系。因为pro-脱落属性与其他任意假设的属性也没有关系,只有输入中有缺少主语的句子才能证明应该设置一个参数。问题是还有没有可见主语的语法表达。这些例子主要有祈使句,见例(\mex{1});省略主语的陈述句,见例(\mex{2}a);甚至是没有虚位的陈述句,见例(\mex{2}b),该例句是 \citet[\page 32]{Valian91a}在《纽约时报》中找到的。
%As discussed on page~\pageref{fn-Expletiva-Pro-Drop}, there is no relation between the pro-drop property and the presence of expletives
%in a language. Since the pro-drop property does not correlate with any of the other putative properties either, only the existence
%of subject-less sentences in the input constitutes decisive evidence for setting a parameter. The problem is that there are grammatical utterances
%where there is no visible subject. Examples of this are imperatives such as (\mex{1}), declaratives
%with a dropped subject as in (\mex{2}a) and even declarative sentences without an expletive such as the  
%example in (\mex{2}b) found by  \citet[\page 32]{Valian91a} in the New York Times.
\eal
\label{Beispiel-Imperativ-Englisch}
\ex 
\gll Give me the teddy bear!\\  
    给 我 \defart{} 泰迪 熊\\
\mytrans{给我这只泰迪熊!}  
%\ex Give me the teddy bear!
\ex 
\gll Show me your toy!\\  
    展示 我 你的 玩具\\
\mytrans{给我看看你的玩具!}  
%\ex Show me your toy!
\zl
\eal
\ex\label{ex-sings-like-a-dream} 
\gll She'll be a big hit. Sings like a dream.\\
    她.将 \textsc{cop} 一 大 击打 唱 像 一 梦\\
\mytrans{她将会大受欢迎的。歌声像梦一样。}  
%\ex She'll be a big hit. Sings like a dream.\label{ex-sings-like-a-dream}
\ex 
\gll Seems like she always has something twin-related perking.\\  
    好像 像 她 总是 有 一些东西 双-相关  振作\\
\mytrans{好像她总是有些让人振奋的相关东西。}  
%\ex Seems like she always has something twin-related perking.
%TODO: not sure about the meaning.
\zl
下面引述的涅槃乐队\isce{涅槃乐队}{Nirvana}的歌曲题目也来自于Valian的文章:
%The following title of a Nirvana\is{Nirvana} song also comes from the same year as Valian's article:
\ea
\gll Smells like Teen Spirit.\\  
    闻 像 Teen Spirit\\
\mytrans{闻起来像Teen Spirit。}  
%Smells like Teen Spirit.\\
\z
Teen Spirit指的是一种除臭剂而smell是一个动词,在德语和英语中都需要一个指称性主语,但是也可以是一个虚位it作主语。但是,例(\mex{0})没有主语,Kurt Cobain心里想的用法无法被重建\footnote{%
 参见 \url{http://de.wikipedia.org/wiki/Smells_Like_Teen_Spirit},\zhdate{2016/03/06}。
},。祈使句确实会出现在儿童的输入中,所以对习得是有作用的。 \citet[\page 33]{Valian91a}这样说:
%Teen Spirit refers to a deodorant and \emph{smell} is a verb that, both in German and English, requires a referential subject but can also be used with an expletive \emph{it} as subject.
%The usage that Kurt Cobain had in mind cannot be reconstructed\footnote{%
%  See \url{http://de.wikipedia.org/wiki/Smells_Like_Teen_Spirit}. 06.03.2016.
%}, independent of the intended meaning, however, the subject in (\mex{0}) is missing.
%Imperatives do occur in the input children have and are therefore relevant for acquisition.
% \citet[\page 33]{Valian91a} says the following about them:
\begin{quotation}
成人语言社团中可以接受的表达构成了儿童的语言输入,它们也是儿童必须掌握的部分。这里所说的“可以接受”并不合乎英语语法(因为英语没有小代语主语,也无法被描述为一个简单VP)。这些句子缺少主语,因此违背了我们所假设的扩展的投射原则\isce{扩展的投射原则(EPP)}{Extended Projection Principle (EPP)}\citep{Chomsky81a}。

儿童也可以接触到完全合法的不带主语的表达,即祈使句。他们也可以接触到可以接受但不完全合乎语法的表达,例如 [(\ref{ex-sings-like-a-dream})]以及类似于Want lunch now?的表达。美国儿童在长大成人之后一定要知道显性的主语在语法上是必需的,也要知道主语什么时候可以被省略。儿童不仅必须掌握正确的语法,还要掌握使得语法可以变通的语篇条件。 \citep[\page 33]{Valian91a}\bracketfootnotequote{%
What is acceptable in the adult community forms part of the child's input, and
is also part of what children must master. The utterances that I have termed
``acceptable'' are not grammatical in English (since English does not have pro
subjects, and also cannot be characterized as a simple VP). They lack subjects
and therefore violate the extended projection principle \citep{Chomsky81a}, which we are assuming.

   Children are exposed to fully grammatical utterances without subjects, in the
form of imperatives. They are also exposed to acceptable utterances which are
not fully grammatical, such as [(\ref{ex-sings-like-a-dream})], as well as forms like, ``Want lunch now?'' The
American child must grow into an adult who not only knows that overt subjects
are grammatically required, but also knows when subjects can acceptably be
omitted. The child must not only acquire the correct grammar, but also master
the discourse conditions that allow relaxation of the grammar.}
\end{quotation}
这一段将语法理论与语法现象之间的关系搞反了:如果一个特定的语法理论不能涵盖一些语言现象,我们不能得出结论:这些语言现象不应该用该理论来描写。而是应该修改这个不兼容的语法,或者如果无法修改的话,就应该放弃该理论。因为祈使句是完全规则的,所以没有理由将其认定为不遵守语法规则的表达。上述引文要求学习者习得两种语法:一种语法对应天赋语法,另一种语法部分地压缩了天赋语法的规则也增加了一些额外规则。
%This passage turns the relations on their head: we cannot conclude from the fact that a particular grammatical
%theory is not compatible with certain data, 
%that these data should not be described by this theory, instead we should modify the incompatible
%grammar or, if this is not possible, we should reject it.
%Since utterances with imperatives are entirely regular, there is no reason to categorize them as utterances that
%do not follow grammatical rules. The quotation above represents a situation where a learner has to acquire two
%grammars: one that corresponds to the innate grammar and a second that partially suppresses the rules of innate grammar
%and also adds some additional rules.

我们会针对该观点提出以下问题:儿童怎样区分他所听到的句子对应于两种语法中的哪一种呢?\iscesub[|)]{参数}{parameter}{pro-脱落}{pro-drop}
%The question we can pose at this point is: how does a child distinguish which of the data it hears are relevant for which of the two grammars?
%\is{parameter!pro-drop|)}

 \citet[\page 347]{Fodor98a}追求另外一种不同的解释,这种解释不会遇到上述问题。该分析不是假设一个学习者从很多语法中寻找对的那一种语法,而是假设儿童使用包含所有可能性的单一的语法。她提出使用部分树(小树)而不是参数。这些小树也可以不充分赋值,而且在极端情况下一个小树可以包含单一特征\citep[\page 6]{Fodor98b}。一个语言学习者可以从一个特定小树的使用来推断一种语言是否有一种给定属性。她举出了一个VP小树的例子,该VP包含一个动词和一个介词短语。这一小树必须用于分析出现在Look at the frog中的VP。相似地,在分析带有前置who的疑问句时,也必须用到一个小树,该小树需要在补足语中有一个wh-NP充当限定语。(见第\pageref{Abb-GB-Wh}页的图\ref{Abb-GB-Wh})。在Fodor版的\pptc 中,该小树将是允准(显性的)语法中wh-移位的参数。Fodor假设,即便是没有设定或只设定了极少的参数,也有缺省值\iscesub{参数}{parameter}{缺省值}{default value}允许学习者分析一个句子。这反之也让学习者能从他无法使用的表达中学习语法,因为这些表达有很多种分析的可能性。假设一个缺省值可能会导致错误分析,但是:因为一个缺省值,就可以设定第二个参数,因为一个表达就可以使用t$_1$和t$_3$进行分析。但是t$_1$不适用于正在研究的某一特定语言,那么话段就要用非缺省小树t$_2$和小树t$_{17}$进行分析。因此,在这一习得模型中,必须要有可以纠正参数设定过程中的错误决定的可能性。Fodor因此假设参数有一个基于频率的激活度(第365页):在分析中经常使用的有高的激活度,而不经常使用的则激活度较低。以这种方式,在排除其他参数时就不需要假设一个特定参数了。
% \citet[\page 347]{Fodor98a} pursues a different analysis that does not suffer from many of the aforementioned problems.
% Rather than assuming that learners try to find a correct grammar among a billion others, she instead assumes that
% children work with a single grammar that contains all possibilities.
%She suggests using parts of trees (\emph{treelets}) rather than parameters. These treelets can also be underspecified and
%in extreme cases, a treelet can consist of a single feature \citep[\page 6]{Fodor98b}.
%A language learner can deduce whether a language has a given property from the usage of a particular treelet.
%As an example, she provides a VP treelet consisting of a verb and a prepositional phrase. 
%This treelet must be used for the analysis of the VP occurring in \emph{Look at the
%  frog}. Similarly, the analysis of an interrogative clause with a fronted \emph{who} would make use
%of a treelet with a \emph{wh}-NP in the specifier of a complementizer phrase (see Figure~\ref{Abb-GB-Wh} on page~\pageref{Abb-GB-Wh}).
%In Fodor's version of \ppt, this treelet would be the parameter that licenses \emph{wh}-movement in (overt) syntax.
%Fodor assumes that there are defaults\is{parameter!default value} that allow a learner to parse a sentence even when no or very few
%parameters have been set. This allows one to learn from utterances that one would have not otherwise been able to use since there would have
%been multiple possible analyses for them. Assuming a default can lead to misanalyses, however: due to a default value, a second
%parameter could be set because an utterance was analyzed with a treelet t$_1$ and t$_3$, for example, but t$_1$ was not suited to the particular
%language in question and the utterance should have instead been analyzed with the non-default treelet t$_2$ and the treelet t$_{17}$.
%In this acquisition model, there must therefore be the possibility to correct wrong decisions in the parameter setting process. 
%Fodor therefore assumes that there is a frequency-based degree of activation for parameters (p.\,365): treelets that are often
%used in analyses have a high degree of activation, whereas those used less often have a lower degree of activation.
%In this way, it is not necessary to assume a particular parameter value while excluding others.

另外,Fodor提出参数应该按照层级进行组织,也就是说只有当一个参数有一个特定取值时,去思考指定的其他参数值才会有意义。
%Furthermore, Fodor proposes that parameters should be structured hierarchically, that is, only if a parameter has a particular value
%does it then make sense to think about specific other parameter values.

Fodor的分析⸺正如其自己所说的那样\citep[\page 385]{Fodor2001a}⸺与HPSG\indexhpsg 和TAG\indextag 等理论兼容。 \citet[\page 147]{ps}将UG描述为所有普遍可用原则的交集。
%Fodor's analysis is -- as she herself notes \citep[\page 385]{Fodor2001a} -- compatible with theories such as HPSG\indexhpsg and TAG\indextag.
% \citet[\page 147]{ps} characterize UG as the conjunction of all universally applicable principles:
\ea
UG = P$_1$ $\wedge$ P$_2$ $\wedge$ \ldots{} $\wedge$ P$_n$
\z
正如原则可以适用于所有的语言,也存在其他原则适用于一种特定语言或一类语言。Pollard \& Sag给出了只适用于英语的成分排序原则。如果假设P$_{n + 1}$--P$_m$是针对特定语言的原则,L$_{1}$--L$_p$是词项的完整列表,R$_{1}$--R$_q$是支配程式(dominance schemata)的完整列表,那么英语\ilce{英语}{English}就可以描述如下:
%As well as principles that hold universally, there are other principles that are specific to a particular language
%or a class of languages. Pollard \& Sag give the example of the constituent ordering principle that only holds for English.
%English\il{English} can be characterized as follows if one assumes that P$_{n + 1}$--P$_m$ are language-specific principles,  
%L$_{1}$--L$_p$ a complete list of lexical entries and R$_{1}$--R$_q$ a list of dominance schemata relevant for English.
\ea
English = P$_1$ $\wedge$ P$_2$ $\wedge$ \ldots{} $\wedge$ P$_m$ $\wedge$ (L$_{1}$ $\vee$ \ldots{}
$\vee$ L$_p$ $\vee$ R$_{1}$ $\vee$ \ldots{} $\vee$  R$_q$)
\z
在Pollard \& Sag的概念体系中,只有适用于所有语言的语言属性才是UG的一部分。 Pollard \& Sag并不把支配规则当做UG的一部分。但是,也可以将UG描述如下:
%In Pollard \& Sag's conception, only those properties of language that equally hold for all languages are
%part of UG. Pollard \& Sag do not count the dominance schemata as part of this. However, one can indeed also describe
%UG as follows:
\ea
%UG = P$_1$ $\wedge$ P$_2$ $\wedge$ \ldots{} $\wedge$ P$_n$ $\wedge$ (R$_{en\mbox{-}1}$ $\vee$ \ldots{} $\vee$
%R$_{en\mbox{-}q}$  $\vee$ R$_{de\mbox{-}1}$ $\vee$ \ldots{} $\vee$ R$_{de\mbox{-}r}$ $\vee$ \ldots )
$
\mathrm{UG} = 
  \mathrm{P}_1 
  \wedge 
  \mathrm{P}_2 
  \wedge 
  \ldots 
  \wedge 
  \mathrm{P}_n 
  \wedge
              (\mathrm{R}_{\mathrm{en}\mathdash1} 
              \vee 
              \ldots 
              \vee 
              \mathrm{R}_{\mathrm{en}\mathdash q} 
              \vee  
              \mathrm{R}_{\mathrm{de}\mathdash1} 
              \vee 
              \ldots 
              \vee 
              \mathrm{R}_{\mathrm{de}\mathdash r}
	      \vee 
	      \ldots)
$
\z
和前文所述一样,P$_1$--P$_n$是普遍适用的原则,$\mathrm{R}_{\mathrm{en}\mathdash1}$--$\mathrm{R}_{\mathrm{en}\mathdash q}$是英语的(核心)支配规则,$\mathrm{R}_{\mathrm{de}\mathdash1}$--$\mathrm{R}_{\mathrm{de}\mathdash r}$是德语的支配规则。(\mex{0})中的支配规则是通过析取方式组合起来的,也就是说,并非所有项都要在具体语言中实现。原则可以适用于词项的特定属性而排除特定短语实现。如果一种语言只包括能出现在句末的中心语,那么需要一个出现在句首中心语的句法规则就永远无法与这些中心语及其投射组合。另外,有一个类型系统的理论也与Fodor的语言习得方法兼容,因为约束易于被不完全赋值。因此,UG中的限制不必约束语法规则中的所有属性:原则可以指向特征取值,针对特定语言的取值自身并不一定得是UG中现有的。相似地,如果一个上位类型描述具有相似但具有特定语言实现的支配原则,该上位类型也可以包括在UG中,但是针对特定语言的细节仍然是开放的,并且学习者一分析就可以推断出来(见\citealp[第9.2节]{AW98a})。Fodor假设的激活方面的差异可以通过为限制加权来获得: $\mathrm{R}_{\mathrm{en}\mathdash1}$--$\mathrm{R}_{\mathrm{en}\mathdash q}$等支配规则是特征取值的集合也是路径等式的集合。正如第\ref{Abschnitt-Diskussion-Performanz}章所解释的,既可以给这些限制加权也可以给限制集合加权。在Fodor的习得模型中,给定一个德语输入,英语规则的权重就会减少,德语规则的权重就会上升。不像Fodor的模型,Pollard \& Sag的习得模型并没有参数设定的激发词\isce{激发词}{trigger}。另外,以前析取作为UG一部分的属性现在可以直接习得。使用小树t$_{17}$(或者一个可能不充分赋值的支配规则),并不是用取值“+”给参数P$_5$进行设定,而使t$_{17}$的激活潜力提升了,所以t$_{17}$在未来的分析中就会优先使用。
\isce[|)]{参数}{parameter}\isce[|)]{原则 \& 参数}{Principles \& Parameters}
%P$_1$--P$_n$ are, as before, universally applicable principles and
%$\mathrm{R}_{\mathrm{en}\mathdash1}$--$\mathrm{R}_{\mathrm{en}\mathdash q}$ are the
%(core) dominance schemata of English and $\mathrm{R}_{\mathrm{de}\mathdash1}$--$\mathrm{R}_{\mathrm{de}\mathdash r}$ are the dominance schemata in
%German. The dominance schemata in (\mex{0}) are combined by means of disjunctions, that is, not every disjunct needs to have a realization in a specific language. Principles can make reference
%to particular properties of lexical entries and rule out certain phrasal configurations.
%If a language only contains heads that are marked for final-position in the lexicon, then grammatical rules that
%require a head in initial position as their daughter can never be combined with these heads or their projections.
%Furthermore, theories with a type system are compatible with Fodor's approach to language acquisition because 
%constraints can easily be underspecified. As such, constraints in UG do not have to make reference to all properties
%of grammatical rules: principles can refer to feature values, the language-specific values themselves do not have to
%already be contained in UG. Similarly, a supertype describing multiple dominance schemata that have
%similar but language-specific instantiations can also be part of UG, however the language-specific details remain open and are then deduced by the learner
%upon parsing (see \citealp[Section~9.2]{AW98a}). The differences in activation assumed by Fodor can be captured
%by weighting the constraints: the dominance schemata $\mathrm{R}_{\mathrm{en}\mathdash1}$--$\mathrm{R}_{\mathrm{en}\mathdash q}$ etc.\ are sets of
%feature-value pairs as well as path equations. As explained in Chapter~\ref{Abschnitt-Diskussion-Performanz}, 
%weights can be added to such constraints and also to sets of constraints. In Fodor's acquisition model, given a German input, the weights for the rules
%of English would be reduced and those for the German rules would be increased. Note that in Pollard
%\& Sag's acquisition scenario, there are no triggers\is{trigger} for parameter
%setting unlike in Fodor's model.
%Furthermore, properties that were previously disjunctively specified as part of UG will now be
%acquired directly. Using the treelet t$_{17}$ (or rather a possibly underspecified dominance
%schema), it is not the case that the value `+' is set for a parameter P$_5$  but rather the activation
%potential of t$_{17}$ is increased such that t$_{17}$ will be prioritized for future
%analyses.
%\is{parameter|)}\is{Principles \& Parameters|)}

\section{原则和词库}
%\section{Principles and the lexicon}

UG理论驱动的语言习得理论会假设,原则是普遍的适用于所有语言的,并且单个语言只是在词库上存在差异。原则是指组合实体的属性。参数因此从原则转移到词库\citep[\page 2]{Chomsky99a}。参见 \citet{MR2010a}以这一模型对罗曼语系语言进行研究以及前面章节讨论的 \citet[\page 395]{SonS2008a}对Snyder例子的分析。
%A variant of the UG-driven theory of language acquisition would be to assume that principles are so general that they hold
%for all languages and individual languages simply differ with regard to their lexicon.
%Principles then refer to properties of combined entities. Parameters therefore migrate from principles into the lexicon
%\citep[\page 2]{Chomsky99a}. See  \citet{MR2010a} for a study of Romance languages in this model and
% \citet[\page 395]{SonS2008a} for an analysis of Snyder's examples that were discussed in the
%previous subsection.

在这一点上,可以发现很多方法都有相似之处:这里讨论的大多数理论都为中心语与其论元组合假设了一个非常通用的结构。例如,在范畴语法和最简方案中,总是有二元的函数-变元组合。在一种特定语言中,成分的序列取决于构成元素的词汇属性。
%At this point, one can observe an interesting convergence in these approaches: most of the theories discussed here assume a very
%general structure for the combination of heads with their arguments. For example, in Categorial Grammar and the Minimalist Program,
%these are always binary functor-argument combinations. The way in which constituents can be ordered in a particular language depends
%on the lexical properties of the combined elements.

现在激烈争论的问题是词汇属性束是否由UG决定\citep[\page 6--7]{Chomsky2007a},以及是否语言的所有方面都可以用同样的组合可能性来描述(见\ref{Abschnitt-Phrasale-Konstruktionen}对于短语构式的论述)。
%The question that is being discussed controversially at present is whether the spectrum of lexical properties is determined by UG 
%\citep[\page 6--7]{Chomsky2007a} and whether all areas of the language can be described with the same general combinatorial possibilities (see Section~\ref{Abschnitt-Phrasale-Konstruktionen} on phrasal constructions).

在\ref{Abschnitt-PP}中,我已经展示了假设天赋语言特定知识的习得理论是什么样的,以及这类习得理论的变体跟我们已经讨论的所有理论兼容。在讨论时,应该记住这一问题:假设英国儿童在其习得英语过程中会使用一部分汉语语法这一说法是否有意义(正如\citealp[\page 453]{Yang2004a}所提出的那样),或者相关现象是否可以用别的方式解释。下面我将展示一些其他方法,这些方法没有预设天赋语言的特定知识,但是假设语言可以仅从输入中习得。下面的章节将会处理基于模型的方法,\ref{Abschnitt-Selektionsbasierter-Spracherwerb}会讨论基于输入语言习得的以词汇为导向的变体。
%In Section~\ref{Abschnitt-PP}, I have shown what theories of acquisition assuming innate language
%specific knowledge can look like and also that variants of such acquisition theories are compatible
%with all the theories of grammar we have discussed.  During this discussion, one should bear in
% mind the question of whether it makes sense at all to assume that English children
%use parts of a Chinese grammar during some stages of their acquisition process (as suggested by
%\citealp[\page 453]{Yang2004a}), or whether the relevant phenomena can be explained in different ways.
%In the following, I will present some alternative approaches that do not presuppose innate language
%specific knowledge, but instead assume that language can simply be acquired from the input. The
%following section will deal with pattern-based approaches and
%Section~\ref{Abschnitt-Selektionsbasierter-Spracherwerb} will discuss the lexically-oriented variant
%of input-based language acquisition.

\section{基于模式的方法}
%\section{Pattern-based approaches}
\label{Abschnitt-musterbasiert}

\mbox{}\citet[\page 7--8]{Chomsky81a}提出语言可以分为核心\isce{核心语法}{core grammar}和边缘\isce{边缘现象}{periphery}两种。核心现象包含语言所有规则的方面。一种语言的核心语法可以看作是UG的实现。熟语\isce{习语}{idiom}和语言中其他不规则的部分是边缘现象。原则 \& 参数理论模型的批评者指出,熟语性以及不规则结构构成了我们语言中相当大的一部分,而且核心与边缘的区分是可变并且任意的,是一种理论内部的划分(\citealp[第7章]{Jackendoff97a};\citealp{Culicover99a-u};\citealp[\page 5]{GSag2000a-u};\citealp[\page 48]{Newmeyer2005a};\citealp[\page 619]{Kuhn2007a})。例如,可以发现很多熟语与句法之间存在互动\citep*{NSW94a}。德语中包含动词成分的熟语大部分都允许动词移动到句首位置,见例 (\mex{1}b);有些熟语允许熟语的一部分前置,见例 (\mex{1}c),有些熟语可被动化,见(\mex{1}d)。
%\citet[\page 7--8]{Chomsky81a} proposed that languages can be divided into a core
%area\is{core grammar} and a periphery\is{periphery}. The core contains all regular aspects of
%language. The core grammar of a language is seen as an instantiation of UG. Idioms\is{idiom} and
%other irregular parts of language are then part of the periphery.  Critics of the Principles \&
%Parameters model have pointed out that idiomatic and irregular constructions constitute a relatively
%large part of our language and that the distinction, both fluid and somewhat arbitrary, is only
%motivated theory-internally (\citealp[Chapter~7]{Jackendoff97a}; \citealp{Culicover99a-u};
%\citealp[\page 5]{GSag2000a-u}; \citealp[\page 48]{Newmeyer2005a}; \citealp[\page 619]{Kuhn2007a}).
%For example, it is possible to note that there are interactions between various idioms and syntax
%\citep*{NSW94a}.  Most idioms in German with a verbal component allow the verb to be moved to
%initial position (\mex{1}b), some allow that parts of idioms can be fronted (\mex{1}c) and some can
%undergo passivization (\mex{1}d).

\eal
\ex 
\gll dass er ihm den Garaus macht\\
     \textsc{comp} 他 他 \defart{} \textsc{garaus} 做\\
\mytrans{他结束了他(杀了他)}
%\gll dass er ihm den Garaus macht\\
%	 that he him the \textsc{garaus} makes\\
%\mytrans{that he finishes him off (kills him)}
\ex 
\gll Er macht ihm den Garaus.\\
	 他 做 他 \defart{} \textsc{garaus}\\
\mytrans{他结束了他。}
%\gll Er macht ihm den Garaus.\\
%	 he makes him the \textsc{garaus}\\
%\mytrans{He finishes him off.}
\ex
%In Amerika sagte man der Kamera nach, die größte Kleinbildkamera der Welt zu sein. Sie war laut
%Schleiffer am Ende der Sargnagel der Mühlheimer Kameraproduktion.\\
\gll 
\emph{Den} \emph{Garaus} \emph{machte} ihr die Diskussion um die Standardisierung des 16-Millimeter-Filmformats,
an dessen Ende die DIN-Norm 19022 (Patrone mit Spule für 16-Millimeter-Film) stand, die im März 1963
zur Norm wurde.\footnotemark\\
\defart{} \textsc{garaus} 做 她
\defart{} 讨论 \textsc{prep} \defart{} 标准化 \defart{} 16-毫米-胶片.版式 \textsc{prep} 谁的 结束 \defart{} DIN-标准 19022
\spacebr{}墨盒 \textsc{prep} 盘 \textsc{prep} 16-毫米-胶片 站 \textsc{rel} \textsc{prep}.\defart{} 三月 1963 \textsc{prep}.\defart{} 规范
变成\\
\footnotetext{%
《法兰克福评论报》(\emph{Frankfurter Rundschau}),\zhdate{1997/06/28},第2页。 %, Ressort: LOKAL-RUNDSCHAU; Fotoapparatesammler Karl-Christian Schelzke regt Ausstellungen an
}
\mytrans{
%在美国,有人说这种相机是世界上最大的袖珍相机。按照Schleiffer的说法,这种相机是导致Mühlheim生产相机的最终决定因素。
结束它生产的是有关标准的16毫米版式的讨论,这催生了DIN标准19022(适用于16毫米胶片的带油的墨盒),它在1963年3月成为规范。}
%\gll 
%\emph{Den} \emph{Garaus} \emph{machte} ihr die Diskussion um die Standardisierung des 16-Millimeter-Filmformats,
%an dessen Ende die DIN-Norm 19022 (Patrone mit Spule für 16-Millimeter-Film) stand, die im März 1963
%zur Norm wurde.\footnotemark\\
%the \textsc{garaus} made her
%the discussion around the standardization of.the 16-millimeter-film.format at whose end the DIN-norm 19022
%\spacebr{}cartridge with coil for 16-millimeter-film stood that in March 1963 to.the norm
%became\\
%\footnotetext{%
%Frankfurter Rundschau, 28.06.1997, p.\,2. %, Ressort: LOKAL-RUNDSCHAU; Fotoapparatesammler Karl-Christian Schelzke regt Ausstellungen an
%}
%\glt `In America, one says that this camera was the biggest compact camera in the world. According to Schleiffer, it was the
%last nail in the coffin for camera production in Mühlheim. What finished it off was the discussion about standardizing
%the 16 millimeter format, which resulted in the DIN-Norm 19022 (cartridge with coil for 16 millimeter film) that became
%the norm in March 1963.'
\ex
\gll in Heidelberg wird "`parasitären Elementen"' unter den Professoren \emph{der} \emph{Garaus} \emph{gemacht}\footnotemark\\
	 \textsc{prep} 海德堡 \passiveprs{} \hspaceThis{"`}寄生的 成分 \textsc{prep} \defart{} 教授 \defart{} \textsc{garaus} 做\\
\footnotetext{%
《曼海姆晨报》(\emph{Mannheimer Morgen}), \zhdate{1999/06/28}。
%运动;光凭螺丝钉是不够的。%%M99/906.41526 Mannheimer
}
\mytrans{在海德堡,教授中的‘寄生虫们’正在被一个接一个地杀死。}
%\gll in Heidelberg wird "`parasitären Elementen"' unter den Professoren \emph{der} \emph{Garaus} \emph{gemacht}\footnotemark\\
%	 in Heidelberg are \hspaceThis{"`}parasitic elements among the professors the \textsc{garaus} made\\
%\footnotetext{%
%Mannheimer Morgen, 28.06.1999, Sport; Schrauben allein genügen nicht.%%M99/906.41526 Mannheimer
%}
%\mytrans{In Heidelberg, ``parasitic elements'' among professors are being killed off.}
\zl
\noindent
有人假设边缘成分和词库不是UG的组成部分(\citealp[\page 150--151]{Chomsky86a};\citealp[\page 343]{Fodor98a}),而是利用其他学习方法习得的⸺换句话说直接从输入归纳而来。批评者提出的问题是为什么这些方法对语言中的规则方面没起作用(\citealp[\page 20]{Abney96a};\citealp[\page 222]{Goldberg2003b};\citealp[\page 100]{Newmeyer2005a};Tomasello\citeyear[\page 36]{Tomasello2006a};\citeyear[\page 20]{Tomasello2006c})。所谓的核心部分比边缘成分更加规则,所以应该更容易学习。
%It is assumed that the periphery and lexicon are not components of UG (\citealp[\page 150--151]{Chomsky86a}; \citealp[\page 343]{Fodor98a})
%but rather are acquired using other learning methods -- namely inductively directly from the input. The question posed by critics is now why these methods should not work for regular aspects of the language as well (\citealp[\page 20]{Abney96a}; 
% steht da nicht \citealp[\page 9]{Culicover99a-u};\note{check, zitiert nach Newmeyer2005} 
%\citealp[\page 222]{Goldberg2003b}; \citealp[\page
%100]{Newmeyer2005a}; Tomasello \citeyear[\page 36]{Tomasello2006a}; \citeyear[\page
%20]{Tomasello2006c}): the areas of the so-called `core' are by definition more regular then components of the periphery, which is why
%they should be easier to learn.

 \citet{Tomasello2000a,Tomasello2003a}曾指出,基于原则 \& 参数理论的语言习得模型与观察到的现象不相符。\pptc 预测,一旦儿童正确设定了某一特定参数,那么他们就不会在语法的特定领域犯错了(见\citealp[\page 146]{Chomsky86a},\citealp[\page 21--22]{Radford90a-u}和\citealp[\page175]{Lightfoot97a})。另外,还假设一个参数会对语法非常不同的部分起作用(见\ref{Abschnitt-PP}对于pro-脱落参数的论述)。当一个参数值设定完成之后,大量现象都会有突然的提升\citep[\page 174]{Lightfoot97a}。但是,实际上不是这样。相反,儿童从输入的话段中学习语言并且从一个特定的年纪开始概括。依赖于输入,他们可以重新对一些助动词而不对其他的助动词进行排序,虽然在英语\ilce{英语}{English}中助动词移位\isce{助动词倒装}{auxiliary inversion}是强制的。\dotfootnote{%
这里,Yang建议将语法与特定概率结合起来对于解决问题并没有帮助,因为必须假设儿童为不同的助动词采用了不同的语法,这是极不可能的。
} 一个反对这类基于输入的理论的论述是儿童可以产生很多输入中并没有的话语。这类现象中被经常讨论的是所谓的主句不定式\isce{主句不定式}{root Infinitive}(RI)和非强制不定式\isce{非强制不定式}{optional infinitive}(OI)\citep{Wexler98a}。\ilce[|(]{英语}{English}存在一些不定形式而非限定动词可以出现在非嵌套句(主句)中。非强制不定式是指儿童既使用一个限定式(\mex{1}a),也使用一个非限定式(\mex{1}b)形式\citep[\page 59]{Wexler98a}:
% \citet{Tomasello2000a,Tomasello2003a}\todostefan{ \citet{Behrens2009a}} has pointed out that a Principles \& Parameters model of language acquisition
%is not compatible with the observable facts. The \ppt predicts that children should no longer make mistakes in a particular area of grammar once they have set
%a particular parameter correctly (see \citealp[\page 146]{Chomsky86a}, \citealp[\page 21--22]{Radford90a-u} and \citealp[\page
%175]{Lightfoot97a}).
%Furthermore, it is assumed that a parameter is responsible for very different areas of grammar (see the discussion of the pro-drop parameter in
%Section~\ref{Abschnitt-PP}). When a parameter value is set, then there should be sudden developments with regard to a number
%of phenomena \citep[\page 174]{Lightfoot97a}. This is, however, not the case. Instead, children acquire language from utterances in their input
%and begin to generalize from a certain age. Depending on the input, they can reorder certain auxiliaries and not others, although
%movement of auxiliaries\is{auxiliary inversion} is obligatory in English\il{English}.\footnote{%
%	Here, Yang's suggestion to combine grammars with a particular probability does not help since one
%	would have to assume that the child uses different grammars for different auxiliaries, which
%        is highly unlikely.
%}
%One argument put forward against these kinds of input-based theories is that children produce utterances that cannot be observed
%to a significant frequency in the input. One much discussed phenomenon of this kind are so called \emph{root infinitives}\is{root Infinitive} (RI) or \emph{optional
%  infinitives}\is{optional infinitive} (OI) \citep{Wexler98a}.\il{English|(} These are infinitive forms that can be used in non-embedded clauses (\emph{root sentences})
%instead of a finite verb. Optional infinitives are those where children use both a finite (\mex{1}a) and non-finite (\mex{1}b) form \citep[\page 59]{Wexler98a}:
\eal
\ex 
\gll Mary likes ice cream.\\
	 Mary 喜欢 冰 奶油\\
\mytrans{Mary喜欢冰淇淋。}
%\ex Mary likes ice cream.
\ex 
\gll Mary like ice cream.\\
	 Mary 喜欢 冰 奶油\\
\mytrans{Mary喜欢冰淇淋。}
%\ex Mary like ice cream.
\zl
 \citet*[\page 656]{WKG2001a}展示了荷兰\ilce[|(]{荷兰语}{Dutch}儿童使用双词短语时,90\,\%的情况下使用宾语--不定式的顺序,而在这些儿童的母语中包含动词的情况下使用这种顺序只有不到10\,\%的几率。如例 (\mex{1})所示的情态词位于首位的复合动词形式也是这一范式的例子,这种包含动词的输入只有30\,\%\citep*[\page 647]{WKG2001a}的几率。
% \citet*[\page 656]{WKG2001a} showed that Dutch\il{Dutch|(} children use the order object infinitive 90\,\% of the time during the two-word phase although
%these orders occur in less than 10\,\% of their mother's utterances that contained a verb.
%Compound verb forms, \eg with a modal in initial position as in (\mex{1}) that contain another instance of this pattern only occurred in 30\,\% of the input containing
%a verb \citep*[\page 647]{WKG2001a}.
\ea
\gll Willst du Brei essen?\\
     想   你 粥 吃\\
\mytrans{你想喝粥吗?}
%\gll Willst du Brei essen?\\
%     want   you porridge eat\\
%\mytrans{Do you want to eat porridge?}
\z
初看好像输入与儿童的表达之间存在矛盾。但是,这一偏差也可以通过学习中的话段-居末偏向\isce{偏见}{bias}来解释(\citealp{WKG2001a};\citealp*{FPG2006a})。很多因素对于动词末位这一特点起作用:1)婴幼儿的大脑限制。研究显示,人类(包括儿童和成人)在话段中会忘记单词,也就是说激活潜力会逐渐降低。因为儿童认知能力的限制,所以为何话段末位的成分有重要意义,这就显而易见了。2)话段的末位成分更容易被切分\isce{切分}{segmentation}。在话段的最后,对于听话者来说部分切分任务消失了:听话者首先要将音素序列切分为单个单词,然后才能理解它们,并将它们组合产生更大的句法实体。在话语最后,切分任务更加简单,因为词语界限已经由话语结尾给出了。另外,按照 \citet*[\page 637]{WKG2001a}的说法,话语末尾的单词语音更长并且有一个频率重音。这一效应在儿童语言中更常见。
%At first glance, there seems to be a discrepancy between the input and the child's utterances.
%However, this deviation could also be explained by an utterance-final bias\is{bias} in learning (\citealp{WKG2001a}; \citealp*{FPG2006a}).
%A number of factors can be made responsible for the salience of verbs at the end of an utterance:
%1) restrictions of the infant brain. It has been shown that humans (both children and adults) forget words during the course of an
%utterance, that is, the activation potential decreases. Since the cognitive capabilities of small children are restricted,
%it is clear why elements at the end of an utterance have an important status. 2) Easier segmentation\is{segmentation} at
%the end of an utterance. At the end of an utterance, part of the segmentation problem for hearers disappears:
%the hearer first has to divide a sequence of phonemes into individual words before he can understand them and
%combine them to create larger syntactic entities.
%This segmentation is easier at the end of an utterance since the word boundary is already given by the end of the utterance.
%Furthermore according to  \citet*[\page 637]{WKG2001a}, utterance-final words have an above average length and do bear a pitch
%accent. This effect occurs more often in language directed at children.

\citet*{FPAG2007a}已经为英语\ilce{英语}{English}、德语、荷兰语\ilce{荷兰语}{Dutch}和西班牙语\ilce[|(]{西班牙语}{Spanish}的语言习得建立了模型。这一电脑模型可以基于输入来重建这些语言之间的差异。初看起来,令人吃惊的是,德语和荷兰语之间,英语和西班牙语之间在不定式使用方面甚至存在差异。因为德语和荷兰语的句法十分相似(都是SOV+V2)。与之相似,英语和西班牙语都是SVO语言。但是,学习英语的儿童会犯OI错误,而学习西班牙语的儿童则几乎不会犯这种错误。
% \citet*{FPAG2007a} have modeled language acquisition for English\il{English}, German, Dutch\il{Dutch}, and Spanish\il{Spanish|(}.
%The computer model could reproduce differences between these languages based on input. At first glance, it
%is surprising that there are even differences between German and Dutch and between English and Spanish with regard to the use of infinitives as
%German and Dutch have a very similar syntax (SOV+V2). Similarly, English and Spanish are both languages with SVO order.
%Nevertheless, children learning English make OI mistakes, whereas this is hardly ever the case for children learning Spanish.

\citet*{FPAG2007a}将出错频率的差异归因于不同语言中的分布差异:他们注意到英语中75\,\%动词居末短语\footnote{%
对于英语来说,作者只是统计了主语是第三人称单数的情况,因为只有在这种情况下限定和不定的形态差异才会比较清楚。%
}都包含复合动词(有定动词$+$依存动词,例如,Can he go?),但是荷兰语中只有30\,\%的动词居末短语包含复合动词。
% \citet*{FPAG2007a} trace the differences in error frequencies back to the distributional differences in each language:
%the authors note that 75\,\% of verb-final utterances\footnote{%
%	For English, the authors only count utterances with a subject in third person singular since it is only in these cases
%	that a morphological difference between the finite and infinitive form becomes clear.%
%}
%in English consist of compound verbs (finite verb $+$ dependent verb, \eg \emph{Can he go?}), whereas this is only the case
%30\,\% of the time in Dutch.

在话段末位不定式的数量方面,德语也与荷兰语存在差异。荷兰语有一种进行形式,而标准德语中没有:
%German also differs from Dutch with regard to the number of utterance-final infinitives. Dutch has a progressive form
%that does not exist in Standard German:
\ea
\gll Wat ben je aan het doen?\\
     什么 \textsc{cop} 你 \textsc{prep} 它 做.\textsc{inf}\\
\mytrans{你在做什么?}
%\gll Wat ben je aan het doen?\\
%     what are you on it do.\textsc{inf}\\
%\mytrans{What are you doing?}
\z
另外,像zitten(去做)、lopen(去跑)和staan(去站)可以与不定式并列使用来描述发生在那一时刻的事件:
%Furthermore, verbs such as \emph{zitten} `to sit', \emph{lopen} `to run' and \emph{staan}
%`to stand' can be used in conjunction with the infinitive to describe events happening in that moment:
\ea
\gll Zit je te spelen?\\
     坐 你 \textsc{inf} 玩耍\\
\mytrans{你要坐下来玩耍吗?} 
%\gll Zit je te spelen?\\
%     sit you to play\\
%\mytrans{Are you sitting and playing?} 
\z
另外,荷兰语还有一种由ga(走)构成的将来\isce{将来式}{future}式。这些因素使得荷兰语比德语多20\,\%的话段末尾不定式。
%Furthermore, there is a future form\is{future} in Dutch that is formed with \emph{ga} `go'. These factors
%contribute to the fact that Dutch has 20\,\% more utterance-final infinitives than German.

西班牙语与英语的差异在于它有宾语附着\isce{附着}{clitic}形式:
%Spanish differs from English in that it has object clitics\is{clitic}:
\ea
\gll (Yo) Lo quiero.\\
     \hspaceThis{(}我 它 想\\
\mytrans{我想要它。}
%\gll (Yo) Lo quiero.\\
%     \hspaceThis{(}I it want\\
%\mytrans{I want it.}
\z
例(\mex{0})中的短代词lo在定式动词之前出现,所以动词出现在句末。但是,在英语中宾语在动词之后。另外,英语输入中复合动词形式的出现频率(70\,\%)远远高于西班牙语(25\,\%)。这是因为英语中的进行体\isce{进行体}{progressive}出现频率高而且问句形式中会出现do支撑\isceat{do-支撑}{\emph{do}-支撑}{do-Support}{\emph{do}-Support}。\ilce[|)]{西班牙语}{Spanish}
%Short pronouns such as \emph{lo} in (\mex{0}) are realized in front of the finite verb so that the verb
%appears in final position. In English, the object follows the verb, however. Furthermore, there are
%a greater number of compound verb forms in the English input (70\,\%) than in Spanish (25\,\%).
%This is due to the higher frequency of the progressive\is{progressive} in English and the
%presence of \emph{do}-support\is{do-Support@\emph{do}-Support} in question formation.\il{Spanish|)}

这里提出的习得模型可以很好地反映不定式的相关分布差异,但是其他的假设儿童拥有成人语法但是用不定式代替定式形式的方法不能解释这一现象的渐进性质。
%The relevant differences in the distribution of infinitives are captured correctly by the proposed acquisition model,
%whereas alternative approaches that assume that children possess an adult grammar but use infinitives
%instead of the finite forms cannot explain the gradual nature of this phenomenon.

\citet*{FPG2009a}甚至指出,就NP短语和不定式的分布而言,基于输入学习比其他解释更有说服力。它们可以解释在德语\ilce{德语}{German}和荷兰语\ilce{荷兰语}{Dutch}中为什么这一语序经常用于情态义(例如,to want)\iscesub{动词}{verb}{情态}{modal}\citep{IT96a}。在这些语言中,在相应疑问句中不定式与情态动词共现。其他方法认为,当前研究的语言结构和成人的语言结构相似,差别只在于一个情态动词不发音。这种方法不能解释为什么学习德语和荷兰语的儿童说出的所有语句并不都有情态义。另外,对于英语来说最大的差异不能解释:在英语中情态义的数量少得多。基于输入的模型可以很好地预测这一点,因为英语可以使用形式动词do来组成问句:
% \citet*{FPG2009a} could even show that input-based learning is superior to other explanations for the distribution of
%NPs and infinitives. They can explain why this order is often used with a modal meaning (\eg \emph{to want})\is{verb!modal}
%in German\il{German} and Dutch\il{Dutch} \citep{IT96a}. 
%In these languages, infinitives occur with modal verbs in the corresponding interrogative clauses. Alternative approaches that
%assume that the linguistic structures in question correspond to those of adults and only differ from them in that a modal verb
%is not pronounced cannot explain why not all utterances of object and verb done by children learning German and Dutch do have a modal meaning.
%Furthermore, the main difference to English cannot be accounted for: in English, the number of modal meanings is considerably
%less. Input-based models predict this exactly since English can use the dummy verb \emph{do} to form questions:
\eal
\ex 
\gll Did he help you?\\
     \textsc{aux} 他 帮助 你\\
\mytrans{他帮助你了吗?} 
%\ex Did he help you?
\ex 
\gll Can he help you?\\
      能够 他 帮助 你\\
\mytrans{他会帮助你吗?} 
%\ex Can he help you?
\zl
如果更大的实体从语句的末端习得,那么he help you必须要有一个情态和一个非情态语境。因为德语和荷兰语通常不会使用助动词tun(做),所以相关的语句末尾总是与情态语境相联系。这就可以解释为什么德语和荷兰语的不定式表达更是常有情态义。\ilce[|)]{英语}{English}\ilce[|)]{荷兰语}{Dutch} 
%If larger entities are acquired from the end of an utterance, then there would be both a modal and non-modal
%context for \emph{he help you}. Since German and Dutch normally do not use the auxiliary \emph{tun} `do', 
%the relevant endings of utterances are always associated with modals contexts.  One can thereby explain
%why infinitival expressions have a modal meaning significantly more often in German and Dutch than in English.\il{English|)}\il{Dutch|)} 

根据这个反对基于输入习得理论的论据,我要转向Tomasello的基于模式的方法\indexcxgstart。按照 \citet[第4.2.1节]{Tomasello2003a}的说法,儿童听到诸如例(\mex{1})的句子,并认识到特定的槽位可被随意填充(也可以参见 \citew{Dabrowska2001a}在认知语法\isce{认知语法}{Cognitive Grammar}框架中提出的相似方案)。
%Following this discussion of the arguments against input-based theories of acquisition, I will turn to Tomasello's pattern-based approach\indexcxgstart.
%According to  \citet[Section~4.2.1]{Tomasello2003a}, a child hears a sentence such as (\mex{1}) and realizes that particular slots can
%be filled freely (see also  \citew{Dabrowska2001a} for analogous suggestions in the framework of Cognitive Grammar\is{Cognitive Grammar}).
\eal
\ex 
\gll Do you want more juice / milk?\\
     \textsc{aux} 你 想 更多的 果汁 {} 牛奶\\
\mytrans{你想要更多的果汁/牛奶吗?} 
%\ex Do you want more juice/milk?
\ex 
\gll Mommy is gone.\\
    妈妈 \textsc{aux} 不见了\\
\mytrans{妈妈不见了。} 
%\ex Mommy is gone.
\zl
从这些表达中,就可以推导出所谓的轴心图式\isce{轴心图式}{pivot schema},正如例(\mex{1})所示,词语可以插入其中。
%From these utterances, it is possible to derive so-called pivot schemata\is{pivot schema} such as those in (\mex{1}) into which words
%can then be inserted:
\eal
\ex 
\gll  more \_\_\_ $\to$ more juice / milk\\
     更多的 {} {} 更多的 果汁 {} 牛奶\\
%\ex more \_\_\_ $\to$ more juice/milk
\ex 
\gll  \_\_\_ gone $\to$ mommy / juice gone\\
     {} 不见了 {} 妈妈 {} 果汁 不见了\\
%\ex \_\_\_ gone $\to$ mommy/juice gone
\zl
在这一发展阶段(22个月),儿童并不会使用这些图式进行概括,这些图式是构式岛并且没有任何句法 \citep{TADR97a}。在SVO语序中将之前不知道的动词与主语、宾语一同使用,在三岁到四岁之间习得很慢\citep[\page 128--129]{Tomasello2003a}。更加复杂的句法和语义关系只会随时间慢慢浮现:当多次遇到及物构式后,儿童就能概括这一构式了:
%In this stage of development (22 months), children do not generalize using these schemata, these schemata are instead construction islands
%and do not yet have any syntax \citep{TADR97a}. The ability to use previously unknown verbs with a subject and an object in an SVO order
%is acquired slowly between the age of three and four \citep[\page 128--129]{Tomasello2003a}.
%More abstract syntactic and semantic relations only emerge with time: when confronted with multiple instantiations of the transitive construction,
%the child is then able to generalize:
\eal
\label{Beispiele-fuer-Transitivkonstruktion}
\ex
\gll  {}[\sub{S} [\sub{NP} The man / the woman] sees  [\sub{NP} the dog / the rabbit / it]].\\
     {} {} \defart{} 男士 {} \defart{} 女士 看  {} \defart{} 狗 {} \defart{} 兔子 {} 它\\
%\ex {}[\sub{S} [\sub{NP} The man/the woman] sees  [\sub{NP} the dog/the rabbit/it]].
\mytrans{男人/女人看见狗/兔子/它。}
\ex 
\gll  {}[\sub{S} [\sub{NP} The man / the woman] likes [\sub{NP} the dog / the rabbit / it]].\\
     {} {} \defart{} 男士 {} \defart{} 女士 喜欢 {} \defart{} 狗 {} \defart{} 兔子 {} 它\\
%\ex {}[\sub{S} [\sub{NP} The man/the woman] likes [\sub{NP} the dog/the rabbit/it]].
\mytrans{男人/女人喜欢狗/兔子/它。}
\ex 
\gll  {}[\sub{S} [\sub{NP} The man / the woman] kicks [\sub{NP} the dog / the rabbit / it]].\\
     {} {} \defart{} 男士 {} \defart{} 女士 踢 {} \defart{} 狗 {} \defart{} 兔子 {} 它\\
%\ex {}[\sub{S} [\sub{NP} The man/the woman] kicks [\sub{NP} the dog/the rabbit/it]].
\mytrans{男人/女人踢狗/兔子/它。}
\zl
按照 \citet[\page 107]{Tomasello2003a}的观点,这一抽象结构(abstraction)的形式是[Sgj TrVerb Obj],Tomasello的方法可以立即奏效,因为人可以认识到抽象结构是怎样工作的:它是反复出现的模型的概括。每一个模式被分配一个语义解释。这些概括可以通过承继层级来获得(见~\pageref{Seite-Typhierarchie}页)\citep[\page  26]{Croft2001a}。但是,这一方法的问题在于,不能解释语言现象不同部分之间的互动:这一方法可以表征例(\mex{0})中所使用的及物动词简单模式,但是不能表征及物动词与语法其余部分之间的互动,例如与否定的互动。如果想要将带有否定的及物动词与及物构式联系起来,那么就会遇到一个问题,因为在承继层级中不能实现这一点。
%According to  \citet[\page 107]{Tomasello2003a}, this abstraction takes the form [Sbj TrVerb Obj]. 
%Tomasello's approach is immediately plausible since one can recognize how abstraction works:
%it is a generalization about reoccurring patterns. Each pattern is then assigned a semantic contribution.
%These generalizations can be captured in inheritance hierarchies (see page~\pageref{Seite-Typhierarchie}) \citep[\page
%  26]{Croft2001a}.
%The problem with this kind of approach, however, is that it cannot explain the interaction between different areas of phenomena in the
%language: it is possible to represent simple patterns such as the use of transitive verbs in
%(\mex{0}), but transitive verbs interact with other areas of the grammar such as negation. If one wishes to connect the construction one assumes for the negation
%of transitive verbs with the transitive construction, then one arrives at a problem since this is not possible in
%inheritance hierarchies.
\ea
\gll The woman did not kick the dog.\\
    \defart{} 女士 \textsc{aux} \textsc{neg} 踢 \defart{} 狗\\
\mytrans{这位女士并没有踢这只狗。} 
%The woman did not kick the dog.
\z
问题是及物构式有一个特定的语义,但是否定及物构式的意义与之相反。\textsc{sem}特征的取值因此会相反。虽然有技术技巧可以避免这一问题,但是因为语言中存在大量类似的句法和语义之间的互动,从认知的角度来看,这种技术方法十分不可行\citep{Mueller2006d,Mueller2007d,MuellerLehrbuch1,MuellerPersian,MWArgSt}。对于Croft方法的讨论,参见\ref{Abschnitt-Croft}。
%The problem is that the transitive construction has a particular semantic contribution but that
%negated transitive construction has the opposite meaning. The values of \textsc{sem} features would
%therefore be contradictory. There are technical tricks to avoid this problem, however, since there
%are a vast number of these kinds of interactions between syntax and semantics, this kind of
%technical solution will result in something highly implausible from a cognitive perspective
%\citep{Mueller2006d,Mueller2007d,MuellerLehrbuch1,MuellerPersian,MWArgSt}. 
%For discussion of Croft's analysis, see Section~\ref{Abschnitt-Croft}.

就这一问题,基于模式分析的支持者会极力争论说这类问题完全是形式框架\isce{形式化}{formalization}不完备造成的,如果不进行形式化就不会有这类问题\citep[第5节]{Goldberg2009a}。但是,这一观点并不能帮助解决这个问题,因为问题不在于形式框架本身,而是形式框架让人更好地看清楚问题。
%At this point, proponents of pattern-based analyses might try and argue that these kinds of problems are only
%the result of a poor/inadequate formalization\is{formalization} and would rather do without a formalization
%\citep[Section~5]{Goldberg2009a}. However, this does not help here as the problem is not the formalization itself, rather
%the formalization allows one to see the problem more clearly.

\begin{sloppypar}
与完全建立在承继上的方法不同,有一种类似于TAG的方法允许向短语构式中插入句法成分。这个方法可以参见\ref{sec-ECG}。动变构式语法流派的 \citet[\page
  170]{BC2005a}提出了一个主动-双及物构式,形式为\mbox{[RefExpr Verb RefExpr RefExpr]},其中RefExpr代表一个指称表达,第一个RefExpr和动词可能不邻接。以这种方式,就可以分析(\mex{1}a,b),并排除(\mex{1}c):
%An alternative to an approach built entirely on inheritance is a TAG-like approach that allows one to insert syntactic material
%into phrasal constructions. Such a proposal was discussed in Section~\ref{sec-ECG}.  \citet[\page
%  170]{BC2005a} working in Embodied
%Construction Grammar suggest an Active-Ditransitive Construction with the form \mbox{[RefExpr Verb
%RefExpr RefExpr]}, where RefExpr stands for a referential expression and the first RefExpr and the
%verb may be non-adjacent. In this way, it is possible to analyze (\mex{1}a,b), while ruling out
%(\mex{1}c):
\end{sloppypar}
\eal
\ex[]{
\gll Mary tossed me a drink.\\
    Mary 扔 我 一 饮料\\
\mytrans{Mary扔给我一瓶饮料。} 
%Mary tossed me a drink.
}
\ex[]{
\gll Mary happily tossed me a drink.\\
    Mary 高兴地 扔 我 一 饮料\\
\mytrans{Mary高兴地扔给我一瓶饮料。} 
%Mary happily tossed me a drink.
}
\ex[*]{
\gll Mary tossed happily me a drink.\\
    Mary 扔 高兴地 我 一 饮料\\
%Mary tossed happily me a drink.
}
\zl
虽然强制要求动词和宾语邻接的条件正确地预测出(\mex{0}c)被排除了,但是对应限制同时也排除了(\mex{1})所示的并列\isce{并列}{coordination}结构:
%While the compulsory adjacency of the verb and the object correctly predicts that (\mex{0}c) is
%ruled out, the respective constraint also rules out coordinate structures\is{coordination} such as (\mex{1}):
\ea
\gll Mary tossed me a juice and Peter a water.\\
    Mary 扔 我 一 果汁 并且 Peter 一 水\\
\mytrans{Mary扔给我一瓶果汁并且扔给Peter一瓶水。} 
%Mary tossed me a juice and Peter a water.
\z
这个句子的部分意义对应双及物构式赋予Mary tossed Peter a water的意义。但是tossed和Peter间有个空位。同样,可以造出双及物构式两个宾语之间存在空位的例子:
%Part of the meaning of this sentence corresponds to what the ditransitive construction contributes to
%\emph{Mary tossed Peter a water}. There is, however, a gap between \emph{tossed} and \emph{Peter}.
%Similarly, one can create examples where there is a gap between both objects of a ditransitive construction:
\ea
\gll He showed me and bought for Mary the book that was recommended in the Guardian last week.\\
    他 展示 我 并且 买 \textsc{prep} Mary \defart{} 书 \textsc{rel} \passivepst{} 推荐 \textsc{prep} \defart{} 《卫报》 上一个 周\\
\mytrans{他向我展示并且为Mary买了那本上周《卫报》推荐的书。} 
%He showed me and bought for Mary the book that was recommended in the Guardian last week.
\z
在例 (\mex{0})中,me与the book \ldots{}并不邻接,这里我的目的不是想要一个并列分析。并列结构是一个非常复杂的现象,大部分理论对此都没有一个直观的分析(见\ref{Abschnitt-Koordination})。相反,我想指出以下事实:构式可以不连续实现, 这一事实给那些声称语言习得完全基于模式的方法提出了一个问题。观点如下:为了理解语言中的并列现象,说话者必须知道有论元在句中实现的动词与这些论元组合有特定的意义。但是,实际模式[Sbj V Obj1 Obj2]可以在任何位置插入成分。除了并列现象,还有可能将成分从模板中移出,移到左边或右边。总之,我们可以说语言学习者必须习得函数与其变元之间有关系。这是基于模式的方法剩下的所有内容,但我们将在下一节讨论的基于选择的方法也涵盖了这一点。
%In (\mex{0}), \emph{me} is not adjacent to \emph{the book \ldots}. It is not my aim here to request a coordination
%analysis. Coordination is a very complex phenomenon for which most theories do not have a straightforward analysis (see Section~\ref{Abschnitt-Koordination}).
%Instead, I would simply like to point out that the fact that constructions can be realized discontinuously poses a problem for approaches that claim that language acquisition is exclusively
%pattern-based. The point is the following: in order to understand coordination data in a language, a speaker must learn that a verb which has its arguments somewhere in the sentence has
%a particular meaning together with these arguments. The actual pattern [Sbj V Obj1 Obj2] can, however, be interrupted in all positions.
%In addition to the coordination examples, there is also the possibility of moving elements out of the pattern either to the left or the right.
%In sum, we can say that language learners have to learn that there is a relation between functors and their arguments. This is all that is left of
%pattern-based approaches but this insight is also covered by the selection-based approaches that we will discuss in the following section.

基于模式方法的支持者可能会反对说:有一个相关构式可以处理例 (\mex{0}) ,将所有语料组合在一起。这意味着有一个构式,其形式是[Sbj V Obj1 Conj V PP Obj2]。这需要实验或者语料库研究来证明这一方式是否有效。语言学家做出的概括是,具有相同句法属性的范畴可以并列(N, \nbar, NP, V, \vbar, VP, \ldots)。对于动词或动词性投射的并列来说,并列短语需要相同的论元:
%A defender of pattern-based approaches could perhaps object that there is a relevant construction for (\mex{0}) that combines
%all material. This means that one would have a construction with the form [Sbj V Obj1 Conj V PP Obj2].
%It would then have to be determined experimentally or with corpus studies whether this actually makes sense.
%The generalization that linguists have found is that categories with the same syntactic properties can be coordinated 
%(N, \nbar, NP, V, \vbar, VP, \ldots).\todostefan{Bei DG diskutieren} For the coordination of verbs or verbal projections, it must hold that the coordinated
%phrases require the same arguments:
\eal
\ex 
\gll Er [arbeitet] und [liest viele Bücher].\\
     他 \spacebr{}工作 并且 \spacebr{}阅读 很多 书\\
%\gll Er [arbeitet] und [liest viele Bücher].\\
%     he \spacebr{}works and \spacebr{}reads many books\\
\mytrans{他工作而且读很多书。}
\ex 
\gll Er [kennt und liebt] diese Schallplatte.\\
     他 \spacebr{}知道 并且 喜欢 这 唱片\\
%\gll Er [kennt und liebt] diese Schallplatte.\\
%     he \spacebr{}knows and loves this record\\
\mytrans{他知道并且喜欢这张唱片。}
\ex 
\gll Er [zeigt dem Jungen] und [gibt der Frau] die Punk-Rock-CD.\\
     他 \spacebr{}展示 \defart{} 男孩 并且 \spacebr{}给 \defart{} 女士 \defart{} {朋克 摇滚 CD}\\
%\gll Er [zeigt dem Jungen] und [gibt der Frau] die Punk-Rock-CD.\\
%     he \spacebr{}shows the boy and \spacebr{}gives the woman the {punk rock CD}\\
\mytrans{他给男孩看,并送给这位女士这张朋克摇滚CD。}
\ex 
\gll Er [liebt diese Schallplatte] und [schenkt ihr ein Buch].\\
     他 \spacebr{}喜欢 \defart{} 唱片 并且 \spacebr{}给 她 一 书\\
%\gll Er [liebt diese Schallplatte] und [schenkt ihr ein Buch].\\
%     he \spacebr{}loves this record and \spacebr{}gives her a book\\
\mytrans{他喜欢这张唱片,并送给她一本书。}
\zl
在一个只包含模式的方法中,必须要假设大量的构式,迄今为止我们只是考虑了包含两个并列项的并列式。但是,上面讨论的现象不仅限于两个元素的并列。如果我们不想放弃语言能力\isce{语言能力}{competence}和语言运用\isce{语言运用}{performance}的差异(参见第\ref{Abschnitt-Diskussion-Performanz}章),那么并列项的数量是完全没有限制的(按照能力语法):
%In an approach containing only patterns, one would have to assume an incredibly large number of constructions and so far we are
%only considering coordinations that consist of exactly two conjuncts. However, the phenomenon discussed above is not only restricted
%to coordination of two elements. If we do not wish to abandon the distinction between competence and performance\is{competence}\is{performance}
%(see Chapter~\ref{Abschnitt-Diskussion-Performanz}), then the number of conjuncts is not constrained at all (by the competence grammar):
\ea
\gll Er [kennt, liebt und verborgt] diese Schallplatte.\\
	 他 \spacebr{}知道 喜欢 并且 借出 这 唱片\\
%\gll Er [kennt, liebt und verborgt] diese Schallplatte.\\
%	 he \spacebr{}knows loves and lends.out this record\\
\mytrans{他知道、喜欢并且借了这张唱片。}
\z
所以学习者不太可能在输入中获得所有可能的现象。更加有可能的情况是:学习者像语言学家那样从他们接触到的语言现象中得出概括:具有相同句法属性的词或短语可以并列。
如果这一点是真的的话,那么基于模式的方法剩下的就只是构式可以不连续实现的假设,以及不邻接成分之间也存在依存关系这一假设。习得问题就与基于选择的方法(将在下一章节讨论)一样:最终必须学习的是元素之间的依存或者说是配价关系(见 \citew[\page 439]{Behrens2009a},作者在考虑了多种因素之后得到了相似的结论)。%
\indexcxgend
%It is therefore extremely unlikely that learners have patterns for all possible cases in their input. It is much more likely that they draw
%the same kind of generalizations as linguists from the data occurring in their input: words and phrases with the same syntactic
%properties can be coordinated. 
%If this turns out to be true, then all that is left for pattern-based approaches is the assumption of discontinuously realized constructions
%and thus a dependency between parts of constructions that states that they do not have to be immediately adjacent to one another.
%The acquisition problem is then the same as for selection-based approaches that will be the topic of the following section: what ultimately has
%to be learned are dependencies between elements or valences (see  \citew[\page 439]{Behrens2009a}, the author reaches the same conclusion
%following different considerations).%
%\indexcxgend

\section{基于选择的方法}
%\section{Selection-based approaches}
\label{Abschnitt-Selektionsbasierter-Spracherwerb}

我将不同于基于模式的方法称作“基于选择的方法”。基于选择的方法是由 \citet{Green-Grammar-Growth}提出的。
%I will call the alternative to pattern-based approaches \emph{selection-based}. A selection-based approach has
%been proposed by  \citet{Green-Grammar-Growth}.  

例(\ref{Beispiele-fuer-Transitivkonstruktion})中模式的概括与动词的配价类别有关。在范畴语法中,模式[Sbj TrVerb Obj]对应着词条(s\bs np)/np(关于用这类词条推导出句子的内容可参见\pageref{abb-the-cat-chased-Mary}页的图\ref{abb-the-cat-chased-Mary})。第\pageref{Abbildung-Max-likes-Anouk}页给出了likes的一棵TAG树。在这里,可以很清楚地看到在这些模型中词项决定了句子的结构。不像基于模式的方法,这些分析为语义嵌套提供了足够的空间:范畴语法中的词项可以与附加语组合,在TAG中初级树也允许附加到相关结点上。
%The generalizations about the pattern in (\ref{Beispiele-fuer-Transitivkonstruktion}) pertain to the valence class of the verb.
%In Categorial Grammar, the pattern [Sbj TrVerb Obj] corresponds to the lexical entry (s\bs np)/np (for the derivation of a sentence
%with this kind of lexical entry, see Figure~\ref{abb-the-cat-chased-Mary} on page~\pageref{abb-the-cat-chased-Mary}).
%A TAG tree for \emph{likes} was given on page~\pageref{Abbildung-Max-likes-Anouk}.
%Here, one can see quite clearly that lexical entries determine the structure of sentences in these models. Unlike pattern-based approaches, these analyses allow
%enough room for semantic embedding: the lexical entries in Categorial Grammar can be combined with adjuncts, and elementary trees in TAG also allow for adjunction
%to the relevant nodes.

现在,我们面对的问题是,从轴心图式向一个带有一个论元结构的词项的跳跃是怎样发生的。在Tomasello的方法中,两者之间没有停顿。轴心图式是短语模式而[Sbj TrVerb Obj]也是一个短语模式。两种模式都有空位允许一些特定成分插入。在基于选择的方法中,情形是类似的:在基于选择的方法中,轴心图式的固定部分是函数\isce{函数}{functor}。 \citet{Green-Grammar-Growth}在HPSG框架中提出了一种习得理论,这种理论可以在不使用UG的情况下起作用。对于双词短语,她假设where’s是例(\mex{1}) 中一个表达的中心语并且where’s选择Robin作为其变元。
%Now, we face the question of how the jump from a pivot schema to a lexical entry with an argument structure takes place. In Tomasello's approach, there is no break between them. Pivot schemata
%are phrasal patterns and [Sbj TrVerb Obj] is also a phrasal pattern. Both schemata have open slots into which certain elements can be inserted.
%In selection-based approaches, the situation is similar: the elements that are fixed in the pivot schema are functors\is{functor} in the selection-based approach.
% \citet{Green-Grammar-Growth} proposes a theory of acquisition in HPSG that can do without UG. For the two-word phase, she assumes that  \emph{where's} is the head
%of an utterance such as (\mex{1}) and that \emph{where's} selects \emph{Robin} as its argument.
\ea
\gll Where's Robin?\\
    哪里\textsc{cop} Robin\\
\mytrans{Robin在哪里?} 
%Where's Robin?
\z
这意味着,她不假设存在一个包含可容纳一人或一物的空位X的短语模式Where’s X?;而是假设存在一个词项where’s,该词项包括它与另外一个成分组合的信息。同样,需要学习的是:存在一个特定的成分,其必须与其他成分一起才能形成一个完整的表达。
%This means that, rather than assuming that there is a phrasal pattern \emph{Where's} X? with an empty slot X for a person
%or thing, she assumes that there is a lexical entry \emph{where's}, which contains the information that it  needs to be combined
%with another constituent. What needs to be acquired is the same in each case: there is particular material that has to be combined
%with other material in order to yield a complete utterance.

在她的文章中,Green展示了长距离依存关系和英语助动词怎样在发展的后期习得的。语法的习得是单调递增的,也就是说新的知识在增加⸺例如,成分可以在局域语境之外实现⸺以前的知识不必修正。在她的模型中,习得过程中的错误实际上是给词项分配配价类型时产生的错误。这些错误必须是可以纠正的。
%In her article, Green shows how long-distance dependencies and the position of English auxiliaries can be acquired in later stages of development.
%The acquisition of grammar proceeds in a monotone fashion, that is, knowledge is added -- for example, knowledge about the fact that
%material can be realized outside of the local context -- and previous knowledge does not have to be revised.
%In her model, mistakes in the acquisition process are in fact mistakes in the assignment of lexical
%entries to valence classes. These mistakes have to be correctable. 

总之,可以说Tomasello所有的观点都可以直接运用到基于选择的方法上,基于模式方法遇到的问题在基于选择的方法中没有显现出来。这里必须要再次清楚地指出一点,这里讨论的基于选择的方法也是一种基于构式的方法。构式是词汇性的而非短语性的。重要的一点是:在两种方法中,词和更加复杂的短语都是形式和意义的配对,而且都可以这样习得。
%In sum, one can say that all of Tomasello's insights can be applied directly to selection-based approaches and the problems with pattern-based
%approaches do not surface with selection-based approaches. It is important to point out explicitly once again here that the selection-based approach
%discussed here also is a construction-based approach. Constructions are just lexical and not phrasal.
%The important point is that, in both approaches, words and also more complex phrases are pairs of form and meaning and can be acquired as such.

在第\ref{Abschnitt-Phrasal-Lexikalisch}章,我们将进一步讨论基于模式的方法并将进一步探索需要假设短语模式的语法领域。
%In Chapter~\ref{Abschnitt-Phrasal-Lexikalisch}, we will discuss pattern-based approaches further
%and we will also explore areas of the grammar where phrasal patterns should be assumed.

%\section{Summary}
\section{总结}

我们可以从上述讨论得出以下结论:假设一种语法是从一系列通过设置二元参数的语法中选择出来的语言习得模型实际上是不够好的。所有借助参数的理论都是假设,因为并不存在这一模型支持者能达成一致的重要的参数集合。实际上甚至并不存在一个不重要的参数集合。
%We should take from the preceding discussion that models of language acquisition that assume that a grammar is chosen
%from a large set of grammars by setting binary parameters are in fact inadequate.
%All theories that make reference to parameters have in common that they are purely hypothetical since there is no
%non-trivial set of parameters that all proponents of the model equally agree on. In fact there is
%not even a trivial one.

在多个实验中,Tomasello及其同事都表明,最初形式的原则 \& 参数理论作出的预测有误,而且相比P\&P分析的支持者所认为的,语言学习更多还是基于模式的。句法能力从动词岛开始。有赖于输入的频率,某些动词构式可以被掌握,即便是带有较低频率动词的同一构式并没有被习得。
%In a number of experiments, Tomasello and his colleagues have shown that, in its original form, the Principles \& Parameters model makes incorrect
%predictions and that language acquisition is much more pattern-based than assumed by proponents of P\&P analyses.
%Syntactic competence develops starting from verb islands. Depending on the frequency of the input, certain verbal constructions can be
%mastered even though the same construction has not yet been acquired with less frequent verbs.

与语法其余部分的交互对于基于模式的方法来说仍然有问题:在大量著作中,可以看到在复杂表达中这一现象的交互实际上不能用短语模式来解释,因为嵌套无法用承继层级来表示。而使用基于选择的方式就不存在这一问题。但是,Tomasello的所有实验结果和观点都可以成功扩展到基于选择的方法上去。
\isce[|)]{习得}{acquisition}
%The interaction with other areas of grammar still remains problematic for pattern-based approaches: in a number of publications, it has been shown
%that the interaction of phenomena that one can observe in complex utterances can in fact not be explained with phrasal patterns since
%embedding cannot be captured in an inheritance hierarchy. This problem is not shared by selection-based approaches. All experimental results and insights
%of Tomasello can, however, be successfully extended to selection-based approaches.%
%\is{acquisition|)}

%\section*{延伸阅读}
%\section*{Further reading}
%\bigskip
\furtherreading{
\citet{Meisel95a}很好地综述了基于原则 \& 参数理论的语言习得模型。
% \citet{Meisel95a} gives a very good overview of theories of acquisition in the Principles \& Parameters model.

Adele Goldberg和Michael Tomasello是最著名的构式语法的支持者,构式语法尽力摆脱天赋语言知识假说。他们发表了很多关于构式语法和语言习得的文章和专著。最重要的专著可能是 \citew{Goldberg2006a}和 \citew{Tomasello2003a}。
%Adele Goldberg and Michael Tomasello are the most prominent proponents of Construction Grammar, a theory that explicitly tries
%to do without the assumption of innate linguistic knowledge. They published many papers and books
%about topics related to Construction Grammar and acquisition. The most important books probably are  \citew{Goldberg2006a} and  \citew{Tomasello2003a}.

用德文写成的对于不同语言习得理论的综述可参见 \citet{KD2008a} ,英文写成的综述有 \citew{AL2011a-u}。
%An overview of different theories of acquisition in German can be found in  \citet{KD2008a} an
%English overview is  \citew{AL2011a-u}.
}

%      <!-- Local IspellDict: en_US-w_accents -->
