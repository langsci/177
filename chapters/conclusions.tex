%% -*- coding:utf-8 -*-
\chapter{结论}
%\chapter{Conclusion}

\largerpage[2]
本书讨论的分析都呈现出了一些相似性。所有框架都使用了复杂范畴来描述语言对象。这在GPSG\indexgpsg、
LFG\indexlfg、HPSG\indexhpsg、CxG\indexcxg 和FTAG\indextag 中尤为明显。不过,GB/最简方案和范畴语法也将第三人称单数作为语言对象来描写,跟描写这些语言对象相关的词类、人称和数的特征就构成了复杂范畴。GB理论中,有带二元值的特征N和V\citep[\page 199]{Chomsky70a}, \citet[\page
119]{Stabler92a-u}用特征值偶对构成“语障”(Barriers),并且 \citet[\page
290--291]{SE2002a}主张在最简方案\indexmp 中使用特征值偶对。同样,参见 \citet{Veenstra98a}使用类型特征描写构建的基于约束的最简分析。依存语法的变体,如Hellwig的依存合一语法\isce{依存合一语法}{Dependency Unification Grammar (DUG)},也使用了特征值偶对\citep[\page 612]{Hellwig2003a}。
%The analyses discussed in this book show a number of similarities. All frameworks use complex
%categories to describe linguistic objects. This is most obvious for GPSG\indexgpsg,
%LFG\indexlfg, HPSG\indexhpsg, CxG\indexcxg and FTAG\indextag, however, GB/Minimalism and Categorial Grammar also talk about NPs in third person singular and the relevant features for part %of
%speech, person and number form part of a complex category. In GB, there are the feature N and V with binary values \citep[\page 199]{Chomsky70a},  \citet[\page
%119]{Stabler92a-u} formalizes \emph{Barriers} with feature-value pairs and  \citet[\page
%290--291]{SE2002a} argue for the use of feature-value pairs in a Minimalist theory\indexmp. Also, see  \citet[\page]{Veenstra98a} for a constraint-based formalization
%of a Minimalist analysis using typed feature descriptions. Dependency Grammar dialects like
%Hellwig's Dependency Unification Grammar\is{Dependency Unification Grammar (DUG)} also use feature-value
%pairs \citep[\page 612]{Hellwig2003a}.

再者,所有当代理论框架(除了构式语法和依存语法)在分析德语句子结构时都体现出了一致性:德语是一个SOV和V2型语言。在结构上,动词首位的小句语序跟动词末位的小句语序是一致的。定式动词要么移位(GB),要么位于跟动词末位位置有关的成分上(HPSG)。动词第二位的小句由动词首位的小句构成,其中有一个成分被提取出了。可以看到,被动的分析也趋于一致:HPSG采用了 \citet{Haider84b,Haider85b,Haider86}最早在GB框架下构建的思想。构式语法的一些变体也利用了特殊标记的“指定论元”\iscesub{论元}{argument}{指定论元}{designated} \citep[\page55--57]{MR2001a}。
%Furthermore, there is a consensus in all current frameworks (with the exception of Construction
%Grammar and Dependency Grammar) about how the sentence structure of German should
%be analyzed: German is an SOV and V2 language. Clauses with verb-initial order resemble verb-final ones in terms of structure. The finite verb is
%either moved (GB) or stands in a relation to an element in verb-final position (HPSG). Verb-second
%clauses consist of  verb-initial clauses out of which one constituent has been extracted. It is also possible to see some convergence with regard to the analysis of the passive: some ideas %originally formulated
%by  \citet{Haider84b,Haider85b,Haider86} in the framework of GB have been adopted by HPSG. Some
%variants of Construction Grammar also make use of a specially marked `designated
%argument'\is{argument!designated} \citep[\page55--57]{MR2001a}.

个别理论框架的新发展很清楚地显示,其所提出的分析的性质常有极大的差异。而CG、LFG、HPSG和CxG都是表层导向的,有时在最简方案中会提出非常抽象的结构,而在有些情况下,有人试图将所有语言都归结于一个基本结构(普遍性本质假说\isce{普遍性本质假说}{Universal Base Hypothesis})。\footnote{%
需要指出的是,现在有很多子变体和最简方案研究人员的个别观点,所以只能像CxG那样讨论倾向性。
}只有我们认为所有语言共享的基本结构和生成表层结构那些必要操作的天赋语言知识是存在的,这种方法才有意义。正如在第\ref{chap-innateness}章所展示的,所有支持天赋语言知识的假说要么是站不住脚的,要么至少是有争议的。语言能力的习得可以在很大程度上得到基于输入的解释(\ref{Abschnitt-UDOP}、\ref{Abschnitt-musterbasiert}和\ref{Abschnitt-Selektionsbasierter-Spracherwerb})。不是所有关于习得的问题都被一次性解决,但是,对于那些对天赋的语言知识持谨慎态度的人来说,基于输入的方法至少貌似正确。
%If we consider new developments in the individual frameworks, it becomes clear that the nature of the proposed analyses can sometimes differ drastically.
%Whereas CG, LFG, HPSG and CxG are surface-oriented, sometimes very abstract structures are assumed in Minimalism and in some cases, one tries to trace all languages back
%to a common base structure (Universal Base Hypothesis\is{Universal Base Hypothesis}).\footnote{%
%  It should be noted that there are currently many subvariants and individual opinions in the Minimalist community so that it
%  is only possible -- as with CxG -- to talk about tendencies.%
%} This kind of approach only makes sense if one assumes that there is innate linguistic knowledge about this base structure
%common to all languages as well as about the operations necessary to derive the surface structures.
%As was shown in Chapter~\ref{chap-innateness}, all arguments for the assumption of innate linguistic knowledge are either not tenable
%or controversial at the very least.
%The acquisition of linguistic abilities can to a large extent receive an input-based explanation
%(Section~\ref{Abschnitt-UDOP}, Section~\ref{Abschnitt-musterbasiert} and
%Section~\ref{Abschnitt-Selektionsbasierter-Spracherwerb}). Not all questions about acquisition have
%been settled once and for all, but input-based approaches are at least plausible enough for one to be very cautious about any assumption of innate linguistic knowledge.
% \citep[\page 4]{Chomsky2007a}

诸如LFG、CG、HPSG、CxG和TAG的模型都符合语言运用的数据,而这些数据在特定的基于转换的方法中是不可能的,这种基于转换的方法被看作是语言能力的理论,而不是语言运用的理论。MGG认为还有其他机制作用于语言知识,如结合“组块”(语言材料的片段)的机制。如果我们想做出这些假设,那么就需要解释组块和组块的处理是如何获得的,而不是转换的复杂系统和转换对比的限制是如何获得的。这就意味着语言习得的问题会非常不同。如果我们认同基于组块的方法,那么关于普遍的转换基础的天赋知识就只能被用来生成表层导向的语法。那么,这就引出了描写语言能力的语法中转换的证据到底是什么的问题,而且是不是应该直接认为描写语言能力的语法属于LFG、CG、HPSG、CxG或TAG所假设的类型。由此,我们可以总结出这样的结论,基于约束的分析与允许基于约束的重新理论建构的转换方法的类型是跟现在的语言事实唯一兼容的方法,而其他所有的分析都需要额外的理论假设。
%Models such as LFG, CG, HPSG, CxG and TAG are compatible with performance data, something that is
%not true of certain transformation-based approaches, which are viewed as theories of competence
%that do not make any claims about performance. In MGG, it is assumed that there are other mechanisms
%for working with linguistic knowledge, for example, mechanisms that combine `chunks' (fragments of linguistic material). If one wishes to make these assumptions,
%then it is necessary to explain how chunks and the processing of chunks are acquired and not how a complex system of transformations and transformation-comparing
%constraints is acquired. This means that the problem of language acquisition would be a very different one. If one assumes a chunk-based approach, then the innate
%knowledge about a universal transformational base would only be used to derive a surface-oriented grammar. This then poses the question of what exactly the evidence
%for transformations in a competence grammar is and if it would not be preferable to simply assume that the competence grammar is of the kind assumed by LFG, CG, HPSG,
%CxG or TAG. One can therefore conclude that constraint-based analyses and the kind of
%transformational approaches that allow a constraint-based reformulation are the only approaches that are compatible with the current facts, whereas all other analyses require additional %assumptions.

最简方案的一些研究区别于其他框架下的研究,因为他们认为结构只能从其他语言的事实所激发。这就可以简化生成不同结构的整个机制,但是这个方法的所有代价并没有减少:有些代价只是被转化为UG的组成部分。他们得到的抽象语法是无法从输入学会的。
%A number of works in Minimalism differ from those in other frameworks in that they assume structures (sometimes also invisible structures) that can only be motivated
%by evidence from other languages. This can streamline the entire apparatus for deriving
%different structures, but the overall costs of the approach are not reduced: some amount of the
%cost is just transferred to the UG component. The abstract grammars that result cannot be learned from the input.

我们可以从这个讨论获知,只有基于约束的、表层导向的模型是合适的且具有解释力的:它们也符合心理语言学的事实,并且从习得的角度来看也是合理的。
%One can take from this discussion that only constraint-based, surface-oriented models are adequate
%and explanatory: they are also compatible with psycholinguistic facts and plausible from the point of view of acquisition.

如果我们现在比较这些方法,我们会发现一些分析可以互相转换。LFG(以及CxG与DG的一些变体)与其他所有理论不同的地方在于,它将主语和宾语这类语法功能\isce{语法功能}{grammatical function}看作是基本类型。如果我们不想这样分析,那么就可以将这些标签替换为论元1、论元2等。论元的数量对应于它们相对的旁格性。这样,LFG就会更像HPSG了。另外,我们可以在HPSG和CxG中根据语法功能来标记论元。在被动分析中就是这样做的(“指定的论元”)。
%If we now compare these approaches, we see that a number of analyses can be translated into one
%another. LFG (and some variants of CxG and DG) differ from all other theories in that grammatical
%functions\is{grammatical function} such as subject and object are primitives of the theory. If one does not want this, then it is possible to replace these labels with Argument1, Argument2,
%etc. The numbering of arguments would correspond to their relative obliqueness. LFG would then move
%closer to HPSG. Alternatively, one could mark arguments in HPSG and CxG with regard to their grammatical function additionally. This is what is done for the analysis of the passive 
%(\textsc{designated argument}).

LFG、HPSG、CxG和范畴语法的变体\citep{MCKRZ89a-u,Briscoe2000a,Villavicencio2002a}有对知识进行层级式组织的方式\isce{承继}{inheritance},这对普遍规律的捕捉是非常重要的。当然,可以按照这个方式来扩展其他任何的框架,但是除了计算机的应用之外,这从未被明确地表示出来,而且承继层级体系在其他框架的理论构建方面没有起到积极的作用。
% Word Grammar auch Hudson90
%LFG, HPSG, CxG and variants of Categorial Grammar \citep{MCKRZ89a-u,Briscoe2000a,Villavicencio2002a}
%possess means for the hierarchical organization of knowledge\is{inheritance}, which is important for capturing generalizations.
%It is, of course, possible to expand any other framework in this way, but this has never been done
%explicitly, except in computer implementations and inheritance hierarchies do not play an active role in theorizing in the other frameworks.

在HPSG和CxG中,词根、词干、词、形态规则和句法规则都是可以按照相同方式来描述的对象。这就允许我们总结出影响不同对象的普遍规律(见第\ref{Abschnitt-UG-mit-Hierarchie}章)。
在LFG中,c-结构被认为在根本上是不同的,这就是为什么这种普遍化是不可能的。在跨语言的研究中,有一种在f-结构中捕捉相似性的尝试,c-结构不太重要,甚至在一些研究中都没有被提及。再者,针对不同语言的计算实现有较大的差异。基于此,我个人倾向于使用相同方式描写所有语言对象的理论框架,即HPSG和CxG。形式上,使用特征值偶对对LFG语法的c-结构进行描写并没有什么障碍,所以在可预见的未来,理论之间会有更多的地方趋于一致。例如,有关HPSG和LFG的混合形式参见 \citew{AW98a}和 \citew{HH2004a-u}。
%In HPSG and CxG, roots, stems, words, morphological and syntactic rules are all objects that can be
%described with the same means. This then allows one to make generalizations that affect very
%different objects (see Chapter~\ref{Abschnitt-UG-mit-Hierarchie}).   
%In LFG, c-structures are viewed as something fundamentally different, which is why this kind of generalization is not possible. In cross-linguistic
%work, there is an attempt to capture similarities in the f-structure, the c-structure is less important and is not even discussed in a number of
%works. Furthermore, its implementation from language to language can differ enormously. For this reason, my personal preference is for frameworks
%that describe all linguistic objects using the same means, that is, HPSG and CxG. Formally, nothing stands in the way of a description of the c-structure
%of an LFG grammar using feature-value pairs so that in years to come there could be even more
%convergence between the theories. For hybrid forms of HPSG and LFG, see  \citew{AW98a} and  \citew{HH2004a-u}, for example.

如果我们比较CxG和HPSG,CxG研究中形式化的程度明显是相对较低的,而且有一些问题没有得到解答。CxG(除了动变构式语法)中较为形式化的方法是HPSG的变体。构式语法中相对来说有较少的精细化构造的分析,而且没有针对德语的描写,这点是无法跟本书中介绍的其他方法相比的。公平来说,必须要指出的是,构式语法是这里讨论的理论中最为年轻的。它对语言学理论的重要贡献被整合进了HPSG和LFG这类理论框架中。
%If one compares CxG and HPSG, it becomes apparent that the degree of formalization in CxG works is relatively low and a number of questions remain
%unanswered. The more formal approaches in CxG (with the exception of Fluid Construction Grammar) are variants of HPSG. There are relatively few precisely worked-out analyses in %Construction Grammar
%and no description of German that would be comparable to the other approaches presented in this
%book. To be fair, it must be said that Construction Grammar
%is the youngest of the theories discussed here. Its most important contributions to linguistic theory have been integrated into frameworks such as HPSG and LFG.

未来的理论将是表层导向、基于约束和模型论方法的整合,诸如CG\indexcg、LFG\indexlfg、HPSG\indexhpsg 和构式语法\indexcxg 的模型论方法,以及TAG\indextag 和GB/最简方案\indexgb \indexmp 的相关变体将会变革为基于约束的方法。最简方案(的变体)和构式语法(的变体)是目前最为广泛采用的方法。实际上,我猜想真相就介于二者之间。未来的语言学将是数据驱动的。内省式\isce{内省}{introspection}的方法作为唯一的数据搜集的方法被证明是不可靠的\citep{Mueller2007c,MM2009a},而且更多地需要辅以实验和基于语料库\isce{语料库语言学}{corpus linguistics}的分析。
%The theories of the future will be a fusion of surface-oriented, constraint-based and
%model-theoretic approaches like CG\indexcg, LFG\indexlfg, HPSG\indexhpsg,
%Construction Grammar\indexcxg, equivalent variants of TAG\indextag and GB/Minimalist\indexgb\indexmp approaches that will be reformulated as constraint-based.
%(Variants of) Minimalism and (variants of) Construction Grammar are the most widely adopted approaches at present. I actually suspect the truth to lie somewhere
%in the middle. The linguistics of the future will be data-oriented. Introspection\is{introspection} as the sole method of data collection
%has proven unreliable \citep{Mueller2007c,MM2009a} and is being increasingly complemented by experimental and corpus-based\is{corpus linguistics} analyses.\todostefan{AL 16.05.2010: zur %grammatikalität finde ich ja den aufsatz 'grammar without grammaticality' von sampson im CLLt gut (und auch die anderen Artikel im selben heft, die sich darauf beziehen)}

统计信息和统计程序在机器翻译中发挥了重要的作用,而且在狭义的语言学方面变得更重要了\citep{Abney96a}。我们已经看到,统计信息在习得过程中是非常重要的,而且Abney讨论了语言的其他方面,如语言变化、句法分析倾向性和合格性判断递差。
随着统计程序越来越受到重视,现在计算语言学面临着向混合形式的转向,\footnote{%
有关HPSG语法的话语识别器的混合请参考 \citew{KP2007a}和 \citew{Kaufmann2009a-u}。
}
因为有研究发现仅仅通过统计方法无法超越某种质量水平\citep{Steedman2011a,Church2011a,Kay2011a}。
跟前述一样的是:真相位于二者之间,即整合的系统。为了整合,相关的语言学理论首先需要进一步发展。正如Manfred Pinkal所说的:“我们不可能在没有了解语言的情况下构造出理解语言的系统。”
%Statistical information and statistical processes play a very important role in machine translation and are becoming more important
%for linguistics in the narrow sense
%\citep{Abney96a}. We\todostefan{AL 16.05.2010: müsste man hier nicht unbedingt Bod/Hay/Jannedy Probabilistic linguistics zitieren? }
%have seen that statistical information is important in the acquisition process and Abney discusses cases of other
%areas of language such as language change, parsing preferences and gradience with grammaticality judgments.
%Modelle wie GB \citep{FC94a}, TAG \citep{Resnik92a}, HPSG \citep{Brew95a}, LFG \citep{HK2007a-u}
%und CG \citep*{OB97a,CHS2002a-u} werden auch mit statistischen Elementen kombiniert.
%Following a heavy focus on statistical procedures, there is now a transition to hybrid forms in computational linguistics,\footnote{%
%See  \citew{KP2007a} and  \citew{Kaufmann2009a-u} for the combination of a speech recognizer with a HPSG grammar.
%}
%since it has been noticed that it is not possible to exceed certain levels of quality with statistical methods alone \citep{Steedman2011a,Church2011a,Kay2011a}. 
%The same holds here as above: the truth is somewhere in between, that is, in combined systems. In order to have something to combine, the relevant linguistic theories first
%need to be developed. As Manfred Pinkal said: ``It is not possible to build systems that understand language without understanding language.''


% lulu/wsun/sisi: DONE
%      <!-- Local IspellDict: en_US-w_accents -->
