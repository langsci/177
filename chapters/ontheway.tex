\section*{本书的出版过程}
%\section*{On the way this book is published}

我从1994年开始写我的毕业论文,并在1997年成功通过答辩。这一阶段的手稿可以在我的网页上获取。在答辩之后,我必须要找到出版商。我很高兴收到了Niemeyer的“语言学研究”系列丛书的邀请,但是同时我对价格感到震惊不已,当时每本书需要186德国马克,这还是在我没有出版商的任何帮助的情况下自己写书和排版(这个价格是纸版小说的二十倍)。\footnote{%
与此同时,Niemeyer被de Gruyter收购,并停止营业了。这本书的价格现在是139.95欧元 / 196.00美元。欧元的价格相当于273.72德国马克。
}这基本上意味着我的书是没有出版的:直到1998年,才能在我的网站上看到这本书,并随后在图书馆可以查询到。我的教授转正著作由CSLI出版社出版,价格相对来说合理多了。在我开始写教科书的时候,我就寻找不同的出版渠道,并跟无名印刷需求的出版社协商。Brigitte Narr负责Stauffenburg出版集团,她说服我在他们的出版社出版HPSG的教材。这本书的德语版属于我,这样我就可以在我的主页上出版。这一合作是成功的,由此我还可以跟Stauffenburg出版我的第二本关于语法理论的教科书。我想这本书具有更为广泛的相关性,并且可以供非德语的读者阅读。由此,我决定将它翻译为英语。不过,Stauffenburg重点出版德语书籍,我必须找到另一家出版社。幸运的是,出版界的情况与1997年相比发生了戏剧性的翻天覆地的变化:我们现在有高水平的出版社,不仅有严格的同行评审,还有着完全公开的途径。我很高兴Brigitte Narr将本书的版权卖回给我,我现在就可以在CC-BY版权下由语言科学出版社出版这本英文版教材了。
%I started to work on my dissertation in 1994 and defended it in 1997. During the whole time the
%manuscript was available on my web page. After the defense, I had to look for a publisher. I was
%quite happy to be accepted to the series \emph{Linguistische Arbeiten} by Niemeyer, but at the same time I
%was shocked about the price, which was 186.00 DM for a paperback book that was written and typeset
%by me without any help by the publisher (twenty times the price of a paperback novel).\footnote{%
 % As a side remark: in the meantime Niemeyer was bought by de Gruyter and closed down. The price of the book is now
 % 139.95 \euro / \$ 196.00. The price in Euro corresponds to 273.72 DM. 
%%This is a price increase of 47\,\%.
%} This
%basically meant that my book was depublished: until 1998 it was available from my web page and after
%%this it was available in libraries only. My Habilitationsschrift was published by CSLI Publications
%for a much more reasonable price. When I started writing textbooks, I was looking for alternative
%distribution channels and started to negotiate with no-name print on demand publishers. Brigitte Narr,
%who runs the Stauffenburg publishing house, convinced me to publish my HPSG textbook with her. The
%\textsc{cop}yrights for the German version of the book remained with me so that I could publish it on my web page. The collaboration was successful so that I also published my second textbook about
%grammatical theory with Stauffenburg. I think that this book has a broader relevance and should be
%accessible for non-German-speaking readers as well. I therefore decided to have it translated into
%English. Since Stauffenburg is focused on books in German, I had to look for another publisher. Fortunately the situation in the publishing sector changed quite dramatically in comparison
%to 1997: we now have high profile publishers with strict peer review that are entirely open access. I am very
%glad about the fact that Brigitte Narr sold the rights of my book back to me and that I can now 
%publish the English version with Language Science Press under a CC-BY license.

%      <!-- Local IspellDict: en_US-w_accents -->
