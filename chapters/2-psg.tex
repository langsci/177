%% -*- coding:utf-8 -*-

\chapter{短语结构语法}
%\chapter{Phrase structure grammar}
\label{Kapitel-PSG}
\isc{短语结构语法|(}\is{phrase structure grammar|(}

本章讨论短语结构语法,该语法在我们接下来要讲解的理论中起到了至关重要的作用。
%This chapter deals with phase structure grammars (PSGs), which play an important role in several of the theories we will encounter in later chapters.

\section{符号与重写规则}
%\section{Symbols and rewrite rules}

根据词的屈折形态和句法分布,我们可以判断出它们的词性。由此,例(\mex{1})中的weil(因为)是连词\isc{连词}\is{conjunction},而das(这)和dem(这)是冠词\isc{冠词}\is{article},后两者也叫做限定词\isc{限定词}\is{determiner}。进而,Buch(书)和Mann(人)是名词\isc{名词}\is{noun} ,而gibt(给)是动词\isc{动词}\is{verb}。
%Words can be assigned to a particular part of speech on the basis of their inflectional properties
%and syntactic distribution. Thus, \emph{weil} `because' in (\mex{1})
%is a conjunction\is{conjunction}, whereas \emph{das} `the' and \emph{dem} `the' are
%articles\is{article} and therefore classed as determiners\is{determiner}. Furthermore, \emph{Buch} `book' and \emph{Mann} `man' are nouns\is{noun} 
%and \emph{gibt} `gives' is a verb\is{verb}.
\ea\label{bsp-weil-er-das-buch-dem-mann-gibt}
\gll weil er das Buch dem Mann gibt\\
	 因为 他 \textsc{det} 书 \textsc{det} 人 给\\
\glt `因为他给人这本书'
\z
%\ea\label{bsp-weil-er-das-buch-dem-mann-gibt}
%\gll weil er das Buch dem Mann gibt\\
%	 because he the book the man gives\\
%\glt `because he gives the man the book'
%\z
在\ref{konstituententests}中,我们介绍了组成成分测试的若干方法。采用这些方法,我们可以判断出das Buch(这本书)和dem Mann(这个人)这两组字符串分别构成了组成成分。那么这些成分需要特定的符号来指称他们。因为在这些短语中,名词起到了重要的作用,它们就叫做“名词短语”(noun phrase)或简称为NP。因为代词er(他)也可以出现在完整的名词短语所在的位置中,所以代词也应归入NP这一类别中。
%Using the constituency tests we introduced in Section~\ref{konstituententests}, we can show that
%individual words as well as the strings \emph{das Buch} `the book' and \emph{dem Mann} `the man',
%form constituents. These get then assigned certain symbols. Since nouns form an important part of
%the phrases \emph{das Buch} and \emph{dem Mann}, these are referred to as \emph{noun phrases} or
%NPs, for short. The pronoun \emph{er} `he' can occur in the same positions as full NPs and can
%therefore also be assigned to the category NP.

短语结构语法用规则来说明符号是如何指派到某类词中,并且这些词是如何构成更复杂的单位的。如(\mex{1})所示,这是用来分析例句(\mex{0})的一个简单的短语结构语法:
\footnote{%
我们暂不分析weil(因为)。因为对于德语的动词位于第一位和动词位于第二位的小句的确切分析来说,我们还需要一些额外的假设,所以在这一章只讨论动词位于句末的小句。
}$^,$\footnote{\label{fn-np-pron-ps-rule}%
NP $\to$ er 这条规则看起来有些奇怪。我们本可以用PersPron $\to$ er这条规则来说明,但是这就需要另一条规则来说明能够代替整个NP的人称代词NP $\to$ PersPron。(\mex{1})将前面两条规则整合成一条,并且说明er(他)出现在名词短语可以出现的位置上。
}
%Phrase structure grammars come with rules specifying which symbols are assigned to certain kinds of words and how these are combined to create more
%complex units. A simple phrase structure grammar which can be used to analyze (\mex{0}) is given in (\mex{1}):\footnote{%
%	I ignore the conjunction \emph{weil} `because' for now. Since the exact analysis of
%        German verb"=first and verb"=second clauses requires a number of additional assumptions, we will restrict ourselves to verb"=final clauses in this chapter.
%}$^,$\footnote{\label{fn-np-pron-ps-rule}%
%	The rule NP $\to$ er may seem odd. We could assume the rule PersPron $\to$ er instead but then would have to posit a further rule which
%	would specify that personal pronouns can replace full NPs: NP $\to$ PersPron. The rule in (\mex{1}) combines the two aforementioned rules and states
%	that \emph{er} `he' can occur in positions where noun phrases can.
%}

\ea
\label{bsp-grammatik-psg}
\begin{tabular}[t]{@{}l@{ }l}
{NP} & {$\to$ Det N}\\          
{S}  & {$\to$ NP NP NP V}
\end{tabular}\hspace{2cm}%
\begin{tabular}[t]{@{}l@{ }l}
{NP} & {$\to$ er}\\
{Det}  & {$\to$ das}\\
{Det}  & {$\to$ dem}\\
\end{tabular}\hspace{8mm}
\begin{tabular}[t]{@{}l@{ }l}
{N} & {$\to$ Buch}\\
{N} & {$\to$ Mann}\\
{V} & {$\to$ gibt}\\
\end{tabular}
\z
由此,我们将NP $\to$\isc{$\to$}\is{$\to$} Det N这一规则解读为一个名词短语,这个名词短语被赋予了NP这个符号,并且包括一个限定词(D)和一个名词(N)。
%We can therefore interpret a rule such as NP $\to$\is{$\to$} Det N as meaning that a noun phrase, that is, something which is assigned the symbol NP, can consist
%of a determiner (Det) and a noun (N).

对于例(\mex{-1})这个句子,我们可以应用(\mex{0})中的文法并按照如下的方式来分析:
首先,我们选取句中的第一个词,然后看是否有规则来说明该词出现在规则的右边。如果有,那么我们就用规则左边的符号来替换它。这一过程可以参考(\mex{1})\vpageref*{bsp-anwendung-grammatik}中的第2—4,6—7和9行的推导过程。比如说,第二行中er被NP代替。如果在规则的右边有两个以上的符号,那么这些符号都被左边的符号所代替。这一过程可以参考第5、8和10行。例如,在第5行和第8行中,Det和N被重写为NP。
%We can analyze the sentence in (\mex{-1}) using the grammar in (\mex{0}) in the following way:
%first, we take the first word in the sentence and check if there is a rule in which this word occurs on the right"=hand
%side of the rule. If this is the case, then we replace the word with the symbol on the left"=hand side of the rule. This happens
%in lines 2--4, 6--7 and 9 of the derivation in (\mex{1})\vpageref*{bsp-anwendung-grammatik}. For
%instance, in line~2 \emph{er} is replaced by NP.
%If there are two or more symbols which occur together on the right"=hand side of a rule, then all
%these words are replaced with the symbol on the left. This happens in lines 5, 8 and 10. For
%instance, in line 5 and 8, Det and N are rewritten as NP.
%\begin{figure}
\ea
\label{bsp-anwendung-grammatik}
\begin{tabular}[t]{@{}r|llllll@{\hspace{1.7cm}}l@{}}
 & \multicolumn{6}{l@{}}{词和符号} & 应用的规则\\\midrule
 %& \multicolumn{6}{l@{}}{words and symbols} & rules that are applied\\\hline
 1 & er            & das          & Buch          & dem          & Mann & gibt                \\
 2 & {NP}          & das          & Buch          & dem          & Mann & gibt & {NP $\to$ er}  \\
 3 & NP            & Det          & Buch          & dem          & Mann & gibt & {Det $\to$ das}  \\
 4 & NP            & Det          & N             & dem          & Mann & gibt & {N $\to$ Buch} \\
 5 & NP            &              & NP            & dem          & Mann & gibt & {NP $\to$ Det N}\\
 6 & NP            &              & NP            & Det          & Mann & gibt & {Det $\to$ dem}  \\
 7 & NP            &              & NP            & Det          & N    & gibt & {N $\to$ Mann} \\
 8 & NP            &              & NP            &              & NP   & gibt & {NP $\to$ Det N}\\
 9 & NP            &              & NP            &              & NP   & {V} & {V $\to$ gibt}  \\
10 &               &              &               &              &      & {S} & {S $\to$ NP NP NP V}\\
\end{tabular}
\z
%\vspace{-\baselineskip}\end{figure}%
如例(\mex{0})所示,我们从一串词开始,然后按照给定的短语结构文法,我们可以推导出句子的结构。我们也可以按照相反的方向来分析:从句子标号S开始,我们可以应用第9—1步,最后到词串。如要从文法中选择重写符号的不同规则,我们可以用(\mex{-1})中的文法来从S分析到字符串“er dem Mann das Buch gibt”(他男人书给)。我们可以说,该文法允准(或生成)\label{Seite-generiert}了一组句子。\isc{生成语法}\is{Generative Grammar}
%In (\mex{0}), we began with a string of words and it was shown that we can derive the structure of a sentence by applying the rules of
%a given phrase structure grammar. We could have applied the same steps in reverse order: starting
%with the sentence symbol S, we would have applied the steps~9--1 and arrived at the string of words.
%Selecting different rules from the grammar for rewriting symbols, we could use the grammar in (\mex{-1}) to get
%from S to the string \emph{er dem Mann das Buch gibt} `he the man the book gives'.
%We can say that this grammar licenses (or generates)\label{Seite-generiert} a set of sentences.\is{Generative Grammar}

例(\mex{0})中的推导过程可以表示为一棵树,如~\vref{er-das-buch-dem-mann-gibt-flat}所示。
%例(\mex{0})中的推导过程可以表示为一棵树,如\chinesevref{图}{er-das-buch-dem-mann-gibt-flat}所示。
%The derivation in (\mex{0}) can also be represented as a tree. This is shown by Figure~\vref{er-das-buch-dem-mann-gibt-flat}.
% \clearpage
\begin{figure}
\centerline{
\begin{forest}
sm edges
[S
  [NP [er;他] ]
  [NP
    [Det [das;\textsc{det}] ]
    [N [Buch;书] ] 
  ]
  [NP
    [Det [dem;\textsc{det}] ]
    [N [Mann;人] ] 
  ]
  [V [gibt;给] ]
]
\end{forest}
%
%% \begin{tikzpicture}
%% \tikzset{level 1+/.style={level distance=3\baselineskip}}
%% \tikzset{level 2+/.style={level distance=2\baselineskip}}
%% \tikzset{frontier/.style={distance from root=8\baselineskip}}
%% \tikzset{every leaf node/.append style={text depth=0pt}}
%% \Tree[.S
%%        [.NP er\\he ]
%%        [.NP
%%          [.Det das\\the ]
%%          [.N Buch\\book ] ]
%%        [.NP
%%          [.Det der\\the ]
%%          [.N Frau\\woman ] ]
%%        [.V gibt\\gives ] ]
%% \end{tikzpicture}
}
\caption{\label{er-das-buch-dem-mann-gibt-flat}“er das Buch dem Mann gibt”(他这本书这个人给)的分析}
%\caption{\label{er-das-buch-dem-mann-gibt-flat}Analysis of \emph{er das Buch dem Mann gibt} `he the
%book the woman gives'}
\end{figure}%

树中的符号叫做“结点”(node)\isc{结点}\is{node}。S直接支配NP结点和V结点\isc{支配}\is{dominance}。树中的其他结点也被支配,只不过不是受到S的直接支配\isc{支配!直接支配}\is{dominance!immediate}。如果我们想讨论结点之间的关系,我们通常会用亲属词来表示。在图\ref{er-das-buch-dem-mann-gibt-flat}中,S是三个NP结点和V结点的“父结点”(mother
node)\isc{结点!父结点}\is{node!mother}。NP结点和V结点是“兄弟结点”\isc{结点!兄弟结点}\is{node!sister},因为他们有共同的父结点。如果一个结点有两个“子结点”\isc{结点!子结点}\is{node!daughter},那么我们就会得到一个二叉\isc{二元}\is{binary}结构\isc{分支!二叉}\is{branching!binary}。如果只有一个子结点,则会得到一个单分支结构\isc{分支!单叉}\is{branching!unary}。如果有两个成分是直接相连的,我们就说它们具有“邻接性”(adjacent)\isc{邻接}\is{adjacency}。
%The symbols in the tree are called \emph{nodes}\is{node}. We say that S immediately dominates the NP nodes and the V node\is{dominance}.
%The other nodes in the tree are also dominated, but not immediately dominated, by S\is{dominance!immediate}. If we want to talk about the
%relationship between nodes, it is common to use kinship terms. In Figure~\ref{er-das-buch-dem-mann-gibt-flat}, S is the \emph{mother node}\is{node!mother}
%of the three NP nodes and the V node.
%The NP node and V are \emph{sisters}\is{node!sister} since they have the same mother node.
%If a node has two daughters\is{node!daughter}, then we have a binary\is{binary} branching structure\is{branching}\is{branching!binary}.
%If there is exactly one daughter, then we have a unary\is{unary}\is{branching!unary} branching
%structure. Two constituents are said to be \emph{adjacent}\is{adjacency}
%if they are directly next to each other.

语言学著作中通常会省略短语结构规则。相反,作者们倾向于使用树图或者紧缩的框式结构,如(\mex{1})所示。
%Phrase structure rules are often omitted in linguistic publications. Instead, authors opt for tree diagrams or the compact equivalent bracket notation
%such as (\mex{1}).
\ea
\gll {}[\sub{S} [\sub{NP} er] [\sub{NP} [\sub{Det} das] [\sub{N} Buch]]  [\sub{NP} [\sub{Det} dem] [\sub{N} Mann]] [\sub{V} gibt]]\\
     {}         {}       他  {}        {}       \textsc{det}  {}      书    {}        {}       \textsc{det}  {}       人     {}      给\\  
%      {}         {}        he  {}        {}       the  {}       book    {}        {}       the  {}       man     {}      gives\\ 
\z
无论采取哪种形式,文法规则是最重要的,因为这些规则表示了语法知识,而这是与具体的结构没有太多关系的。这样,我们可以利用(\ref{bsp-grammatik-psg})中的文法来剖析或生成例(\mex{1}),该句中的宾语顺序与前面例(\ref{bsp-weil-er-das-buch-dem-mann-gibt})的不同:
%Nevertheless, it is the grammatical rules which are actually important since these represent grammatical knowledge which is independent of specific structures.
%In this way, we can use the grammar in (\ref{bsp-grammatik-psg}) to parse or generate the sentence
%in (\mex{1}), which differs from (\ref{bsp-weil-er-das-buch-dem-mann-gibt}) in the order of objects: 
\ea
\gll {}[weil] er dem Mann das Buch gibt\\
	 {}\spacebr{}因为 他.\nom{} \textsc{det}.\dat{} 人 \textsc{det}.\acc{} 书 给\\
\glt `因为他给这个人这本书'
\z
%\ea
%\gll {}[weil] er dem Mann das Buch gibt\\
%	 {}\spacebr{}because he.\nom{} the.\dat{} man the.\acc{} book gives\\
%\glt `because he gives the man the book'
%\z
与例(\ref{bsp-weil-er-das-buch-dem-mann-gibt})相比,这句话中用来替代限定词和名词的规则采用了不同的顺序。具体来说,这里并不是用das(这)来替代第一个Det,也不是用Buch(书)来替代第一个名词,而是用dem(这)来替代第一个Det,并用Mann(人)替代第一个名词。
%The rules for replacing determiners and nouns are simply applied in a different order than in (\ref{bsp-weil-er-das-buch-dem-mann-gibt}). Rather than replacing the first Det with \emph{das} %`the' and the first noun with \emph{Buch} `book', the first Det is replaced with \emph{dem} `the' and the first noun with \emph{Mann}.

在此,我需要指出的是,(\ref{bsp-grammatik-psg})中的文法并不是针对(\ref{bsp-weil-er-das-buch-dem-mann-gibt})中的例句的唯一一套文法。实际上,对于这类句子(参看练习\ref{ua-psg-eins})而言,还有无限多\label{page-unendlich-viele-grammatiken}可能的文法。另一套文法如例(\mex{1})所示:
%At this juncture, I should point out that the grammar in (\ref{bsp-grammatik-psg}) is not the only possible grammar for the example sentence in
%(\ref{bsp-weil-er-das-buch-dem-mann-gibt}). There is an infinite\label{page-unendlich-viele-grammatiken} number of possible grammars which could
%be used to analyze these kinds of sentences (see exercise \ref{ua-psg-eins}). Another possible grammar is given in (\mex{1}):

\ea\label{psg-binaer}
\begin{tabular}[t]{@{}l@{ }l@{}}
NP & $\to$ Det N  \\
V  & $\to$ NP V\\
\end{tabular}\hspace{2cm}%
\begin{tabular}[t]{@{}l@{ }l}
{NP}  & {$\to$ er}\\
{Det} & {$\to$ das}\\
{Det} & {$\to$ dem}\\
\end{tabular}\hspace{8mm}
\begin{tabular}[t]{@{}l@{ }l}
{N} & {$\to$ Buch}\\
{N} & {$\to$ Mann}\\
{V} & {$\to$ gibt}\\
\end{tabular}
\z
该文法允准了如图\vref{er-das-buch-dem-mann-gibt-bin}所示的二叉树结构。
%This grammar licenses binary branching structures as shown in Figure~\vref{er-das-buch-dem-mann-gibt-bin}.
\begin{figure}
\centerline{
\begin{forest}
sm edges
[V
  [NP [er;他] ]
  [V
    [NP
      [Det [das;\textsc{det}] ]
      [N [Buch;书] ] ]
    [V
      [NP
        [Det [dem;\textsc{det}] ]
        [N [Mann;人] ] ]
      [V [gibt;给] ] ] ] ]
\end{forest}
}
\caption{\label{er-das-buch-dem-mann-gibt-bin}“er das Buch dem Mann gibt”(他这本书这个人给)的二叉树结构分析}
%\caption{\label{er-das-buch-dem-mann-gibt-bin}Analysis of \emph{er das Buch dem Mann gibt} with a
%  binary branching structure}
\end{figure}%

实际上,(\mex{0})和(\ref{bsp-grammatik-psg})中的文法都不够精确。如果我们在文法中加入额外的词条,如ich(我)和den(\textsc{det}的宾格形式),那么就会生成不合乎语法的句子,如例(\mex{1}b--d)所示:\footnote{%
应用(\ref{psg-binaer})中的文法,我们还会遇到别的问题,即我们无法确定什么时候话语是完整的,因为符号V被用来表示V和NP的所有组合。由此,我们也可以用这个文法来分析(i)中的句子:
%Both the grammar in (\mex{0}) and (\ref{bsp-grammatik-psg}) are too imprecise.
%If we adopt additional lexical entries for \emph{ich} `I' and \emph{den} `the' (accusative) in our grammar, then we would incorrectly
%license the ungrammatical sentences in (\mex{1}b--d):\footnote{%
%	With the grammar in (\ref{psg-binaer}), we also have the additional problem that we cannot determine when an utterance is complete
%	since the symbol V is used for all combinations of V and NP. Therefore, we can also analyze the sentence in (i) with this grammar:
  
\eal
\ex[*]{
\gll der Mann erwartet\\
     \textsc{det} 人 希望\\
}
\ex[*]{
\gll des        Mannes     er        das Buch dem Mann gibt\\
     \textsc{det}.\gen{} 人.\gen{} 他.\nom{} \textsc{det}.\acc{} 书 \textsc{det}.\dat{} 人 给\\
}
\zl
动词所带的论元数量应该在文法中显示出来。在下面的章节中,我们将介绍动词对论元的选择(价)在不同的语法理论中是如何表述的。
%The number of arguments required by a verb must be somehow represented in the grammar. In the following chapters, we will see exactly
%how the selection of arguments by a verb (valence) can be captured in various grammatical theories.
}
\eal
\ex[]{
\gll er        das Buch        dem Mann       gibt\\
     他.\nom{} \textsc{det}.\acc{} 书 \textsc{det}.\dat{} 人 给\\
\glt `他把书给这个人。'
}
\ex[*]{
\gll ich      das Buch        dem Mann       gibt\\
     我.\nom{} \textsc{det}.\acc{} 书 \textsc{det}.\dat{} 人 给\\
}
\ex[*]{
\gll er        das        Buch den        Mann gibt\\
     他.\nom{} \textsc{det}.\acc{} 书 \textsc{det}.\acc{} 人 给\\
}
\ex[*]{
\gll er        den        Buch       dem Mann gibt\\
     他.\nom{} \textsc{det}.\mas{} 书(\neu) \textsc{det} 人  给\\
}
\zl
例(\mex{0}b)违反了主谓一致原则\isc{主谓一致|(}\is{agreement|(} ,具体来说:ich(我)和gibt(给)并不搭配。例(\mex{0}c)是不合乎语法的,因为没有满足动词对格的约束条件。最后,例(\mex{0}d)也是不合乎语法的,因为在限定词和名词之间缺少一致关系。我们不能将den(这)这个阳性的宾格成分与Buch(书)这个中性词搭配。基于上述原因,这两个成分的性属性不同,所以说,二者不能搭配。
%In (\mex{0}b), subject"=verb agreement\is{agreement|(} has been violated, in other words: \emph{ich} `I' and \emph{gibt} `gives' do not fit together.
%(\mex{0}c) is ungrammatical because the case requirements of the verb have not been satisfied: \emph{gibt} `gives' requires a dative object. Finally, (\mex{0}d) is ungrammatical
%because there is a lack of agreement between the determiner and the noun. It is not possible to combine \emph{den} `the', which is masculine and bears accusative case, 
%and \emph{Buch} `book' because \emph{Buch} is neuter gender. For this reason, the gender properties
%of these two elements are not the same and the elements can therefore not be combined.

在下面的内容中,我们将思考如何让我们的语法可以杜绝生成例(\mex{0}b--d)中的句子。如果我们想说明主谓一致关系,那么我们就需要说明德语中的六种格,并且动词必须与主语在人称(1、2、3)\isc{人称}\is{person} 和数(sg(单数)、pl(复数))\isc{数}\is{number}上保持一致\isc{主谓一致}\is{agreement}:
%In the following, we will consider how we would have to change our grammar to stop it from licensing the sentences in (\mex{0}b--d).
%If we want to capture subject"=verb agreement, then we have to cover the following six cases in German, as the verb has to agree with the
%subject in both person\is{person} (1, 2, 3) and number\is{number} (sg, pl)\is{agreement}:
\eal\jamwidth=8cm\relax%\settowidth\jamwidth{(3, sg)}
\ex 
\gll Ich schlafe.\\
     我 睡觉\\      \jambox{(1,sg)}
\ex 
\gll Du schläfst.\\
     你 睡觉\\      \jambox{(2,sg)}
\ex 
\gll Er schläft.\\
     他 睡觉\\      \jambox{(3,sg)}
\ex 
\gll Wir schlafen.\\
     我们 睡觉\\       \jambox{(1,pl)}
\ex 
\gll Ihr schlaft.\\
     你们 睡觉\\       \jambox{(2,pl)}
\ex 
\gll Sie schlafen.\\   
     他们 睡觉\\      \jambox{(3,pl)}
\zl
我们可以通过增加符号的数量来用语法规则描述这些关系。我们用下面的符号来替换S $\to$ NP NP NP V:
%It is possible to capture these relations with grammatical rules by increasing 
%the number of symbols we use. Instead of the rule S $\to$ NP NP NP V, we can use
%the following:
\ea
\begin{tabular}[t]{@{}l@{ }l@{~~}l@{~~}l@{~~}l}
S  & $\to$ NP\_1\_sg & NP & NP & V\_1\_sg\\
S  & $\to$ NP\_2\_sg & NP & NP & V\_2\_sg\\
S  & $\to$ NP\_3\_sg & NP & NP & V\_3\_sg\\
S  & $\to$ NP\_1\_pl & NP & NP & V\_1\_pl\\
S  & $\to$ NP\_2\_pl & NP & NP & V\_2\_pl\\
S  & $\to$ NP\_3\_pl & NP & NP & V\_3\_pl\\
\end{tabular}
\z
这样就意味着,我们需要六种不同的符号来分别表示名词短语和动词,还需要六条规则,而不是一条。
%This would mean that we need six different symbols for noun phrases and verbs respectively, as well as six rules rather than one.

为了说明动词的格指派关系,我们可以按照类似的方式将格信息整合到符号中。这样,我们可以得到如下的规则:
%In order to account for case assignment by the verb, we can incorporate case information into the symbols in an analogous way. We would then
%get rules such as the following:
\ea
\label{ditrans-ps-regeln}
\begin{tabular}[t]{@{}l@{ }l@{~~}l@{~~}l@{~~}l}
S  & $\to$ NP\_1\_sg\_nom & NP\_dat & NP\_acc & V\_1\_sg\_nom\_dat\_acc\\
S  & $\to$ NP\_2\_sg\_nom & NP\_dat & NP\_acc & V\_2\_sg\_nom\_dat\_acc\\
S  & $\to$ NP\_3\_sg\_nom & NP\_dat & NP\_acc & V\_3\_sg\_nom\_dat\_acc\\
S  & $\to$ NP\_1\_pl\_nom & NP\_dat & NP\_acc & V\_1\_pl\_nom\_dat\_acc\\
S  & $\to$ NP\_2\_pl\_nom & NP\_dat & NP\_acc & V\_2\_pl\_nom\_dat\_acc\\
S  & $\to$ NP\_3\_pl\_nom & NP\_dat & NP\_acc & V\_3\_pl\_nom\_dat\_acc\\
\end{tabular}
\z
考虑到名词需要区分四种不同的格,我们给主格NP设置了六种符号,其他格的NP需要三种符号。由于动词需要与NP相搭配,我们还需要区分(\mex{1})中带三个论元,或者只带一个或两个论元的动词,所以说,我们需要增加表示动词的符号数量。\isc{价}\is{valence}
%Since it is necessary to differentiate between noun phrases in four cases, we have a total of six symbols for NPs in the nominative and three symbols for NPs with
%other cases. Since verbs have to match the NPs, that is, we have to differentiate between verbs which select three arguments and those selecting only one or two (\mex{1}),
%we have to increase the number of symbols we assume for verbs.\is{valence}
\eal
\ex[]{
\gll Er schläft.\\
	 他 睡觉\\
\glt `他正在睡觉。'
}
\ex[*]{
\gll Er schläft das Buch.\\
	 他 睡觉 \textsc{det} 书\\
}
\ex[]{
\gll Er kennt das Buch.\\
	 他 认识 \textsc{det} 书\\
\glt `他认识这本书。'
}
\ex[*]{
\gll Er kennt.\\
	 他 认识\\
}
\zl
%\eal
%\ex[]{
%\gll Er schläft.\\
%	 he sleeps\\
%\glt `He is sleeping.'
%}
%\ex[*]{
%\gll Er schläft das Buch.\\
%	 he sleeps the book\\
%}
%\ex[]{
%\gll Er kennt das Buch.\\
%	 he knows the book\\
%\glt `He knows the book.'
%}
%\ex[*]{
%\gll Er kennt.\\
%	 he knows\\
%}
%\zl
在上面的规则中,有关动词所需论元数量的信息在“nom\_dat\_acc”这个标记中有所说明。
%In the rules above, the information about the number of arguments required by a verb is included in the marking `nom\_dat\_acc'.

为了说明例(\mex{1})中的限定词与名词的一致关系,我们需要整合性\isc{性}\is{gender}(阴性、阳性、中性)、数\isc{数}\is{number}(单数、复数)、格\isc{格}\is{case}(主格、属格、与格、宾格),以及屈折类型(强、弱)的信息\footnote{%
这些是形容词的屈折变化类型,它们也同样适用于某些名词,如Beamter(公务员)、Verwandter(亲戚)和Gesandter(公使)。有关形容词类型的更多信息请参考第\pageref{page-Flexionsklasse-Wunderlich}页。
}。
%In order to capture the determiner"=noun agreement in (\mex{1}), we have to incorporate information about gender\is{gender} (fem, mas, neu),
%number\is{number} (sg, pl), case\is{case} (nom, gen, dat, acc) and the inflectional classes (strong, weak)\footnote{%
%These are inflectional classes for adjectives which are also relevant for some nouns such as \emph{Beamter} `civil servant', 
%\emph{Verwandter} `relative', \emph{Gesandter} `envoy'.
%For more on adjective classes see page~\pageref{page-Flexionsklasse-Wunderlich}.%
%}.
\eal\settowidth\jamwidth{(屈折类型)}
\ex 
\gll der Mann, die Frau, das Buch\\
	 \textsc{det}.\mas{} 人(\mas) \textsc{det}.\fem{} 女人(\fem) \textsc{det}.\neu{} 书(\neu)\\\jambox{(性)}
\ex 
\gll das Buch, die Bücher\\
	 \textsc{det} 书.\sg{} \textsc{det} 书.\pl\\\jambox{(数)}
\ex 
\gll des Buches, dem Buch\\
	 \textsc{det}.\gen{} 书.\gen{} \textsc{det}.\dat{} 书\\\jambox{(格)}
\ex\isc{屈折类型}\is{inflectional class} 
\gll ein Beamter, der Beamte\\
	 一 公务员 \textsc{det} 公务员\\\jambox{(屈折类型)}
\zl
%\eal\settowidth\jamwidth{(Inflectional class)}
%\ex 
%\gll der Mann, die Frau, das Buch\\
%	 the.\mas{} man(\mas) the.\fem{} woman(\fem) the.\neu{} book(\neu)\\\jambox{(gender)}
%\ex 
%\gll das Buch, die Bücher\\
%	 the book.\sg{} the books.\pl\\\jambox{(number)}
%\ex 
%\gll des Buches, dem Buch\\
%	 the.\gen{} book.\gen{} the.\dat{} book\\\jambox{(case)}
%\ex\is{inflectional class} 
%\gll ein Beamter, der Beamte\\
%	 a civil.servant the civil.servant\\\jambox{(inflectional class)}
%\zl
我们不用NP $\to$ Det N这条规则,我们将要用到下面这些规则,如例(\mex{1})所示:\footnote{%
为了方便,这些规则没有包括屈折类型相关的信息。
}  
%Instead of the rule NP $\to$ Det N, we will have to use rules such as those in (\mex{1}):\footnote{%
%  To keep things simple, these rules do not incorporate information regarding the inflection class.
%}
\ea
%\resizebox{\linewidth}{!}{
\begin{tabular}[t]{@{}l@{ }l@{~~}l}
NP\_3\_sg\_nom  & $\to$ Det\_fem\_sg\_nom & N\_fem\_sg\_nom \\
NP\_3\_sg\_nom  & $\to$ Det\_mas\_sg\_nom & N\_mas\_sg\_nom \\
NP\_3\_sg\_nom  & $\to$ Det\_neu\_sg\_nom & N\_neu\_sg\_nom \\
NP\_3\_pl\_nom  & $\to$ Det\_fem\_pl\_nom & N\_fem\_pl\_nom \\
NP\_3\_pl\_nom  & $\to$ Det\_mas\_pl\_nom & N\_mas\_pl\_nom \\
NP\_3\_pl\_nom  & $\to$ Det\_neu\_pl\_nom & N\_neu\_pl\_nom \\[2mm]
\end{tabular}

\begin{tabular}[t]{@{}l@{ }l@{~~}l}
NP\_3\_sg\_nom  & $\to$ Det\_fem\_sg\_nom & N\_fem\_sg\_nom \\
NP\_3\_sg\_nom  & $\to$ Det\_mas\_sg\_nom & N\_mas\_sg\_nom \\
NP\_3\_sg\_nom  & $\to$ Det\_neu\_sg\_nom & N\_neu\_sg\_nom \\
NP\_3\_pl\_nom  & $\to$ Det\_fem\_pl\_nom & N\_fem\_pl\_nom \\
NP\_3\_pl\_nom  & $\to$ Det\_mas\_pl\_nom & N\_mas\_pl\_nom \\
NP\_3\_pl\_nom  & $\to$ Det\_neu\_pl\_nom & N\_neu\_pl\_nom \\[2mm]
\end{tabular}
\z
例(\mex{0})说明了主格名词短语的规则。我们还需要针对属格、与格和宾格的类似规则。这样,我们就需要24个符号来说明限定词($3*2*4$),24个符号说明名词,以及24条规则,而不是一条规则。如果要考虑屈折类型的话,符号的数量和规则的数量都要翻倍。\isc{主谓一致|)}\is{agreement|)}
%(\mex{0}) shows the rules for nominative noun phrases. We would need analogous rules for genitive,
%dative, and accusative. We would then require 24 symbols for determiners ($3*2*4$), 24 symbols for nouns and
%24 rules rather than one. If inflection class is taken into account, the number of symbols and the
%number of rules doubles.\is{agreement|)} 

\section{短语结构语法中特征的运用}
%\section{Expanding PSG with features}
\label{sec-PSG-Merkmale}

短语结构语法如果只用原子式的符号是有问题的,因为他们无法描写一些特定的扩展形式。语言学家能够辨识出NP\_3\_sg\_nom指的是名词短语,因为这里面包括字母NP。但是,在形式化的术语中,这个符号与文法中的其他任意一个符号是一样的。我们并不能找到可以用来指代NP的所有符号的共同特征。进而,非结构化的符号无法捕捉到这样的事实,即例(\mex{0})中的规则是有共同点的。在形式化的术语中,规则之间的唯一共性在于规则左边有一个符号,而右边有两个。
%Phrase structure grammars which only use atomic symbols are problematic as they cannot capture certain generalizations.
%We as linguists can recognize that NP\_3\_sg\_nom stands for a noun phrase because it contains the letters NP. 
%However, in formal terms this symbol is just like any other symbol in the grammar and we cannot capture the commonalities
%of all the symbols used for NPs. Furthermore, unstructured symbols do not capture the fact that the rules in (\mex{0}) 
%all have something in common. In formal terms, the only thing that the rules have in common is that there is one symbol on the
%left"=hand side of the rule and two on the right.

我们可以引进特征来解决这一问题。我们给范畴符号指派一些特征,并且允许我们的规则将这些特征的值包含进来。举例来说,我们可以假设范畴符号NP具有人称、数和格的特征。对于限定词和名词来说,我们可以采用一条附加的特征来表示性,另一条特征来表示屈折类型。例(\mex{1})给出了两条规则,并用括号分别表示出特征值:\footnote{%
第\ref{chap-feature-descriptions}章介绍了属性值结构。这些结构总是包括一对属性名称和属性值。在这样的情况下,值的顺序就不重要了,因为每个值都是独一无二的,并由相应的特征名称来指定。因为我们在(\mex{0})这样的格式中没有属性名称,所以值的顺序是重要的。
}
%We can solve this problem by introducing features which are assigned to category symbols and therefore allow for the values of
%such features to be included in our rules. For example, we can assume the features person, number and case for the category
%symbol NP. For determiners and nouns, we would adopt an additional feature for gender and one for
%inflectional class. (\mex{1}) shows two rules augmented by the respective values in brackets:\footnote{%
%  Chapter~\ref{chap-feature-descriptions} introduces attribute value structures. In these structure we always
%have pairs of a feature name and a feature value. In such a setting, the order of values is not
% important, since every value is uniquely identified by the corresponding feature name. Since we do not have a feature name
%in schemata like (\mex{0}), the order of the values is important.
%}

\ea
\begin{tabular}[t]{@{}l@{ }l}
NP(3,sg,nom)  & $\to$ Det(fem,sg,nom) N(fem,sg,nom)\\
NP(3,sg,nom)  & $\to$ Det(mas,sg,nom) N(mas,sg,nom)\\
\end{tabular}
\z
如果我们在例(\mex{0})中要用到变量而不是值,那么我们就会得到例(\mex{1})中的规则格式:
%If we were to use variables rather than the values in (\mex{0}), we would get rule schemata as the
%one in (\mex{1}):
\ea
\label{Regel-mit-Variablen}
\begin{tabular}[t]{@{}l@{ }l@{ }l}
NP({3},{Num},{Case}) & $\to$ & Det(Gen,{Num},{Case}) N(Gen,{Num},{Case})\\
\end{tabular}
\z
这里变量的值并不重要,重要的是他们是搭配的。为了保证规则的有效性,这些值要按照顺序排列是非常重要的;也就是说,在限定词的范畴中,性永远排在第一位,数第二,其他往后排。根据规则,人称特征的值位于NP(3,Num,Case)的第一个位置上,这里锁定为第三人称。当然,值的这种限制也可以由词汇来决定:
%The values of the variables here are not important. What is important is that they match. For this
%to work, it is important that the values are ordered; that is, in the category of a determiner, the gender is always first, number
%second and so on. The value of the person feature (the first position in the NP(3,Num,Case)) is fixed at `3' by the rule. These
%kind of restrictions on the values can, of course, be determined in the lexicon: 
\ea
\begin{tabular}[t]{@{}l@{ }l}
NP(3,sg,nom)  & $\to$ es\\
Det(mas,sg,nom)  & $\to$ des\\
\end{tabular}
\z

\noindent
(\ref{ditrans-ps-regeln})中的规则可以如例(\mex{1})所示,归并到一条范式中:
%The rules in (\ref{ditrans-ps-regeln})  can be collapsed into a single schema as in (\mex{1}):
\ea
\label{ditrans-schema}
\begin{tabular}[t]{@{}l@{ }l@{ }l}
S  & $\to$ & NP({Per1},{Num1},{nom}) \\
   &       & NP(Per2,Num2,{dat})\\
   &       & NP(Per3,Num3,{acc})\\
   &       & V({Per1},{Num1},ditransitive)\\
\end{tabular}
\z
Per1和Num1在动词和主语中的实现保证了主谓一致。对于其他NP来说,这些特征的值是无关的。这些NP的格也是明确的。
%The identification of Per1 and Num1 on the verb and on the subject ensures that there is subject"=verb agreement.
%For the other NPs, the values of these features are irrelevant. The case of these NPs is explicitly determined.
\isc{短语结构语法|)}\is{phrase structure grammar|)}

\section{语义}
%\section{Semantics}
\label{sec-PSG-Semantik}

在导言和前面几个章节中,我们主要分析了语言的句法方面,而且这本书后面的部分也主要探讨句法问题。
但是,有必要提醒大家的是,我们是为了交流而使用语言的,也就是说,针对某些场景、话题或者观点来交换信息。
如果我们想准确地解释我们的语言能力,那么我们还要解释我们所说的话的意义。
为了达到这一目标,我们就有必要理解句法结构,但是单是这样是不够的。进而,那些只关注句法结构的习得的语言习得理论也是不全面的。
由此,句法语义接口(syntax"=semantics interface)\isc{句法语义接口}\is{syntax"=semantics interface}问题的重要性尤为突出,每个语法理论都要说明句法和语义是如何互动的。
在下面,我将讲解我们是如何将短语结构规则与语义信息相结合的。为了表示意义,我会用一阶谓词逻辑和$\lambda$-演算\isc{$\lambda$-演算}\is{$\lambda$"=calculus}。不过遗憾的是,我们无法介绍基础逻辑知识的详细信息,以帮助没有相关经验的读者来理解下面的内容,但是我们这里举的这些简单的例子应该可以为句法和语义的互动关系提供一些基本的认识,进而得到一个能够对它进行解释的语言学理论。
%In the introductory chapter and the previous sections, we have been dealing with syntactic aspects
%of language and the focus will remain very much on syntax for the remainder of this book. It is,
%however, important to remember that we use language to communicate, that is, to transfer information
%about certain situations, topics or opinions. If we want to accurately explain our capacity for
%language, then we also have to explain the meanings that our utterances have. To this end, it is
%necessary to understand their syntactic structure, but this alone is not enough. Furthermore,
%theories of language acquisition that only concern themselves with the acquisition of syntactic
%constructions are also inadequate. The syntax"=semantics interface\is{syntax"=semantics interface}
%is therefore important and every grammatical theory has to say something about how syntax and
%semantics interact. In the following, I will show how we can combine phrase structure rules with
%semantic information. To represent meanings, I will use first"=order predicate logic and
%$\lambda$"=calculus\is{$\lambda$"=calculus}. Unfortunately, it is not possible to provide a detailed
%discussion of the basics of logic so that even readers without prior knowledge can follow all the
%details, but the simple examples discussed here should be enough to provide some initial
%insights into how syntax and semantics interact and furthermore, how we can develop a linguistic
%theory to account for this.

为了显示句子的意义是如何从它的组成部分推导出来的,我们来看一下例(\mex{1}a)。我们将例(\mex{1}b)中的意义指派给例(\mex{1}a)这个句子。
%To show how the meaning of a sentence is derived from the meaning of its parts, we will consider (\mex{1}a). We
%assign the meaning in (\mex{1}b) to the sentence in (\mex{1}a). 
\eal
\ex\label{Bsp-Max-schlaeft}
\gll Max schläft.\\
     Max 睡觉\\
\glt `Max正在睡觉。'
%\glt `Max is sleeping.'
\ex\label{Bsp-schlafen-max} 
\relation{schlafen}(\relation{max})
\zl
这里,我们假定\relation{schlafen}表示了schläft(睡觉)的含义。我们用初始符号来表示我们处理的是词义而不是实在的词。乍看上去,我们用\relation{schlafen}来表示(\mex{1}a)的意义并没有太大的变化,因为它只不过是动词schläft(睡觉)的另一种形式。但是,我们有必要集中在一个单一的动词形式上,因为屈折变化与意义是无关的。我们可以通过比较例(\mex{1}a)和(\mex{1}b)中的句子来对此进行分析:
%Here, we are assuming \relation{schlafen} to be the meaning of  \emph{schläft} `sleeps'. We use prime symbols to indicate
%that we are dealing with word meanings and not actual words. At first glance, it may not seem that we have really gained anything
%by using \relation{schlafen} to represent the meaning of (\mex{0}a), since it is just another form of the verb \emph{schläft} `sleeps'.
%It is, however, important to concentrate on a single verb form as inflection is irrelevant when it comes to meaning. We can see this by comparing the 
%examples in (\mex{1}a) and (\mex{1}b):
\eal
\ex 
\gll Jeder Junge schläft.\\
     每个 男孩 睡觉\\
\glt `每个男孩都在睡觉。'
\ex 
\gll Alle Jungen schlafen.\\
     所有 男孩 睡觉\\
\glt `所有的男孩都在睡觉。'	 
\zl
%\eal
%\ex 
%\gll Jeder Junge schläft.\\
%     every boy sleeps\\
%\glt `Every boy sleeps.'
%\ex 
%\gll Alle Jungen schlafen.\\
   %  all boys sleep\\
%\glt `All boys sleep.'	 
%\zl

\noindent
为了提高可读性,我从现在开始在语义表示中使用谓词的英语释义。\footnote{%
需要注意的是,我并不是说英语适合表示语言的语义关系与概念,其实它们也可以用其他语言来表示。
}
%To enhance readability I use English translations of the predicates in semantic representations
%from now on.\footnote{%
%  Note that I do not claim that English is suited as representation language for semantic relations
 % and concepts that can be expressed in other languages.
%}
所以说,(\mex{-1}a)的意义表示为(\mex{1}),而不是(\mex{-1}b):
%So the meaning of (\mex{-1}a) is represented as (\mex{1}) rather then (\mex{-1}b):
\ea
\label{sleep-max}
\relation{sleep}(\relation{max})
\z
当我们分析(\mex{-0})的意义时,我们可以看出每个词都表示了哪部分含义。凭直觉来看, \relation{max}来自Max。但是棘手的问题是,schläft(睡觉)贡献了哪些含义。如果我们考虑一个“睡觉”事件的特征的话,那么我们就会知道典型的情况是有一个人在睡觉。这个信息属于动词schlafen(睡觉)的含义的一部分。但是,动词的含义并不包括睡觉的个体,因为这个动词可以跟不同的主语搭配:
%When looking at the meaning in (\mex{-0}), we can consider which part of the meaning comes from each word.
%It seems relatively intuitive that \relation{max} comes from \emph{Max}, but the trickier question is what exactly
%\emph{schläft} `sleeps' contributes in terms of meaning. If we think about what characterizes a `sleeping' event, we
%know that there is typically an individual who is sleeping. This information is part of the meaning of the verb \emph{schlafen}
%`to sleep'. The verb meaning does not contain information about the sleeping individual, however, as this verb can be used
%with various subjects:
 \eal
\ex 
\gll Paul schläft.\\
     Paul 睡觉\\
\glt `Paul正在睡觉。'
\ex 
\gll Mio schläft.\\
     Mio 睡觉\\
\glt `Mio正在睡觉。'
\ex 
\gll Xaver schläft.\\
     Xaver 睡觉\\
\glt `Xaver正在睡觉。'
\zl
% \eal
%\ex 
%\gll Paul schläft.\\
%     Paul sleeps\\
%\glt `Paul is sleeping.'
%\ex 
%\gll Mio schläft.\\
%     Mio sleeps\\
%\glt `Mio is sleeping.'
%\ex 
%\gll Xaver schläft.\\
%     Xaver sleeps\\
%\glt `Xaver is sleeping.'
%\zl
所以说,我们可以抽象出\relation{sleep}的任意一种具体用法。相反,我们用变量(如$x$)来表示,如(\mex{-2}b)中的\relation{max}。这个$x$可以在指定的句子中被替换为\relation{paul}、\relation{mio}或\relation{xaver}。为了保证我们能在给定的含义中接触到这些变量,我们在它们前面写上$\lambda$。这样,schläft(睡觉)的含义可以表示如下:
%We can therefore abstract away from any specific use of \relation{sleep} and instead of, for example, \relation{max} in (\mex{-2}b), we
%use a variable (\eg $x$). This $x$ can then be replaced by \relation{paul}, \relation{mio} or \relation{xaver} in a given sentence. To allow us
%to access these variables in a given meaning, we can write them with a $\lambda$ in front. Accordingly, \emph{schläft} `sleeps' will have
%the following meaning:
\ea
$\lambda x~\relation{sleep}(x)$
\z
%
从(\ref{sleep-max})到(\mex{0})的步骤叫做“$\lambda$"=抽象”(lambda abstraction)\isc{lambda-抽象@$\lambda$-抽象}\is{lambda"=abstraction@$\lambda$"=abstraction}。(\mex{0})这个表达式与它的论元的意义整合过程是按照下面的方式进行的:我们去除$\lambda$和相应的变量,然后将变量的所有实例替换为论元的意义。如果我们将(\mex{0})和(\mex{1})中的\relation{max}整合在一起,我们就会得到(\ref{Bsp-schlafen-max})中的含义。
%The step from (\ref{sleep-max}) to (\mex{0}) is referred to as \emph{lambda abstraction}\is{lambda"=abstraction@$\lambda$"=abstraction}.
%The combination of the expression (\mex{0}) with the meaning of its arguments happens in the following way: we remove the $\lambda$ and the 
%corresponding variable and then replace all instances of the variable with the meaning of the
%argument. If we combine (\mex{0}) and \relation{max} as in (\mex{1}),
%we arrive at the meaning in (\ref{Bsp-schlafen-max}). 
\ea
$\lambda x~\relation{sleep}(x)$ \relation{max}
\z
%CAUTION:
这一过程叫做$\beta$"=约归($\beta$"=reduction)\isc{beta-约归@$\beta$-约归}\is{beta"=reduction@$\beta$"=reduction}或$\lambda$"=变换($\lambda$"=conversion)\isc{lambda-变换@$\lambda$-变换}\is{lambda"=conversion@$\lambda$"=conversion}。为了深入说明这一概念,我们用及物动词的例子来分析。例(\mex{1}a)中的句子具有(\mex{1}b)所表示的含义:
%The process is called $\beta$"=reduction\is{beta"=reduction@$\beta$"=reduction} or 
%$\lambda$"=conversion\is{lambda"=conversion@$\lambda$"=conversion}. To show this further, let us consider an example with a transitive verb. The sentence
%in (\mex{1}a) has the meaning given in (\mex{1}b):
\eal
\ex\label{Bsp-Max-mag-Lotte} 
\gll Max mag Lotte.\\
     Max 喜欢 Lotte\\
\glt `Max喜欢Lotte.'
%\gll Max mag Lotte.\\
 %    Max likes Lotte\\
%\glt `Max likes Lotte.'
\ex \relation{like}(\relation{max}, \relation{lotte})
\zl
mag(喜欢)的$\lambda$"=抽象如例(\mex{1})所示:
%The $\lambda$"=abstraction of \emph{mag} `likes' is shown in (\mex{1}):
\ea
$\lambda y \lambda x~\relation{like}(x, y)$
\z
需要注意的是,第一个$\lambda$总是需要用在第一位的。变量$y$对应于mögen的宾语。对于英语这种语言来说,宾语和动词一起构成动词短语(VP),并且这个VP是与主语相搭配的。
德语与英语不同的地方在于,德语在语序方面允许更大的自由度。造成形式意义匹配的问题在不同的理论中有不同的解决方案。我们将在下面的章节中来说明具体的解决方案。
%Note that it is always the first $\lambda$ that has to be used first. The variable $y$ corresponds
%to the object of \emph{mögen}. For languages like English it is assumed that the object forms a verb
%phrase (VP) together with the verb and this VP is combined with the subject. German differs from
%English in allowing more freedom in constituent order. The problems that result for form meaning
%mappings are solved in different ways by different theories. The respective solutions will be
%addressed in the following chapters.

如果我们将(\mex{0})中的表达式与其宾语Lotte整合在一起,我们会得到(\mex{1}a),然后通过$\beta$"=约归得到(\mex{1}b):
%If we combine the representation in (\mex{0}) with that of the object \emph{Lotte}, we arrive at (\mex{1}a), and following
%$\beta$"=reduction, (\mex{1}b):
\eal
\label{lambda-moegen}
\ex $\lambda y \lambda x~\relation{like}(x, y) \relation{lotte}$
\ex $\lambda x~\relation{like}(x, \relation{lotte})$
\zl

\noindent
这一意义可以反过来与主语相整合,进而我们得到(\mex{1}a),以及通过$\beta$"=约归后的(\mex{1}b):
%This meaning can in turn be combined with the subject and we then get (\mex{1}a) and (\mex{1}b) after $\beta$"=reduction:
\eal
\ex $\lambda x~\relation{like}(x, \relation{lotte}) \relation{max}$
\ex \relation{like}(\relation{max}, \relation{lotte})
\zl

\begin{sloppypar}
\noindent
在介绍完$\lambda$-演算后,我们将组合性语义整合进我们的短语结构中就比较简单了。动词跟主语组合的规则需要进一步扩展,以囊括进动词的语义贡献、主语的语义贡献,以及二者组合(整个句子)后的语义所占据的位置。
  完整的语义是按照正确顺序排列的个体语义的组合。由此,我们可以将(\mex{1}a)中的简单规则转化为(\mex{1}b):
%After introducing lambda calculus, integrating the composition of meaning into our phrase structure rules is simple. A rule for the
%combination of a verb with its subject has to be expanded to include positions for the semantic contribution of the verb, the semantic
%contribution of the subject and then the meaning of the combination of these two (the entire sentence). The complete meaning is the
%combination of the individual meanings in the correct order. We can therefore take the simple rule in (\mex{1}a) and turn it into (\mex{1}b):
\end{sloppypar}
\eal
\ex S $\to$ NP(nom) V
\ex S(V$'$ NP$'$) $\to$ NP(nom, NP$'$) V(V$'$)
\zl
V$'$表示V的意义,NP$'$表示NP(nom)的意义。V$'$ NP$'$ 表示V$'$和NP$'$的组合性意义。当我们分析(\ref{Bsp-Max-schlaeft})时,V$'$ 的意义是$\lambda x~\relation{sleep}(x)$,而NP$'$的意义是\relation{max}。V$'$ NP$'$的组合对应于(\mex{1}a),或者在$\beta$"=约归后对应于(\ref{Bsp-schlafen-max})——这里重复表示为(\mex{1}b):
%V$'$ stands for the meaning of V and NP$'$ for the meaning of the NP(nom). V$'$ NP$'$ stands for the combination of V$'$ and NP$'$. When analyzing
%(\ref{Bsp-Max-schlaeft}), the meaning of V$'$ is $\lambda x~\relation{sleep}(x)$ and the meaning of NP$'$ is \relation{max}. The combination of V$'$ NP$'$
%corresponds to (\mex{1}a) or after $\beta$"=reduction to (\ref{Bsp-schlafen-max}) -- repeated here as (\mex{1}b):
\eal
\ex $\lambda x~\relation{sleep}(x) \relation{max}$
\ex \relation{sleep}(\relation{max})
\zl

\noindent
针对例(\ref{Bsp-Max-mag-Lotte})中的及物动词,我们可以提出(\mex{1})中的规则来处理:
%For the example with a transitive verb in (\ref{Bsp-Max-mag-Lotte}), the rule in (\mex{1}) can be proposed:
\ea
S(V$'$ NP2$'$ NP1$'$) $\to$ NP(nom, NP1$'$) V(V$'$) NP(acc, NP2$'$)
\z
动词(V$'$)的意义首先与宾语(NP2$'$)的意义组合,然后跟主语(NP1$'$)的意义组合。
%The meaning of the verb (V$'$) is first combined with the meaning of the object (NP2$'$) and then with the meaning of the subject (NP1$'$). 

在这一点,我们可以看到上述的短语结构规则有几条不同的语义规则。按照这一方式来分析语言的理论假说叫做“规则对应假说”(rule"=to"=rule hypothesis)\isc{规则对应假说}\is{rule"=to"=rule hypothesis}\citep[\page 184]{Bach76a}。针对语言表达式的意义的推导过程将在\ref{Sec-GPSG-Sem}中详细展开。
%At this point, we can see that there are several distinct semantic rules for the phrase structure rules above. The hypothesis that we should analyze language
%in this way is called the \emph{rule"=to"=rule hypothesis}\is{rule"=to"=rule hypothesis}
%\citep[\page 184]{Bach76a}. A more general process for deriving the
%meaning of linguistic expression will be presented in Section~\ref{Sec-GPSG-Sem}.

\section{部分德语句法的短语结构规则}
%\section{Phrase structure rules for some aspects of German syntax}

因为德语中相对自由的语序允许我们通过移位测试来判断句中的直接成分,所以说对句中直接成分的判断是相对容易的。相较而言,名词短语内各部分的分析则是比较困难的。我们在这一节就来集中解决这一问题。为了对\ref{sec-xbar}中有关 \xbar 句法理论的讲解作准备,我们还会讲到介词短语。
%Whereas determining the direct constituents of a sentence is relative easy, since we can very much rely on the movement test due to the
%somewhat flexible order of constituents in German, it is more difficult to identify the parts of the noun phrase. This is the problem
%we will focus on in this section. To help motivate assumptions about \xbar~syntax to be discussed in Section~\ref{sec-xbar},
%we will also discuss prepositional phrases.

\subsection{名词短语}
%\subsection{Noun phrases}
\label{sec-psg-np}

到目前为止,我们提出了名词短语的一个相对简单的结构:我们的规则规定了一个名词短语包括一个限定词和一个名词。名词短语可以是像例(\mex{1}a)那样较为复杂的结构,如例(\mex{1})所示:
%Up to now, we have assumed a relatively simple structure for noun phrases: our rules state that a noun phrase consists of a determiner and a
%noun. Noun phrases can have a distinctly more complex structure than (\mex{1}a). This is shown by the following examples in (\mex{1}):
\eal
\label{Beispiele-NP-Adjunkte}
\ex 
\gll eine Frau\\
	 一 女人\\
\ex 
\gll eine Frau, die wir kennen\\
	 一 女人 \textsc{rel} 我们 认识\\
\ex 
\gll eine Frau aus Stuttgart\\
	 一 女人 \textsc{prep} 斯图加特\\
\ex 
\gll eine kluge Frau\\
	 一 聪明 女人\\
\ex 
\gll eine Frau aus Stuttgart, die wir kennen\\
	 一 女人 \textsc{prep} 斯图加特 \textsc{rel} 我们 认识\\
\ex 
\gll eine kluge Frau aus Stuttgart\\
	 一 聪明 女人 \textsc{prep} 斯图加特\\
\ex 
\gll eine kluge Frau, die wir kennen\\
	 一 聪明 女人 \textsc{rel} 我们 认识\\
\ex 
\gll eine kluge Frau aus Stuttgart, die wir kennen\\
	 一 聪明 女人 \textsc{prep} 斯图加特 \textsc{rel} 我们 认识\\
\zl
%\eal
%\label{Beispiele-NP-Adjunkte}
%\ex 
%\gll eine Frau\\
%	 a woman\\
%\ex 
%\gll eine Frau, die wir kennen\\
%	 a woman who we know\\
%\ex 
%\gll eine Frau aus Stuttgart\\
%	 a woman from Stuttgart\\
%\ex 
%\gll eine kluge Frau\\
%	 a smart woman\\
%\ex 
%\gll eine Frau aus Stuttgart, die wir kennen\\
%	 a woman from Stuttgart who we know\\
%\ex 
%\gll eine kluge Frau aus Stuttgart\\
%	 a smart woman from Stuttgart\\
%\ex 
%\gll eine kluge Frau, die wir kennen\\
%	 a smart woman who we know\\
%\ex 
%\gll eine kluge Frau aus Stuttgart, die wir kennen\\
%	 a smart woman from Stuttgart who we know\\
%\zl

\noindent
与限定词和名词一样的是,名词短语也可以包括形容词、介词短语和关系从句。例(\mex{0})中附加的成分是附接语\isc{附接语|(}\is{adjunct|(}。它们限制了名词短语所指称的物体的集合。而(\mex{0}a)是指具有女人的属性的某个实体,(\mex{0}b)中的所指也必须具有为我们所知的属性。
%As well as determiners and nouns, noun phrases can also contain adjectives, prepositional phrases and relative clauses. 
%The additional elements in (\mex{0}) are adjuncts\is{adjunct|(}. They restrict the set of objects which the noun phrase 
%refers to. Whereas (\mex{0}a) refers to a being which has the property of being a woman, the referent of (\mex{0}b) must
%also have the property of being known to us.

前面的名词短语规则只是简单地将名词和限定词组合在一起,这样只能用来分析例(\mex{0}a)。我们现在面临的问题是如何修改这条规则,以适应于分析例(\mex{0})中的其他名词短语。除了规则(\mex{1}a)之外,我们可以提出跟(\mex{1}b)一样的一条规则。\footnote{%
参看\citew[\page 238]{Eisenberg2004a}中有关名词短语中平铺结构的假说。
}$^,$\footnote{%
当然,还有诸如性和数的其他特征也应该在这一节里讨论。但是,为了降低解释的难度,我们暂不考虑这些问题。
}
%Our previous rules for noun phrases simply combined a noun and a determiner and can therefore only be used to
%analyze (\mex{0}a). The question we are facing now is how we can modify this rule or which additional rules we would
%have to assume in order to analyze the other noun phrases in (\mex{0}). In addition to rule (\mex{1}a), one could propose 
%a rule such as the one in (\mex{1}b).\footnote{%
%	See \citew[\page 238]{Eisenberg2004a} for the assumption of flat structures in noun phrases.
%}$^,$\footnote{%
%	There are, of course, other features such as gender and number, which should be part of all the rules
%	discussed in this section. I have omitted these in the following for ease of exposition.
%}
%\todostefan{These footnotes have to be blocked from moving to the next page.}
% reformulating one line to be shorter plus enlarging the page by two lines did the trick.
\eal
\ex NP $\to$ Det N
\ex NP $\to$ Det A N
\zl
但是,这条规则并不能用来分析诸如(\mex{1})的名词短语:
%However, this rule would still not allow us to analyze noun phrases such as (\mex{1}):
\ea
\label{Beispiel-alle-weitern-schlagkraeftigen-Argumente}
\gll alle weiteren schlagkräftigen Argumente\\
	 所有 进一步 强有力 论证\\
\glt `所有其他的强有力的论证'
\z
%\ea
%\label{Beispiel-alle-weitern-schlagkraeftigen-Argumente}
%\gll alle weiteren schlagkräftigen Argumente\\
%	 all further strong arguments\\
%\glt `all other strong arguments'
%\z
为了分析(\mex{0}),我们需要例(\mex{1})中的规则:
%In order to be able to analyze (\mex{0}), we require a rule such as (\mex{1}): 
\ea 
NP $\to$ Det A A N
\z
在名词短语中,总是有可能增加形容词的数量。而给这些形容词制定一个上限是相当武断的。即使我们选择如下的简称,仍是有问题的:
%It is always possible to increase the number of adjectives in a noun phrase and setting an upper limit for
%adjectives would be entirely arbitrary. Even if we opt for the following abbreviation, there are still problems:

\ea 
NP $\to$ Det A* N
\z
例(\mex{0})中的星号\isc{*}\is{*}表示任意的迭代次数。所以说,(\mex{0})包括了没有形容词的规则,以及有一个、两个,或更多个形容词的规则。
%The asterisk\is{*} in (\mex{0}) stands for any number of iterations. Therefore, (\mex{0}) encompasses rules with no adjectives
%as well as those with one, two or more.

% because of moving footnotes
问题是,根据(\mex{0})中的规则,形容词和名词并不组成一个成分,这样我们就解释不了为什么例(\mex{1})中的并列结构仍是可能的:
%The problem is that according to the rule in (\mex{0}) adjectives and nouns do not form a constituent and we can therefore not explain why coordination 
%is still possible in (\mex{1}):
\ea
\gll alle [[geschickten Kinder] und [klugen Frauen]]\\
	 所有  \spacebr{}\spacebr{}有能力 孩子 和  \spacebr{}聪明 女人\\
\glt `所有有能力的孩子和聪明的女人'	 
%\gll alle [[geschickten Kinder] und [klugen Frauen]]\\
%	 all  \spacebr{}\spacebr{}skillful children and  \spacebr{}smart women\\
%\glt `all the skillful children and smart women'	 
\z
如果我们假设并列涉及到具有相同句法属性的两个或两个以上的字符串的组合的话,那么我们就需要认为形容词和名词构成了一个单位。
%If we assume that coordination involves the combination of two or more word strings with the same syntactic properties, then we would have to assume
%that the adjective and noun form a unit.

%The noun phrases with adjectives discussed thus far can be captured by the following rules:
%
% Shorter for layout:
下面的规则说明了截至目前所讨论的带形容词的名词短语:
%The following rules capture the noun phrases with adjectives discussed thus far:
\eal
\label{NP-Regeln}
\ex NP $\to$ Det \nbar
\ex\label{NP-Regeln-Adj} \nbar $\to$ A \nbar
\ex\label{NP-Regeln-Nbar-N} \nbar $\to$ N
\zl

\noindent
这些规则说明了:一个名词短语包括一个限定词和一个名词性成分(\nbarc)。这个名词性成分包括一个形容词和一个名词性成分(\mex{0}b),或者只包括一个名词(\mex{0}c)。因为\nbarc 也位于(\mex{0}b)中规则的右边,所以我们可以多次应用这一规则,并且用它来说明诸如(\ref{Beispiel-alle-weitern-schlagkraeftigen-Argumente})的带有多个形容词的名词短语。图\vref{Abbildung-Adjektive-in-NP} 给出了不带形容词的名词短语、带一个形容词的名词短语和带两个形容词的名词短语的结构。
%These rules state the following: a noun phrase consists of a determiner and a nominal element (\nbar). This nominal element
%can consist of an adjective and a nominal element (\mex{0}b), or just a noun (\mex{0}c). Since \nbar is also on the right"=hand side
%of the rule in (\mex{0}b), we can apply this rule multiple times and therefore account for noun phrases with multiple adjectives such as
%(\ref{Beispiel-alle-weitern-schlagkraeftigen-Argumente}). Figure~\vref{Abbildung-Adjektive-in-NP} shows the structure of a noun phrase
%without an adjective and that of a noun phrase with one or two adjectives.
\begin{figure}
\hfill%
\begin{forest}
sm edges
[NP
   [Det [eine;一] ]
   [\nbar
      [N [Frau;女人] ] ] ]
\end{forest}
\hfill
\begin{forest}
sm edges
[NP
   [Det [eine;一] ]
   [\nbar
      [A [kluge;聪明] ]
      [\nbar
        [N [Frau;女人] ] ] ] ]
\end{forest}
%
\hfill
\begin{forest}
sm edges
[NP
  [Det [eine;一] ]
    [\nbar
    [A [glückliche;快乐] ]
       [\nbar
       [A [kluge;聪明] ]
         [\nbar
         [N [Frau;女人] ] ] ] ] ]
\end{forest}
\hfill\mbox{}
%
\caption{\label{Abbildung-Adjektive-in-NP}带有不同数量的形容词的名词短语}
%\caption{\label{Abbildung-Adjektive-in-NP}Noun phrases with differing numbers of adjectives}
\end{figure}%
形容词klug(聪明)限定了名词短语所指的集合。如果我们假设另一个形容词,如glücklich(开心),那么它只是指那些既开心又聪明的女人。这种名词短语可以用在如下的语境中:
%The adjective \emph{klug} `smart' restricts the set of referents for the noun phrase. If we assume an
%additional adjective such as \emph{glücklich} `happy', then it only refers to those women who are happy
%as well as smart. These kinds of noun phrases can be used in contexts such as the following:

\ea
\label{Beispiel-Iteration-Adjektive}
\gll A: Alle klugen Frauen sind unglücklich.\\
\spacebr{} 所有 聪明 女人 \textsc{cop} 不开心\\
%\spacebr{} all smart women are unhappy\\

\gll B: Nein, ich kenne eine glückliche kluge Frau.\\
	\spacebr{} 不 我 认识 一 开心 聪明 女人\\
%	\spacebr{} no I know a happy smart woman\\
\z
我们观察到,这段话可以加上Aber alle glücklichen
  klugen Frauen sind schön(但是所有开心和聪明的女人都是漂亮的),以及一个相应的回答。在我们的规则系统中,(\mex{-1})也允许了在名词短语中增加更多形容词的情况,如eine glückliche kluge Frau(一个开心的、聪明的女人)。在规则(\ref{NP-Regeln-Adj})中,\nbarc 既出现在规则的左边,也出现在规则的右边。这类规则被看作是“递归的”(recursive)\isc{递归}\is{recursion}。
%We observe that this discourse can be continued with \emph{Aber alle glücklichen
%  klugen Frauen sind schön} `but all happy, smart women are beautiful' and a corresponding answer. The possibility
  %to have even more adjectives in noun phrases such as \emph{eine glückliche kluge Frau} `a happy, smart
 % woman'\todoandrew{Komma zwischen den Adjektiven?} is accounted
  %for in our rule system in (\mex{-1}). In the rule (\ref{NP-Regeln-Adj}), \nbar occurs on the left as well as the right"=hand
  %side of the rule. This kind of rule is referred to as \emph{recursive}\is{recursion}.
\isc{附接语|)}\is{adjunct|)}

我们现在构建了一套精巧的语法,可以用它来分析带有形容词性修饰语的名词短语。结果是,形容词和名词的组合构成一个组成成分。我们也可以想到这个问题,即限定词和形容词是否能够构成一个成分,如下类名词短语:
%We have now developed a nifty little grammar that can be used to analyze noun phrases containing
%adjectival modifiers. As a result, the combination of an adjective and noun is given constituent
%status. One may wonder at this point if it would not make sense to also assume that determiners and
%adjectives form a constituent, as we also have the following kind of noun phrases: 
\ea
\gll diese schlauen und diese neugierigen Frauen\\
	 这些 聪明 和 这些 好奇 女人\\
%	 these smart and these curious women\\
\z
这里,我们分析一个不同的结构。虽然有两个完整的NP并列\isc{并列}\is{coordination}在一起,但是第一个并列成分的一部分被删除了。
%Here, we are dealing with a different structure, however. Two full NPs have been
%conjoined\is{coordination} and part of the first conjunct has been deleted.
\ea
\gll diese schlauen \st{Frauen} und diese neugierigen Frauen\\
	  这些 聪明 女人 和 这些 好奇 女人\\
%	 these smart women and these curious women\\
\z
我们也可以在句子,甚至是词层面发现类似的现象:
%One can find similar phenomena at the sentence and even word level:
\eal
\ex 
\gll dass Peter dem Mann das Buch \st{gibt} und Maria der Frau die Schallplatte gibt\\
     \textsc{comp} Peter \textsc{det} 人 \textsc{det} 书 给 和 Maria \textsc{det} 女人 \textsc{det} 唱片 给\\
\glt `Peters给这个人这本书,并且Maria给这个女人这张唱片'
%	 that Peter the man the book gives and Maria the woman the record gives\\
%\glt `that Peters gives the book to the man and Maria the record to the woman'
\ex 
\gll be- und ent"=laden\\
	 \prt{} 和 \prt{}"=装载\\
\glt `装载和卸载'
%	 \prt{} and \prt{}"=load\\
%\glt `load and unload'
\zl
% Dass in (\mex{-1}) wirklich keine normale symmetrische Koordination vorliegt, sieht man, wenn man
% (\mex{-1}) mit (\mex{1}) vergleicht:
% \ea
% diese schlauen Frauen und klugen Männer
% \z
% Mit (\mex{0}) verweist man auf eine Gruppe, die aus schlauen Frauen und klugen Männern besteht,
% wohingegen man mit (\mex{-2}) auf zwei Gruppen verweist, nämlich

\noindent
到目前为止,我们讨论了如何能够将形容词整合进我们的有关名词短语的规则当中。其他附接语,如介词短语或关系小句,也可以按照类似的方式将形容词加入\nbarc 之中。
%Thus far, we have discussed how we can ideally integrate adjectives into our rules for the structure of noun phrases.
%Other adjuncts such as prepositional phrases or relative clauses can be combined with \nbar in an analogous way to adjectives:
\eal
\ex\label{xbar-PP-Adjunkt-an-N} \nbar $\to$ \nbar PP
\ex \nbar $\to$ \nbar relative clause
\zl
有了这些规则和(\ref{NP-Regeln})的那些规则,我们就可以设想PP和关系小句的规则,并用它们来分析(\ref{Beispiele-NP-Adjunkte})中的例子。
%With these rules and those in (\ref{NP-Regeln}), it is possible -- assuming the corresponding rules for PPs and
%relative clauses -- to analyze all the examples in (\ref{Beispiele-NP-Adjunkte}).

(\ref{NP-Regeln}c)说明了 \nbarc 有可能只带一个名词。但我们还没有讨论更为重要的规则:我们需要另一个规则来整合名词与它们所带的论元成分。如例(\mex{1}a--b)所示,这些名词有Vater(爸爸)、Sohn(儿子)和Bild(照片)这类“关系名词”(relational nouns)\isc{名词!关系名词}\is{noun!relational}。例(\mex{1}c)说明了动词与其所带论元的名词化过程:
%(\ref{NP-Regeln}c) states that it is possible for \nbar to consist of a single noun. A further important rule has not yet been
%discussed: we need another rule to combine nouns such as \emph{Vater} `father', \emph{Sohn} `son' or \emph{Bild} `picture', 
%so"=called \emph{relational nouns}\is{noun!relational}, with their arguments. Examples of these can be found in (\mex{1}a--b).
%(\mex{1}c) is an example of a nominalization of a verb with its argument:
\eal
\label{Beispiele-NP-relationale-Nomina}
\ex 
\gll der Vater von Peter\\
	 \textsc{det} 爸爸 \textsc{prep} Peter\\
\glt `Peter的爸爸'
%	 the father of Peter\\
%\glt `Peter's father'
\ex 
\gll das Bild vom Gleimtunnel\\
	 \textsc{det} 照片 \textsc{prep}.\textsc{det} 格莱姆隧道\\
\glt `格莱姆隧道的照片'
%	 the picture of.the Gleimtunnel\\
%\glt `the picture of the Gleimtunnel'
\ex 
\gll das Kommen des Installateurs\\
	 \textsc{det} 来 \textsc{det} 水管工\\
\glt `水管工的检查'
%	 the coming of.the plumber\\
%\glt `the plumber's visit'
\zl
\noindent
下面是用来分析(\ref{Beispiele-NP-relationale-Nomina}a、b)的规则,如(\mex{1})所示:
%The rule that we need to analyze (\ref{Beispiele-NP-relationale-Nomina}a,b) is given in (\mex{1}):
%\todostefan{Martin: It is not said why relational nouns should behave in a special way.}
\ea
\nbar $\to$ N PP
\z
%
图\ref{Abbildung-NP-mit-PP-Argument}说明了带PP论元的两种结构。右边的树形图还包括一个额外的PP附接语,并由(\ref{xbar-PP-Adjunkt-an-N})中的规则来允准。
%Figure~\ref{Abbildung-NP-mit-PP-Argument} shows two structures with PP"=arguments. The tree on the right also contains an additional PP"=adjunct, which is licensed
%by the rule in (\ref{xbar-PP-Adjunkt-an-N}).
\begin{figure}
\centerfit{%
\begin{forest}
sm edges
[NP
 [Det [das;\textsc{det}] ]
 [\nbar
   [N [Bild;照片] ]
   [PP [vom Gleimtunnel;\textsc{prep}.\textsc{det} 格莱姆隧道,roof ] ] ] ]
\end{forest}%
\hspace{2em}%
\begin{forest}
sm edges
[NP
  [Det [das;\textsc{det}] ]
  [\nbar
    [\nbar
      [N [Bild;照片] ]
      [PP [vom Gleimtunnel;\textsc{prep}.\textsc{det} 格莱姆隧道,roof ] ] ] 
    [PP [im Gropiusbau;\textsc{prep}.\textsc{det} 格罗皮乌斯博物馆,roof ] ] ] ]
\end{forest}}
\caption{\label{Abbildung-NP-mit-PP-Argument}名词与位于右侧的作为附接语的PP补足语“vom Gleimtunnel”的组合}
%\caption{\label{Abbildung-NP-mit-PP-Argument}Combination of a noun with PP complement
%  \emph{vom Gleimtunnel} to the right with an adjunct PP}
\end{figure}%

除了上述讨论的NP结构,还有限定语或名词缺失的其他结构。名词可以通过省略而被删除。(\mex{1})给出了名词短语的一个例子,这里不需要补足语的名词被省略了。例(\mex{2})说明了名词只有一个限定语,而且其补足语被实现了,而不是名词本身。下划线表明了名词正常应该出现的位置。
%In addition to the previously discussed NP structures, there are other structures where the determiner or noun is missing.
%Nouns can be omitted via ellipsis. (\mex{1}) gives an example of noun phrases, where a noun that does not require a complement
%has been omitted. The examples in (\mex{2}) show NPs in which only one determiner and complement of the noun has been realized,
%but not the noun itself. The underscore marks the position where the noun would normally occur. 
\eal
\label{ex-nounless-np}
\ex 
\gll eine kluge \_\\
	 一 聪明\\
\glt `一个聪明的'
\ex 
\gll eine kluge große \_\\
     一    聪明 高\\
\glt `一个聪明的高的'

\ex 
\gll eine kluge \_ aus Hamburg\\
	 一 聪明 {} \textsc{prep} 汉堡\\
\glt `一位从汉堡来的聪明的'
\ex 
\gll eine kluge \_, die alle kennen\\
	 一 聪明 {} \textsc{rel} 所有人 知道\\
\glt `一个大家都知道的聪明的'
\zl

\eal
\label{ex-nounless-np-relational-noun}
\ex 
\gll (Nein, nicht der Vater von Klaus), der \_ von Peter war gemeint.\\
	\spacebr{}不 不  \textsc{det} 爸爸 \textsc{prep} Klaus \textsc{det} {} \textsc{prep} Peter \textsc{aux} 指的是\\
\glt `不,指的不是Klaus的爸爸,而是Peter的爸爸。'
\ex 
\gll (Nein, nicht das Bild von der Stadtautobahn), das \_ vom Gleimtunnel war beeindruckend.\\
	 \spacebr{}不 不  \textsc{det} 图片 \textsc{prep} \textsc{det} 公路 \textsc{det} {} \textsc{prep}.\textsc{det} 格莱姆隧道 \textsc{cop} 印象深刻\\
\glt `不,不是公路的照片让人印象深刻,而是格莱姆隧道的照片。'
\ex 
\gll (Nein, nicht das Kommen des Tischlers), das \_ des Installateurs ist wichtig.\\
	 \spacebr{}不 不  \textsc{det} 来 \textsc{det} 木匠 \textsc{det} {} \textsc{det} 水管工 \textsc{cop} 重要\\
\glt `不,木匠来不重要,水管工来很重要。'
\zl
%\eal
%\label{ex-nounless-np}
%\ex 
%\gll eine kluge \_\\
%	 a smart\\
%\glt `a smart one'
%\ex 
%\gll eine kluge große \_\\
%     a    smart tall\\
%\glt `a smart tall one'
%
%\ex 
%\gll eine kluge \_ aus Hamburg\\
%	 a smart {} from Hamburg\\
%\glt `a smart one from Hamburg'
%\ex 
%\gll eine kluge \_, die alle kennen\\
%	 a smart {} who everyone knows\\
%\glt `a smart one who everyone knows'
%\zl
%
%\eal
%\label{ex-nounless-np-relational-noun}
%\ex 
%\gll (Nein, nicht der Vater von Klaus), der \_ von Peter war gemeint.\\
%	\spacebr{}no not the father of Klaus the {} of Peter was meant\\
%\glt `No, it wasn't the father of Klaus, but rather the one of Peter that was meant.'
%\ex 
%\gll (Nein, nicht das Bild von der Stadtautobahn), das \_ vom Gleimtunnel war beeindruckend.\\
%	 \spacebr{}no not the picture of the motorway the {} of.the Gleimtunnel was impressive\\
%\glt `No, it wasn't the picture of the motorway, but rather the one of the Gleimtunnel that was impressive.'
%\ex 
%\gll (Nein, nicht das Kommen des Tischlers), das \_ des Installateurs ist wichtig.\\
%	 \spacebr{}no not the coming of.the carpenter the {} of.the plumber is important\\
%\glt `No, it isn't the visit of the carpenter, but rather the visit of the plumber that is important.'
%\zl
在英语中,代词one必须用在相应的位置上,\footnote{%
参看\citet[\S~4.12]{FLGR2012a} ,在英语中这些例子没有代词one。
} 但是德语中,名词\isc{名词}\is{noun}就被简单地省略了。
%In English, the pronoun
%\emph{one} must often be used in the corresponding position,\footnote{%
  %See \citet[Section~4.12]{FLGR2012a} for English\il{English} examples without the
  %pronoun \emph{one}.
%} but in German the noun\is{noun} is
%simply omitted.
%\todostefan{ich habe hier mit absicht ``often'' hinzugefügt, weil in den obengenannten beispielen
%man nicht in allen Fällen ``one'' benutzen kann, vgl. (\mex{0}c)}
在短语结构语法中,这就叫做“$\epsilon$生成式”(epsilon production)\isc{$\epsilon$生成式}\is{epsilon production}\isc{空成分}\is{empty element}。例(\mex{1}a)中,这些规则被一个空成分所替代。例(\mex{1}b)中的规则是与术语“$\epsilon$生成式”(epsilon production)相关的变量:
%In phrase structure grammars, this can be described by a so"=called \emph{epsilon production}\is{epsilon production}\is{empty element}.
%These rules replace a symbol with nothing (\mex{1}a). The rule in (\mex{1}b) is an equivalent variant which is responsible for the term \emph{epsilon production}:
\eal
\label{np-epsilon}
\ex N $\to$
\ex N $\to$ $\epsilon$
\zl 

\noindent
相应的树形图如图\vref{Abbildung-NP-ohne-Nomen}所示。
%The corresponding trees are shown in Figure~\vref{Abbildung-NP-ohne-Nomen}.
\begin{figure}
\hfill
\begin{forest}
sm edges
[NP
  [Det [eine;一] ]
  [\nbar
    [A [kluge;聪明] ]
    [\nbar
      [N [\trace ] ] ] ] ]
\end{forest}
\hfill
\begin{forest}
sm edges
[NP
  [Det [das;\textsc{det}] ]
  [\nbar
    [N [\trace] ]
    [PP [vom Gleimtunnel;\textsc{prep}.\textsc{det} 格莱姆隧道, roof] ] ] ]
\end{forest}
\hfill%
\mbox{}
\caption{\label{Abbildung-NP-ohne-Nomen}没有明显中心语的名词短语}
%\caption{\label{Abbildung-NP-ohne-Nomen}Noun phrases without an overt head}
\end{figure}%
回到盒子的比喻,例(\mex{0})中的规则像是贴有同样标签的空盒子,就好像是普通名词的盒子一样。正如我们在前面所强调的,当我们要思考的是将这些盒子放在哪里的问题时,装在盒子里的内容就不重要了。由此,例(\ref{Beispiele-NP-Adjunkte})中的名词短语可以在同样的句子中出现。空名词盒子也可以像一个真正的名词那样。如果我们不打开这个空盒子,我们就不会知道它与装满东西的盒子有什么不同。
%Going back to boxes, the rules in (\mex{0}) correspond to empty boxes with the same labels as the boxes
%of ordinary nouns. As we have considered previously, the actual content of the boxes is unimportant when
%considering the question of where we can incorporate them. In this way, the noun phrases in (\ref{Beispiele-NP-Adjunkte})
%can occur in the same sentences. The empty noun box also behaves like one with a genuine noun. If we
%do not open the empty box, we will not be able to ascertain the difference to a filled box. 

我们不仅可以在名词短语中省略名词,在某些语境中,限定词也可以不出现。例(\mex{1})显示了复数\isc{复数}\is{plural}的名词短语:
%It is not only possible to omit the noun from noun phrases, but the determiner can also remain unrealized in certain contexts.
%(\mex{1}) shows noun phrases in plural\is{plural}:
\eal
\ex 
\gll Frauen\\
	 女人\\
%	 women\\
\ex 
\gll Frauen, die wir kennen\\
	 女人 \textsc{rel} 我们 认识\\
%	 women who we know\\
\ex 
\gll kluge Frauen\\
	 聪明 女人\\
%	 smart women\\
\ex 
\gll kluge Frauen, die wir kennen\\
	 聪明 女人 \textsc{rel} 我们 认识\\
%	 smart women who we know\\
\zl
如果名词是集合名词\isc{名词!集合名词}\is{noun!mass}的话,限定词在单数时可以被省略:
%The determiner can also be omitted in singular if the noun denotes a mass noun\is{noun!mass}: 
\eal
\ex 
\gll Getreide\\
	 粮食\\
%	 grain\\
\ex 
\gll Getreide, das gerade gemahlen wurde\\
	 粮食 \textsc{rel} 刚 种 \textsc{aux}\\
\glt `刚种的粮食'
%	 grain that just ground was\\
%\glt `grain that has just been ground'
\ex 
\gll frisches Getreide\\
	 新鲜 粮食\\
%	 fresh grain\\
\ex 
\gll frisches Getreide, das gerade gemahlen wurde\\
	 新鲜 粮食 \textsc{rel} 刚刚 种 \textsc{aux}\\
\glt `刚种的新鲜的粮食'
%	 fresh grain that just ground was\\
%\glt `fresh grain that has just been ground'
\zl
最后,限定词和名词都可以被省略:
%Finally, both the determiner and the noun can be omitted: 
\eal
\ex 
\gll Ich helfe klugen.\\
	 我 帮助 聪明\\
\glt `我帮助聪明人。'
%	 I help smart\\
%\glt `I help smart ones.'
\ex 
\gll Dort drüben steht frisches, das gerade gemahlen wurde.\\
	 那儿 那边 有 新鲜 \textsc{rel} 刚 种 \textsc{aux}\\
\glt `在那边有一些刚种的新鲜的(粮食)。'
%	 there over stands fresh that just ground was\\
%\glt `Over there is some fresh (grain) that has just been ground.'
\zl
图\vref{Abbildung-NP-ohne-Det}展示了相应的树形图。
%Figure~\vref{Abbildung-NP-ohne-Det} shows the corresponding trees. 

\begin{figure}
\hfill
\begin{forest}
sm edges
[NP
  [Det [\trace] ]
  [\nbar
    [N [Frauen;女人] ] ] ]
\end{forest}
\hfill
\begin{forest}
sm edges
[NP
  [Det [\trace] ]
  [\nbar
    [A [klugen;聪明] ]
    [\nbar
      [N [\trace] ] ] ] ]
\end{forest}
\hfill
\mbox{}
\caption{\label{Abbildung-NP-ohne-Det}没有明显限定词的名词短语}
%\caption{\label{Abbildung-NP-ohne-Det}Noun phrases without overt determiner}
\end{figure}%

针对我们目前提出的规则,有两点需要说明:截至目前,我一直在说形容词。但是,在名词前的位置上可以有非常复杂的形容词短语。这些可以是带补足语(\mex{1}a、b)的形容词或者是形容词性助词(\mex{1}c、d):
%It is necessary to add two further comments to the rules we have developed up to this point: up to now, I have
%always spoken of adjectives. However, it is possible to have very complex adjective phrases in pre"=nominal position.
%These can be adjectives with complements (\mex{1}a,b) or adjectival participles (\mex{1}c,d):

\eal
\ex 
\gll der seiner Frau treue Mann\\
	 \textsc{det} 他的.\dat{} 妻子 忠诚 男人\\
\glt `忠于妻子的男人'
\ex 
\gll der auf seinen Sohn stolze Mann\\
	 \textsc{det} \textsc{prep} 他的.\acc{} 儿子 骄傲 男人\\
\glt `以儿子为骄傲的男人'
\ex 
\gll der seine Frau liebende Mann\\
	 \textsc{det} 他的.\acc{} 女人 爱 男人\\
\glt `爱妻子的男人'
\ex 
\gll der von seiner Frau geliebte Mann\\
     \textsc{det} \textsc{prep} 他的.\dat{} 妻子 爱 男人\\
\glt `被妻子爱的男人'	 
\zl
%\eal
%\ex 
%\gll der seiner Frau treue Mann\\
%	 the his.\dat{} wife faithful man\\
%\glt `the man faithful to his wife'
%\ex 
%\gll der auf seinen Sohn stolze Mann\\
%	 the on his.\acc{} son proud man\\
%\glt `the man proud of his son'
%\ex 
%\gll der seine Frau liebende Mann\\
%	 the his.\acc{} woman loving man\\
%\glt `the man who loves his wife'
%\ex 
%\gll der von seiner Frau geliebte Mann\\
%     the by his.\dat{} wife loved man\\
%\glt `the man loved by his wife'	 
%\zl
把这点考虑进来的话,我们需要将规则(\ref{NP-Regeln-Adj})按照如下的方式来修改:
%Taking this into account, the rule (\ref{NP-Regeln-Adj}) has to be modified in the following way:
\ea
\label{NP-Regeln-AP} 
\nbar $\to$ AP \nbar
\z
一个形容词短语(AP)可以包括一个NP和一个形容词,一个PP和一个形容词或者仅仅是一个形容词:
%An adjective phrase (AP) can consist of an NP and an adjective, a PP and an adjective or just an adjective:
\eal
\ex AP $\to$ NP A
\ex AP $\to$ PP A
\ex AP $\to$ A
\zl
就目前我们提出的规则而言,还有两点不如意的结果。就是例(\mex{0}c)和例(\ref{NP-Regeln-Nbar-N})中不带补足语的形容词或名词的规则,这里重复显示为例(\mex{1}):
%There are two imperfections resulting from the rules we have developed thus far. These are the rules for adjectives
%or nouns without complements in (\mex{0}c) as well as (\ref{NP-Regeln-Nbar-N}) -- repeated here as (\mex{1}):
\ea
\nbar $\to$ N
\z
如果我们应用这些规则,就会生成一元的子树,即其父结点只有一个子结点的树。如图\ref{Abbildung-NP-ohne-Det} 所示。如果我们援引盒子的比喻,这就意味着一个盒子中只装有一个盒子。
%If we apply these rules, then we will generate unary branching subtrees, that is trees with a mother that
%only has one daughter. See Figure~\ref{Abbildung-NP-ohne-Det} for an example of this. If we maintain the
%parallel to the boxes, this would mean that there is a box which contains another box which is the one with 
%the relevant content.

原则上来说,没有人能阻止我们将这一信息放进更大的盒子中。我们不用(\mex{1})的规则,只是简单应用(\mex{2})中的规则:
%In principle, nothing stops us from placing this information directly into the larger box. Instead of
%the rules in (\mex{1}), we will simply use the rules in (\mex{2}):
\eal
\ex A $\to$ kluge
\ex N $\to$ Mann
\zl
\eal
\label{Lexikon-Projektion}
\ex AP $\to$ kluge
\ex \nbar $\to$ Mann
\zl
例(\mex{0}a)说明了kluge(聪明)与完整的形容词短语具有相同的属性。特别是,它不能带补足语。这与(\ref{bsp-grammatik-psg})和(\ref{psg-binaer})的语法中,代词er(他)作为一个NP的范畴类型是一致的。
%(\mex{0}a) states that \emph{kluge}  `smart' has the same properties as a full adjective phrase, in particular that it cannot be combined
%with a complement. This is parallel to the categorization of the pronoun \emph{er} `he' as an NP in the grammars
%(\ref{bsp-grammatik-psg}) and (\ref{psg-binaer}).

将\nbarc 指派给不需要补足语的名词是有优势的,我们不需要解释为什么(\mex{1}b)和(\mex{1}a)中的分析是可能的,尽管二者在意义上没有什么分别。
%Assigning \nbar to nouns which do not require a complement has the advantage that we do not have to explain why the analysis in (\mex{1}b) is possible as well
%as (\mex{1}a) despite there not being any difference in meaning.
\eal
\ex 
\gll {}[\sub{NP} einige [\sub{\nbar} kluge [\sub{\nbar} [\sub{\nbar} [\sub{N} Frauen ] und  [\sub{\nbar} [\sub{N} Männer ]]]]]]\\
	 {}      一些   {}           聪明 {}          {}           {}       女人  {} 和 {} {}          男人\\
%	 {}      some   {}           smart {}          {}           {}       women  {} and {} {}          men\\
\ex 
\gll {}[\sub{NP} einige [\sub{\nbar} kluge [\sub{\nbar} [\sub{N} [\sub{N} Frauen ] und [\sub{N} Männer
]]]]]\\
	{}       一些   {}           聪明 {}          {}       {}       女人  {} 和 {} 男人\\
%	{}       some   {}           smart {}          {}       {}       women  {} and {} men\\
\zl
%
例(\mex{0}a)中,两个名词被投射到\nbarc,并且随后并列起来。属于同一范畴的两个成分并列的结果一定会是属于该范畴的一个新成分。在(\mex{0}a)的例子中,这也是\nbarc。这个成分随后与形容词和限定词相组合。在(\mex{0}b)中,名词本身是一个并列结构。并列的结果一定是跟其组成成分具有一样的范畴。在这个例子中,这个范畴是N。这个N变成\nbarc,然后与形容词相组合。如果不带补足语的名词被看作是\nbarc,而不是N,我们就会遇到伪歧义(spurious ambiguities)\isc{歧义!伪歧义}\is{ambiguity!spurious}问题。例(\mex{1})中的结构是唯一可能的分析结果。
%In (\mex{0}a), two nouns have projected to \nbar and have then been joined by coordination. The result of coordination
%of two constituents of the same category is always a new constituent with that category. In the case of (\mex{0}a), this
%is also \nbar. This constituent is then combined with the adjective and the determiner. In (\mex{0}b), the nouns themselves
%have been coordinated. The result of this is always another constituent which has the same category as its parts. In this case,
%this would be N. This N becomes \nbar and is then combined with the adjective. If nouns which do not require complements were
%categorized as \nbar rather than N, we would not have the problem of spurious ambiguities\is{ambiguity!spurious}. The
%structure in (\mex{1}) shows the only possible analysis.
\ea
\gll {}[\sub{NP} einige [\sub{\nbar} kluge [\sub{\nbar} [\sub{\nbar} Frauen ] und [\sub{\nbar} Männer
]]]]\\
      {}         一些    {}           聪明 {}          {}           女人  {}  和 {} 男人\\
%      {}	some    {}           smart {}          {}           women  {}  and {} men\\
\z

\subsection{介词短语}
%\subsection{Prepositional phrases}
\label{Abschnitt-PP-Syntax}
\isc{介词|(}\is{preposition|(}
与名词短语的句法相比,介词短语(PP)的句法相对来说就比较简单了。PP通常包括一个介词和一个名词短语,这个名词短语的格属性由介词决定。我们可以用下面的规则来表示:
%Compared to the syntax of noun phrases, the syntax of prepositional phrases (PPs) is relatively straightforward. PPs normally 
%consist of a preposition and a noun phrase whose case is determined by that preposition. We can capture this with the following
%rule:
\ea
\label{Regel-PP-einfach}
PP $\to$ P NP
\z
当然,这个规则也必须包括NP的格信息。为了方便,我在上面的NP"=规则和 AP"=规则中省略了格信息。
%This rule must, of course, also contain information about the case of the NP. I have omitted this for ease of exposition as I did
%with the NP"=rules and AP"=rules above.

下面(\mex{1})中的语法选自《杜登语法》\citep[\S 1300]{Duden2005-Authors},这里的介词短语进一步界定了介词所述方式的语义,如下所示:
%The Duden grammar \citep[\S 1300]{Duden2005-Authors} offers examples such as those in (\mex{1}), which show that certain prepositional phrases
%serve to further define the semantic contribution of the preposition by indicating some measurement, for example:
\eal
\ex\label{Beispiel-Schritt-vor-dem-Abgrund} 
\gll {}[[Einen Schritt] vor dem Abgrund] blieb er stehen.\\
	 {}\spacebr{}\spacebr{}一 步 \textsc{prep} \textsc{det} 深渊 保留 他 站\\
\glt `他在离深渊一步之遥的地方站住了。'
\ex 
\gll {}[[Kurz] nach dem Start] fiel die Klimaanlage aus.\\
	 {}\spacebr{}\spacebr{}短暂 \textsc{prep} \textsc{det}  起飞 坏 \textsc{det} 空调 出来\\
\glt `起飞后不久,空调就停止工作了。'
\ex 
\gll {}[[Schräg] hinter der Scheune] ist ein Weiher.\\
	 {}\spacebr{}\spacebr{}斜 \textsc{prep} \textsc{det} 谷仓 \textsc{cop} 一 池塘\\
\glt `在谷仓的斜对面有一个池塘。'
\ex 
\gll {}[[Mitten] im Urwald] stießen die Forscher auf einen alten Tempel.\\
	 {}\spacebr{}\spacebr{}中间 \textsc{prep}.\textsc{det} 丛林 碰到 \textsc{det} 研究人员 \textsc{prep} 一 古老 寺庙\\
\glt `在丛林当中,研究者碰到了一栋古老的寺庙。'
\zl
%\eal
%\ex\label{Beispiel-Schritt-vor-dem-Abgrund} 
%\gll {}[[Einen Schritt] vor dem Abgrund] blieb er stehen.\\
%	 {}\spacebr{}\spacebr{}one step before the abyss remained he stand\\
%\glt `He stopped one step in front of the abyss.'
%\ex 
%\gll {}[[Kurz] nach dem Start] fiel die Klimaanlage aus.\\
%	 {}\spacebr{}\spacebr{}shortly after the take.off fell the air.conditioning out\\
%\glt `Shortly after take off, the air conditioning stopped working.'
%\ex 
%\gll {}[[Schräg] hinter der Scheune] ist ein Weiher.\\
%	 {}\spacebr{}\spacebr{}diagonally behind the barn is a pond\\
%\glt `There is a pond diagonally across from the barn.'
%\ex 
%\gll {}[[Mitten] im Urwald] stießen die Forscher auf einen alten Tempel.\\
%	 {}\spacebr{}\spacebr{}middle in.the jungle stumbled the researchers on an old temple\\
%\glt `In the middle of the jungle, the researches came across an old temple.'
%\zl
为了分析例(\mex{0}a、b)中的句子,我们可以提出下面这些规则,如(\mex{1})所示:
%To analyze the sentences in (\mex{0}a,b), one could propose the following rules in (\mex{1}):
\eal
\ex PP $\to$ NP PP
\ex PP $\to$ AP PP
\zl
这些规则利用PP来表示动作的方式。最后得到的成分是另一个PP。我们有必要在(\mex{-1}a、b)中利用这些规则来分析介词短语,但是我们无法应用这些规则来分析(\mex{1})中的句子:
%These rules combine a PP with an indication of measurement. The resulting constituent is another PP. It is
%possible to use these rules to analyze prepositional phrases in (\mex{-1}a,b), but it unfortunately also allows
%us to analyze those in (\mex{1}):
\eal
\ex[*]{
\gll [\sub{PP} einen Schritt [\sub{PP} kurz [\sub{PP} vor dem Abgrund]]]\\
	 {} 一 步 {} 短 {} \textsc{prep} \textsc{det}  深渊\\
	 %	 {} one step {} shortly {} before the abyss\\
}
\ex[*]{
\gll [\sub{PP} kurz [\sub{PP} einen Schritt [\sub{PP} vor dem Abgrund]]]\\
	 {} 短 {} 一 步 {} \textsc{prep} \textsc{det}  深渊\\
%	 {} shortly {} one step {} before the abyss\\
}
\zl
(\mex{-1})中的所有规则被用来分析(\mex{0})中的例子。由于在规则的左边和右边都有PP,我们可以按照我们所希望的任意顺序和次数来应用这些规则。
%Both rules in (\mex{-1}) were used to analyze the examples in (\mex{0}). Since the symbol PP occurs on both the left 
%and right"=hand side of the rules, we can apply the rules in any order and as many times as we like.
% Fn: Semantik hilft nicht.

为了避免产生这些不需要的副作用,我们可以修改前面提出的规则:
%We can avoid this undesired side"=effect by reformulating the previously assumed rules:
\eal
\ex PP $\to$ NP \pbar
\ex PP $\to$ AP \pbar
\ex PP $\to$ \pbar\label{Regel-PP-P}
\ex \pbar $\to$ P NP
\zl
规则(\ref{Regel-PP-einfach})变成了(\mex{0}d)。(\mex{0}c)中的规则证明了PP可以包括\pbarc。图\ref{Abbildung-PP}显示了应用(\mex{0}c)和(\mex{0}d)对(\mex{1})的分析,以及在规则(\mex{0}b)和(\mex{0}d)的约束下形容词位于首位的例子:
%规则(\ref{Regel-PP-einfach})变成了(\mex{0}d)。(\mex{0}c)中的规则证明了PP可以包括\pbar 。图%\vref{Abbildung-PP}显示了应用(\mex{0}c)和(\mex{0}d)对(\mex{1})的分析,以及在规则(\mex{0}b)和(\mex{0}%d)的约束下形容词位于首位的例子:
%Rule (\ref{Regel-PP-einfach}) becomes (\mex{0}d). The rule in (\mex{0}c) states that a PP can consist of \pbar.
%Figure~\vref{Abbildung-PP} shows the analysis of (\mex{1}) using (\mex{0}c) and (\mex{0}d) as well as the analysis
%of an example with an adjective in the first position following the rules in (\mex{0}b) and (\mex{0}d):
\ea
\gll vor dem Abgrund\\
	 \textsc{prep} \textsc{det} 深渊\\
\glt `在深渊前'
%	 before the abyss\\
%\glt `in front of the abyss'
\z
\begin{figure}
\hfill
\begin{forest}
sm edges
[PP
  [\pbar
    [P [vor;\textsc{prep}] ]
    [NP [dem Abgrund;\textsc{det} 深渊, roof] ] ] ]
\end{forest}
\hfill
\begin{forest}
sm edges
[PP
  [AP [kurz;短暂,roof] ]
  [\pbar
    [P [vor;\textsc{prep}] ]
    [NP [dem Abgrund;\textsc{det} 深渊,roof] ] ] ]
\end{forest}
%% \begin{tikzpicture}
%% \tikzset{level 1+/.style={level distance=2\baselineskip}}
%% \tikzset{frontier/.style={distance from root=8\baselineskip}}
%% \Tree[.PP
%%        [.{\pbar}
%%          [.P vor;before ]
%%          [.NP \edge[roof]; {dem Abgrund;the abyss} ] ] ] 
%% \end{tikzpicture}
%% \hfill
%% \begin{tikzpicture}
%% \tikzset{level 1+/.style={level distance=2\baselineskip}}
%% \tikzset{frontier/.style={distance from root=8\baselineskip}}
%% \Tree[.PP
%%        [.AP \edge[roof]; {kurz;shortly} ]
%%        [.{\pbar}
%%          [.P vor;before ]
%%          [.NP \edge[roof]; {dem Abgrund;the abyss} ] ] ] 
%% \end{tikzpicture}
\hfill
\mbox{}
  \caption{\label{Abbildung-PP}带和不带量度的介词短语}
%\caption{\label{Abbildung-PP}Prepositional phrases with and without measurement}
\end{figure}%

有读者可能对图\ref{Abbildung-PP}有疑问,为什么左图中没有表示空量度的短语?这与图\ref{Abbildung-NP-ohne-Det}中所示的空限定词是相似的。图\ref{Abbildung-NP-ohne-Det}中的空限定词是指没有限定词的名词短语与有限定词的名词短语表示的意义是一样的。通常由可见的限定词表示的意义以某种方式整合进名词短语的结构之中了。如果我们在空限定词中没有表示这层意思的话,将会导致有关语义整合的更为复杂的问题:我们只是需要了解\ref{sec-PSG-Semantik}表示的机制,而且这些信息是比较普遍的。意义是由这些词本身决定的,而不是由规则决定的。如果我们假设在图\ref{Abbildung-PP}中左边的那棵树那样有一条一元的规则,而不是空限定词的话,那么这个一元规则就应该提供限定词的语义。有些学者提出了这种分析。更多的有关空成分的内容参看第\ref{Abschnitt-Diskussion-leere-Elemente}章。
%At this point, the attentive reader is probably wondering why there is no empty measurement phrase in
%the left figure of Figure~\ref{Abbildung-PP}, which one might expect in analogy to the empty determiner in Figure~\ref{Abbildung-NP-ohne-Det}.
%The reason for the empty determiner in Figure~\ref{Abbildung-NP-ohne-Det} is that the entire noun phrase
%without the determiner has a meaning similar to those with a determiner. The meaning normally contributed
%by the visible determiner has to somehow be incorporated in the structure of the noun phrase. If we
%did not place this meaning in the empty determiner, this would lead to more complicated assumptions about semantic
%combination: we only really require the mechanisms presented in Section~\ref{sec-PSG-Semantik} and these are
%very general in nature. The meaning is contributed by the words themselves and not by any rules. If we were
%to assume a unary branching rule such as that in the left tree in Figure~\ref{Abbildung-PP} instead of the
%empty determiner, then this unary branching rule would have to provide the semantics of the determiner. This
%kind of analysis has also been proposed by some researchers.\todostefan{Martin: obscure} See Chapter~\ref{Abschnitt-Diskussion-leere-Elemente} for
%more on empty elements.

与没有限定词的NP不同的是,不带有关程度或方式的介词短语并不缺少任何意义的组成部分。所以说,我们没有必要假设有一个表示量度的空成分,并对整个PP有语义贡献。由此,(\ref{Regel-PP-P})中的规则说明了介词短语包括\pbarc,即由P和NP构成。\isc{介词|)}\is{preposition|)}
%Unlike determiner"=less NPs, prepositional phrases without an indication of degree or measurement do
%not lack any meaning component for composition. It is therefore not necessary to assume an empty indication of measurement, which
%somehow contributes to the meaning of the entire PP. Hence, the rule in (\ref{Regel-PP-P}) states that a
%prepositional phrase consists of \pbar, that is, a combination of P and NP.\is{preposition|)}

\section{\xbar 理论}
%\section{\xbart}
\label{sec-xbar}

如果我们回顾前面章节\isc{X 理论@\xbar 理论|(}\is{X theory@\xbar theory|(}中介绍的规则,我们会发现中心语永远和他们的补足语相组合,以构成一个新的成分(\mex{1}a、b),然后再与其它成分(\mex{1}c、d)相组合:
%If\is{X theory@\xbar theory|(}  we look again at the rules that we have formulated in the previous section, we see that heads are always 
%combined with their complements to form a new constituent (\mex{1}a,b), which can then be combined with further constituents (\mex{1}c,d):

\eal
\ex \nbar $\to$ N PP
\ex \pbar $\to$ P NP
\ex\label{Regel-NP-Xbar}
    NP $\to$ Det \nbar
\ex PP $\to$ NP \pbar
\zl
%
研究英语的语法学家发现,平行结构可以用来表示那些以形容词或动词作为中心语的短语。在这里,我先讨论形容词短语,之后在第\ref{Kapitel-GB}章讨论动词短语。与德语一样,英语中可以带补足语的形容词是有严格限制的,即这些带补足语的形容词短语在英语中不能出现在名词性成分之前。如下例(\mex{1})所示:
%Grammarians working on English\il{English} noticed that parallel structures can be used for phrases which have adjectives or verbs as their head.
%I discuss adjective phrases at this point and postpone the discussion of verb phrases to Chapter~\ref{Kapitel-GB}. As in German, certain adjectives 
%in English can take complements with the important restriction that adjective phrases with complements cannot realize these pre"=nominally in English. 
%(\mex{1}) gives some examples of adjective phrases:
\eal
\ex 
\gll He is proud.\\
     他 \textsc{cop} 骄傲\\
\glt `他感到很骄傲。'
\ex 
\gll He is very proud.\\
他 \textsc{cop} 非常 骄傲\\
\glt `他感到非常骄傲。'
\ex 
\gll He is proud of his son.\\
他 \textsc{cop} 骄傲 \textsc{prep} 他的 儿子\\
\glt `他为他的儿子感到骄傲。'
\ex 
\gll He is very proud of his son.\\
他 \textsc{cop} 非常 骄傲 \textsc{prep} 他的 儿子\\
\glt `他为他的儿子感到非常骄傲。'
\zl
与介词短语不同的是,带补足语的形容词通常是可选的。proud(骄傲)可以带或不带PP。表达程度的very(很)也是可选的。
%Unlike prepositional phrases, complements of adjectives are normally optional. \emph{proud} can be used with or without a PP.
%The degree expression \emph{very} is also optional.

(\mex{1})中有针对这一分析的规则,在图\ref{Abbildung-AP}中显示了相应的结构。
%(\mex{1})中有针对这一分析的规则,在图\vref{Abbildung-AP}中显示了相应的结构。
%The rules which we need for this analysis are given in (\mex{1}), with the corresponding structures in Figure~\vref{Abbildung-AP}.

\begin{samepage}
\eal
\ex AP $\to$ \abar
\ex AP $\to$ AdvP \abar
\ex \abar $\to$ A PP
\ex \abar $\to$ A
\zl
\end{samepage}

\begin{figure}
\hfill
\begin{forest}
sm edges
[AP
  [\abar
    [A [proud;骄傲] ] ] ]
\end{forest}
%% \begin{tikzpicture}
%% \tikzset{level 1+/.style={level distance=2\baselineskip}}
%% \tikzset{frontier/.style={distance from root=6\baselineskip}}
%% \Tree[.AP
%%        [.{\abar}
%%          [.A proud ] ] ]
%% \end{tikzpicture}
\hfill
\begin{forest}
sm edges
[AP
  [AdvP [very;非常] ]
  [\abar
    [A [proud;骄傲] ] ] ]
\end{forest}
%% \begin{tikzpicture}
%% \tikzset{level 1+/.style={level distance=2\baselineskip}}
%% \tikzset{frontier/.style={distance from root=6\baselineskip}}
%% \Tree[.AP
%%        [.AdvP very ]
%%        [.{\abar}
%%          [.A proud ] ] ]
%% \end{tikzpicture}
\hfill
\begin{forest}
sm edges
[AP
  [\abar
    [A [proud;骄傲] ]
    [PP [of his son;\textsc{prep} 他的 儿子,roof] ] ] ]
\end{forest}
%% \begin{tikzpicture}
%% \tikzset{level 1+/.style={level distance=2\baselineskip}}
%% \tikzset{frontier/.style={distance from root=6\baselineskip}}
%% \Tree[.AP
%%        [.{\abar}
%%          [.A proud ]
%%          [.PP \edge[roof]; {of his son} ] ] ]
%% \end{tikzpicture}
\hfill
\begin{forest}
sm edges
[AP
  [AdvP [very;非常] ]
  [\abar
    [A [proud;骄傲] ]
    [PP [of his son;\textsc{prep} 他的 儿子,roof] ] ] ]
\end{forest}
%% \begin{tikzpicture}
%% \tikzset{level 1+/.style={level distance=2\baselineskip}}
%% \tikzset{frontier/.style={distance from root=6\baselineskip}}
%% \Tree[.AP
%%        [.AdvP very ]
%%        [.{\abar}
%%          [.A proud ]
%%          [.PP \edge[roof]; {of his son} ] ] ]
%% \end{tikzpicture}
\hfill
\mbox{}
\caption{\label{Abbildung-AP}英语的形容词短语}
%\caption{\label{Abbildung-AP}English adjective phrases}
\end{figure}%

正如在\ref{sec-PSG-Merkmale}中所显示的,我们可以对具体的短语结构规则进行扩展,并最终得到更具有普遍性的规则。按照这一方式,诸如人称、数和性这些属性不再按照范畴符号来进行编码,而是采用更为简单的符号,如NP、Det和N。我们只需要明确那些在给定规则的语境中相关的特征的值。我们可以进一步抽象为:与其采用明确的范畴符号,如词范畴的N、V、P和A,以及短语范畴的NP、VP、PP和AP,不如简单地采用词类的变量,并用X和XP来表示。
%As was shown in Section~\ref{sec-PSG-Merkmale}, it is possible to generalize over very specific
%phrase structure rules and thereby arrive at more general rules. In this way, properties such as
%person, number and gender are no longer encoded in the category symbols, but rather only simple
%symbols such as NP, Det and N are used. It is only necessary to specify something about the values
%of a feature if it is relevant in the context of a given rule. We can take this abstraction a step
%further: instead of using explicit category symbols such as N, V, P and A for lexical categories and
%NP, VP, PP and AP for phrasal categories, one can simply use a variable for the word class in question and speak of X and XP.

这种抽象的方式就叫做\xbarc 理论 (或者X"=杠理论)。该理论由\citet{Chomsky70a}提出,并由\citet{Jackendoff77a}完善。这种形式的抽象规则在很多不同的理论中发挥了重要的作用。比如说:管辖与约束理论(第\ref{Kapitel-GB}章)、扩展的短语结构语法(第\ref{Kapitel-GPSG}章)和词汇功能语法(第\ref{Kapitel-LFG}章)。在中心语驱动的短语结构语法(第\ref{Kapitel-HPSG}章)中,\xbarc 理论也起到了重要的作用,但是并没有采纳\xbarc 范式的所有限制条件。
%This form of abstraction can be found in so"=called \xbart (or X"=bar theory, the term \emph{bar}
%refers to the line above the symbol), which was developed by \citet{Chomsky70a} and refined by
%\citet{Jackendoff77a}. This form of abstract rules plays an important role in many different
%theories. For example: Government \& Binding\indexgb (Chapter~\ref{Kapitel-GB}), Generalized Phrase
%Structure Grammar\indexgpsg (Chapter~\ref{Kapitel-GPSG}) and Lexical Functional Grammar\indexlfg
%(Chapter~\ref{Kapitel-LFG}). In HPSG\indexhpsg (Chapter~\ref{Kapitel-HPSG}), \xbar theory also plays
%a role, but not all restrictions of the \xbar schema have been adopted.

例(\mex{1})说明了应用\xbarc 规则的可能的实例, X范畴被放在了N的位置上,就好像这些词串的例子都是由这些规则生成的:
%(\mex{1}) shows a possible instantiation of \xbar rules, where the category X has been used in place of N, as well as examples of word strings
%which can be derived by these rules:
%\todostefan{martin meint ich solle hier NP statt Strichnotation verwenden}
\eanoraggedright
\label{psg-xbar-schema}
\begin{tabular}[t]{@{}l@{\hspace{5mm}}l@{\hspace{5mm}}l@{}}
\xbar\mbox{ 规则} & \mbox{带有特定的范畴} & \mbox{例子}\\[2mm]
%\xbar\mbox{ rule} & \mbox{with specific categories} & \mbox{example strings}\\[2mm]
$\overline{\overline{\mbox{X}}} \rightarrow \overline{\overline{\mbox{specifier}}}$~~\xbar &
$\overline{\overline{\mbox{N}}} \rightarrow \overline{\overline{\mbox{\textsc{det}}}}$~~\nbar & \mbox{the [picture of Paris]} \\
$\xbar \rightarrow$ \xbar~~$\overline{\overline{\mbox{adjunct}}}$            & \nbar $\rightarrow$ \nbar~~$\overline{\overline{\mbox{REL\_CLAUSE}}}$ & \mbox{[picture of Paris]}\\
                            &                                              & \mbox{[that everybody knows]}\\
\xbar $\rightarrow \overline{\overline{\mbox{adjunct}}}$~~\xbar            & \nbar $\rightarrow \overline{\overline{\mbox{A}}}$~~\nbar & \mbox{beautiful [picture of Paris]}\\
\xbar $\rightarrow$ \mbox{X}~~$\overline{\overline{\mbox{complement}}}*$   & \nbar $\rightarrow$ \mbox{N}~~$\overline{\overline{\mbox{P}}}$ & \mbox{picture [of Paris]}\\
\end{tabular}
\z

任何一个词类都可以替代X(如V、A或P)。在上面的规则中,没有横杠的X表示一个词汇项。如果想把横杠明确地显示出来,就需要写成\xnullc。正如(\ref{Regel-mit-Variablen})中的规则所示,我们并没有明确指定限定词或名词的格的值,而是简单地要求在规则右边的值与之相符,(\mex{0})中的规则要求规则右边(X 或 \xbarc)的元素的词类与规则左边(\xbarc 或 $\overline{\overline{\mbox{X}}}$)的元素的词类是一致的。
%Any word class can replace X (\eg V, A or P). The X without the bar stands for a lexical item in
%the above rules. If one wants to make the bar level explicit, then it is possible to write \xnull. 
%Just as with the rule in (\ref{Regel-mit-Variablen}), where we did not specify the case value of the
%determiner or the noun but rather simply required that the values on the right"=hand side of the
%rule match, the rules in (\mex{0}) require that the word class of an element on the right"=hand side
%of the rule (X or \xbar) matches that of the element on the left"=hand side of the rule (\xbar or
%$\overline{\overline{\mbox{X}}}$).

一个词汇项能跟它的补足语\isc{补足语}\is{complement}相组合。最后一条规则中的`*'\isc{*}\is{*}表示它后面的符号可以不受限制地重复。一个特殊的情况是补足语的缺失。在das Bild(这张图片)中, Bild(图片)没有PP担当补足语,由此变成了\nbarc。词汇项及其补足语结合的结果是X的一个新的投射层:投射层1,并由一个横杠来表示。之后,\xbarc 可以跟附接语相组合。这一过程发生在\xbarc 的左边或右边。这一组合过程的结果仍然是一个\xbarc,也就是说,将一个附接语\isc{附接语}\is{adjunct}组合进来并没有改变投射层。最大投射由两个横杠表示。我们也可以将XP写成带有两个横杠的X。一个XP包括一个指定语\isc{指定语}\is{specifier}和\xbarc。在有些理论中,句子中的主语和NP中的限定词都是指定语。进而,形容词短语中的程度修饰词和介词短语中的方式指示词也被算作是指定语。
%A lexical element can be combined with all its complements\is{complement}. The `*'\is{*} in the last rule stands for
%an unlimited amount of repetitions of the symbol it follows. A special case is zero"fold occurrence of complements. There is no
%PP complement of \emph{Bild} `picture' present in \emph{das Bild} `the picture' and thus N becomes \nbar. The result of the
%combination of a lexical element with its complements is a new projection level of X: the projection level 1, which is marked by
%a bar. \xbar can then be combined with adjuncts. These can occur to the left or right of \xbar. The result of this combination is
%still \xbar, that is the projection level is not changed by combining it with an adjunct\is{adjunct}.
%Maximal projections are marked by two
%bars. One can also write XP\is{XP} for a projection of X with two bars. An XP consists of a specifier\is{specifier} and \xbar. Depending
%on one's theoretical assumptions, subjects of sentences (\citealp{Haider95b-u,Haider97a}\todostefan{Haider noch mal lesen};
%\citealp[Section~3.2.2]{Berman2003a}) and determiners in NPs \citep[\page
%210]{Chomsky70a} are specifiers. Furthermore, degree modifiers \citep[\page
%210]{Chomsky70a} in adjective phrases and measurement indicators in prepositional phrases are also counted as specifiers.

非中心语的位置只能满足最大投射,所以补足语、附接语和指定语总是有两个横杠。
图\ref{Abb-GB-Min-Max}显示了短语结构的最小结构和最大结构。
%图\vref{Abb-GB-Min-Max}显示了短语结构的最小结构和最大结构。
%Non"=head positions can only host maximal projections and therefore complements, adjuncts and specifiers always have two bars. 
%Figure~\vref{Abb-GB-Min-Max} gives an overview of the minimal and maximal structure of phrases.
\begin{figure}
\hfill
\begin{forest}
sm edges
[XP
  [\xbar [X] ] ]
\end{forest}
\hfill
\begin{forest}
%where n children=0{}{},
%sm edges
%for tree={parent anchor=south, child anchor=north,align=center,base=bottom}
[XP
  [指定语(specifier)]
%  [specifier]
  [\xbar
    [附接语(adjunct)]
%    [adjunct]
    [\xbar
      [补足语(complement)] [X] ] ] ]
%      [complement] [X] ] ] ]
\end{forest}
\hfill\mbox{}
\caption{\label{Abb-GB-Min-Max}最小和最大的短语结构}
%\caption{\label{Abb-GB-Min-Max}Minimal and maximal structure of phrases}
\end{figure}%

有些范畴没有指定语或者只有一个。附接语也是可选的,所以,并不是所有结构都包括带有附接语子结点的\xbarc。除了右手边图上的分支所示的,带附接语的XP和中心语"=附接语\isc{附接语!中心语-附接语}\is{adjunct!head}有时也是成立的。在(\ref{psg-xbar-schema})中,只有一条规则是针对中心语在补足语之前的情况,当然,补足语在中心语之前的顺序也是可能的,如图\ref{Abb-GB-Min-Max}所示。
%Some categories do not have a specifier or have the option of having one. Adjuncts are optional and therefore
%not all structures have to contain an \xbar with an adjunct daughter.
% In addition to the branching shown in the right"=hand figure, adjuncts to
%XP and head"=adjuncts\is{adjunct!head} are sometimes possible. There is only a single rule in (\ref{psg-xbar-schema})
%for cases in which a head precedes the complements, however an order in which the complement precedes the head is
%of course also possible. 
%This is shown in Figure~\ref{Abb-GB-Min-Max}.

图\ref{Abb-das-schoene-Bild-von-Paris}显示了对NP结构das Bild(这张图片)和das schöne Bild von Paris(这张巴黎的美丽图片)的分析。
%图\vref{Abb-das-schoene-Bild-von-Paris}显示了对NP结构das Bild(这张图片)和das schöne Bild von Paris(这张巴黎%的美丽图片)的分析。
图\ref{Abb-das-schoene-Bild-von-Paris}中的NP结构和图\ref{Abbildung-AP}中的树形图都是最小产出结构的例子。图\ref{Abb-das-schoene-Bild-von-Paris}中左边的树形图也是没有附接语的结构的一个例子。图\ref{Abb-das-schoene-Bild-von-Paris}中的右手结构是最大产出结构的一个例子:指定语、附接语和补足语都是齐全的。
%Figure~\vref{Abb-das-schoene-Bild-von-Paris} shows the analysis of the NP structures \emph{das Bild} `the picture'
%and \emph{das schöne Bild von Paris} `the beautiful picture of Paris'. The NP structures in Figure~\ref{Abb-das-schoene-Bild-von-Paris}
%and the tree for \emph{proud} in Figure~\ref{Abbildung-AP} show examples of minimally populated structures.
%The left tree in Figure~\ref{Abb-das-schoene-Bild-von-Paris} is also an example of a structure without an adjunct. The right"=hand structure
%in Figure~\ref{Abb-das-schoene-Bild-von-Paris} is an example for the maximally populated structure:
%specifier, adjunct, and complement are present.

\begin{figure}
\hfill
\begin{forest}
sm edges
[NP
  [DetP
    [\detbar
      [Det [das;\textsc{det}] ] ] ]
  [\nbar
    [N [Bild;图片] ] ] ]
\end{forest}
\hfill
\begin{forest}
sm edges
[NP
  [DetP
    [\detbar
      [Det [das;\textsc{det}] ] ] ]
  [\nbar
    [AP
      [\abar
        [A [schöne;美丽] ] ] ]
    [\nbar
      [N [Bild;图片] ]
      [PP 
        [\pbar
          [P [von;\textsc{prep}] ]
          [NP
            [\nbar
              [N [Paris;巴黎] ] ] ] ] ] ] ] ]
\end{forest}
%
\hfill\mbox{}
\caption{\label{Abb-das-schoene-Bild-von-Paris}das Bild(这张图片)和das schöne Bild von Paris(这张巴黎的美丽图片)的\xbarc 分析}
%\caption{\label{Abb-das-schoene-Bild-von-Paris}\xbar~analysis of \emph{das Bild} `the picture'
%  and \emph{das schöne Bild von Paris} `the beautiful picture of Paris'}
\end{figure}%

图\ref{Abb-das-schoene-Bild-von-Paris}中的分析说明了规则中的所有非中心语都是短语。这样,我们就需要认为这是一个限定短语,即使限定词没有跟其他元素相组合。限定词的一元结构并不好看,但是它是自足的。\footnote{%
有的\xbarc 理论并不认为限定词有复杂的结构,请看\citew{Muysken82a}。
}
%The analysis given in Figure~\ref{Abb-das-schoene-Bild-von-Paris} assumes that all non"=heads in a rule are
%phrases. One therefore has to assume that there is a determiner phrase even if the determiner is not combined with other elements.
%The unary branching of determiners is not elegant but it is consistent.\footnote{%
%	For an alternative version of \xbar theory which does not assume elaborate structure for determiners see \citew{Muysken82a}.
%}
在图\ref{Abb-das-schoene-Bild-von-Paris}中,Paris(巴黎)这个NP的一元分支看起来也有点奇怪,但是如果我们把它放进更为复杂的名词短语之中, 它们就不那么奇怪了:
%The unary branchings for the NP \emph{Paris} in Figure~\ref{Abb-das-schoene-Bild-von-Paris} may also seem somewhat odd, but they actually become more
%plausible when one considers more complex noun phrases:
\eal
\ex 
\gll das Paris der dreißiger Jahre\\
	 \textsc{det} 巴黎 \textsc{det} 三十 年\\
\glt `30年代的巴黎'
%	 the Paris of.the thirty years\\
%\glt `30's Paris'
\ex 
\gll die Maria aus Hamburg\\
	 \textsc{det} Maria \textsc{prep} 汉堡\\
\glt `来自汉堡的Maria'
%	 the Maria from Hamburg\\
%\glt `Maria from Hamburg'
\zl
虽然一元投射有些不够优雅,但是我们不用过于担心,因为在(\ref{Lexikon-Projektion})有关词汇项的讨论中,大部分情况下一元分支结点都是可以避免的,或者说我们争取避免这样的结构。如果不这样做的话,我们就会得到伪歧义结构\isc{歧义!伪歧义}\is{ambiguity!spurious}。在下面的章节中,我们将讨论范畴语法和中心语驱动的短语结构语法,这些语法理论没有针对限定词、形容词和名词的一元规则。
%Unary projections are somewhat inelegant but this should not concern us too much here, as we have
%already seen in the discussion of the lexical entries in (\ref{Lexikon-Projektion})
%that unary branching nodes can be avoided for the most part and that it is indeed desirable to avoid
%such structures. Otherwise, one gets spurious ambiguities\is{ambiguity!spurious}. In the following
%chapters, we will discuss approaches such as Categorial Grammar and HPSG, which do not assume
%unary rules for determiners, adjectives and nouns. 

再者,其他的\xbarc 理论与本书所讨论的理论也有不同的看法。特别是,我们将不予理会非中心语永远作为最大投射\isc{投射!最大投射}\is{projection!maximal}这一条假设。\citet{Pullum85a}和\citet{KP90a}认为,有些理论并不一定没有严格的\xbarc 理论的限制多。也可以参考\ref{sec-Diskussion-X-Bar}中的讨论。
%Furthermore, other \xbar~theoretical assumptions will not be shared by several theories discussed in this book. In particular, the assumption that non"=heads always have
%to be maximal projections\is{projection!maximal} will be disregarded. \citet{Pullum85a} and
%\citet{KP90a} have shown that the respective theories are not necessarily less restrictive
%than theories which adopt a strict version of the \xbar theory. See also the discussion in Section~\ref{sec-Diskussion-X-Bar}.
\isc{X 理论@\xbar 理论|)}\is{X theory@\xbar theory|)}

%\section*{思考题}
%\section*{Comprehension questions}

\bigskip
\questions{
\begin{enumerate}
\item 为什么只用原子式范畴的短语结构语法不足以描写自然语言呢?
\item 让我们来看一下(\ref{psg-binaer})中的文法,为了分析到例(\mex{1})中的符号V,我们应该采取哪些(替换符号的)步骤?
%\item Why are phrase structure grammars that use only atomic categories inadequate for the description of natural languages?
%\item Assuming the grammar in (\ref{psg-binaer}), state which steps (replacing symbols) one has to take to get to the symbol 
%	 V in the sentence (\mex{1}).
\ea
\gll er das Buch dem Mann gibt\\
	 他 \textsc{det} 书 \textsc{det} 人 给\\
\glt `他给这个人这本书。'
%	 he the book the man gives\\
%\glt `He gives the book to the man.'
\z
你的回答应该与(\ref{bsp-anwendung-grammatik})中的分析保持一致。
%Your answer should resemble the analysis in (\ref{bsp-anwendung-grammatik}).
%\pagebreak
\item 请应用谓词逻辑来表示例(\mex{1})的意义:
%\item Give a representation of the meaning of (\mex{1}) using predicate logic:
\eal
\ex 
\gll Ulrike kennt Hans.\\
%	 Ulrike knows Hans\\
	 Ulrike 认识 Hans\\
\ex 
\gll Joshi freut sich.\\
	 Joshi 高兴 \refl{}\\
\glt `Joshi很高兴。'
%	 Joshi is.happy \refl{}\\
%\glt `Joshi is happy.'
\zl
\end{enumerate}
}

%\section*{练习题}
%\section*{Exercises}

\exercises{
\begin{enumerate}
\item\label{ua-psg-eins}
在第\pageref{page-unendlich-viele-grammatiken}页,我认为句子(\ref{bsp-weil-er-das-buch-dem-mann-gibt})的分析可以有无限多种可能的文法。为什么这种说法是正确的?
%\item\label{ua-psg-eins}
%	On page~\pageref{page-unendlich-viele-grammatiken}, I claimed that there is an infinite number of
%	grammars we could use to analyze (\ref{bsp-weil-er-das-buch-dem-mann-gibt}).
%	Why is this claim correct?
\item 请说明哪个文法或哪些文法是最好的?
%\item Try to come up with some ways in which we can tell which of these possible grammars is or are the best?

\item\label{uebung-np-empty} 在\ref{sec-psg-np}中,我们给出了名词短语的一部分句法。为什么说例(\mex{1})中的规则是有问题的?
%\item\label{uebung-np-empty} A fragment for noun phrase syntax was presented in Section~\ref{sec-psg-np}.
%Why is the interaction of the rules in (\mex{1}) problematic?
\eal
\ex NP $\to$ Det \nbar
\ex \nbar $\to$ N
\ex Det $\to$ $\epsilon$
\ex N $\to$ $\epsilon$
\zl

\item 为什么在词典中把books标记为NP是不合适的?
%\item Why is it not a good idea to mark \emph{books} as NP in the lexicon?

\item 为什么对于books这类名词来说,下面的规则并不适用?
%\item Can you think of some reasons why it is not desirable to assume the following rule for nouns such as \emph{books}:
\ea
NP $\to$ Modifier* books Modifier*
\z

(\mex{0})中的规则说明了有无限数量的修饰语与名词books相组合,并且其后也有无限数量的修饰语。我们可以应用这个规则来生成例(\mex{1})中的短语:
%The rule in (\mex{0}) combines an unlimited number of modifiers with the noun \emph{books} followed by an unlimited number
%of modifiers. We can use this rule to derive phrases such as those in (\mex{1}):
\eal
\ex 
\gll books\\
书\\
\glt `书'
\ex 
\gll interesting books\\
有趣 书\\
\glt `有趣的书'
\ex 
\gll interesting books from Stuttgart\\
有趣 书 \textsc{prep} 斯图加特\\
\glt `斯图加特的有趣的书'
\zl

请在你的回答中考虑到并列的语料。请假设对称的并列结构\isc{并列}\is{coordination}要求其并列的短语或词具有相同的句法属性。
%Make reference to coordination data in your answer. Assume that symmetric coordination\is{coordination} requires that both
%coordinated phrases or words have the same syntactic category.

\item \citet{FLGR2012a}提出,对于例(\mex{1})中的无名词结构,应该将其看作是由限定词the和形容词相组合的短语结构。
%\item \citet{FLGR2012a} suggested treating nounless structures like those in (\mex{1}) as involving
%  a phrasal construction combining the determiner \emph{the} with an adjective.
\eal
\ex 
\gll Examine the plight of the very poor.\\
检查 \textsc{det} 困境 \textsc{prep} \textsc{det} 非常 穷\\
\glt `了解穷人的困境'
\ex 
\gll Their outfits range from the flamboyant to the functional.\\
他们的 套装 范围 \textsc{prep} \textsc{det} 华丽 \textsc{prep} \textsc{det} 功能性\\
\glt `他们的套装从华丽的到功能性的都有。'
\ex 
\gll The unimaginable happened.\\
\textsc{det} 不可想象的 发生\\
\glt `不可思议的事情发生了。'
\zl
例(\mex{1})说明了对应于其结构的短语结构规则:
%(\mex{1}) shows a phrase structure rule that corresponds to their construction:
\ea
NP $\to$ the Adj
\z
Adj表示poor这样的单个词或者very poor这样的复合体。
%Adj stands for something that can be a single word like \emph{poor} or complex like \emph{very
%  poor}.

再来看(\ref{ex-nounless-np})和(\ref{ex-nounless-np-relational-noun})中的德语语料,并说明为什么这么分析,而且为什么例(\mex{1})中的更为普遍的分析不能扩展到德语的应用之中。
%Revisit the German data in (\ref{ex-nounless-np}) and (\ref{ex-nounless-np-relational-noun})  and
%explain why such an analysis and even a more general one as in (\mex{1}) would
%not extend to German.
\ea
NP $\to$ Det Adj
\z

\item 为什么\xbarc 理论不能说明没有额外假设的德语形容词短语呢?(这一问题只适合于德语母语者。)
%\item Why can \xbart not account for German adjective phrases without additional assumptions? (This
%  task is for (native) speakers of German only.)

\item 请写出能够分析例(\mex{1})中的句子,并且能排除例(\mex{2})的句子的短语结构文法。
%\item Come up with a phrase structure grammar that can be used to analyze the sentence in (\mex{1}), but also
%rules out the sentences in (\mex{2}).

      \eal
      \ex[]{
      \gll Der Mann hilft der Frau.\\
            \textsc{det}.\nom{} 男人 帮助 \textsc{det}.\dat{} 女人\\
      \glt `这个男人帮助这个女人。'
      %           the.\nom{} man helps the.\dat{} woman\\
      %\glt `The man helps the woman.'
      }
      \ex[]{
      \gll Er gibt ihr das Buch.\\
                 他.\nom{} 给 她.\dat{} \textsc{det} 书\\
      \glt `他给她这本书。'
%           he.\nom{} gives her.\dat{} the book\\
%      \glt `He gives her the book.'
      }
      \ex[]{
      \gll Er wartet auf ein Wunder.\\
           他.\nom{} 等 \textsc{prep} 一 奇迹\\
	  \glt `他在等奇迹发生。'
	  %           he.\nom{} waits on a miracle\\
	  %\glt `He is waiting for a miracle.'
      }
%      \ex[]{
%       Er wartet neben dem Bushäuschen auf ein Wunder.
%       }
      \zl%\enlargethispage{1\baselineskip}
      \eal
      \ex[*]{
      \gll  Der Mann hilft er.\\
		    \textsc{det}.\nom{} 人 帮助 他.\nom{}\\
		   % the.\nom{} man helps he.\nom{}\\
      }
      \ex[*]{
      \gll  Er gibt ihr den Buch.\\
            他.\nom{} 给 她.\dat{} \textsc{det}.\mas{} 书.\neu{}\\
           %he.\nom{} gives her.\dat{} the.\mas{} book.\neu{}\\
      }
      \zl
\item 为了分析下述句子,你应该对上述习题中完成的文法做哪些修改?
%\item Consider which additional rules would have to be added to the grammar you developed in the previous exercise
%	  in order to be able to analyze the following sentences:

\eal
\ex 
\gll Der Mann hilft der Frau jetzt.\\
     \textsc{det}.\nom{} 男人 帮助 \textsc{det}.\dat{} 女人 现在\\
\glt `这个男人在帮助这个女人。'
%     the.\nom{} man helps the.\dat{} woman now\\
%\glt `The man helps the woman now.'
\ex 
\gll Der Mann hilft der Frau neben dem Bushäuschen.\\
     \textsc{det}.\nom{} 男人 帮助 \textsc{det}.\dat{} 女人 \textsc{prep} \textsc{det} 公交车候车亭\\
\glt `这个男人帮助公交车候车亭旁边的那个女人。'
%     the.\nom{} man helps the.\dat{} woman next.to the bus.shelter\\
%\glt `The man helps the woman next to the bus shelter.'
\ex 
\gll Er gibt ihr das Buch jetzt.\\
     他.\nom{} 给 她.\dat{} \textsc{det}.\acc{} 书 现在\\
\glt '他现在给她这本书。'
%     he.\nom{} gives her.\dat{} the.\acc{} book now\\
%\glt 'He gives her the book now.'
\ex 
\gll Er gibt ihr das Buch neben dem Bushäuschen.\\
     他.\nom{} 给 她.\dat{} \textsc{det}.\acc{} 书 \textsc{prep} \textsc{det} 公交车候车亭\\
\glt `他给她公交车候车亭旁的那本书。'
%     he.\nom{} gives her.\dat{} the.\acc{} book next.to the bus.shelter\\
%\glt `He gives her the book next to the bus shelter.'
\ex 
\gll Er wartet jetzt auf ein Wunder.\\
     他.\nom{} 等 现在 \textsc{prep} 一 奇迹\\
\glt `他在等奇迹发生。' 
%     he.\nom{} waits now on a miracle\\
%\glt `He is waiting for a miracle now.'
\ex 
\gll Er wartet neben dem Bushäuschen auf ein Wunder.\\
      他.\nom{} 等 \textsc{prep} \textsc{det}.\dat{} 公交车候车亭 \textsc{prep} 一 奇迹\\
\glt `他在等公交车候车亭旁的奇迹发生。'
%     he.\nom{} waits next.to the.\dat{} bus.shelter on a miracle\\
%\glt `He is waiting for a miracle next to the bus shelter.'
\zl
\item 请安装Prolog系统(如SWI"=Prolog\footnote{%
\url{http://www.swi-prolog.org} 
})
%\item Install a Prolog system (\eg SWI"=Prolog\footnote{%
%\url{http://www.swi-prolog.org} 
%})
并测试你的语法。详细的使用说明可以在对应手册中定子句文法(Definite Clause Grammar,DCG)\isc{定子句文法}\is{Definite Clause Grammar (DCG)}的关键词下面找到。
%and try out your grammar. Details for the notation can be found in the corresponding handbook
%under the key word Definite Clause Grammar (DCG)\is{Definite Clause Grammar (DCG)}.
\end{enumerate}
}

%\section*{延伸阅读}
%\section*{Further reading}

\furtherreading{
在短语结构语法中加入特征的扩展最早是由\citet{Harman63a}在1963年提出的。
%The expansion of phrase structure grammars to include features was proposed as early as 1963 by \citet{Harman63a}.

本章所讨论的有关名词短语的短语结构语法覆盖了名词短语句法的大部分内容,但是仍不能解释某些特定的NP结构。而且,练习(\ref{uebung-np-empty})也说明了目前文法的一些问题。\citew{Netter98a}和\citew{Kiss2005a}在HPSG理论框架下讨论了这一问题,并提出了解决方案。
%The phrase structure grammar for noun phrases discussed in this chapter covers a large part of the syntax
%of noun phrases but cannot explain certain NP structures. Furthermore, it has the problem, which exercise~\ref{uebung-np-empty}
%is designed to show. A discussion of these phenomena and a solution in the framework of HPSG can be found in \citew{Netter98a} and \citew{Kiss2005a}.

针对语义信息整合进短语结构语法中的讨论是比较少的。有关谓词逻辑的讨论以及它如何整合进短语结构语法中,还有量词辖域的讨论,可以参考\citew{BB2005a}。
 %The discussion of the integration of semantic information into phrase structure grammars was very short. A detailed discussion of predicate logic and
%its integration into phrase structure grammars -- as well as a discussion of quantifier s\textsc{cop}e -- can be found in \citew{BB2005a}. 
}

% wsun: DONE
%      <!-- Local IspellDict: en_US-w_accents -->
