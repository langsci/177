%% -*- coding:utf-8 -*-
\exewidth{(235)}%

\chapter{语言知识的天赋性}
%\chapter{The innateness of linguistic knowledge}
\label{Abschnitt-Angeborenheit}\label{chap-innateness}
%
% Haspelmath2010c:391 tendencies

如果我们试着将本书提到的诸多理论进行比较的话,我们会发现这些理论之间有很多相似之处。\footnote{\label{fn-ffs}%
我们需要区分理论(theory)和框架(framework)这两个术语。框架是指构建理论时使用的一组共同的假设和工具。在本书中,我讨论了德语的理论。这些理论是在特定框架(GB、GPSG、HPSG、LFG\ldots)下形成的,另外当然有其他语言的其他理论也遵循同样的基本假设。这些理论虽然不同于这里讨论的德语理论,但它们是在同一框架内形成的。 \citet{Haspelmath2010c}倡导一种无框架的语法理论。如果语法理论使用了不相容的工具,那么就难以展开语言的对比。所以针对英语的非局部依存关系所提出的转换假设跟为德语提出的\slaschc 机制之间就无法进行比较了。我同意Haspelmath所说的形式工具可能会导致偏见的观点,但是不管怎样语言事实终究需要描写。如果理论之间没有共性,我们得到的将是依据个人的框架得出的孤立的理论。如果一个框架具有共享的词汇和建立无框架语法理论的标准,那么这一框架就是无框架语法理论。进一步论述见 \citet{MuellerCoreGram}以及本书的第\ref{Abschnitt-UG-mit-Hierarchie}章。
}
%If we try and compare the theories presented in this book, we notice
%that there are a number of similarities.\footnote{\label{fn-ffs}%
%  The terms \emph{theory} and \emph{framework} may require clarification. A framework is a common
%  set of assumptions and tools that is used when theories are formulated. In this book, I discussed
%  theories of German. These theories were developed in certain frameworks (GB, GPSG, HPSG, LFG, \ldots) and of course there are
%  other theories of other languages that share the same fundamental assumptions. These theories
%  differ from the theories of German presented here but are formulated in the same
%  framework.  \citet{Haspelmath2010c} argues for framework-free grammatical theory. If grammatical
%  theories used incompatible tools, it would be difficult to compare languages. So assuming
%  transformations for English nonlocal dependencies and a \slasch mechanism for German would make
%  comparison impossible. I agree with Haspelmath that the availability of formal tools may lead to
%  biases, but in the end the facts have to be described somehow. If nothing is shared between
%  theories, we end up with isolated theories formulated in one man frameworks. If there \emph{is} shared
%  vocabulary and if there are standards for doing framework"=free grammatical theory, then the
%  framework is framework"=free grammatical theory. See
%   \citet{MuellerCoreGram} and Chapter~\ref{Abschnitt-UG-mit-Hierarchie} of this book for further discussion.
%}
在所有理论框架中都有利用属性"=值偶对来描述语言对象的理论变体。这些理论提出的句法结构有时是相似的。然而,有一些差异会引起不同流派之间的激烈争论。理论之间的差异在于它们是否会提出转换、空成分、基于短语或基于词的分析,二叉或扁平结构这样的假设。
%In all of the frameworks, there are variants of theories that use feature"=value pairs to describe linguistic objects.
%The syntactic structures assumed are sometimes similar. Nevertheless, there are some differences that have often led to fierce debates
%between members of the various schools. Theories differ with regard to whether they assume transformations, empty elements, phrasal or lexical analyses,
%binary branching or flat structures.

每一种理论不仅要描述自然语言,还要解释自然语言。为某一给定语言设定允准其结构的无穷多个语法是可行的(见第\pageref{ua-psg-eins}页的练习\ref{ua-psg-eins})。这些语法在观察上是充分的(observationally adequate)\isc{观察的充分性}\is{observational adequacy}。如果一种语法能够符合观察以及母语者语感的话,该语法就实现了描写的充分性(descriptive adequacy)\isc{描写的充分性}\is{descriptive adequacy}。\footnote{%
因为有主观因素的作用,所以这一术语有时不是特别有用。并不是每一个人都直觉地认为如下假设是正确的,即假设世界上语言的每一种所能观察到的语序都可以从共同的“限定语-中心语-补足语”的构型中推导出来,并且也只能通过移位移到左侧(见\ref{Abschnitt-Kaynesche-Modelle})。
}
%Every theory has to not only describe natural language, but also explain it. It is possible to
%formulate an infinite number of grammars that license structures for a given language (see Exercise~\ref{ua-psg-eins} on page~\pageref{ua-psg-eins}). These grammars are \emph{observationally adequate}\is{observational %adequacy}.
%A grammar achieves \emph{descriptive adequacy}\is{descriptive adequacy} if it corresponds to observations and the intuitions of native speakers.\footnote{%
%This term is not particularly useful as subjective factors play a role. Not everybody finds grammatical theories intuitively correct where it is assumed that every
%observed order in the languages of the world has to be derived from a common
%Specifier"=Head"=Complement configuration, and also only by movement to the left (see Section~\ref{Abschnitt-Kaynesche-Modelle} for the discussion of such proposals).
%}
如果一种语言学理论能够用于为每一种自然语言建立一个描写充分的语法,那么这种语言学理论就是描写充分的。但是,能够实现描写充分的语法不一定具有解释充分性( explanatory adequacy\isc{解释的充分性}\is{explanatory adequacy})。能够实现解释充分性的语法需要与语言习得\isc{语言习得}\is{language acquisition}的数据相符,这些数据就是那些貌似能被人类说话者习得的语法\citep[\page 24--25]{Chomsky65a}。
%A linguistic theory is descriptively adequate if it can be used to formulate a descriptively adequate grammar for every natural language. However, grammars achieving descriptive
%adequacy do not always necessarily reach \emph{explanatory adequacy}\is{explanatory adequacy}. Grammars that achieve explanatory adequacy are those that are compatible with
%acquisition data\is{language acquisition}, that is, grammars that could plausibly be acquired by human speakers \citep[\page
%  24--25]{Chomsky65a}.

 \citet[\page 25]{Chomsky65a}假设儿童已经具有语法在特定领域中原则上应该是怎样的知识,然后他们从语言输入中提取出一个特定语法事实上应该是怎样的信息。主流的生成语法(MGG)中最著名的习得理论变体就是原则和参数理论,它认为参数化的原则限定了可能的语法结构,并且儿童只需要在语言习得中设置参数(见\ref{Abschnitt-GB-Paramater})。
% \citet[\page 25]{Chomsky65a} assumes that children already have domain"=specific knowledge about what grammars could, in principle, look like and then extract information about what
%a given grammar actually looks like from the linguistic input. The most prominent variant of acquisition theory in Mainstream Generative Grammar (MGG) is the Principles \& Parameters
%theory, which claims that parametrized principles restrict the grammatical structures possible and
%children just have to set parameters during language acquisition
%(see Section~\ref{Abschnitt-GB-Paramater}).

这些年来,天赋性假说\isc{天赋论}\is{nativism}(也称作先天论)历经多次修改。尤其是关于先天语言知识组成部分的假设,即所谓的普遍语法(UG)\isc{普遍语法}\is{Universal Grammar (UG)},一直在变化。
%Over the years, the innateness hypothesis, also known as nativism\is{nativism}, has undergone a number of modifications.
%In particular, assumptions about exactly what forms part of the innate linguistic knowledge,
%so"=called Universal Grammar (UG)\is{Universal Grammar (UG)}, have often been subject to change.

先天论经常遭到构式语法\indexcxgc、认知语法\isc{认知语法}\is{Cognitive Grammar}的支持者以及其他理论学派研究人员的反对。其他的解释一般用来支持语法范畴、句法结构或者句法结构中语言对象关系具有天赋性的事实。受到批评的另外一点是,由于很多规定被简单化地假设为普遍语法的一部分,语法的实际复杂性被模糊掉了。下面是GB/最简方案分析中一些论证的基本程序。
%Nativism is often rejected by proponents of Construction Grammar\indexcxg, Cognitive
%Grammar\is{Cognitive Grammar} and by many other researchers working in other theories. Other
%explanations are offered for the facts normally used to argue for the innateness of grammatical
%categories, syntactic structures or relations between linguistic objects in syntactic structures.
%Another point of criticism is that the actual complexity of analyses is blurred by the fact that many of the stipulations are simply assumed to be part of UG.
%The following is a caricature of a certain kind of argumentation in GB/Minimalism analyses: 

\begin{enumerate}
\item 我已经为S语言中的P现象提供了一种分析;
%\item I have developed an analysis for the phenomenon P in the language S.
\item 这种分析是优雅的/概念上简洁的/我的\footnote{%
    参见
    \url{http://www.youtube.com/watch?v=cAYDiPizDIs}。 \zhdate{2015/12/01}。\nocite{Zappa86a}\aimention{Frank Zappa}
};
%\item The analysis is elegant/conceptually simple/mine\footnote{%
%    Also, see
%    \url{http://www.youtube.com/watch?v=cAYDiPizDIs}. 01.12.2015.\nocite{Zappa86a}\aimention{Frank Zappa}
%}.
\item 不可能去学习相关的结构或原则;
%\item There is no possibility to learn the relevant structures or principles.
\item 所以,这一分析中从A1到An的假设一定是说话者天赋知识的一部分。
%\item Therefore, the assumptions A$_1$ through A$_n$ that are made in this analysis must be part of the innate knowledge
%of speakers.
\end{enumerate}
通过向UG中随意增加假设,可以让其余的分析十分简单。
%By attributing arbitrary assumptions to UG, it is possible to keep the rest of the analysis very
%simple.

下面的章节会简要评价一些支持语言特有的天赋知识的证据。我们会发现所有证据都存在争议。在下面的章节中,我会讨论语法体系结构的一些基本问题,语言能力与语言运用之间的差异,如何为语言使用现象建模,语言习得理论以及其他存在争议的问题,例如,在语言表征中设置空成分是否理想以及语言是应该主要基于词的属性还是短语的模式进行解释。
%The following section will briefly review some of the arguments for language"=specific innate knowledge.
%We will see that none of these arguments are uncontroversial. In the following chapters, I will discuss fundamental
%questions about the architecture of grammar, the distinction between competence and performance and how to model
%performance phenomena, the theory of language acquisition as well as other controversial questions, \eg 
%whether it is desirable to postulate empty elements in linguistic representations and whether language should
%be explained primarily based on the properties of words or rather phrasal patterns.

在我们转向这些激烈争论的话题之前,我想先讨论一下争论最为激烈的一个话题,即天赋语言知识的问题。在文献中,有人找到了以下支持天赋知识的证据:
%Before we turn to these hotly debated topics, I want to discuss the one that is most fiercely
%debated, namely the question of innate linguistic knowledge. In the literature, one finds the
%following arguments for innate knowledge:

\begin{itemize}
\item 存在句法普遍性,
\item 习得的速度,
\item 语言习得存在一个“关键期”,
\item 所有儿童都可以习得语言,但是灵长类动物不可以,
\item 儿童会自发地将皮钦语规范化,
\item 语言处理位于大脑的特定部分,
\item 所谓的语言能力与普遍认知能力的分离:
%\item the existence of syntactic universals,
%\item the speed of acquisition,
%\item the fact that there is a `critical period' for language acquisition,
%\item the fact that all children learn a language, but primates do not,
%\item the fact that children spontaneously regularize pidgin languages, 
%\item the localization of language processing in particular parts of the brain,
%\item the alleged dissociation of language and general cognition:
\begin{itemize}
\item 威廉氏综合症,
\item FoxP2基因发生突变的KE氏家族,以及
%\item Williams Syndrome,
%\item the KE family with FoxP2 mutation and
\end{itemize}
\item 刺激贫乏论。
%\item the Poverty of the Stimulus Argument.
\end{itemize}
 \citet{Pinker94a}对这些证据作了很好的概述。 \citet{Tomasello95a}对这本书作了批评性评论。其中个别观点会在下面进行讨论。
% \citet{Pinker94a} offers a nice overview of these arguments.  \citet{Tomasello95a} provides a critical review of this book. The individual points will be discussed in what follows.

\section{句法普遍性}
%\section{Syntactic universals}
\label{sec-syntactic-universals}

句法\isc{普遍性|(}\is{universal|(}的普遍性被作为论据来证明语言知识的天赋性(如\citealp[\page 33]{Chomsky98a-u}、\citealp[\page 237--238]{Pinker94a})。在文献中,对于什么是普遍性的、什么是语言特定的有不同的看法。支持普遍性的突出代表有:\footnote{%
Frans Plank\aimention{Frans Plank}在康斯坦斯有一个关于什么是普遍性的存档文集\citep{PF2000a}:\url{http://typo.uni-konstanz.de/archive/intro/}。截至\zhdate{2015/12/23},一共有2029条记录。这些记录根据它们的质量来进行标注,结果发现,许多的普遍性具有统计上的普遍性,它们适用于绝大多数语言,但是有一些例外。一些普遍性被标记为绝对的,也就说,极少有例外。有1153条被标记为带有问号的绝对,有1021条被标记为不带问号的绝对值。许多普遍性是通过蕴含的普遍性\isc{普遍性!蕴涵的普遍性}\is{universal!implicational}捕捉到的,即,它们具有这样的形式:如果一种语言具有属性X,那么它也有属性Y。在存档文集中列出的普遍性部分上是非常具体的,而且指向具体语法属性的历时变化。比如说,第四条表示:“如果呼格的典型元素是一个前缀,那么这个前缀来自于第一人称领有者或第二人称主语。”
}\nocite{Harbour2011a}
%The\is{universal|(} existence of syntactic universals has been taken as an argument for the innateness of linguistic knowledge
%(\eg \citealp[\page 33]{Chomsky98a-u}; %\citealp[\page 46--47]{Chomsky88a-u}; %\citealp[\page 46--47]{Chomsky88a}, ist in Stanford falsch zitiert
%\citealp[\page 237--238]{Pinker94a}). There are varying claims in the literature with regard to what is universal and 
%language"=specific. The most prominent candidates for universals are:\footnote{%
%Frans Plank\aimention{Frans Plank} has an archive of universals in Konstanz \citep{PF2000a}:
%\url{http://typo.uni-konstanz.de/archive/intro/}. On 23.12.2015, it contained 2029 entries.
%The entries are annotated with regard to their quality, and it turns out that many of the universals
%are statistical universals, that is, they hold for the overwhelming majority of languages, but there are
%some exceptions. Some of the universals are marked as almost absolute, that is, very few exceptions are known.
%1153 were marked as absolute or absolute with a question mark. 1021 of these are marked as absolute without
%a question mark. Many of the universals captured are implicational universals\is{universal!implicational}, that is, they have the form:
%if a language has the property X, then it also has the property Y. The universals listed in the archive
%are, in part, very specific and refer to the diachronic development of particular grammatical
%properties. For example, the fourth entry states that: \emph{If the exponent of
%  vocative is a prefix, then this prefix has arisen from 1st person possessor or a 2nd person
%  subject.} 
%}\nocite{Harbour2011a}

\begin{itemize}
\item 中心语导向参数
\item \xbarc 结构
\item 语法功能(如主语或宾语)
\item 约束原则
\item 长距离依存的属性
\item 时、体、态的语法要素
\item 词性
\item 递归或自嵌套
%\item the Head Directionality Parameter
%\item \xbar structures
%\item grammatical functions such as subject or object
%\item binding principles
%\item properties of long"=distance dependencies
%\item grammatical morphemes for tense, mood and aspect
%\item parts of speech
%\item recursion or self"=embedding
\end{itemize}

\noindent
我们将在下面的内容中逐一讨论这些普遍性。应该强调的是,人们对于这些普遍性的认定并没有达成一致,而且这些观察到的属性实际上需要假定天赋的语言知识。
%These supposed universals will each be discussed briefly in what follows. One should emphasize that there is by no means
%a consensus that these are universal and that the observed properties actually require postulating innate linguistic
%knowledge.

\subsection{中心语导向参数}
%\subsection{Head Directionality Parameter}
\label{Abschnitt-Kopfstellungsparameter}

% a month ago -> head final
% counterexamples notwithstanding
\mbox{}\isc{参数!中心语导向参数|(}\is{parameter!head direction|(}%
中心语导向参数在第\ref{Abschnitt-GB-Paramater}节就已经介绍过了。第\pageref{Bsp-Kopfstellungsparameter}页(\ref{Bsp-Kopfstellungsparameter})中的例子在这里重复表示为下面的(\mex{1}),这些例子说明了日语的结构是英语结构的镜像:
%The Head Directionality Parameter was already introduced in Section~\ref{Abschnitt-GB-Paramater}. The examples in (\ref{Bsp-Kopfstellungsparameter}) on
%page~\pageref{Bsp-Kopfstellungsparameter}, repeated below as (\mex{1}), show that the structures in Japanese are the mirror image of the English structures:
\eal
\label{Bsp-Kopfstellungsparameter-zwei}
\ex 
be showing pictures of himself
\ex
\gll zibun  -no syasin-o mise-te iru\\
     他自己 \hspaceThis{-}\textsc{prep} 照片 显示 \textsc{cop}\\
%     himself \hspaceThis{-}of picture showing be\\
\zl
为了捕捉到这些事实,我们提出一个参数来表示中心语相对于其所带论元的位置(如Chomsky \citeyear[\page 146]{Chomsky86}、\citeyear[\page 70]{Chomsky88a-u})。
%In order to capture these facts, a parameter was proposed that is responsible for the position of the head relative to its
%arguments (\eg Chomsky \citeyear[\page 146]{Chomsky86}; \citeyear[\page 70]{Chomsky88a-u}). 

Radford (\citeyear[\page 60--61]{Radford90a-u}; \citeyear[\page 19--22]{Radford97a-u})、 \citet[\page 234, 238]{Pinker94a}、 \citet[\page 350]{Baker2003b}和其他作者声称,在假定了中心语导向性参数的基础上,在动词所管辖的方向和介词所管辖的方向之间具有某种外在或隐含的关系,也就是说,动词末位语序的语言具有后置词,或具有VO语序的语言具有前置词。这一观点可以通过英语\il{English}和日语\il{Japanese}这两种语言来验证,如例(\mex{0})所示:在介词短语中,no出现在代词的后面,名词syasin-o(图片)位于其所属的PP后面,主动词在它的宾语后面,助词iru位于主动词mise-te之后。具体的短语就是英语中各自短语的镜像。
%By assuming a Head Directionality Parameter, Radford (\citeyear[\page 60--61]{Radford90a-u}; \citeyear[\page 19--22]{Radford97a-u}),  \citet[\page 234, 238]{Pinker94a},  \citet[\page 350]{Baker2003b}
%and other authors claim, either explicitly or implicitly, that there is a correlation between the direction of government of verbs and that of adpositions, that is, languages
%with verb"=final order have postpositions and languages with VO order have prepositions. This claim
%can be illustrated with the language pair English\il{English}/""Japanese\il{Japanese} and the
%examples in (\mex{0}): the \emph{no} occurs after the pronoun in the prepositional phrase, the noun \emph{syasin-o} `picture' follows the PP belonging to it, the main verb follows its object and the
%auxiliary \emph{iru} occurs after the main verb \emph{mise-te}. The individual phrases are the exact mirror image of
%the respective phrases in English.

对此,一个反例足以说明这一说法是站不住脚的。实际上,像波斯语\il{Persian}这种语言是遵循动词末位的语序的,但它是有前置词的,如(\mex{1})所示:
%A single counterexample is enough to disprove a universal  claim and in fact, it is possible to
%find a language that has verb"=final order but nevertheless has prepositions.
%Persian\il{Persian} is such a language. An example is given in (\mex{1}):
\ea
\gll man ketâb-â-ro be Sepide dâd-am.\\
     我 书-\pl-\RA{} \textsc{prep} Sepide  给-1\sg\\
\mytrans{我把书给了Sepide。}
%     I book-\pl-\RA{} to Sepide gave-1\sg\\
%\mytrans{I gave the books to Sepide.}
\z
在\ref{Abschnitt-X-Bar}中,我们说明了德语不能简单地按照这一参数来描写:德语是动词位于末位的语言,但是它既有前置词,也有后置词。《语言结构的世界地图》这本书囊括了41种带有VO语序和后置词的语言,还有14种带有OV语序和前置词的语言\citep{wals-83,wals-85}。\footnote{%
  \url{http://wals.info/combinations/83A_85A\#2/15.0/153.0},\zhdate{2015/12/23}。
} \citet{Dryer92a}更早之前在小范围样本的语言研究中也指出,按照中心语导向参数所预测的结果是有例外的。
%In Section~\ref{Abschnitt-X-Bar}, it was shown that German cannot be easily described with this parameter: German is a verb"=final language but has both
%prepositions and postpositions. The World Atlas of Language Structures lists 41 languages with VO
%order and postpositions and 14 languages with OV order and prepositions \citep{wals-83,wals-85}.\footnote{%
%  \url{http://wals.info/combinations/83A_85A\#2/15.0/153.0}, 23.12.2015.
%} An earlier study by  \citet{Dryer92a} done with a smaller sample of languages also points out that
%there are exceptions to what the Head Directionality Parameter would predict. 

不仅如此, \citet[\page 422]{GW94a}指出,用一个参数来表示中心语的位置是不够的,因为英语和德语及荷兰语中标补语可以出现在他们的补足语前面;不过,英语是一个VO语序的语言,而德语和荷兰语是OV语序的语言。
%Furthermore,  \citet[\page 422]{GW94a} point out that a single parameter for the position of heads would not be enough since complementizers in both English and German/Dutch
%occur before their complements; however, English is a VO language, whereas German and Dutch count as OV languages.

如果我们希望通过句法范畴来决定管辖的方向(\citealp[\page 422]{GW94a};\citealp[\page 15]{Chomsky2005a}),那么我们就必须假设句法范畴属于普遍语法的一部分(更多内容参见\ref{Abschnitt-UG-Wortarten})。对于这类假设来说,前置词和后置词还是有问题的,因为它们通常都被指派给同一个范畴(P)。如果我们要为前置词和后置词引入特殊的范畴,那么像第\pageref{Tabelle-Merkmalszerlegung-Wortarten}页所示的词类的四分法就不管用了。相反,我们需要一个额外的二元特征,这样便会自动预测出八种范畴,虽然只有五种(四种常规的,加上一个额外的)是实际上需要的。
%If one wishes to determine the direction of government based on syntactic categories (\citealp[\page 422]{GW94a}, \citealp[\page 15]{Chomsky2005a}), then one has to assume
%that the syntactic categories in question belong to the inventory of Universal Grammar (see Section~\ref{Abschnitt-UG-Wortarten}, for more on this).
%Difficulties with prepositions and postpositions also arise for this kind of assumption as these are normally assigned to the same category (P).
%If we were to introduce special categories for both prepositions and postpositions, then a four"=way
%division of parts of speech like the one on page~\pageref{Tabelle-Merkmalszerlegung-Wortarten} would
%no longer be possible. One would instead require an additional binary feature and one would thereby
%automatically predict eight categories although only five (the four commonly assumed plus an extra one) are actually needed.

我们可以看到,Pinker作为普遍规则构建出的管辖方向之间的关系实际上在作为一种倾向性时才是正确的,而不是作为严格的规则。也就是说,有许多语言,其中前置词或后置词的使用与动词的位置之间具有相关性\citep[\page 83]{Dryer92a}。\footnote{%
 \citet[\page 234]{Pinker94a}在他的观点中使用了usually(通常)这个词。由此,他暗示了这里是有例外的,介词的顺序与动词管辖方向之间的关系实际上是一种倾向性,而不是一种具有普遍意义的可应用的规则。但是,在随后的内容中,他认为,中心语导向参数构成了天赋的语言知识的一部分。 \citet[\page
55]{Travis84a-u}讨论了现代汉语中一些跟她所提出的相关性不符的实例。之后她提出将中心语导向参数作为一种缺省的参数,这个参数可以被语言中的其他限制所覆盖。
} 
%One can see that the relation between direction of government that Pinker formulated as a universal
%claim is in fact correct but rather as a tendency than as a strict rule, that is, there are many languages where
%there is a correlation between the use of prepositions or postpositions and the position the verb
%\citep[\page 83]{Dryer92a}.\footnote{%
% \citet[\page 234]{Pinker94a} uses the word \emph{usually} in his formulation. He thereby implies
%that there are exceptions and that the correlation between the ordering of adpositions and the
%direction of government of verbs is actually a tendency rather than
%a universally applicable rule. However, in the pages that follow, he argues that the Head Directionality Parameter forms part of innate linguistic knowledge.
% \citet[\page 55]{Travis84a-u} discusses data from Mandarin Chinese that do not correspond to the correlations she assumes. She then proposes treating the Head Directionality Parameter
%as a kind of Default Parameter that can be overridden by other constraints in the language.
%} 

许多语言中,介词由动词演变而来。在现代汉语\isc{现代汉语}\is{Mandarin Chinese}语法中,有一类词通常被称作副动词\isc{副动词}\is{coverb}。这些词可以用作介词,也可以用作动词。如果我们历时地观察语言,那么我们就可以找到这些倾向性的解释,而不用参考天赋的语言学知识(参见\citealp[\page 445]{EL2009a})。
%In many languages, adpositions have evolved from verbs. In Chinese\is{Mandarin Chinese} grammar, it is commonplace to refer to a particular class of words as coverbs\is{coverb}.
%These are words that can be used both as prepositions and as verbs. If we view languages historically, then we can find explanations for these tendencies that do not have to make
%reference to innate linguistic knowledge (see \citealp[\page 445]{EL2009a}). 

进而,我们可以解释与语言处理的倾向性相关的一些事实:具有相同管辖方向的语言(图\ref{fig-head-position}a--b)与具有相反管辖方向的语言(图\ref{fig-head-position}c--d)相比,其动词和前/后置词之间的距离更小。
%Furthermore, it is possible to explain the correlations with reference to processing preferences: in languages with the same direction of government, the distance between the verb
%and the pre-/postposition is less (Figure~\ref{fig-head-position}a--b) than in languages with
%differing directions of government (Figure~\ref{fig-head-position}c--d).
\begin{figure}
\hfill
%\begin{tabular}{cc}
\subfloat[带有前置词的SVO(常见)]{
%\subfloat[SVO with prepositions (common)]{
\makebox[.4\textwidth]{
\begin{tikzpicture}
\tikzset{level 1+/.style={level distance=2\baselineskip}}
%\tikzset{frontier/.style={distance from root=24\baselineskip}}
\Tree[.IP NP
       [.VP \node(v){V}; NP [.PP \node(p){P}; NP ] 
       ]
]
\draw (v) |-  ([yshift=-5mm]v |- p) -| (p);
\end{tikzpicture}}}
\hfill
\subfloat[带有后置词的SOV(常见)]{
%\subfloat[SOV with postpositions (common)]{
\makebox[.4\textwidth]{
\begin{tikzpicture}
\tikzset{level 1+/.style={level distance=2\baselineskip}}
%\tikzset{frontier/.style={distance from root=24\baselineskip}}
\Tree[.IP NP
       [.VP [.PP NP \node(p){P}; ] NP \node(v){V};
       ]
]
\draw (v) |-  ([yshift=-5mm]v |- p) -| (p);
\end{tikzpicture}}}\hfill\mbox{}

\hfill
\subfloat[带有后置词的SOV(少见)]{
%\subfloat[SVO with postpositions (rare)]{
\makebox[.4\textwidth]{
\begin{tikzpicture}
\tikzset{level 1+/.style={level distance=2\baselineskip}}
%\tikzset{frontier/.style={distance from root=24\baselineskip}}
\Tree[.IP NP
       [.VP \node(v){V}; NP [.PP NP \node(p){P}; ] 
       ]
]
\draw (v) |-  ([yshift=-5mm]v |- p) -| (p);
\end{tikzpicture}}}
\hfill
\subfloat[带有前置词的SOV(少见)]{
%\subfloat[SOV with prepositions (rare)]{
\makebox[.4\textwidth]{
\begin{tikzpicture}
\tikzset{level 1+/.style={level distance=2\baselineskip}}
%\tikzset{frontier/.style={distance from root=24\baselineskip}}
\Tree[.IP NP
       [.VP [.PP \node(p){P}; NP ] NP \node(v){V};
       ]
]
\draw (v) |-  ([yshift=-5mm]v |- p) -| (p);
\end{tikzpicture}}}
\hfill\mbox{}
\caption{基于 \citet[\page 221]{Newmeyer2004b}的遵循不同的中心语语序的动词与介词之间的距离}\label{fig-head-position}
%\caption{Distance between verb and preposition for various head orders according to  \citet[\page 221]{Newmeyer2004b}}\label{fig-head-position}
\end{figure}%
从语言处理的角度来看,具有相同管辖方向的语言更易于被人们接受,因为他们允许听者更好地识别动词短语的组成部分( \citet[\page 219--221]{Newmeyer2004b}引用了 \citew[\page 32]{Hawkins2004a-u}有关一般处理的偏好的讨论,也可以参考 \citew[\page 131]{Dryer92a})。由此,这一倾向性可以解释为语言运用偏好的语法化(参见第\ref{Abschnitt-Diskussion-Performanz}章有关语言能力和语言运用的区分),而且这并不必然依赖于天赋的语言特有的知识。
%From the point of view of processing, languages with the same direction of government should be preferred since they allow the hearer to better identify the parts
%of the verb phrase ( \citet[\page 219--221]{Newmeyer2004b} cites  \citew[\page 32]{Hawkins2004a-u}
%with a relevant general processing preference, see also  \citew[\page 131]{Dryer92a}). This tendency can thus be explained as the
%grammaticalization of a performance preference (see Chapter~\ref{Abschnitt-Diskussion-Performanz}
%for the distinction between competence and performance) and recourse to innate language"=specific
%knowledge is not necessary.%
\isc{参数!中心语导向参数|)}\is{parameter!head direction|)}

\subsection{\xbar 结构}
%\subsection{\xbar structures}
\label{sec-Diskussion-X-Bar}

通常认为\isc{X理论@\xbar 理论|(}\is{X theory@\xbar theory|(},所有语言都有对应于\xbarc 模式的句法结构(参见\ref{sec-xbar})(\citealp[\page
238]{Pinker94a};\citealp[\page 11, 14]{Meisel95a};\citealp[\page 216]{PJ2005a})。但是,也有像迪尔巴尔语\il{Dyirbal}(澳大利亚)这种语言,其中句子是无法用层级结构来表示的。由此, \citet[\page 110]{Bresnan2001a}认为,他加禄语\il{Tagalog}、匈牙利语\il{Hungarian}、马拉雅拉姆语\il{Malayalam}、瓦勒皮里语\il{Warlpiri}、Jiwarli语\il{Jiwarli}、Wambaya语\il{Wambaya}、雅卡语\il{Jakaltek}和其他相应的语言没有VP节点,而是带有(\mex{1})这样格式的规则:
%It\is{X theory@\xbar theory|(} is often assumed that all languages have syntactic structures that
%correspond to the \xbar schema (see Section~\ref{sec-xbar}) (\citealp[\page
%238]{Pinker94a}; \citealp[\page 11, 14]{Meisel95a}; \citealp[\page 216]{PJ2005a}). There are, however, languages such as Dyirbal\il{Dyirbal} (Australia) where it does not seem
%to make sense to assume hierarchical structure for sentences.
%Thus,  \citet[\page 110]{Bresnan2001a} assumes that Tagalog\il{Tagalog}, Hungarian\il{Hungarian},
%Malayalam\il{Malayalam}, Warlpiri\il{Warlpiri}, Jiwarli\il{Jiwarli}, Wambaya\il{Wambaya},
%Jakaltek\il{Jakaltek}
%and other corresponding languages do not have a VP node, but rather a rule taking the form of (\mex{1}): 
\ea
S $\to$ C$^*$
\z
这里,C$^*$ 表示任意数目的短语成分,而且结构中没有中心语。其他不带中心语的结构的例子将在\ref{Abschnitt-Phrasale-Konstruktionen}讨论。
%Here, C$^*$ stands for an arbitrary number of constituents and there is no head in the structure.
%Other examples for structures without heads will be discussed in Section~\ref{Abschnitt-Phrasale-Konstruktionen}.

\xbarc 结构被用来限制可能规则的形式。相应假设是这些限制减少了我们可以构建的语法类别,并且,使得语言更容易被习得。但是正如 \citet{KP90a}所展示的,\xbarc 结构的假说不能限制可能语法的数量,如果我们允准空的中心语的话。管约论使用了很多空的中心语\isc{空成分}\is{empty element},而且在最简方案\indexmpc 中,它们还有显著数量的增加。比如说,(\mex{0})中的规则可以被重构为:
%\xbar structure was introduced to restrict the form of possible rules. The assumption was that these restrictions reduce the class
%of grammars one can formulate and thus -- according to the assumption -- make the grammars easier to
%acquire. But as  \citet{KP90a} have shown,
%the assumption of \xbar structures does not lead to a restriction with regard to the number of possible grammars if one allows for empty heads\is{empty element}.
%In GB, a number of null heads were used and in the Minimalist Program\indexmp, there has been a
%significant increase of these. For example, the rule in (\mex{0}) can be reformulated as follows:
\ea
V$'$ $\to$ \vnull C$^*$
\z
这里,\vnullc 是一个空的中心语。由于限定语是可选的,V$'$可以投射到VP上,这样我们就得到了一个对应于\xbarc 模式的结构。
%Here, \vnull is an empty head. Since specifiers are optional, V$'$ can be projected to VP and we arrive at a structure corresponding to
%the \xbar schema.

除了具有自由语序的语言带来的问题以外,附加结构还有一些问题:Chomsky在\xbarc 理论中有关形容词性结构的分析(\citealp[\page 210]{Chomsky70a};还可以参考本书\ref{sec-xbar},尤其是图\vref{Abbildung-AP})不能直接应用到德语,因为,与英语不同的是,德语中的形容词短语是中心语后置,并且程度修饰语必须直接位于形容词的前面:
%Apart from the problem with languages with very free constituent order, there are further problems with adjunction structures: Chomsky's analysis of adjective structure
%in \xbart (\citealp[\page 210]{Chomsky70a}; see also Section~\ref{sec-xbar} of this book, in
%particular Figure~\vref{Abbildung-AP}) is not straightforwardly applicable to German since, unlike
%English, adjective phrases in German are head"=final and degree modifiers must directly precede the
%adjective: 
\eal
\ex[]{
\gll der auf seinen Sohn sehr stolze Mann\\
	 \textsc{det} \textsc{prep} 他的 儿子 非常 骄傲 男人\\
\mytrans{对自己的儿子感到非常骄傲的男人}
%	 the of his son very proud man\\
%\mytrans{the man very proud of his son}
}
\ex[*]{
\gll der sehr auf seinen Sohn stolze Mann\\
	 \textsc{det} 非常 \textsc{prep} 他的 儿子 骄傲 男人\\
%	 the very of his son proud man\\
}
\ex[*]{
\gll der auf seinen Sohn stolze sehr Mann\\
	 \textsc{det} \textsc{prep} 他的 儿子 骄傲 非常 男人\\
%	 the of his son proud very man\\
}
\zl
根据\xbarc 模式,auf seinen Sohn必须与stolze相组合,而且只有这样才能得到与它的限定语相组合的\abarc 投射(参见图\vref{Abbildung-AP}中英语形容词性短语的结构)。这样就只可能推导出诸如(\mex{0}b)或(\mex{0}c)的语序。而这两个语序在德语中都是不可能的。如果我们假设德语跟英语完全一致,而且基于某种原因,形容词的补足语必须移到左边的话,这样才有可能拯救\xbarc 模式。如果我们允许这种补救措施,那么当然任何一种语言都可以用\xbarc 模式来进行描写。结果是我们必须针对许多语言提出巨量的移位规则,而这从心理语言学的角度来看是异常复杂和困难的。请看第\ref{Abschnitt-Diskussion-Performanz}章有关与语言运用相适应的语法讨论。 
%Following the \xbar schema, \emph{auf seinen Sohn} has to be combined with \emph{stolze} and only then can the
%resulting \abar projection be combined with its specifier (see Figure~\vref{Abbildung-AP} for the structure of adjective
%phrases in English). It is therefore only possible to derive orders such as (\mex{0}b) or (\mex{0}c). Neither of these
%is possible in German. It is only possible to rescue the \xbar schema if one assumes that German is
%exactly like English and, for some reason, the complements of adjectives must be moved to the left. If we allow this kind of repair
%approaches, then of course any language can be described using the \xbar schema. The result would be that one would have
%to postulate a vast number of movement rules for many languages and this would be extremely complex and difficult
%to motivate from a psycholinguistic perspective. See Chapter~\ref{Abschnitt-Diskussion-Performanz} for grammars compatible with performance.

\xbarc 理论的最严格形式的问题在\ref{sec-xbar}进行了说明,它是由hydra小句\isc{hydra小句}\is{hydra clause}引起的 \citep{PR70a,Link84a-u,Kiss2005a}:
%A further problem for \xbart in its strictest form as presented in Section~\ref{sec-xbar} is posed by so"=called hydra clauses\is{hydra clause} \citep{PR70a,Link84a-u,Kiss2005a}:
\eal
\ex {}
\gll [[der Kater] und [die Katze]], die einander lieben\\
     \spacebr{}\spacebr{}\textsc{det} 公猫 和 \textsc{det} 猫 \textsc{rel} 互相 爱\\
\mytrans{公猫和(母)猫相互爱慕}
%     \spacebr{}\spacebr{}the tomcat and the cat that each.other love\\
%\mytrans{the tomcat and the (female) cat that love each other}
\ex {}约会的[[男孩儿] 和 [女孩儿]]是我的朋友。 
%\ex {}[[The boy] and [the girl]] who dated each other are friends of mine. 
\zl
因为(\mex{0})中的关系从句指称一组对象,他们只能附加在并列的结果上。整个并列结构是一个NP,但是,附加语实际上应该附加在\xbarc 层面上。与德语和英语中的关系小句相反的情况是波斯语中的形容词: \citet{Samvelian2007a}提出的分析是,形容词直接与名词相组合,而且只有名词和形容词的组合才能与PP论元相组合。
%Since the relative clauses in (\mex{0}) refer to a group of referents, they can only attach to the result of the coordination.
%The entire coordination is an NP, however, and adjuncts should actually be attached at the \xbar level. The reverse case of relative clauses
%in German and English is posed by adjectives in Persian:  \citet{Samvelian2007a} argues for an analysis where adjectives are combined with nouns
%directly, and only the combination of nouns and adjectives is then combined with a PP argument.

关于德语和英语的讨论说明了限定语和附加语的导入不能在具体的投射层面进行约束,而且前面非构型语言(non"=configurational languages)的讨论已经显示了中间层的假设并不适用于每一种语言。
%The discussion of German and English shows that the introduction of specifiers and adjuncts cannot be restricted to particular projection levels, and
%the preceding discussion of non"=configurational languages has shown that the assumption of intermediate levels does not make sense for every language.

还需要指出的是,Chomsky本人在1970年提出,有的语言可以脱离\xbarc 模式\citeyearpar[\page 210]{Chomsky70a}。
%It should also be noted that Chomsky himself assumed in 1970 that languages can deviate from the \xbar
%schema \citeyearpar[\page 210]{Chomsky70a}.

如果我们希望把所有的信息编码在词汇中,那么我们就需要能够表示共性的极为抽象的组合规则。这种组合性规则的一个例子是范畴语法\isc{范畴语法}\is{Categorial Grammar (CG)} 的多重应用规则(参见第\ref{Kapitel-CG}章),以及最简方案\indexmpc 中的合并\isc{合并}\is{Merge}(参见\ref{Abschnitt-MP})。上面这些规则仅仅说明了两个语言学对象被组合在一起。这种组合性当然存在于每一种语言中。但是,对于完全词汇化的语法而言,只有我们允许空中心语并且提出某些特别的假设的话,才有可能来描写语言。这将在\ref{Abschnitt-Phrasale-Konstruktionen}具体讨论。\isc{X理论@\xbar 理论|)}\is{X theory@\xbar theory|)} 
%If one is willing to encode all information about combination in the lexicon, then one could get by with very abstract combinatorial rules that would hold universally.
%An example of this kind of combinatorial rules is the multiplication rules of Categorial Grammar\is{Categorial Grammar (CG)} (see Chapter~\ref{Kapitel-CG}) 
%as well as Merge\is{Merge} in the Minimalist Program\indexmp (see Section~\ref{Abschnitt-MP}).
%The rules in question simply state that two linguistic objects are combined. These kinds of
%combination of course exist in every language. With completely lexicalized grammars, however, it is only possible to describe languages
%if one allows for null heads and makes certain ad hoc assumptions. This will be discussed in
%Section~\ref{Abschnitt-Phrasale-Konstruktionen}.\is{X theory@\xbar theory|)} 

\subsection{主语和宾语的语法功能}
%\subsection{Grammatical functions such as subject and object}
\label{Abschnitt-UG-EPP}

\mbox{} \citet[\page xxv]{BK82a}、 \citet[\page 236--237]{Pinker94a}\isc{主语|(}\is{subject|(}\isc{宾语|(}\is{object|(}\isc{语法功能|(}\is{grammatical function|(}、 \citet[\page 349]{Baker2003b}和其他人认为所有的语言都有主语和宾语。为了说明这个观点到底意味着什么,我们必须探究这些术语。对于大多数的欧洲语言来说,很容易说明主语和宾语是什么(参见\ref{Abschnitt-GF});但是,并不是所有的语言都是这样的,或者说在有些语言中用这些术语是完全没有意义的(\LATER{\citealp{Durie85a};
}\citealp[\S~4]{Croft2001a}; \citealp[\S~4]{EL2009a})。
%\mbox{} \citet[\page xxv]{BK82a},  \citet[\page 236--237]{Pinker94a}\is{subject|(}\is{object|(}\is{grammatical function|(},  \citet[\page 349]{Baker2003b} 
%and others assume that all languages have subjects and objects. In order to determine what exactly this claim means, we have to explore the terms
%themselves. For most European languages, it is easy to say what a subject and an object is (see Section~\ref{Abschnitt-GF}); however,
%it has been argued that it is not possible for all languages or that it does not make sense to use
%these terms at all (\LATER{\citealp{Durie85a}; }\citealp[Chapter~4]{Croft2001a}; \citealp[Section~4]{EL2009a}).

在Pinker研究的LFG\indexlfgc{}这类理论中,语法功能扮演着重要的角色。事实上,是该把句子视为主语、宾语还是特殊定义的子句型论元(\xcompc)\citep*{DL2000a-u,Berman2003b-u,Berman2007a-u,AMM2005a-u,Forst2006a-u}仍然是有争议的。这就意味着当我们讨论语法功能如何分配给论元时是有一定自由度与灵活性的。由此,我们有可能在所有语言中找到一种将语法功能指派给函项论元的形式。
%In theories such as LFG\indexlfg{} -- the one in which Pinker worked -- grammatical functions play a primary role. The fact that it is still
%controversial whether one should view sentences as subjects, objects or as specially defined sentential arguments (\xcomp) \citep*{DL2000a-u,Berman2003b-u,Berman2007a-u,AMM2005a-%u,Forst2006a-u}
%serves to show that there is at least some leeway for argumentation when it comes to assigning
%grammatical functions to arguments. It is therefore likely that one can find an assignment of
%grammatical functions to the arguments of a functor in all languages. 

与LFG不同的是,语法功能在GB(请看\citealp{Williams84a,Sternefeld85a})和范畴语法\indexcgc 中是不重要的,在GB中语法功能只能间接地通过指向树的位置来决定。由此,在第\ref{Kapitel-GB}章讨论的方法中,主语是位于IP的限定语位置上的短语。
%Unlike LFG, grammatical functions are irrelevant in GB (see \citealp{Williams84a,Sternefeld85a}) and Categorial Grammar\indexcg. In GB, grammatical functions can only be
%determined indirectly by making reference to tree positions. Thus, in the approach discussed in
%Chapter~\ref{Kapitel-GB}, the subject is the phrase in the specifier position of IP.

在乔姆斯基式理论的后期,也有一些看起来对应于语法功能的功能性节点(AgrS\isc{范畴!功能范畴!AgrS}\is{category!functional!AgrS},
AgrO\isc{范畴!功能范畴!AgrO}\is{category!functional!AgrO}, AgrIO\isc{范畴!功能范畴!AgrIO}\is{category!functional!AgrIO}请看第\pageref{Seite-AgrO}页)。但是, \citet[\S~4.10.1]{Chomsky95a-u}指出,这些功能性范畴只是出于理论内部的原因而提出的,而且应该被排除在普遍语法的范畴之外。请看 \citew{Haider97a}和 \citew[\page 509--510]{Sternefeld2006a-u}对德语的描写,其中无功能性的投射不能被激发。
%In later versions of Chomskyan linguistics, there are functional nodes that seem to correspond to grammatical functions (AgrS\is{category!functional!AgrS},
%AgrO\is{category!functional!AgrO}, AgrIO\is{category!functional!AgrIO}, see
%page~\pageref{Seite-AgrO}). However,  \citet[Section~4.10.1]{Chomsky95a-u} remarks that these functional categories were only assumed for theory internal reasons
%and should be removed from the inventory of categories that are assumed to be part of UG. See  \citew{Haider97a} and  \citew[\page 509--510]{Sternefeld2006a-u} for a description of German that %does without functional projections
%that cannot be motivated in the language in question.

HPSG\indexhpsgc 所持的观点较为中立:它用一个特殊的配价特征来表示主语(在德语语法中,有一个中心语特征还可以对非定式动词的主语进行表征)。但是,\subjfc 的值是从更为普遍的理论性角度推导出来的:在德语中,带有结构格\isc{格!结构格}\is{case!structural}的层级最低的旁格\isc{旁格}\is{obliqueness}成分就是主语(Müller \citeyear[\page 153]{Mueller2002b}; \citeyear[\page 311]{MuellerLehrbuch1})。
%The position taken by HPSG\indexhpsg is somewhere in the middle: a special valence feature is used for subjects (in grammars of German, there is a head feature that contains a
%representation of the subject for non"=finite verb forms). However, the value of the \subjf is
%derived from more general theoretical considerations: in German, the least oblique\is{obliqueness}
%element with structural case\is{case!structural} is the subject (Müller \citeyear[\page 153]{Mueller2002b}; \citeyear[\page 311]{MuellerLehrbuch1}).

在\gbc 理论(扩展的投射原则,EPP\isc{扩展的投射原则}\is{Extended Projection Principle (EPP)}、 \citew[\page 10]{Chomsky82a-u})和LFG\indexlfg (主语条件\isc{主语条件}\is{Subject Condition})中,都有用来确保每个句子都有一个主语的原则。通常认为这些原则是具有语言共性的。\footnote{%
但是,  \citet[\page 27]{Chomsky81a}允许语言不带主语。他假设,这可以通过一个参数来解决。 \citet[\page 311]{Bresnan2001a}提出了主语条件,但是在脚注中指出,也许有必要将这一条件参数化,从而使得它适合于某些语言。%
} 
%In \gbt (Extended Projection Principle, EPP\is{Extended Projection Principle (EPP)},  \citew[\page 10]{Chomsky82a-u}) and also in LFG\indexlfg (Subject Condition\is{Subject Condition}), there are %principles
%ensuring that every sentence must have a subject. It is usually assumed that these principles hold universally.\footnote{%
%  However,  \citet[\page 27]{Chomsky81a} allows for languages not to have a subject. He assumes that this is handled by a parameter.
 % \citet[\page 311]{Bresnan2001a} formulates the Subject Condition, but mentions in a footnote that it might be necessary
%to parameterize this condition so that it only holds for certain languages.%
%} 
 
正如前面提到的, GB中没有语法功能,但是有对应于语法功能的结构位置。对应于主语的位置是IP的限定语。EPP说明了在SpecIP中必须有一个元素。如果我们假定这个原则具有语言共性,那么每一种语言都必须在这个位置有一个元素。正如我们看到的,这个普遍性假设是有反例的:德语。德语有无人称被动句(\mex{1}a),其中有无主语的动词(\mex{1}b、c)和形容词(\mex{1}d--f)。\footnote{%
	更多有关德语中无主语动词的讨论,请参考 \citew[\S~6.2.1, 6.5]{Haider93a}、 \citew{Fanselow2000b}、
    \citew[\page912]{Nerbonne86b}和 \citew[\S~3.2]{MuellerLehrbuch1}。
}
%As previously mentioned, there are no grammatical functions in GB, but there are structural positions that correspond to grammatical functions.
%The position corresponding to the subject is the specifier of IP. The EPP states that there must be an element in SpecIP. If we assume universality
%of this principle, then every language must have an element in this position. As we have already seen, there is a counterexample to this
%universal claim: German. German has an impersonal passive (\mex{1}a) and there are also subjectless verbs (\mex{1}b,c) and adjectives (\mex{1}d--f).\footnote{%
%	For further discussion of subjectless verbs in German, see  \citew[Sections~6.2.1, 6.5]{Haider93a},  \citew{Fanselow2000b},
%    \citew[\page912]{Nerbonne86b} and  \citew[Section~3.2]{MuellerLehrbuch1}.
%}
\eal
\ex 
\gll dass noch gearbeitet wird\\
	 \textsc{comp} 仍然 工作 \passiveprs{}\\
\mytrans{人们仍在工作}
%	 that still worked is\\
%\mytrans{that people are still working}
\ex 
\gll Ihm graut vor der Prüfung.\\
     他.\dat{} 害怕 \textsc{prep} \textsc{det} 考试\\
\mytrans{他怕考试。}
%     him.\dat{} dreads before the exam\\
%\mytrans{He dreads the exam.}
\ex 
\gll Mich friert.\\
	 我.\acc{} 冷\\
\mytrans{我冷死了。}
%	 me.\acc{} freezes\\
%\mytrans{I am freezing.}
\ex\label{ex-schulfrei}
\gll weil schulfrei ist\\
	 因为 假期 \textsc{cop}\\
\mytrans{因为今天不用上学}
%	 because school.free is\\
%\mytrans{because there is no school today}
\ex\label{ex-schlecht-ist}
\gll weil ihm schlecht ist\\
	 因为 他.\dat{} 生病 \textsc{cop}\\
\mytrans{因为他生病了}
%	 because him.\dat{} ill is\\
%\mytrans{because he is not feeling well}
\ex\label{ex-fuer-dich-ist-immer-offen}\iw{offen}
\gll Für dich ist immer offen.\footnotemark\\
	 对于 你 \textsc{cop} 总是 开放\\
%	 for you is always open\\
\footnotetext{%
  \citew[\page 18]{Haider86}。
}
\mytrans{我们一直为你敞开大门。}
%\mytrans{We are always open for you.}
\zl
大多数不带主语的谓词也可以带形式主语。如例(\mex{1})所示:
%Most of the predicates that can be used without subjects can also be used with an expletive
%subject. An example is given in (\mex{1}):
\ea
\gll dass es ihm vor der Prüfung graut\\
	 \textsc{comp} \expl{} 他 \textsc{prep} \textsc{det} 考试 害怕\\
\mytrans{他怕考试。}
%	 that \expl{} him before the exam dreads\\
%\mytrans{He dreads the exam.}
\z
但是,也有不能带形式主语es的动词,如 \citet[\page 185]{Reis82}提出的例(\mex{1}a)中的liegen(位于)。
%However, there are verbs such as \emph{liegen} `lie' in example (\mex{1}a) from  \citet[\page 185]{Reis82} that cannot occur with
%an \emph{es} `it'.

\eal
\ex[]{
\gll Mir liegt an diesem Plan.\\
	 我.\dat{} 位于 \textsc{prep} 这 计划\\
\mytrans{这个计划对我很重要。}
%	 me.\dat{} lies on this plan\\
%\mytrans{This plan matters a lot to me.}
}
\ex[*]{
\gll Mir liegt es an diesem Plan.\\
	 我.\dat{} 位于 it \textsc{prep} 这 计划\\
%	 me.\dat{} lies it on this plan\\
}
\zl

\noindent
尽管如此,EPP和主语条件也有时用在德语中。 \citet[\page 1311]{Grewendorf93}提出了一个空形式主语\isc{空成分}\is{empty element}\isc{代词!虚指代词}\is{pronoun!expletive}来填充到无主语结构的主语位置上。
%Nevertheless, the applicability of the EPP and the Subject Condition is sometimes also assumed for German.
%\todostefan{ \citet{JS89a-u}} \citet[\page 1311]{Grewendorf93}\todostefan{Safir85 cited by  \citet[\page 259]{Koster87a-u}}
%assumes that there is an empty expletive\is{empty element}\is{pronoun!expletive} that fills the subject position of subjectless constructions.

Berman(\citeyear[\page 11]{Berman99a}、
\citeyear[\S~4]{Berman2003a})在\lfgc 框架下提出,动词形态可以在德语中填充主语角色,这样即使在无主句中,主语位置也在f"=结构中被填充了。约束条件是所有不带\predvc 的f"=结构必须是第三人称单数才能应用到无主语的f"=结构中。定式动词的主谓一致信息必须匹配到无主语的f"=结构,这样无主语结构的动词屈折变化就被限制为第三人称单数\citep{Berman99a}。
%Berman (\citeyear[\page 11]{Berman99a};
%\citeyear[Chapter~4]{Berman2003a}), working in \lfg, assumes that verbal morphology can fulfill the subject role
%in German and therefore even in sentences where no subject is overtly present, the position for the subject is filled
%in the f"=structure. A constraint stating that all f"=structures without a \predv must be third
%person singular applies to the f"=structure of the unexpressed subject. The agreement information in
%the finite verb has to match the information in the f"=structure of the unexpressed subject and
%hence the verbal inflection in subjectless constructions is restricted to be 3rd person singular \citep{Berman99a}. 

正如我们在第\pageref{Seite-leeres-Objekt}页看到的,有些在最简方案下工作的研究人员甚至认为每个句子都有一个宾语(Stabler\aimention{Edward P. Stabler} 在
 \citet[\page 61, 124]{Veenstra98a}中这样引述)。单价动词的宾语被假定为一个空成分\isc{空成分}\is{empty element}。
%As we saw on page~\pageref{Seite-leeres-Objekt}, some researchers working in the Minimalist Program even
%assume that there is an object in every sentence (Stabler\aimention{Edward P. Stabler} quoted in
% \citet[\page 61, 124]{Veenstra98a}). Objects of monovalent verbs are assumed to be empty
%elements\is{empty element}.
 
如果我们允准这样的机制,当然就很容易维持许多普遍性的假设:我们假设一种语言X具有属性Y,然后假设结构性的成分是看不见而且没有意义的。这种分析只能用在理论内部以取得理论的一致性\isc{一致性}\is{uniformity}(参见\citealp[\S~2.1.2]{CJ2005a})。\footnote{%
 关于语言习得的内容,参见第\ref{chap-acquisition}章。
 }
%%If we allow these kinds of tools, then it is of course easy to maintain the existence of many universals: we claim that a language X has the property Y and then assume that
%the structural items are invisible and have no meaning. These analyses can only be justified theory"=internally with the goal of uniformity\is{uniformity}
%(see \citealp[Section~2.1.2]{CJ2005a}).\footnote{%
%	For arguments from language acquisition, see Chapter~\ref{chap-acquisition}.
%	}
\isc{主语|)}\is{subject|)}\isc{宾语|)}\is{object|)}\isc{语法功能|)}\is{grammatical function|)}

\subsection{约束原则}
%\subsection{Binding principles}

管辖代词约束\isc{约束理论|(}\is{Binding Theory|(}的原则也被看作是普遍语法的一部分(\citealp[\page 33]{Chomsky98a-u};\citealp*[\page 146]{CTK2009a};\citealp[\page 468]{Rizzi2009a})。\gbc 理论中的约束理论包括三条原则:原则A规定诸如sich或himself的反身代词指称某个局部域(约束域)内的一个成分(先行词)。简言之,我们可以这样说,一个反身代词必须要指称一个共指的论元成分。
%The principles\is{Binding Theory|(} governing the binding of pronouns are also assumed to be part of UG (\citealp[\page 33]{Chomsky98a-u}; \citealp*[\page 146]{CTK2009a}; 
%\citealp[\page 468]{Rizzi2009a}). Binding Theory in \gbt has three principles: principle A states that reflexives such as \emph{sich} or \emph{himself} refer to
%an element (antecedent) inside of a certain local domain (binding domain). Simplyfying a bit, one could say
%that a reflexive has to refer to a co-argument.
\ea
\gll Klaus$_i$ sagt, dass Peter$_j$ sich$_{*i/j}$ rasiert hat.\\
     Klaus     说 \textsc{comp} Peter 他自己 刮胡子 has\\
%     Klaus     says that Peter himself shaved has\\
\z
原则B是针对人称代词的,它规定人称代词不能指称它们的约束域内的成分。
%Principle B holds for personal pronouns and states that these cannot refer to elements inside of their
%binding domain.
\ea
\gll Klaus$_i$ sagt, dass Peter$_j$ ihn$_{i/*j}$ rasiert hat.\\
	 Klaus 说 \textsc{comp} Peter 他 刮胡子 has\\ 
%	 Klaus says that Peter him shaved has\\ 
\z
原则C规定了有指的表达式的指称内容。根据原则C,表达式A$_1$不能指称表达式A$_2$,如果A$_2$c"=统制\isc{c-统制}\is{c"=command} A$_1$的话。c"=统制被界定为话语的结构。对于c"=统制有许多不同的定义;一个简单的版本是:如果在短语成分结构中有一条路径从A向上直到下一个节点,然后只能往下到B,那么A c"=统制B。
%Principle C determines what referential expressions can refer to. According to Principle~C, an expression A$_1$ cannot refer to another expression
%A$_2$ if A$_2$ c"=commands\is{c"=command} A$_1$. c"=command is defined with reference to the structure of the utterance. There are various definitions
%of c"=command; a simple version states that A c"=commands B if there is a path in the constituent structure that goes upwards from A to the next branching
%node and then only downwards to B.

对于(\mex{1}a)中的例句来说,这就意味着Max和er(他)不能指称相同的个体,因为er c"=统制Max。
%For the example in (\mex{1}a), this means that \emph{Max} and \emph{er} `he' cannot refer to the same individual since \emph{er} c"=commands
%\emph{Max}.

\eal
\ex 
\gll Er sagt, dass Max Brause getrunken hat.\\
	 他 说 that Max 苏打水 喝 \textsc{aux}\\
\mytrans{他说Max喝了苏打水。}
%	 he says that Max soda drunk has\\
%\mytrans{He said that Max drank soda.}
\ex 
\gll Max sagt, dass er Brause getrunken hat.\\
	 Max 说 that 他 苏打水 喝 \textsc{aux}\\
\mytrans{Max说他喝了苏打水.}
%	 Max said that he soda drunk has\\
%\mytrans{Max said that he drank soda.}
\ex 
\gll Als er hereinkam, trank Max Brause.\\
	 当 他 进来 喝 Max 苏打水\\
\mytrans{当他进来的时候,Max正在喝苏打水。}
%	 as he came.in drank Max soda\\
%\mytrans{As he came in, Max was drinking soda.}
\zl
但是er和Max指称相同个体在(\mex{0}b)中却是可能的,因为它们没有c"=统制关系。对于er(他)来说,只有它在不指称动词getrunken(喝)的另一个论元的时候才具有c"=统制关系,而却是(\mex{0}b)中的情况。相似地,(\mex{0}c)的er(他)和Max之间也没有统制关系,因为代词er位于复杂结构中。 因此er(他)和Max可以指称相同或不同的个体,正如(\mex{0}b)和(\mex{0}c)所示。
%This is possible in (\mex{0}b), however, as there is no such c"=command relation. For \emph{er} `he', it must only be the case that it does not
%refer to another argument of the verb \emph{getrunken} `drunk' and this is indeed the case in (\mex{0}b). Similarly, there is no c"=command
%relation between \emph{er} `he' and \emph{Max} in (\mex{0}c) since the pronoun \emph{er} is inside a complex structure.
%\emph{er} `he' and \emph{Max} can therefore refer to the the same or different individuals in (\mex{0}b) and (\mex{0}c).

 \citet*[\page 147]{CTK2009a} 指出,(\mex{0}b、c)和对应的英语\il{English}例子是有歧义的,而(\mex{0}a)不是,这跟原则C有关。这意味着有一种含义是不适用的。为了获得正确的约束原则,学习者需要知道哪些意义是表达式不具备的。作者指出,儿童在三岁就已经熟练掌握了原则C,而且他们由此得到结论说,原则C貌似属于天赋的语言知识。(这是一个经典的论述。关于刺激贫乏论\isc{刺激贫乏论}\is{Poverty of the Stimulus},请看\ref{Abschnitt-PSA},而有关相反证据的更多内容请看\ref{Abschnitt-negative-Evidenz})。
% \citet*[\page 147]{CTK2009a} point out that (\mex{0}b,c) and the corresponding English\il{English} examples are ambiguous, whereas
%(\mex{0}a) is not, due to Principle C. This means that one reading is not available. In order to acquire the correct binding principles,
%the learner would need information about which meanings expressions do not have. The authors note that children already master Principle C
%at age three and they conclude from this that Principle C is a plausible candidate for innate linguistic knowledge. (This is a classic kind of argumentation.
%For Poverty of the Stimulus arguments\is{Poverty of the Stimulus}, see Section~\ref{Abschnitt-PSA} and for more on negative evidence, see Section~\ref{Abschnitt-negative-Evidenz}).

 \citet[\page 483]{EL2009b}指出,原则C具有很强的跨语言的倾向性,但还是有一些例外的。比如说,他们提到Abaza语\il{Abaza}中的相互表达式,其中表示each other(互相)的词缀出现在主语位置上,而不是宾语位置上,Guugu Yimidhirr语\il{Guugu Yimidhirr}也是这样,其中上层句可以跟从属句中的完全NP共指。
% \citet[\page 483]{EL2009b} note that Principle C is a strong cross"=linguistic tendency but
%it nevertheless has some exceptions. As an example, they mention both reciprocal expressions
%in Abaza\il{Abaza}, where affixes that correspond to \emph{each other} occur in subject position rather than object position as well as Guugu Yimidhirr\il{Guugu Yimidhirr}, where
%pronouns in a superordinate clause can be coreferent with full NPs in a subordinate clause.

此外, \citet[\page 351]{Fanselow92b}提出例(\mex{1})来说明原则C不是一个好的句法原则。
%Furthermore,  \citet[\page 351]{Fanselow92b} refers to the examples in (\mex{1}) that show that Principle C is a poor candidate for a syntactic
%principle.
\eal
\ex 
\gll Mord ist ein Verbrechen.\\
     谋杀 是 一 犯罪\\
%     murder is a crime\\
\ex 
\gll Ein gutes Gespräch hilft Probleme überwinden.\\
     一 好 谈话 帮助 问题 克服\\
\mytrans{一次好的交谈可以帮助解决问题。}
%     a good conversation helps problems overcome\\
%\mytrans{A good conversation helps to overcome problems.}
\zl
(\mex{0}a)是说当某人杀死他人时,这是一项犯罪,而(\mex{0}b)是倾向于跟他人聊天,而不是跟自己聊天。在这些句子中,名词化的Mord(谋杀)和Gespräch(谈话)使用时没有带上原始动词的任何论元成分。所以说,没有表示句法统制关系的任何论元成分。尽管如此,名词化的动词的论元不能互指。所以说,这样就有一条原则说明,谓词的论元槽上必须解释为非互指的,只要论元的指认没有通过语言手段明确地指出来。
%(\mex{0}a) expresses that it is a crime when somebody kills someone else, and (\mex{0}b) refers to conversations with another
%person rather than talking to oneself. In these sentences, the nominalizations \emph{Mord} `murder'
%and \emph{Gespräch} `conversation' are used without any arguments of the original verbs. So there
%aren't any arguments that stand in a syntactic command relation to one another. Nevertheless the arguments
%of the nominalized verbs cannot be coreferential. Therefore it seems that there is a principle at work that says that
%the argument slots of a predicate must be interpreted as non"=coreferential as long as the identity of the arguments is not explicitly expressed
%%by linguistic means.\todostefan{aber was hat das mit den Fällen zu tun, in denen die Sachen im Baum
 % realisiert sind. Noch mal Fanselow lesen}

综上所述,约束理论还有一些尚未解决的问题。英语中原则A--C的\hpsgc 理论变体不能用于德语\citep[\S~20]{Mueller99a}。在LFG理论框架下, \citet{Dalrymple93a}提出了一个约束理论的变体,其中代词性表达的约束属性在词汇中决定。按照这个方法,就可以描述具体语言的代词属性了。\isc{约束理论|)}\is{Binding Theory|)}
%In sum, one can say that there are still a number of unsolved problems with Binding Theory. The
%\hpsg variants of Principles A--C in English cannot
%even be applied to German \citep[Chapter~20]{Mueller99a}. Working in LFG,  \citet{Dalrymple93a} proposes a variant of Binding Theory where the binding
%properties of pronominal expressions are determined in the lexicon. In this way, the language"=specific properties of pronouns can be accounted for.\is{Binding Theory|)}

\subsection{长距离依存的属性}
%\subsection{Properties of long"=distance dependencies}
\label{Abschnitt-Fernabhängigkeiten}

在\isc{邻接|(}\is{subjacency|(}前面章节中讨论的长距离依存受限于某种条件。例如,在英语中,属于名词短语的句子中的任何成分都是无法被提取出来的。 \citet[\page
  70]{Ross67}把这个限制称为复杂NP限制(Complex NP Constraint\isc{复杂NP限制}\is{Complex NP
  Constraint})。在后面的研究中, \citet[\S~5.1.2]{Ross67}还构建了右界限制(Right Roof Constraint\isc{右界限制}\is{Right Roof Constraint})
。对此,有人倾向于把这些限制归纳为一个单一的、更具有普遍性的约束,即邻接原则(Chomsky \citeyear[\page 271]{Chomsky73a};\citeyear[\page 40]{Chomsky86b};\citealp{Baltin81a,Baltin2006a})。
邻接原则被认为是具有语言共性的。邻接限制规定移位操作可以跨越一个约束节点\isc{约束节点}\is{bounding node},而到底什么是约束节点,则取决于研究的语言(Baltin \citeyear[\page 262]{Baltin81a};\citeyear{Baltin2006a};\citealp[\page 57]{Rizzi82b};\citealp[\page 38--40]{Chomsky86b})。\LATER{ \citew{PB90a,Newmeyer91a}}\footnote{%
 \citet[\page 539--540]{Newmeyer2004a}指出了约束节点的不同语言差异的一个概念性的问题:邻接性是天赋的语言特有的原则,因为它太抽象了,说话者是无法学习的。但是,如果参数化需要说话者从语言输入中挑选一些范畴的话,那么相应的限制就必须从输入中推导出来,至少是有可能判断出所涵盖的范畴。这就导致一个问题,最初有关语言习得不可能性的论断是对的。请看\ref{Abschnitt-PSA}有关刺激缺乏论的内容,以及\ref{Abschnitt-PP}有关基于参数的语言习得的理论\isc{语言习得}\is{language acquisition}。

值得注意的是,具有某种词性的参数成分也要求在普遍语法中保有相应的词性信息。
}
%The\is{subjacency|(}\todostefan{\cite{Kroch89a-u,SWP2012a-u}} 
%long"=distance dependencies discussed in the preceding chapters are subject
%to some kind of restrictions. For example, nothing can be extracted out of sentences that are part of a noun phrase in English.  \citet[\page
%%  70]{Ross67} calls the relevant constraint the \emph{Complex NP Constraint}\is{Complex NP
 % Constraint}. In later work, the attempt was made to group this, and other constraints such as the
%\emph{Right Roof Constraint}\is{Right Roof Constraint} also formulated by   \citet[Section~5.1.2]{Ross67}, into a single, more general constraint, namely the Subjacency Principle
%	(Chomsky \citeyear[\page 271]{Chomsky73a}; \citeyear[\page 40]{Chomsky86b}; \citealp{Baltin81a,Baltin2006a}).
%Subjacency was assumed to hold universally.
%The Subjacency Constraint states that movement operations can cross at most one bounding node\is{bounding node}, whereby what exactly counts as a bounding node
%depends on the language in question (Baltin \citeyear[\page 262]{Baltin81a};
%\citeyear{Baltin2006a}; \citealp[\page 57]{Rizzi82b}; \citealp[\page 38--40]{Chomsky86b}).\LATER{ \citew{PB90a,Newmeyer91a}}\footnote{%
%	  \citet[\page 539--540]{Newmeyer2004a} points out a conceptual problem following from the language"=specific determination of bounding nodes: it is argued
%	 that subjacency is an innate language"=specific principle since it is so abstract that it is impossible for speakers to learn it. However, if parameterization
%	 requires that a speaker chooses from  a set of categories in the linguistic input, then the corresponding constraints must be derivable from the input at least to
%	 the degree that it is possible to determine the categories involved. This raises the question as to whether the original claim of the impossibility of acquisition
%	 is actually justified. See Section~\ref{Abschnitt-PSA} on the  \emph{Poverty of the Stimulus}\is{Poverty of the Stimulus}
%	 and Section~\ref{Abschnitt-PP} on parameter"=based theories of language
 %        acquisition\is{language acquisition}.

%         Note also that a parameter that has as the value a part of speech requires the respective
%         part of speech values to be part of UG.
%}

目前,对于邻接是否应该被看作是天赋的语言知识这个问题,在GB和最简方案中有不同的看法。 \citet*{HCF2002a}认为邻接并不属于天赋的语言能力的一部分,至少严格来说不是,而是广义上的跟语言相关的限制,这些限制可以从更为普遍的认知角度推导出来(请看第\pageref{Seite-Subjazenz-Performanz}页)。由于邻接在其他当代的文献(Newmeyer \citeyear[\page 15, 74--75]{Newmeyer2005a};\citeyear[\page 184]{Newmeyer2004b};\citealp{Baltin2006a}\footnote{%
不过,请参考 \citew[\page 552]{Baltin2004a}。
}、\citealp{Baker2009a}、\citealp{Freidin2009a}和\citealp{Rizzi2009a,Rizzi2009b})中还被看作是普遍语法原则的重要一环,我们在这里会对邻接原则进行更为详细的说明。
%Currently, there are varying opinions in the GB/Minimalism tradition with regard to the question of whether subjacency should be considered as part of innate linguistic knowledge.
% \citet*{HCF2002a} assume that subjacency does not form part of language"=specific abilities, at least not in the strictest sense, but rather is a linguistically relevant constraint
%in the broader sense that the constraints in question can be derived from more general cognitive ones (see
%p.\,\pageref{Seite-Subjazenz-Performanz}). Since subjacency still plays a role as a UG principle in other contemporary works (Newmeyer \citeyear[\page 15, 74--75]{Newmeyer2005a};
%\citeyear[\page 184]{Newmeyer2004b}; 
%\citealp{Baltin2006a}\footnote{%
%However, see  \citew[\page 552]{Baltin2004a}.
%}; \citealp{Baker2009a}; \citealp{Freidin2009a}; \citealp{Rizzi2009a,Rizzi2009b}), 
%the Subjacency Principle will be discussed here in some further detail.

我们可以区分两种移位:左向移位(通常叫做提取\isc{提取}\is{extraction})和右向移位(通常叫做外置\isc{外置}\is{extraposition})。所有的移位类型都构成长距离依存关系。在下面的章节中,我将讨论外置的一些限制。提取将在随后的\ref{Abschnitt-Subjazenz-Extraktion}详细讨论。
%It is possible to distinguish two types of movement: movement to the left (normally called extraction\is{extraction}) and movement to the right (normally referred to as
%extraposition\is{extraposition}). Both movement types constitute long"=distance dependencies.
%In the following section, I will discuss some of the restrictions on extraposition. Extraction will be discussed in Section~\ref{Abschnitt-Subjazenz-Extraktion} following it.

\subsubsection{外置}
%\subsubsection{Extraposition}

\mbox{} \citet{Baltin81a}\isc{外置|(}\is{Extraposition|(}和 \citet[\page 40]{Chomsky86b}声称,例(\mex{1})中的外置关系从句必须要根据嵌套的NP进行解释,即这些句子不匹配那些出现在t位置上的关系从句,而是对应于出现在t$'$位置上的例子。\il{English}
%\mbox{} \citet{Baltin81a}\is{Extraposition|(} and  \citet[\page 40]{Chomsky86b} claim that the extraposed relative clauses in (\mex{1}) have to be interpreted with
%reference to the embedding NP, that is, the sentences are not equivalent to those where the relative clause would occur in the position marked with t, but rather
%they correspond to examples where it would occur in the position of the t$'$.\il{English}
\eal
\label{ex-chomsky-sub}
\ex 
\gll {}[\sub{NP} Many books [\sub{PP} with         [stories      t]] t$'$]  were         sold [that I wanted to read].\\
     {}          许多 书     {}       \textsc{prep} \spacebr{}故事 {}  {}     \passivepst{} 卖   \spacebr\textsc{rel} 我 想 \textsc{inf} 读\\
\mytrans{有许多我想读的带故事的书都被卖了。}
\ex
\gll {}[\sub{NP} Many proofs [\sub{PP} of            [the theorem t]] t$'$] appeared [that I wanted to think about].\\
     {}          许多  证据    {}       \textsc{prep} \spacebr\textsc{det} 定理 {} {} 出现 \spacebr\textsc{rel} 我 想 \textsc{inf} 思考 \textsc{prep}.\\
\mytrans{我想思考的这个定理的许多证据都显现出来了。}    
\zl
这里,我们假定NP、PP、VP和AP(至少在英语中)都是右向移位的约束节点,这里所指的释义就被邻接原则排除了\citep[\page 262]{Baltin81a}。
%Here, it is assumed that NP, PP, VP and AP are bounding nodes for rightward movement (at least in English) and the interpretation in question here
%is thereby ruled out by the Subjacency Principle \citep[\page 262]{Baltin81a}. 

如果我们构建一个对应于(\mex{0}a)的德语例子,并且把嵌套的名词替换了,这样它就被规则排除了,或者不能作为所指,这样我们就得到了(\mex{1}):
%If we construct a German example parallel to (\mex{0}a) and replace the embedding noun so that it is ruled out or dispreferred as a referent, then we arrive at (\mex{1}):
\ea
\gll weil viele Schallplatten mit Geschichten verkauft wurden, die ich noch lesen wollte\\
	 因为 许多 唱片 \textsc{prep} 故事 卖 \passivepst{} \textsc{rel} 我 仍 读 想\\
\mytrans{因为很多带有我想读的故事的唱片被卖了}
%	 because many records with stories sold were that I still read wanted\\
%\mytrans{because many records with stories were sold that I wanted to read}
\z
这句话可以在这样的场景中使用,其中某人在一个唱片店中看到有些特殊的唱片,并且想起来他以前想读这些唱片上的童话故事。因为他并没有读这些唱片,附加到上位名词是不可能的,所以就附加到Geschichten(故事)上了。通过仔细地挑选名词,就有可能构建出诸如(\mex{1})的例子了,这些例子表示外置允许跨越多个NP节点:\footnote{%
请看 \citew[\page 211]{Mueller99a}和Müller (\citeyear{Mueller2004d};\citeyear[\S~3]{Mueller2007c})。相关的荷兰语\il{Dutch}的例子,请看 \citew[\page 52]{Koster78b-u}。
}
%This sentence can be uttered in a situation where somebody in a record store sees particular records and remembers that he had
%wanted to read the fairy tales on those records. Since one does not read records, adjunction to the superordinate noun is implausible
%and thus adjunction to \emph{Geschichten} `stories' is preferred. By carefully choosing the nouns, it is possible to construct examples such as
%(\mex{1}) that show that extraposition can take place across multiple NP nodes:\footnote{%
%  See  \citew[\page 211]{Mueller99a} and Müller (\citeyear{Mueller2004d};
%  \citeyear[Section~3]{Mueller2007c}). For parallel examples from
 % Dutch\il{Dutch}, see  \citew[\page 52]{Koster78b-u}.  
%}

\eal
\ex 
\gll Karl hat mir [ein Bild [einer Frau \_$_i$]] gegeben, [die schon lange tot ist]$_i$.\\
     Karl \textsc{aux} 我  \spacebr{}一 照片  \spacebr{}一 女人 {} 给 \spacebr{}\textsc{rel} \textsc{part} 长 死 is\\
\mytrans{Karl给了我一张死了一段时间的女人的照片。}
%	 Karl has me  \spacebr{}a picture  \spacebr{}a woman {} given \spacebr{}that PRT long dead is\\
%\mytrans{Karl gave me a picture of a woman that has been dead \erro{for} some time.}
\ex 
\gll Karl hat          mir [eine Fälschung [des Bildes                 [einer Frau \_$_i$]]] gegeben, [die schon lange tot ist]$_i$.\\
     Karl \textsc{aux} 我  \spacebr{}一 伪造 \spacebr{}\textsc{det} 照片 \spacebr{}一 女人 {} 给 \spacebr{}\textsc{rel} \textsc{part} 长 死 \textsc{cop}\\
\mytrans{Karl给了我一张伪造的死了一段时间的女人的照片。}
%	Karl has me \spacebr{}a forgery \spacebr{}of.the picture \spacebr{}of.a woman {} given \spacebr{}that PRT long dead is\\
%\mytrans{Karl gave me a forgery of the picture of a woman that has been dead for some time.}
\ex 
\gll Karl hat mir [eine Kopie [einer Fälschung [des Bildes [einer Frau \_$_i$]]]] gegeben, [die schon lange tot ist]$_i$.\\
	 Karl \textsc{aux} 我 \spacebr{}一 复制 \spacebr{}一 伪造 \spacebr{}\textsc{det} 照片 \spacebr{}一 女人 {} 给 \spacebr{}\textsc{rel} \textsc{part} 长 死 \textsc{cop}\\
\mytrans{Karl给了我一张复制的伪造的死了一段时间的女人的照片。}
%	 Karl has me \spacebr{}a \textsc{cop}y \spacebr{}of.a forgery \spacebr{}of.the picture \spacebr{}of.a woman {} given \spacebr{}that PRT long dead is\\
%\mytrans{Karl gave me a \textsc{cop}y of a forgery of the picture of a woman that has been dead for some time.}
\zl
这种嵌套可以无限地继续下去,只要能找到语义上可行的嵌套的名词。NP在德语中被看作是约束节点(Grewendorf \citeyear[\page 81]{Grewendorf88a};\citeyear[\page 17--18]{Grewendorf2002a};\citealp[\page 285]{Haider2001a})。这些例子说明右向的外置关系从句可以跨越任意数量的约束节点。
%This kind of embedding could continue further if one were to not eventually run out of nouns that
%allow for semantically plausible embedding.
%NP is viewed as a bounding node in German (Grewendorf \citeyear[\page 81]{Grewendorf88a};
%\citeyear[\page 17--18]{Grewendorf2002a}; \citealp[\page 285]{Haider2001a}). These examples show that it is possible for rightward extraposed relative clauses
%to cross any number of bounding nodes.

 \citet[\page 52--54]{Koster78b-u}讨论了例(\mex{0})中语言事实的一些可能的解释,其中关系从句被看作是移位到了NP/PP的边界,然后由此移向更远的位置(这种移位需要所谓的逃跑机制\isc{逃跑机制}\is{escape hatch}或者逃跑路径)。他认为,这些方法也适用于那些被邻接规则排除在外的句子,也就是说,那些像(\ref{ex-chomsky-sub})的例子。这就意味着,要么诸如(\ref{ex-chomsky-sub})的事实可以根据邻接原则来解释,则(\mex{0})中的句子是反例,要么存在一个逃跑机制,则(\ref{ex-chomsky-sub})中的例子是无关的,推导出的句子并不能用邻接原则进行解释。
% \citet[\page 52--54]{Koster78b-u} discusses some possible explanations for the data in (\mex{0}), where it is assumed that relative clauses move to the NP/PP border and are then
%moved on further from there (this movement requires so"=called escape hatches\is{escape hatch} or escape routes). He argues that
%these approaches will also work for the very sentences that should be ruled out by subjacency, that is, for examples such as (\ref{ex-chomsky-sub}). This means that either
%data such as (\ref{ex-chomsky-sub}) can be explained by subjacency and the sentences in (\mex{0})  are counterexamples, or there are escape hatches and the examples in
%(\ref{ex-chomsky-sub}) are irrelevant, deviant sentences that cannot be explained by subjacency.

在(\mex{0})的例句中,关系从句都被外置了。这些关系从句被看作是附加语,而且有人认为这些外置的附加语并不是移位而来的,而是在他们的位置上原位生成的,而且通过特殊的机制来表示共指关系\citep{Kiss2005a}。对于这类分析的支持者来说,(\mex{0})中的例句对邻接原则来说是无效的,因为邻接原则只对移位进行限制。但是,外置可以跨越短语的界限这一点并不局限于关系从句;句子的补足语也可以被外置:
%In the examples in (\mex{0}), a relative clause was extraposed in each case. These relative clauses
%are treated as adjuncts and there are analyses that assume that extraposed adjuncts are not moved but rather base"=generated in their position,
%and coreference/""coindexation is achieved by special mechanisms \citep{Kiss2005a}.
%For proponents of these kinds of analyses, the examples in (\mex{0}) would be irrelevant to the subjacency discussion as the Subjacency Principle
%only constrains movement. However, extraposition across phrase boundaries is not limited to relative clauses; sentential complements can also be extraposed:
\eal
\ex 
\gll Ich habe       [von                    [der                    Vermutung \_$_i$]] gehört, [dass es Zahlen gibt, die die folgenden Bedingungen erfüllen]$_i$.\\
     我 \textsc{aux} \spacebr{}\textsc{prep} \spacebr{}\textsc{det} 猜测 {} 听说 \spacebr{}\textsc{comp} \expl{} 数字 给 \textsc{rel} \textsc{det} 下面的 要求 满足\\
\mytrans{我听说了一种猜测,有能够满足如下要求的数字。}
%	 I have \spacebr{}from \spacebr{}the conjecture {} heard \spacebr{}that \expl{} numbers gives that the following requirements fulfill\\
%\mytrans{I have heard of the conjecture that there are numbers that fulfill the following requirements.}
\ex 
\gll Ich habe [von [einem Beweis [der Vermutung \_$_i$]]] gehört, [dass es Zahlen gibt, die die folgenden Bedingungen erfüllen]$_i$.\\
	我 \textsc{aux} \spacebr{}\textsc{prep} \spacebr{}一 证据 \spacebr{}\textsc{det} 猜测 {} 听说 \spacebr{}\textsc{comp} \expl{} 数字 给 \textsc{rel} \textsc{det} 下面的 要求 满足\\
\mytrans{我听说了猜测的证据,有能够满足如下要求的数字。}
%	I have \spacebr{}from \spacebr{}a proof \spacebr{}of.the conjecture {} heard \spacebr{}that \expl{} numbers give that the following requirements fulfill\\
%\mytrans{I have heard of the proof of the conjecture that there are numbers that fulfill the following requirements.}
\ex 
\gll Ich habe [von [dem Versuch [eines Beweises [der Vermutung \_$_i$]]]] gehört, [dass es Zahlen gibt, die die folgenden Bedingungen erfüllen]$_i$.\\
     我 \textsc{aux} \spacebr{}\textsc{prep} \spacebr{}\textsc{det} 尝试 \spacebr{}一 证据 \spacebr{}\textsc{det} 猜测 {} 听说 \spacebr{}\textsc{comp} \expl{} 数字 给 \textsc{rel} \textsc{det} 下面的 要求 满足\\
\mytrans{我听说了试图证明猜测的事情,有能够满足如下要求的数字。}
%     I have \spacebr{}from \spacebr{}the attempt \spacebr{}of.a proof \spacebr{}of.the conjecture {} heard \spacebr{}that \expl{} numbers gives that the following requirements fulfill\\
%\mytrans{I have heard of the attempt to prove the conjecture that there are numbers that fulfill the following requirements.}
\zl
由于有的名词选择带zu的不定式或介词短语,而且由于它们可以像上述句子一样被外置,就必须确保后置成分的句法范畴对应于名词所需要的范畴。这就意味着,管辖的名词和外置的成分之间必须存在某种关系。据此,(\mex{0})中的例句必须被分析为外置的例子,并且针对上述讨论的观点提供相反的证据。 
%Since there are nouns that select \emph{zu} infinitives or prepositional phrases and since these can
%be extraposed like the sentences above, it must be ensured that the syntactic category of the postposed element corresponds to the category required by the noun.
%This means that there has to be some kind of relation between the governing noun and the extraposed element. For this reason, the examples in
%(\mex{0}) have to be analyzed as instances of extraposition and provide counter evidence to the claims discussed above.

如果想要讨论循环嵌套的可能性,那么我们就必须想到如例(\mex{-1})中那些跨越几组句子的可能性,而例(\mex{0})是非常远的。但是,我们有可能找到深度嵌套的个例:
(\mex{1})给出了Tiger语料库\isc{Tiger语料库}\is{Tiger corpus}中找到的关系从句外置和补足语外置的一些例子\footnote{%
有关Tiger语料库的更多信息请参考 \citew{BDEHKLRSU2004a}。
}(\citealp[\page 78--79]{Mueller2007c};\citealp[\S~2.1]{MM2009a})。
%If one wishes to discuss the possibility of recursive embedding, then one is forced to refer to constructed examples as the likelihood of stumbling across groups of sentences
%such as those in (\mex{-1}) and (\mex{0}) is very remote. It is, however, possible to find some individual cases of deep embedding:
%(\mex{1}) gives some examples of relative clause extraposition and complement extraposition taken from the Tiger corpus\is{Tiger corpus}\footnote{%
%  See  \citew{BDEHKLRSU2004a} for more information on the Tiger corpus.
%} (\citealp[\page 78--79]{Mueller2007c}; \citealp[Section~2.1]{MM2009a}).
\eal
\ex 
\gll Der 43jährige will nach eigener Darstellung damit [\sub{NP} den Weg [\sub{PP}~für [\sub{NP} eine
  Diskussion [\sub{PP} über [\sub{NP} den künftigen Kurs [\sub{NP} der stärksten
  Oppositions\-gruppierung]]]]]] freimachen, [die aber mit 10,4 Prozent
  der Stimmen bei der Wahl im Oktober weit hinter den Erwartungen zurückgeblieben war]. (s27639)\\
  \textsc{det} 43岁 想要 \textsc{prep} 自己的 描述 那儿.\textsc{prep} {} \textsc{det} 道路 \hspaceThis{[\sub{PP}~}\textsc{prep} {} 一 讨论 {} 关于 {} \textsc{det} 未来 路线 {} \textsc{det} 强
  反对.派 自由.做 \spacebr{}\textsc{rel} 但是 \textsc{prep} 10.4 百分点
  \textsc{det} 选票 \textsc{prep} \textsc{det} 选举 \textsc{prep}.\textsc{det} 十月 远 \textsc{prep} \textsc{det} 预期 停留.回 \textsc{aux}\\
\mytrans{按照他自己的话来说,这位43岁的人想要为最强的反对派的有关未来路线的讨论扫清道路,但是,他的表现低于预期,在十月的选举中只得到了10.4个百分点。}
%\gll Der 43jährige will nach eigener Darstellung damit [\sub{NP} den Weg [\sub{PP}~für [\sub{NP} eine
%  Diskussion [\sub{PP} über [\sub{NP} den künftigen Kurs [\sub{NP} der stärksten
%  Oppositions\-gruppierung]]]]]] freimachen, [die aber mit 10,4 Prozent
%  der Stimmen bei der Wahl im Oktober weit hinter den Erwartungen zurückgeblieben war]. (s27639)\\
%  the 43.year.old wants after own depiction there.with {} the way \hspaceThis{[\sub{PP}~}for {} a discussion {} about {} the future course {} of.the strongest
%  opposition.group free.make \spacebr{}that however with 10.4 percent
%  of.the votes at the election in October far behind the expectations stayed.back was\\
%\mytrans{In his own words, the 43-year old wanted to clear the way for a discussion about the future course of the strongest opposition group that had, however, performed well below expectations %gaining only 10.4 percent of the votes at the election in October.}
\ex 
{\raggedright
\gll {}[\ldots] die Erfindung der Guillotine könnte [\sub{NP} die Folge [\sub{NP} eines verzweifelten
    Versuches des gleichnamigen Doktors] gewesen sein, [seine Patienten ein für allemal von
    Kopfschmerzen infolge schlechter Kissen zu befreien]. (s16977)\\
    {}  \textsc{det} 发明 \textsc{det} 断头台 能够 {} \textsc{det} 结果 {} 一 绝望 尝试 \textsc{det} 相同.名字 医生 \textsc{cop} \textsc{aux} \spacebr{}他的 病人
一次 \textsc{prep} 永远 \textsc{prep} 头疼 由于 坏 枕头 \textsc{inf} 解放\\
\par}
\mytrans{断头台的发明可能是由于与之同名的医生想要永远解除那些由于差枕头而导致头疼的病人的痛苦所做一种绝望的尝试。}
%{\raggedright
%\gll {}[\ldots] die Erfindung der Guillotine könnte [\sub{NP} die Folge [\sub{NP} eines verzweifelten
%    Versuches des gleichnamigen Doktors] gewesen sein, [seine Patienten ein für allemal von
%    Kopfschmerzen infolge schlechter Kissen zu befreien]. (s16977)\\
 %   {}  the invention of.the guillotine could {} the result {} of.a desperate attempt the same.name doctor have been \spacebr{}his patients
%once for all.time of headaches because.of bad pillows to free\\
%\par}
%\glt `The invention of the guillotine could have been the result of a desperate attempt of the
%eponymous doctor to rid his patients once and for all of headaches from bad pillows.'
\zl

\noindent
我们也可以造出违反邻接条件的英语句子。 \citet[\page 2333]{Uszkoreit90a}举出了下面的例子:
%It is also possible to construct sentences for English that violate the Subjacency Condition.
% \citet[\page 2333]{Uszkoreit90a} provides the following example:
\ea
\gll {}[\sub{NP} Only letters [\sub{PP} from          [\sub{NP} those people \_$_i$]]] remained unanswered [that had received our earlier reply]$_i$.\\
     {}          只有   信     {}        \textsc{prep} {}        那些   人     {}        保持      未回答的    \spacebr\textsc{rel} \textsc{aux} 收到 我们的 早些 回复\\
\mytrans{只有那些一直没有回答的人收到了我们的回信。}
\z
%
Jan Strunk\aimention{Jan Strunk} (p.\,c.\, 2008)找到了外置的例子,既包括跨越多重短语界限的限制性关系从句,也包括跨越多重短语界限的非限制性关系从句:
%Jan Strunk\aimention{Jan Strunk} (p.\,c.\, 2008) has found examples for extraposition of both restrictive and non"=restrictive relative clauses across
%multiple phrase boundaries:
\eal
\ex 
\gll For example, we understand that Ariva buses have won [\sub{NP} a number [\sub{PP} of [\sub{NP} contracts [\sub{PP} for [\sub{NP} routes in London \_$_i$ ]]]]] recently, [which will not be run by low floor accessible buses]$_i$.\\
     \textsc{prep} 例子 我们 理解 \textsc{comp} Ariva 巴士 \textsc{aux} 赢 {} 一 数字 {} \textsc{prep} {} 合同 {} \textsc{prep} {} 路线 \textsc{prep} 伦敦 {} {} 最近 \spacebr\textsc{rel} 将 不 \passive{} 运营 \textsc{prep} 低 层 可及的 巴士\\
\mytrans{例如,我们知道Ariva巴士最近已经取得了一些伦敦路线的合同,这些路线不会由低层的巴士运营了。'}\footnote{%
\url{http://www.publications.parliament.uk/pa/cm199899/cmselect/cmenvtra/32ii/32115.htm},
\zhdate{2007/02/24}.
%\url{http://www.publications.parliament.uk/pa/cm199899/cmselect/cmenvtra/32ii/32115.htm},
%24.02.2007.
}
\ex 
\gll I picked up [\sub{NP} a \textsc{cop}y of [\sub{NP} a book \_$_i$ ]] today, by a law professor, about law, [that is not assigned or in any way required to read]$_i$.\\
我 拾 起 {} 一 复印本 \textsc{prep} {} 一 书 {} {} 今天 \textsc{prep} 一 法律 教授 关于 法律 \spacebr\textsc{rel} \passive{} 不 指派 或者 \textsc{prep} 任何 方式 要求 \textsc{inf} 阅读\\
\mytrans{我拾起了一本今天法律教授讲的有关法律的一本书的复印本,不过这本书并没有被要求阅读。'}\footnote{%
\url{http://greyhame.org/archives/date/2005/09/}, \zhdate{2008/09/27}.
%\url{http://greyhame.org/archives/date/2005/09/}, 27.09.2008.
}
\ex 
\gll We drafted [\sub{NP} a list of [\sub{NP} basic demands \_$_i$ ]] that night [that had to be unconditionally met or we would stop making and delivering pizza and go on strike]$_i$.\\
我们 起草 {} 一 列表 \textsc{prep} {} 基本 需求  {} {} 那 晚上 \spacebr\textsc{rel} \textsc{aux} \textsc{inf} \passive{} 无条件地 满足 或者 我们 可以 停止 制作 和 递送 披萨 和 走 \textsc{prep} 罢工\\
\mytrans{我们那晚上起草了一个基本需求的清单,这些需求必须无条件地满足,不然我们就不再做披萨和运送披萨,而是继续罢工。'}\footnote{%
\url{http://portland.indymedia.org/en/2005/07/321809.shtml}, \zhdate{2008/09/27}.
%\url{http://portland.indymedia.org/en/2005/07/321809.shtml}, 27.09.2008.
}
\zl
 \citew[\page 111]{SS2013b-u}也引用过(\mex{0}a)。我们可以在这篇论文中找到更多的德语和英语的例子。
%(\mex{0}a) is also published in  \citew[\page 111]{SS2013b-u}. Further attested examples from German and
%English can be found in this paper.

上述分析表明,右向的邻接限制并不适用于英语或德语,由此不能被看作是具有语言普遍性的。我们只能简单地说NP和PP在英语或德语中不是约束节点。那么,这些外置的事实对于认定邻接原则的理论来说就不存在问题了。不过,邻接限制也被看作是左向移位。我们将在下一节来详细讨论。\isc{外置|)}\is{extraposition|)} 
%The preceding discussion has shown that subjacency constraints on rightward movement do not hold for English or German and thus cannot be
%viewed as universal. One could simply claim that NP and PP are not bounding nodes in English or German. Then, these extraposition data would
%no longer be problematic for theories assuming subjacency. However, subjacency constraints are also
%assumed for leftward movement. This is discussed in more detail in the following section.\is{extraposition|)} 

\subsubsection{提取}
%\subsubsection{Extraction}
\label{Abschnitt-Subjazenz-Extraktion}

在\isc{提取|(}\is{extraction|(}特定条件下,有些成分是不能左向移位的\citep{Ross67}。这些成分被称为提取的孤岛\isc{提取!提取的孤岛}\is{extraction!island}。 \citet[\S~4.1]{Ross67}构建了复杂NP限制\isc{复杂NP限制}\is{Complex NP
  Constraint} (CNPC),该限制表明复杂名词短语是无法提取的。比如说在一个名词短语内部的关系小句的提取就是不可行的: 
%Under\is{extraction|(} certain conditions, leftward movement is not possible from certain constituents \citep{Ross67}. 
%These constituents are referred to as islands for extraction\is{extraction!island}.  \citet[Section~4.1]{Ross67} formulated the \emph{Complex NP Constraint}\is{Complex NP
%  Constraint} (CNPC) that states that extraction is not possible from complex noun phrases. An example of extraction
%  from a relative clause inside a noun phrase is the following:

\ea
[*]{
\gll Who$_i$ did he just read [\sub{NP} the report [\sub{S} that was about \_$_i$]?\\
谁 \textsc{aux} 他 刚 读 {} \textsc{det} 报告 {} \textsc{rel} \textsc{cop} 关于\\
}
\z
尽管(\mex{0})在语义上是行得通的,这个句子仍是不合语法的。因为这里的疑问代词超越了关系小句的界限,也超越了NP的界限,由此它跨越了两个约束节点。有人认为,复杂NP限制适用于所有语言。但是,事实并非如此,因为在丹麦语\il{Danish} \citep[\page 55]{EL79a}、挪威语\il{Norwegian}、瑞典语\il{Swedish}、日语\il{Japanese}、韩语\il{Korean}、泰米尔语\il{Tamil}和库阿语\il{Akan}中是可以的(请看 \citew[\page 245, 262]{Hawkins99a}
和其中的参考文献)。由于复杂NP限制这个条件被整合进邻接原则中,相应地邻接原则并不具有普遍适用性,除非有人认为在分析的语言中,NP不是一个约束节点。然而,确实是大部分语言不允许复杂名词短语的提取。Hawkins基于处理难度来\isc{运用}\is{performance}分析与之相关的结构(\S~4.1)。他解释了允许这类提取的语言和不允许这类提取的语言之间的区别,这与各自语言中结构的不同处理容量是有关的,这些结构来自于与其他语法属性的提取的互动,如动词位置和其他约定的语法结构(\S~4.2)。
%Although (\mex{0}) would be a semantically plausible question, the sentence is still ungrammatical. This is explained by the fact that
%the question pronoun has been extracted across the sentence boundary of a relative clause and then across the NP boundary and has therefore
%crossed two bounding nodes. It is assumed that the CNPC holds for all languages. This is not the case, however, as the corresponding structures
%are possible in Danish\il{Danish} \citep[\page 55]{EL79a}, Norwegian\il{Norwegian},
%Swedish\il{Swedish}, Japanese\il{Japanese}, Korean\il{Korean}, Tamil\il{Tamil}
%and Akan\il{Akan} (see  \citew[\page 245, 262]{Hawkins99a} and references therein).
%Since the restrictions of the CNPC are integrated into the Subjacency Principle, it follows that the Subjacency Principle cannot be universally
%applicable unless one claims that NP is not a bounding node in the problematic languages. However, it seems
%indeed to be the case that the majority of languages do not allow extraction from complex noun phrases. Hawkins explains this on the basis of the processing difficulties\is{performance}
%associated with the structures in question (Section~4.1). He explains the difference between languages that allow this kind of extraction and languages that do not
%with reference to the differing processing load for structures that stem from the interaction of extraction with other grammatical properties such as verb position
%and other conventionalized grammatical structures in the respective languages (Section~4.2).

与复杂名词短语的提取不同的是,邻接原则并不排除跨越单句界限的提取,如(\mex{1})。
%Unlike extraction from complex noun phrases, extraction across a single sentence boundary (\mex{1}) is not ruled out by the Subjacency Principle.
\ea
\gll Who$_i$ did she think that he saw \_$_i$?\\
谁 \textsc{aux} 她 认为 \textsc{comp} 他 看见\\
\mytrans{她认为他看到的是谁?}
\z
正如在前面章节提到的,跨越多重句子边界的移位在转换理论中被叫做循环移位\isc{循环!转换的循环}\is{cycle!transformational}:疑问代词移到限定语位置,然后移到下一个最高的限定语上。这些移位步骤的每一步都受到邻接原则的限制。邻接原则一举排除了长距离移位。
%Movement across multiple sentence boundaries, as discussed in previous chapters, is explained by so"=called cyclic\is{cycle!transformational} movement
%in transformational theories: a question pronoun is moved to a specifier position and can then be
%moved further to the next highest specifier. Each of these
%movement steps is subject to the Subjacency Principle. The Subjacency Principle rules out
%long"=distance movement in one fell swoop.

邻接原则无法解释为什么嵌套在言说动词(\mex{1}a)或叙实动词(\mex{1}b)下的句子的提取是异常的\citep[\page 68--69]{EL79a}。
%The Subjacency Principle cannot explain why extraction from sentences embedded under verbs that specify the kind of utterance (\mex{1}a) or factive verbs (\mex{1}b)
%is deviant \citep[\page 68--69]{EL79a}. 
\eal
\ex[??]{
\gll Who$_i$ did she mumble that he saw \_$_i$?\\
谁 \textsc{aux} 她 咕哝 \textsc{comp} 他 看见\\
}
\ex[??]{
\gll Who$_i$ did she realize that he saw \_$_i$?\\
谁 \textsc{aux} 她 意识到 \textsc{comp} 他 看见\\
}
\zl
这些句子的结构看起来跟(\mex{-1})是一样的。在完整的句法理论中,有学者还试图将这些差异解释为邻接违反或对Ross限制的违反。由此,\citet[\page 401--402]{Stowell81a-u}认为,(\mex{0})中的句子与(\mex{-1})中的句子具有不同结构。
%The structure of these sentences seems to be the same as (\mex{-1}). In entirely syntactic approaches, it was also attempted to explain these differences as subjacency
%violations or as a violation of Ross' constraints. It has therefore been assumed \citep[\page 401--402]{Stowell81a-u} that the sentences in (\mex{0}) have a  structure different
%rom those in (\mex{-1}). 
Stowell将这些表示言说方式的动词的子句型论元看作是附加语。由于附加语从句对提取来说是孤岛条件,这就可以解释为什么(\mex{0}a)是标记性的了。附加语的分析也跟这些子句型论元是可以省略的事实是一致的。
%Stowell treats these sentential arguments of manner of speaking verbs as adjuncts. Since adjunct clauses are islands for extraction by assumption, this would explain why
%(\mex{0}a) is marked. The adjunct analysis is compatible with the fact that these sentential arguments can be omitted:
\eal
\ex 
\gll She shouted that he left.\\
她 大喊 \textsc{comp} 他 离开\\
\mytrans{她大喊着说他离开了。}
\ex 
\gll She shouted.\\
她 大喊\\
\mytrans{她大喊大叫。}
\zl
 \citet[\page 352]{AG2008a}指出,将这些从句看作是附加语的分析是不合理的,因为他们只跟非常有限的动词类别有关,即言说类动词和认为类动词。这一属性是论元的属性\isc{论元}\is{argument},不是附加语的属性\isc{附加语}\is{adjunct}。像地点修饰语这类附加语可以适用于很多动词类。而且,如果(\mex{0}b)中的句子论元被省略,它的含义也会发生改变:而(\mex{0}a)要求传达一些信息,而(\mex{0}b)并不需要。也可以像(\mex{1})那样用NP来替代句子论元,此时,我们一定不希望将它们处理为附加语。
% \citet[\page 352]{AG2008a} have pointed out that treating such clauses as adjuncts is not justified as they are only possible with a very restricted
%class of verbs, namely verbs of saying and thinking. This property is a property of arguments\is{argument} and not of adjuncts\is{adjunct}.
%Adjuncts such as place modifiers are possible with a wide number of verb classes. Furthermore, the meaning changes if the sentential argument is omitted as in (\mex{0}b):
%whereas (\mex{0}a) requires that some information is communicated, this does not have to be the case with (\mex{0}b). It is also possible to replace the sentential argument
%with an NP as in (\mex{1}), which one would certainly not want to treat as an adjunct.
\ea
\gll She shouted the remark/the question/something I could not understand.\\
她 大喊 \textsc{det} 话/\textsc{det} 问题/某事 我 能够 不 理解\\
\mytrans{她大喊着我听不懂的话/问题/某事}
\z
将这些句子论元归类为附加语不能扩展到叙实动词上,因为他们的句子论元不是可选的\citep[\page 352]{AG2008a}:
%The possibility of classifying sentential arguments as adjuncts cannot be extended to factive verbs
%as their sentential argument is not optional \citep[\page 352]{AG2008a}:

\eal
\judgewidth{??}
\ex[]{
\gll She realized that he left.\\
她 意识到 \textsc{comp} 他 离开\\
}
\ex[??]{
\gll She realized.\\
她 意识到\\
}
\zl

\noindent
 \citet{KK70a}针对叙实动词提出了一个带有名词性中心语的复杂名词短语。一个可选的“事实”删除"=转换\isc{转换!事实-删除@\emph{事实}-删除}\is{transformation!fact-Deletion@\emph{fact}"=Deletion}去掉了诸如(\mex{1}a)句中NP的中心语名词和限定语,从而推导出诸如(\mex{1}b)的句子(第159页)。
% \citet{KK70a} suggest an analysis of factive verbs that assumes a complex noun phrase with a nominal head. An optional
%\emph{fact} Deletion"=Transformation\is{transformation!fact-Deletion@\emph{fact}"=Deletion} removes the head noun and the determiner of the
%NP in sentences such as (\mex{1}a) to derive sentences such as (\mex{1}b) (page~159). 
\eal
\ex 
\gll She realized [\sub{NP} the fact [\sub{S} that he left]].\\
     她   意识到   {}         \textsc{det} 事实 {} \textsc{comp} 他 离开\\
\mytrans{她意识到他离开了的事实。}
\ex 
\gll She realized [\sub{NP} [\sub{S} that he left]].\\
她 意识到 {} {} \textsc{comp} 他 离开\\
\mytrans{她意识到他离开了。}
\zl
从这类句子中提取的不可能性可以通过假定两个约束节点有交叉来解释,这种交叉被认为是不可能的(有关结构的孤岛地位,请看\citealp[\S~4]{KK70a})。这个分析预测了从叙实动词的补足语从句中提取就跟在明显的NP论元中提取一样是不可行的,因为这两个结构是一样的。
%The impossibility of extraction out of such sentences can be explained by assuming that two boundary
%nodes were crossed, which was assumed to be impossible (on
%the island status of this construction, see \citealp[Section~4]{KK70a}). This analysis predicts that
%extraction from complement clauses of factive verbs should be just as bad as extraction from overt
%NP arguments since the structure for both is the same.
%\todostefan{Naja, ein Unterschied ist eben, ob
%  Material da ist, oder nicht.} 
但是,根据 \citet[\page 353]{AG2008a},事实并不是如此:
%According to  \citet[\page 353]{AG2008a}, this is, however, not the case: 
\eal
\judgewidth{??}
\ex[*]{
\gll Who did she realize the fact that he saw  \_$_i$?\\
谁 \textsc{aux} 她 意识到 \textsc{det} 事实 \textsc{comp} 他 看见\\
}
\ex[??]{
\gll Who did she realize that he saw  \_$_i$?\\
谁 \textsc{aux} 她 意识到 \textsc{comp} 他 看见\\
}
\zl

\noindent
跟 \citet{Erteschik81a}、 \citet{EL79a}、 \citet{Takami88a}\LATER{ \citet{Erteschik-Shir79a,Erteschik-Shir98a}}
和 \citet{vanValin98a}一样, \citet[\S~7.2]{Goldberg2006a}认为,空位一定属于可以潜在构成话语的焦点\isc{焦点}\is{focus}的那部分(请看 \citew{Cook2001a}、 \citew{deKuthy2002a},德语的请看
 \citew{Fanselow2003a})。这就意味着这个部分一定不能被预设\isc{预设}\is{presupposition}。\footnote{%
不管整句话是不是否定的,预设为真的仍为真。所以说,下面(i.a)和(i.b)都表明有一位法国国王。
  \eal
  \ex 
  \gll The King of France is bald.\\
  \textsc{det} 国王 \textsc{prep} 法国 \textsc{cop} 秃的\\
  \mytrans{法国国王是秃子。}
  \ex 
  \gll The King of France is not bald.\\
  \textsc{det} 国王 \textsc{prep} 法国 \textsc{cop} 不 秃的\\
  \mytrans{法国国王不是秃子。}
  \zllast
}
%Together with  \citet{Erteschik81a},  \citet{EL79a},  \citet{Takami88a}\LATER{ \citet{Erteschik-Shir79a,Erteschik-Shir98a}}
%and  \citet{vanValin98a},  \citet[Section~7.2]{Goldberg2006a} assumes that the gap must be in a part of the utterance
%that can potentially form the focus\is{focus} of an utterance (see  \citew{Cook2001a},  \citew{deKuthy2002a} and
% \citew{Fanselow2003a} for German). This means that this part must not be presupposed\is{presupposition}.\footnote{%
%  Information is presupposed if it is true regardless of whether the utterance is negated or not.
%  Thus, it follows from both (i.a) and (i.b) that there is a king of France.
%  \eal
%  \ex The King of France is bald.
%  \ex The King of France is not bald.
%  \zllast
%}
如果我们考虑这对邻接原则的事实意味着什么,那么我们就会注意到每个例子中,提取都发生在预设的材料中:
%If one considers what this means for the data from the subjacency discussion, then one notices that in each case extraction
%has taken place out of presupposed material:
\eal
\ex 复杂NP\\
%\ex Complex NP\\
\gll She didn't         see the          report that      was          about him. $\to$ The          report was         about him.\\
     她 \textsc{aux}.不 看见 \textsc{det} 报告 \textsc{rel} \textsc{cop} 关于 他     {}    \textsc{det} 报告   \textsc{cop} 关于 他\\
\mytrans{她没看见关于他的报告。$\to$ 报告是关于他的。}
\ex 认为动词或言说动词的补足语\\
%\ex Complement of a verb of thinking or saying\\
\gll She didn't whisper that he left. $\to$ He left.\\
她 \textsc{aux}.不 低声说 \textsc{comp} 他 离开 {} 他 离开\\
\mytrans{她没有低声说他离开了。$\to$ 他离开了。 }
\ex 叙实动词\\
%\ex Factive verb\\
\gll She didn't realize that he left. $\to$ He left.\\
她 \textsc{aux}.不 意识到 \textsc{comp} 他 离开 {} 他 离开\\
\mytrans{她没意识到他离开了。$\to$ 他离开了。}
\zl
Goldberg指出,属于背景信息的短语成分是孤岛(Backgrounded constructions are islands (BCI))。 \citet{AG2008a}对这个语义/""语用分析进行了实验测试,并且将之与纯粹的句法方法进行了比较。他们能够证明信息结构\isc{信息结构}\is{information structure}属性对于成分的提取能力发挥了重要的作用。 \citet[\S~3.H]{Erteschik73a-u}与 \citet[\page 375]{AG2008a}认为,不同语言为了排除提取,在有关多少成分必须属于背景信息方面是有差异的。任何情况下,我们不应该认为所有语言都应排除附加语的提取,因为像丹麦语\il{Danish}这些语言是可以从关系从句中进行提取的\isc{关系小句}\is{relative clause}。\footnote{%
在讨论基于普遍语法的方法是否是可被检验的问题时, \citet*[\page
  2669]{CKT2010a}声称是不可能从关系从句中进行提取的,而且这些语言的存在质疑了普遍语法这个概念。(“如果儿童习得任何语言时,都可以学会从关系从句中提取语言表达式,那么这就会对普遍语法的基本原则提出严重的质疑。”)由此,他们反驳了Evans和Levinson,以及Tomasello,后面的这些学者认为普遍语法方法是不可被检验的\isc{普遍语法!可证伪性}\is{Universal Grammar (UG)!falsifiability}。如果Crain、Khlentzos 和Thornton的观点是正确的,那么(\mex{1})就会证明普遍语法是假的,这样讨论就结束了。
}
%Goldberg assumes that constituents that belong to backgrounded information are islands (\emph{Backgrounded constructions are islands} (BCI)).
% \citet{AG2008a} have tested this semantic/""pragmatic analysis experimentally and compared it to a purely syntactic approach.
%They were able to confirm that information structural properties\is{information structure} play a significant role for the extractability
%of elements. Along with  \citet[Section~3.H]{Erteschik73a-u},  \citet[\page 375]{AG2008a} assume that languages differ with regard to how
%much constituents have to belong to background knowledge in order to rule out extraction.
%In any case we should not rule out extraction from adjuncts for all languages as there are languages such as Danish\il{Danish} where it is possible to
%extract from relative clauses\is{relative clause}.\footnote{%
%Discussing the question of whether UG"=based approaches are falsifiable,  \citet*[\page
%  2669]{CKT2010a} claim that it is not possible to extract from relative clauses and the existence
%of such languages would call into question the very concept of UG. (``If a child acquiring any language
%could learn to extract linguistic expressions from a relative clause, then this would seriously
%cast doubt on one of the basic tenets of UG.'') They thereby contradict Evans and Levinson as well as Tomasello, who claim that UG approaches are
%not falsifiable\is{Universal Grammar (UG)!falsifiability}. If the argumentation of Crain, Khlentzos and Thornton were correct, then (\mex{1}) would falsify
%UG and that would be the end of the discussion.
%}
 \citet[\page 61]{Erteschik73a-u}举出了如下的例子:
% \citet[\page 61]{Erteschik73a-u} provides the following examples, among others:
\eal
\label{Beispiel-Extraktion-Adjunkt}
\ex
\gll Det$_i$ er   der mange [der kan lide \_$_i$].\\
     \textsc{pron} \textsc{cop} 那  许多 \hspaceThis{[}\textsc{rel} 可以 喜欢\\
\mytrans{有许多喜欢那个的人。' (直译: `那个,有许多人喜欢。})
%     that are there  many \hspaceThis{[}that can like\\
%\mytrans{There are many who like that.' (lit.: `That, there are many who like.})
\ex
\gll Det    hus$_i$  kender jeg en    mand [som har købt \_$_i$].\\
     \textsc{det}   房子  认识  我 一 男人 \hspaceThis{[}\textsc{comp} \textsc{aux} 买\\
\mytrans{我认识一个买了那栋房子的人。' (直译: `这个房子,我认识一个买了它的人。})
%     that   house  know  I a man \hspaceThis{[}that has bought\\
%\mytrans{I know a man that has bought that house.' (lit.: `This house, I know a man that has bought.})
\zl

%\noindent 
Rizzi提出的有关邻接限制的参数化\isc{参数!邻接}\is{parameter!subjacency}在许多作品中被摒弃了,而且相关的效应被归为语法其他方面的差异\citep{Adams84a,CMC83a,Grimshaw86b,Kluender92a}。\il{English|)}
%Rizzi's parameterization\is{parameter!subjacency} of the subjacency restriction has been abandoned in many works, and the relevant effects have been
%ascribed to differences in other areas of grammar \citep{Adams84a,CMC83a,Grimshaw86b,Kluender92a}.\il{English|)}

在本小节,我们看到有一些不属于结构的句法属性因素,导致了左向移位可能受到了限制。除了信息结构的属性,语言处理方面的因素\label{Seite-Subjazenz-Performanz}\isc{运用}\is{performance}也起到了重要的作用\citep*{Grosu73a,EC2000a,Gibson98a,KK93a,Hawkins99a,SHS2007a}。所包括成分的长度,填充语和空位之间的距离、句法结构的复杂性,以及在填充语和空位之间的空间内相似的话语所指之间的干扰效应都是话语的可接受度的重要因素。由于不同语言在它们的句法结构方面是不同的,自然会有语言运用的不同效应,比如那些在外置和提取方面的差异。\isc{提取|)}\is{extraction|)}
%We have seen in this subsection that there are reasons other than syntactic properties of structure as to why leftward movement might be blocked.
%In addition to information structural properties, processing considerations\label{Seite-Subjazenz-Performanz}\is{performance}
%also play a role \citep*{Grosu73a,EC2000a,Gibson98a,KK93a,Hawkins99a,SHS2007a}.
%The length of constituents involved, the distance between filler and gap, definiteness, complexity of syntactic structure and
%interference effects between similar discourse referents in the space between the filler and gap are all important factors for
%the acceptability of utterances. Since languages differ with regard to their syntactic structure,
%varying effects of performance, such as the ones found for extraposition and extraction, are to be expected.\is{extraction|)} 

总而言之,我们可以说邻接限制不能说明德语和英语中的外置现象;另外,与使用邻接原则相比,借助信息结构和语言加工现象可以更好地解释提取限制。由此,假定邻接原则作为普遍的语法能力中的句法限制来解释语言事实就是不必要的了。
%In sum, we can say that subjacency constraints do not hold for extraposition in either German or English and furthermore that one can better
%explain constraints on extraction with reference to information structure and processing phenomena than with the Subjacency Principle.
%Assuming subjacency as a syntactic constraint in a universal competence grammar is therefore unnecessary to explain the facts.
\isc{邻接|)}\is{subjacency|)}

\subsection{表示时态、情态和体的语法语素}
%\subsection{Grammatical morphemes for tense, mood and aspect}

 \citet[\page 238]{Pinker94a}认为许多语言都有表示时体、情态、体、格和否定的语素的观点是正确的。但是,在关于一种语言中包括哪些语法属性以及他们是如何表达的方面是有着极大的差异的。
% \citet[\page 238]{Pinker94a} is correct in claiming that there are morphemes for tense, mood, aspect, case and negation in many languages. However, there is a great
%deal of variation with regard to which of these grammatical properties a language has and
%how they are expressed.

时态系统方面的差异的例子请看 \citew{wals-65,wals-66}。现代汉语\il{Mandarin Chinese}显然是一个例子:它几乎没有形态。在几乎每一种语言中,相同的语素可以按照一种或另一种形式出现的事实可以归结为某些事物需要被反复表达的事实,然后那些经常重复出现的事物就被语法化了。
%For examples of differences in the tense system see  \citew{wals-65,wals-66}. Mandarin
%Chinese\il{Mandarin Chinese} is a clear case: it has next to no morphology. The fact that the same morphemes occur in one form or another in almost every language
%can be attributed to the fact that certain things need to be expressed repeatedly and then things which are constantly repeated become grammaticalized.


\subsection{词类}
%\subsection{Parts of speech}
\label{Abschnitt-UG-Wortarten}

在第\ref{Abschnitt-neues-GB}节,我们提到了所谓的制图方法\isc{制图}\is{cartography},其中有些人提出了超过三十种功能范畴(请看第\pageref{Tabelle-Cinque}页有关Cinque\aimention{Guglielmo
  Cinque}的功能性中心语的表格\ref{Tabelle-Cinque}),并且认为这些范畴与相应固定的句法结构一起构成了普遍语法的一部分。 \citet[\page 55, 57]{CR2010a}甚至提出了超过400种被认为在所有语言的语法中起到了重要作用的功能范畴\isc{范畴!语法范畴}\is{category!functional}
。\footnote{%
问题是这些范畴到底是否属于普遍语法是不确定的。
}
%In Section~\ref{Abschnitt-neues-GB}, so"=called cartographic approaches\is{cartography} were mentioned, some of which assume over thirty functional categories
%(see Table~\ref{Tabelle-Cinque} on page~\pageref{Tabelle-Cinque} for Cinque's\aimention{Guglielmo
%  Cinque} functional heads) and assume that these categories form part of UG together with corresponding fixed syntactic structures.
% \citet[\page 55, 57]{CR2010a} even assume over 400 functional categories\is{category!functional}
%that are claimed to play a role in the grammars of all languages.\footnote{%
%	The question of whether these categories form part of UG is left open.
%}
再有,像\mbox{Infl}\isc{范畴!功能范畴!I}\is{category!functional!I}(屈折变化)和Comp\isc{范畴!功能范畴!C}\is{category!functional!C} (标补语)的特殊词类在构建所谓的普遍性的原则时也被提及了(Baltin \citeyear[\page
262]{Baltin81a};\citeyear{Baltin2006a};\citealp{Rizzi82b};\citealp[\page 38]{Chomsky86b};\citealp[\page 397]{Hornstein2013a})。
%Also, specific parts of speech such as \mbox{Infl}\is{category!functional!I} (inflection) and
%Comp\is{category!functional!C} (complementizer) are referred to when formulating principles that are assumed to be universal (Baltin \citeyear[\page
%262]{Baltin81a}; \citeyear{Baltin2006a}; \citealp{Rizzi82b}; \citealp[\page 38]{Chomsky86b};
%\citealp[\page 397]{Hornstein2013a}). 

Chomsky(\citeyear[\page 68]{Chomsky88a-u};\citeyear{Chomsky91a-u};\citeyear[\page 131]{Chomsky95a-u})、 \citet[\page 284, 286]{Pinker94a}、 \citet[\page 270]{Briscoe2000a}和 \citet[\page 621]{Wunderlich2004a}关于词类的天赋机制做了相对较少的假设:Chomsky认为所有的词汇范畴\isc{范畴!词汇范畴}\is{category!lexical}(动词\isc{动词}\is{verb}、名词\isc{名词}\is{noun}、形容词\isc{形容词}\is{adjective}和介词\isc{前置词}\is{adposition})都属于普遍语法,而语言具有处理他们的权利。Pinker、Briscoe和Wunderlich认为所有的语言都有名词和动词。再者,对普遍语法的批评提出了这样的疑问,我们能否在其他语言中找到这些句法范畴,并且它们具有诸如德语和英语这类语言中我们所熟知的形式。
%Chomsky (\citeyear[\page 68]{Chomsky88a-u}; \citeyear{Chomsky91a-u};
%\citeyear[\page 131]{Chomsky95a-u}),  \citet[\page 284, 286]{Pinker94a},  \citet[\page 270]{Briscoe2000a} and
% \citet[\page 621]{Wunderlich2004a} make comparatively fewer assumptions about the innate inventory of parts of speech:
%Chomsky assumes that all lexical categories\is{category!lexical} (verbs\is{verb},  nouns\is{noun},
%adjectives\is{adjective} and adpositions\is{adposition}) belong to UG and languages have these at their disposal.
%Pinker, Briscoe and Wunderlich assume that all languages have nouns and verbs.
%Again critics of UG raised the question as to whether these syntactic categories can be found in other languages in the form known to us from languages such as German and English.

 \citet[\page 72]{Braine87a}认为,诸如动词和名词的词类应该被看作是从像论元和谓词这样的基础概念推导而来的(请看 \citew[\page 257]{Wunderlich2008a})。这就意味着对于这些范畴的存在有一个独立的解释,他们并不是基于语言特有的天赋知识的。
% \citet[\page 72]{Braine87a} argues that parts of speech such as verb and noun should be viewed as derived from fundamental concepts like argument and predicate
%(see also  \citew[\page 257]{Wunderlich2008a}). This means that there is an independent explanation for the presence of these categories that is not
%based on innate language"=specific knowledge.

 \citet[\S~2.2.4]{EL2009a}讨论了类型学的文献,并且举出了缺少副词和形容词的语言的例子。作者引用了Straits Salish\il{Straits Salish}作为一种在动词和名词之间没有区别的语言的例子(请看 \citealp[\page 481]{EL2009b})。他们评论说,对于经常使用的排在前四位或前五位的非印欧语系的语言来说,是有必要增加额外的词类如状貌词\isc{状貌词}\is{ideophone}、方位词\isc{方位词}\is{positional}、副动词(coverb)\isc{副动词}\is{coverb}、量词\isc{量词}\is{classifier}等来分析这些语言。\footnote{%
相反的观点,请看 \citew[\page 465]{JP2009a}。
} 
% \citet[Section~2.2.4]{EL2009a} discuss the typological literature and give examples of languages which lack adverbs and adjectives.
%The authors cite Straits Salish\il{Straits Salish} as a language in which there may be no difference
%between verbs and nouns (see also \citealp[\page 481]{EL2009b}).
%They remark that it does make sense to assume the additional word classes ideophone\is{ideophone}, positional\is{positional}, coverb\is{coverb},
%classifier\is{classifier} for the analysis of non Indo"=European languages on top of the four or five normally used.\footnote{%
%	For the opposite view, see  \citew[\page 465]{JP2009a}.
%} 
如果我们认为语言可以从可能性的集合(工具包)\isc{普遍语法)!作为工具包的普遍语法}\is{Universal Grammar (UG)!as a toolkit}中进行选择且不会穷尽的话,这样对于基于UG的理论就不是问题了(\citealp[\page
263]{Jackendoff2002a-u};\citealp[\page 11]{Newmeyer2005a};\citealp*[\page 204]{FHC2005a};\citealp[\page 6--7]{Chomsky2007a};\citealp[\page 55, 58, 65]{CR2010a})。
但是,容忍这个观点是很武断的。我们可以针对至少一个语言去设定任意的一套词类系统,然后声称它是普遍语法的一部分,并进一步声明大部分(也许甚至是所有)语言并不会利用所有的词类。 \citet[\page 157]{Villavicencio2002a}就是这样认为的,他在范畴语法的框架下提出了范畴S、NP、N、PP和PRT。这种假设是不可证伪的\isc{普遍语法!可证伪性}\is{Universal Grammar (UG)!falsifiability}(关于相似情况的讨论以及更为普遍性的讨论参见\citealp[\page 436]{EL2009a}、\citealp[\page 471]{Tomasello2009a})。
%This situation is not a problem for UG"=based theories if one assumes that languages can choose from an inventory of possibilities (a toolkit)\is{Universal Grammar (UG)!as a toolkit}
%but do not have to exhaust it (\citealp[\page
%263]{Jackendoff2002a-u}; \citealp[\page 11]{Newmeyer2005a}; \citealp*[\page 204]{FHC2005a};
%\citealp[\page 6--7]{Chomsky2007a}; \citealp[\page 55, 58, 65]{CR2010a}). 
%However, if we condone this view, then there is a certain arbitrariness. It is possible to assume any parts of speech that one requires
%for the analysis of at least one language, attribute them to UG and then claim that most (or maybe even all) languages do not make use of
%the entire set of parts of speech. This is what is suggested by  \citet[\page 157]{Villavicencio2002a}, working in the framework
%of Categorial Grammar, for the categories S, NP, N, PP and PRT. This kind of assumption is not falsifiable\is{Universal Grammar (UG)!falsifiability} (see
%\citealp[\page 436]{EL2009a}; \citealp[\page 471]{Tomasello2009a} for a discussion of similar cases and a more general discussion).

而Evans和Levinson认为,我们需要其他范畴, \citet[\page
458]{Haspelmath2009a}和 \citet[\page 453]{Croft2009a}进而否定了跨语言共性的词类的存在。我认为这过于极端,我认为更好的研究策略是试图找到不同语言的共同点。\footnote{%
对比 \citew[\page 2]{Chomsky99a}:
“在缺乏相反的强有力的证据下,假定语言是统一的,带有简单可预测的话语属性的不同限制。”
} 
我们应该期待能够找到不能套用我们带有印欧语系偏见的语法概念的语言。
%Whereas Evans and Levinson assume that one needs additional categories,  \citet[\page
%458]{Haspelmath2009a} and  \citet[\page 453]{Croft2009a} go so far as to deny the existence of cross"=linguistic
%parts of speech. I consider this to be too extreme and believe that a better research strategy is to try and find commonalities between
%languages.\footnote{%
%  Compare  \citew[\page 2]{Chomsky99a}:
%``In the absence of compelling evidence to the contrary, assume languages to be uniform, with variety
%restricted to easily detectable properties of utterances.''
%} 
%One should, however, expect to find languages that do not fit into our Indo"=European"=biased conceptions of grammar.

\subsection{递归与无限}
%\subsection{Recursion and infinitude}
\label{Abschnitt-Rekursion}

在《科学》杂志的一篇文章上, \citet*{HCF2002a}进一步阐明了只有递归才具有具体领域的普遍性,“提供一个有限成分集合生成无限表达的能力”(请看第\pageref{NP-Regeln-Adj}页关于递归的短语结构规则的例子 (\ref{NP-Regeln-Adj}))。\footnote{%
在\emph{Cognition}中的一篇讨论性文章中, \citet*{FHC2005a}澄清了他们的观点,递归是唯一的针对具体语言和具体人类的属性,这是一个假设,而且可以根本就没有语言特有/种族特有的属性。然后,能力与属性的具体组合应该是人类特有的(第182--201页)。他们提出的另一个观点是,天赋的语言特有的知识具有对应于早期主流的生成语法所谓的复杂性(第182页)。
 \citet[\page 7]{Chomsky2007a}指出合并\isc{合并}\is{Merge}可以是非具体语言的操作,但是仍将之归为普遍语法。
}这个假设是具有争议的,而且既有理论模型的也有实证分析方面的反对理由。
%In an article in \emph{Science},  \citet*{HCF2002a} put forward the hypothesis that the only
%domain"=specific universal is recursion, ``providing the capacity to generate an infinite range of expressions from a finite set of elements'' (see (\ref{NP-Regeln-Adj}) on page~\pageref{NP-Regeln-%Adj} for an example of a
%recursive phrase structure rule).\footnote{%
%	In a discussion article in \emph{Cognition},  \citet*{FHC2005a} clarify that their claim that recursion is the only language"=specific and
%	human"=specific property is a hypothesis and it could be the case that are not any language"=specific/species"=specific properties at all.
%	Then, a particular combination of abilities and properties would be specific to humans
 %       (p.\,182--201). An alternative they consider is that innate language-specific knowledge has a complexity corresponding to what was assumed in earlier versions of Mainstream Generative %Grammar (p.\,182).
%
% removed paragraph because of layout
% \citet[\page 7]{Chomsky2007a} notes that Merge\is{Merge} could be a non language"=specific operation but still attributes it to UG.%
%} This assumption is controversial and there have been both formal and empirical objections to it.

\subsubsection{形式化的问题}
%\subsubsection{Formal problems}

语言能力是无限的这个观点是广为流传的,而且早在Humboldt的论著中就能够找到相关的说法:
%The claim that our linguistic capabilities are infinite is widespread and can already be found
%in Humboldt's work:\footnote{%
%The process of language is not simply one where an individual instantiation is created; at the same time it must allow for an
%indefinite set of such instantiations and must above all allow the expression of the conditions imposed by thought.
%Language faces an infinite and truly unbounded subject matter, the epitome of everything one can think of. Therefore, it must make
%infinite use of finite means and this is possible through the identity of the power that is
%responsible for the production of thought and language.
%}
\begin{quotation}
语言的生成并不是简单地创造一个话语的过程;与之同时,它还允许这类表达的无限集合,其中最为重要的是允许思想赋予表达的条件。语言面对着一个无限且真正没有边界的主观实在,一个人们可能想象的所有事情的缩影。由此,它必须能够利用有限的手段来表达无限的内容,而且这是可以通过负责思维和语言的生成的能力进行确认的。\citep[\page 108]{Humboldt88a-u}\footnote{%
Das Verfahren der Sprache ist aber nicht bloß ein solches, wodurch eine einzelne Erscheinung zustande kommt;
es muss derselben zugleich die Möglichkeit eröffnen, eine unbestimmbare Menge solcher Erscheinungen und unter allen,
ihr von dem Gedanken gestellten Bedingungen hervorzubringen.
Denn sie steht ganz eigentlich einem unendlichen und wahrhaft grenzenlosen Gebiete, dem Inbegriff alles
Denk\-baren gegenüber. Sie muss daher von endlichen Mitteln einen unendlichen Gebrauch machen, und
vermag dies durch die Identität der gedanken- und sprache\-erzeugenden Kraft.
}
\end{quotation}

\noindent
如果我们只看语言事实,我们能看到语句的长度是有上限的。这是因为特别长的语句无法处理,说话者需要睡觉或者最终会在某个时刻死去。如果我们设定一个长度为10万个语素的句子,然后假设一个语素库X,我们就可以构造出不到X$^{100,000}$个表达式。如果我们在每个100,000位置上使用每个语素的话,就会得到 X$^{100,000}$这个数。由于不是所有这些表达式都是合乎语法的,那么实际上是少于X$^{100,000}$个可能的表达式的(请看\citealp{Weydt72a}中类似的但是更为具体的说明)。这个数已经够大的了,但是仍是有限的。对于思维来说也是一样的:我们不需要无限多的思想(如果“无限”这个概念是按照这个词的数学含义来理解的话),尽管Humboldt和 \citet[\page 137]{Chomsky2008a}的观点是相反的。\footnote{%
 \citet{Weydt72a}讨论了Chomsky关于无限多句子的可能性的观点,以及Chomsky谈及Humboldt是否是合理的。Chomsky在《当下语言学理论的问题》\citep[\page 17]{Chomsky70b-ut}的引述中遗漏了下面这句话,即Denn sie steht ganz eigentlich einem unendlichen und
    wahrhaft grenzenlosen Gebiete, dem Inbegriff alles Denkbaren gegenüber。 \citet[\page 266]{Weydt72a}认为,Humboldt、Bühler\aimention{B{\"u}hler, Karl}和Martinet\aimention{Martinet, Andr{\'e}}声称有无限的思想可以被表达。Weydt认为这并不意味着句子可以有任意长度。相反,他认为文本的长度没有上限。这个论述是很有意思的,但我猜想文本只不过是更大的单位,而Weydt针对语言句子长度没有上限的说法也同样适用于文本。文本可以通过(i)中相当简化的规则生成,它将一个句子U和一个文本T组合成更大的文本T:
  \ea
  T $\to$ T U
  \z
U可以是一个文本的一个句子或者一个短语。如果我们准备好承认文本的长度没有上限的话,这就意味着句子的长度也没有上限,因为我们可以通过“和(and)”来将文本中所有的短语连接起来构成长句。这样的长句是将短句连接在一起的产物,它们跟乔姆斯基观点下所允准的非常长的句子在本质上是不同的,因为它们不包括一个任意深度(请看\ref{chap-competence-performance})的自我中心的嵌套\isc{自嵌套}\is{self"=embedding},但是无论如何从任意长度的文本造出的句子的数量是无限的。

对于任意长的文本来说,有一个有趣的问题:让我们来设想一个人造句并将这些句子加进一个现有的文本中。当这个人死了的时候,这项工作就被打断了。我们可以说另一个人可以接下这个文本,直到这个人去世,再这样继续下去。同样,问题是人们能否理解这个有几千万页的文本的意义和结构。如果这不足以成为一个问题,人们可以自问一直加句子到文本中直到2731年的这个人的语言还是不是在2015年开始这个文本的人的语言。如果这个问题的答案是否定的,那么这个文本就不是由一个语言L的句子构成的,而是由许多种语言构成的,这样就与争论的焦点无关了。
} 
%If we just look at the data, we can see that there is an upper bound for the length of utterances. This has to do with the fact that
%extremely long instances cannot be processed and that speakers have to sleep or will eventually die at some point.
%If we set a generous maximal sentence length at 100,000 morphemes and then assume a morpheme inventory of X then one can form less than
%X$^{100,000}$ utterances. We arrive at the number X$^{100,000}$ if we use each of the morphemes at each of the 100,000 positions.
%Since not all of these sequences will be well"=formed, then there are actually less than
%X$^{100,000}$ possible utterances (see also \citealp{Weydt72a} for a similar but more elaborate argument). This number
%is incredibly large, but still finite. The same is true of thought: we do not have infinitely many
%possible thoughts (if \emph{infinitely} is used in the mathematical sense of the word), despite
%claims by Humboldt and  \citet[\page 137]{Chomsky2008a} to the contrary.\footnote{%
%   \citet{Weydt72a} discusses Chomsky's statements regarding the existence of infinitely many
%  sentences and whether it is legitimate for Chomsky to refer to Humboldt. Chomsky's quote in
%  \emph{Current Issues in Linguistic Theory} \citep[\page 17]{Chomsky70b-ut} leaves out the sentence \emph{Denn sie steht ganz eigentlich einem unendlichen und
%    wahrhaft grenzenlosen Gebiete, dem Inbegriff alles Denkbaren gegenüber.}  \citet[\page 266]{Weydt72a} argues that
%  Humboldt, Bühler\aimention{B{\"u}hler, Karl} and Martinet\aimention{Martinet, Andr{\'e}} claimed that
%  there are infinitely many thoughts that can be expressed. Weydt
%  claims that it does not follow that sentences may be arbitrarily long. Instead he suggests that
%  there is no upper bound on the length of texts. This claim is interesting, but I guess texts are
%  just the next bigger unit and the argument that Weydt put forward against languages without an
%  upper bound for sentence length also applies to texts. A text can be generated by the rather
%  simplified rule in (i) that combines an utterance U with a text T resulting in a larger text T:
%  \ea
%  T $\to$ T U
%  \z
%  U can be a sentence or another phrase that can be part of a text. If one is ready to admit that
%  there is no upper bound on the length of texts, it follows that there cannot be an upper bound on
%  the length of sentences either, since one can construct long sentences by joining all phrases of a
%  text with \emph{and}. Such long sentences that are the product of conjoining short sentences are different in nature from very long sentences that are admitted under the
%  Chomskyan view in that they do not include center-self embeddings\is{self"=embedding} of an
%  arbitrary depth (see Section~\ref{chap-competence-performance}), but nevertheless the number of
%  sentences that can be produced from arbitrarily long texts is infinite.

%  As for arbitrarily long texts there is an interesting problem: Let us assume that a person
%  produces sentences and keeps adding them to an existing text. This enterprise will be interrupted
%  when the human being dies. One could say that another person could take up the text extension
%  until this one dies and so on. Again the question is whether one can understand the meaning and the structure
%  of a text that is several million pages long. 42. If this is not enough of a problem, one may ask
%  oneself whether the language of the person who keeps adding to the text in the year 2731 is still
%  the same that the person who started the text spoke in 2015. If the answer to this question is no,
%  then the text is not a document containing sentences from one language L but a mix from several
%  languages and hence irrelevant for the debate. 
%} 

在文献中,我们有时可以发现我们可以创造出无限长的句子的说法(请看 \citew*[\page 117]{NKN2001a}, \citew[\page 3]{KS2008a-u},以及 \citew{OW2012a}中Dan Everett的论述)。当然不是这样的。我们在第\ref{Kapitel-PSG}章提到的重写语法也不是这样的,这些语法允许无限句子的生成,因为规则右手边的符号集合被界定为有限的。但是,我们有可能生成无限数量的句子,句子本身不能是无限的,因为一个符号总是被无限多的其他符号所替代,由此会得到非有限符号的序列。
%In the literature, one sometimes finds the claim that it is possible to produce infinitely long
%sentences (see for instance  \citew*[\page 117]{NKN2001a} and  \citew[\page 3]{KS2008a-u} and Dan Everett in  \citew{OW2012a} at 25:19).
%This is most certainly not the case. It is also not the case that the rewrite grammars we encountered in
%Chapter~\ref{Kapitel-PSG} allow for the creation of infinite sentences as the set of symbols
%of the right"=hand side of the rule has to be finite by definition. While it is possible to derive
%an infinite number of sentences, the sentences themselves cannot be infinite, since it is always one
%symbol that is replaced by finitely many other symbols and hence no infinite symbol sequence may result.

 \citet[\S~I.1]{Chomsky65a}在 \citet{Saussure16a}\nocite{Saussure16a-Fr}影响下,区分语言能力\isc{语言能力}\is{competence}和语言运用\isc{语言运用}\is{performance}:语言能力是有关哪种语言结构是合乎语法的知识,而语言运用是对这种知识的应用(请看\ref{Abschnitt-Kompetenz-Performanz-TAG}和第\ref{Abschnitt-Diskussion-Performanz}章)。我们有限的大脑容量以及其他限制导致我们无法处理任意数量的嵌套,而且我们无法造出超过100,000个语素长度的句子。语言能力与语言运用之间的区分是有意义的,并且允许我们构造出对诸如(\mex{1})的句子的分析的规则:
% \citet[Section~I.1]{Chomsky65a} follows  \citet{Saussure16a}\nocite{Saussure16a-Fr} and draws a distinction between competence\is{competence} and performance\is{performance}:
%competence is the knowledge about what kind of linguistic structures are well"=formed, and performance is the application of this knowledge (see
%Section~\ref{Abschnitt-Kompetenz-Performanz-TAG} and Chapter~\ref{Abschnitt-Diskussion-Performanz}).
%Our restricted brain capacity as well as other constraints are responsible for the fact that we cannot deal with an arbitrary amount of embedding
%and that we cannot produce utterances longer than 100,000 morphemes. The separation between competence and performance makes sense and allows us to
%formulate rules for the analysis of sentences such as (\mex{1}):
\eal 
\label{Beispiel-Satzeinbettung}
\ex 
\gll Richard is sleeping.\\
Richard \textsc{aux} 睡觉\\
\mytrans{Richard在睡觉。}
\ex 
\gll Karl suspects that Richard is sleeping.\\
Karl 怀疑 \textsc{comp} Richard \textsc{aux} 睡觉\\
\mytrans{Karl怀疑Richard在睡觉。 }
\ex 
\gll Otto claims that Karl suspects that Richard is sleeping.\\
Otto 声称 \textsc{comp} Karl 怀疑 \textsc{comp} Richard \textsc{aux} 睡觉\\
\mytrans{Otto声称Karl怀疑Richard在睡觉。}
\ex 
\gll Julius believes that Otto claims that Karl suspects that Richard is sleeping.\\
Julius 认为 \textsc{comp} Otto 声称 \textsc{comp} Karl 怀疑 \textsc{comp} Richard \textsc{aux} 睡觉\\
\mytrans{Julius认为Otto声称Karl怀疑Richard在睡觉。}
\ex 
\gll Max knows that Julius believes that Otto claims that Karl suspects that Richard is sleeping.\\
Max 知道 \textsc{comp} Julius 认为 \textsc{comp} Otto 声称 \textsc{comp} Karl 怀疑 \textsc{comp} Richard \textsc{aux} 睡觉\\
\mytrans{Max 知道Julius认为Otto声称Karl怀疑Richard在睡觉。}
\zl
规则采用如下的形式:将一个名词短语跟某个类别的动词和一个从句相组合。通过连续地应用这条规则,就可以造出任意长度的字符串。 \citet{PS2010a}指出,我们必须把两件事区分开:语言是一个递归系统,还是我们能够设计的一种具体语言的最佳模型碰巧是递归的。更多有关这个方面和大脑处理的内容,请看 \citew{LL2011a}。当我们应用上面的系统来构造字符串时,我们不能说明(一个特别的)语言是无限的,即使通常是这样认为的(\citealp[\page 105--106]{Bierwisch66a};\citealp[\page 86]{Pinker94a};\citealp*[\page 1571]{HCF2002a}; \citealp[\page 1]{MuellerLehrbuch1};
\citealp*[\page 7]{HNG2005a};\citealp[\page 3]{KS2008a-u})。
%The rule takes the following form: combine a noun phrase with a verb of a certain class and a clause.
%By applying this rule successively, it is possible to form strings of arbitrary length.
% \citet{PS2010a} point out that one has to keep two things apart: the question of whether language
%is a recursive system and whether it is just the case that the best models that we can devise for a particular
%language happen to be recursive. For more on this point and on processing in the brain, see  \citew{LL2011a}.
%When constructing strings of words using the system above, it cannot be shown that (a particular) language is
%infinite, even if this is often claimed to be the case (\citealp[\page 105--106]{Bierwisch66a}; \citealp[\page 86]{Pinker94a}; \citealp*[\page 1571]{HCF2002a}; \citealp[\page 1]{MuellerLehrbuch1};
%\citealp*[\page 7]{HNG2005a}; \citealp[\page 3]{KS2008a-u}).

与证明没有最大自然数的证据相似的是,语言无限性的“证据”被看作是一个间接证据(\citealp[\page 105--106]{Bierwisch66a};\citealp[\page 86]{Pinker94a})。在自然数方面,是这样运作的:假设$x$是最大自然数。然后构造$x + 1$,因为这在定义上是一个自然数,我们现在找到一个比$x$更大的自然数。由此,我们说明了$x$是最大的数的假设导致了一个矛盾,所以说,就不可能有最大自然数这个说法。
%The ``proof'' of this infinitude of language is led as an indirect proof parallel to the proof that shows
%that there is no largest natural number (\citealp[\page 105--106]{Bierwisch66a}; \citealp[\page 86]{Pinker94a}). In the domain of natural numbers, this works
%as follows: assume $x$ is the largest natural number. Then form $x + 1$ and, since this is by
%definition a natural number, we have now found a natural number that is greater than $x$. We have
%therefore shown that the assumption that $x$ is the highest number leads to a contradiction and thus
%that there cannot be such a thing as the largest natural number. 

当我们把这个证据转化到自然语言领域,出现的问题是我们是否仍旧希望有一个1,000,000,000词的字符串作为我们想要描述的语言的一部分。如果我们还想这样做,那么这个证据就不能用了。
%When transferring this proof into the domain of natural language, the question arises as to whether one would still want to class 
%a string of 1,000,000,000 words as part of the language we want to describe. If we do not want this, then this proof will not work.

如果我们把语言看作是一个生物构造,那么我们就需要接受它是有限的事实。否则,我们必须要假定它是无限的,但是生物学上的真实个体的一个无限大的部分在生物学上是不存在的\citep[\page
111]{Postal2009a}。 \citet{LL2011a}将语言看作是物理学上不可数但是是字符串的有限集合。他们指出一定要做出这样的区分,即想象无限地扩展句子的能力,还是从一个不可数的字符串集合中拿出一个句子,然后真正地将之扩展的能力。我们具有第一种能力,不具有第二种。
%If we view language as a biological construct, then one has to accept the fact that it is finite. Otherwise, one is forced to assume
%that it is infinite, but that an infinitely large part of the biologically real object is not biologically real \citep[\page
%111]{Postal2009a}.  \citet{LL2011a} refer to languages as physically uncountable but finite sets of strings.
%They point out that a distinction must be made between the ability to imagine extending a sentence indefinitely and the ability
%to take a sentence from a non"=countable set of strings and really extend it. We possess the first ability but not the second.

为语言的无限性提供论据的一种可能是,认为只有创建了符合语法规则的话语集合的生成语法\isc{生成语法}\is{Generative Grammar}才适合于模拟语言,而且我们需要递归规则来捕捉语言事实,这就是为什么心智表达式具有递归机制,从而生成无限数量的表达式(Chomsky, \citeyear[\page 115]{Chomsky56a-u};\citeyear[\page
86--87]{Chomsky2002a-u}),然后这就暗示了语言包括无限多的表达式。 \citet{PS2010a}指出,这个论据有两个错误:即使我们认同生成语法,仍有可能是,即使有递归规则,对语境敏感的语法也只能生成有限的集合。 \citet[120--121]{PS2010a}引述了Andr{\'a}s Kornai\aimention{Andr{\'a}s Kornai}举的一个有趣的例子。
%One possibility to provide arguments for the infinitude of languages is to claim that only generative grammars\is{Generative Grammar},
%which create sets of well"=formed utterances, are suited to modeling language and that we need recursive rules to capture the data, which
%is why mental representations have a recursive procedure that generates infinite numbers of expressions (Chomsky, \citeyear[\page 115]{Chomsky56a-u}; \citeyear[\page
%86--87]{Chomsky2002a-u}), which then implies that languages consist of infinitely many expressions.
%There are two mistakes in this argument that have been pointed out by  \citet{PS2010a}: 
%even if one assumes generative grammars, it can still be the case that a context"=sensitive grammar can still only generate a finite set even with 
%recursive rules.  \citet[120--121]{PS2010a} give an interesting example from Andr{\'a}s Kornai\aimention{Andr{\'a}s Kornai}.

更为重要的问题是,没有必要假设语言生成集合。有三个明显的形式化的方案,但是只有第三种在这里提到了,即模型理论,也是基于约束的\isc{基于约束的语法}\is{constraint"=based grammar}方法\isc{模型理论的语法}\is{model"=theoretic grammar}(请看第\ref{Abschnitt-Generativ-Modelltheoretisch}章)。Johnson \& Postal的弧对语法\isc{弧对语法}\is{Arc Pair Grammar}
\citeyearpar{JP80a-u}、对 \citet{Kaplan95a}形式化的LFG\indexlfgc、对 \citet{Rogers97a}重新形式化的GPSG\indexgpsg,以及根据 \citet{King99a-u}假设的HPSG\indexhpsgc, \citet{Pollard99a}和 \citet{Richter2007a}采用的是模型理论的方法。在基于约束的理论中,我们可以分析像(\ref{Beispiel-Satzeinbettung})的例子,并且说特定的态度动词选择一个名词性NP和一个that从句,而且这些只能在特定的局部句法配置中出现,句法配置中所含成分之间具有特殊的关系。其中一种关系就是主谓一致。按照这个方式,我们可以表示诸如(\ref{Beispiel-Satzeinbettung})的表达式,并且不需要说明有多少句子是可以嵌套的。这就意味着基于约束的理论与结构的数量是有限还是无限这一问题的两个答案都兼容。通过使用按照这个方式构建出的语言能力语法,就有可能开发出语言使用的模型。它能够解释为什么有些字符串,比如说特别长的那些,是不可接受的(请看第\ref{Abschnitt-Diskussion-Performanz}章)。
%The more important mistake is that it is not necessary to assume that grammars generate
%sets. There are three explicitly formalized alternatives of which only the third is mentioned here,
%namely the model"=theoretic and therefore constraint"=based\is{constraint"=based grammar} approaches\is{model"=theoretic grammar} (see 
%Chapter~\ref{Abschnitt-Generativ-Modelltheoretisch}). Johnson \& Postal's Arc Pair Grammar\is{Arc Pair Grammar} \citeyearpar{JP80a-u}, LFG\indexlfg
%in the formalization of  \citet{Kaplan95a}, GPSG\indexgpsg in the reformalization of  \citet{Rogers97a} and HPSG\indexhpsg
%with the assumptions of  \citet{King99a-u},  \citet{Pollard99a} and  \citet{Richter2007a} are examples of model"=theoretic approaches.
%In constraint"=based theories, one would analyze an example like (\ref{Beispiel-Satzeinbettung}) saying that certain attitude verbs select a nominative NP and a 
%\emph{that} clause and that these can only occur in a certain local configuration where a particular relation holds between the elements involved.
%One of these relations is subject"=verb agreement. In this way, one can represent expressions such as (\ref{Beispiel-Satzeinbettung})
%and does not have to say anything about how many sentences can be embedded.
%This means that constraint"=based theories are compatible with both answers to the question of whether there is a finite or infinite number of structures.
%Using competence grammars formulated in the relevant way, it is possible to develop performance
%models that explain why certain strings -- for instance very long ones -- are unacceptable
%(see Chapter~\ref{Abschnitt-Diskussion-Performanz}).

\subsubsection{语言事实的问题}
%\subsubsection{Empirical problems}

有时人们认为所有的自然语言都是递归的,而且在所有语言中一个任意长度的句子是可能的(概述请看\citealp*[\page 7]{HNG2005a} ,更多内容请看 \citew[Section~2]{PS2010a})。当有人提到递归的时候,通常是指带有自我嵌套的结构,正如我们在(\ref{Beispiel-Satzeinbettung})的分析\citep{Fitch2010a}中所看到的。但是,有可能有的语言不允许自我嵌套。 \citet{Everett2005a-u}认为Pirah{\~a}\il{Pirah{\~a}}就是这样一种语言(但是,请看 \citew*{NPR2009a-u}和
 \citew{Everett2009a-u})。
没有递归的语言的另一个例子是Warlpiri\il{Warlpiri}语,它通常跟 \citew{Hale76a}一起引用。不过,Hale的规则针对带关系从句的句子的组合是递归的(第85页)。在第98页递归被明确地表示出来。\footnote{%
但是,他在第78页说明关系从句通过逗号跟包含中心语名词的句子区分开。Warlpiri语中的关系从句总是外围的,也就是说,它们出现在其所指名词的句子左边或右边。相似结构可以在德语中找到:
\ea
\gll Es war einmal ein Mann. Der hatte sieben Söhne.\\
	 那 \textsc{cop}  从前 一 男人 他 有 七 儿子\\
\mytrans{从前有个男人。他有七个儿子。}
\z
也有可能是我们在文本层\isc{文本}\is{text}处理句子的连接,而非带有递归的句子层。
} \citet[\page 131]{PS2010a}讨论了Hixkaryána\il{Hixkaryána}语,这是属于加勒比语族的亚马逊语,它与Pirah{\~a}语没有关系。这个语言有嵌套,但是嵌套的材料与其主句具有不同的形式。这些嵌套不能带有非限定性的特征。在Hixkaryána语中,也没有并列\isc{并列}\is{coordination}短语和从句( \citew[\page
45]{Derbyshire79a-u}援引 \citew[\page 131]{PS2010a}),这就是为什么这种语言中无法产生嵌套的句子。其他没有自嵌套的语言有Akkadian\il{Akkadian}语、Dyirbal\il{Dyirbal}语和原始"=Uralic\il{Proto"=Uralic}语。
%It is sometimes claimed that all natural languages are recursive and that sentences of an arbitrary
%length are possible in all languages (\citealp*[\page 7]{HNG2005a} for an overview, and see  \citew[Section~2]{PS2010a} for further references). When one speaks of recursion,
%what is often meant are structures with self"=embedding as we saw in the analysis of (\ref{Beispiel-Satzeinbettung}) \citep{Fitch2010a}. 
%However, it is possible that there are languages that do not allow self"=embedding.  \citet{Everett2005a-u} claims that
%Pirah{\~a}\il{Pirah{\~a}} is such a language (however, see  \citew*{NPR2009a-u} and
% \citew{Everett2009a-u}). 
%A further example of a language without recursion, which is sometimes cited with reference to  \citew{Hale76a}, is Warlpiri\il{Warlpiri}.
%Hale's rules for the combination of a sentence with a relative clause are recursive, however (page~85). This recursion is made
%explicit on page~98.\footnote{%
%	However, he does note on page~78 that relative clauses are separated from the sentence
%        containing the head noun by a pause. Relative clauses in Warlpiri are always peripheral, that is, they occur to the left or right of a sentence with the noun they refer to. Similar
%	constructions can be found in German:
%\ea
%\gll Es war einmal ein Mann. Der hatte sieben Söhne.\\
%	 there was once a man he had seven sons\\
%\mytrans{There once was a man. He had seven sons.}
%\z
%It could be the case that we are dealing with linking of sentences at text level\is{text} and not recursion at sentence level.
%}  \citet[\page 131]{PS2010a} discuss Hixkaryána\il{Hixkaryána}, an Amazonian language from the Caribbean language
%family that is not related to Pirah{\~a}. This language does have embedding, but the embedded
%material has a different form to that of the matrix clause. It could be the case that these embeddings cannot be carried out indefinitely. In Hixkaryána,
%there is also no possibility to coordinate\is{coordination} phrases or clauses ( \citew[\page
%45]{Derbyshire79a-u} cited by  \citew[\page 131]{PS2010a}), which is why this possibility of forming recursive sentence embedding does not
%exist in this language either. Other languages without self"=embedding seem to be Akkadian\il{Akkadian}, Dyirbal\il{Dyirbal} and Proto"=Uralic\il{Proto"=Uralic}.
	
当然,说所有语言都是递归的这个观点是有一定意义的:它们遵守这样的规则,一个具体数量的符号可以被组合以构成另一个符号。\footnote{%
 \citet[\page 11]{Chomsky2005a}认为归并\isc{合并}\is{Merge}组合了n个对象。一个特殊的例子是二元归并。\isc{分支!二叉}\is{branching!binary}
}
%There is of course a trivial sense in which all languages are recursive: they follow a rule that says that a particular number of symbols can be combined
%to form another symbol.\footnote{%
%   \citet[\page 11]{Chomsky2005a} assumes that Merge\is{Merge} combines n objects. A special instance of this is binary Merge.\is{branching!binary}
%}
\ea
X $\to$ X \ldots{} X
\z
据此,所有的自然语言都是递归的,而且简单符号的组合以构成更为复杂的符号是语言的一个基本属性\citep[\page
6]{Hockett60a}。关于Pirah{\~a}语的争论是如此的尖锐,这个事实可以说明这不是所谓的那种递归。另参见 \citet{Fitch2010a}。
%In this sense, all natural languages are recursive and the combination of simple symbols to more complex ones is a basic property of language \citep[\page
%6]{Hockett60a}. The fact that the debate about Pirah{\~a} is so fierce could go to show that this is not the kind of recursion that is meant.
%Also, see  \citet{Fitch2010a}.

同样也有人认为,范畴语法\indexcgc 的组合性规则也具有语言的普遍性。我们可以利用这些规则来将函项跟它的论元相组合(\mbox{X/Y $*$ Y = X})。这些规则跟(\mex{0})的规则几乎一样抽象。区别是其中一个成分必须是函项。在最简方案\indexmpc 中也有对应的限制,比如说选择性特征(请看\ref{Abschnitt-MG})和语义角色指派的限制。但是,是否是范畴语法允准了递归结构并不取决于普遍的组合模式,而是词汇项。使用(\mex{1})中的词汇项,只有可能分析两个句子,并且一定不会构建递归的结构。
%It is also assumed that the combinatorial rules of Categorial Grammar\indexcg hold universally. It is possible to use these rules to
%combine a functor with its arguments (\mbox{X/Y $*$ Y = X}). These rules are almost as abstract as the rules in (\mex{0}). The difference
%is that one of the elements has to be the functor. There are also corresponding constraints in the Minimalist Program\indexmp such
%as selectional features (see Section~\ref{Abschnitt-MG}) and restrictions on the assignment of
%semantic roles. However, whether or not a Categorial Grammar licenses recursive structures does not depend on the very general combinatorial schemata, but rather on the lexical entries.
%Using the lexical entries in (\mex{1}), it is only possible to analyze two sentences and certainly not to build recursive structures.

\eal
\ex the: np/n
\ex woman: n
\ex cat: n
\ex sees: (s\bs np)/np
\zl
如果我们扩展词汇来囊括范畴n/n的修饰语,范畴的连词(X\bs X)/X,那么我们就会得到一个递归语法。
%If we expand the lexicon to include modifiers of the category n/n or conjunctions of the category (X\bs X)/X, then we arrive at a recursive
%grammar.

 \citet*[\page 203]{FHC2005a}指出,不允准递归结构的语言事实对于基于普遍语法的理论来说并不是一个问题,因为并不是普遍语法的所有可能性都能够通过具体的语言来应用。按照这个观点,我们实际上跟词类一样面临同样的情况(参见\ref{Abschnitt-UG-Wortarten}),你可以提出任意数量属于普遍语法的属性,然后再根据它们是否发挥重要作用的语言基础来决定具体语言的情况。这个方法的一个极端的变体就是所有语言的语法变成了普遍语法中的一部分(也许带有诸如NP\sub{Spanish}、NP\sub{German}的不同符号)。这个针对语言的有关人类能力的基于普遍语法的变体事实上是不可证伪的\isc{普遍语法!可证伪性}\is{Universal Grammar (UG)!falsifiability}(\citealp[\page 436, 443]{EL2009a};\citealp[\page 471]{Tomasello2009a})。
% \citet*[\page 203]{FHC2005a} note that the existence of languages that do not license recursive structures is not a problem for UG"=based theories
%as not all the possibilities in UG have to be utilized by an individual language.
%With this view, we have actually the same situation as with parts of speech (see
%Section~\ref{Abschnitt-UG-Wortarten}) that you can posit any number of properties belonging to UG
%and then decide on a language by language basis whether they play a role or not. An extreme variant of this approach would be that grammars of all languages become part of 
%UG (perhaps with different symbols such as NP\sub{Spanish}, NP\sub{German}). This variant of a UG"=based theory of the human capacity for language
%would be truly unfalsifiable\is{Universal Grammar (UG)!falsifiability} (\citealp[\page 436, 443]{EL2009a}; \citealp[\page 471]{Tomasello2009a}).

\subsubsection{在认知的其他领域中的递归}
%\subsubsection{Recursion in other areas of cognition}

在语言领域之外也有可以用递归规则描写的现象: \citet*[\page 1571]{HCF2002a}提到了导航、家族关系和计数系统。\footnote{\label{fn-Rekursion-Mathematik}%
但是, \citet[\page 230]{PJ2005a}指出,导航与Chomsky描述的递归系统不同,递归不是在所有的文化中都是计数系统的一部分。他们认为,那些发展出无限计数系统的文化可以这么做(译者注:将递归作为计数系统的一部分),因为他们有语言能力。 \citet*[\page 203]{FHC2005a}也这样认为。后者认为在其他领域中的递归的所有形式都取决于语言。关于这一点,更多内容请参考 \citew[\page 7--8]{Chomsky2007a}。 \citet{LL2011a}指出,自然数被界定为是递归的,但是数学定义并不必然对人类采用的数学运算发挥重要的作用。
}
%There are also phenomena in domains outside of language that can be described with recursive rules:
% \citet*[\page 1571]{HCF2002a} mention navigation, family relations and counting systems.\footnote{\label{fn-Rekursion-Mathematik}%
%   \citet[\page 230]{PJ2005a} note, however, that navigation differs from the kind of recursive system described by Chomsky and that recursion
 % is not part of counting systems in all cultures. They assume that those cultures that have developed
%  infinite counting systems could do this because of their linguistic capabilities. This is also assumed by  \citet*[\page 203]{FHC2005a}.
%  The latter authors claim that all forms of recursion in other domains depend on language. For more on this point, see
%   \citew[\page 7--8]{Chomsky2007a}.  \citet{LL2011a} note that natural numbers are defined
%  recursively, but the mathematical definition does not necessarily play a role for the kinds of arithmetic operations carried out by humans.
%}
我们也可以认为,相关的能力是后来习得的,更高水平的数学属于个人成就,跟大多数人的认知能力是无关的,但是即使是3岁9个月的小孩儿都已经可以产出递归的结构了:
%One could perhaps argue that the relevant abilities are acquired late and that higher mathematics is a matter of individual accomplishments that do not
%have anything to do with the cognitive capacities of the majority, but even children at the age of 3
%years and 9 months are already able to produce recursive structures:
%\todostefan{M: Is this true also  for children of nonlinguists?}
2008年,有篇新闻报道说从一架飞机上拍到了一个土生土长的巴西部落。我把这张照片给我儿子\aimention{Max M{\"u}ller}看,并告诉他土著美洲人用弓和箭射飞机。然后他问我那是哪种飞机。我告诉他你看不到,因为拍照片的人坐在飞机里。然后,他就说,如果你要拍下既有这架飞机,也有土著美洲人的照片的话,你就需要另一架飞机。他对他的主意感到很开心,然后他说“然后又有一架。然后又有一架。一架接着一架。”据此,他一定可以想象出嵌套的结果。
%In 2008, there were newspaper reports about an indigenous Brazilian tribe that was photographed from
%a plane. I showed this picture to my son\aimention{Max M{\"u}ller} and told him that Native Americans shot at the plane with a bow and arrow. He then asked me what kind of plane it was. I told %him that you cannot 
%see that because the people
%who took the photograph were sitting in the plane. He then answered that you would then need another
%plane if you wanted to take a photo that contained both the plane and the Native Americans. He was
%pleased with his idea and said ``And then another one. And then another one. One after the
%other''. He was therefore very much able to imagine the consequence of embeddings.

 \citet[\page 113--114]{CJ2005a}讨论了感知\isc{视觉感知}\is{visual perception}和音乐\isc{音乐}\is{music}作为独立于语言的递归系统。 \citet{Jackendoff2011a}将这个讨论扩展到视觉感知和音乐,并加上了程式域(以做咖啡为例)和无词的连环画。 \citet[\page 7--8]{Chomsky2007a}声称,视觉感知的例子是无关的,但是之后承认构建递归结构的能力可以属于更为普遍的认知能力(第8页)。他还是把这个能力归结为普遍语法。他把普遍语法看作是语言能力的一个子集,也就是说,作为语言需要的非特定域能力(广义的语言官能\isc{语言官能!广义的语言官能}\is{Faculty of Language! in the Broad Sense (FLB)})和特定域能力(狭义的语言官能\isc{语言官能!狭义的语言官能}\is{Faculty of
  Language! in the Narrow Sense (FLN)})的一个子集。
% \citet[\page 113--114]{CJ2005a} discuss visual perception\is{visual perception} and music\is{music} as recursive systems that are
%independent of language.  \citet{Jackendoff2011a} extends the discussion of visual perception and
%music and adds the domains of planning (with the example of making coffee) and wordless comic strips.
% \citet[\page 7--8]{Chomsky2007a} claims that examples from visual perception are irrelevant but then admits that the ability to build up recursive structures
%could belong to general cognitive abilities (p.\,8).
%He still attributes this ability to UG. He views UG as a subset of the Faculty of Language, that is, as a subset of non domain"=specific abilities
%(Faculty of Language in the Broad Sense = FLB\is{Faculty of Language! in the Broad Sense (FLB)}) and the domain"=specific abilities (Faculty of Language in the Narrow Sense = FLN\is{Faculty %of
%  Language! in the Narrow Sense (FLN)}) required for language.

\subsection{小结}
%\subsection{Summary}
\label{Abschnitt-Universalien-Zusammenfassung}

综上所述,我们可以说,已经发现的语言上的共性并未让人们就以下观点达成一致:一定要假设特定领域的天赋知识来解释语言上的共性。在2008年召开的“德国语言学学会”会议上,Wolfgang Klein\aimention{Wolfgang Klein}承诺,如果有人能够指出所有语言共享的不平凡的属性,他就奖励给他100欧元(请参考\citealp{Klein2009a})。这就涉及到对“平凡”的界定。看上去很清楚的是,所有语言都共享谓词论元结构\isc{谓词论元结构}\is{predicate"=argument structure}和某些意义上(\citealp[]{Hudson2010a};\citealp[\page 2701]{LR2010a})的依存关系\isc{依存}\is{dependency},而且,所有语言都有可以根据组合关系构成的复杂表达式(由于Manfred Krifka\aimention{Manfred Krifka}提出了组合性\isc{组合性}\is{compositionality},被允诺了20欧)。但是,正如在很多地方所提到的,普遍性绝不意味着天赋性(\citealp[\page 189]{Bates84a};\citealp[\page 205]{Newmeyer2005a}):Newmeyer举例说,表示太阳和月亮的词几乎在所有的语言中都有。这跟天体在每个人的生活中起到的重要作用的事实有关系,所以人们需要用词来指称它们。不能由此来下结论说相应的概念是天赋的。相似地,用来表示两个物体之间的关系的词(如“接”)必须按照明显的方式与描述这两个物体的词(“我”、“大象”)有联系。但是,这并不必然说明语言的这个属性是天赋的。
%In sum, we can say that there are no linguistic universals for which there is a consensus that one
%has to assume domain"=specific innate knowledge to explain them.  At the 2008 meeting of the
%\emph{Deutsche Gesellschaft für Sprachwissenschaft}, Wolfgang Klein\aimention{Wolfgang Klein}
%promised \euro~100 to anyone who could name a non"=trivial property that all languages share
%(see \citealp{Klein2009a}). This begs the question of what is meant by `trivial'. It seems clear
%that all languages share predicate"=argument structures\is{predicate"=argument structure} and
%dependency relations\is{dependency} in some sense (\citealp[]{Hudson2010a}; \citealp[\page
%  2701]{LR2010a}) and, all languages have complex expressions whose meaning can be determined compositionally (Manfred Krifka\aimention{Manfred Krifka} was promised
%20\,\euro{} for coming up with compositionality\is{compositionality}). However, as has been noted at various
%points, universality by no means implies innateness (\citealp[\page 189]{Bates84a};
%\citealp[\page 205]{Newmeyer2005a}): Newmeyer gives the example that words for sun and moon probably
%exist in all languages. This has to do with the fact that these celestial bodies play an important
%role in everyone's lives and thus one needs words to refer to them.  It cannot be concluded from
%this that the corresponding concepts have to be innate. Similarly, a word that is used to express a
%relation between two objects (\eg \emph{catch}) has to be connected to the words describing both of
%these objects (\emph{I}, \emph{elephant}) in a transparent way. However, this does not necessarily entail that this property of language is innate.

即使我们可以找到所有语言共享的结构属性,我们仍然无法证明天赋语言知识的存在,因为这些相似性可以被追溯到其他因素上。有人认为,所有语言都必须按照这个方式以使得少量的有限资源能够被儿童习得(\citealp[\S~10.7.2]{Hurford2002a};\citealp[\page 433]{Behrens2009a})。按照这种说法,在它发展的相关阶段,我们的大脑就是一个限制因素。语言需要适合我们的大脑,而因为我们的大脑是相似的,语言也在某些方面是相似的(请参考\citealp[\page 251]{Kluender92a})。
%Even if we can find structural properties shared by all languages, this is still not proof of innate linguistic knowledge, as these similarities could
%be traced back to other factors. It is argued that all languages must be made in such a way as to be acquirable with the paucity of resource available to small
%children (\citealp[Section~10.7.2]{Hurford2002a}; \citealp[\page 433]{Behrens2009a}).
%It follows from this that, in the relevant phases of its development, our brain is a constraining factor.
%Languages have to fit into our brains and since our brains are similar, languages are also similar
%in certain respects (see \citealp[\page 251]{Kluender92a}).
\isc{普遍性|)}\is{universal|)}

\section{语言习得的速度}
%\section{Speed of language acquisition}
\label{Abschnitt-Geschwindigkeit-Spracherwerb}

通常\isc{习得!习得速度|(}\is{acquisition!speed|(}认为,儿童学习语言异常快,而这只能是因为他们已经掌握了不需要习得的关于语言的知识(如\citealp[\page 144]{Chomsky76c-u};\citealp[\page
395]{Hornstein2013a})。为了让这个论述更为严谨,它必须证明复杂度相当的其他方面的知识需要更长的习得时间\citep[\page 214--218]{Sampson89a}。但还没有看到相关的证据。语言习得跨越了多年的时间,而且不可能简单地证明语言是通过“短暂地接触”而习得的。Chomsky将语言学跟物理学相比,并指出我们要习得物理学的知识更为困难。但是, \citet[\page 215]{Sampson89a}指出,人们在学校或大学习得的物理学知识并不能构成比较的基础,相反,人们应该考虑我们每天身处的物理世界的知识的习得。比如说,我们想要将液体倒进容器、用跳绳跳的那种知识,或者物体的弹道性能的知识。为了对语言习得进行说明,需要对这些领域的知识进行对比,其复杂性并非微不足道。对于这方面的深入讨论,请参考 \citew[\page 214--218]{Sampson89a}。 \citet[\page 1]{MR98a-u}指出,六岁的儿童可以理解23,700个词汇,并会运用5000多个。据此,在四年半的时间里,他们平均每天学14个新词。这确实是令人惊叹的,但是不能作为天赋的语言知识的论据,因为所有习得的理论都认为词是通过语言事实学习的,而不是由基因确定的普遍句法先天决定的。任何情况下,基因编码的假设对于新造词来说都是不太可能的,如fax(传真)、iPod、 e-mail(电子邮件)、Tamagotchi\isc{电子鸡}\is{Tamagotchi}(电子鸡)。
%It\is{acquisition!speed|(} is often argued that children learn language extraordinarily quickly and
%this can only be explained by assuming that they already possess knowledge about language that does
%not have to be acquired (\eg \citealp[\page 144]{Chomsky76c-u}; \citealp[\page
%  395]{Hornstein2013a}).  In order for this argument to hold up to closer scrutiny, it must be demonstrated
%that other areas of knowledge with a comparable degree of complexity require longer to acquire
%\citep[\page 214--218]{Sampson89a}. This has not yet been shown.  Language acquisition spans several
%years and it is not possible to simply state that language is acquired following \emph{brief
%exposure}.  Chomsky compares languages to physics and points out that it is considerably more
%difficult for us to acquire knowledge about physics.   \citet[\page 215]{Sampson89a} notes, however,
%that the knowledge about physics one acquires at school or university is not a basis for comparison
%and one should instead consider the acquisition of everyday knowledge about the physical world around
%us.  For example, the kind of knowledge we need when we want to pour liquids into a container, skip
%with a skipping rope or the knowledge we have about the ballistic properties of objects. The
%complexity in comparing these domains of knowledge in order to be able to make claims about language
%acquisition may turn out to be far from trivial.  For an in-depth discussion of this aspect, see
% \citew[\page 214--218]{Sampson89a}.   \citet[\page 1]{MR98a-u} point out that children at the age of
%six can understand 23,700 words and use over 5000.  It follows from this that, in the space of four
%and a half years, they learn on average 14 new words every day. This is indeed an impressive
%feat, but cannot be used as an argument for innate linguistic knowledge as all theories of
%acquisition assume that words have to be learned from data rather than being predetermined by a
%genetically"=determined Universal Grammar. In any case the assumption of genetic encoding would be highly
%implausible for newly created words such as \emph{fax}, \emph{iPod}, \emph{e-mail}, \emph{Tamagotchi}\is{Tamagotchi}.

此外,跟第二语言习得\isc{习得!第二语言习得}\is{acquisition!second language}相比,第一语言习得是毫不费力且迅速的这个说法是错误的,正如 \citet[\page 9]{Klein86a-u}的估算所显示的:如果我们假设儿童每天听五个小时的语言会话(保守估计),那么在他们生命中的头五年中,他们有9100小时的语言训练。但是到五岁的时候,他们仍没有习得所有的复杂结构。相比较而言,第二语言学习者,假设有必要的动机,可以在为期六周每天十二个小时(总共500小时)的集训课上学会一门语言的语法。\isc{习得!习得速度|)}\is{acquisition!speed|)}
%Furthermore, the claim that first language acquisition is effortless and rapid when compared to second language acquisition\is{acquisition!second language} is a myth
%as has been shown by estimations by  \citet[\page 9]{Klein86a-u}: if we assume that children hear linguistic utterances for five hours a day (as a conservative
%estimate), then in the first five years of their lives, they have 9100 hours of linguistic training. But at the age of five, they have still not acquired all complex constructions.
%In comparison, second"=language learners, assuming the necessary motivation, can learn the grammar of a language rather well in a six-week crash course with
%twelve hours a day (500 hours in total).\is{acquisition!speed|)}

\section{习得的关键期}
%\section{Critical period for acquisition}

鸭子具有关键期\isc{关键期|(}\is{critical period|(}\isc{习得|(}\is{acquisition|(},期间它们受到父母行为的显著影响。通常,小鸭子跟随它们的妈妈。但是,如果在这个特殊的时期,有一个人出现,而不是它们的妈妈,这些鸭子就会跟着这个人。过了关键期后,这种对它们行为的影响就不存在了\citep{Lorenz70a-u}。这种关键期在其他动物和其他认知领域中也可以被发现,例如灵长类动物的视觉能力的习得。特定的能力在特定的时间框架下习得,而相关输入的存在对决定这个时间框架的开始阶段是重要的。
%Among ducks\is{critical period|(}\is{acquisition|(}, there is a critical phase in which their
%behavior towards parent figures is influenced significantly. Normally,
%baby ducks follow their mother. If, however, a human is present rather than the mother during a particular time span, the ducks will follow the human.
%After the critical period, this influence on their behavior can no longer be identified \citep{Lorenz70a-u}. This kind of critical period can also be identified
%in other animals and in other areas of cognition, for example the acquisition of visual abilities among primates.
%Certain abilities are acquired in a given time frame, whereby the presence of the relevant input is important for determining the start of this
%time frame.\todostefan{ \citew{Hurford91a-u} diskutieren}

 \citet{Lenneberg64a}认为,语言习得只持续到12岁,并且从儿童可以比大人学语言学得更好这点得出结论,这也有赖于关键期,而且语言习得一定具有跟鸭子的印随行为相似的属性,所以说,语言习得的特质必须是天赋的\citep[\S~4]{Lenneberg67a-u}。
% \citet{Lenneberg64a} claims that language acquisition is only possible up to the age of twelve and concludes from the fact that children can learn
%language much better than adults that this is also due to a critical period and that language acquisition must have similar properties to
%the behavior of ducks and hence, the predisposition for language acquisition must be innate \citep[Chapter~4]{Lenneberg67a-u}.

对于关键期的长度的假说有相当大的分歧。有人认为是5年,有人认为是6年、12年,甚至是15年\citep[\page 31]{HBW2003a}。另一种跟关键期相关的假说是假定习得语言的能力随着时间持续减弱。
%The assumptions about the length of the critical period for language acquisition vary considerably. It is possible to find suggestions for 5, 6, 12 and even 15 years
%\citep[\page 31]{HBW2003a}. An alternative assumption to a critical period would be to assume that the ability to acquire languages decreases continuously
%over time. 
 \citet{JN89a}试图判断出第二语言习得\isc{习得}\is{acquisition}的关键期,并且认为第二语言的学习从15岁开始变得尤为糟糕。\nocite{Sorace2003a}
% \citet{JN89a} tried to determine a critical period for second"=language acquisition\is{acquisition} and they claim that a second language is learned
%significantly worse from the age of 15.\nocite{Sorace2003a}
但是, \citet*[\page]{EBJKSPP96a}指出,Johnson和Newport的数据有一个不同的曲线,更适合于个人数据。另一个曲线表明学习语言的能力没有突然的变化,而是一个稳定的下降过程,因此没有证据证明关键期带来的影响。
% \citet*[\page]{EBJKSPP96a} have, however, pointed out that there is a different curve for Johnson and Newport's data that fits the individual data better. The alternative curve
%shows no abrupt change but rather a steady decrease in the ability to learn language and therefore offers no proof of an effect created by a critical period.

 \citet*{HBW2003a}评估了移民到美国的2,016,317位讲西班牙语\il{Spanish}的移民和324,444位讲汉语\il{Mandarin Chinese}的移民所做的调查问卷的数据。他们调查了他们的年龄、移民时间点、教育水平和他们习得的英语水平\il{English}。他们无法确定出从何时开始语言习得受到了严重的限制。相反,随着年龄的增长,他们的学习能力有着稳定的下降过程。这也可以在其他领域中被观察到:比如说,越晚学开车就越难。
% \citet*{HBW2003a} evaluate data from a questionnaire of 2,016,317 Spanish speakers\il{Spanish} and 324,444 speakers of Mandarin Chinese\il{Mandarin Chinese}
%that immigrated to the United States. They investigated which correlations there were between age, the point at immigration, the general level of education of the speakers
%and the level of English they acquired\il{English}.
%They could not identify a critical point in time after which language acquisition was severely restricted.
%Instead, there is a steady decline in the ability to learn as age increases. This can also be observed in other
%domains: for example, learning to drive at an older age is much harder.

由此可见,没有证据证明第二语言习得存在关键期。有时,有人认为第二语言习得不是由天赋的普遍语法驱动的,而实际上是一个在关键期就已经接触知识的学习过程\citep[\page 176]{Lenneberg67a-u}。由此,我们可以说第一语言习得有关键期。但是,出于伦理学的原因,我们不能直接地用实验来控制输入量。我们不能说,找20个孩子,然后让他们在没有语言输入的环境中长到3岁、4岁、5岁、6岁……或15岁,然后比较结果。这种研究有赖于非常少见的出于被忽视的案例。比如说, \citet{Curtiss77a-u}研究了一个叫Genie的女孩儿。那时,Genie13岁,并且一直在隔离的环境中长大。她也被叫做野孩。\isc{野孩}\is{feral child}正如Curtiss所介绍的,她没能再学会特定的语言规则。为了进行客观的比较,我们还需要其他的没有在完全隔离和非人环境中成长的被试的测试数据。唯一收集相关实验数据的可能性是研究到一定年龄还没有接受过任何手语输入的聋儿。 \citet[\page 63]{JN89a}用美国手语\il{sign language!American (ASL)}做了一些相关的试验。试验也表明在学习能力方面有一个线性的下降过程,但是语言习得并没有从某个年龄就突然下降或者能力完全丧失。\isc{关键期|)}\is{critical period|)}\isc{习得|)}\is{acquisition|)}
%Summing up, it seems to be relatively clear that a critical period cannot be proven to exist for second"=language acquisition. Sometimes, it is assumed anyway that second"=language acquisition
%is not driven by an innate UG, but is in fact a learning process that accesses knowledge already
%acquired during the critical period \citep[\page 176]{Lenneberg67a-u}.
%One would therefore have to show that there is a critical period for first"=language
%acquisition. This is, however, not straightforward as, for ethical reasons, one cannot experimentally
%manipulate the point at which the input is available. We cannot, say, take 20 children and let them grow up without linguistic input to the age
%of 3, 4, 5, 6, \ldots{} or 15 and then compare the results. This kind of research is dependent on thankfully very rare cases of neglect. For example,  \citet{Curtiss77a-u}
%studied a girl called Genie. At the time, Genie was 13 years old and had grown up in isolation. She is a so"=called feral child.\is{feral child}
%As Curtiss showed, she was no longer able to learn certain linguistic rules. For an objective comparison, one would need other test subjects that had not grown up
%in complete isolation and in inhumane conditions. The only possibility of gaining relevant
%experimental data is to study deaf subjects that did not receive any input from a sign language up to a certain age.  \citet[\page
%63]{JN89a} carried out relevant experiments with learners of American Sign Language\il{sign language!American (ASL)}. It was also shown here that there is a linear
%decline in the ability to learn, however nothing like a sudden drop after a certain age or even a complete loss of the ability to acquire language.\is{critical period|)}\is{acquisition|)}

\section{非人类灵长动物的习得缺乏}
%\section{Lack of acquisition among non"=human primates}

非人类的灵长动物不能学会自然语言的事实被当作我们的语言能力由基因决定的证据。所有的科学家都认同这样的事实,人类和灵长类动物之间的差异是由基因决定的,而且这跟语言能力是相关的。
%The fact that non-human primates cannot learn natural language is viewed as evidence for the genetic determination of our linguistic ability. All scientists agree on the fact that there are %genetically"=determined
%differences between humans and primates and that these are relevant for linguistic ability.
 \citet{Friederici2009a}对文献进行了综述,并提出,在大猩猩\isc{大猩猩}\is{chimpanzee}和猕猴\isc{猕猴}\is{macaque}(以及儿童)中,它们大脑各部分之间的连接不如成年人的发达。大脑的相连区域一起负责词汇"=语义知识的处理,并且可以构成语言发展的重要前提(第179页)。
% \citet{Friederici2009a} offers an overview of the literature that claims that in chimpanzees\is{chimpanzee} and macaques\is{macaque} (and small children),
%the connections between parts of the brain are not as developed as in adult humans. The connected regions of the brain are together responsible for
%the processing of lexical"=semantic knowledge and could constitute an important prerequisite for the development of language (p.\,179).

但问题是,我们与其他灵长类动物的区别是在于我们具有针对语言的特殊认知能力,还是因为我们在认知的领域一般性(domain-general)方面与灵长类动物不同,由此我们才具备了掌握语言的能力?不过, \citet[\S~2]{Fanselow92b}认为人类特有的形式能力不必然是针对语言的。相似地, \citet[\page 7--8]{Chomsky2007a}考虑归并\isc{合并}\is{Merge}(按照他的观点,这是唯一的构建结构的操作)是否属于语言特有的天赋能力,还是人类特有的一般能力(但是,参见\ref{Abschnitt-Rekursion},尤其是脚注\ref{fn-Rekursion-Mathematik})。
%The question is, however, whether we differ from other primates in having special cognitive
%capabilities that are specific to language or whether our capability to acquire languages is due to
%domain-general differences in cognition.  \citet[Section~2]{Fanselow92b} speaks of a human"=specific formal competence that does not necessarily
%have to be specific to language, however. Similarly,  \citet[\page 7--8]{Chomsky2007a} has considered whether Merge\is{Merge} (the only structure"=building
%operation, in his opinion), does not belong to language"=specific innate abilities, but rather to general human"=specific competence (see, however,
%Section~\ref{Abschnitt-Rekursion}, in particular footnote~\ref{fn-Rekursion-Mathematik}).  

我们可以确知的是非人类灵长动物不懂得特殊的指示手势。人类喜欢模仿。其他灵长类动物也模仿,不过,不是出于社会因素\citep[\page 9--10]{Tomasello2006c}。根据 \citet[\page 676]{TCCBM2005a},只有人类具有能力和动力按照共同的目标和社会合作的行动计划来施行合作的活动。灵长类动物懂得意向性的动作,但是,只有人类在大脑中带着共同的目标进行行动,即共享意识(shared intentionality)。
只有人类使用和懂得手语\citep[\page 685, 724, 726]{TCCBM2005a}。语言在更高的层面进行合作:符号被用来指称物体,并且有时也指向说话者或听话者。为了能够使用这种交流系统,我们必须能够在交谈中设身处地为他人着想,并且发展出共同的期待和目标\citep[\page 683]{TCCBM2005a}。
这样,非人类的灵长动物缺乏语言的社会和认知的前提条件,也就是说人类和其他灵长类动物之间的区别不必用天赋的语言学知识来解释(\citealp[\S~8.1.2]{Tomasello2003a};\citealp{TCCBM2005a})。
%One can ascertain that non"=human primates do not understand particular pointing gestures. Humans like to imitate things. Other primates also imitate, however, not
%for social reasons \citep[\page 9--10]{Tomasello2006c}. According to  \citet[\page 676]{TCCBM2005a}, only humans have the ability and motivation to
%carry out coordinated activities with common goals and socially"=coordinated action plans. Primates do understand intentional actions, however, only humans
%act with a common goal in mind (\emph{shared intentionality}).
%Only humans use and understand hand gestures \citep[\page 685, 724, 726]{TCCBM2005a}. Language is collaborative to a high degree: symbols
%are used to refer to objects and sometimes also to the speaker or hearer. In order to be able to use this kind of communication system, one has to be able to
%put oneself in the shoes of the interlocutor and develop common expectations and goals \citep[\page 683]{TCCBM2005a}.
%Non"=human primates could thus lack the social and cognitive prerequisites for language, that is, the difference between humans and other primates does not have
%to be explained by innate linguistic knowledge (\citealp[Section~8.1.2]{Tomasello2003a};
%\citealp{TCCBM2005a}).

\section{克里奥尔语和手语}
%\section{Creole and sign languages}

当\isc{克里奥尔语|(}\is{creole language|(}\il{sign language|(}说话者没有一个共同语可以互相交流的时候,他们就会发展出所谓的皮钦语\isc{皮钦语}\is{pidgin language}。这些语言使用有限的词汇和非常基础的语法。需要指出的是,这些说皮钦语的人的后代会对这些语言进行规范化,创造一种带有独立语法的语言。这些语言被叫做克里奥尔语(creole languages)。有一种假说认为,从克里奥尔语发展出的语言形式受到天赋的普遍语法的限制\citep{Bickerton84a}.。一般认为,克里奥尔语的参数设置对应于参数的默认值\isc{参数}\is{parameters}(Bickerton \citeyear[\page 217]{Bickerton84b};\citeyear[\page 178]{Bickerton84a}),也就是说,在出生时参数就拥有值,而且这些对应于克里奥尔语具有的值。这些默认的值在学习其他语言的时候需要进行调节。\footnote{%
对于默认值假说带来的问题,请参考 \citew[\page 17]{Meisel95a}。 \citet[\page 56, fn.\,13]{Bickerton97a}不同意克里奥尔语具有默认的参数值的说法。
} 
Bickerton认为克里奥尔语包括语言学习者无法从输入中习得的成分,即从皮钦语中获得。他的论断是经典的刺激贫乏论\isc{刺激贫乏论}\is{Poverty of the Stimulus}的变体,我们将在\ref{Abschnitt-PSA}中具体讨论。
%When\is{creole language|(}\il{sign language|(} speakers that do not share a common language wish to communicate with each other, they develop so"=called
%pidgin languages\is{pidgin language}. These are languages that use parts of the vocabularies of the languages involved but have a very rudimentary
%grammar. It has been noted that children of pidgin speakers regularize these languages. The next generation of speakers creates a new language with an independent
%grammar. These languages are referred to as \emph{creole languages}.
%One hypothesis is that the form of languages that develop from creolization is restricted by an innate UG \citep{Bickerton84a}. It is assumed that
%the parameter setting of creole languages corresponds to the default values of parameters\is{parameters} (Bickerton \citeyear[\page 217]{Bickerton84b};
%\citeyear[\page 178]{Bickerton84a}), that is, parameters already have values at birth and these
%correspond to the values that creole languages have. These default values would have to be modified when learning other languages.\footnote{%
%	For problems that can arise from the assumption of defaults values, see   \citew[\page
%  17]{Meisel95a}.  \citet[\page 56, fn.\,13]{Bickerton97a} distances himself from the claim that creole languages have
%  the default values of parameters.
%} 
%Bickerton claims that creole languages contain elements that language learners could not have
%acquired from the input, that is from the pidgin languages. His argumentation is a variant of the classic Poverty of the Stimulus Argument\is{Poverty of the Stimulus} that will be discussed in more %detail in Section~\ref{Abschnitt-PSA}.

Bickerton的观点倍受争议,因为它无法证实儿童是否从成人的个体语言中得到输入(\citealp[\page 207]{Samarin84a};\citealp[\page 209]{Seuren84a})。考虑到证据缺乏,我们只能说有一些人口统计学上的事实显示,至少对一些克里奥尔语来说是这样的\citep{Arends2008a}。
这意味着儿童接受到的输入不仅包括来自皮钦语的字符串(string)输入,还有来自父母及周围其他人的个人语言中的句子输入。许多研究克里奥尔语的人认为成人对新涌现的语言贡献了具体的语法形式。例如,在夏威夷克里奥尔英语\il{Hawaiian Creole English}中,人们观察到,它受到使用者母语的影响:说日语的人既使用SOV语序,也使用SVO语序,而说菲律宾语的人既使用VOS语序,也使用SVO语序。总之,在语言中有相当多的变体可以追溯到个别说话者的不同母语上。
%Bickerton's claims have been criticized as it cannot be verified whether children had input in the individual languages
%of the adults (\citealp[\page 207]{Samarin84a}; \citealp[\page 209]{Seuren84a}).\todostefan{M: more recent work: Arends, Mufwene, Parkvall} All that can be said considering this lack of evidence %is that there are a number
%of demographic facts that suggest that this was the case for at least some creole languages
%\citep{Arends2008a}.
%This means that children did not only have the strings from the pidgin languages as an input but
%also sentences from the individual languages spoken by parents and others around them. Many creolists assume that adults contribute specific grammatical forms to the emerging language. For %example, in the case of Hawaiian Creole English\il{Hawaiian Creole English}
%one can observe that there are influences from the mother tongues of the speakers involved: Japanese speakers use SOV order as well as SVO and Philippinos use VOS order as well
%as SVO order. In total, there is quite a lot of variation in the language that can be traced back to the various native languages of the individual speakers.

我们也可以不用语言特有的天赋知识的假说来解释克里奥语化的效应:儿童规范语言的事实可以归功于独立于语言的现象。在实验中,给参与者看两个灯泡,然后被试必须预测哪个灯泡接下来会亮。如果其中一个灯泡70\%的时间都是亮的,参与者也会70\%的时间也选这个(尽管他们实际上可以有更高的成功率,如果他们总是选择有70\%的概率会亮的那个灯泡)。这个行为被叫做概率匹配(Probability Matching)\isc{概率匹配}\is{Probability
Matching}。如果我们再加一个灯泡,然后把这个灯泡在70\%的情况下点亮,这样,他们就在这个最为常见的情况的方向下进行了规范\citep{Gardener57a,Weir64a}。
%It is also possible to explain the effects observed for creolization without the assumption of innate language"=specific knowledge:
%the fact that children regularize language can be attributed to a phenomenon independent of language. In experiments, participants were shown two
%light bulbs and the test subjects had to predict which of the light bulbs would be turned on next. If one of the bulbs was switched on 70\% of the
%time, the participants also picked this one 70\% of the time (although they would have actually had a higher success rate if they had always chosen
%the bulb turned on with 70\% probability). This behavior is known as \emph{Probability Matching}\is{Probability
%Matching}. If we add another light bulb to this scenario and then turn this lamp on in 70\% of cases and the other two each 
%15\% of the time, then participants choose the more frequently lit one 80--90\% of the time, that is, they regularize in the direction of
%the most frequent occurrence \citep{Gardener57a,Weir64a}.

儿童比成人更多地进行规范化\citep{HudsonN99a,HKN2005a},这个事实跟有限的脑容量有关(“更少就是更多”—假说,\citealp{Newport90a,Elman93a})。
%Children regularize more than adults \citep{HudsonN99a,HKN2005a}, a fact that can be traced back to their
%limited brain capacity (``less is more''-hypothesis, \citealp{Newport90a,Elman93a}). 

跟克里奥尔语一样,我们可以在手语\isc{手语}\is{sign language}的习得的某些社会环境中找到类似的情境: \citet{SN2004a}展示了,一个学习美国手语(ASL)\isc{手语!美国手语}\is{sign language!American (ASL)}的孩子(Simon)比他的父母犯的错误少很多。他的父母是在15或16岁第一次学习美国手语的,并且只有在70\%的时间运用特殊的规定的动作。Simon则90\%的时间都做这些动作。他把从他的父母那里得到的输入规范化了,其中形式"=意义对儿的持续使用发挥了重要的作用,也就是说,他没有简单地使用概率匹配,而是有选择地学习。 \citet[\page 401]{SN2004a}怀疑,这种规范化在克里奥尔语和手语的涌现过程中也起到了重要的作用。但是,我们还没有可以证明这个假说的相关的统计学数据。
%Like creolization, a similar situation can be found in certain social contexts with the acquisition of sign language\is{sign language}:
% \citet{SN2004a} have shown that a child (Simon) that learned American Sign Language (ASL)\is{sign language!American (ASL)} makes considerably
%less mistakes than his parents. The parents first learned ASL at the age of 15 or 16 and performed particular obligatory movements only
%70\% of the time. Simon made these movements 90\% of the time. He regularized the input from his parents, whereby the consistent use
%of form"=meaning pairs plays an important role, that is, he does not simply use Probability Matching, but learns selectively.
%  \citet[\page 401]{SN2004a} suspect that these kinds of regularizations also play a role for the emergence of creole and sign languages.
%However, the relevant statistical data that one would need to confirm this hypothesis are not available.
\isc{克里奥尔语|)}\is{creole language|)}\il{sign language|)}


\section{大脑部位的定位}
%\section{Localization in special parts of the brain}

通过在语言生成或处理阶段测量大脑的活动,以及探查有脑损伤的病人,我们可以辨认出大脑的某些部分(布洛卡区\isc{布洛卡区}\is{Broca's area}和维尼克区\isc{维尼克区}\is{Wernicke's area})在语言生成和处理中发挥了重要的作用(有关当代研究的概况,请参考 \citew{Friederici2009a})。Chomsky说有一个语言中心,甚至把它比作一个器官(\emph{organ}\isc{器官}\is{organ})(\citealp[\page 164]{Chomsky77c-u};\citealp[\page 1]{Chomsky2005a};\citealp[\page 133]{Chomsky2008a})。这种定位被当作是我们的语言知识的内在基础的证据(也请参考\citealp[\page 297--314]{Pinker94a})。
%By measuring brain activity during speech production/processing and also by investigating patients with brain damage, one can identify
%special parts of the brain (Broca's area\is{Broca's area} and
%Wernicke's area\is{Wernicke's area}) that play an important role for language production and processing (see  \citew{Friederici2009a} for a current overview).
%Chomsky talks about there being a center of language and even calls this (metaphorically) an
%\emph{organ}\is{organ} (\citealp[\page 164]{Chomsky77c-u}; \citealp[\page 1]{Chomsky2005a}; \citealp[\page 133]{Chomsky2008a}).
%This localization was seen as evidence for the innate basis for our linguistic knowledge (see also \citealp[\page 297--314]{Pinker94a}). 

但是,如果这些部位被损坏了,大脑的其他区域可以接管相应的功能。如果损害发生在幼儿早期,语言也可以不在大脑中的这些部位学习(相关资源,请看\citealp[\S~4.1]{Dabrowska2004a})。
%However, it is the case that if these parts are damaged, other areas of the brain can take over the relevant
%functions. If the damage occurs in early childhood, language can also be learned without these special areas
%of the brain (for sources, see \citealp[Section~4.1]{Dabrowska2004a}).

除此之外,也可以观察到,大脑的某些部位在阅读时被激活。如果在大脑的某些部位处理的定位导致语言知识的天赋机制是有效的,那么在阅读时,大脑某些部位的激活就会让我们得出这样的结论,阅读的能力是天赋的(\citealp[\page ]{EBJKSPP96a};\citealp[\page 57]{Bishop2002a})。但是,并没有这样假设(也请参考\citealp*[\page 196]{FHC2005a})。
%Apart from that, it can also be observed that a particular area of the brain is activated when reading. If the conclusion about the localization of processing 
%in a particular part of the brain leading to the innateness of linguistic knowledge were valid, then the activation of certain
%areas of the brain during reading should also lead us to conclude that the ability to read is innate (\citealp[\page ]{EBJKSPP96a};
%\citealp[\page 57]{Bishop2002a}). This is, however, not assumed (see also \citealp*[\page 196]{FHC2005a}). 

还可以指出的是,语言处理影响大脑的几个区域,而且不只是布洛卡区和维尼克区(\citealp[\page
11]{FM2005a};\citealp{Friederici2009a})。另一方面,布罗卡区和维尼克区在非语言的任务中也是活跃的,如模仿、运动协调和音乐的处理\isc{音乐}\is{music}\citep{MKGF2001a}。有关内容的概括和更多的资源,请参考 \citew{FM2005a}。
%It should also be noted that language processing affects several areas of the brain and not just Broca's and Wernicke's areas (\citealp[\page
%11]{FM2005a}; \citealp{Friederici2009a}). On the other hand, Broca's and Wernicke's areas are also active during non"=linguistic tasks
%such as imitation, motoric coordination and processing of music\is{music} \citep{MKGF2001a}. For an overview and further sources,
%see  \citew{FM2005a}.

 \citet{MMGRRBW2003a}考察了第二语言习得中的大脑活动。他们给德语为母语者意大利语\il{Italian}和日语\il{Japanese}的语料,然后观察布洛卡区的活动。然后,他们将这个与用意大利语和日语词的人工语言进行对比,这些语言并不符合作者提出的普遍语法的原则。比如说,他们的人工语言的处理包括例(\mex{1})中显示的调换语序的问句。
% \citet{MMGRRBW2003a} investigated brain activity during second"=language acquisition. They gave German native speakers data from Italian and Japanese and noticed
%that there was activation in Broca's area. They then compared this to artificial languages that used Italian\il{Italian} and Japanese\il{Japanese} words but did
%not correspond to the principles of Universal Grammar as assumed by the authors. An example of the processes assumed in their artificial language is the formation
%of questions by reversing of word order as shown in (\mex{1}).
\eal
\ex 
\gll This is a statement.\\
这 \textsc{cop} 一 声明\\
\mytrans{这是一个声明。}
\ex 
\gll Statement a is this?\\
     声明 一 \textsc{cop} 这\\
\mytrans{这是一个声明吗?}
\zl
然后作者观察大脑在学习人工语言时的不同的活跃区域。这是一个有趣的结果,但是不能说明我们具有天赋的语言知识。它只说明了在我们处理我们的自然语言时活跃的区域在学习其他语言时也是活跃的,而且像调换词序这种玩词游戏也会影响大脑的其他区域。
%The authors then observed that different areas of the brain were activated when learning this artificial language. This is an interesting result, but does not show
%that we have innate linguistic knowledge. It only shows that the areas that are active when processing our native languages are also active when we learn other
%languages and that playing around with words such as reversing the order of words in a sentence affects other areas of the brain.

关于大脑的特殊部位的语言的定位部分的详细讨论可以参考 \citew[\S~4]{Dabrowska2004a}。
%A detailed discussion of localization of languages in particular parts of the brain can be found
%in  \citew[Chapter~4]{Dabrowska2004a}.

\section{语言跟一般认知的区别}
%\section{Differences between language and general cognition}

不认同天赋的语言知识的学者们相信语言可以通过一般的认知手段来获得。如果可以证明带有严重认知缺陷的人仍能获得正常的语言能力或者具有正常智力水平的人的语言能力十分有限的话,那么我们就可以说明语言和一般的认知是没有关系的。
%Researchers who believe that there is no such thing as innate linguistic knowledge assume that language can be acquired with general cognitive
%means. If it can be shown that humans with severely impaired cognition can still acquire normal linguistic abilities or that there are people
%of normal intelligence whose linguistic ability is restricted, then one can show that language and general cognition are independent.

\subsection{威廉综合症}
%\subsection{Williams Syndrome}

有\isc{威廉综合症|(}\is{Williams Syndrome|(} 一些智商(IQ\isc{IQ}\is{IQ})特别低,但是能说出合法句子的人。这些人当中有的人有威廉综合症(关于具有威廉综合症的人的能力的讨论,请参考 \citew*{BLJLG2000a})。 \citet{Yamada81a}将这些作为证据来证明是有一个独立于其他智力的语法模型的。
%There\is{Williams Syndrome|(} are people with a relatively low IQ\is{IQ}, who can nevertheless produce grammatical utterances.
%Among these are people with Williams Syndrome (see  \citew*{BLJLG2000a} for a discussion of the abilities of people with
%Williams Syndrome).  \citet{Yamada81a} takes the existence of such cases as evidence for a separate module of grammar, independent
%of the remaining intelligence.

IQ是根据实足年龄在智商测验(心理年龄)中取得的分数决定的。被研究的青少年都有对应于四到六岁儿童的心理年龄。不过这个年龄的儿童在很多方面已经接近成年人的语言能力。 \citet*[\page 295]{GSP94a}说明了具有威廉综合症的儿童确实显示了语言缺陷,并且他们的语言能力跟他们的心理年龄相称。对于患有威廉综合症的人在形态句法方面的问题,请见 \citew{KGBDHU97a}。有关威廉综合症的讨论在 \citew{Karmiloff-Smith98a}中有很好的总结。
%IQ is determined by dividing a score in an intelligence test (the mental age) by chronological age. The teenagers that were studied
%all had a mental age corresponding to that of a four to six year-old child. Yet children at this age already boast impressive
%linguistic ability that comes close to that of adults in many respects.  \citet*[\page 295]{GSP94a} have shown that children
%with Williams Syndrome do show a linguistic deficit and that their language ability corresponds to what would be expected
%from their mental age. For problems of sufferers of Williams Syndrome in the area of morphosyntax, see  \citew{KGBDHU97a}.
%The discussion about Williams Syndrome is summarized nicely in  \citew{Karmiloff-Smith98a}.
\isc{威廉综合症|)}\is{Williams Syndrome|)}

\subsection{带有FoxP2基因突变的KE家族}
%\subsection{KE family with FoxP2 mutation}

这\isc{FoxP2|(}\is{FoxP2|(}\isc{基因|(}\is{gene|(}是一个有语言问题的英国家庭,即所谓的KE家族。这个备受语言问题困扰的家族成员具有基因缺陷。 \citet{FVKWMP98a}和 \citet{LFHVM2001a}发现,这跟FoxP2基因(FoxP2表示\emph{Forkhead-Box P2})的变异有关。 \citet{GC91a}从形态方面的问题遗传于基因缺陷这样的事实中总结出,一定有负责某个具体的语法模型(形态\isc{形态}\is{morphology})的基因。但是, \citet[\page 930]{VKWAFP95a}证明了,KE家族不只在形态句法上有问题:受到影响的家庭成员在智力、语言、以及面部肌肉的运用上都有问题。考虑到面部肌肉受到相当多的限制动作,可以假定他们的语言困难也来自于运动的问题\citep[\page 285]{Tomasello2003a}。不过,KE家族的语言学问题不仅限于语言生成的问题,也有语言理解的问题\citep[\page 58]{Bishop2002a}。尽管如此,我们不能将语言缺陷直接联系到FoxP2上,因为还有一些其他能力也受到FoxP2变异的影响:发音受阻、形态和句法,也对非语言的智商和面部肌肉的运动问题有影响,以及非语言任务的处理也有问题\citep{VKWAFP95a}。
%There\is{FoxP2|(}\is{gene|(} is a British family -- the so"=called KE family -- that has problems with language.
%The members of this family who suffer from these linguistic problems have a genetic defect.  \citet{FVKWMP98a} and  \citet{LFHVM2001a}
%discovered that this is due to a mutation of the FoxP2 gene (FoxP2 stands for \emph{Forkhead-Box P2}).
% \citet{GC91a} conclude from the fact that deficits in the realm of morphology are inherited with genetic defects that
%there must be a gene that is responsible for a particular module of grammar (morphology\is{morphology}).
% \citet[\page 930]{VKWAFP95a} have demonstrated, however, that the KE family did not just have problems with morphosyntax:
%the affected family members have intellectual and linguistic problems together with motoric problems with facial muscles.
%Due to the considerably restricted motion in their facial muscles, it would make sense to assume that their linguistic difficulties also
%stem from motory problems \citep[\page 285]{Tomasello2003a}. The linguistic problems in the KE family are not just limited
%to production problems, however, but also comprehension problems \citep[\page 58]{Bishop2002a}.
%Nevertheless, one cannot associate linguistic deficiencies directly with FoxP2 as there are a number of other abilities that
%are affected by the FoxP2 mutation: as well as hindering pronunciation, morphology and syntax, it also has an effect on
%non"=verbal IQ and motory problems with the facial muscles, dealing with non-linguistic tasks, too \citep{VKWAFP95a}.

%% Furthermore, FoxP2 also occurs in animals. For example, the human gene differs from the analogous
%% gene of a mouse\is{mouse} in only three amino acid positions, and from those of
%% chimpanzees\is{chimpanzee}, gorillas\is{gorilla} and rhesus apes\is{rhesus ape} by only two
%% positions \citep{EPFLWKMP2002a}.
%
%In addition, 
进而,FoxP2也出现在其他身体问题上:它还负责肺、心、肠和大脑的许多区域\citep{MF2003a}。 \citet[\page
260--261]{MF2003a}指出,FoxP2也许不是直接对器官和器官的区域的发展有影响,而是规范了一串不同的基因。由此,FoxP2不能被叫做语言基因,它只是一个以复杂方式跟其他基因相互影响的基因。只不过,跟其它基因相比,它对我们的语言能力十分重要,但是,同样称FoxP2为语言基因是不正确的,就像没人会因为肌病而不能直立行走,就将遗传的肌肉功能失调叫做“行走基因”\citep[\page 58]{Bishop2002a}。 \citet[\page
  392]{Karmiloff-Smith98a}也有相似的观点:
  有一种基因缺陷导致有些人在十岁开始失去听力,并在三十岁时彻底聋了。这个基因缺陷导致耳朵内部用来听的毛发发生了变化。这样,人们也就不愿意说“听觉基因”这样的词了。
%Furthermore, FoxP2 also occurs in other body tissues: it is also responsible for the development of
%the lungs, the heart, the intestine and various regions of the brain \citep{MF2003a}.  \citet[\page
%  260--261]{MF2003a} point out that FoxP2 is probably not directly responsible for the development
%of organs or areas of organs but rather regulates a cascade of different genes. FoxP2 can therefore
%not be referred to as the language gene, it is just a gene that interacts with other genes in
%complex ways.  It is, among other things, important for our language ability, however, in the same
%way that it does not make sense to call FoxP2 a language gene, nobody would connect a hereditary
%muscle disorder with a `walking gene' just because this myopathy prevents upright walking
%\citep[\page 58]{Bishop2002a}.  A similar argument can be found in  \citet[\page
%  392]{Karmiloff-Smith98a}: 
%there is a genetic defect that leads some people to begin to lose their hearing from the age of ten
%and become completely deaf by age thirty. This genetic defect
%causes changes in the hairs inside the ear that one requires for hearing. In this case, one would
%also not want to talk about a `hearing gene'.

 \citet*[\page 190]{FHC2005a}也认同FoxP2不是语言知识的根源。对这个话题的概述,请参考 \citew{Bishop2002a}和 \citew[\S~6.4.2.2]{Dabrowska2004a},一般意义上的基因问题,请参考 \citew{FM2005a}。
% \citet*[\page 190]{FHC2005a} are also of the opinion that FoxP2 cannot be responsible for linguistic knowledge. For an overview of this topic,
%see  \citew{Bishop2002a} and  \citew[Section~6.4.2.2]{Dabrowska2004a} and for genetic questions in general, see  \citew{FM2005a}. 
\isc{FoxP2|)}\is{FoxP2|)}\isc{基因|)}\is{gene|)}

\section{刺激贫乏}
%\section{Poverty of the Stimulus}
\label{Abschnitt-PSA}

关于\isc{刺激贫乏论|(}\is{Poverty of the Stimulus|(}\isc{习得|(}\is{acquisition|(}语言知识的天赋机制的一个重要证据是所谓的刺激贫乏论(PSA)\citep[\page 34]{Chomsky80b-u}。在文献中可以找到不同的版本, \citet{PS2002a}对此进行了详细的讨论。在讨论这些变体之后,他们总结了论证的逻辑结构,如下所示(第18页):
%An\is{Poverty of the Stimulus|(}\is{acquisition|(} important argument for the innateness of the linguistic knowledge is the so"=called
%Poverty of the Stimulus Argument (PSA) \citep[\page 34]{Chomsky80b-u}. Different versions of it can be found in the literature and have been carefully discussed
%by  \citet{PS2002a}. After discussing these variants, they summarize the logical structure of the
%argument as follows (p.\,18):
\eal
\ex 人类儿童学习第一语言时,要么用数据驱动的学习方法,要么用内在知识支持的方法(假说的选言前提)
\ex 如果儿童学习第一语言是通过数据驱动的方法,那么他们就无法获得他们没有得到必要根据的任何知识(数据驱动学习的定义)
\ex 但是,儿童确实会学习他们之前没有(实证的前提)的关键性证据。
\ex 所以说,儿童不是通过数据驱动来学习第一语言的。( b和c的拒取式(modus tollens)) 
\ex 结论:儿童学习语言是通过天赋知识支持的学习过程。(a和d的析取结论)
%\ex Human children learn their first language either by data"=driven learning or by learning supported by innate knowledge (a disjunctive premise by assumption)
%\ex If children learn their first language by data"=driven learning, then they could not acquire anything for which they did not have the necessary evidence
%(the definition of data"=driven learning)
%\ex However, children do in fact learn things that they do not seem to have decisive evidence for (empirical prerequisite)
%\ex Therefore, children do not learn their first language by data"=driven learning. (\emph{modus tollens} of b and c)
%\ex Conclusion: children learn language through a learning process supported by innate knowledge. (disjunctive syllogism of a and d)
\zl
Pullum和Scholz随后讨论了作为天赋的语言知识的短语成分证据的四种现象。他们有英语复合词开头部分的复数\citep{Gordon86a}、英语助动词的顺序\citep{Kimball73b-u}、英语无指的one\citep{Baker78a-u} ,以及英语助动词的位置\citep[\page 29--33]{Chomsky71a-u}。在\ref{PSA-cases}分析这些问题之前,我将讨论一个PSA的变体,并将之作为短语结构语法的形式属性。
%Pullum and Scholz then discuss four phenomena that have been claimed to constitute evidence for there being innate linguistic knowledge.
%These are plurals as initial parts of compounds in English \citep{Gordon86a}, sequences of auxiliaries in English
%\citep{Kimball73b-u}, anaphoric \emph{one} in English \citep{Baker78a-u} and the position of auxiliaries in English \citep[\page 29--33]{Chomsky71a-u}.
%Before I turn to these cases in Section~\ref{PSA-cases}, I will discuss a variant of the PSA that refers to the formal properties of
%phrase structure grammars.

\subsection{Gold定理}
%\subsection{Gold's Theorem}
\label{Abschnitt-Golds-Theorem}

在形式语言理论中\isc{语言!形式语言}\is{language!formal},语言被看作是包括这门语言的所有表达式的集合。这种集合可以通过各种复杂的重写文法\isc{重写文法}\is{rewrite grammar}来获得。在第\ref{Kapitel-PSG}章,我们介绍了一种重写文法,即上下文无关文法。在上下文无关文法\isc{上下文无关文法}\is{context"=free grammar}中,在规则的左边总是有一个符号(所谓的非终结符),而在规则的右边有更多的符号。在右边还可以是符号(所谓的非终结符\isc{符号!非终结符}\is{symbol!non-terminal})或所描写语言的词/语素(所谓的终结符\isc{符号!终结符}\is{symbol!terminal})。语法中的词也叫做词汇(V)。部分形式语法的起始符,通常是S。在文献中,这点备受争议,因为不是所有的表达式都是句子(请看\citealp[\page 44]{Deppermann2006a})。但是,这样假设是没有必要的。我们可以用话语作为起始符,然后界定生成S、NP、VP的规则或区分出其他语句的任何其他条件。\footnote{%
在第\pageref{HPSG-Rootnode}页,我讨论了短语结构语法中属于S符号的描写。如果我们在这个描写中省略中心语特征的区分,那么就会得到所有完整句子的描述,即the man(男人)或 now(现在)。
}
%In theories of formal languages\is{language!formal}, a language is viewed as a set containing all
%the expressions belonging to a particular language. This kind of set can be captured using various
%complex rewrite grammars. A kind of rewrite grammar\is{rewrite grammar} -- so"=called context"=free 
%grammars\is{context"=free grammar} -- was presented in Chapter~\ref{Kapitel-PSG}.
%In context"=free grammars, there is always exactly one symbol on the left"=hand side of the rule
%(a so"=called non"=terminal symbol) and there can be more of these on the right"=hand side of the
%rule. On the right side there can be symbols (so"=called non"=terminal symbols\is{symbol!non-terminal}) or words/morphemes
%of the language in question (so"=called terminal symbols\is{symbol!terminal}). The words in a grammar are also referred to as vocabulary (V). Part of a formal grammar is
%a start symbol, which is usually S. In the literature, this has been criticized since not all expressions
%are sentences (see \citealp[\page 44]{Deppermann2006a}). It is, however, not necessary to assume
%this. It is possible to use Utterance as the start symbol and define rules that derive S, NP, VP or whatever else
%one wishes to class as an utterance from Utterance.\footnote{%
%	On page~\pageref{HPSG-Rootnode}, I discussed a description that corresponds to the
%	S symbol in phrase structure grammars. If one omits the specification of head features
%	in this description, then one gets a description of all complete	phrases, that is,
%	also \emph{the man} or \emph{now}.
%}

从起始符开始,我们可以在一个文法中应用短语结构规则,直到我们得到只包括词(终结符)的序列。我们能够生成的所有序列的集合属于该文法所允准的语言的表达式。这个集合是任意组合可以得到的所有词或语素的子集。包含所有可能序列的集合叫做V$^*$。
%Beginning with the start symbol, one can keep applying phrase structure rules in a grammar  
%until one arrives at sequences that only contain words (terminal symbols). The set of all sequences
%that one can generate are the expressions that belong to the language that is licensed by the grammar.
%This set is a subset of all sequences of words or morphemes that can be created by arbitrary 
%combination. The set that contains all possible sequences is referred to as V$^*$.   
  
\cite{Gold67a}证明了,在环境E下,只给出有限的语言输入,且没有额外的知识,是不可能解决识别特殊的语言类型下任意一种语言的问题的。Gold关心从给定的语言类型中对一种语言的识别。当在t$_n$时间内的某个点上,操某种语言的人可以认定语言L是当下的语言,而且没有改变这个看法的话,这门语言L就算被识别出来了。不过,这个时间点不是提前定好的,识别总要在某个时刻实现。Gold将之叫做“受限的识别”(identification in the limit\isc{受限的识别|(}\is{identification in the limit|(})。
环境是任意无限的句子序列\phonliste{ a$_1$, a$_2$, a$_3$,
\ldots },其中,语言中的每个句子都至少在这个序列中出现一次。为了说明识别问题甚至不能解决非常简单的语言类别的问题,Gold认为包括词表V中所有可能的词的序列的语言类别会有一个序列:让V是词表,而且x$_1$, x$_2$, x$_3$, \ldots{}是出自这个词表的词语序列。出自这个词表的所有字符串的集合是V$^*$。对于例(\mex{1})中的语言类型来说,它包括V中所有可能的成分序列和所应有的一个序列,我们可以证明人们是如何从文本中学会这些语言的。
%\cite{Gold67a} has shown that in an environment E, it is not possible to solve the identification
%problem for any language from particular languages classes, given a finite amount of linguistic input, without additional knowledge. Gold is concerned
%with the identification of a language from a given class of languages. A language L counts as identified if at a given point in time
%t$_n$, a learner can determine that L is the language in question and does not change this hypothesis.
%This point in time is not determined in advance, however, identification has to take place at some point.
%Gold calls this \emph{identification in the limit}\is{identification in the limit|(}.
%The environments are  arbitrary infinite sequences of sentences \phonliste{ a$_1$, a$_2$, a$_3$,
%\ldots }, whereby each sentence in the language must occur at least once in this sequence. In order
%to show that the identification problem cannot be solved for even very simple language classes, Gold
%considers the class of languages that contain all possible sequences of words from the vocabulary V expect
%for one sequence: let V be the vocabulary and x$_1$, x$_2$, x$_3$, \ldots{} the sequences of words from this vocabulary.
%The set of all strings from this vocabulary is V$^*$. For the class of languages in (\mex{1}), which consist of all possible
%sequences of elements in V with the exception of one sequence, it is possible to state a process of how one could
%learn these languages from a text.
\ea
L$_1$ = V$^* - x_1$, L$_2$ = V$^* - x_2$, L$_3$ = V$^* - x_3$, \ldots
\z

\noindent
在每次输入后,我们可以猜想语言是V$^* - \sigma$,其中$\sigma$表示按照字母顺序排列的第一个最短长度,但是还没有看到的序列。如果这个序列后来出现了,那么这个假说就被相应地修正了。按照这个方式,我们最终会得到正确的语言。
%After every input, one can guess that the language is V$^* - \sigma$, where $\sigma$ stands for the alphabetically first
%sequence with the shortest length that has not yet been seen. If the sequence in question occurs later, then this hypothesis
%is revised accordingly. In this way, one will eventually arrive at the correct language.

如果我们从必须选择的V$^*$语言的集合中进行扩展,那么我们的学习过程就不再起作用了,因为,如果V$^*$是目标语,那么猜想最终会得到错误的结果。如果有一个能够学会这种语言类型的程序,那么它就必须在一些输入后正确地识别出V$^*$。让我们假定,这个输入是x$_k$。学习的过程如何能够告诉我们在这点上,我们要找的语言不是为了$j \neq k$的V$^* - x_j$?如果x$_k$导致人们猜出了错误的语法V$^*$,那么随后的每个输入会跟正确结果和错误结果都兼容。因为我们只有正向的数据,没有输入允许我们在这些假说之间进行区分,并且给出我们找到的所寻找的语言的超集信息。Gold指出,没有一种形式语言理论(比如说,正则文法\isc{正则语言}\is{regular language}、上下文无关文法\isc{上下文无关文法}\is{context"=free
grammar}和上下文相关文法\isc{上下文相关文法}\is{context"=sensitive grammar})假定的语法类型可以在有一些例句的文本的输入后按照有限的步骤识别出来。这对包括所有有限语言和至少一种无限语言的所有类型来说都是正确的。如果正向证据和负向证据\isc{负向证据}\is{negative
evidence}都被用来学习,而不是文本的话,情况会有不同。
%If we expand the set of languages from which we have to choose by V$^*$, then our learning process will no longer work since, if
%V$^*$ is the target language, then the guessing will perpetually yield incorrect results.
%If there were a procedure capable of learning this language class, then it would have to correctly identify V$^*$ after a certain
%number of inputs. Let us assume that this input is x$_k$. How can the learning procedure tell us at this point that the language
%we are looking for is not V$^* - x_j$ for $j \neq k$? If x$_k$ causes one to guess the wrong grammar
%V$^*$, then every input that comes after that will be compatible
%with both the correct (V$^* - x_j$) and incorrect (V$^*$) result. Since we only have positive
%data, no input allows us to distinguish between either of the
%hypotheses and provide the information that we have found a superset of the language we are looking for.
%Gold has shown that none of the classes of grammars assumed in the theory of formal languages (for example, regular\is{regular language}, context"=free\is{context"=free
%grammar} and context"=sensitive\is{context"=sensitive grammar} languages) can be identified after a finite amount of steps given the input of a text with example utterances.
%This is true for all classes of languages that contain all finite languages and at least one infinite language. The situation is different if positive and negative data\is{negative
%evidence}
%are used for learning instead of text.

从Gold的结果中得出的结论是,对于语言习得来说,人们需要帮助他们从最开始就避免特殊假说的知识。 \citew{Pullum2003a}批评了将Gold的发现作为语言知识必须是天赋的证据。他列出了为了得到Gold的结果与自然语言习得相关的一些观点。然后,他证明这些中的每一条都不是没有争议的。
%The conclusion that has been drawn from Gold's results is that, for language acquisition, one requires knowledge that helps to avoid particular hypotheses from the start.
% \citew{Pullum2003a} criticizes the use of Gold's findings as evidence for the fact that linguistic knowledge must be innate. He lists a number of assumptions that have
%to be made in order for Gold's results to be relevant for the acquisition of natural languages. He then shows that each of these is not uncontroversial.

\begin{enumerate}
\item 自然语言可以属于可学习的文本的类型,这跟上文提到的上下文无关语法相反。
%\item Natural languages could belong to the class of text-learnable languages as opposed to the class of context"=free grammars mentioned above.

\item 学习者能知道哪些词语序列是不合语法的(请看第453--454页Gold的文章中相似的观点)。正如在那之后展示的,儿童确实有直接的负向证据\isc{负向证据}\is{negative evidence},而且也有非直接的负向证据(请看\ref{Abschnitt-negative-Evidenz})。
%\item Learners could have information about which sequences of words are not grammatical (see p.\,453--454 of Gold's essay for a similar
%conjecture). As has been shown since then, children do have direct negative evidence\is{negative evidence} and there is also indirect
%negative evidence (see Section~\ref{Abschnitt-negative-Evidenz}).

\item 学习者是否将他们自己限制到一个文法中是不清楚的。 \citet{Feldman72a}发展了一个学习程序,淘汰了某点上所有不正确的文法,而且是无限次正确的,但是并不总是需要选择一个正确的文法,然后坚持相应的假说。
  应用这个程序,我们就有可能学会所有的递归可枚举语言\isc{递归可枚举语言}\is{recursively enumerable language},也就是说,所有的语言都有一个生成语法。Pullum指出,即使Feldman的学习机制也能证明是过于具有限制性的。它需要学习者花费整整一生的时间去追求正确的文法,而且他们在这个过程中还会有不正确的、但是好一些的假说。
%\item It is not clear whether learners really restrict themselves to exactly one grammar.  \citet{Feldman72a} has developed a learning procedure that
%eliminates all incorrect grammars at some point and is infinitely many times correct but it does not have to always choose one correct grammar
%and stick to the corresponding hypothesis.
%Using this procedure, it is possible to learn all recursively enumerable languages\is{recursively enumerable language}, that is, all languages
%for which there is a generative grammar. Pullum notes that even Feldman's learning procedure could prove to be too restrictive. It could take an entire
%lifetime for a learner to reach the correct grammar and they could have incorrect yet increasingly better hypotheses along the way.

\item 学习者可以在改进的条件下工作。如果人们允许某种程度的容忍,那么习得就会更为简单,而且它还会有可能学会递归可枚举语言的类型\citep{Wharton74a}。
%\item Learners could work in terms of improvements. If one allows for a certain degree of tolerance, then acquisition is easier and it even becomes
%possible to learn the class of recursively enumerable languages  \citep{Wharton74a}.

\item 语言习得并不必然构成关于序列的特定集合的知识的习得,也就是说,生成语法的习得可以创造出这个集合。如果文法被看作是部分描写语言结构的约束的集合,而不必要是语言结构的唯一集合的话,情况就完全不同了(关于这点的更多内容,请看\ref{sec-modelle-theorien}和第\ref{Abschnitt-Generativ-Modelltheoretisch}章)。
%\item Language acquisition does not necessarily constitute the acquisition of knowledge about a particular set of sequences, that is, the acquisition
%of a generative grammar capable of creating this set. The situation is completely different if grammars are viewed as a set of constraints that
%partially describe linguistic structures, but not necessarily a unique set of linguistic structures (for more on this point, see
%      Section~\ref{sec-modelle-theorien} and Chapter~\ref{Abschnitt-Generativ-Modelltheoretisch}).
\end{enumerate}

\noindent
进而,Pullum指出,也有可能在有限步骤内使用正向输入的Gold的程序来学习上下文相关文法的类型,当规则的数量有一个上限$k$的时候,其中$k$是一个任意数。
有可能$k$特别大,以至于人类大脑的认知能力不能使用比其有更多规则的文法了。
因为一般认为自然语言可以根据上下文相关语法\isc{上下文相关文法}\is{context"=sensitive grammar}来描述,所以它可以显示出Gold意义上的自然语言的句法可以通过文本来学习(也请看\citealp[\page 195--196]{SP2002b})。
%Furthermore, Pullum notes that it is also possible to learn the class of context"=sensitive grammars
%with Gold's procedure with positive input only in a finite number of steps if there is an upper
%bound $k$ for the number of rules, where $k$ is an arbitrary number.
%It is possible to make $k$ so big that the cognitive abilities of the human brain would not be able to use a grammar with more rules than this.
%Since it is normally assumed that natural languages can be described by context"=sensitive grammars\is{context"=sensitive grammar}, it can therefore
%be shown that the syntax of natural languages in Gold's sense can be learned from texts (see also \citealp[\page 195--196]{SP2002b}).  

 \citet{Johnson2004a}补充说,在有关语言习得的讨论中还有一个重要方面被忽略了。Gold的识别问题跟先天论大讨论中起到重要作用的语言习得的问题是不同的。为了让区别明晰化,Johnson区分了(Gold意义上的)识别能力和语言习得意义上的先天主义。对于语种类型C的识别意味着必须有一个函项$f$,当目标语处于有限时间内时,对于每个环境$E$,$C$中的每一种语言$L$永远收敛到假说$L$。
% \citet{Johnson2004a} adds that there is another important point that has been overlooked in the discussion about language acquisition. Gold's problem
%of identifiability is different from the problem of language acquisition that has played an important role in the nativism debate.
%In order to make the difference clear, Johnson differentiates between identifiability (in the Goldian sense) and learnability in the sense of
%language acquisition. Identifiability for a language class C means that there must be a function $f$ that for each environment $E$ for each
%language $L$ in $C$ permanently converges on hypothesis $L$ as the target language in a finite amount of time.

Johnson对学习力(learnability\isc{学习力}\is{learnability})(第585页)做了如下的界定:
“自然语言的类型$C$是可以学习的,当且仅当,任何一个普通的人类儿童和$C$的任意一种语言$L$的几乎任何正常的语言环境下,儿童会在1岁到5岁间把$L$(或某种足够类似于$L$的语言)作为母语来习得。”Johnson补充道,这个定义跟心理语言学中的学习能力理论没有关系,而是在习得的现实概念上的一个暗示。
%Johnson proposes the following as the definition of \emph{learnability}\is{learnability} (p.\,585):
%\emph{A class $C$ of natural languages is learnable iff, given almost any normal human child and almost any
%normal linguistic environment for any language $L$ in $C$, the child will acquire $L$ (or something sufficiently similar to $L$) as a native language
%between the ages of one and five.} Johnson adds the caveat that this definition does not correspond to
%any theory of learnability in psycholinguistics, but rather it is a hint in the direction of a
%realistic conception of acquisition.

Johnson指出,在大部分对Gold理论的解读中,识别力和学习力被当作同一回事,而这在逻辑上是不正确的:这两个概念的主要区别在于两个量词的使用。对于属于类型$C$中的“一”种语言$L$的识别需要学习者在有限时间内在每个环境中收敛到$L$上。这个时间在不同的环境中的区别可以非常大。
这还不是时间的上限。
对于语言$L$,我们可以直接构造出环境的序列$E_1$, $E_2$, \ldots{},这样在环境$E_i$中的学习者不会在早于时间$t_i$前猜出$L$。与识别力不同,学习力是指在每个正常环境中一个时间点之后,“每个”正常的孩子都可以收敛到正确的语言上。这就意味着儿童是在特定的时间段之后习得语言的。
 Johnson引用了 \citet[\page 352]{Morgan89a}的话,他说,儿童在大约听到4,280,000个句子后学会他们的母语的。如果我们假设学习力的概念对于可能时间有一个有限的上限的话,那么很少的语言类型会在限制内被识别出。Johnson是这样说明的:让$C$是包括$L$和$L'$语言的类型,其中$L$和$L'$有一些共同的成分。我们有可能构造出这样的文本,其中第一个句子在$L$和$L'$中都有。
如果学习者将$L$当作是工作假说,那么继续这个属于$L'$的句子的文本,如果他将$L'$作为他的假说,则继续属于$L$的句子。每一种情况下,学者着都在$n$个步骤后得到一个错误的假说。这就意味着识别力不是语言习得的一个貌似正确的模型。
%Johnson notes that in most interpretations of Gold's theorem, identifiability and learnability are viewed as one and the same
%and shows that this is not logically correct: the main difference between the two depends on the use of two quantifiers.
%Identifiability of  \emph{one} language $L$ from a class $C$ requires that the learner converges on $L$ in \emph{every} environment
%after a finite amount of time. This time can differ greatly from environment to environment.
%There is not even an upper bound for the time in question.
%It is straightforward to construct a sequence of environments $E_1$, $E_2$, \ldots{} for $L$, so that a learner in the environment $E_i$ will not
%guess $L$ earlier than the time  $t_i$. Unlike identifiability, learnability means that there is a point in time after which in every
%normal environment, \emph{every} normal child has converged on the correct language. This means that children acquire their language
%after a particular time span.
% Johnson quotes  \citet[\page 352]{Morgan89a} claiming that children learn their native language after they have heard approximately
% 4,280,000 sentences. If we assume that the concept of learnability has a finite upper"=bound for available time, then very few language
% classes can be identified in the limit. Johnson has shown this as follows: let $C$ be a class of languages containing $L$ and $L'$, where
% $L$ and $L'$ have some elements in common. It is possible to construct a text such that the first $n$ sentences are contained both in
% $L$ and in $L'$.
%If the learner has $L$ as its working hypothesis  then continue the text with sentences from $L'$, if he has $L'$ as his hypothesis,
%then continue with sentences from $L$. In each case, the learner has entertained a false hypothesis after $n$ steps. This means that identifiability
%is not a plausible model for language acquisition.

除了识别能力不具备心理现实性的事实外,它跟学习能力也不相容\citep[\page 586]{Johnson2004a}。对于识别能力来说,只需要找到一个学习者(上面提到的功能$f$),但是,学习能力对(几乎)所有正常的孩子都适用。如果我们把所有的因素都保持不变,那么相比于学习力,更容易证明语言类型的识别力。
%Aside from the fact that identifiability is psychologically unrealistic, it is not compatible with learnability \citep[\page 586]{Johnson2004a}.
%For identifiability, only one learner has to be found (the function $f$ mentioned above), learnability, however, quantifies over
%(almost) all normal children. If one keeps all factors constant, then it is easier to show the identifiability of a language class rather than
%its learnability.
一方面,识别能力普遍地满足所有环境的要求,不管这些是不是看起来很奇怪或者它们能包括多少重复。另一方面,学习能力具有除了正常环境的(几乎)通用的条件。所以说,跟识别能力相比,学习能力指更少的环境,这样就导致有问题的文本作为输入以使得语言是不可学习的可能性更少。进而,学习能力是这样界定的,学习者不需要真正地学习$L$,而是学会某种相似的识别能力。所以说,学习能力不在识别能力之后,反之亦然。\isc{受限的识别|)}\is{identification in the limit|)}
%On the one hand, identifiability quantifies universally over all environments, regardless of whether these may seem odd or of how many repetitions these may contain.
%Learnability, on the other hand, has (almost) universal quantification exclusively over normal environments. Therefore, learnability refers to fewer environments
%than identifiability, such that there are less possibilities for problematic texts that could occur as an input and render a language unlearnable.
%Furthermore, learnability is defined in such a way that the learner does not have to learn $L$ exactly, but rather learn something sufficiently similar
%to $L$. With respect to this aspect, learnability is a weaker property of a language class than
%identifiability. Therefore, learnability does not follow from identifiability nor the reverse.\is{identification in the limit|)}

最后,Gold分析了没有考虑语义信息的句法知识的习得。但是,儿童在学习语言时,从上下文中获得了大量的信息\citep{TCCBM2005a}。 正如 \citet[\page 44]{Klein86a-u}指出的,如果把人放在屋子中,人没有学会任何东西,而是汉语\isc{现代汉语}\is{Mandarin Chinese}的句子在跟他们做游戏。语言是在社会和文化语境中习得的。
%Finally, Gold is dealing with the acquisition of syntactic knowledge without taking semantic knowledge into consideration.  
%However, children possess a vast amount of information from the context that they employ when acquiring a language \citep{TCCBM2005a}.
%As pointed out by  \citet[\page 44]{Klein86a-u}, humans do not learn anything if they are placed in a room and sentences in
%Mandarin Chinese\is{Mandarin Chinese} are played to them. Language is acquired in a social and cultural context.
 
总之,我们应该指出,天赋的语言知识的存在性不能从语言学习能力在数学上的发现推导出来。 
%In sum, one should note that the existence of innate linguistic knowledge cannot be derived from mathematical
%findings about the learnability of languages.  

\subsection{四个案例}
%\subsection{Four case studies}
\label{PSA-cases}

\mbox{} \citet{PS2002a}深入地探讨了刺激贫乏论的四个著名例子。我们将在下面讨论这些内容。Pullum和Scholz的文章收录在一本讨论性的文集中。由 \citet{SP2002b}发表的反对他们文章的观点的文章也收录在这本文集中。 \citet{Eisenberg92b}反驳了 \citet{Chomsky86}和德语文献中有关PoS论元的观点。
%\mbox{} \citet{PS2002a} have investigated four prominent instances of the Poverty of the Stimulus Argument in more detail.
%These will be discussed in what follows. Pullum and Scholz's article appeared in a discussion volume. Arguments against their
%article are addressed by  \citet{SP2002b} in the same volume. Further PoS arguments from  \citet{Chomsky86} and
%from literature in German have been disproved by  \citet{Eisenberg92b}.

\subsubsection{名名组合的复数}
%\subsubsection{Plurals in noun-noun compounding}

\mbox{} \citet{Gordon86a}\il{English|(}提出,英语中的复合词\isc{组构}\is{composition}只允许不规则的复数变化,即mice-eater,而不是\noword{rats-eater}。Gordon认为,以不规则的复数作为开头成分的复合词非常少见,以致于儿童不能仅从语言数据中学会这样的复合词。
%\mbox{} \citet{Gordon86a}\il{English|(} claims that compounds\is{composition} in English only allow
%irregular plurals in compounds, that is, \emph{mice-eater} but ostensibly not
%\noword{rats-eater}. Gordon claims that compounds with irregular plurals as first element are so rare that children could not have learned the fact that such
%compounds are possible purely from data.

在第25--26页,Pullum和Scholz讨论了英语中的数据,并且说明规则的复数形式确实只能在复合词的第一个成分中出现(chemicals-maker、forms-reader、generics-maker、securities-dealer、drinks trolley、rules committee、publications
  catalogue)。\footnote{%
也可以参考 \citew[\page 7]{Abney96a}摘自《华尔街日报》中的例子。
}
这说明,所谓的没有从语言事实中学习实际上在语言学上是不充分的理由,所以不能用来解释习得。\il{English|)}
%On pages 25--26, Pullum and Scholz discuss data from English that show that regular plurals can
%indeed occur as the first element of a compound (\emph{chemicals-maker}, \emph{forms-reader}, \emph{generics-maker},
%\emph{securities-dealer}, \emph{drinks trolley}, \emph{rules committee}, \emph{publications
%  catalogue}).\footnote{%
%  Also, see  \citew[\page 7]{Abney96a} for examples from the Wall Street Journal.
%}
%This shows that what could have allegedly not been learned from data is in fact not linguistically adequate and one therefore does not have to explain 
%its acquisition.\il{English|)}

\subsubsection{助动词的位置}
%\subsubsection{Position of auxiliaries}

第二项\il{English|(}研究分析情态动词\isc{动词!情态动词}\is{verb!modal}和助动词\isc{动词!助动词}\is{verb!auxiliary}的位置。 \citet[\page 73--75]{Kimball73b-u}讨论了例(\mex{1})中的数据和(\mex{2})中的规则,该规则类似于 \citet[\page 39]{Chomsky57a}提出的一种规则,并且用来描述下面的语言事实:
%The\il{English|(} second study deals with the position of modal\is{verb!modal} and auxiliary verbs\is{verb!auxiliary}.  \citet[\page
%  73--75]{Kimball73b-u} discusses the data in (\mex{1}) and the rule in (\mex{2}) that is similar to one of the rules suggested by
%  \citet[\page 39]{Chomsky57a} and is designed to capture the following data: 
\eal
\label{Aux-Beispiele}
\ex 
\gll It rains.\\
     \expl{} 下雨\\
\mytrans{下雨了。}
\ex\label{It-may-rain} 
\gll It may rain.\\
     \expl{} 会 下雨\\
\mytrans{会下雨。}
\ex 
\gll It may have rained.\\
     \expl{} 会 \textsc{aux} 下雨\\
\mytrans{可能已经下雨了。}
\ex 
\gll It may be raining.\\
     \expl{} 会 \textsc{aux} 下雨\\
\mytrans{可能正在下雨。}
\ex 
\gll It has rained.\\
     \expl{} \textsc{aux} 下雨\\
\mytrans{已经下雨了。}
\ex 
\gll It has been raining.\\
     \expl{} \textsc{aux} \textsc{aux} 下雨\\
\mytrans{一直在下雨。}
\ex 
\gll It is raining.\\
     \expl{} \textsc{aux} 下雨\\
\mytrans{正在下雨。}
\ex\label{It-may-have-been-raining} 
\gll It may have been raining.\\
     \expl{} 会 \textsc{aux} \textsc{aux} 下雨\\
\mytrans{有可能一直在下雨。}
\zl
\ea
\label{Regel-Aux}
Aux $\to$ T(M)(have+en)(be+ing)
\z
T表示时态,M代表情态动词,en代表分词语速(been/seen/\ldots{}中的\suffix{en}和rained中的\suffix{ed})。这里的括弧表明了表达式的可选择性。Kimball指出,如果(\mex{-1}h)是合乎语法的,就只能构造出这个规则。如果不是这样,那么我们就必须识别出规则中的材料,这样才能覆盖(M)(have+en)、(M)(be+ing)和(have+en)(be+ing)这三种情况。
Kimball假定,儿童掌握了复杂的规则,因为他们知道诸如(\mex{-1}h)这样的句子是合乎语法的,以及他们知道情态动词和助动词必须出现的位次。Kimball认为,儿童对于(\mex{-1}h)的语序没有正向的证据,并由此得出结论说,关于(\mex{0})的规则的知识必须是天赋的。
%T stands for tense, M for a modal verb and \emph{en} stands for the participle morpheme (\suffix{en} in
%\emph{been}/\emph{seen}/\ldots{} and \suffix{ed} in \emph{rained}). The brackets here indicate the optionality
%of the expressions. Kimball notes that it is only possible to formulate this rule if (\mex{-1}h) is well"=formed.
%If this were not the case, then one would have to reorganize the material in rules such that the three cases
%(M)(have+en), (M)(be+ing) and (have+en)(be+ing) would be covered.
%Kimball assumes that children master the complex rule since they know that sentences such as
%(\mex{-1}h) are well-formed and since
%they know the order in which modal and auxiliary verbs must occur. Kimball assumes that children do not have positive
%evidence for the order in (\mex{-1}h) and concludes from this that the knowledge about the rule in (\mex{0}) must
%be innate.

Pullum和Scholz指出了这个刺激贫乏论的两个问题:首先,他们找到了上百个例子,其中有些是来源于儿童故事的,所以Kimball的观点是,诸如(\mex{-1}h)的“非常少”的句子应该被纳入研究范围。对于PSA来说,我们至少应该区分多少情况是被允许的,如果我们仍然希望说明没有什么能够从中学到的话\citep[\page
  29]{PS2002a}。
%Pullum and Scholz note two problems with this Poverty of the Stimulus Argument: first, they have found hundreds of examples, among them some from
%children's stories, so that the Kimball's claim that sentences such as (\mex{-1}h) are ``vanishingly rare'' should
%be called into question. For PSA arguments, one should at least specify how many occurrences there are allowed to be if one still wants to claim
%that nothing can be learned from them \citep[\page
%  29]{PS2002a}. 

  第二个问题是,认为(\mex{-1}h)的规则在我们的语言知识中发挥了作用是讲不通的。语言事实方面的发现表明这个规则在描写上是不充分的。如果(\ref{Regel-Aux})中的规则在描写上不充分,那么它就不具备解释上的充分性,也就不能解释它是如何获得的。
%The second problem is that it does not make sense to assume that the rule in  (\mex{-1}h) plays a role in our linguistic knowledge.
%Empirical findings have shown that this rule is not descriptively adequate. If the rule in (\ref{Regel-Aux}) is not descriptively
%adequate, then it cannot achieve explanatory adequacy and therefore, one no longer has to explain how it can be acquired.

除了(\ref{Regel-Aux})的规则之外,这里所有的理论都认为助动词或情态动词嵌套了一个短语,即没有包括所有助动词和情态动词的Aux节点,但是有像下面这样的(\ref{It-may-have-been-raining})的结构:
%Instead of a rule such as (\ref{Regel-Aux}), all theories discussed here currently assume that auxiliary or modal verbs embed a phrase, that is,
%one does not have an Aux node containing all auxiliary and modal verbs, but rather a structure for (\ref{It-may-have-been-raining}) that looks
%as follows:
\ea
\gll It [may [have [been raining]]].\\
     \expl{} \spacebr{}会 \spacebr\textsc{aux} \spacebr\textsc{aux} 下雨\\
\mytrans{有可能一直在下雨。}
\z
这里,助动词和情态动词总是选择嵌套的短语。现在,习得问题看起来完全不同了:说话者必须学会动词投射中选择助动词或情态动词的中心语动词的形式。如果这个信息学会了,那么嵌套的动词性投射有多复杂就是无关的了:may可以跟非定式词汇动词(\ref{It-may-rain})或非定式助动词(\ref{Aux-Beispiele}c、d)相组合。\il{English|)}
%Here, the auxiliary or modal verb always selects the embedded phrase. The acquisition problem now looks completely different: a speaker has to learn the form
%of the head verb in the verbal projection selected by the auxiliary or modal verb. If this information has been learned, then it is irrelevant
%how complex the embedded verbal projections are: \emph{may} can be combined with a non"=finite lexical verb (\ref{It-may-rain}) or a non"=finite
%auxiliary (\ref{Aux-Beispiele}c,d).\il{English|)}

\subsubsection{one的指称}
%\subsubsection{Reference of \emph{one}}

Pullum和Scholz调查的第三个方面\il{English|(}是英语的代词one。
 \citet[\page 413--425, 327--340]{Baker78a-u}认为,儿童学不会one可以指代比单个词大的成分,如例(\mex{1})。
%The\il{English|(} third case study investigated by Pullum and Scholz deals with the pronoun \emph{one} in English.
% \citet[\page 413--425, 327--340]{Baker78a-u} claims that children cannot learn that \emph{one} can refer to constituents
%larger than a single word as in (\mex{1}).
\eal
\ex 
\gll I would like to tell you another funny story, but I've already told you the only \emph{one} I
know.\\
我 要 喜欢 \textsc{inf} 告诉 你 另一个 有趣的 故事 但是 我.\textsc{aux} 已经 告诉 你 \textsc{det} 唯一 \textsc{pron} 我 知道\\
\mytrans{我想给您讲另一个有趣的故事,但是我已经给你讲了我唯一知道的一个故事。}
\ex 
\gll The old man from France was more erudite than the young \emph{one}.\\
\textsc{det} 老 人 从 法国 \textsc{aux} 更加 博学 比 \textsc{det} 年轻 \textsc{pron}\\
\mytrans{从法国来的老人比年轻人更加博学。}
\zl
Baker(第416--417页)认为,one决不能指代NPs内部的单个名词,并且以(\mex{1})中的例句作为证据:
%Baker (\page 416--417) claims that \emph{one} can never refer to single nouns inside of NPs and supports this
%with examples such as (\mex{1}):
\ea[*]{
\gll The student of chemistry was more thoroughly prepared than the one of physics.\\
\textsc{det} 学生 \textsc{prep} 化学 \textsc{cop} 更加 全看 准备 比 \textsc{det} \textsc{pron} \textsc{prep} 物理学\\
\mytrans{化学专业的同学比物理学专业的同学准备更为充分。}
}
\z
按照Baker的观点,学习者需要负向数据\isc{负向证据}\is{negative evidence}来获得不合乎语法的知识。根据他的推论,由于学习者从没有接触过负向证据,他们就不可能学会相关的知识,并且由此必须是已经掌握它了。
%According to Baker, learners would require negative data\is{negative evidence} in order to acquire this knowledge about ungrammaticality.
%Since learners -- following his argumentation -- never have access to negative evidence, they cannot possibly have learned the relevant
%knowledge and must therefore already possess it.

 \citet[\page 33]{PS2002a}指出,有带有相同结构的可以接受的例子,如(\mex{0})中的例子所示:
% \citet[\page 33]{PS2002a} point out that there are acceptable examples with the same structure as
%the examples in (\mex{0}): 
\eal
\ex 
\gll I'd rather teach linguistics to a student of mathematics than to
one of any discipline in the humanities.\\
我.\textsc{aux} 更 教 语言学 \textsc{inf} 一 学生 \textsc{prep} 数学 比 \textsc{inf} \textsc{pron} \textsc{prep}  任何 专业 \textsc{prep} \textsc{det} 人文学科\\
\mytrans{我更愿意给数学专业的学生教语言学,而不是人文学科中的任何一个专业的学生。}
\ex 
\gll An advocate of Linux got into a heated discussion with one of Windows NT and the rest of the evening was nerd talk.\\
一 支持者 \textsc{prep} Linux 卷 进 一 热烈的 讨论 \textsc{prep} \textsc{pron} \textsc{prep} Windows NT 和 \textsc{det} 剩余的 \textsc{prep} \textsc{det} 晚上 \textsc{cop} 书呆子 谈话\\
\mytrans{一位Linux系统的支持者被卷进了一场跟Windows NT操作系统的支持者的热烈的讨论之中,晚上的剩余时间就成了书呆子间的讨论。}
\zl
这意味着,关于(\mex{-1})中结构的合格性,没有什么是要学习的。
而且,要获得one可以指代更大的成分这个事实的数据并不像Baker(第416页)声称的那样毫无希望:有例子表明,只有在one指代一个更大的字符串时是可以解释的。Pullum和Scholz从各种语料中寻找例子。他们也从CHILDES语料库\isc{CHILDES语料库}\is{CHILDES}中找到了例子,CHILDES语料库是一个包括了儿童的交流语言的语料库\citep{MacWhinny95a-u}。下面的例子选自一档日间电视节目:
%This means that there is nothing to learn with regard to the well"=formedness of the structure in (\mex{-1}).
%Furthermore, the available data for acquiring the fact that \emph{one} can refer to larger constituents is not as hopeless
%as Baker (p.\,416) claims: there are examples that only allow an interpretation where \emph{one} refers to a larger
%string of words. Pullum and Scholz offer examples from various corpora. They also provide examples from the CHILDES corpus\is{CHILDES},
%a corpus that contains communication with children \citep{MacWhinny95a-u}. The following example is from a daytime TV show:
\eanoraggedright
\begin{tabular}[t]{@{}l@{~}p{10.9cm}}
A: & ``Do you think you will ever remarry again? I don't.''\\
   & “你认为你还会再婚吗?我不会。”\\
B: & ``Maybe I will, someday. But he'd have to be somebody very special. Sensitive and supportive, giving. Hey, wait a minute, where
   do they make guys like this?''\\
   & “也许有一天我会。但是他必须是非常特别的人。既感性又支持我,还乐于奉献。哎,稍等一下,上哪儿去找这样的人呢?”\\
A: & \begin{minipage}[t]{10.9cm}
\gll ``I don't know. I've never seen one up close.''\\
     \hspaceThis{``}我 \textsc{aux.neg} 知道 我.\textsc{aux} 从未 看见 \textsc{pron} 上 近\\
\mytrans{我不知道。我身边从没见过这样的人。}
\end{minipage}
\end{tabular}
\z
显然,这里的one不能指代guys,因为A已经看到了guys。
相反,它指代guys like this,即感性又愿意提供帮助的人。
%Here, it is clear that \emph{one}  cannot refer to \emph{guys} since A has certainly already seen \emph{guys}.
%Instead, it refers to \emph{guys like this}, that is, men who are sensitive and supportive.   

再次,有问题出现了。听者需要听多少个例子才能算作是PSA理论的支持者眼中认可的例子呢。\il{English|)}
%Once again, the question arises here as to how many instances a learner has to hear for it to count as evidence in the eyes
%of proponents of the PSA.\il{English|)}

\subsubsection{极性问句中助动词的位置}
%\subsubsection{Position of auxiliaries in polar questions}
\label{Abschnitt-Hilfsverbumstellung}

Pullum\isc{助动词倒装|(}\is{auxiliary inversion|(}\il{English|(}和Scholz提出的PoS的第四个问题源自Chomsky,它是关于英语助动词在极性问句中的位置的。正如在第\pageref{Seite-GB-Entscheidungsfragen-Englisch}页所展现的,在\gbtc 中,极性问句是由助动词从句中I位置移到开头位置C而生成的。在转换语法的早期版本中,确切的分析是不同的,但是主要的观点还是最高阶的助动词被移到了从句的开头。
  \citet[\page 29--33]{Chomsky71a-u}讨论了(\mex{1})中的句子,并且认为,儿童知道他们必须移动最高阶的助动词,即使没有正向的证据。\footnote{%
 助动词变换的例子也用在最近的PoS观点中,如 \citew*{BPYC2011a}和 \citew[\page 39]{Chomsky2013a}。 \citet{Bod2009a}的工作并没有得到讨论。更多有关Bod的方法,请看\ref{Abschnitt-UDOP}。
 }比如说,如果他们乐于提出这样的观点,人们简单地将第一个助动词放在句子的开头的话,那么这个假说就会针对(\mex{1}b)而得到正确的结论(\mex{1}a),而不是针对(\mex{1}c)的,因为极性问句应该是(\mex{1}d),而不是(\mex{1}e)。
%The\is{auxiliary inversion|(}\il{English|(} fourth PoS argument discussed by Pullum and Scholz comes from Chomsky and pertains to the
%position of the auxiliary in polar interrogatives in English. As shown on page~\pageref{Seite-GB-Entscheidungsfragen-Englisch},
%it was assumed in \gbt that a polar question is derived by movement of the auxiliary from the I position to the initial position C of
%the sentence. In early versions of Transformational Grammar, the exact analyses were different, but the main point was that the
%highest auxiliary is moved to the beginning of the clause. 
%  \citet[\page 29--33]{Chomsky71a-u}
%discusses the sentences in (\mex{1}) and claims that children know that they have to move the highest auxiliary verb even without having
%positive evidence for this.\footnote{%
%	Examples with auxiliary inversion are used in more recent PoS arguments too, for example in 
%   \citew*{BPYC2011a} and  \citew[\page 39]{Chomsky2013a}. Work by  \citet{Bod2009a} is not discussed by the authors.
%  For more on Bod's approach, see Section~\ref{Abschnitt-UDOP}.% 
%} If, for example, they entertained the hypothesis that one simply places the first auxiliary at the beginning of the sentence, then
%this hypothesis would deliver the correct result (\mex{1}b) for (\mex{1}a), but not for (\mex{1}c) since the polar question should
%be (\mex{1}d) and not (\mex{1}e).
\eal
\ex[]{
\gll The dog in the corner is hungry.\\
\textsc{det} 狗 在 \textsc{det} 角落 \textsc{cop} 饿\\
\mytrans{在角落里的狗饿了。}
}
\ex[]{
\gll Is the dog in the corner hungry?\\
\textsc{cop} \textsc{det} 狗 在 \textsc{det} 角落 饿\\
\mytrans{在角落里的狗饿了吗?}
}
\ex[]{
\gll The dog that is in the corner is hungry.\\
\textsc{det} 狗 \textsc{rel} \textsc{cop} 在 \textsc{det} 角落 \textsc{cop} 饿\\
\mytrans{在角落里的那只狗饿了。}
}
\ex[]{\label{Hilfsverbinversion-mit-RS}
\gll Is the dog that is in the corner hungry?\\
\textsc{cop} \textsc{det} 狗 \textsc{rel} \textsc{cop} 在 \textsc{det} 角落 饿\\
\mytrans{在角落里的那只狗饿了吗?}
}
\ex[*]{\label{Hilfsverbinversion-ungrammatisch}
\gll Is the dog that in the corner is hungry?\\
\textsc{cop} \textsc{det} 狗 \textsc{rel} 在 \textsc{det} 角落 \textsc{cop} 饿\\
}
\zl
Chomsky表示,儿童没有任何证据说明这样的事实,人们简单地把第一个助动词线性地提前是错的,这就是为什么他们可以按照数据驱动的学习过程来验证这个观点。他甚至进一步提出,操英语者只是很少,甚至是从未生成过(\ref{Hilfsverbinversion-mit-RS})这样的例子( \citew[\page 114--115]{Piattelli-Palmarini80a-u})。
在语料库数据和貌似真实的构造出的例子的帮助下, \citet{Pullum96a}证明了这个观点明显是错误的。
  \citet{Pullum96a}在《华尔街日报》中找到了例子,并且 \citet{PS2002a}更为细致地讨论了相关的例子,并且加上CHILDES语料库\isc{CHILDES语料库}\is{CHILDES}中的例子,一同说明成年人不能只造出相关类型的句子,而是他们也出现在儿童的输入中。\footnote{%
 关于该点的更多信息,请参考 \citew[\page 223]{Sampson89a}。Sampson引用了英语学校研究的William Blake的诗的一部分,以及一本儿童百科全书。这些例子当然在助动词位置的习得中没有起到重要的作用,因为这个语序是在3岁2个月时学会的,也就是说,在孩子们达到上学的年龄时早就已经学会了。
 }
CHILDES\isc{CHILDES语料库}\is{CHILDES}语料库中的例子反驳了第一助动词需要前置的假说,如(\mex{1})所示:\footnote{%
请参考 \citew{LE2001a}。语言习得方面的研究者认同,在跟孩子们的交谈中,这类句子出现的频率实际上是非常低的。请参考 \citew[\page 223]{ARP2008a}。
 }
%Chomsky claims that children do not have any evidence for the fact that the hypothesis that one
%simply fronts the linearly first auxiliary is wrong, which is why they could pursue this hypothesis in a data"=driven learning process. He even goes so
%far as to claim that speakers of English only rarely or even never produce examples such as (\ref{Hilfsverbinversion-mit-RS})
%(Chomsky in
% \citew[\page 114--115]{Piattelli-Palmarini80a-u}). 
%With the help of corpus data and plausibly constructed examples,  \citet{Pullum96a} has shown that this claim is clearly wrong.
%  \citet{Pullum96a} provides examples from the Wall Street Journal and  \citet{PS2002a} discuss the relevant examples in more detail
% and add to them with examples from the CHILDES corpus\is{CHILDES} showing  that adult speakers cannot only produce the relevant
% kinds of sentences, but also that these occur in the child's input.\footnote{%
%For more on this point, see  \citew[\page 223]{Sampson89a}. Sampson cites part of a poem by William Blake, that is studied in English schools, as well as
%a children's encyclopedia. These examples surely do not play a role in acquisition of auxiliary position since this order is learned at the age of
%3;2, that is, it has already been learned by the time children reach school age.%
%}
%Examples from CHILDES\is{CHILDES} that disprove the hypothesis that the first auxiliary has to be fronted are given in (\mex{1}):\footnote{%
%  See  \citew{LE2001a}. Researchers on language acquisition agree that the frequency of this kind of examples in communication with children is in fact
%  very low. See  \citew[\page 223]{ARP2008a}.
%}
\eal
\label{aux-fronting-childes}
\ex 
\gll Is the ball you were speaking of in the box with the bowling pin?\\
\textsc{cop} \textsc{det} 球 你 \textsc{aux} 谈 \textsc{prep} 在 \textsc{det} 盒子 \textsc{prep} \textsc{det} 保龄球 瓶\\
\mytrans{你说的盒子里的那个球是保龄球瓶吗?}
\ex 
\gll Where's          this little boy who's                     full of smiles?\\
     哪儿.\textsc{cop} 这   小     男孩 \textsc{rel}.\textsc{cop} 充满 \textsc{prep} 笑容\\
\mytrans{这个满脸笑容的小男孩哪儿去了?}
\ex\label{aux-fronting-Adjunktsatz} 
\gll While you're sleeping, shall I make the breakfast?\\
当 你.\textsc{aux} 睡觉 可以 我 做 \textsc{det} 早饭\\
\mytrans{在你睡觉的时候,我能做早饭吗?}
\zl

\noindent
Pullum和Scholz指出,诸如(\mex{1}b)的wh"=问句也是相关的,如果我们认为他们是从极性问句派生而来的话(请看本书第\pageref{Seite-GB-Entscheidungsfragen-Englisch}页),而且如果我们希望展示儿童是如何学会基于结构的假说的话。这可以通过(\mex{1})中的例子来解释:(\mex{1}a)被派生出的基础形式是(\mex{1}b)。如果我们要把(\mex{1}b)中的第一个助动词前置,那么我们就会得到(\mex{1}c)。
%Pullum and Scholz point out that \emph{wh}"=questions such as (\mex{0}b) are also relevant if one assumes that these are derived from
%polar questions (see page~\pageref{Seite-GB-Entscheidungsfragen-Englisch} in this book) and if one wishes to show how the child can
%learn the structure"=dependent hypothesis. This can be explained with the examples in (\mex{1}): the
%base form from which (\mex{1}a) is derived is (\mex{1}b). If we were to front the first auxiliary in (\mex{1}b), we would produce (\mex{1}c).
\eal
\ex[]{
\gll Where's the application Mark promised to fill out?\\
哪儿.\textsc{cop} \textsc{det} 申请表 Mark 承诺 \textsc{inf} 填 \textsc{adv}\\
\mytrans{Mark承诺要填的表在哪儿呢?'}footnote{%
  译自CHILDES语料库中的一档电视节目\isc{CHILDES语料库}\is{CHILDES}。
}
%Where's the application Mark promised to fill out?\footnote{%
%  From the transcription of a TV program in the CHILDES corpus\is{CHILDES}.
%}
}
\ex[]{
\gll the          application Mark [\sub{\textsc{aux}} PAST] promised to            fill out          [\sub{\textsc{aux}} is] there\\
     \textsc{det} 申请表       Mark {}                  过去]  承诺      \textsc{inf} 填   \textsc{adv} {}                  \textsc{cop} 那儿\\
\mytrans{Mark过去承诺要填的表在那儿}
}
\ex[*]{
\gll Where did the application Mark promised to fill out is?\\
哪儿 \textsc{aux} \textsc{det} 申请表 Mark 承诺 \textsc{inf} 填 \textsc{adv} \textsc{cop}\\
}
\zl
但是,(\mex{0}c)是不正确的证据也可以在跟孩子有关的语言中找到。
Pullum和Scholz举出了(\mex{1})中的例句:\footnote{%
这些句子选自NINA05。CHA in DATABASE/ENG/SUPPES/。
}
%Evidence for the fact that (\mex{0}c) is not correct can, however, also be found in language addressed to children.
%Pullum and Scholz provide the examples in (\mex{1}):\footnote{%
%	These sentences are taken from NINA05.CHA in DATABASE/ENG/SUPPES/.
%}
\eal
\label{wh-Fragen-Hilfsverbinversion}
\ex 
\gll Where's the little blue crib that was in the house before?\\
哪儿.\textsc{cop} \textsc{det} 小 蓝色 螃蟹 \textsc{rel} \textsc{cop} \textsc{prep} \textsc{det} 房子 以前\\
\mytrans{以前房子里的小蓝螃蟹在哪儿?}
\ex 
\gll Where's the other dolly that was in here?\\
哪儿.\textsc{cop} \textsc{det} 另一个 娃娃  \textsc{rel} \textsc{cop} \textsc{prep} 这儿\\
\mytrans{原来在这儿的另一个娃娃在哪儿?}
\ex 
\gll Where's the other doll that goes in there?\\
哪儿.\textsc{cop} \textsc{det} 另一个 娃娃 \textsc{rel} 走 在 那儿 \\
\mytrans{放在那儿的另一个娃娃在哪儿?}
\zl
这些问句具有Where's NP?这样的形式,其中NP包含一个关系从句。
%These questions have the form \emph{Where's NP?}, where NP contains a relative clause.

在(\ref{aux-fronting-Adjunktsatz})中,有另一个从句位于实际的疑问句的前面,一个包括助动词的状语从句。这个句子就可以证明线性上第一位助动词必须前置的假说是错误的\citep[\page 223]{Sampson89a}。
%In (\ref{aux-fronting-Adjunktsatz}), there is another clause preceding the actual interrogative, an adjunct clause containing an
%auxiliary as well. This sentence therefore provides evidence for falsehood of the hypothesis that the linearly first auxiliary must be fronted
%\citep[\page 223]{Sampson89a}. 

总之,在儿童的语言输入中,有许多可验证的句子类型允许他们在两种假说中选择。再一次,问题来了,有多少证据可以被认为是足够的呢?
%In total, there are a number of attested sentence types in the input of children that would allow them to choose between the two
%hypotheses. Once again, the question arises as to how much evidence should be viewed as sufficient.

 \citet{LU2002a}和 \citet{LY2002a}评论了Pullum和Scholz的文章。Lasnik和Uriagereka论证道,习得问题要比Pullum和Scholz提出的大得多,因为当一个学习者不知道他要习得语言的任何知识时,这个学习者不能只有我们已经讨论过的(\mex{1})中的假说,还应该有(\mex{2})中的其他假说:
%Pullum und Scholz's article has been criticized by  \citet{LU2002a} and  \citet{LY2002a}. Lasnik and Uriagereka argue that the acquisition
%problem is much bigger than presented by Pullum and Scholz since a learner without any knowledge about the language he was going to acquire
%could not just have the hypothesis in (\mex{1}) that were discussed already but also the additional hypotheses in (\mex{2}):
\eal
\label{Hilfsverbhypothesen}
\ex 将第一个助动词放在从句的开头。
%\ex Place the first auxiliary at the front of the clause.
\ex\label{Hypothese-I-C} 
将第一个助动词放在从句的开头的matrix-Infl上。
%\ex\label{Hypothese-I-C} 
%Place the first auxiliary in matrix-Infl at the front of the clause.
\zl
\eal
\ex 将任意一个助动词放在从句的开头。
\ex 将任意一个定式的助动词放在从句的开头。
%\ex Place any auxiliary at the front of the clause.
%\ex Place any finite auxiliary at the front of the clause.
\zl
(\mex{0})中的所有假说都可以通过(\mex{1})中的句子得到许可:
%Both hypotheses in (\mex{0}) would be permitted by the sentences in (\mex{1}):
\eal
\ex[]{
\gll Is the dog in the corner hungry?\\
\textsc{cop} \textsc{det} 狗 \textsc{prep} \textsc{det} 角落 饿\\
\mytrans{在角落里的狗饿吗?}
}
\ex[]{
\gll Is the dog that is in the corner hungry?\\
\textsc{cop} \textsc{det} 狗 \textsc{rel} \textsc{cop} \textsc{prep} \textsc{det} 角落 饿\\
\mytrans{在角落里的那只狗饿吗?}
}
\zl
但是,他们也可以允准(\mex{1})中的句子:
%They would, however, also allow sentences such as  (\mex{1}):
\ea[*]{
\gll Is the dog that in the corner is hungry?\\
\textsc{cop} \textsc{det} 狗 \textsc{rel} \textsc{prep} \textsc{det} 角落 \textsc{cop} 饿\\
}
\z
必须要指出的问题是为什么所有允准(\mex{0})的假说应该被丢弃,这是因为学习者在他们的自然语言的输入中并没有关于(\mex{0})是不可能的任何信息。他们缺乏负向证据。\isc{负向证据}\is{negative evidence}如果(\mex{-1}b)跟正向证据一起出现,那么这就无论如何暗示了(\ref{Hypothese-I-C})中的假说必然是正确的了。Lasnik和Uriagereka提出了跟(\mex{-1}b)也相容的假说,如下所示:
%The question that must now be addressed is why all hypotheses that allow (\mex{0}) should be discarded since the learners do
%not have any information in their natural"=linguistic input about the fact that (\mex{0}) is not possible. They are lacking
%negative evidence.\is{negative evidence} If (\mex{-1}b) is present as positive evidence, then this by no means implies
%that the hypothesis in (\ref{Hypothese-I-C}) has to be the correct one. Lasnik and Uriagereka present the following hypotheses
%that would also be compatible with (\mex{-1}b):
\eal
\ex 将第一个助动词放在首位(随后会有声调上的变化)。
\ex 将第一个助动词放在首位(随后是第一个完整的成分)。
\ex 将第一个助动词放在首位(随后是第一个剖析的语义单位)。
%\ex Place the first auxiliary in initial position (that follows a change in intonation).
%\ex Place the first auxiliary in initial position (that follows the first complete constituent).
%\ex Place the first auxiliary in initial position (that follows the first parsed semantic unit).
\zl
这些假说不能说明像(\mex{1})这样包括连词的句子:
%These hypotheses do not hold for sentences such as (\mex{1}) that contain a conjunction:
\ea[]{
\label{Beispiel-Hilfsverbvoranstellung-Koordination}
\gll Will those who are coming and those who are not coming raise their hands?\\
将 那些 \textsc{rel} \textsc{aux} 来 和 那些 \textsc{rel} \textsc{aux} 不 来 举起 他们的 手\\
\mytrans{来的和不来的能举起手吗?}
}
\z
(\mex{-1})中的假说也可以允准诸如(\mex{1})的句子:
%The hypotheses in (\mex{-1}) would also allow for sentences such as (\mex{1}):
\ea[*]{
\gll Are those who are coming and those who not coming will raise their hands?\\
\textsc{aux} 那些 \textsc{rel} \textsc{aux} 来 和 那些 \textsc{rel} 将 不 来 举起 他们的 手\\
}
\z
听者听到(\mex{-1})这样的句子时会反对假说(\mex{-2}),并由此排除(\mex{0})。但是,仍有可能想到一个跟之前讨论的所有数据相同的一个相似的看似正确的假说。
%Speakers hearing sentences such as (\mex{-1}) can reject the hypotheses (\mex{-2}) and thereby rule out (\mex{0}), however, it is still possible
%to think of analogous implausible hypotheses that are compatible with all data previously discussed.

 \citet{LY2002a}接受了Pullum和Scholz的挑战,并明确表示要习得一个特定的现象,人们需要多少次事件。
他们这样写道:
% \citet{LY2002a} take up the challenge of Pullum and Scholz and explicitly state how many occurrences one needs to acquire a particular phenomenon.
%They write the following:

\begin{quotation}
   假设我们有两个不同的习得问题,P$_1$和P$_2$,其中每个问题都包括一个二元决策。对于P$_1$来说,让F$_1$是可以解决是P$_1$还是其他的数据的频率,然后F$_2$是P$_2$的频率。进一步假设,儿童在几乎相同的发展阶段成功地习得了P$_1$和P$_2$。那么,在任何有关语言发展的定量研究的理论中,我们期望F$_1$和F$_2$也大体是一致的。相反,如果F$_1$和F$_2$的结果非常不同,那么P$_1$和P$_2$就必须表示性质不同的学习问题了。 

现在,让P$_1$是助动词变换的问题。两个选项是基于结构的假说(3b-i)和第一助动词假说(3a-i)。\citep[\page 155]{LY2002a}\footnote{%
   Suppose we have two independent problems of acquisition, P$_1$ and P$_2$, each
of which involves a binary decision. For P$_1$, let F$_1$ be the frequency of the
data that can settle P$_1$ one way or another, and for P$_2$, F$_2$. Suppose further
that children successfully acquire P$_1$ and P$_2$ at roughly the same developmental
stage. Then, under any theory that makes quantitative predictions of language
development, we expect F$_1$ and F$_2$ to be roughly the same. Conversely, if F$_1$ and
F$_2$ turn out significantly different, then P$_1$ and P$_2$ must represent qualitatively
different learning problems.

   Now let P$_1$ be the auxiliary inversion problem. The two choices are the
structure-dependent hypothesis (3b-i) and the first auxiliary hypothesis (3a-i). 
}
\end{quotation}

\noindent
有关英语中助动词的位置的知识是在儿童的3岁2个月时习得的。根据Legate和Yang,我们需要另一个在3岁2个月时学会的习得现象来进行比较。作者重点讨论了主语脱落问题\isc{参数!pro-脱落参数|(}\is{parameter!pro"=drop|(}\footnote{%
这个现象也叫做pro"=脱落(pro"=drop)。关于pro"=脱落参数的更为详细的讨论请看\ref{sec-pro-drop}。
},该现象是在36个月时学会的(比助动词变换早两个月)。根据作者的观点,习得问题包括一个二元决策过程:第一步,我们需要在(\ref{Hilfsverbhypothesen})中的两个假说进行选择。第二步,学习者需要决定一种语言是否使用显性主语。作者认为,像there的虚位词\isc{代词!虚指代词}\is{pronoun!expletive}的使用,可以用来证明学习者所学习的语言并不是可以带可选主语的那种语言。然后,他们在CHILDES语料库\isc{CHILDES语料库}\is{CHILDES}中计算了包括there"=主语的句子,并且估算了学习中听到的1,2\,\%句子的F$_2$。按照他们的观点,由于我们这里处理的是相同难度的现象,助动词变换要是能够可以学会的话,诸如(\ref{Hilfsverbinversion-mit-RS})和(\ref{wh-Fragen-Hilfsverbinversion})的句子就应该包括1.2\,\%的输入。
%The position of auxiliaries in English is learned by children at the age of 3;2. According to Legate
%and Yang, another acquisition phenomenon that is learned at the age of 3;2 is needed for
%comparison. The authors focus on subject drop\is{parameter!pro"=drop}\footnote{%
%  This phenomenon is also called \emph{pro"=drop}. For a detailed discussion of the pro"=drop
%  parameter see Section~\ref{sec-pro-drop}.
%}, that is learned
%at 36 months (two months earlier than auxiliary inversion). According to the authors, acquisition problems involve a binary decision:
%in the first case, one has to choose between the two hypotheses in (\ref{Hilfsverbhypothesen}). In the second case, the learner has to determine
%whether a language uses overt subjects. The authors assume that the use of expletives\is{pronoun!expletive} such as \emph{there} serves as
%evidence for learners that the language they are learning is not one with optional subjects. They then count the sentences in the CHILDES corpus\is{CHILDES}
%that contain \emph{there}"=subjects and estimate F$_2$ at 1,2\,\% of the sentences heard by the learner.
%Since, in their opinion, we are dealing with equally difficult phenomena here, sentences such as (\ref{Hilfsverbinversion-mit-RS}) and (\ref{wh-Fragen-Hilfsverbinversion})
%should constitute 1.2\,\% of the input in order for auxiliary inversion to be learnable.

然后,作者检索了Nina和Adam的语料(都属于CHILDES\isc{CHILDES语料库}\is{CHILDES}),并注意到0,068到0,045\,\%的语料具有(\ref{wh-Fragen-Hilfsverbinversion})的形式,而没有语料具有(\ref{Hilfsverbinversion-mit-RS})的形式。他们总结道,这个数字不足以作为正向证据。
%The authors then searched in the Nina and Adam corpora (both part of CHILDES\is{CHILDES}) and note that 0,068 to 0,045\,\% of utterances have the form of
%(\ref{wh-Fragen-Hilfsverbinversion}) and none have the form of (\ref{Hilfsverbinversion-mit-RS}). They conclude that this number is not sufficient as positive evidence.

在指出Pullum和Scholz从《华尔街日报》上得到的数据不必然跟语言习得相关这点上,Legate和Yang是正确的。而且,他们还指出了在数据中没有发现带有复杂主语名词短语的例子,或者至少是可以忽略不计的。但是,他们的论述还是有三个严重的问题:首先,在虚位主语和pro"=脱落语言的属性之间没有关系:加利西亚语\il{Galician} \citep[\S~2.5]{RU90a-u}是带有虚位代词的pro"=脱落语言。意大利语有一个虚位成分ci,\footnote{%
不过,并不是所有人都把ci看作是虚位成分。相关的概述请看 \citew{Remberger2009a}。
} 尽管意大利语\il{Italian}可以算作是pro"=脱落语言, \citet{Franks95a-u}把上索布语\il{Sorbian!Upper}和下索布语\il{Sorbian!Lower}列为pro"=脱落语言,他们的主语位置有虚位成分。
因此,由于虚位代词跟pro"=脱落参数没有关系,他们的出现频率就跟参数值的习得无关。如果在省略主语的可能性和主语虚位的出现频率之间有关系的话,那么说挪威语和丹麦语\il{Danish}的儿童就应该比说英语的儿童更早学会他们的语言必须要有主语,因为虚位成分在丹麦语和挪威语\il{Norwegian}中的出现频率更高\citep[\page 220]{SP2002b}。
在丹麦语中,对应于英语there"=结构的出现频率是英语的两倍。目前,在习得比率上是否真的存在差异仍是不清楚的\citep[\page 246]{Pullum2009a}。
%Legate and Yang are right in pointing out that Pullum and Scholz's data from the Wall Street Journal are not necessarily relevant for language acquisition and also in pointing
%out that examples with complex subject noun phrases do not occur in the data or at least to a
%negligible degree. There are, however, three serious problems with their argumentation: first, there
%is no correlation between the occurrence of expletive subjects and the property of being a pro"=drop
%language: Galician\il{Galician} \citep[Section~2.5]{RU90a-u} is a pro"=drop language with subject
%expletive pronouns, in Italian\il{Italian} there is an existential expletive \emph{ci},\footnote{%
%	However, \emph{ci} is not treated as an expletive by all authors. See  \citew{Remberger2009a} for an overview.
%} even though Italian counts as a pro"=drop language,  \citet{Franks95a-u} lists Upper\il{Sorbian!Upper} and Lower Sorbian\il{Sorbian!Lower} as pro"=drop languages
% that have expletives in subject position.
%Since therefore expletive pronouns have nothing to do with the pro"=drop parameter, their frequency is irrelevant for the acquisition of a parameter value. If there were a correlation
%between the possibility of omitting subjects and the occurrence of subject expletives, then Norwegian and Danish\il{Danish} children should learn that there has to be a subject
%in their languages earlier than children learning English since expletives occur a higher percentage of the time in Danish and Norwegian\il{Norwegian} \citep[\page 220]{SP2002b}.
%In Danish, the constructions corresponding to \emph{there}"=constructions in English are twice as frequent. It is still unclear whether there are actually differences in
%rate of acquisition \citep[\page 246]{Pullum2009a}.

第二,在构造他们的刺激贫乏论时,Legate和Yang认为是有天赋的语言知识的(pro"=脱落参数\isc{参数!pro-脱落参数|)}\is{parameter!pro"=drop|)})。
这样,他们就陷入了循环论证,因为他们理应说明内在的语言知识的假说是不可缺少的。\citep[\page 220]{SP2002b}。
%Second, in constructing their Poverty of the Stimulus argument, Legate and Yang assume that there is innate linguistic knowledge (the pro"=drop parameter\is{parameter!pro"=drop|)}).
%Therefore their argument is circular since it is supposed to show that the assumption of innate linguistic knowledge is indispensable \citep[\page 220]{SP2002b}. 

Legate和Yang的观点的第三个问题是,他们认为转换分析是唯一可能的分析。下面的引述清楚地说明了这个观点\citep[\page 153]{LY2002a}:
%The third problem in Legate and Yang's argumentation is that they assume that a transformational analysis is the only possibility. This becomes clear
%from the following citation \citep[\page 153]{LY2002a}:
\begin{quotation}
当然,构成问句的正确操作是基于结构的:它包括将句子分析为结构化组织的短语,并将位于主语NP后的助词前置,这可以是任意长度的 :\footnote{%
The correct operation for question formation is, of course, structure dependent: it involves parsing
the sentence into structurally organized phrases, and fronting the auxiliary that follows the
subject NP, which can be arbitrarily long:
}
\begin{exe}
\exi{(4)}
\begin{xlist}
\ex 
\gll Is [the woman who is singing] e happy?\\
\textsc{cop} \spacebr\textsc{det} 女人 \textsc{rel} \textsc{aux} 唱歌 {} 高兴\\
\mytrans{唱歌的女人高兴吗?}
\ex 
\gll Has [the man that is reading a book] e eaten supper?\\
\textsc{aux} \spacebr\textsc{det} 人 \textsc{rel} \textsc{aux} 读 一 书 {} 吃 晚饭\\
\mytrans{看书的那个人吃晚饭了吗?}
\end{xlist}
\end{exe}
\end{quotation}

\noindent
由Chomsky推进的这个分析是基于转换\isc{转换}\is{transformation}的(请看第\pageref{Seite-GB-Entscheidungsfragen-Englisch}页),也就是说,学习者需要按照Legate和Yang所描述的学习:助动词必须移到主语名词短语的前面。不过,还有其他分析说不需要变换或其他相当的机制。
如果我们的语言知识不包括任何有关变换的信息,那么他们关于需要学习什么的观点就是错误的。例如,我们可以假设,像在范畴语法\isc{范畴语法}\is{Categorial Grammar (CG)}中,助动词构成了一组具有特殊分布属性的词类。有可能替代他们是疑问句中观察到的首位,另一种是在主语后\citep[\page 104]{Villavicencio2002a}。
这样就需要习得主语是在中心语的左边还是右边实现的信息。除了这个基于词汇的分析,另一种方法是,我们可以采用构式语法\isc{构式语法}\is{Construction Grammar (CxG)}(Fillmore \citeyear[\page44]{Fillmore88a};\citeyear{Fillmore99a};\citealp[\page 18]{KF99a})、认知语法\isc{认知语法}\is{Cognitive Grammar} \citep[\S~9]{Dabrowska2004a}或HPSG理论\indexhpsg \citep{GSag2000a-u}来分析。在这些框架中,只有简单的两个模式\footnote{%
	 \citet{Fillmore99a}提出了主语助动词变换构式的次类型,因为这类变换并不只在问句中出现。
}来分析这两个序列,他们是根据动词和主语的语序来赋予不同的意义的。习得的问题就是学习者需要在输入中识别相应的短语范式。他们需要认识到Aux NP VP在英语中是合乎语法的,并且有疑问的语义。
构式语法导向的文献对习得的相关理论作了非常好的论述(请看\ref{Abschnitt-musterbasiert}和\ref{Abschnitt-Selektionsbasierter-Spracherwerb})。基于构式的习得理论也被我们能看到的频率效应这样的事实所支持,即,助词变换首先由儿童用在一些助词上,而且只在发展的后期,然后扩展到所有的助动词上。如果说话者学会助词构式具有Aux NP
VP的范式,那么Lasnik和Uriagereka在(\ref{Beispiel-Hilfsverbvoranstellung-Koordination})中提出的并列数据不再是问题了。这是因为,如果我们只将第一个并列成分指派到范式Aux NP VP的NP上,那么并列结构(and those who are not coming)中剩下的部分还是未分析的,而且不能被整合进整个句子中。由此,听者被迫修改他的假设,即will those who are coming对应于Aux NP VP中Aux NP的序列,然后使用整个NP成分those who are coming and those who are not coming。
%The analysis put forward by Chomsky (see page~\pageref{Seite-GB-Entscheidungsfragen-Englisch}) is a transformation"=based\is{transformation} one, that is, a learner
%has to learn exactly what Legate and Yang describe: the auxiliary must move in front of the subject noun phrase. There are, however, alternative analyses that
%do not require transformations or equivalent mechanisms.
%If our linguistic knowledge does not contain any information about transformations, then their claim about what has to be learned is wrong.
%For example, one can assume, as in Categorial Grammar\is{Categorial Grammar (CG)}, that auxiliaries form a word class with particular distributional properties.
%One possible placement for them is initial positions as observed in questions, the alternative is after the subject \citep[\page 104]{Villavicencio2002a}.
%There would then be the need to acquire information about whether the subject is realized to the
%left or to the right of its head. As an alternative to this lexicon"=based analysis,
%one could pursue a Construction Grammar\is{Construction Grammar (CxG)} (Fillmore \citeyear[\page44]{Fillmore88a};
%\citeyear{Fillmore99a}; \citealp[\page 18]{KF99a}), Cognitive Grammar\is{Cognitive Grammar} \citep[Chapter~9]{Dabrowska2004a}, or HPSG\indexhpsg \citep{GSag2000a-u} approach.
%In these frameworks, there are simply two\footnote{%
%	 \citet{Fillmore99a} assumes subtypes of the Subject Auxiliary Inversion Construction since this kind of inversion does not
%	only occur in questions.
%}
%schemata for the two sequences that assign different meanings according to the order of verb and subject. The acquisition problem is then that the learners have
%to identify the corresponding phrasal patterns in the input. They have to realize that Aux NP VP is a well"=formed structure in English that has interrogative
%semantics.
%The relevant theories of acquisition in the Construction Grammar"=oriented literature have been very well worked out (see Section~\ref{Abschnitt-musterbasiert} and
%\ref{Abschnitt-Selektionsbasierter-Spracherwerb}). Construction"=based theories of acquisition are also supported by the fact that one can see that there are
%frequency effects, that is, auxiliary inversion is first produced by children for just a few auxiliaries and only in later phases of development is it then extended to
%all auxiliaries. If speakers have learned that auxiliary constructions have the pattern Aux NP VP, then the coordination data provided by Lasnik and Uriagereka in 
%(\ref{Beispiel-Hilfsverbvoranstellung-Koordination}) no longer pose a problem since, if we only assign the first conjunct to the NP in the pattern Aux NP VP, then
%the rest of the coordinate structure (\emph{and those who are not coming}) remains unanalyzed and cannot be incorporated into the entire sentence.
%The hearer is thereby forced to revise his assumption that \emph{will those who are coming} corresponds to the sequence Aux NP in Aux NP VP and instead to
%use the entire NP \emph{those who are coming and those who are not coming}.
由此,对于习得英语来说,先学一些助动词的范式是Aux NP VP,然后再学习到所有助动词都符合这个模式,这个学习方式是充分的。 \citet{LE2001a}也证明了这个观点,他训练了一个神经网络\isc{神经网络|(}\is{neural network|(},其中使用了助动词构式中不包括带有关系从句的NP的数据。但是,关系小句在其他结构中有。训练语料的复杂度一点一点增加,就像儿童接收语言输入一样\citep{Elman93a}。\footnote{%
  这里有文化差异。在一些文化中,成年人不跟还没掌握全部语言能力的儿童讲话\citep{Ochs82a,OS85a}(也请参考\ref{Abschnitt-negative-Evidenz})。由此,儿童就必须从环境中学习,即他们听到的句子反映了语言的全部复杂性。
} 神经网络可以预测一个词语序列的下一个符号。对于带有疑问语序的句子来说,预测的结果是正确的。即使是(\mex{1})中的关系代词也被预测了,尽管序列Aux Det N Relp在训练语料中并没有出现。
%For acquisition, it is therefore enough to simply learn the pattern Aux NP VP first for some and then eventually for all auxiliaries in English.
%This has also been shown by  \citet{LE2001a}, who trained a neural network\is{neural network|(} exclusively with data that did not contain NPs with relative
%clauses in auxiliary constructions. Relative clauses were, however, present in other structures. The complexity of the training material was increased bit
%by bit just as is the case for the linguistic input that children receive
%\citep{Elman93a}.\footnote{%
%	There are cultural differences. In some cultures, adults do not talk to children that have not attained
%	full linguistic competence \citep{Ochs82a,OS85a} (also see
%  Section~\ref{Abschnitt-negative-Evidenz}). Children have to therefore learn the language from their environment, that is, the sentences that
%  they hear reflect the full complexity of the language.
%} The neural network can predict the next symbol after a sequence of words. For sentences with interrogative word order, the predictions are correct.
%Even the relative pronoun in (\mex{1}) is predicted despite the sequence Aux Det N Relp never occurring in the training material.
\ea
\gll Is the boy who is smoking crazy?\\
\textsc{cop} \textsc{det} 男孩 \textsc{rel} \textsc{aux} 抽烟 疯狂\\
\mytrans{抽烟的那个男孩疯狂吗?}
\z
进而,如果这个网络显示了不合乎语法的句子(\mex{1}),那么系统就会给出一个出错的信号:
%Furthermore, the system signals an error if the network is presented with the ungrammatical sentence (\mex{1}):
\ea[*]{
\gll Is the boy who smoking is crazy?\\
\textsc{aux} \textsc{det} 男孩 \textsc{rel} 抽烟 \textsc{cop} 疯狂\\
}
\z
关系代词后面不应该是动名词,应该是定式动词。构建的神经网络当然不是我们在习得和言语生成的过程中对我们大脑中活动的充分模拟。\footnote{%
请看 \citet[\page 324]{Hurford2002a}和 \citet[\S~6.2]{Jackendoff2007a}有关神经网络的特定类型带来的问题,以及 \citet{Pulvermueller2003a,Pulvermueller2010a}提出的不具有这些问题的另一种观点。
}但是,实验表明,学习者接收到的输入包括可以用来习得语言的丰富的统计\isc{统计学}\is{statistics}信息。\isc{神经网络|)}\is{neural network|)}Lewis和Elman指出,输入中词的分布的统计信息不是说话者所有的唯一信息。除了关于分布的信息,他们也暴露在上下文\isc{上下文}\is{context}的信息中,而且可以利用词在语音上的相似性。
%A gerund is not expected after the relative pronoun, but rather a finite verb. The constructed neural network is of course not yet an adequate model of what is
%going on in our heads during acquisition and speech production.\footnote{%
%  See  \citet[\page 324]{Hurford2002a} and  \citet[Section~6.2]{Jackendoff2007a} for problems that arise for certain kinds of neural
%  networks and  \citet{Pulvermueller2003a,Pulvermueller2010a} for an alternative
%  architecture that does not have these problems.
%} The experiment shows, however, that the input that the learner receives contains rich statistical\is{statistics} information that can be
%used when acquiring language.\is{neural network|)} Lewis and Elman point out that the statistical information about the distribution of words in the input
%is not the only information that speakers have. In addition to information about distribution, they are also exposed to information about the context\is{context}
%and can make use of phonological similarities in words.

跟(\mex{0})中的不合乎语法的句子有关的是,有人认为,永远不会造出这类句子的事实说明儿童已经知道语法操作是基于结构的,而且这就是为什么他们不会有这样的想法,只有线性上第一个动词移位了\citep{CN87a-u}。这个观点并不易于验证,因为儿童一般不说相对复杂的句子。由此,测试他们会犯的相应的错误的非法句子是唯一可能。 \citet{CN87a-u}做了这样的实验。他们的研究被 \citet*{ARP2008a}批评了,因为这些作者可以证明儿童真的在对助动词前置的时候会犯错误。作者将Crain和Nakayama的第一次研究的结果跟Crain和Nakayama的研究中助动词的错误选择区分开。由于助动词is的使用,不合乎语法的例子具有从不或很少互相挨着出现的词对儿(\mex{1}a中的who running)。
%In connection to the ungrammatical sentences in (\mex{0}), it has been claimed that the fact that such sentences can never be produced shows
%that children already know that grammatical operations are structure"=dependent and this is why they do not entertain the hypothesis that it is simply
%the linearly first verb that is moved \citep{CN87a-u}. The claim simply cannot be verified since children do not normally form the relevant complex
%utterances. It is therefore only possible to experimentally illicit utterances where they could make the relevant mistakes.
% \citet{CN87a-u} have carried out such experiments. Their study has been criticized by  \citet*{ARP2008a} since these authors could show that children
%do really make mistakes when fronting auxiliaries. The authors put the difference to the results of the first study by Crain and Nakayama down to unfortunate choice
%of auxiliary in Crain and Nakayama's study. Due to the use of the auxiliary \emph{is}, the ungrammatical examples had pairs of words that never or only
%very rarely occur next to each other (\emph{who running} in (\mex{1}a)). 
\eal
\ex 
\gll  \hspaceThis{*~} The boy who is running fast can jump high. \\
%\gll \hspaceThis{*~} The boy who is running fast can jump high. \\
      {}              \textsc{det} 男孩 \textsc{rel} \textsc{aux} 跑 快 能 跳 高\\
\glt \hspaceThis{*~}\quotetrans{跑得快的那个男孩跳的高。}\\
$\to$\\
 {}* \gll Is the boy who running fast can jump high?\\
          \textsc{aux} \textsc{det} 男孩 \textsc{rel} 跑 快 能 跳 高\\
\ex 
\gll \hspaceThis{*~} The boy who can run fast can jump high. \\
     {}              \textsc{det} 男孩 \textsc{rel} \textsc{aux} 跑 快 能 跳 高\\
\glt \hspaceThis{*~}\quotetrans{跑得快的男孩可以跳得高。}\\
$\to$\\
 {}* \gll Can the boy who run fast can jump high?\\
          \textsc{aux} \textsc{det} 男孩 \textsc{rel} 跑 快 能 跳 高\\
\zl
如果我们使用助动词can,这个问题就消失了,因为who和run一定是一起出现的。然后这就导致儿童实际上会犯他们不应该犯的错误,因为不正确的句子确实违反了理应属于天赋的语言知识的那部分。
%If one uses the auxiliary \emph{can}, this problem disappears since \emph{who} and \emph{run} certainly do appear together. This then leads to the children
%actually making mistakes that they should not have, as the incorrect utterances actually violate a constraint that is supposed to be part of innate
%linguistic knowledge.

 \citet{Estigarribia2009a}具体调查了英语的极性问句。他指出,孩子们输入的具有Aux NP VP形式的极性问句连一半都不到(第74页)。相反,父母会用简化的方式跟孩子沟通,并且使用如下的的句子:
% \citet{Estigarribia2009a} investigated English polar questions in particular. He shows that not even half of the polar questions in children's input have
% the form Aux NP VP (p.\,74).
%Instead, parents communicated with their children in a simplified form and used sentences such as:
\eal
\ex 
\gll That your tablet?\\
那 你的 药片\\
\mytrans{那是你的药片?}
\ex 
\gll He talking?\\
他 说话\\
\mytrans{他在说话?}
\ex 
\gll That taste pretty good?\\
\textsc{det} 品尝 特别 好\\
\mytrans{那个尝起来特别好吃?}
\zl
Estigarribia将不同的范式分成复杂的类型,如下所示:
%Estigarribia divides the various patterns into complexity classes of the following kind:
\textsc{frag}
(fragmentary)、\textsc{spred} (subject predicate) 和 \textsc{aux-in} (auxiliary
  inversion)。(\mex{1})指出了相应的例子:
%(\emph{fragmentary}), \textsc{spred} (\emph{subject predicate}) and \textsc{aux-in} (\emph{auxiliary
%  inversion}). (\mex{1}) shows corresponding examples:
\eal\settowidth\jamwidth{(\textsc{aux-in})}
\ex 
\gll coming tomorrow?         \jambox{(\textsc{frag})}\\
来 明天\\
\mytrans{明天来吗?}
\ex 
\gll you coming tomorrow?     \jambox{(\textsc{spred})}\\
你 来 明天\\
\mytrans{你明天来吗?}
\ex 
\gll Are you coming tomorrow? \jambox{(\textsc{aux-in})}\\
\textsc{aux} 你 来 明天\\
\mytrans{你明天来吗?}
\zl
我们看到的是,复杂度一类比一类高。Estigarribia提出了一种语言习得的系统,其中更简单的类型在更为复杂的类型之前习得,而后者从更为简单类型的外围修饰成分发展而来(第76页)。他认为,问句形式是从右到左学习的(right to left elaboration\isc{右向左的精细化}\is{right to left
  elaboration}),也就是说,(\mex{0}a)是先学的,然后是(\mex{0}b)中在(\mex{0}a)的基础上包括主语的范式,再然后,在第三步,才出现附加的助动词模式(\mex{0}c)(第82页)。
在这种学习过程中,没有包括助动词的变换。这个观点跟基于约束的分析是一致的,如 \citet{GSag2000a-u}。 \citet*{FPAG2007a}提出的一个类似的方法将在\ref{Abschnitt-musterbasiert}讨论。
%What we see is that the complexity increases from class to class. Estigarribia suggests a system of language
%acquisition where simpler classes are acquired before more complex ones and the latter ones develop from peripheral
%modifications of more simple classes
%(p.\,76). He assumes that question forms are learned from right to left
% (\emph{right to left elaboration}\is{right to left
%  elaboration}), that is, (\mex{0}a) is learned first, then the pattern in (\mex{0}b) containing a subject in addition to the material in (\mex{0}a), and
% then in a third step, the pattern (\mex{0}c) in which an additional auxiliary occurs (p.\,82). 
%In this kind of learning procedure, no auxiliary inversion is involved. This view is compatible with constraint"=based analyses such as that of
%  \citet{GSag2000a-u}. 
%A similar approach to acquisition by  \citet*{FPAG2007a} will be discussed in Section~\ref{Abschnitt-musterbasiert}.

 \citet{Bod2009a}提出了一项更为有趣的研究。他指出,如果我们假设带有任意种类分叉的树的话,是有可能学会助动词变换的,即使在输入中没有复杂名词短语的助动词变换。他使用的分析策略和他得到的结果非常有趣,我们将在第\ref{Abschnitt-UDOP}节来详细探讨。
%A further interesting study has been carried out by  \citet{Bod2009a}. He shows that it is possible to learn auxiliary inversion
%assuming trees with any kind of branching even if there is no auxiliary inversion with complex noun phrases present
%in the input. The procedure he uses as well as the results he gains are very interesting and will be discussed 
%in Section~\ref{Abschnitt-UDOP} in more detail.

总之,我们可以说儿童在助动词位置的使用上是会犯错误的,而这些错误也许不会犯,如果这些相关的知识是天赋的话。关于输入中词的分布的统计信息足以学会输入中实际不带这种复杂句子的复杂句子的结构。
%In conclusion, we can say that children do make mistakes with regard to the position of auxiliaries that they
%probably should not make if the relevant knowledge were innate. Information about the statistical
%distribution of words in the input is enough to learn the structures of complex sentences without
%actually having this kind of complex sentences in the input.% 
\isc{助动词倒装|)}\is{auxiliary inversion|)}\il{English|)}

\subsubsection{小结}
%\subsubsection{Summary}

\mbox{} \citet[\page 19]{PS2002a}提出,刺激贫乏论(APS)应按照下面的结构来表示:
%\mbox{} \citet[\page 19]{PS2002a} show what an Argument from Poverty of the Stimulus (APS) would have to look like if it were
%constructed correctly:
\ea
\begin{tabular}[t]{@{}l@{~~}p{10.8cm}@{}}
\multicolumn{2}{@{}l@{}}{APS的具体策略:}\\
a. & 习得特征:详细描述应该知道的知识。\\
b. & 缺陷规范:识别出学习者必须接触到的句子的集合,这样关于习得的数据驱动的学习就会得到支持。\\
c. & 不可或缺论:给出如果学习是数据驱动的理论,然后在没有接触到缺陷的句子的时候,习得就不会发生。\\
d. & 不可及证据:支持有缺陷的句子的类例在习得过程中对学习者来说是接触不到的观点。\\
e. & 习得证据:给出理由以相信习得实际上在童年时期就被学习者熟知了。\\
%\multicolumn{2}{@{}l@{}}{APS specification schema:}\\
%a. & ACQUIRENDUM CHARACTERIZATION: describe in detail what is alleged to be known.\\
%b. & LACUNA SPECIFICATION: identify a set of sentences such that if the learner had access to them, the claim of data-driven learning
%of the acquirendum would be supported.\\
%c. & INDISPENSABILITY ARGUMENT: give reason to think that if
%learning were data-driven, then the acquirendum could not be
%learned without access to sentences in the lacuna.\\
%d. & INACCESSIBILITY EVIDENCE: support the claim that tokens of sentences in the lacuna were not available to the learner during the acquisition process.\\
%e. & ACQUISITION EVIDENCE: give reason to believe that the acquirendum does in fact become known to learners during childhood.\\
\end{tabular}
\z
正如上面四个问题的研究所示,有许多反对习得论的理由。如果没有必要获得习得,那么就没有任何内在的语言知识的证据了。
习得论必须至少是足够可描述地。这是语言学家可以回答的语言事实方面的问题。在Pullum和Scholz讨论的四个刺激贫乏论中的三个观点中,有没有被充分描写的部分。在前面的章节中,我们已经接触到了其他的刺激贫乏论,其中包括语言数据无法在事实上支持的观点(如邻接原则)。
对于(\mex{0})中剩下的观点,需要跨学科的研究工作:缺陷的具体化要落实到形式语言\isc{形式语言}\is{formal language}的理论中(句子的集合的具体化),不可或缺论是一个学习理论\isc{学习理论}\is{learning theory}领域中的数学任务,不可及的证据是一个语言事实方面的问题,可以通过语料库获得,最终习得的证据是实验发展心理学的问题\citep[\page 19--20]{PS2002a}。
%As the four case studies have shown, there can be reasons for rejecting the acquirendum. If the acquirendum does not have to be acquired, than there is no
%longer any evidence for innate linguistic knowledge.
%The acquirendum must at least be descriptively adequate. This is an empirical question that can be answered by linguists. In three of the four PoS arguments discussed
%by Pullum and Scholz, there were parts which were not descriptively adequate. In previous sections, we already encountered other PoS arguments that involve
%claims regarding linguistic data that cannot be upheld empirically (for example, the Subjacency Principle). 
%For the remaining points in (\mex{0}), interdisciplinary work is required: the specification of the lacuna falls into the theory of formal language\is{formal language}
%(the specification of a set of utterances), the argument of indispensability is a mathematical task
%from the realm of learning theory\is{learning theory}, the evidence for inaccessibility is an
%empirical question that can be approached by using corpora, and finally the evidence for acquisition
%is a question for experimental developmental psychologists \citep[\page 19--20]{PS2002a}. 

 \citet[\page 46]{PS2002a}指出了关于(\mex{0}c)的一个有趣的悖论:没有学习的数学理论的结果,我们无法获得(\mex{0}c)。如果我们希望提出有效的刺激贫乏理论,我们就需要自动在学习理论中得到进步,也就是说,有可能比之前认为的学会更多。
% \citet[\page 46]{PS2002a} point out an interesting paradox with regard to (\mex{0}c):
%without results from mathematical theories of learning, one cannot achieve (\mex{0}c). If one wishes to provide a valid
%Poverty of the Stimulus Argument, then this should automatically lead to improvements in theories of learning, that is, it is possible
%to learn more than was previously assumed.

\subsection{无指导的数据驱动的剖析(U-DOP)}
%\subsection{Unsupervised Data-Oriented Parsing (U-DOP)}
\label{Abschnitt-UDOP}

\mbox{} \citet{Bod2009a}\isc{助动词倒装|(}\is{auxiliary inversion|(}\isc{统计学|(}\is{statistics|(}\isc{无监督的面向数据的句法分析(U-DOP)|(}\is{Unsupervised Data-Oriented Parsing (U-DOP)|(}提出了不需要任何有关句中的词类或词间关系信息的程式。
我们唯一需要做的假设是,有着某种结构。这个程式包括三个步骤:
%\mbox{} \citet{Bod2009a}\is{auxiliary inversion|(}\is{statistics|(}\is{Unsupervised Data-Oriented Parsing (U-DOP)|(} 
%has developed a procedure that does not require any information about word classes or relations between words
%contained in utterances.\todostefan{Integrate  \citew{CL2011a-u,CL2013a-u,CL2012a-u,LS2007a-u}}
%The only assumption that one has to make is that there is some kind of structure. The procedure consists of three steps:
\begin{enumerate}
\item 针对给定句子的集合计算所有可能的(不带范畴符号的)(二叉)树\isc{分支!二叉}\is{branching!binary}。
\item 将这些树分成子树。
\item 计算出每个句子的理想树。
%\item Compute all possible (binary"=branching) trees\is{branching!binary} (without category symbols) for a set
%of given sentences.
%\item Divide these trees into sub"=trees.
%\item Compute the ideal tree for each sentence.
\end{enumerate}
这个过程可以解释(\mex{1})中的句子:
%This process will be explained using the sentences in  (\mex{1}):
\eal
\ex 
\gll Watch the dog.\\
小心 \textsc{det} 狗\\
\mytrans{小心狗。}
\ex 
\gll The dog barks.\\
\textsc{det} 狗 叫\\
\mytrans{狗叫。}
\zl
指派给这些语句的树只使用了范畴符号X,因为相关短语的范畴还未知。为了让例子具有可读性,这些词本身不会被给予范畴X,尽管我们当然可以这样做。图\vref{Abbildung-unlabeled-trees}说明了(\mex{0})的树。
%The trees that are assigned to these utterances only use the category symbol X since the categories for the relevant phrases
%are not (yet) known. In order to keep the example readable, the words themselves will not be given the category X, although
%one can of course do this. Figure~\vref{Abbildung-unlabeled-trees} shows the trees for (\mex{0}).
\begin{figure}
\hfill
\begin{forest}
sn edges
[X
	[X
		[watch\\
		 看]
		[the\\
		\textsc{det}]]
	[dog\\
	 狗]]
\end{forest}
\hfill
\begin{forest}
sn edges
[X
	[watch\\
	  看]
	[X
		[the\\
		\textsc{det}]
		[dog\\
		 狗]]]
\end{forest}
\hfill\mbox{}
\\[3ex]
\hfill\begin{forest}
sn edges
[X
	[X
		[the\\
		\textsc{det}]
		[dog\\
		狗]]
	[barks\\
	叫]]
\end{forest}
\hfill
\begin{forest}
sn edges
[X
	[the\\
	\textsc{det}]
	[X
		[dog\\
		狗]
		[barks\\
		叫]]]
\end{forest}
\hfill\mbox{}
\caption{\label{Abbildung-unlabeled-trees}Watch the
    dog和The dog barks的可能的二叉树结构.}
%\caption{\label{Abbildung-unlabeled-trees}Possible binary"=branching structures for \emph{Watch the
%    dog} and \emph{The dog barks}.}
%\vspace{-\baselineskip}
\end{figure}%
%%
%%
下一步,这些树被分成了子树。图\ref{Abbildung-unlabeled-trees}中的树具有图\vref{Abbildung-Teilbaume}中可见的子树。
%In the next step, the trees are divided into subtrees. The trees in Figure~\ref{Abbildung-unlabeled-trees} have the subtrees that can be seen in Figure~\vref{Abbildung-Teilbaume}.
\begin{figure}
\hfill\begin{forest}
sn edges
[X
	[X
		[watch\\
		看]
		[the\\
		\textsc{det}]]
	[dog\\
	狗]]
\end{forest}
\hfill
\begin{forest}
sn edges
[X
	[X [,phantom ]]
	[dog\\
	狗]]
\end{forest}
\hfill
\begin{forest}
[X
	[watch\\
	看]
	[the\\
	\textsc{det}]]
\end{forest}\hfill\mbox{}
\\[3ex]
\hfill\begin{forest}
sn edges
[X
	[watch\\
	看]
	[X
		[the\\
		\textsc{det}]
		[dog\\
		狗]]]
\end{forest}
\hfill
\begin{forest}
sn edges
[X
	[watch\\
	看]
	[X [,phantom ]]]
\end{forest}
\hfill
\begin{forest}
[X
	[the\\
	\textsc{det}]
	[dog\\
	狗]]
\end{forest}\hfill\mbox{}
\\[3ex]
\hfill\begin{forest}
sn edges
[X
	[X
		[the\\
		\textsc{det}]
		[dog\\
		狗]]
	[barks\\
	叫]]
\end{forest}
\hfill
\begin{forest}
sn edges
[X
	[X [,phantom ]]
	[barks\\
	叫]]
\end{forest}
\hfill
\begin{forest}
[X
	[the\\
	\textsc{det}]
	[dog\\
	狗]]
\end{forest}\hfill\mbox{}
\\[3ex]
\hfill\begin{forest}
sn edges
[X
	[the\\
	\textsc{det}]
	[X
		[dog\\
		狗]
		[barks\\
		叫]]]
\end{forest}
\hfill
\begin{forest}
sn edges
[X
	[the\\
	\textsc{det}]
	[X [,phantom ]]]
\end{forest}
\hfill
\begin{forest}
[X
	[dog\\
	狗]
	[barks\\
	叫]]
\end{forest}
\hfill\mbox{}
\caption{\label{Abbildung-Teilbaume}图\ref{Abbildung-unlabeled-trees}中的树的子树}
%\caption{\label{Abbildung-Teilbaume}Subtrees for the trees in Figure~\ref{Abbildung-unlabeled-trees}}
\end{figure}%
第三步,我们现在需要计算每个语句的最优树。对于The dog
  barks.来说,子树的集合中有两棵树完全对应于这个语句。但是,也可以从子树中构造结构。由此,针对The dog
  barks.这句话,就有多重推导的可能性,他们都使用了图\ref{Abbildung-Teilbaume}中的树:
  一方面是使用了整棵树的许多小的推导过程,另一方面是从小的子树构建树的推导过程。图\ref{Abbildung-Analyse}展现了这些子树是如何构造的。
%In the third step, we now have to compute the best tree for each utterance. For \emph{The dog
%  barks.}, there are two trees in the set of the subtrees that correspond exactly to this utterance.
%But it is also possible to build structures out of subtrees. There are therefore multiple derivations possible
%for \emph{The dog
 % barks.} all of which use the trees in Figure~\ref{Abbildung-Teilbaume}: 
%  one the one hand, trivial derivations that use the entire tree, and on the other, derivations that
%  build trees from smaller subtrees.
%Figure~\ref{Abbildung-Analyse} gives an impression of how this construction of subtrees happens.
\begin{figure}
\hfill
\adjustbox{valign=c}{%
\scalebox{.9}{%
\begin{forest}
sn edges
[X
	[the\\
	\textsc{det}]
	[X
		[dog\\
		狗]
		[barks\\
		叫]]]
\end{forest}}}
由
%is created by
\adjustbox{valign=c}{%
\scalebox{.9}{%
\begin{forest}
sn edges
[X
	[the\\
	\textsc{det}]
	[X
		[dog\\
		狗]
		[barks\\
		叫]]]
\end{forest}}}
和
%and
\adjustbox{valign=c}{%
\scalebox{.9}{%
\begin{forest}
sn edges
[X
	[the\\
	\textsc{det}]
	[X [,phantom ]]]
\end{forest}}}
$\circ$
\adjustbox{valign=c}{%
\scalebox{.9}{%
\begin{forest}
[X
	[dog\\
	狗]
	[barks\\
	叫]]
\end{forest}}}\hfill\mbox{}
生成
\\[3ex]
\hfill\adjustbox{valign=c}{%
\scalebox{.9}{%
\begin{forest}
sn edges
[X
	[X
		[the\\
		\textsc{det}]
		[dog\\
		  狗]]
	[barks\\
	 叫]]
\end{forest}}}
由
%is created by
\adjustbox{valign=c}{%
\scalebox{.9}{%
\begin{forest}
sn edges
[X
	[X
		[the\\
		\textsc{det}]
		[dog\\
		狗]]
	[barks\\
	叫]]
\end{forest}}}
和
\adjustbox{valign=c}{%
\scalebox{.9}{%
\begin{forest}
sn edges
[X
	[X [,phantom ]]
	[barks\\
	叫]]
\end{forest}}}
$\circ$
\adjustbox{valign=c}{%
\scalebox{.9}{%
\begin{forest}
[X
	[the\\
	\textsc{det}]
	[dog\\
	狗]]
\end{forest}}}
生成
\hfill\mbox{}
\caption{\label{Abbildung-Analyse}应用图\ref{Abbildung-Teilbaume}中的子树的The dog barks的分析}
%\caption{\label{Abbildung-Analyse}Analysis of \emph{The dog barks} using subtrees from Figure~\ref{Abbildung-Teilbaume}}
\end{figure}%
如果我们现在想判断(\mex{1})中的哪个分析是最优的,我们就必须要计算每棵树的概率。
%If we now want to decide which of the analyses in (\mex{1}) is the best, then we have to compute the probability of each tree.
\eal
\ex \gll {}[[the dog] barks]\\
\hspaceThis{[[}\textsc{det} 狗 叫\\
\mytrans{狗叫}
\ex \gll {}[the [dog barks]]\\
         \spacebr\textsc{det} \spacebr{}狗 叫\\
\mytrans{狗叫}
\zl
一棵树的概率是它的所有分析的可能性的总和。
我们可以在图\ref{Abbildung-Analyse}中找到(\mex{0}b)的两种分析。(\mex{0}b)的第一种分析的可能性对应于从所有子树的集合中选择[the [dog barks]]的完整树的概率。因为有十二棵子树,选择这个的概率是1/12。第二个分析的概率是组合的子树的概率的结果,由此是1/12 $\times$ 1/12 = 1/144。(\mex{0}b)的分析的概率就是1/12 $+$ (1/12 $\times$ 1/12) = 13/144。
我们可以就此按照相同方式计算(\mex{0}a)中的树的概率。这里唯一的区别是[the dog]的树在子树集合中出现了两次。由此,它的概率是2/12。[[the dog] barks]这棵树的概率就是:1/12 $+$ (1/12 $\times$ 2/12) = 14/144。这样,我们就从语料中提取了貌似正确的结构知识。这个知识也可以在人们听到一个没有完整树的新的语句时使用。然后,就有可能使用已经知道的子树来计算这个新语句的可能分析的概率了。
Bod的模型也考虑了权重:那些说话者很长时间以前听到的句子会得到较低的权重。据此,我们也可以解释这样的事实,儿童并不是使用他们同时可听到的所有句子。
这个扩展使得UDOP模型对于语言习得\isc{习得}\is{acquisition}来说更为可信了。
%The probability of a tree is the sum of the probabilities of all its analyses.
%There are two analyses for (\mex{0}b), which can be found in Figure~\ref{Abbildung-Analyse}.
%The probability of the first analysis of (\mex{0}b) corresponds to the probability of choosing exactly the complete tree for [the [dog barks]] from
%the set of all subtrees. Since there are twelve subtrees, the probability of choosing that one is 1/12. The probability of the second
%analysis is the product of the probabilities of the subtrees that are combined and is therefore 1/12 $\times$ 1/12 = 1/144.
%The probability of the analysis in (\mex{0}b) is therefore 1/12 $+$ (1/12 $\times$ 1/12) = 13/144.
%One can then calculate the probability of the tree in (\mex{0}a) in the same way. The only difference here is that the tree for
%[the dog] occurs twice in the set of subtrees. Its probability is therefore
% 2/12. The probability of the tree [[the dog] barks] is therefore:
% 1/12 $+$ (1/12 $\times$ 2/12) = 14/144. We have thus extracted knowledge about plausible structures from the corpus. This knowledge can
%also  be applied whenever one hears a new utterance for which there is no complete tree. It is then possible to use already known
% subtrees to calculate the probabilities of possible analyses of the new utterance.
%Bod's model can also be combined with weights: those sentences that were heard longer ago by the speaker, will receive a lower weight.
%One can thereby also account for the fact that children do not  have all sentences that they have ever heard available simultaneously. 
%This extension makes the UDOP model more plausible for language acquisition\is{acquisition}.

在上面的例子中,我们没有给词指派范畴信息。如果我们这么做了,就会得到图\vref{Abbildung-diskontinuierlich}中的树作为子树。
%In the example above, we did not assign categories to the words. If we were to do this, then we
%would get the tree in Figure~\vref{Abbildung-diskontinuierlich} as a possible
subtree.
\begin{figure}
\centering
\begin{forest}
[X
	[X
		[watch\\
		看,tier=word]]
	[X
		[X]
		[X
			[dog\\
			 狗,tier=word]]]]
\end{forest}
\caption{\label{Abbildung-diskontinuierlich}非连续的部分树}
%\caption{\label{Abbildung-diskontinuierlich}Discontinuous partial tree}
\end{figure}%
如果我们想获得在给定树的不同子树间出现成分的依存关系,这些非连续的子树是很重要的。比如说下面的句子:
%These kinds of discontinuous subtrees are important if one wants to capture dependencies between elements that occur in different subtrees
%of a given tree. Some examples are the following sentences:

\eal
\ex 
\gll BA carried \emph{more} people \emph{than} cargo in 2005.\\
波音 运载 更多 人 比 货 在 2005\\
\mytrans{在2005年,波音公司运载的人比货多。}
\ex 
\gll \emph{What's} this scratch \emph{doing} on the table?\\
什么.\textsc{aux} 这 划痕 做 \textsc{prep} \textsc{det} 桌子\\
\mytrans{桌子上怎么有个划痕?}
\ex 
\gll Most software \emph{companies} in Vietnam \emph{are} small sized.\\
大多数 软件 公司 在 越南 \textsc{cop} 小 型号\\
\mytrans{大多数越南的软件公司都是小型的。}
\zl

\noindent
然后,也有可能用这些非连续的树来学习英语的助动词变换。为了能得到正确的句子(\mex{2}a),而不是错误的句子(\mex{2}b),我们所需要的是(\mex{1})中的两个句子的树结构。
%It is then also possible to learn auxiliary inversion in English with these kinds of discontinuous
%trees. All one needs are tree structures for the two sentences in (\mex{1}) in order to prefer the correct sentence (\mex{2}a) over the incorrect one (\mex{2}b).

\eal
\label{Beispiel-Inversion}
\ex 
\gll The man who is eating is hungry.\\
\textsc{det} 人 \textsc{rel} \textsc{aux} 吃 \textsc{cop} 饿\\
\mytrans{正吃东西的那个人饿了。}
\ex 
\gll Is the boy hungry?\\
\textsc{cop} \textsc{det} 男孩 饿\\
\mytrans{男孩饿了吗?}
\zl

\eal
\ex[]{\label{Bsp-Is-the-man-who-is-eating-hungry}
\gll Is the man who is eating hungry?\\
\textsc{cop} \textsc{det} 人 \textsc{rel} \textsc{aux} 吃 饿\\
\mytrans{吃东西的那个人饿了吗?}
}
\ex[*]{
\gll Is the man who eating is hungry?\\
\textsc{aux} \textsc{det} 人 \textsc{rel} 吃 \textsc{cop} 饿\\
}
\zl

\noindent
U-DOP可以从(\mex{1})中的句子学会图\vref{Abbildung-Strukturen-fuer-Fragen-und-RS}中(\mex{-1})的结构。
%U-DOP can learn the structures for (\mex{-1}) in Figure~\vref{Abbildung-Strukturen-fuer-Fragen-und-RS} from the sentences in (\mex{1}):

\eal
\label{Hilfsverbinversion-Input}
\ex\label{Bsp-The-man-who-is-eatin-mumbled}
\gll The man who is eating mumbled.\\
\textsc{det} 人 \textsc{rel} \textsc{aux} 吃 嘟囔\\
\mytrans{正吃东西的那个人嘟囔了。}
\ex \gll The man is hungry.\\
\textsc{det} 人 \textsc{cop}  饿\\
\mytrans{那个人饿了。}
\ex \gll The man mumbled.\\
\textsc{det} 人 嘟囔\\
\mytrans{那个人嘟囔了。}
\ex \gll The boy is eating.\\
\textsc{det} 男孩 \textsc{aux} 吃\\
\mytrans{男孩正在吃。}
\zl

\noindent
请注意这些句子不包括(\mex{-1}a)中的任何结构的实例。
%Note that these sentences do not contain any instance of the structure in (\mex{-1}a).
\begin{figure}
\hfill
\scalebox{.9}{\begin{forest}
sn edges
[X
	[X
		[X
			[X
				[the\\
				\textsc{det}]]
			[X
				[man\\
				人]]]
		[X
			[X
				[who\\
				\textsc{rel}]]
			[X
				[X
					[is\\
					\textsc{aux}]]
				[X
					[eating\\
					 吃]]]]]
	[X
		[X
			[is\\
			\textsc{cop}]]
		[X
			[hungry\\
			饿]]]]
\end{forest}}
\hfill
\scalebox{.9}{%
\begin{forest}
sn edges
[X
	[X
		[is\\
		\textsc{cop}]]
	[X
		[X
			[X
				[the\\
				\textsc{det}]]
			[X
				[boy\\
				男孩]]]
		[X
			[hungry\\
			饿]]]]
\end{forest}}
\hfill\mbox{}
\caption{\label{Abbildung-Strukturen-fuer-Fragen-und-RS}从(\ref{Beispiel-Inversion})和(\ref{Hilfsverbinversion-Input})中的例子学会的U-DOP的结构}
%\caption{\label{Abbildung-Strukturen-fuer-Fragen-und-RS}Structures that U-DOP learned from the examples in (\ref{Beispiel-Inversion}) and (\ref{Hilfsverbinversion-Input})}
\end{figure}%
根据这里学会的结构,有可能证明助词的位置的最短推导过程也有可能是正确的:正确的语序是Is the man who is eating
  hungry?只需要图\vref{Abbildung-Kombination-fuer-grammatischen-Satz}中的部分被组合起来,而*Is the man who eating is hungry?的结构需要将图\ref{Abbildung-Strukturen-fuer-Fragen-und-RS}中的至少四棵子树互相组合起来。这如图\vref{Abbildung-Kombination-fuer-ungrammatischen-Satz}所示。
%With the structures learned here, it is possible to show that the shortest possible derivation for
%the position of the auxiliary is also the correct one: the correct order
%\emph{Is the man who is eating
 % hungry?} only requires that the fragments in Figure~\vref{Abbildung-Kombination-fuer-grammatischen-Satz} are combined, whereas the structure for
%  * \emph{Is the man who eating is hungry?} requires  at least four subtrees from Figure~\ref{Abbildung-Strukturen-fuer-Fragen-und-RS} to be combined
 % with each other. This is shown by Figure~\vref{Abbildung-Kombination-fuer-ungrammatischen-Satz}.

\begin{figure}
\hfill
\adjustbox{valign=c}{%
\begin{forest}
[X
	[X
		[is\\
		\textsc{cop},tier=word]]
	[X
		[X]
		[X
			[hungry\\
			饿,tier=word]]]]
\end{forest}
}
\hfill
$\circ$
\hfill
\hspace{5mm}\adjustbox{valign=c}{%
\begin{forest}
sn edges
[X
	[X
		[X
			[the\\
			\textsc{det}]]
		[X
			[man\\
			人]]]
	[X
		[X
			[who\\
			\textsc{rel}]]
		[X
			[X
				[is\\
				\textsc{aux}]]
			[X
				[eating\\
				吃]]]]]
\end{forest}}
\hfill\mbox{}
\caption{\label{Abbildung-Kombination-fuer-grammatischen-Satz}使用图\ref{Abbildung-Strukturen-fuer-Fragen-und-RS}中的两棵子树的助词组合的正确结构的推导过程}
%\caption{\label{Abbildung-Kombination-fuer-grammatischen-Satz}Derivation of the correct structure for combination with an auxiliary using two subtrees from
%Figure~\ref{Abbildung-Strukturen-fuer-Fragen-und-RS}}
\end{figure}%
%
%
%
%
\begin{figure}
\hfill
\adjustbox{valign=c}{%
\begin{forest}
empty nodes
[X
	[X
	      [
        	[is\\
	\textsc{cop}]] ]
	[X
		[X]
		[X]]]
\end{forest}}
\hfill
$\circ$
\hfill
\adjustbox{valign=c}{%
\begin{forest}
[X
	[X
		[X
			[the\\
			\textsc{det},tier=word]]
		[X
			[man\\
			人,tier=word]]]
	[X
		[X
			[who\\
			\textsc{rel},tier=word]]
		[X]]]
\end{forest}}
\hfill
$\circ$
\hfill
\adjustbox{valign=c}{%
\begin{forest}
sn edges
[X
	[eating\\
	吃]]
\end{forest}}
\hfill
$\circ$
\hfill
\adjustbox{valign=c}{%
\begin{forest}
sn edges
[X
	[X
		[is\\
		\textsc{cop}]]
	[X
		[hungry\\
		 饿]]]
\end{forest}}
\hfill\mbox{}
\caption{\label{Abbildung-Kombination-fuer-ungrammatischen-Satz}使用图\ref{Abbildung-Strukturen-fuer-Fragen-und-RS}中的两棵子树的助词组合的错误结构的推导过程}
%\caption{\label{Abbildung-Kombination-fuer-ungrammatischen-Satz}Derivation of the incorrect structure
%for the combination with an auxiliary using two subtrees from Figure~\ref{Abbildung-Strukturen-fuer-Fragen-und-RS}}
\end{figure}%

我们总是选择那些包括最少子树的动机是因为我们对已知的材料进行最大化的类比。
%The motivation for always taking the derivation that consists of the least subparts is that one maximizes similarity to already known material.

包括一个助词的(\mex{1})的树也可以从只有两棵子树(带有[\sub{X} is\sub{X} X]的树和The man who is eating is hungry的整棵树)的图\ref{Abbildung-Strukturen-fuer-Fragen-und-RS}中得到。
%The tree for (\mex{1}) containing one auxiliary too many can also be created from Figure~\ref{Abbildung-Strukturen-fuer-Fragen-und-RS} with just two subtrees 
%(with the tree [\sub{X} is\sub{X} X] and the entire tree for \emph{The man who is eating is hungry}).
\ea[*]{
\gll Is the man who is eating is hungry?\\
\textsc{cop}/\textsc{aux} \textsc{det} 人 \textsc{rel} \textsc{aux} 吃 \textsc{cop} 饿\\
}
\z
有趣的是,儿童确实会造出这类错误的句子(\citealp[\page 530]{CN87a-u};\citealp*{ARP2008a})。
但是,如果我们考虑到子树加上所组合的部分的数量的概率,我们就会得到正确的结果,即(\ref{Bsp-Is-the-man-who-is-eating-hungry}),而不是(\mex{0})。
这是因为the man who is eating在语料库中出现了两次,一个是在(\ref{Bsp-Is-the-man-who-is-eating-hungry})中,一个是在(\ref{Bsp-The-man-who-is-eatin-mumbled})中。
所以,the man who is eating的概率跟the man who is eating is hungry的概率一样高,这样就得到图\ref{Abbildung-Kombination-fuer-grammatischen-Satz}中的推导式,而不是(\mex{0})中的推导式。
这适用于这里构建的例子,但是,我们可以想象在一个真实的语料库中,具有the man who is eating形式的序列比具有更多词的序列更为常用,因为the man who is eating也可以在其他语境中出现。
Bod将这个过程用到了成人语言的语料库(英语\il{English}、德语\il{German}和汉语\il{Mandarin Chinese})上,也用到了CHILDES语料库\isc{CHILDES语料库}\is{CHILDES}的Eve语料上,他这样做是为了考察类比构造\isc{类比}\is{analogy}是否构成了人类语言习得的模型\isc{习得}\is{acquisition}。他可以证明,我们上面展示的例子也适用于大规模的自然语言的语料库:尽管在Eve语料中没有跨越复杂NP的助词移位的例子,有可能通过类比来学会复杂NP内的助词是不能前置的。
%Interestingly, children do produce this kind of incorrect sentences (\citealp[\page 530]{CN87a-u}; \citealp*{ARP2008a}). 
%However, if we consider the probabilities of the subtrees in addition to the the number of combined
%subparts, we get the correct result, namely (\ref{Bsp-Is-the-man-who-is-eating-hungry}) and not (\mex{0}).
%This is due to the fact that \emph{the man who is eating} occurs in the corpus twice, in (\ref{Bsp-Is-the-man-who-is-eating-hungry}) and in
%(\ref{Bsp-The-man-who-is-eatin-mumbled}).
%Thus, the probability of \emph{the man who
%  is eating} is just as high as the probability of \emph{the man who is eating is hungry} and thus derivation in Figure~\ref{Abbildung-Kombination-fuer-grammatischen-Satz} 
%  is preferred over the one for (\mex{0}).
%This works for the constructed examples here, however one can imagine that in a realistic corpus, sequences of the form \emph{the man who is eating} are more frequent
%than sequences with further words since \emph{the man who is eating} can also occur in other contexts.
%Bod has applied this process to corpora of adult language (English\il{English}, German\il{German} and Chinese\il{Mandarin Chinese}) as well as
%applying it to the Eve corpus from the CHILDES database\is{CHILDES} in order to see whether analogy formation\is{analogy} constitutes a plausible model
%for human acquisition of language\is{acquisition}. He was able to show that what we demonstrated for
%the sentences above also works for a larger corpus of
%naturally occurring language: although there were no examples for movement of an auxiliary across a complex NP in the Eve corpus, it is possible to learn
%by analogy that the auxiliary from a complex NP cannot be fronted.

这样就有可能从没有关于词类或语言的抽象属性的先验知识的语料库中学会句法结构。
Bod所作的唯一一个假设是,(二叉)\isc{分支!二叉}\is{branching!binary}结构是存在的。二叉性的假设并不是必要的。但是如果我们将平铺结构囊括进计算中,树的集合就会变得相当大。这样,Bod在他的实验中只使用了二叉的结构。在他的树中,X包括两个其他的X's或一个词。我们就在分析循环\isc{循环}\is{recursion}结构。由此,Bod的工作提出了只需要循环的句法结构的习得理论,而循环被 \citet*{HCF2002a}看作是语言的基本属性。
%It is therefore possible to learn syntactic structures from a corpus without any prior knowledge
%about parts of speech or abstract properties of language.
%The only assumption that Bod makes is that there are (binary"=branching)\is{branching!binary} structures. The assumption of binarity is not really
%necessary. But if one includes flat branching structures into the computation, the set of trees will
%become considerably bigger. Therefore, Bod only used binary"=branching structures in his
%experiments. In his trees, X consists of two other X's or a word. We are therefore dealing with
%recursive\is{recursion} structures. Therefore, Bod's work proposes a theory of the acquisition of
%syntactic structures that only requires recursion, something that is viewed by  \citet*{HCF2002a} as a basic property of language.

正如在\ref{Abschnitt-Rekursion}展示的,有证据显示循环并不限于语言,这样我们可以总结出,为了能够从现有输入中学会句法结构,我们没有必要假设内在的语言学知识。
%As shown in Section~\ref{Abschnitt-Rekursion}, there is evidence that recursion is not restricted to language and thus one can conclude that it is not 
%necessary to assume innate linguistic knowledge in order to be able to learn syntactic structures from the given input.

尽管如此,这里有必要指出的是:Bod证明的是句法结构是可以学习的。在他的结构中还没有总结的所涵盖的每个词的词性信息也可以用统计\isc{统计学}\is{statistics}的方法推导出来\citep{RCF98a,Clark2000a}。\footnote{%
用来区分词类的计算语言学的算法是考察整个语料的。但是儿童总是处理其中的一部分。那么,相应的学习过程也一定会包括一个记忆的曲线。请参考 \citew[\page
  67]{Braine87a}。
}
%Nevertheless, it is important to point out something here: what Bod shows is that syntactic structures can be learned.
%The information about the parts of speech of each word involved which are not yet included in his structures can also be derived using
%statistical\is{statistics} methods \citep{RCF98a,Clark2000a}.\footnote{%
%	Computational linguistic algorithms for determining parts of speech often look at an entire corpus. But children are always
%	dealing with just a particular part of it. The corresponding learning process must then also include a
%	curve of forgetting. See  \citew[\page
 % 67]{Braine87a}. 
%} 
在所有的可能性中,可以学习的结构对应于表层导向的语言学理论也会假设的结构。但是,并不是所有的语言学分析都是必要的。在Bod的模型中,只考察了结构中词的出现情况。
没有说明词之间是否具有一个具体的常规关系(例如,连接被动分词和完成时分词的词汇规则)。进而,没有说明表达式的寓意是如何表示的(他们是按照构式语法所说的整体含义还是从词汇中投射的?)。还有跟理论语言学(请看第\ref{Abschnitt-Phrasal-Lexikalisch}章)有关的问题,并且不能直接从词的统计分布以及由其计算出的结构中推导出来(关于这点的更多内容请看\ref{Abschnitt-U-Dop-phrasal})。\isc{助动词倒装|)}\is{auxiliary inversion|)}
%In all probability, the structures that can be learned correspond to structures that surface"=oriented linguistic theories would also assume. However, not
%all aspects of the linguistic analysis are acquired. In Bod's model, only occurrences of words in structures are evaluated.
%Nothing is said about whether words stand in a particular regular relationship to one another or not (for example, a lexical rule connecting a passive
%participle and perfect participle). Furthermore, nothing is said about how the meaning of expressions arise (are they rather  holistic in the sense of Construction
%Grammar or projected from the lexicon?). These are questions that still concern theoretical linguists (see Chapter~\ref{Abschnitt-Phrasal-Lexikalisch}) 
%and cannot straightforwardly be derived from the statistic distribution of words and the structures computed from them (see Section~\ref{Abschnitt-U-Dop-phrasal}
%for more on this point).\is{auxiliary inversion|)}

还需要指出的是:我们已经看到统计信息可以用来推导出复杂的语言表达式的结构。现在,问题的实质是这跟乔姆斯基早期反对的统计学方法有什么关系(\citealp[\page 16]{Chomsky57a})。 \citet[\S~4.2]{Abney96a}详细地讨论了这点。乔姆斯基的早期观点的问题是Chomsky讨论的是马尔科夫模型\isc{马尔科夫模型}\is{Markov model}。这是有限状态自动机的统计版本。有限状态自动机\isc{自动机!有限状态自动机}\is{automaton!finite}只能描写3型语言\isc{复杂类型}\is{complexity class},这样就不适合分析自然语言了。但是,Chomsky的评论不能适用于普遍的统计学方法。\isc{统计学|)}\is{statistics|)}\isc{无监督的面向数据的句法分析(U-DOP)|)}\is{Unsupervised Data-Oriented Parsing (U-DOP)|)}
%A second comment is also needed: we have seen that statistical information can be used to derive the structure of complex linguistic expressions. This now
%begs the question of how this relates to Chomsky's earlier argumentation against statistical approaches
%(\citealp[\page 16]{Chomsky57a}).  \citet[Section~4.2]{Abney96a} discusses this in detail. The problem with his earlier argumentation is that Chomsky referred
 %to Markov models\is{Markov model}. These are statistical versions of finite automatons. Finite automatons\is{automaton!finite} can only describe
%type 3 languages\is{complexity class} and are therefore not appropriate for analyzing natural
%language. However, Chomsky's criticism cannot be applied to statistical methods
%in general.\is{statistics|)}\is{Unsupervised Data-Oriented Parsing (U-DOP)|)}

\subsection{负向证据}
%\subsection{Negative evidence}
\label{Abschnitt-negative-Evidenz}

在\isc{负向证据|(}\is{negative evidence|(}一些认为支持天赋知识的研究中,他们提出儿童没有接触到负向证据,即没有人告诉他们诸如(\ref{Hilfsverbinversion-ungrammatisch})\isc{助动词倒装}\is{auxiliary inversion}的句子——这里重复为(\mex{1})——是不合乎语法的(\citealp[\page 42--52]{BH70a};\citealp{Marcus93a})。
%In\is{negative evidence|(} a number of works that assume innate linguistic knowledge, it is claimed that children do not have access to negative evidence, that is,
%nobody tells them that sentences such as (\ref{Hilfsverbinversion-ungrammatisch})\is{auxiliary inversion} -- repeated here as (\mex{1})
%-- are ungrammatical (\citealp[\page 42--52]{BH70a}; \citealp{Marcus93a}). 
\ea[*]{\label{Hilfsverbinversion-ungrammatisch-zwei}
\gll Is the dog that in the corner is hungry?\\
\textsc{cop} \textsc{det} 狗 \textsc{rel} \textsc{prep} \textsc{det} 角落 \textsc{cop} 饿\\
}
\z
确实是,大人们不会用不合乎语法的句子来每天叫醒他们的孩子,但是,儿童实际上能够接触到不同类型的负向证据。例如, \citet{CC2003a}指出,说英语\il{English}和法语\il{French}的父母会纠正孩子们不合乎语法的句子。
例如,他们重复那些没有对动词正确变位的句子。儿童可以从重复的话语中和重复修正错误的变化中推导出事实,并且Chouinard和Clark也证实了,他们确实是这样做的。作者们观察了五个儿童的数据,他们的父母都有学术背景。他们还讨论了其它文化中父母和子女之间的关系(相关概述请参考 \citew{Ochs82a,OS85a}和 \citew[\page 71]{Marcus93a}),并且涉及了低社会经济阶层的美国家庭的情况(第660页)。
%It is indeed correct that adults do not wake up their children with the ungrammatical sentence of the day, however, children do in fact have access
%to negative evidence of various sorts. For example,  \citet{CC2003a} have shown that English\il{English} and French speaking\il{French} parents
%correct the utterances of their children that are not well"=formed.
%For example, they repeat utterances where the verb was inflected incorrectly. Children can deduce from the fact that the utterance was repeated and from what was changed
%in the repetition that they made a mistake and Chouinard and Clark also showed that they actually do this. The authors looked at data from five children whose
%parents all had an academic qualification. They discuss the parent"=child relationship in other cultures, too (see  \citew{Ochs82a,OS85a} and  \citew[\page 71]{Marcus93a}
%for an overview) and refer to studies of America families with lower socio"=economic status (page~660). 

负向证据的深层形式是间接的负向证据\isc{证据!负向证据!间接负向证据}\is{evidence!negative!indirect}, \citet[\page 9]{Chomsky81a}提出这也在习得中起到了作用。 \citet[\S~5.2]{Goldberg95a}举出(\mex{1}a)中的句子作为例子:\footnote{%
也请参考 \citew[\page 277]{Tomassello2006b-u}。
}
%A further form of negative evidence is indirect negative evidence\is{evidence!negative!indirect}, which  \citet[\page 9]{Chomsky81a} also assumes could play a role
%in acquisition.  \citet[Section~5.2]{Goldberg95a} gives the utterance in (\mex{1}a) as an example:\footnote{%
%Also, see  \citew[\page 277]{Tomassello2006b-u}.
%}
\eal
\ex[]{
\gll Look! The magician made the bird disappear.\\
看 \textsc{det} 魔术师 使得 \textsc{det} 鸟 消失\\
\mytrans{看!魔术师把鸟变没了。}
}
\ex[*]{
\gll The magician disappeared the bird.\\
\textsc{det} 魔术师 消失 \textsc{det} 鸟\\
}
\zl
儿童可以从成人使用包含make的更加复杂的致使结构这一现象得出以下结论:与其它动词(例如,melt)不同,动词disappear不能用作及物动词。间接负向证据起到作用的直接例子来自形态学。有一些产生式规则无论如何无法使用,如果有词限制\isc{限制}\is{blocking}了这个规则的应用的话。一个例子是德语中的名词化\isc{名词化}\is{nominalization}后缀\suffix{er}。通过在动词词干上加上\suffix{er},我们可以得到一个名词,它指(通常是习惯上)采取某种行动的人,Raucher(吸烟者)、Maler(画家)、Sänger(歌手)、Tänzer(舞者)。但是,Stehler(小偷)是非常特殊的。Stehler(小偷)的构成被Dieb(贼)的存在限制住了。由此,语言学习者必须从Stehler(小偷)的缺失中推导出的名词化规则并不适用于stehlen(偷)。
%The child can conclude from the fact that adults use a more involved causative construction with \emph{make}
%that the verb \emph{disappear}, unlike other verbs such as \emph{melt}, cannot be used transitively. 
%An immediately instructive example for the role played by indirect negative evidence comes from morphology.
%There are certain productive rules that can however still not be applied if there is a word that blocks\is{blocking}
%the application of the rule. An example is the \suffix{er} nominalization\is{nominalization} suffix in German.
%By adding an \suffix{er} to a verb stem, one can derive a noun that refers to someone who carries out a particular
%action (often habitually) (\emph{Raucher} `smoker', \emph{Maler} `painter', \emph{Sänger} `singer', \emph{Tänzer} `dancer').
%However, \emph{Stehler} `stealer' is very unusual. The formation of \emph{Stehler} is blocked by the existence of \emph{Dieb} `thief'.
%Language learners therefore have to infer from the non"=existence of \emph{Stehler} that the nominalization rule does not apply to \emph{stehlen} `to steal'.

相似地,对于方式副词的位置没有任何限制的英语\il{English}语法而言,具有这种知识的说话者会认为(\mex{1})中的所有语序都是可能的\citep[\page 206]{SP2002b}:
%Similarly, a speaker with a grammar of English\il{English} that does not have any restrictions on
%the position of manner adverbs would expect that both orders in (\mex{1}) are possible \citep[\page 206]{SP2002b}:
\eal
\ex[]{
\gll call the police immediately\\
叫 \textsc{det} 警察 马上\\
\mytrans{马上叫警察}
}
\ex[*]{
\gll call immediately the police\\
叫 马上 \textsc{det} 警察\\
}
\zl
学习者可以从这样的事实中间接地得出结论,诸如(\mex{0}b)的动词短语(几乎)从不出现在不属于这个语言的输入中。这可以用相关的统计学习的算法进行模拟。
%Learners can conclude indirectly from the fact that verb phrases such as (\mex{0}b) (almost) never occur in the input that these are probably not part
%of the language. This can be modeled using the relevant statistical learning algorithms.

截至目前,负向证据提供的例子更多的是貌似正确的论断。
 \citet{Stefanowitsch2008a}\isc{统计学|(}\is{statistics|(}将语料库语言学\isc{语料库语言学}\is{corpus linguistics}的研究整合进可接受性实验的统计分布中,并且证明了所期待的频率中获得的负向证据跟说话者的可接受性的判断有关。我们在下面简短地讨论一下这个过程:Stefanowitsch提出了下面的原则:
%The examples for the existence of negative evidence provided so far are more arguments from plausibility.
% \citet{Stefanowitsch2008a}\is{statistics|(} has combined corpus linguistic\is{corpus linguistics} studies
%on the statistical distribution with acceptability experiments and has shown that negative evidence gained from
%expected frequencies correlates with acceptability judgments of speakers. This process will be discussed now briefly: Stefanowitsch
%assumes the following principle:
\ea
根据语言特征或要素出现的个体频率来构造出他们的共现频率规律,并且根据共现的实际频率来核查这些规律。\citep[\page 518]{Stefanowitsch2008a}
%Form expectations about the frequency of co"=occurrence of linguistic features or elements on the basis of their individual frequency of occurrence
%and check these expectations against the actual frequency of co"=occurrence. \citep[\page 518]{Stefanowitsch2008a}
\z
Stefanowitsch研究的是包括英式英语的英语国际语料库(International Corpus of English\il{English},ICE-GB)。在这个语料库中,动词say出现了3333次,带有双及物动词的句子(Subj Verb Obj Obj)出现了1824次。数据库中动词的总数达136551个。如果所有的动词在所有类型的句子中以相同的频率出现的话,那么我们会希望得到这样的结果,say在双及物构式中出现了44.52次(X / 1,824 = 3,333 / 136,551 所以 X = 1,824 $\times$ 3,333 / 136,551)。但是,这个表达式出现的实际数量是0,跟(\mex{1}b)不同,说英语的人不适用于(\mex{1}a)这样的句子。
%Stefanowitsch works with the part of the \emph{International Corpus of English} that contains British English\il{English} (ICE-GB). In this corpus, the
%verb \emph{say} occurs 3,333 times and sentences with ditransitive verbs (Subj Verb Obj Obj) occur 1,824 times. The entire total of verbs in the corpus
%is 136,551. If all verbs occurred in all kinds of sentences with the same frequencies, then we would expect \emph{say} to occur 44.52 times
%(X / 1,824 = 3,333 / 136,551 and hence X = 1,824 $\times$ 3,333 / 136,551) in the ditransitive construction. But the number of actual
%occurrences is actually 0 since, unlike (\mex{1}b), sentences such as (\mex{1}a) are not used by
%speakers of English. 
\eal
\ex[*]{
\gll Dad said Sue something nice.\\
爸爸 说 Sue 某事 好\\
}
\ex[]{
\gll Dad said something nice to Sue.\\
爸爸 说 某事 好 \textsc{prep} Sue\\
\mytrans{爸爸跟Sue说了某件好事。}
}
\zl

Stefanowitsch证明了双及物句式中say没有出现是非常重要的。进而,他考察了可接受性判断是如何跟特定构式中动词不出现的频率相比较的。在第一个实验中,他能够证明特殊构式中不出现要素的频率与说话者的可接受性判断有关,而跟构式中动词的出现频率无关。\isc{统计学|)}\is{statistics|)}
%Stefanowitsch shows that the non"=occurrence of \emph{say} in the ditransitive sentence pattern is significant. Furthermore, he investigated how acceptability
%judgments compare to the frequent occurrence or non"=occurrence of verbs in certain constructions. 
%In a first experiment, he was able to show that the frequent non"=occurrence of elements in particular constructions correlates with the acceptability judgments of speakers, whereas
%this is not the case for the frequent occurrence of a verb in a construction.\is{statistics|)}

总之,我们可以说间接的负向证据可以从语言输入中推导出来,而且它在语言习得中起到了重要的作用。
%In sum, we can say that indirect negative evidence can be derived from linguistic input and that it
%seems to play an important role in language acquisition.
\isc{负向证据|)}\is{negative evidence|)}% 

\section{总结}
%\section{Summary}

综上所述,没有一个支持内在的语言知识的论断是没有争议的。
这当然不会排除有内在的语言知识的可能性,但是那些希望将这个假说整合进他们的理论的学者们要比之前想要证明他们假设的内在性真正属于我们的语言知识并且能够只从语言输入学习时更为小心了。
%It follows from all this that not a single one of the arguments in favor of innate linguistic knowledge remains uncontroversial.
%This of course does not rule out there still being innate linguistic knowledge but those who wish to incorporate
%this assumption into their theories have to take more care than was previously the case to prove that what they assume to be innate
%is actually part of our linguistic knowledge and that it cannot be learned from the linguistic input alone.%
\isc{刺激贫乏论|)}\is{Poverty of the Stimulus|)}%
\isc{习得|)}\is{acquisition|)}


%\section*{思考题}
%\section*{Comprehension questions}

%\bigskip
\pagebreak
%~\newline\vspace*{-9mm}
\questions{
\begin{enumerate}
\item 哪些学说假设了天赋的语言学知识?
%Which arguments are there for the assumption of innate linguistic knowledge?
\end{enumerate} 
}


%\section*{延伸阅读}
%\section*{Further reading}

\furtherreading{
Pinker的\citeyearpar{Pinker94a}这本书是关于语言的天赋模型的最好的一本书。
%Pinker's book \citeyearpar{Pinker94a} is the best written book arguing for nativist models of language.

 \citet*{EBJKSPP96a}讨论了支持天赋语言知识的所有观点,并且证明了相关的现象可以有不同的解释。作者们采用了联结主义的观点。他们用神经网络进行研究,神经网络被认为是相对准确地模拟我们大脑工作过程的方法。这本书也包括了遗传学的基本知识和大脑结构的章节内容,并深入讨论了为什么将语言知识直接编码进我们的基因组是不可能的。
% \citet*{EBJKSPP96a} discuss all the arguments that have been proposed in favor of innate linguistic knowledge and show
%that the relevant phenomena can be explained differently. The authors adopt a connectionist view. They work with neuronal
%networks, which are assumed to model what is happening in our brains relatively accurately.
%The book also contains chapters about the basics of genetics and the structure of the brain, going into detail about why
%a direct encoding of linguistic knowledge in our genome is implausible. 

有些使用了神经网络的方法遭到了批评,这是因为他们无法捕捉到人类能力的某些方法,如循环或者话语中相同词的多次使用。
  \citet{Pulvermueller2010a}讨论了一个具有记忆的架构,并且使用它来分析循环结构。在他的概述性文章中,引用了一些研究来证明许多更为抽象的规则或理论语言学中认为理所当然的模式都可以在神经层面进行证明。但是,Pulvermüller并不认为语言知识是内在的(第173页)。
%Certain approaches using neuronal networks have been criticized because they cannot capture certain aspects of human abilities
%such as recursion or the multiple usage of the same words in an utterance.
%  \citet{Pulvermueller2010a} discusses an architecture that has memory and uses this to analyze recursive structures. In his overview article,
% certain works are cited that show that the existence of more abstract rules or schemata of the kind theoretical linguists take for granted
% can be demonstrated on the neuronal level. Pulvermüller does not, however, assume that linguistic knowledge is innate (p.\,173).

Pullum和Scholz详细地分析了刺激贫乏论\citep{PS2002a,SP2002b}。
%Pullum and Scholz have dealt with the Poverty"=of"=the"=Stimulus argument in detail
% \citep{PS2002a,SP2002b}.

 \citet{Goldberg2006a}和 \citet{Tomasello2003a}是最为著名的构式语法学家,构式语法明确地不支持天赋语言知识的假说。
% \citet{Goldberg2006a} and  \citet{Tomasello2003a} are the most prominent proponents of Construction Grammar, a theory that explicitly tries
%to do without the assumption of innate linguistic knowledge.
}

% lulu/wsun/sisi: DONE
%      <!-- Local IspellDict: en_US-w_accents -->
