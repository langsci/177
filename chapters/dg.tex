%% -*- coding:utf-8 -*-
\chapter{依存语法}
%\chapter{Dependency Grammar}
\label{Kapitel-DG}

依存语法是本书介绍的最古老的一种语法理论。它的现代版本是由法国语言学家Lucien Tesnière(1893--1954)进一步发展而来的。他的奠基性著作《结构句法基础》(\emph{Eléments de syntaxe structurale})是在1938年完成的,这是在Ajdukiewicz发表范畴语法的论文的三年后\citeyearpar{Ajdukiewicz35a-u},但是真正的出版被推迟到1959年,也就是在他去世五年后\nocite{Tesniere59a-u}。因为配价是依存语法的核心,它有时也叫做配价语法。Tesnière的思想在今天广为流传。几乎所有的现代理论都有配价和依存的概念\citep[\page 262--263, 284]{AF2010a}。
%Dependency Grammar is the oldest framework described in this book. Its modern version was developed by %the
%French linguist Lucien Tesnière (1893--1954). His foundational work \emph{Eléments de syntaxe
%  structurale} `Elements of structural syntax' was basically finished in 1938 only three years after
%Ajdukiewicz's paper on \cg \citeyearpar{Ajdukiewicz35a-u}, but the publication
%was delayed until 1959, five years after his death\nocite{Tesniere59a-u}. Since valence is central
%in Dependency Grammar, it is sometimes also referred to as Valence Grammar. 
%Tesnière's ideas are wide-spread nowadays. The conceptions of valence and dependency are present in
%almost all of the current theories \citep[\page 262--263, 284]{AF2010a}.
% zitieren Schmidt

尽管有一些研究英语的文献\citep{Anderson71a-u,Hudson84a-u},依存语法在欧洲中部更为流行,尤其是在德国\citep[\page 56--57]{Engel96a}。 \citet[\page 250]{AF2010a}指出了一个可能的原因:\tes 的原始著作直到最近才有英译本\citep{Tesniere2015a-not-crossreferenced},但是德语译本都已经有35年了\citep{Tesniere80a-u}。由于依存语法的重点在于依存关系,而不是成分的线性顺序,通常认为它更适于分析自由语序的语言,这也是它在斯拉夫语\il{Slavic}的研究中比较流行的一个原因:在1960年开始的以Sgall、Hajičová和Panevova代表的新布拉格学派\isc{新布拉格学派}\is{New Prague School}深入发展了依存语法(全面的信息请参考\citealp{HS2003a-u})。Igor\,A.\ Meľčuk和A.\,K.\ Žolkovskij在1960年的苏联开始研究一个叫做\mttc 的模型,它也被用于机器翻译项目中\citep{Melcuk64a-u,Melcuk81a,Melcuk88a-u,Kahane2003a-u}。1970年,\mel 离开了苏联,前往加拿大,现在蒙特利尔工作。
%Although there is some work on English \citep{Anderson71a-u,Hudson84a-u}, Dependency Grammar is most popular in central Europe and especially %so in Germany \citep[\page
%  56--57]{Engel96a}.  \citet[\page 250]{AF2010a} identified a possible reason for this: \tes's
%original work was not available in English until very recently \citep{Tesniere2015a-not-crossreferenced}, but there has
%been a German translation for more than 35 years now \citep{Tesniere80a-u}. Since Dependency Grammar focuses on dependency relations rather %than
%linearization of constituents, it is often felt to be more appropriate for languages with freer
%constituent order, which is one reason for its popularity among researchers working on Slavic\il{Slavic}
%languages: the New Prague School\is{New Prague School} represented by Sgall, Hajičová and Panevova developed Dependency Grammar further,
%beginning in the 1960s (see \citealp{HS2003a-u} for an overview).  Igor\,A.\ Meľčuk and
%A.\,K.\ Žolkovskij started in the 1960s in the Soviet Union to work on a model called \mtt, which was also used in machine
%translation projects \citep{Melcuk64a-u,Melcuk81a,Melcuk88a-u,Kahane2003a-u}. \mel left the
%Soviet Union towards Canada in the 1970s and now works in Montréal. 

依存语法在德国及世界范围内的学者中广为流传。它在德语作为外语的教学中非常成功\citep{HB69a-u,HB98a}。在东德的莱比锡工作的Helbig和Buscha开始编纂配价辞典\citep{HS69a-u} ,之后的研究者在曼海姆的德语语法研究所(Institut für Deutsche Sprache)工作,他们也展开了类似的辞典编纂工程\citep{SKSR2004a-u}。
%Dependency Grammar is very wide-spread in Germany and among scholars of German linguistics
%worldwide. It is used very successfully for teaching German as a foreign language
%\citep{HB69a-u,HB98a}. Helbig and Buscha, who worked in Leipzig, East Germany, started to
%compile valence dictionaries \citep{HS69a-u} and later researchers working at the Institut für
%Deutsche Sprache (Institute for German Language) in Mannheim began similar lexicographic projects \citep{SKSR2004a-u}. 

下面列出的是可能在德国工作或工作过的语言学家的不完整列表:
%The following enumeration is a probably incomplete list of linguists who are/were based in Germany: 
Vilmos  \citet{Agel2000a-u}, 卡塞尔;
Klaus  \citet{Baumgaertner65a-u,Baumgaertner70a}, 莱比锡 之后是 斯图加特;
%Bernd Bohnet, Computer Science Stuttgart;
Ulrich  \citet{Engel77,Engel2014a}, IDS 曼海姆; 
Hans-Werner  \citet{Eroms85a,Eroms87b-u,Eroms2000a}, 帕绍; 
Heinz Happ, 图宾根;
Peter  \citet{Hellwig78a-u,Hellwig2003a}, 海德堡;
Jürgen  \citet{Heringer96a-u}, 奥格斯堡; 
Jürgen  \citet{Kunze68a-u,Kunze75a-u}, 柏林;
Henning  \citet{Lobin93a-u}, 基森;
Klaus  \citet{Schubert87a-u}, 希尔德斯海姆;
Heinz Josef  \citet{Weber97a}, 特里尔;
Klaus  \citet{Welke88a-u,Welke2011a-u}, 洪堡大学;
Edeltraud  \citet{Werner93a-u}, 哈雷-维滕贝格。
%\pagebreak
%Vilmos  \citet{Agel2000a-u}, Kassel; 
%Klaus  \citet{Baumgaertner65a-u,Baumgaertner70a}, Leipzig later Stuttgart;
%Bernd Bohnet, Computer Science Stuttgart;
%Ulrich  \citet{Engel77,Engel2014a}, IDS Mannheim; 
%Hans-Werner  \citet{Eroms85a,Eroms87b-u,Eroms2000a}, Passau; 
%Heinz Happ, Tübingen;
%Peter  \citet{Hellwig78a-u,Hellwig2003a}, Heidelberg;
%Jürgen  \citet{Heringer96a-u}, Augsburg; 
%Jürgen  \citet{Kunze68a-u,Kunze75a-u}, Berlin;
%Henning  \citet{Lobin93a-u}, Gießen;
%Klaus  \citet{Schubert87a-u}, Hildesheim;
%Heinz Josef  \citet{Weber97a}, Trier;
%Klaus  \citet{Welke88a-u,Welke2011a-u}, Humboldt University Berlin;
%Edeltraud  \citet{Werner93a-u}, Halle-Wittenberg.\pagebreak

尽管从1959年起的连续几十年间,许多国家都有相关的研究,但是阶段性的全球性会议直到2011年才开始举办。\footnote{%
  \href{http://depling.org/}{http://depling.org/}. \zhdate{2015/04/10}。
%  \url{http://depling.org/}. 10.04.2015.
}$^,$\footnote{%
  \mtt 的会议从2003年开始每两年举办一次。
}
%Although work has been done in many countries and continuously over the decades since 1959, a
%periodical international conference was established as late as 2011.\footnote{%
% \url is replaced here, since it does cruel things to (our) fonts, which means that the first footnote number
% comes out too high. 04.03.2016 bug 
%  \href{http://depling.org/}{http://depling.org/}. 10.04.2015.
%  \url{http://depling.org/}. 10.04.2015.
%}$^,$\footnote{%
%  A conference on \mtt has taken place biannually since 2003.
%}

依存语法很早就被用于计算项目中。Meľčuk在苏联研究机器翻译,而David G.\ Hays在美国研究机器翻译。Jürgen Kunze在东德的德国科学院任计算语言学主席,他从1960年也开始机器翻译的研究。 \citew{Kunze75a-u}这本书讲述了语言学研究的形式化背景。还有许多研究者从1973年到1986年在萨尔布吕肯的合作研究中心100-电子语言学研究(SFB 100, Elektronische Sprachforschung)工作。这个研究中心的主要内容也是机器翻译。这些工程研究了俄语\il{Russian}到德语、法语\il{French}到德语、英语\il{English}到德语以及世界语到德语的翻译。对于这一范围内萨尔布吕肯的工作,请参阅 \citew{Klein71a-u}、 \citew{Rothkegel76a-u}和 \citew{Weissgerber83a-u}。 \citet{MIF85a}在一个分析日语和生成英语的工程中使用了依存语法。Richard Hudson从1980年开始研究基于依存语法框架的词语法\indexwgc\citep{Hudson84a-u,Hudson2007a-u} ,而Sleator和Temperly从1990年开始研究链语法\isc{链语法}\is{Link Grammar}\citep{ST91a-u,GLS95a-u}。Fred Karlsson的约束语法\citeyearpar{Karlsson90a-u}被应用到许多语言中(可用的大规模的语法片段有丹麦语\il{Danish}、葡萄牙语\il{Portuguese}、西班牙语\il{Spanish}、英语\il{English}、瑞典语\il{Swedish}、挪威语\il{Norwegian}、法语\il{French}、德语\il{German}、世界语\il{Esperanto}、意大利语\il{Italian}和荷兰语\il{Dutch})和学校的教学、语料库标注\isc{语料库标注}\is{corpus annotation}和机器翻译\isc{机器翻译}\is{machine translation}中。在项目网站上可以观看在线的演示视频。\footnote{%
  \url{http://beta.visl.sdu.dk/constraint_grammar}. \zhdate{2015/07/24}.
}

%From early on, Dependency Grammar was used in computational projects. Meľčuk worked on machine
%translation in the Soviet Union \citep{Melcuk64a-u} and David G.\ Hays worked on machine translation
%in the United States \citep{HZ60a-u}. Jürgen Kunze, based in East Berlin at the German Academy of
%Sciences, where he had a chair for computational linguistics, also started to work on machine
%translation in the 1960s. A book that describes the formal background of the linguistic work was
%published as  \citew{Kunze75a-u}.  Various researchers worked in the Collaborative Research Center
%100 \emph{Electronic linguistic research} (SFB 100, Elektronische Sprachforschung) from 1973--1986
%in Saarbrücken. The main topic of this SFB was machine translation as well. There were projects on
%Russian\il{Russian} to German, French\il{French} to German, English\il{English} to German, and
%Esperanto to German translation. For work from Saarbrücken in
%this context see  \citew{Klein71a-u},  \citew{Rothkegel76a-u}, and  \citew{Weissgerber83a-u}.
% \citet{MIF85a} used Dependency Grammar in a project that analyzed Japanese\il{Japanese} and
%generated English\il{English}.
%Richard Hudson started to work in a dependency grammar-based framework called Word Grammar\indexwg
%in the 1980s \citep{Hudson84a-u,Hudson2007a-u} and Sleator and Temperly have been working on Link
%Grammar\is{Link Grammar} since the 1990s \citep{ST91a-u,GLS95a-u}.
%Fred Karlsson's Constraint Grammars \citeyearpar{Karlsson90a-u} are developed for many languages (bigger fragments are available
%for Danish\il{Danish}, Portuguese\il{Portuguese}, Spanish\il{Spanish}, English\il{English}, Swedish\il{Swedish}, Norwegian\il{Norwegian}, French%\il{French}, German\il{German}, Esperanto\il{Esperanto}, Italian\il{Italian}, and
%Dutch\il{Dutch}) and are used for school teaching,
%corpus annotation\is{corpus annotation} and machine translation\is{machine translation}. An online
%demo is available at the project website.\footnote{%
%  \url{http://beta.visl.sdu.dk/constraint_grammar}. 24.07.2015.
%}
%\todostefan{SK:  \citew{KP91a-u,IKKLP92a-u,Coch96a}} 
% Hays 1964
% Melcuk Machine Translation, Coch ist auch MTT
% Somers86a-u

% LR87a Englisch (und Französisch) nur Generierung, keine Fernabhängigkeiten
% Coch96a MTT, 
% IKKLP92a French English, MTT

% Baumgärtner, Heringer, Kunze



% Eroms2003a-u zur Geschichte in Deutschland

%% wikipedia: whose most distinctive characteristic is its use of dependency grammar, an approach to
%% syntax in which the sentence's structure is almost entirely contained in the information about
%% individual words, and syntax is seen as consisting primarily of principles for combining words. The
%% central syntactic relation is that of dependency between words; constituent structure is not
%% recognized except in the special case of coordinate structures.
%
%% wikipedia: Word grammar is an example of cognitive linguistics, which models language as part of general knowledge and not as a specialized mental faculty.[1] This is in contrast to the nativism of Noam Chomsky and his students.

%Osborne 23.11.2009
%The dependency grammar I am developing overlaps in important areas with HPSG.  It is nonderivational, monostratal, and strongly lexical.  My interest at present concerning HPSG is to determine the extent to which the HPSG understanding of the lexicon can be adopted as a basis for my dependency grammar.  The notion of lexical rules that relate lexical entries to each other seems promising to me. 


近年来,依存语法在计算语言学中越来越受到欢迎。原因是许多标注语料库(树库)都包括依存信息。\footnote{%
根据 \citet{Kay2000a-u},Hays开发的史上第一个树库就标注了依存关系。}统计分析器可以在这样的树库中进行训练\citep{YM2003a-u,Attardi2006a-u,Nivre2003a-u,KMcDN2009a-u,Bohnet2010a-u}。许多句法剖析器都适用于多种语言,因为它们采用的一般方法是独立于语言的。对依存关系进行一致的标注是较为容易的,因为分析的可能性是比较少的。
%In recent years, Dependency Grammar became more and more popular among computational linguists. The
%reason for this is that there are many annotated corpora (tree banks) that contain dependency
%information.\footnote{%
%  According to  \citet{Kay2000a-u}, the first treebank ever was developed by Hays and did
%  annotate dependencies.
%} Statistical parsers are trained on such tree banks \citep{YM2003a-u,Attardi2006a-u,Nivre2003a-u,KMcDN2009a-u,Bohnet2010a-u}. Many of
%the parsers work for multiple languages since the general approach is language independent. It is
%easier to annotate dependencies consistently since there are fewer possibilities to do
%so.
%% \todostefan{S: also because it is closer to semantics and clearly lexicalized (see Kahane 2000, introduction to TAL, written before the swing)
%% and it is easier to evaluate (see UAS and LAS)}
支持基于短语成分模型的句法学家会关心以下一些区别:二叉的分支结构还是平铺模型,附加语是高附加性还是低附加性,空成分的有无等等,并且会对这些问题进行激烈的讨论。与之相对,话语中的依存关系是比较清楚的。由此,我们可以很容易地保持标注的一致性,并在这些标注的语料上训练统计剖析器。
%While
%syntacticians working in constituency"=based models may assume binary branching or flat models, high
%or low attachment of adjuncts, empty elements or no empty elements and argue fiercely about this,
%it is fairly clear what the dependencies in an utterance are. Therefore it is easy to annotate
%consistently and train statistical parsers on such annotated data.

除了统计模型,还有所谓的深层处理系统,这些系统是依赖于手工构造的、语言学驱动的语法。我已经提到了Meľčuk在机器翻译方面做了一些工作; \citet{HZ60a-u}开发了针对俄语\il{Russian}的剖析器; \citet{SN86a}开发了一个使用英语\il{English}语法的句法剖析器, \citet*{JLV86a-u}开发了一个用芬兰语\il{Finnish}演示的剖析器, \citet{Hellwig86a-u,Hellwig2003a,Hellwig2006a}开发了英语\il{English}的词语法, \citet{Covington90a}开发了俄语\il{Russian}和拉丁语\il{Latin}的剖析器,这个系统可以剖析非连续的成分,还有 \citet{Menzel98a-u}实现了一个高质量的德语依存语法的剖析器。 \citet{Kettunen86a-u}、 \citet{Lehtola86a-u}和 \citet{MS98a-u}还提及了其他计算剖析器。下面列出了有依存语法的语法片段的若干种语言:
%Apart from statistical modeling, there are also so-called deep processing systems, that is, systems
%that rely on a hand-crafted, linguistically motivated grammar. I already mentioned Meľčuk's work in
%the context of machine translation;  \citet{HZ60a-u} had a parser for Russian\il{Russian};
% \citet{SN86a} developed a parser that was used with an English\il{English}
%grammar,  \citet*{JLV86a-u} developed a parser that was demoed with Finnish\il{Finnish},  \citet{Hellwig86a-u,Hellwig2003a,Hellwig2006a}
%implemented grammars of German in the framework of Dependency Unification Grammar,  \citet{Hudson89a}
%developed a Word Grammar for English\il{English},
% \citet{Covington90a} developed a parser for Russian\il{Russian} and Latin\il{Latin}, which can parse discontinuous constituents, and
% \citet{Menzel98a-u} implemented a robust parser of a Dependency Grammar of German.
%Other work on computational parsing to be mentioned is
% \citew*{Kettunen86a-u,Lehtola86a-u,MS98a-u}.
%The following is a list of languages for which Dependency Grammar
%fragments exist:
% Melcuk Machine Translation, Coch ist auch MTT
% Somers86a-u
% Baumgärtner, Heringer, Kunze

\begin{itemize}
\item 丹麦语(Danish)\il{丹麦语}       \citep{Bick2001a-u,BN2007a-u}
\item 英语(English)\il{英语}     \citep{MIF85a,SN86a,LR87a,Hudson89a,ST91a-u,VHA92a-u,IKKLP92a-u,Coch96a}
\item 世界语(Esperanto)\il{世界语} \citep{Bick2009a-u} 
\item 爱沙尼亚语(Estonian)\il{爱沙尼亚语}   \citep*{Mueuerisep99a-u,MPMKRU2003a-u}
\item 法罗语(Faroese)\il{法罗语}     \citep{Trosterud2009a-u}
\item 芬兰语(Finnish)\il{芬兰语}     \citep*{NJL84a-u,JLV86a-u}
\item 法语(French)\il{法语}       \citep{IKKLP92a-u,Coch96a,Bick2010a-u}
\item 德语(German)\il{德语}       \citep{Hellwig86a-u,Coch96a,HKMS98a-u,MS98c-u,Hellwig2003a,Hellwig2006a,GK2001a}
\item 爱尔兰语(Irish)\il{爱尔兰语}         \citep{DvG2006a-u}
\item 日语(Japanese)\il{日语}   \citep*{MIF85a}
\item 拉丁语(Latin)\il{拉丁语}         \citep{Covington90a}
\item 现代汉语(Mandarin Chinese)\il{现代汉语} \citep{LW2006a-u,Liu2009a-u}
\item 挪威语(Norwegian)\il{挪威语}               \citep*{HBN2000a-u},
\item 古冰岛语(Old Icelandic)\il{古冰岛语}       \citep{Maas77a}
\item 葡萄牙语(Portuguese)\il{葡萄牙语}             \citep{Bick2003a-u} 
\item 俄语(Russian)\il{俄语}                   \citep{HZ60a-u,Melcuk64a-u,Covington90a}
\item 西班牙语(Spanish)\il{西班牙语}                   \citep{Coch96a,Bick2006a-u}
\item 斯瓦希里语(Swahili)\il{斯瓦希里语}                   \citep{Hurskainen2006a-u}
%\item Danish\il{Danish}       \citep{Bick2001a-u,BN2007a-u}
%\item English\il{English}     \citep{MIF85a,SN86a,LR87a,Hudson89a,ST91a-u,VHA92a-u,IKKLP92a-u,Coch96a}
%\item Esperanto\il{Esperanto} \citep{Bick2009a-u} 
%\item Estonian\il{Estonian}   \citep*{Mueuerisep99a-u,MPMKRU2003a-u}
%\item Faroese\il{Faroese}     \citep{Trosterud2009a-u}
%\item Finnish\il{Finnish}     \citep*{NJL84a-u,JLV86a-u}
%\item French\il{French}       \citep{IKKLP92a-u,Coch96a,Bick2010a-u}
%\item German\il{German}       \citep{Hellwig86a-u,Coch96a,HKMS98a-u,MS98c-u,Hellwig2003a,Hellwig2006a,GK2001a}
%\item Irish\il{Irish}         \citep{DvG2006a-u}
%\item Japanese\il{Japanese}   \citep*{MIF85a}
%\item Latin\il{Latin}         \citep{Covington90a}
%\item Mandarin Chinese\il{Mandarin Chinese} \citep{LW2006a-u,Liu2009a-u}
%\item Norwegian\il{Norwegian}               \citep*{HBN2000a-u},
%\item Old Icelandic\il{Old Icelandic}       \citep{Maas77a}
%\item Portuguese\il{Portuguese}             \citep{Bick2003a-u} 
%\item Russian\il{Russian}                   \citep{HZ60a-u,Melcuk64a-u,Covington90a}
%\item Spanish\il{Spanish}                   \citep{Coch96a,Bick2006a-u}
%\item Swahili\il{Swahili}                   \citep{Hurskainen2006a-u}
\end{itemize}
约束语法的网页\footnote{%
  \url{http://beta.visl.sdu.dk/constraint_grammar_languages.html}
}另外列出的还有巴斯克语(Basque)\il{巴斯克语}、加泰罗尼亚语(Catalan)\il{加泰罗尼亚语}、英语(English)、芬兰语(Finnish)\il{芬兰语}、德语(German)\il{德语}、意大利语(Italian)\il{意大利语}、萨米语(Sami)\il{萨米语}和瑞典语(Swedish)\il{瑞典语}的语法。
%The Constraint Grammar webpage\footnote{%
%  \url{http://beta.visl.sdu.dk/constraint_grammar_languages.html}
%} additionally lists grammars for
%Basque\il{Basque},
%Catalan\il{Catalan},
%English,
%Finnish\il{Finnish},
%German\il{German},
%Italian\il{Italian},
%Sami\il{Sami}, and
%Swedish\il{Swedish}.
%
%Arppe, Antti (2000). "Developing a grammar checker for Swedish". In: Nordgård, T. (ed.) Nodalida'99 Proceedings. Department of Linguistics, University of Trondheim. pp. 13-27.
% Coch96a AlethGen uses MTT and generates English, French, Spanish, German
% KP91a   MTT, Paper fehlt
% IKKLP92a English, French
% LR87a Englisch (und Französisch) nur Generierung, keine Fernabhängigkeiten
% Coch96a MTT, 
% IKKLP92a French English, MTT
\LATER{Maybe it is worthy to add the following sentences in this section: 

还有基于依存树库的一些语言学研究。这些研究通过基于经验主义的数据的分析探讨了句法上的问题。
%Some linguistic studies based on dependency treebanks were also done. These studies begin to explore syntactic problems by empirical data.  

Liu, Haitao. Probability distribution of dependency distance. Glottometrics 15, 2007, 1-12. 
Liu, Haitao. The complexity of Chinese dependency syntactic networks. Physica A 387 (2008) 3048-3058. 
Liu, Haitao & Hu Fengguo. What role does syntax play in a language network? EPL (Europhysics Letters), 83 (2008) 18002.  
Liu, Haitao. Dependency distance as a metric of language comprehension difficulty. Journal of Cognitive Science. 2008, 9(2):159-191.
Liu, Haitao. Probability Distribution of Dependencies based on Chinese Dependency Treebank. Journal of Quantitative Linguistics. 2009, 16 (3): 256–273
Liu, Haitao, Richard Hudson, Zhiwei Feng. Using a Chinese treebank to measure dependency distance. Corpus Linguistics and Linguistic Theory. 2009, 5(2): 161-174. 
Liu, Haitao. Dependency direction as a means of word-order typology a method based on dependency treebanks. Lingua. 2010, 120(6): 1567-1578.
Liu, Haitao. Quantitative properties of English verb valency. Journal of Quantitative Linguistics. 2011, 18(3): 207-233.
Syntactic Variation in Chinese-English Code-switching. Lingua. 2013, (1): 58-73.
Liu, Haitao & Xu Chunshan. Quantitative typological analysis of Romance languages. Poznań Studies in Contemporary Linguistics. 2012, 48(4): 597-625. 
Liu, Haitao & Cong Jing. Empirical Characterization of Modern Chinese as a Multi-level System from the Complex Network Approach. Journal of Chinese Linguistics. 2014 (1): 1-38.
Jingyang Jiang and Haitao Liu*. The Effects of Sentence Length on Dependency Distance, Dependency Direction and the Implications - Based on a Parallel English-Chinese Dependency Treebank. Language Sciences. 2015, 50: 93-104.
}


% \citet{Starosta2003b-u} Lexicase Grammar aber keine Referenzen


%% Kim:
%% Tree banks = dependency tree banks

%% Statistische Systeme brauchen zum Lernen kohärente Analysen
%% Wenn arbiträre Strukturen gelernt werden hat der statistische Parser keine Chance
%% Phrasenstrukturfragen: 
%% VP vs flache Strukturen, binäre vs. flach, rechts- oder linksverzweigend
%% Als Zwischenstop zwischen Bedeutung und Text hat Dependenz weniger Überhang


%% \begin{itemize}

%% \item Menzel

%% \citep{YM2003a}

%% \citep{Attardi2006a-u}    DeSR A statistical shift/reduce dependency parser

%% \citep{Nivre2003a-u}    MaltParser A system for data-driven dependency parsing
%%     MST Parser A non-projective dependency parser that searches for maximum spanning trees over directed graphs
%%     Mate Parser Joint non-projective labeled dependency parser and part-of-speech tagger
%%     MST Parser (C\#) A non-projective dependency parser that searches for maximum spanning trees over directed graphs (C\# conversion of the Java code)
%%     RelEx An open source parser that generates a dependency parse for the English language, by applying graph rewriting to the output of the link grammar parser
%%     ClearParser A statistical, transition-based dependency parser.
%%     Stanford parser A statistical phrase-structure parser which provides a tool to convert the output into a form of dependency graph called "Stanford Dependencies"
%%     TULE A linguistic framework that takes a natural language sentence in input (Italian) and returns a full dependency tree describing its syntactic structure
%%     XDG Development Kit An integrated development environment for Extensible Dependency Grammar (XDG)

%% \end{itemize}

\section{有关表示形式的一般说明}
%\section{General remarks on the representational format}

\subsection{价、结与卫星结构}
%\subsection{Valence information, nucleus and satellites}

依存语法的核心概念是“价”(请参阅\ref{sec-intro-arg-adj})。这个概念的核心隐喻是稳定分子的信息,这在化学中解释为电子的层级关系。
%The central concept of Dependency Grammar is valence (see Section~\ref{sec-intro-arg-adj}). The
%central metaphor for this is the formation of stable molecules, which is explained in chemistry with
%reference to layers of electrons.
%\todostefan{Sylvain: Due to  \citet{Jespersen37a-u}} 
%
% Jespersen, O. (1937). Analytic syntax. London: Allen and Unwin.
%
% Tesnière, L. (1934). Comment construire une syntaxe. Bulletin de la Faculté des Lettres de
% Strasbourg 7–12iéme année (pp. 219–229).
%
化合物与语言结构之间的区别在于化合物是没有方向的,也就是说,我们说氧原子在构成水的过程中比氢原子更重要是没有道理的。与之相比,动词比跟它一起构成完整小句的名词短语更为重要。在诸如英语和德语的语言中,动词决定了它的依存成分的形式,比如说他们的格。
%A difference between chemical compounds and linguistic structures
%is that the chemical compounding is not directed, that is, it would not make sense to claim that oxygen is more
%important than hydrogen in forming water. In contrast to this, the verb is more important than
%the nominal phrases it combines with to form a complete clause. In languages like English and
%German, the verb determines the form of its dependents, for instance their case.
 
描述依存关系的一个方法如图\vref{fig-the-child-reads-the-book-dg}所示。
%One way to depict dependencies is shown in Figure~\vref{fig-the-child-reads-the-book-dg}.
% moved this on top of the figure
最高阶的结点是动词reads。它的价是一个主格NP(主语)和一个宾格NP(宾语)。
%The highest node is the verb \emph{reads}. Its valence is a nominative NP (the subject) and an
%accusative NP (an object). 
\begin{figure}
\centerline{%
\begin{forest}
dg edges
[V
  [N
    [D [the;\textsc{det}] ]
     [child;孩子] ]
  [reads;读]
  [N
    [D [a;一] ]
    [book;书] ] ]
\end{forest}
}
\caption{\label{fig-the-child-reads-the-book-dg}The child reads a book.的分析}
%\caption{\label{fig-the-child-reads-the-book-dg}Analysis of \emph{The child reads a book.}}
\end{figure}%

这通过在表示动词的结点和表示各自名词的结点之间的依存关系来描述。名词本身需要一个限定词,它是分别通过the和a的依存关系来表示的。需要注意的是,这里展示的分析对应于HPSG中假定的NP分析,即名词选择它的限定词(请参阅\ref{Abschnitt-Spr}节)。不过,需要指出的是,NP与DP分析哪一个更合理?这个问题在依存语法社团中也展开了讨论(\citealp[\page 90]{Hudson84a-u};\citealp{vanLangendonck94a,Hudson2004a})。请参阅 \citet{Engel77}关于N作为中心语的分析以及 \citet[\page 31]{Welke2011a-u}有关限定词作为中心语的分析。
%This is depicted by the dependency links between the node representing the verb and the
%nodes representing the respective nouns. The nouns themselves require a determiner, which again is shown by the dependency
%links to \emph{the} and \emph{a} respectively. Note that the analysis presented here corresponds to
%the NP analysis that is assumed in HPSG for instance, that is, the noun selects its specifier (see
%Section~\ref{Abschnitt-Spr}). It should be noted, though, that the discussion whether an NP or a DP
%analysis is appropriate also took place within the Dependency Grammar community
%(\citealp[\page 90]{Hudson84a-u}; \citealp{vanLangendonck94a,Hudson2004a}). See  \citet{Engel77} for
%an analysis with the N as head and  \citet[\page 31]{Welke2011a-u} for an analysis with the
%determiner as head.
% \citet[Section~8.2]{Eroms2000a} suggests an analysis where the relation between determiner
%and noun is marked to be special and neither depends on the other. Nevertheless he treats the noun
%as the head.

动词是小句的中心语,名词叫做从属词(dependants)。另一套针对中心语与从属语的术语是结(nucleus)\isc{结}\is{nucleus}与卫星结构(satellite)\isc{卫星结构}\is{satellite}。
%The verb is the head of the clause and the nouns are called \emph{dependents}. Alternative terms
%for head and dependent are \emph{nucleus}\is{nucleus} and \emph{satellite}\is{satellite}, respectively. 

依存语法的变体——词语法(word grammar)\indexwgc 使用另外一种描写依存关系的方法\citep{Hudson2007a-u},如图\vref{fig-the-child-reads-the-book-dg-sentence}所示。
%An alternative way to depict the dependencies, which is used in the Dependency Grammar variant Word Grammar\indexwg \citep{Hudson2007a-u}, is %provided in Figure~\vref{fig-the-child-reads-the-book-dg-sentence}.
% moved this on top of the figure
这张图显示了语法功能,而不是词类信息,但是除此以外,它与图\ref{fig-the-child-reads-the-book-dg}是一致的。图\ref{fig-the-child-reads-the-book-dg}中的最高结点被标记为图\ref{fig-the-child-reads-the-book-dg-sentence}中的\textsc{root}箭头。向下的关系是由箭头的方法来预测的。
%This graph displays the grammatical functions rather than information about part of speech, but apart
%from this it is equivalent to the representation in Figure~\ref{fig-the-child-reads-the-book-dg}. The highest node in Figure~\ref{fig-the-child-reads-the-%book-dg} is labeled with the \textsc{root}
%arrow in Figure~\ref{fig-the-child-reads-the-book-dg-sentence}. Downward links are indicated by the direction of the arrows.
\begin{figure}
%http://en.wikibooks.org/wiki/LaTeX/Linguistics#Syntactic_trees
\centerline{%
\begin{dependency}[theme = simple]
   \begin{deptext}[column sep=1em]
      The \& child \& reads \& a \& book. \\
   \end{deptext}
   \deproot{3}{ROOT}
   \depedge{2}{1}{\textsc{det}}
   \depedge{3}{2}{SBJ}
   \depedge{3}{5}{OBJ}
   \depedge{5}{4}{\textsc{det}}
\end{dependency}
}
\caption{\label{fig-the-child-reads-the-book-dg-sentence}The child reads a book.的分析的另一种表示形式}
%\caption{\label{fig-the-child-reads-the-book-dg-sentence}Alternative presentation of the analysis of \emph{The child reads a book.}}
\end{figure}%

第三种表示相同的依存关系的形式如图\vref{fig-the-child-reads-the-book-dg-words}所示,这里又有树格式了。如果我们将图中的根结点向上表示,就得到了这棵树。
%A third form of representing the same dependencies provided in Figure~\vref{fig-the-child-reads-the-book-dg-words} has the tree format again.
% moved this before the figure
%This tree results if we pull the root node in Figure~\ref{fig-the-child-reads-the-book-dg-sentence}
%upwards. 
%
\begin{figure}
%http://en.wikibooks.org/wiki/LaTeX/Linguistics#Syntactic_trees
\centerline{%
\begin{forest}
sm edges
[reads
  [child,edge label={node[midway,left,font=\scriptsize]{SBJ~~}} 
    [the,edge label={node[midway,left,font=\scriptsize]{\textsc{det}}}] ]
  [book,edge label={node[midway,right,font=\scriptsize]{~OBJ}} 
    [a,edge label={node[midway,right,font=\scriptsize]{\textsc{det}}}] ] ]
\end{forest}
}
\caption{\label{fig-the-child-reads-the-book-dg-words}The child reads a book.的分析的另一种表示形式}
%\caption{\label{fig-the-child-reads-the-book-dg-words}Alternative presentation of the analysis of \emph{The child reads a book.}}
\end{figure}%
由于我们对从属词上的结的依存关系有清晰的可视化表示,我们就不用箭头来表示这个信息了。但是,有些依存语法的变体,比如说\wg{},使用了相互依存关系。由此,比如说,有些理论认为在his child的分析中,his依存于child,而且child依存于his。如果必须要描述这种相互依存的关系,则需要对所有的依存关系使用箭头或者有些层级树中的依存关系用下划线表示,而其他依存关系用箭头表示。
%Since we have a clear visualization of the dependency relation that represents the nucleus
%above the dependents, we do not need to use arrows to encode this information. However, some
%variants of Dependency Grammar -- for instance \wg{} -- use mutual dependencies. So for instance, some
%theories assume that \emph{his} depends on \emph{child} and \emph{child} depends on
%\emph{his} in the analysis of \emph{his child}. If mutual
%dependencies have to be depicted, either arrows have to be used for all dependencies or some
%dependencies are represented by downward lines in hierarchical trees and other dependencies by arrows.

当然,词类信息也可以加进图\ref{fig-the-child-reads-the-book-dg-sentence}和图\ref{fig-the-child-reads-the-book-dg-words}中,语法功能标签可以加在图\ref{fig-the-child-reads-the-book-dg}中,而语序可以加在图\ref{fig-the-child-reads-the-book-dg-words}中。
%Of course part of speech information can be added to the Figures~\ref{fig-the-child-reads-the-book-dg-sentence}
%and~\ref{fig-the-child-reads-the-book-dg-words}, grammatical function labels could be added to
%Figure~\ref{fig-the-child-reads-the-book-dg}, and word order can be added to Figure~\ref{fig-the-child-reads-the-book-dg-words}.

上图描述了中心语与各自的从属词之间的依存关系。这可以按照$n$-ary规则\label{page-rule-format-dg}来表示得更为形式化,该规则与第\ref{Kapitel-PSG}章讨论的短语结构规则是十分类似的(\citealp[\page 305]{Gaifman65a};\citealp[\page 513]{Hays64a-u};\citealp[\page 61]{Baumgaertner70a};\citealp[\S~4.1]{Heringer96a-u})。比如说,Baumgärtner提出了(\mex{1})中的规则:
%The above figures depict the dependency relation that holds between a head and the respective
%dependents. This can be written down more formally as an $n$-ary rule\label{page-rule-format-dg} that is similar to phrase
%structure rules that were discussed in Chapter~\ref{Kapitel-PSG} (\citealp[\page 305]{Gaifman65a}; \citealp[\page 513]{Hays64a-u}; \citealp[\page 61]
%{Baumgaertner70a}; \citealp[Section~4.1]{Heringer96a-u}). For instance Baumgärtner suggests the
%rule in (\mex{1}):
\ea
$\chi \to \varphi_1 \ldots \varphi_i * \varphi_{i+2} \ldots \varphi_n, where~0 < i \leq n$
\z
(\mex{0})中的星号对应范畴$\chi$中的词。在我们的例子中,$\chi$可以是V,`$*$'的位置可以由\emph{reads}表示,而且$\varphi_1$和$\varphi_3$可以是N。与(\mex{1}b)中关于限定词—名词组合的规则一起,(\mex{1}a)中的规则可以允准图\ref{fig-the-child-reads-the-book-dg}中的依存树。
%The asterisk in (\mex{0}) corresponds to the word of the category $\chi$. In our example, $\chi$
%would be V, the position of the `$*$' would be taken by \emph{reads}, and $\varphi_1$ and
%$\varphi_3$ would be N. Together with the rule in (\mex{1}b) for the determiner-noun combination, the rule in (\mex{1}a) would license
%the dependency tree in Figure~\ref{fig-the-child-reads-the-book-dg}.%\pagebreak
\eal
\ex V $\to$ N $*$ N
\ex N $\to$ D $*$
\zl
%\pagebreak

\noindent
另外,许多二叉规则可以被看作是与它的主语、直接宾语或间接宾语的组合\citep{Kahane2009a}。我们会在\ref{sec-dependency-vs-constituency}详细讨论依存规则,以及依存语法与短语结构语法的对比。
%Alternatively, several binary rules can be assumed that combine a head with
%its subject, direct object, or indirect object \citep{Kahane2009a}. Dependency rules will be discussed in
%more detail in Section~\ref{sec-dependency-vs-constituency}, where dependency grammars are compared with phrase structure grammars.
%% \ea
%% $\chi (\varphi_1 \ldots \varphi_n), where~0 < n$
%% \z
%% $\chi$ and $\varphi_1 \ldots \varphi_n$ are the category labels of lexemes. An alternative way to
%% write this down is provided in (\mex{1}):
%% \ea
%% $\chi \to \varphi_1 \ldots \varphi_i * \varphi_{i+2} \ldots \varphi_n, where~0 < i \leq n$
%% \z
%% The asterisk in (\mex{0}) corresponds to the word of the category $\chi$.


% Linke/Nussbaumer/Portmann 1996, S.112


%% Nucleus/Kern: 
%% Element des Satzes, dass in einer Abhängigkeitsbeziehung zu einem anderen steht
%% Konnexion: 
%% Verbindung zweier Kerne, strukturelle Beziehung zwischen zwei Elementen 
%%  Abhängigkeitsbeziehung

%% Nexus/Knoten: 
%% Das Verb bildet den obersten Knoten, von dem alle Konstituenten des Satzes mittelbar oder unmittelbar abhängen (Dependentien)
%% Dependentien
%% Aktanten: Lebewesen oder Dinge, die aktiv oder passiv an durch das Verb beschriebenen Aktionen beteiligt sind (z.B. Subjekt, Objekt)
%% Angaben: zur näheren Bestimmung der Aktion (z.B. Adverbiale)
%% Indices: von Aktanten und Angaben abhängig (Artikel, Adjektive, Pronomina)

%% Regentien: Dependentien, die anderen Elementen übergeordnet sind
%% Junktive: quantitative Veränderung des Satzes (z.B. durch Konjunktionen)
%% Translative: qualitative Veränderung des Satzes durch (semantisch) „leere“ Wörter (Überführung einer Kategorie in eine andere)

\subsection{附加语}
%\subsection{Adjuncts}

\tes 使用的另一个隐喻是“小戏”。一个事件的核心参与者是行动元(actants)\isc{行动元}\is{actant} ,除此之外就是背景,即舞台,一般的布置。行动元这个概念在其他理论中是论元,而描述为舞台的实体被叫做场景(circumstants)。这些场景是修饰语和在本书介绍的其他理论中通常所说的附加语。在依存关系的表示中,依存语法的论元和附加语之间没有太大的区别。图\vref{fig-the-child-often-reads-the-book-slowly}表示了(\mex{1})的分析:
%Another metaphor that was used by \tes is the drama metaphor. The core participants of an event are
%the \emph{actants}\is{actant} and apart from this there is the background, the stage, the general setting. The actants
%are the arguments in other theories and the stage-describing entities are called
%\emph{circumstants}. These circumstants are modifiers and usually analyzed as adjuncts in the other
%theories described in this book. As far as the representation of dependencies is concerned, there is
%not much of a difference between arguments and adjuncts in Dependency Grammar.
%Figure~\vref{fig-the-child-often-reads-the-book-slowly} shows the analysis of (\mex{1}):
\ea
\gll The child often reads the book slowly.\\
\textsc{det} 孩子 经常 读 \textsc{det} 书 慢慢地\\
\mytrans{那个孩子经常慢慢地读书。}
\z
\begin{figure}
\centerline{%
    \begin{forest}
    dg edges
    [V,l sep=2\baselineskip
      [N
        [D [the;\textsc{det}] ]
         [child;孩子] ]
      [Adv,dg adjunct [often;经常]]
      [reads;读]
      [N
        [D [the;\textsc{det}] ]
        [book;书] ]
      [Adv,dg adjunct [slowly;慢慢地]] ]
    \end{forest}
}
\caption{\label{fig-the-child-often-reads-the-book-slowly}The child often reads the book slowly.的分析}
%\caption{\label{fig-the-child-often-reads-the-book-slowly}Analysis of \emph{The child often reads
%    the book slowly.}}
\end{figure}%
依存关系的标注使用了 \citet{Engel77}提出来的描写不同依存关系的技术手段:附加语用一条从附加语结点向上的额外的线来表示(也请参阅\citealp{Eroms2000a})。另一种区分论元/附加语,或者行动元/场景元的方法是作为论元和附加语的地位的明确的区别。所以我们可以使用针对附加语和论元的明确标记,就像在前面标记语法功能一样。德语语法和配价辞典经常使用标签E和A来分别表示Ergänzung(补足语)和Angabe(说明语)。
%The dependency annotation uses a technical device suggested by  \citet{Engel77} to depict
%different dependency relations: adjuncts are marked with an additional line upwards from the adjunct
%node (see also \citealp{Eroms2000a}). An alternative way to specify the argument/adjunct, or rather the actant/circumstant distinction, is of course an %explicit
%specification of the status as argument or adjunct. So one can use explicit labels for adjuncts and
%arguments as it was done for grammatical functions in the preceding. German grammars and valence dictionaries often
%use the labels E and A for \emph{Ergänzung} and \emph{Angabe}, respectively.


\subsection{语序次序化}
%\subsection{Linearization}
\label{sec-dg-linearization}

目前,我们已经看到了依存图,它们连接了按照一定线性顺序排列的词。不过,从属词的语序在原则上不是由依存关系决定的,由此依存语法必须使用额外的手段来表示语言对象(词根、形素、词)的合理语序。 \citet[\page 50]{Engel2014a}提出了针对(\mex{1})中的句子的依存图\vref{fig-ich-bei-tom-gestern}。\footnote{%
Engel使用E\sub{sub}来表示主语,还有E\sub{acc}、E\sub{dat}和E\sub{gen}来表示具有不同格的宾语。
}
%So
%% \todostefan{S: OK but you still don't have presented a DG. How a DG produce a dependency tree?
%% And in this paper you don't explain how DGs formalize the linearization. See Kahane 2001 for a detailled discussion.
%% And Gerdes \& Kahane 2001 or Debusmann \& Duchier 2001 for topological DG.}
%far we have seen dependency graphs that had connections to words that were linearized in a
%certain order. The order of the dependents, however, is in principle not determined by the dependency
%and therefore a Dependency Grammar has to contain additional statements that take care of the
%proper linearization of linguistic objects (stems, morphemes, words).  \citet[\page 50]{Engel2014a}
%assumes the dependency graph in Figure~\vref{fig-ich-bei-tom-gestern} for the sentences in
%(\mex{1}).\footnote{%
%  Engel uses E\sub{sub} for the subject and E\sub{acc}, E\sub{dat}, and E\sub{gen} for the objects
%  with respective cases.
%}
\eal
\label{ex-gestern-war-ich-bei-tom}
\ex 
\gll Gestern war ich bei Tom.\\
     昨天 \textsc{cop} 我 \textsc{prep} Tom\\
\mytrans{我昨天在Tom家。}
%     yesterday was I with Tom\\
%\mytrans{I was with Tom yesterday.}
\ex 
\gll Ich war gestern bei Tom.\\
     我   \textsc{cop} 昨天 \textsc{prep} Tom\\
%     I   was yesterday with Tom\\
\ex 
\gll Bei Tom war ich gestern.\\
     \textsc{prep} Tom  \textsc{cop} 我 昨天\\
%     with Tom  was I yesterday\\
\ex 
\gll Ich war bei Tom gestern.\\
     我 \textsc{cop} \textsc{prep} Tom 昨天\\
%     I was with Tom yesterday\\
\zl
\begin{figure}[htb]
\centerline{
\begin{forest}
[V\sub{fin, \sliste{ \normalfont sub, sit }}\\
 war\\
 \textsc{cop}
%  was
 [E\sub{sub}\\
  ich\\
  我]
%    I]
 [E\sub{sit}\\
  bei Tom\\
  \textsc{prep} Tom]
%    with Tom]
 [A\sub{temp}\\
  gestern\\
  昨天]]
%    yesterday]]
\end{forest}
}
\caption{\label{fig-ich-bei-tom-gestern}根据 \citet[\page 50]{Engel2014a}的ich、war、bei Tom和gestern(我昨天在Tom家。)的不同语序的依存图}
%\caption{\label{fig-ich-bei-tom-gestern}Dependency graph for several orders of \emph{ich},
%  \emph{war}, \emph{bei Tom}, and \emph{gestern} `I was with Tom yesterday.' according to
 %  \citet[\page 50]{Engel2014a}}
\end{figure}%
根据 \citet[\page 50]{Engel2014a},正确的语序是由表层句法规则决定的,比如说,证明在陈述性主句的前场位置总有一个成分的规则,以及在第二位是定式动词的规则。\footnote{\label{fn-Engel-linearization}%
``Die korrekte Stellung ergibt sich dann zum Teil aus oberflächensyntaktischen Regeln (zum Beispiel:
im Vorfeld des Konstativsatzes steht immer genau ein Element; das finite Verb steht an zweiter
Stelle) [\ldots]''
}$^,$\footnote{%
   \citet[\page 81]{Engel70a}针对\vfc 中只有一个成分的说法提出了反例。相关的例子将在\ref{sec-dg-multiple-frontings}讨论。
} 进而,还有关于语用属性的线性规则,比如说旧信息位于新信息之前。另一条规则保证了弱代词被置于前场或中场的开头。这一线性顺序的概念在经验与概念上都是有问题的,我们会在\ref{sec-linearization-problems-dg}再来讨论。这里需要指出的是,只处理依存关系的方法承认中心语和它的从属词的非连续的实现形式。除了任何更进一步的约束,依存语法还有一个在\ref{sec-ECG}的\pageref{ex-dass-die-frauen-tueren-oeffnen-disc}页中已经讨论过的有关嵌套构式语法以及\ref{sec-fcg-nld}关于流体构式语法的问题。其中,一个论元可以打断另一个论元,如图\vref{fig-dass-die-Frauen-Tueren-oeffnen-dg}所示。
%According to  \citet[\page 50]{Engel2014a}, the correct order is enforced by surface syntactic rules as for
%instance the rules that states that there is always exactly one element in the Vorfeld in
%declarative main clauses and that the finite verb is in second position.\footnote{\label{fn-Engel-linearization}%
%``Die korrekte Stellung ergibt sich dann zum Teil aus oberflächensyntaktischen Regeln (zum Beispiel:
%im Vorfeld des Konstativsatzes steht immer genau ein Element; das finite Verb steht an zweiter
%Stelle) [\ldots]''
%}$^,$\footnote{%
%   \citet[\page 81]{Engel70a} provides counterexamples to the claim that there is exactly one element
%  in the \vf. Related examples will be discussed in Section~\ref{sec-dg-multiple-frontings}.
%} Furthermore, there are linearization rules that concern pragmatic properties, as for instance
%given information before new information. Another rule ensures that weak pronouns are placed into the Vorfeld or at the beginning of the Mittelfeld.
%This conception of linear order is problematic both for empirical and conceptual reasons and we will
%turn to it again in Section~\ref{sec-linearization-problems-dg}. It should be noted here that approaches
%that deal with dependency alone admit discontinuous realizations of heads and their
%dependents. Without any further constraints, Dependency Grammars would share a problem that was
%already discussed on page~\pageref{ex-dass-die-frauen-tueren-oeffnen-disc} in Section~\ref{sec-ECG}
%on Embodied Construction Grammar and in Section~\ref{sec-fcg-nld} with respect to Fluid Construction Grammar: one argument could
%interrupt another argument as in Figure~\vref{fig-dass-die-Frauen-Tueren-oeffnen-dg}.
\begin{figure}
\centerline{%
\begin{forest}
dg edges
[V
  [N
    [D,no edge,name=die [die;\textsc{det}] ]
     [Frauen;女人] ]
%    [D,no edge,name=die [die;the] ]
%     [Frauen;women] ]
  [N,name=Türen
     [Türen;门] ] 
  [öffnen;开]
%     [Türen;doors] ] 
%  [öffnen;open]
]
\draw (Türen.south)--(die.north);
\end{forest}
}
\caption{\label{fig-dass-die-Frauen-Tueren-oeffnen-dg}dass die Frauen Türen
    öffnen(这个女人开门)的不想要的分析}
%\caption{\label{fig-dass-die-Frauen-Tueren-oeffnen-dg}Unwanted analysis of \emph{dass die Frauen Türen
%    öffnen} `that the women open doors'}
\end{figure}%
为了排除语言中的这种不可能的线性排列,有时会认为分析需要是可投射的\isc{可投射性}\is{projectivity},即如图\ref{fig-dass-die-Frauen-Tueren-oeffnen-dg}中交叉的分支是不被允许的。这实际上重新将直接成分的概念引入了框架中,因为这就意味着中心语的所有从属词都必须在中心语周边实现,除非故意使用特殊的机制(请参阅\ref{sec-nld-dg}关于非局部依存的例子)。\footnote{\label{fn-projective-dg-vs-constituents}%
尽管这可以得到短语结构语法中所谓的单位(unit),但这里也是有区别的:在短语结构语法中,单位具有功能标签(如NP),这在依存语法中是不同的。在依存语法中,我们只指出中心语的标签(如图\ref{fig-the-child-often-reads-the-book-slowly}中属于child的N)或者直接指出中心词(如图\ref{fig-the-child-reads-the-book-dg-words}中的词child)所以在依存语法表示中有更少的结点(但是请参阅\ref{sec-dg-is-simpler}的讨论)。
}有些作者明确地使用短语结构成分来构成成分的线性约束\citep{GK2001a,Hellwig2003a}。
%In order to exclude such linearizations in languages in which they are impossible, it is sometimes assumed that analyses have to be projective
%\is{projectivity},
%that is crossing branches like those in Figure~\ref{fig-dass-die-Frauen-Tueren-oeffnen-dg} are not
%allowed. This basically reintroduces the concept of constituency into the framework, since this
%means that all dependents of a head have to be realized close to the head unless special mechanisms for
%liberation are used (see for instance Section~\ref{sec-nld-dg} on nonlocal dependencies).\footnote{\label{fn-projective-dg-vs-constituents}%
%  While this results in units that are also assumed in phrase structure grammars, there is a
%  difference: the units have category labels in phrase structure grammars (for instance NP), which is not the case in
%  Dependency Grammars. In Dependency Grammars, one just refers to the label of the head (for instance
%  the N that belongs to \emph{child} in Figure~\ref{fig-the-child-often-reads-the-book-slowly}) or
%  one refers to the head word directly (for instance, the word \emph{child} in
 % Figure~\ref{fig-the-child-reads-the-book-dg-words}). So there are fewer nodes in Dependency
%  Grammar representations (but see the discussion in Section~\ref{sec-dg-is-simpler}).
%} Some
%authors explicitly use a phrase structure component to be able to formulate restrictions on
%serializations of constituents \citep{GK2001a,Hellwig2003a}. 

% HE2003a-u DG und lineare Ordnung

\subsection{语义}
%\subsection{Semantics}

\tes 已经按照后来理论中常见的语义角色区分了动词的参与者。他指出,第一个行动元是施事,第二个是受事,第三个是受益者\citep[\S~106]{Tesniere2015a-not-crossreferenced}。
%\tes already distinguished the participants of a verb in a way that was later common in theories of
%semantic roles. He suggested that the first actant is the agent, the second one a patient and the
%third a benefactive \citep[Chapter~106]{Tesniere2015a-not-crossreferenced}.
% \citealp[\page 258]{AF2010a}).
% \citet{Welke2003a-u} and  \citet{Fillmore2003a-u} discuss the assignment of semantic roles in
%overview articles on Dependency Grammar. 
考虑到依存语法是一个基于词汇的框架,所有关于论元连接的词汇化方法都可以被采用。但是,论元联接\isc{联接}\is{linking}与语义角色\isc{语义角色}\is{semantic role}指派只是当自然语言表达需要指派意义时亟需解决的一小部分问题。附加语和量词的辖域的问题是需要被解决的,而且很清楚的是没有考虑到用线性语序表示依存关系的依存图是不够的。一个没有排序的依存图将语法功能指派给中心语的从属词,并且它在很多方面都类似于\lfgc 的f"=结构。\footnote{%
Tim Osborne(p.\,c.\ 2015)提醒我,并非所有情况都是这样的:比如说非谓词性介词没有在f"=结构中显示,但是他们必然出现在依存图中。
} 对于第\pageref{ex-david-devoured-a-sandwich-at-noon-yesterday}页的(\ref{ex-david-devoured-a-sandwich-at-noon-yesterday})这类句子,这里重复为(\mex{1}),我们得到第\pageref{fstruc-david-devoured-a-sandwich-at-noon-yesterday}页的(\ref{fstruc-david-devoured-a-sandwich-at-noon-yesterday})中的f"=结构。这个f"=结构包括一个主语(David)、一个宾语(a sandwich),以及带有两个成分的附加语的集合(at noon和yesterday)。
%Given that Dependency Grammar is a lexical framework, all
%lexical approaches to argument linking\is{linking} can be adopted. However, argument linking and semantic
%role\is{semantic role} assignment are just a small part of the problem that has to be solved when natural language
%expressions have to be assigned a meaning. 
%Issues regarding the scope of adjuncts and quantifiers
%have to be solved and it is clear that dependency graphs representing dependencies without taking
%into account linear order are not sufficient. An unordered dependency graph assigns grammatical
%functions to a dependent of a head and hence it is similar in many respects to an \lfg
%f"=structure.\footnote{%
%Tim Osborne (p.\,c.\ 2015) reminds me that this is not true in all cases: for instance non"=predicative prepositions are not reflected in f"=structures, but %of course they are present in
%dependency graphs.
%} For a sentence
%like (\ref{ex-david-devoured-a-sandwich-at-noon-yesterday}) on page~\pageref{ex-david-devoured-a-sandwich-at-noon-yesterday}, repeated here as 
%(\mex{1}), one gets
%the f"=structure in (\ref{fstruc-david-devoured-a-sandwich-at-noon-yesterday}) on
%page~\pageref{fstruc-david-devoured-a-sandwich-at-noon-yesterday}. This f"=structure contains a
%subject (\emph{David}), an object (\emph{a sandwich}), and an adjunct set with two elements
%(\emph{at noon} and \emph{yesterday}).
\ea
\label{ex-david-devoured-a-sandwich-at-noon-yesterday-two}
\gll David devoured a sandwich at noon yesterday.\\
David 吞食 一 三明治 在 中午 昨天\\
\mytrans{David昨天中午狼吞虎咽地吃光了一个三明治。}
\z
这就是未排序的依存图中的编码形式。由于这个平行特征, \citet[\page 308]{Broeker2003a-u}针对依存语法也提出了粘着语义学就不足为奇了\isc{粘着语义学}\is{glue semantics}(\citealp*{DLS93a-u};\citealp[\S~8]{Dalrymple2001a-u})。我们在\ref{glue-semantics}已经介绍过粘着语义学了。
%This is exactly what is encoded in an unordered dependency graph. Because of this parallel it comes
%as no surprise that  \citet[\page 308]{Broeker2003a-u} suggested to use glue semantics\is{glue semantics}
%(\citealp*{DLS93a-u}; \citealp[Chapter~8]{Dalrymple2001a-u}) for Dependency
%Grammar as well. Glue semantics was already introduced in Section~\ref{glue-semantics}.

依存语法的有些变体对语义有明确的处理。一个例子是意义文本理论(\mttc)\citep{Melcuk88a-u}。词语法是另一个例子(Hudson \citeyear[\S~7]{Hudson91a-u};\citeyear[\S~5]{Hudson2007a-u})。我们不在这里介绍这些理论的概念。需要指出的是,像Hudson的词语法这类理论对线性顺序是十分严格的,而且并不认为(\ref{ex-gestern-war-ich-bei-tom})中的所有句子都具有相同的依存结构(请参阅\ref{sec-nld-dg})。词语法更接近于短语结构语法,并且具有跟基于成分的理论相同的成分序列与语义的互动关系。
%There are some variants of Dependency Grammar that have explicit treatments of
%semantics. One example is \mtt \citep{Melcuk88a-u}. Word Grammar is another one
%(Hudson \citeyear[Chapter~7]{Hudson91a-u}; \citeyear[Chapter~5]{Hudson2007a-u}). The notations of
%these theories cannot be introduced here. It should be noted though that theories like Hudson's Word
%Grammar are rather rigid about linear order and do not assume that all the sentences in
%(\ref{ex-gestern-war-ich-bei-tom}) have the same dependency structure (see Section~\ref{sec-nld-dg}). Word
%Grammar is closer to phrase structure grammar and therefore can have a semantics that interacts with
%constituent order in the way it is known from constituent"=based theories.

% Hudson2003b hat was

%% Broeker2003a-u: in fact, several dependency theories include
%% such semantic structures, either as
%% separate strata (Meaning-Text Theory,
%% Functional Generative Description) or as
%% part of the overall structure (Dependency
%% Unification Grammar, Word Grammar,
%% DACHS).

\section{被动}
%\section{Passive}
\label{Abschnitt-Passiv-DG}

依存\isc{被动|(}\is{passive|(}语法是一个基于词汇的理论,而且价是其核心概念。基于这个原因,采用基于词汇的方法分析被动就不足为奇了。也就是说,我们假定有一个被动分词,它具有与主动动词不同的配价需求(\citealp[\S~12]{Hudson90a-u};\citealp[\S~10.3]{Eroms2000a};\citealp[\page 53--54]{Engel2014a})。
%Dependency\is{passive|(} Grammar is a lexical theory and valence is the central concept. For this reason, it is not surprising that
%the analysis of the passive is a lexical one. That is, it is assumed that there is a passive
%participle that has a different valence requirement than the active verb
%(\citealp[Chapter~12]{Hudson90a-u}; \citealp[Section~10.3]{Eroms2000a}; \citealp[\page 53--54]{Engel2014a}).

(\mex{1})中的标准例子被分析为图\vref{fig-passive-dg}中所示的形式。
%Our standard example in (\mex{1}) is analyzed as shown in Figure~\vref{fig-passive-dg}.
\ea
\gll [dass] der Weltmeister geschlagen wird\\
     \spacebr{}\textsc{comp} \textsc{det} 世界.冠军 击败 \passiveprs{}\\
\mytrans{世界冠军被击败了} 
%     \spacebr{}that the world.champion beaten is\\
%\mytrans{that the world champion is (being) beaten} 
\z
\begin{figure}
\centering
\begin{forest}
dg edges
[V\sub{fin, \sliste{ \normalfont prt }}, s sep=8mm
  [V\sub{prt, \sliste{ \normalfont sub $\Rightarrow\varnothing$, akk $\Rightarrow$ sub}}
    [N
      [D [der;\textsc{det}] ]
      [Weltmeister;世界.冠军] ]
    [geschlagen;击败] ] 
  [wird;\passiveprs]]
%        [D [der;the] ]
%     [Weltmeister;world.champion] ]
%    [geschlagen;beaten] ] 
%  [wird;is]]
\end{forest}
\caption{\label{fig-passive-dg}[dass] der Weltmeister geschlagen wird(世界冠军被击败了)的分析类似于 \citet[\page 53--54]{Engel2014a}提出的分析}
%\caption{\label{fig-passive-dg}Analysis of [\emph{dass}] \emph{der Weltmeister geschlagen wird}
%  `that the world champion is (being) beaten' parallel to the analyses provided by  \citet[\page 53--54]{Engel2014a}}
\end{figure}%
这张图是针对被动结构的直觉上的描述。对于人称被动来说,其形式化很有可能会落实到词汇规则上。请参阅 \citew[\page 629--630]{Hellwig2003a}关于英语中被动分析的词汇规则的明确建议。
% This figure is an intuitive depiction of what is going on in passive constructions. A formalization would
% probably amount to a lexical rule for the personal passive. See  \citew[\page
%   629--630]{Hellwig2003a} for an explicit suggestion of a lexical rule for the analysis of the
% passive in English.

请注意,der Weltmeister(世界冠军)不是Engel的分析中被动助词wird的论元。这意味着主语--动词的一致关系不能受限于局部,而且我们需要为了一致关系\isc{主谓一致}\is{agreement}来开发一些精细化的机制。\footnote{%
这个问题对于所谓的远被动\isc{远被动}\is{remote passive}来说是更为迫切的问题:
%Note that \emph{der Weltmeister} `the world champion' is not an argument of the passive auxiliary
%\emph{wird} `is' in Engel's analysis. This means that subject--verb agreement\is{agreement} cannot be determined
%locally and some elaborated mechanism has to be developed for ensuring agreement.\footnote{%
%This problem would get even more pressing for cases of the so-called remote passive\is{remote passive}:
\eal
\ex
\gll weil der Wagen zu reparieren versucht wurde\\
     因为 \textsc{det}.\sg.\nom{} 汽车 \textsc{inf} 修理 尝试 \passivepst.\sg\\
\mytrans{因为试着修理这辆汽车}
%     because the.\sg.\nom{} car to repair tried was\\
%\mytrans{because it was tried to repair the car}
\ex
\gll weil die Wagen zu reparieren versucht wurden\\
     因为 \textsc{det}.\pl.\nom{} 汽车 \textsc{inf} 修理 尝试 \passivepst.\pl\\
\mytrans{因为试着修理这辆汽车}
%     because the.\pl.\nom{} cars to repair tried were\\
%\mytrans{because it was tried to repair the cars}
\zl
这里,zu reparieren的宾语与助动词wurde(单数)和wurden(复数)保持一致,这个宾语是嵌套在两层深的动词的宾语。但是,关于如何分析这些远被动的问题在Engel的系统中是有待解决的问题,而且这个问题的解决方法可能会包含HPSG中应用的机制:zu reparieren的论元被提升到统治动词versucht上,被动应用于这个动词,并将宾语转化为主语,这是通过助词提升的。这就解释了zu reparieren(修理)的隐含宾语与wurde(单数)之间的一致关系。 \citet{Hudson97a}在词语法的框架下提出了德语的动词性补足语的分析,这个分析包括了他所谓的概化提升(generalized raising)的过程。他指出,主语和补足语一起提升到了统治中心语中。请注意,这样包括概化提升的分析可以直接对(i)这类句子进行分析,因为宾语将依存于与主语相同的中心语上,即hat(\textsc{aux}),由此可以放在主语前。
%Here the object of \emph{zu reparieren}, which is the object of a verb which is embedded two levels deep,
%agrees with the auxiliaries \emph{wurde} `was' and \emph{wurden} `were'. However, the question how to analyze these remote
%passives is open in Engel's system anyway and the solution of this problem would probably involve the mechanism
%applied in HPSG: the arguments of \emph{zu reparieren} are raised to the governing verb
%\emph{versucht}, passive applies to this verb and turns the object into a subject which is then
%raised by the auxiliary. This explains the agreement between the underlying object of \emph{zu
%  reparieren} `to repair' and \emph{wurde} `was'.  \citet{Hudson97a}, working in the framework of \wg, suggests an analysis of verbal
%complementation in German that involves what he calls \emph{generalized raising}. He assumes
%that both subjects and complements may be raised to the governing head. Note that such an analysis
%involving generalized raising would make an analysis of sentences like (i) straightforward, since
%the object would depend on the same head as the subject, namely on \emph{hat} `has' and hence can be
%placed before the subject.
\ea
\label{ex-gestern-hat-sich-der-spieler-verletzt}
\gll Gestern hat sich der Spieler verletzt.\\
     昨天 \textsc{aux} 自己 \textsc{det} 选手 受伤\\
\mytrans{这个选手昨天伤到了自己。}
%     yesterday has self the player injured\\
%\mytrans{The player injured himself yesterday.}
\z
关于Groß \& Osborne对(\ref{ex-gestern-hat-sich-der-spieler-verletzt})的讨论,请参阅第\pageref{fig-gestern-hat-sich-der-spieler-verletzt-dg-rising}页。
%For a discussion of Groß \& Osborne's account of (\ref{ex-gestern-hat-sich-der-spieler-verletzt}) see page~\pageref{fig-gestern-hat-sich-der-spieler-%verletzt-dg-rising}.
} 
 \citet{Hudson90a-u}、 \citet[\S~5.3]{Eroms2000a}和 \citet{GO2009a}认为主语依存于助动词,而不是主动词。
% \citet{Hudson90a-u},  \citet[Section~5.3]{Eroms2000a} and  \citet{GO2009a} assume that subjects depend on auxiliaries
%rather than on the main verb.
\LATER{S: the classical ref for that is \mel 1988:129-144, who  gives a bunch of criteria.} 
% Silvain Kahane 11.04.2015
%% for the auxiliary as head Melcuk 1988, Hudson 1984, 1990, 2001, etc
%
%% for the participle as head:
%
%% Mertens, Piet (2008)
%% Factorisation des contraintes syntaxiques dans un analyseur de dépendance.
%% Actes du Congrès TALN 2008 (Traitement Automatique du Langage Naturel), Avignon, 9-13 juin 2008. PDF
%
%de Marneffe, M.-C., Manning D. (2008). Stanford typed dependencies manual. Technical report, Stanford University.
%
这需要范畴语法(请参阅\ref{Kategorialgrammatik-Komposition})和HPSG语法\citep{HN94a}中较为普遍的论元转换。更合适的分析是将分词的主语当作助词的主语,如图\vref{fig-passive-subj-raised-dg}所示。\isc{被动|)}\is{passive|)}
%This requires some argument transfer as it is common in \cg (see
%Section~\ref{Kategorialgrammatik-Komposition}) and%
%\hpsg \citep{HN94a}. The adapted analysis that
%treats the subject of the participle as a subject of the auxiliary is given in Figure~\vref{fig-passive-subj-raised-dg}.\is{passive|)}
\begin{figure}
\centering
\begin{forest}
dg edges
[V\sub{fin, \sliste{ \normalfont sub, prt }}, s sep=8mm
  [N
    [D [der; \textsc{det}] ]
    [Weltmeister;世界.冠军] ]
%    [D [der;the] ]
%    [Weltmeister;world.champion] ]
  [V\sub{prt, \sliste{ \normalfont sub $\Rightarrow\varnothing$, akk $\Rightarrow$ sub}}
    [geschlagen;击败] ] 
  [wird;\passiveprs{}]]
%    [geschlagen;beaten] ] 
%  [wird;is]]
\end{forest}
\caption{\label{fig-passive-subj-raised-dg}带有主语作为助词从属词的[dass] der Weltmeister geschlagen wird
(世界冠军被打了)的分析}
%\caption{\label{fig-passive-subj-raised-dg}Analysis of [\emph{dass}] \emph{der Weltmeister geschlagen wird}
%  `that the world champion is (being) beaten' with the subject as dependent of the auxiliary}
\end{figure}%

\section{动词位置}
%\section{Verb position}

在许多德语的依存语法的著作中,并没有处理线性顺序问题,而且作者只关注依存关系。动词及其论元之间的依存关系基本上等同于动词位于首位与动词位于末位的句子。如果我们比较图\ref{fig-vl-dg}和图\ref{fig-vi-dg}中给出的(\mex{1})中的例子的依存图的话,我们会看到只有动词的位置是不同的,但是依存关系就如他们应当的那样是相同的。\footnote{%
 \citet{Eroms2000a}用词性Pro来表示jeder(每人)这类代词。如果词类的信息在选择中发挥了重要的作用,这就使得管辖名词性表达的中心语的所有配价框架的析取的区分是十分必要的,因为他们要么跟带有内部结构的NP,要么跟带有介词的NP相组合。通过对代词赋予范畴N,我们可以避免这个析取的具体化的问题。代词跟名词的区别在于它的价(当名词需要限定词时,它是完全满足的),而不是它的词性。 \citet[\page 259]{EH2003a}使用符号N\_pro来表示代词。如果pro-部分被理解为带有词性的特殊属性对象,这就跟我们上面讲到的内容不一致了:中心语就可以选择N了。如果N\_pro和N被看作是不同的原子符号,问题还是存在的。

使用N而不是Pron作为代词的词性,这在依存语法的其他版本中是标准的表示,例如词语法 (\citealp[\page 167]{Hudson90a-u};\citealp[\page 190]{Hudson2007a-u})。也请参阅第\pageref{fn-np-pron-ps-rule}页的脚注\ref{fn-np-pron-ps-rule}关于短语结构语法中代词和NP的区别。
}
%In many Dependency Grammar publications on German, linearization issues are not dealt with and
%authors just focus on the dependency relations. The dependency relations between a verb
%and its arguments are basically the same in verb-initial and verb-final sentences. If we compare the
%dependency graphs of the sentences in (\mex{1}) given in
%\pagebreak[4] 
%the Figures~\ref{fig-vl-dg} and~\ref{fig-vi-dg}, we see that only the position of the verb is
%different, but the dependency relations are the same, as it should be.\footnote{%
%   \citet{Eroms2000a} uses the part of speech Pron for pronouns like \emph{jeder}
%  `everybody'. If information about part of speech plays a role in selection, this makes necessary a
 % disjunctive specification of all valence frames of heads that govern nominal expressions, since
 % they can either combine with an NP with internal structure or with a pronoun. By assigning
 % pronouns the category N, such a disjunctive specification is avoided. A pronoun differs from a noun
%  in its valence (it is fully saturated, while a noun needs a determiner), but not in its part of
 % speech.  \citet[\page 259]{EH2003a} use the symbol N\_pro for pronouns. If the pro-part is to be understood as
 % a special property of items with the part of speech N, this is compatible with what I have said
%  above: heads could then select for Ns. If N\_pro and N are assumed to be distinct, atomic symbols,
 % the problem remains.
%
%  Using N rather than Pron as part of speech for pronouns is standard in other versions of
 % Dependency Grammar, as for instance \wg (\citealp[\page 167]{Hudson90a-u}; \citealp[\page
 %   190]{Hudson2007a-u}).
%  See also footnote~\ref{fn-np-pron-ps-rule} on page~\pageref{fn-np-pron-ps-rule} on the distinction
%  of pronouns and NPs in phrase structure grammars.%
%}
\eal
\ex
\gll [dass] jeder diesen Mann kennt\\
     \spacebr{}\textsc{comp} 每人 这 男人 认识\\
\mytrans{每个人都认识这个男人}
%     \spacebr{}that everybody this man knows\\
%\mytrans{that everybody knows this man}
\ex
\gll Kennt jeder diesen Mann?\\
     认识 每人 这 男人\\
\mytrans{每个人都认识这个男人吗?}
%     knows everybody this man\\
%\mytrans{Does everybody know this man?}
\zl

\begin{figure}
\centerline{
\begin{forest}
dg edges
[V
  [N
    [jeder;每人] ]
%    [jeder;everybody ] ]
  [N
    [D [diesen;这] ]
    [Mann;男人] ]
  [kennt;认识] ]
%    [D [diesen;this] ]
%    [Mann;man] ]
%  [kennt;knows] ]
\end{forest}
}
\caption{\label{fig-vl-dg}[dass] jeder diesen Mann kennt(每个人都认识这个男人)的分析}
%\caption{\label{fig-vl-dg}Analysis of [\emph{dass}] \emph{jeder diesen Mann kennt} `that everybody knows this man'}
\end{figure}%
\begin{figure}
\centerline{
\begin{forest}
dg edges
[V
  [kennt;认识]
%  [kennt;knows]
  [N
    [jeder;每人 ] ]
%    [jeder;everybody ] ]
  [N
    [D [diesen;这] ]
    [Mann;男人] ] ]
%    [D [diesen;this] ]
%    [Mann;man] ] ]
\end{forest}
}
\caption{\label{fig-vi-dg}Kennt jeder diesen Mann?(每个人都认识这个男人吗?)的分析}
%\caption{\label{fig-vi-dg}Analysis of \emph{Kennt jeder diesen Mann?} `Does everybody know this
%    man?'}
\end{figure}%
带有论元和附加语的动词的正确的语序由线性约束条件来保证,这些条件与各自的空间位置是相关的。请参阅\ref{sec-dg-linearization}和\ref{sec-linearization-problems-dg}更多有关语序次序化的细节问题。
%The correct ordering of the verb with respect to its arguments and adjuncts is ensured by
%linearization constraints that refer to the respective topological fields. See Section~\ref{sec-dg-linearization} and
%Section~\ref{sec-linearization-problems-dg} for further details on linearization.

\section{局部重新排序}
%\section{Local reordering}

局部重新排序的情况是相同的。(\mex{1}b)中句子的依存关系如图\vref{fig-scrambling-dg}所示。(\mex{1}a)中带有正常语序的句子的分析已经在图\ref{fig-vl-dg}中给出了。
%The situation regarding local reordering is the same. The dependency relations of the sentence
%in (\mex{1}b) are shown in Figure~\vref{fig-scrambling-dg}. The analysis of the sentence with normal order in (\mex{1}a)
%has already been given in Figure~\ref{fig-vl-dg}.
%% Gross und Osborne haben eine
%%   Scrambling-Analyse für Perfektsätze, weil bei Ihnen das Subjekt vom Hilfsverb abhängt und das
%%   Objekt vom Partizip.
\eal
\ex
\gll [dass] jeder diesen Mann kennt\\
     \spacebr{}\textsc{comp} 每人 这 男人 认识\\
\mytrans{每个人都认识这个男人}
%     \spacebr{}that everybody this man knows\\
%\mytrans{that everybody knows this man}
\ex
\gll [dass] diesen Mann jeder kennt\\
     \spacebr{}\textsc{comp} 这 男人 每人 认识\\
\mytrans{每个人都认识这个男人}
%     \spacebr{}that this man everybody knows\\
%\mytrans{that everybody knows this man}
\zl

\begin{figure}
\centerline{
\begin{forest}
dg edges
[V
  [N
    [D [diesen;这个] ]
    [Mann;男人] ]
%    [D [diesen;this] ]
%    [Mann;man] ]
  [N
    [jeder;每人 ] ]
  [kennt;认识] ]
%    [jeder;everybody ] ]
%  [kennt;knows] ]
\end{forest}
}
\caption{\label{fig-scrambling-dg}[dass] diesen Mann jeder kennt (每个人都认识这个人)的分析}
%\caption{\label{fig-scrambling-dg}Analysis of [\emph{dass}] \emph{diesen Mann jeder kennt} `that everybody knows this man'}
\end{figure}%

\section{长距离依存}
%\section{Long"=distance dependencies}
\label{sec-nld-dg}

在\isc{长距离依存|(}\is{long"=Distance dependency|(}依存语法中有许多方法来分析非局部依存关系。最简单的方法我们已经在前面的章节中看到了。许多分析只关注依存关系,并且认为位于第二位的动词只是一种可能的语序次序化的方式(\citealp[\S~9.6.2]{Eroms2000a};\citealp{GO2009a})。图\vref{fig-diesen-mann-kennt-jeder-dg}展示了(\mex{1})的分析:
%There\is{long"=Distance dependency|(} are several possibilities to analyze nonlocal dependencies in Dependency
%Grammar. The easiest one is the one we have already seen in the previous sections. Many analyses just focus on the dependency relations
%and assume that the order with the verb in second position is just one of the possible
%linearization variants (\citealp[Section~9.6.2]{Eroms2000a}; \citealp{GO2009a}). Figure~\vref{fig-diesen-mann-kennt-jeder-dg} shows the analysis of
%(\mex{1}):
\ea
\label{ex-Diesen-Mann-kent-jeder-dg}
\gll {}[Diesen Mann] kennt jeder.\\
	 {}\spacebr{}这 男人 认识 每人\\
\mytrans{每个人都认识这个男人。}
%	 {}\spacebr{}this man knows everybody\\
%\mytrans{Everyone knows this man.}
\z
\begin{figure}
\centerline{
\begin{forest}
dg edges
[V
  [N
    [D [diesen;这] ]
    [Mann;男人] ]
  [kennt;认识] 
%    [D [diesen;this] ]
%    [Mann;man] ]
%  [kennt;knows] 
  [N
    [jeder;每人 ] ]
%    [jeder;everybody ] ]
]
\end{forest}
}
\caption{\label{fig-diesen-mann-kennt-jeder-dg}没有对前置进行特殊处理的Diesen Mann kennt jeder.(这个人,每个人都认识。)的分析}
%\caption{\label{fig-diesen-mann-kennt-jeder-dg}Analysis of \emph{Diesen Mann kennt jeder.} `This
%  man, everybody knows.' without special treatment of fronting}
\end{figure}%
现在,这是最简单的情况,所以让我们看看(\mex{1})中的例子,它确实包括了非局部(nonlocal)的依存关系:
%Now, this is the simplest case, so let us look at the example in (\mex{1}), which really involves a
%\emph{nonlocal} dependency:
\ea
\label{ex-wen-glaubst-du-dass-dg}
\gll Wen$_i$    glaubst        du,        daß  ich       \_$_i$ gesehen habe?\footnotemark\\
     谁.\acc{} 认为.2\sg{} 你.\nom{} \textsc{comp} 我.\nom{} {}      看见    \textsc{aux}\\
%     who.\acc{} believe.2\sg{} you.\nom{} that I.\nom{} {}      seen    have\\
\footnotetext{%
     \citew[\page84]{Scherpenisse86a}。
    }
\mytrans{你认为我看见了谁?}
%\mytrans{Who do you think I saw?}
\z
依存关系的描写如图\vref{fig-wen-glaubst-du-dass-dg}所示。
%The dependency relations are depicted in Figure~\vref{fig-wen-glaubst-du-dass-dg}.
%% \footnote{%
%%  \citet{GO2009a} assume that subjects depend on the auxiliary rather than on the main verb. I am
%% sticking to Engel's analysis here \citeyearpar[\page 49]{Engel2014a}, but this does not affect any of the points made here.
%% }
\begin{figure}
\centering
\begin{forest}
dg edges
[V
  [N,name=nacc,no edge,tier=mytier [wen;谁] ]
  [glaubst;认为] 
  [N [du;你] ]
%  [N,name=nacc,no edge,tier=mytier [wen;who] ]
%  [glaubst;believe] 
%  [N [du;you] ]
  [Subjunction
    [dass;\textsc{comp}]
%    [dass;that]
    [V\sub{fin, \sliste{ \normalfont sub, prt }}
      [N [ich;我 ] ]
%      [N [ich;I ] ]
      [V\sub{prt}, name=vprt
        [N,phantom,tier=mytier]
        [gesehen;看见] ]
      [habe;\textsc{aux}] ] ] ]
 %       [gesehen;seen] ]
%      [habe;have] ] ] ]
\draw (vprt.south)--(nacc.north);
\end{forest}
\caption{\label{fig-wen-glaubst-du-dass-dg}Wen glaubst du, dass ich gesehen habe?(你认为我看见了谁?)的非可投射分析}
%\caption{\label{fig-wen-glaubst-du-dass-dg}Non-projective analysis of \emph{Wen glaubst du, dass ich gesehen habe?}
%  `Who do you think I saw?'}
\end{figure}%%
这张图与我们之前看到的很多图不同,它不具有可投射性\isc{可投射性}\is{projectivity}。这意味着有交叉:对于wen(谁)的V\sub{prt}和N之间的联系与联接glaubst(认为)和du(你)以及他们的范畴符号的关系有所交叉。根据设想的依存语法的版本,这可以被认为是一个问题,或者不是。让我们来看一下两个选项:如果图\ref{fig-wen-glaubst-du-dass-dg}中给出的类型的非联系性被允许用在Heringer和Eroms的语法中(\citealp[\page 261]{Heringer96a-u};\citealp[\S~9.6.2]{Eroms2000a}),\footnote{%
但是,作者提到了将提取的成分向更高的结点提升的可能性。请参阅 \citew[\page 260]{EH2003a}。
}这就需要语法中有一些可以排除那些不合乎语法的非连续体。比如说,图\vref{fig-wen-glaubst-ich-du-dass-dg}中(\mex{1})的分析就应该被排除。
%This graph differs from most graphs we have seen before in not being projective\is{projectivity}. This means that
%there are crossing lines: the connection between V\sub{prt} and the N for \emph{wen} `who' crosses
%the lines connecting \emph{glaubst} `believe' and \emph{du} `you' with their category symbols. Depending on
%the version of Dependency Grammar assumed, this is seen as a problem or it is not. Let us
%explore the two options: if discontinuity of the type shown in
%Figure~\ref{fig-wen-glaubst-du-dass-dg} is allowed for as in Heringer's and Eroms' grammars
%(\citealp[\page 261]{Heringer96a-u}; \citealp[Section~9.6.2]{Eroms2000a}),\footnote{%
%  However, the authors mention the possibility of raising an extracted element to a higher node. See
%  for instance  \citew[\page 260]{EH2003a}.
%} there has to be something in the grammar
%that excludes discontinuities that are ungrammatical. For instance, an analysis of (\mex{1}) as in
%Figure~\vref{fig-wen-glaubst-ich-du-dass-dg} should be excluded.
\ea[*]{
\gll Wen        glaubst        ich      du, dass gesehen habe?\\
     谁.\acc{} 认为.2\sg{} 我.\nom{} 你.\nom{} \textsc{comp} 看见 \textsc{aux}\\
\glt 想说:\quotetrans{你认为我看见谁了?}
%     who.\acc{} believe.2\sg{} I.\nom{} you.\nom{} that seen have\\
%\glt Intended: `Who do you think I saw?'
}
\z
\begin{figure}
\centering
\begin{forest}
dg edges
[V
  [N,name=nacc,no edge,tier=mytier, [wen;谁] ]
  [glaubst;认为 ] 
  [N,name=nich,no edge,tier=vprt [ich;我 ] ]
  [N [du;你] ]
%  [N,name=nacc,no edge,tier=mytier, [wen;who] ]
%  [glaubst;believe] 
%  [N,name=nich,no edge,tier=vprt [ich;I ] ]
%  [N [du;you] ]
  [Subjunction
    [dass;\textsc{comp}]
%    [dass;that]
    [V\sub{fin, \sliste{ \normalfont sub, prt }}, name=vfin
      [V\sub{prt}, name=vprt,tier=vprt
        [N,phantom,tier=mytier]
        [gesehen;看见 ] ]
      [habe;\textsc{aux} ] ] ] ]
%        [gesehen;seen] ]
%      [habe;have] ] ] ]
\draw (vprt.south)--(nacc.north);
\draw (vfin.south)--(nich.north);
\end{forest}
\caption{\label{fig-wen-glaubst-ich-du-dass-dg}*\,Wen glaubst ich
    du, dass gesehen habe?(你认为我看见谁了)的不想得到的依存图}
%\caption{\label{fig-wen-glaubst-ich-du-dass-dg}Unwanted dependency graph of *\,\emph{Wen glaubst ich
%    du, dass gesehen habe?} `Who do you think I saw?'}
\end{figure}%
需要注意的是,(\mex{0})中成分的语序与提出的表示空间位置的观点是兼容的 \citet[\page 50]{Engel2014a}:有一个由wen(谁)填充的\vfc,有一个由glaubst(认为)填充的句子左边界,还有由ich(我)、du(你)和小句论元填充的\mf 。在\mf 有像ich(我)和du(你)这样的代词是完全正常的。问题在于这些代词属于不同的类型:du属于主动词glaubst(认为),而ich (我)依存于(gesehen(看见))habe(\textsc{aux})。一个理论需要覆盖的事实是,前置和外置分别瞄定小句的最左边和最右边。这可以直接按照基于成分的方法来模拟,正如我们在前几章所看到的。
%Note that the order of elements in (\mex{0}) is compatible with statements that refer to
%topological fields as suggested by  \citet[\page 50]{Engel2014a}: there is a \vf filled by \emph{wen}
%`who', there is a left sentence bracket filled by
%\emph{glaubst} `believe', and there is a \mf filled by \emph{ich} `I', \emph{du} `you' and the clausal argument. Having
%pronouns like \emph{ich} and \emph{du} in the \mf is perfectly normal. The problem is that these two
%pronouns come from different clauses: \emph{du} belongs to the matrix verb \emph{glaubst} `believe' while
%\emph{ich} `I' depends on (\emph{gesehen} `seen') \emph{habe} `have'. What has to be covered by a theory is the fact that fronting and
%extraposition target the left-most and right-most positions of a clause, respectively. This can be
%modeled in constituency"=based approaches in a straightforward way, as has been shown in the previous chapters.

作为另一种非连续成分,我们可以假定一个能够在结构中将嵌套的中心语的依存关系提高到更高的结点上的机制。这一分析由 \citet{Kunze68a-u}、 \citet{Hudson97a,Hudson2000a}、 \citet{Kahane97a}、 \citet{KNR98a}和 \citet{GO2009a}提出。
%As an alternative to discontinuous constituents, one could assume additional mechanisms that
%promote the dependency of an embedded head to a higher head in the structure. Such an analysis was
%suggested by  \citet{Kunze68a-u},  \citet{Hudson97a,Hudson2000a},  \citet{Kahane97a},  \citet{KNR98a},
%and  \citet{GO2009a}. 
%% \todostefan{S: This analysis has been proposed many times in DG. I think the first to consider that is Hudson. See his paper Discontinuity of 2000 (on his web page)
%% In Kahane, Nasr and Rambow 1998 (ACL) we formalized this operation and called it lifting.
%% This has been also described by Norbert Bröker (paper in the same issue of TAL) in a LFG style.
%% It has been implemented for German word order by Debusmann \& Duchier (ACL 2001)
%% I wrote dozen of papers on the formalization of extraction showing the link between such an idea,
%% functional uncertainty in LFG, HPSG and CG slash feature, etc.}
接下来,我会用 \citet{GO2009a}提出的有关这类分析的例子。Groß \& Osborne用图\vref{fig-wen-glaubst-du-dass-dg-rising}中的虚线描述了重新组织的依存关系。\footnote{%
 \citet[\page 260]{EH2003a}提出了类似的分析,但是没有说明任何形式化的细节。
}$^,$\footnote{%
需要注意的是, \citet{GO2009a}并没有针对简单和复杂的动词位于第二位的句子做出一个统一的分析。也就是说,对于可以解释为局部重新排序的情况,他们提出了没有提升的分析。他们对(\ref{ex-Diesen-Mann-kent-jeder-dg})的分析如图\ref{fig-diesen-mann-kennt-jeder-dg}所示。这就导致了\ref{sec-linearization-problems-dg}讨论过的问题。
}
%In what follows, I use the analysis by  \citet{GO2009a} as an example for such analyses. Groß \& Osborne depict the reorganized dependencies with a %dashed line as in
%Figure~\vref{fig-wen-glaubst-du-dass-dg-rising}.\footnote{%
%   \citet[\page 260]{EH2003a} make a similar suggestion but do not provide any formal details.%
%}$^,$\footnote{%
%Note that  \citet{GO2009a} do not assume a uniform analysis of simple and complex V2 sentences. That
%is, for cases that can be explained as local reordering they assume an analysis without
%rising. Their analysis of (\ref{ex-Diesen-Mann-kent-jeder-dg}) is the one depicted in
%Figure~\ref{fig-diesen-mann-kennt-jeder-dg}. This leads to problems which will be discussed in Section~\ref{sec-linearization-problems-dg}.
%}
\begin{figure}
\centering
\begin{forest}
dg edges
[V
  [N, edge=dashed [wen;谁] ] 
  [glaubst;认为] 
  [N [du;你] ]
%  [N, edge=dashed [wen;who] ] 
%  [glaubst;believe] 
%  [N [du;you] ]
  [Subjunction
    [dass;\textsc{comp}]
 %   [dass;that]
    [V\sub{fin, \sliste{ \normalfont sub, prt }}
      [N [ich;我 ] ]
%      [N [ich;I ] ]
      [V\sub{prt, g}, 
        [gesehen;看见] ]
      [habe;\textsc{aux}] ] ] ]
%        [gesehen;seen] ]
%      [habe;have] ] ] ]
\end{forest}
\caption{\label{fig-wen-glaubst-du-dass-dg-rising}包括提升的Wen glaubst du, dass
    ich gesehen habe?(你认为我看到了什么?)的可投射分析}
%\caption{\label{fig-wen-glaubst-du-dass-dg-rising}Projective analysis of \emph{Wen glaubst du, dass
%    ich gesehen habe?} `Who do you think I saw?' involving rising}
\end{figure}%%
依存关系(V\sub{prt})的源头被标记为一个g和一个从属词,这个从属词通过虚线连接到他提升\isc{提升}\is{rising}(最高的V)的结点上。除了在局部实现gesehen(看见)的宾格从属词之外,关于缺失成分的信息也传递到了更高的结点上并在那里得到实现。
%The origin of the dependency (V\sub{prt}) is marked with a \emph{g} and the dependent is connected
%to the node to which it has risen\is{rising} (the topmost V) by a dashed line. Instead of realizing the accusative dependent of \emph{gesehen}
%`seen' locally, information about the missing element is transferred to a higher node and
%realized there. 

 \citet{GO2009a}的分析并不是十分准确。这里有一个$g$和一条虚线,但是句子可能会包括多重非局部的依存关系。比如说,例(\mex{1})中,在关系小句den wir alle begrüßt haben(我们都打过招呼的那个人)和die noch niemand hier
  gesehen hat(还没有人在这见过的那个人)中有一个非局部的依存关系:关系代词在关系小句内被前置。短语dem Mann, den wir alle kennen(我们都认识的那个人)是gegeben(给)的前置的与格宾语,而且die noch niemand hier gesehen
  hat(还没有人在这见过的那个人)是从Frau (女人)为中心语的NP中外置而来的。
%The analysis of  \citet{GO2009a} is not very precise. There is a $g$ and there is a dashed line, but
%sentences may involve multiple nonlocal dependencies. In (\mex{1}) for instance, there is a nonlocal dependency
%in the relative clauses \emph{den wir alle begrüßt haben} `who we all greeted have' and \emph{die noch niemand hier
%  gesehen hat} `who yet nobody here seen has': the relative pronouns are fronted inside the relative clauses. The phrase \emph{dem Mann, den wir alle
%  kennen} `the man who we all know' is the fronted dative object of \emph{gegeben} `given' and \emph{die noch niemand hier gesehen
%  hat} `who yet nobody here seen has' is extraposed from the NP headed by \emph{Frau} `woman'.
\ea
\gll Dem Mann, den wir alle begrüßt haben, hat die Frau das Buch gegeben, die noch niemand hier gesehen hat.\\
     \textsc{det} 男人  \textsc{rel} 我们 都 打招呼 \textsc{aux}      \textsc{aux} \textsc{det} 女人 \textsc{det} 书 给 \textsc{rel} 但 没人 这里 看见 \textsc{aux}\\
\mytrans{这里还没有人见过的那个女人把这本书给了这个我们都打过招呼的男人。}
%     the man   who we all greeted have      has the woman the book given who yet nobody here seen has\\
%\mytrans{The woman  who nobody ever saw here gave the book to the man, who all of us greeted.}
\z
所以,这就意味着中心语和移位成分之间的联系(依存关系)已经说的很清楚的。这就是 \citet{Hudson97a,Hudson2000a}在他的词语法中对非局部依存关系的分析:除了连接词与它的主语、宾语和其他成分的依存关系,他还提出了提取成分的更深层次的依存关系。比如说,(\ref{ex-wen-glaubst-du-dass-dg})中的wen(谁),为了简便这里重复为(\mex{1}),是gesehen(看见)的宾语以及glaubst(认为)和dass(\textsc{comp})的提取。
%So this means that the connections (dependencies) between the head and the dislocated element have
%to be made explicit. This is what  \citet{Hudson97a,Hudson2000a} does in his Word
%Grammar\indexwg analysis of nonlocal dependencies: in addition to dependencies that relate a word to
%its subject, object and so on, he assumes further dependencies for extracted elements. For example,
%\emph{wen} `who' in (\ref{ex-wen-glaubst-du-dass-dg}) -- repeated here as (\mex{1}) for convenience -- is the object of \emph{gesehen} `seen' and the %extractee of
%\emph{glaubst} `believe' and \emph{dass} `that': 
\ea
\label{ex-wen-glaubst-du-dass-dg-two}
\gll Wen glaubst du, dass ich gesehen habe?\\
     谁 认为 你 \textsc{comp} 我 看见 \textsc{aux}\\
\mytrans{你认为我看见了谁?}
%     who believe you that I seen have\\
%\mytrans{Who do you believe that I saw?}
\z
Hudson证明了词语法\indexwgc 中多重依存关系的使用对应于HPSG\indexhpsgc 中的结构共享\isc{结构共享}\is{structure sharing}\citep[\page 15]{Hudson97a}。非局部依存关系被模拟为一系列局部依存关系,这跟\gpsgc 和\hpsgc 中所做的是一样的。这是重要的,因为它允许我们捕捉提取路径标记\isc{提取路径标记}\is{extraction path marking}的效果\citep*[\page 1--2, \S~3.2]{BMS2001a}:比如说,有的语言对有成分提取的句子使用一种特殊形式的补足语。图\vref{fig-wen-glaubst-du-dass-wg}给出了词语法中(\ref{ex-wen-glaubst-du-dass-dg-two})的分析。
%Hudson states that the use of multiple dependencies in Word Grammar\indexwg corresponds to structure
%sharing\is{structure sharing} in HPSG\indexhpsg \citep[\page 15]{Hudson97a}. Nonlocal dependencies are modeled as a series
%of local dependencies as it is done in \gpsg and \hpsg. This is important since it allows one to
%capture extraction path marking\is{extraction path marking} effects \citep*[\page 1--2, Section~3.2]{BMS2001a}: for instance,
%there are languages that use a special form of the complementizer for sentences from which an
%element is extracted. Figure~\vref{fig-wen-glaubst-du-dass-wg} shows the analysis of
%(\ref{ex-wen-glaubst-du-dass-dg-two}) in \wg.
\begin{figure}
    \begin{forest}
      wg
      [,phantom
       [wen]
       [glaubst]
       [du]
       [dass]
       [ich]
       [gesehen]
       [habe]
      ]
    % The root
    \draw[deparrow] ([xshift=-3pt,yshift=8ex]glaubst.north) to[out=south, in=north]          ([xshift=-3pt]glaubst.north);
    %
    % surface structure = above the words
\draw[deparrow] ([xshift= 3pt]glaubst.north) .. controls +(up:6mm)  and +(up:6mm)  .. node[above] {s}  ([xshift= 0pt]du.north);
\draw[deparrow] ([xshift= 0pt]glaubst.north) .. controls +(up:12mm) and +(up:12mm) .. node[above] {l} ([xshift=-3pt]dass.north);
%
%    \draw[deparrow] ([xshift= 3pt]glaubst.north) to[out=north, in=north] node[above] {s}     ([xshift= 0pt]du.north);
%    \draw[deparrow] ([xshift= 0pt]glaubst.north) to[out=north, in=north] node[above] {l}     ([xshift=-3pt]dass.north);
%
    \draw[deparrow] ([xshift= 3pt]dass.north)  .. controls +(up:18mm) and +(up:18mm) .. node[above] {c}     ([xshift= 3pt]habe.north);
    \draw[deparrow] ([xshift=-3pt]habe.north)  .. controls +(up:6mm)  and +(up:6mm)  .. node[above] {r}     ([xshift= 0pt]gesehen.north);
    \draw[deparrow] ([xshift= 0pt]habe.north)  .. controls +(up:12mm) and +(up:12mm) .. node[above] {s}     ([xshift= 0pt]ich.north);
    \draw[deparrow] ([xshift= -6pt]glaubst.north) to[out=north, in=north] node[above] {x$<$}  ([xshift= 0pt]wen.north);
    %
    % underground = bellow the words
    \draw[deparrow] ([xshift= 0pt]gesehen.south) to[out=south, in=south] node[below] {x$<$o} ([xshift=-3pt]wen.south);
    \draw[deparrow] ([xshift= 0pt]dass.south)    to[out=south, in=south] node[below] {x$<$}  ([xshift= 0pt]wen.south);
    \end{forest}
\caption{\label{fig-wen-glaubst-du-dass-wg}词语法中带有多重依存关系的Wen glaubst du, dass
    ich gesehen habe?(你认为我看见了谁?)的可投射分析}
%\caption{\label{fig-wen-glaubst-du-dass-wg}Projective analysis of \emph{Wen glaubst du, dass
%    ich gesehen habe?} `Who do you think I saw?' in Word Grammar involving multiple dependencies}
\end{figure}%
词上的连线是通常对主语、宾语和其他论元的依存连线(r是sharer的缩写,它指动词性补足语,l代表clausal complement),在词下的连线是针对提取成分的连线(x$<$)。从gesehen(看见)连到wen(谁)是特殊的,因为它既是一个宾语连线也是一个提取连线(x$<$o)。这个连线是对图\ref{fig-wen-glaubst-du-dass-dg-rising}中由虚线标记的小$g$和N的明确的说明。除了图\ref{fig-wen-glaubst-du-dass-dg-rising}中的情况,图\ref{fig-wen-glaubst-du-dass-wg}也有一个从dass(\textsc{comp})到wen(谁)的提取连线。我们可以用Engel、Eroms和Gross \& Osborne的图形表示法来表示词语法的依存关系:我们可以简单地从V$_{prt}$结点和从属词结点上加虚线到统治wen(谁)的N结点。
%The links above the words are the usual dependency links for subjects (s) and objects (o) and other
%arguments (r is an abbreviation for \emph{sharer}, which refers to verbal complements, l stands for
%\emph{clausal complement}) and the links below the words are links for extractees (x$<$). The link from \emph{gesehen}
%`seen' to \emph{wen} `who' is special since it is both an object link and an extraction link (x$<$o). This
%link is an explicit statement which corresponds to both the little $g$ and the N that is marked by the
%dashed line in Figure~\ref{fig-wen-glaubst-du-dass-dg-rising}. In addition to what is there in
%Figure~\ref{fig-wen-glaubst-du-dass-dg-rising}, Figure~\ref{fig-wen-glaubst-du-dass-wg} also has an
%extraction link from \emph{dass} `that' to \emph{wen} `who'. One could use the graphic representation of Engel,
%Eroms, and Gross \& Osborne to display the Word Grammar dependencies: one would simply add dashed
%lines from the V$_{prt}$ node and from the Subjunction node to the N node dominating \emph{wen}
%`who'.

尽管这看上去比较简单,但是我想指出的是,词语法还应用了其他原则来满足合乎语法的结构。在下面,我来解释无缠绕原则(No-tangling Principle)、不孤单原则(No-dangling Principle)和句子—根原则(Sentence-root Principle)。
%While this looks simple, I want to add that Word Grammar employs further principles that have to be
%fulfilled by well"=formed  structures. In the following I explain the \emph{No-tangling Principle},
%the \emph{No-dangling Principle} and the \emph{Sentence-root Principle}.
\begin{principle}[无缠绕原则]
%\begin{principle}[The No-tangling Principle]
依存箭头不能缠绕。
 % Dependency arrows must not tangle.
\end{principle}

\begin{principle}[不孤单原则]
%\begin{principle}[The No-dangling Principle]
每个词必须有一个母结点。
%Every word must have a parent.
\end{principle}

\begin{principle}[句子—根原则]
%\begin{principle}[The Sentence-root Principle]
在每一个非复合句中,只有一个词,它的母结点不是一个词,而是一个上下文的成分。
%In every non-compound sentence there is just one word whose parent
%is not a word but a contextual element.
\end{principle}

\noindent
非缠绕原则保证了没有交叉的依存线,即它保证了结构是可投射的\citep[\page 23]{Hudson2000a}。因为非局部依存关系是通过具体的依存机制确立的,有人想排除非可投射的分析。这一原则也排除了(\mex{1}b),这里green(绿色)依存于peas(豌豆),但是并不邻接于peas(豌豆)。由于on(\textsc{prep})选择了peas(豌豆),从on(\textsc{prep})到peas(豌豆)的箭头就会与从peas(豌豆)到green(绿色)的箭头交叉。
%The No-tangling Principle ensures that there are no crossing dependency lines, that is, it ensures
%that structures are projective \citep[\page 23]{Hudson2000a}. Since non-local dependency relations are established via the
%specific dependency mechanism, one wants to rule out the non"=projective analysis. This principle
%also rules out (\mex{1}b), where \emph{green} depends on \emph{peas} but is not adjacent to
%\emph{peas}. Since \emph{on} selects \emph{peas} the arrow from \emph{on} to \emph{peas} would cross
%the one from \emph{peas} to \emph{green}.
\eal
\ex[]{
\gll He lives on green peas.\\
他 生存 \textsc{prep} 绿色 豌豆\\
\mytrans{他靠绿色豌豆生存。}
}
\ex[*]{
\gll He lives green on peas.\\
他 生存 绿色 \textsc{prep} 豌豆\\
}
\zl
不孤单原则确保了没有单独的词组连接到结构的主要部分上。没有这条规则,(\mex{0}b)就会被分析为孤立的(green(绿色))\citep[\page 23]{Hudson2000a}。
%The No-dangling Principle makes sure that there are no isolated word groups that are not connected to the main part
%of the structure. Without this principle (\mex{0}b) could be analyzed with the isolated word
%\emph{green} \citep[\page 23]{Hudson2000a}.

句子—根原则需要用来排除那些不止有一个最高成分的结构。glaubst(认为)是图\ref{fig-wen-glaubst-du-dass-wg}中的根。没有其他词统治并选择了它。这个原则确保了没有其他的根。所以这个原则排除了那些短语中的所有成分都是根的情况,因为如果不这样的话,不孤单原则就会失去效力,因为它就会被平凡地实现\citep[\page 25]{Hudson2000a}。
%The Sentence-root Principle is needed to rule out structures with more than one highest
%element. \emph{glaubst} `believe' is the root in Figure~\ref{fig-wen-glaubst-du-dass-wg}. There is
%no other word that dominates it and selects for it. The principle makes sure that there is no other
%root. So the principle rules out situations in which all elements in a phrase are roots, since
%otherwise the No-dangling Principle would lose its force as it could be fulfilled trivially \citep[\page 25]{Hudson2000a}.

我这里加入了相当复杂的原则集合,是为了与基于短语结构的方法进行合理的比较。如果这里针对一般的短语提出一致性的话,就不需要这三个原则了。比如说,\lfgc 和\hpsgc 就不需要这样的原则。
%I added this rather complicated set of principles here in order to get a fair
%comparison with phrase structure"=based proposals. If continuity is assumed for phrases in general,
%the three principles do not have to be stipulated. So, for example, \lfg and \hpsg
%do not need these three principles.

需要注意的是 \citet[\page 16]{Hudson97a}认为\vfc 中的成分被提取出来了,即使是像(\ref{ex-Diesen-Mann-kent-jeder-dg}))这样简单的句子。我将在\ref{sec-linearization-problems-dg}说明为什么我认为这样的分析比那些分析更好,那些分析认为像(\ref{ex-Diesen-Mann-kent-jeder-dg})这样的简单句只是对应于动词位于首位或动词位于末位的语序变体。
%Note that  \citet[\page 16]{Hudson97a} assumes that the element in the \vf is extracted even for simple
%sentences like (\ref{ex-Diesen-Mann-kent-jeder-dg}). I will show in Section~\ref{sec-linearization-problems-dg} why I think that
%this analysis has to be preferred over analyses assuming that simple sentences like
%(\ref{ex-Diesen-Mann-kent-jeder-dg}) are just order variants of corresponding verb-initial or
%verb-final sentences.%
\isc{长距离依存|)}\is{long"=Distance dependency|)}

\section{新的发展与理论变体}
%\section{New developments and theoretical variants}

本节主要介绍\tes 的依存语法的变体。\ref{sec-tesniere-pos}介绍\tes 的词类系统,\ref{sec-connection-junction-transfer}描述由\tes 界定的语言对象的组合的模型。
%This section mainly deals with \tes's variant of Dependency Grammar. Section~\ref{sec-tesniere-pos}
%deals with \tes's part of speech system and Section~\ref{sec-connection-junction-transfer} describes the modes of combinations of
%linguistics objects assumed by \tes.

\subsection{\tes 的词类划分}
%\subsection{\tes's part of speech classification}
\label{sec-tesniere-pos}

正如在导言中提到的,\tes 是依存语法历史上的核心人物,因为他第一次开发了形式化模型\citep{Tesniere59a-u,Tesniere80a-u,Tesniere2015a-not-crossreferenced}。现今有许多依存语法的版本,而且大部分都用其他语言中使用的词类标签(N、P、A、V、Adv、Conj,\ldots)。\tes 的词类系统包括四个主要的范畴:名词、动词、形容词和副词。这些范畴的标签来自于用在世界语\il{Esperanto}的词尾,即分别是O、I、A和E。这些范畴按照语义来界定,如表\ref{table-pos-tesniere}所示。\footnote{%
正如 \citet[\page 77]{Weber97a}指出的,这个范畴不是没有问题的:根据什么说Angst(害怕)是一个实体呢?为什么glauben(认为)是一个具体的过程?也请参阅 \citet[\S~3.4]{Klein71a-u}关于schlagen(击打)和Schlag(击打)以及类似例子的讨论。即使我们认为Schlag是通过转移到范畴O而从\stem{schlag}实际的过程生成的, 这样的O表示实体的假设也是有问题的。
}
%As mentioned in the introduction, \tes is a central figure in the history of Dependency Grammar
%as it was him who developed the first formal model \citep{Tesniere59a-u,Tesniere80a-u,Tesniere2015a-not-crossreferenced}. There are many versions
%of Dependency Grammar today and most of them use the part of speech labels that are used in other
%theories as well (N, P, A, V, Adv, Conj, \ldots). \tes had a system of four major categories: noun,
%verb, adjective, and adverb. The labels for these categories were derived from the endings that are
%used in Esperanto\il{Esperanto}, that is, they are O, I, A, and E, respectively. These categories
%were defined semantically as specified in Table~\ref{table-pos-tesniere}.\footnote{%
%  As 
% \citet[\page ]{Klein71a-u} and 
% Klein hat Beispiele wie Schlag. Die würde Tesnière aber vielleicht als
% Translationen erklären.
% \citet[\page 77]{Weber97a} points out this categorization is not without problems: in what
%  sense is \emph{Angst} `fear' a substance? Why should \emph{glauben} `believe' be a concrete
%  process? See also  \citet[Section~3.4]{Klein71a-u} for the discussion of \emph{schlagen} `to beat' and
%  \emph{Schlag} `the beat' and similar cases. Even if one assumes that \emph{Schlag} is derived from
%  the concrete process \stem{schlag} by a transfer into the category O, the assumption that such
%  Os stand for concrete substances is questionable.
%}
\begin{table}
\begin{tabular}{lll}
\lsptoprule
         & 物质 & 过程\\
\midrule
具体 & 名词 & 动词 \\
抽象 & 形容词 & 副词\\
%         & substance & process\\
%\midrule
%concrete & noun & verb \\
%abstract & adjective & adverb\\
\lspbottomrule
\end{tabular}
\caption{\label{table-pos-tesniere}\tes 提出的语义驱动的词类类型}
%\caption{\label{table-pos-tesniere}Semantically motivated part of speech classification by \tes}
\end{table}%
\tes 认为这些范畴是普遍的,而且认为这些范畴互相依存的方式也是有限制的。
%\tes assumed these categories to be universal and suggested that there are constraints in which way these categories may
%depend on others.

根据\tes ,名词和副词可以依存于动词,形容词依存于名词,而副词依存于形容词或副词。
%According to \tes, nouns and adverbs may depend on verbs, adjectives may depend on nouns, and adverbs
%may depend on adjectives or adverbs.
%% \todostefan{S: This is a distributional definition of POS.
%% This is very good definition for this time, directly inspired by the definition of POS of Jespersen
%% 1924.} 
这个情景在图\vref{fig-tesniere-general-stemma}中的普通依存图中有所描述。`*' 表示Es中的依存关系的任意数量。
%This situation is depicted in the general dependency graph in
%Figure~\vref{fig-tesniere-general-stemma}. The `*' means that there can be an arbitrary number of
dependencies between Es.
\begin{figure}
\begin{forest}
[I [O 
     [A [E*]]]
   [E*]
]
\end{forest}
\caption{\label{fig-tesniere-general-stemma} 根据\tes 的依存关系的普遍配置\\(I = verb, O~=~noun, A = adjective, E = adverb)}
%\caption{\label{fig-tesniere-general-stemma}Universal configuration for dependencies according to
%  \tes\\(I = verb, O~=~noun, A = adjective, E = adverb)}
\end{figure}%
当然,要找到依存于动词的形容词和依存于名词的句子(动词)是比较容易的。这类情况在\tes 的系统中由所谓的转移(transfers)来处理。
%It is of course easy to find examples in which adjectives depend on verbs and sentences (verbs)
%depend on nouns. Such cases are handled via so-called \emph{transfers} in \tes's
%system.
%% \footnote{%
%%  \citet[\page 32]{Weber97a} assumes that the adjective \emph{schön} in (i) is an E, that is, an adverb.
%% \ea
%% \gll Das Buch ist schön.\\
%%      the book is nice\\
%% \mytrans{The book is nice.}  
%% \z
%% Even though \tes is using the E category for nominal
%% expressions like \emph{the whole day}, treating \emph{schön} as adverb seems inappropriate. Time
%% expressions like \emph{the whole day} are used adverbially, so \tes encodes grammatical functions in
%% his symbols. However, the grammatical function `adverbial´ is inappropriate for the adjective in (i),
%% rather it should be called a predicative element.
%% } 
而且,这个范畴的集合中没有连词、限定词和介词。对于这些成分与他们的从属词的组合来说,\tes 使用了特殊的组合性关系:联结和转移。我们将在下面介绍这些概念。
%Furthermore,
%conjunctions, determiners, and prepositions are missing from this set of categories. For the
%combination of these elements with their dependents \tes used special combinatoric relations:
%junction and transfer. We will deal with these in the following subsection.

\subsection{联系、联结与转用}
%\subsection{Connection, junction, and transfer}
\label{sec-connection-junction-transfer}

 \citet{Tesniere59a-u}提出了结点间的三个基本关系:联系、联结与转移。联系是我们在之前的章节中就介绍过的中心语与从属语之间的简单关系。联结是在并列分析中起到重要作用的特殊关系,而转移是允许我们变换词或短语的范畴的工具。
% \citet{Tesniere59a-u} suggested three basic relations between nodes: connection, junction, and
%transfer. Connection is the simple relation between a head and its dependents that we have already
%covered in the previous sections. Junction is a special relation that plays a role in the analysis
%of coordination and transfer is a tool that allows one to change the category of a lexical item
%or a phrase. 

\subsubsection{联结}
%\subsubsection{Junction}
\label{sec-dg-coordination}

图\vref{fig-dg-junction}说明了联结关系:两个并列成分John和Mary通过连词and连到了一起。
%Figure~\vref{fig-dg-junction} illustrates the junction relation: the two conjuncts \emph{John}
%and \emph{Mary} are connected with the conjunction \emph{and}.\LATER{read \citep{Osborne2006a-u}}
%
% moved to top.
有趣的是,并列成分都连接到中心语laugh上。
%It is interesting to note that both of the conjuncts are connected to the head \emph{laugh}. 
\begin{figure}
\begin{forest}
dg edges
[V 
      [N [John;John] ]
      [Conj,dg junction [and;和]]
      [N [Mary;Mary] ]
      [laugh;笑]]
\end{forest}
\caption{\label{fig-dg-junction}对并列应用特殊的联结(junction)关系的分析}
%\caption{\label{fig-dg-junction}Analysis of coordination using the special relation \emph{junction}}
\end{figure}%

对于两个并列名词的情况,我们得到图\vref{fig-dg-junction-all-girls-and-boys}中的依存图。
%In the case of two coordinated nouns we get dependency graphs like the one in Figure~\vref{fig-dg-junction-all-girls-and-boys}.
\begin{figure}
\begin{forest}
dg edges
[V 
      [N [Det,name=det [All;所有] ]
         [girls;女孩] ]
      [Conj,dg junction [and;和]]
      [N,name=n [boys;男孩] ]
      [dance;跳舞]]
\draw (n.south)--(det.north);
\end{forest}
\caption{\label{fig-dg-junction-all-girls-and-boys}对并列应用特殊的联结(junction)关系的分析}
%\caption{\label{fig-dg-junction-all-girls-and-boys}Analysis of coordination using the special relation \emph{junction}}
\end{figure}%
所有的名词都联接到支配动词上,而且所有的名词都支配相同的限定词。
%Both nouns are connected to the dominating verb and both nouns dominate the same determiner.

对并列的另一种分析方法是将连词看作是中心词,并列的成分看作是它的从属词。\footnote{%
我这里不用\tes 的范畴标签,这样读者就不用将I翻译成V,以及O翻译成N。}
这个方法的唯一问题是连词的范畴。它不能是Conj,因为支配动词不选择Conj,而是N。这里可以使用的方法基本上与范畴语法中使用的是一样的(请参阅\ref{sec-coordination-cg}):范畴语法中连词的范畴是(X\bs X)/X。我们有一个函数,它带有同样范畴的两个论元,这个组合的结果是一个跟这两个论元具有相同范畴的对象。要把这个方法翻译到依存语法中,我们就会得到图\vref{fig-dg-coordination-with-conjunction-as-head}中描述的分析,而不是图\ref{fig-dg-junction}和图\ref{fig-dg-junction-all-girls-and-boys}中的分析。
%An alternative to such a special treatment of coordination would be to treat the conjunction as the
%head and the conjuncts as its dependents.\footnote{%
%I did not use \tes's category labels here to spare the
%reader the work of translating I to V and O to N.}
%The only problem of such a proposal would be the
%category of the conjunction. It cannot be Conj since the governing verb does not select a Conj, but
%an N. The trick that could be applied here is basically the same trick as in Categorial Grammar (see
%Section~\ref{sec-coordination-cg}): the category of the conjunction in Categorial Grammar is (X\bs
%X)/X. We have a functor that takes two arguments of the same category and the result of the
%combination is an object that has the same category as the two arguments. Translating this approach
%to Dependency Grammar, one would get an analysis as the one depicted in
%Figure~\vref{fig-dg-coordination-with-conjunction-as-head} rather than the ones in
%Figure~\ref{fig-dg-junction} and Figure~\ref{fig-dg-junction-all-girls-and-boys}.
%\todostefan{Ist das
%  auch der Ansatz von  \citet[\page 625]{Hellwig2003a}?}
\begin{figure}
\hfill
\begin{forest}
dg edges
[V 
      [N [N [John;John] ]
         [and;和]
         [N [Mary;Mary] ] ]
      [laugh;笑]]
\end{forest}
\hfill
\begin{forest}
dg edges
[V 
      [N [Det,name=det [All;所有] ]
         [N [girls;女孩们] ]
            [and;和]
            [N [boys;男孩们] ] ]
      [dance;跳舞]]
\end{forest}
\hfill\mbox{}
\caption{\label{fig-dg-coordination-with-conjunction-as-head}无联结(junction)的将连词看作是中心语的并列的分析}
%\caption{\label{fig-dg-coordination-with-conjunction-as-head}Analysis of coordination without
%  \emph{junction} and the conjunction as head}
\end{figure}%
这个句子的图看起来太奇怪了,因为限定词和两个并列成分都依存于连词,但是由于这两个N选择了Det,并列的结果也是这样的。在范畴语法的概念中,连词的范畴应该是((NP\bs Det)\bs (NP\bs Det))/(NP\bs Det),因为X实例为具有范畴(NP\bs Det)的名词,在这个分析中名词是中心语,限定词是从属语。
%The figure for \emph{all girls and boys} looks rather strange
%% \todostefan{S: see  \citew{Kahane97a-u} and  \citew{Sangati2012a-u} for a formalization with a
%%   coordinative bubble.} 
%since both the determiner and the
%two conjuncts depend on the conjunction, but since the two Ns are selecting a Det, the same is true
%for the result of the coordination. In Categorial Grammar notation, the category of the conjunction
%would be ((NP\bs Det)\bs (NP\bs Det))/(NP\bs Det) since X is instantiated by the nouns which would
%have the category (NP\bs Det) in an analysis in which the noun is the head and the determiner is the
%dependent.

需要注意的是,这两种方法都需要给主语—动词的一致关系提供解释\isc{主谓一致}\is{agreement}。\tes 最开始的分析假定了动词和连词之间有两种依存关系。\footnote{%
 \citet[\page 467]{Eroms2000a}指出了这个一致性问题,并描述了事实。在他的分析中,他将第一个连词连接到统治中心语上,尽管看起来更合适的做法应该是假定一个内在构造的并列结构,然后连接到更高层的连词上。
}因为连词是单数的,动词是复数的,在这个方法中根据依存关系得到的模型无法模拟一致关系。
%Note that both approaches have to come up with an explanation of subject--verb agreement\is{agreement}. \tes's
%original analysis assumes two dependencies between the verb and the individual conjuncts.\footnote{%
%   \citet[\page 467]{Eroms2000a} notes the agreement problem and describes the facts. In his
%  analysis, he connects the first conjunct to the governing head, although it seems to be more
%  appropriate to assume an internally structured coordination structure and then connect the highest conjunction.
%} As the conjuncts are singular and the verb is plural, agreement cannot be modeled in tandem with dependency
%relations in this approach. 
%( \citet[\page 292]{Jung2003a} argues that subject-verb agreement should
%be handled as government relation and hence as a dependency.)  
如果第二种分析找到了描述联结中并列的一致属性,这样一致的事实就是算数的,而且没有任何问题。
%If the second analysis finds ways of specifying the agreement properties
%of the coordination in the conjunction, the agreement facts can be accounted for without problems.

与图\ref{fig-dg-coordination-with-conjunction-as-head}中描述的中心语方法相比的另一种方法是没有中心语的。在基于短语结构的框架下工作的几个作者提出了没有中心语的并列结构的分析。这类分析也在依存语法中被提及\citep{Hudson88a,Kahane97a}。 \citet{Hudson88a}和其他作出了相似假设的学者提出了一个针对并列的短语结构成分:
两个名词和连词被组合在一起构成一个更大的对象,它具有与任何一个组合词语不相对应的属性。
%The alternative to a headed approach as depicted in Figure~\ref{fig-dg-coordination-with-conjunction-as-head} is an unheaded one. Several authors
%working in phrase structure"=based frameworks suggested analyses of coordination without a
%head. Such analyses are also assumed in Dependency Grammar
%\citep{Hudson88a,Kahane97a}.  \citet{Hudson88a} and others who make similar assumptions assume a phrase structure component for
%coordination:\todostefan{Hudson: this is not constituent structure, since it covers non-constituent
%  coordination} 
%the two nouns and the conjunction are combined to form a larger object which has properties which
%do not correspond to the properties of any of the combined words.

相似地,基于联结的并列结构的分析为表达式的解读提出了问题。如果语义角色的指派在依存关系中发生,那么就会有如图\ref{fig-dg-junction}的图式的问题了。因为laugh的语义角色不能同时被John和Mary填充。相反,它由一个实体填充,即指向包括John和Mary的那个集合。这个语义表示属于短语John and Mary,而且在这个并列结构中,这个最高实体的自然候选者是and,因为它涵盖了John和Mary的意义:\relation{and}(\relation{John},~\relation{Mary})。
%Similarly, the junction"=based analysis of coordination poses problems for the interpretation of the
%representations. If semantic role assignment happens in parallel to dependency relations, there would be a
%problem with graphs like the one in Figure~\ref{fig-dg-junction}, since the semantic role of \emph{laugh} cannot be
%filled by \emph{John} and \emph{Mary} simultaneously. Rather it is filled by one entity, namely the
%one that refers to the set containing John and Mary. This semantic representation would belong to
%the phrase \emph{John and Mary} and the natural candidate for being the topmost entity in this
%coordination is \emph{and}, as it embeds the meaning of \emph{John} and the meaning of
%\emph{Mary}: \relation{and}(\relation{John},~\relation{Mary}).

这类联结也适用于动词的并列。但是,也不是没有问题的,因为附加语的辖域可以覆盖到离它最近的连词,或者是整个并列结构。例如下面 \citet[\page 217]{Levine2003a}中的句子:
%Such junctions are also assumed for the coordination of verbs. This is, however, not without problems,
%since adjuncts can have scope over the conjunct that is closest to them or over the whole
%coordination. An example is the following sentence from  \citet[\page 217]{Levine2003a}:
\ea
\gll Robin came in, found a chair, sat down, and whipped off her logging boots in exactly thirty seconds flat.\\
Robin 来 进 找到 一 椅子 坐 下 和 拿 开 她的 测井 靴子 在 确定地 三十 秒 干脆\\
\mytrans{Robin进来,找到了一把椅子,坐下来,并在三十秒内就把她的靴子脱下来了。}
\z
附加语in exactly thirty seconds flat可以指向(\mex{1}a)中的whipped off her logging boots,或者像(\mex{1}b))中那样覆盖所有三个连词:
%The adjunct \emph{in exactly thirty seconds flat} can refer either to \emph{whipped off her logging
%  boots} as in (\mex{1}a) or scope over all three conjuncts together as in (\mex{1}b):
\eal
\ex 
\gll Robin came in, found a chair, sat down, and [[pulled off her logging boots] in exactly thirty seconds flat].\\
     Robin 来   进   找到  一 椅子    坐  下    和  \hspaceThis{[[}拿 开 她的 测井 靴子 在 确定地 三十 秒 干脆\\
\mytrans{Robin进来,找到了一把椅子,坐下来,并在三十秒内就把她的靴子脱下来了。}
\ex\label{ex-Robin-flat-VP}
\gll Robin [[came in, found a chair, sat down, and pulled off her logging boots] in exactly thirty seconds flat].\\
     Robin \hspaceThis{[[}来 进 找到 一 椅子 坐 下 和 拿 开 她的 测井 靴子 在 确定地 三十 秒 干脆\\
\mytrans{Robin在三十秒内进来,找到了一把椅子,坐下来,并把她的靴子脱下来了。}
\zl
图\vref{fig-dg-adjunct-attachment-wrong}中的泰尼埃式的分析对应于(\mex{1}),而如图\vref{fig-dg-adjunct-attachment-right}中将连词作为中心语的分析对应于(\mex{0}b)。
%The Tesnièreian analysis in Figure~\vref{fig-dg-adjunct-attachment-wrong} corresponds to (\mex{1}), while an analysis that
%treats the conjunction as the head as in Figure~\vref{fig-dg-adjunct-attachment-right} corresponds
%to (\mex{0}b).
\ea
\gll Robin came in in exactly thirty seconds flat and Robin found a chair in exactly thirty seconds flat
and Robin pulled off her logging boots in exactly thirty seconds flat.\\
Robin 进 来 在 确定地 三十 秒 干脆 和 Robin 找到 一 椅子 在 确定地 三十 秒 干脆 和 Robin 拿 开 她的 测井 靴子 在 确定地 三十 秒 干脆\\
\mytrans{Robin在三十秒内进来,在三十秒内找到了一把椅子,并在三十秒内把她的靴子脱下来了。}
\z
当附加语分别指向每个连词的,而不是如(\ref{ex-Robin-flat-VP})中由动词短语表示的指向一个累积的事件的时候,就会得到(\mex{0})中的解读。
%The reading in (\mex{0}) results when an adjunct refers to each conjunct individually rather then referring to a
%cumulative event that is expressed by a verb phrase as in (\ref{ex-Robin-flat-VP}).
\begin{figure}
\vspace{-1cm}%
\begin{forest}
dg edges
[\phantom{V}
  [V, l sep+=2ex, name=v1, no edge
    [N,name=n1 [Robin;Robin]]
    [came;来]
    [Part [in;进] ] ]
  [Conj,dg junction [and;和]]
  [V, l sep+=2ex, name=v2, no edge [found;找到]
     [N 
       [Det [a;一]]
       [chair;椅子]]
     [P, dg adjunct, name=p [in;在]
        [N 
          [Det [thirty;三十]]
          [seconds;秒]]]
]]
\draw (v2.south)--(n1.north)
      (v1.south)--(p.north);
\end{forest}
\caption{\label{fig-dg-adjunct-attachment-wrong}带有联结关系的动词并列的分析}
%\caption{\label{fig-dg-adjunct-attachment-wrong}Analysis of verb coordination involving the junction relation}
\end{figure}%
\begin{figure}
\begin{forest}
dg edges
[V, 
  [N,name=n1 [Robin;Robin]]
  [V
    [came;来]
    [Part [in;在] ] ]
  [and;和]
  [V [found;找到]
     [N 
       [Det [a;一]]
       [chair;椅子]] ]
  [P, dg adjunct [in;在]
     [N 
       [Det [thirty;三十]]
       [seconds;秒]]]]
\end{forest}
\caption{\label{fig-dg-adjunct-attachment-right}带有联系关系的动词并列的分析}
%\caption{\label{fig-dg-adjunct-attachment-right}Analysis of verb coordination involving the connection relation}
\end{figure}%

 \citet[\page 217]{Levine2003a}讨论了连接到 \citet*{BMS2001a}提出的提取\isc{提取}\is{extraction}的\hpsgc 分析的这些句子。Bouma, Malouf \& Sag提出了,附加语作为中心语的从属语而从词汇的角度被引入的想法。因为附加语按照词汇的方法来引入,并列结构基本上与泰尼埃式的分析具有相同的结构。有可能会想到正确得到语义组成成分的方式,即使句法并不对应于语义依存关系(请参阅\citealp{Chaves2009a}的观点),但是清楚的是,针对语义中的情况,从句法结构得到语义是更简单的。
% \citet[\page 217]{Levine2003a} discusses these sentences in connection to the \hpsg analysis of
%extraction\is{extraction} by  \citet*{BMS2001a}. Bouma, Malouf \& Sag suggest an analysis in which adjuncts are
%introduced lexically as dependents of a certain head. Since adjuncts are introduced lexically, the
%coordination structures basically have the same structure as the ones assumed in a
%Tesnièreian analysis. It may be possible to come up with a way to get the semantic composition right
%even though the syntax does not correspond to the semantic dependencies (see \citealp{Chaves2009a} for
%suggestions), but it is clear that it is simpler to derive the semantics from a syntactic structure
%which corresponds to what is going on in semantics. 

\subsubsection{转用}
%\subsubsection{Transfer}
\label{sec-transfer-dg}

在\tes 的系统中,转用被用于主要范畴(如名词)中的中心语与次要范畴(如介词)的词相组合的词或短语。另外,转用可以在不需要任何其他词的参与下,变换词或短语的范畴。
%Transfers\todostefan{Liu Haitao: please refer to  \citew{Werner93a-u}.} are used in \tes's system for the combination of words or phrases with a head of %one of
%the major categories (for instance nouns) with words in minor categories (for instance
%prepositions). In addition, transfers can transfer a word or phrase into another category without
%any other word participating.

图\ref{fig-transfer-in-das-traumboot}就是转用的一个例子。
%Figure~\ref{fig-transfer-in-das-traumboot} shows an example of a transfer.
\begin{figure}
\begin{forest}
[steigt (I)\\
 进入\hspaceThis{(I)}
%enter\hspaceThis{(I)}
   [er (O)\\他\hspaceThis{(O)}]
%    [er (O)\\he\hspaceThis{(O)}]
   [E
     [in\\\textsc{prep}]
%       [in\\in]
     [Traumboot (O)\\梦.船\hspaceThis{(O)}
%        [Traumboot (O)\\dream.boat\hspaceThis{(O)}
       [das\\这]
%          [das\\das]
       [Liebe (O)\\爱\hspaceThis{(O)}
%            [Liebe (O)\\love\hspaceThis{(O)}
         [der\\\textsc{det}]]] ] ] 
%              [der\\the]]] ] ]
\end{forest}
%% \begin{forest}
%%     [un exemple
%%       [A
%%         [frapp]
%%          [ant, dg transfer] ] ]
%%     \end{forest}
\caption{\label{fig-transfer-in-das-traumboot}从 \citew[\page 83]{Weber97a}而来的转用的例子}
%\caption{\label{fig-transfer-in-das-traumboot}Transfer with an example adapted from
%   \citew[\page 83]{Weber97a}}
\end{figure}%
介词in引发了范畴的变化:由于Traumboot(梦船)是一个O(名词),介词与名词的组合是一个E。这个例子说明了\tes 使用了语法范畴来对语法功能进行编码。在HPSG这样的理论中,有一个清晰的区别:一方面,这里有关词性的信息,另一方面,也有作为修饰语和谓语的成分的函项。修饰语函项被编码为修饰性特征\textsc{mod},它是独立于词性的。由此,这就可以有修饰性和非修饰性的形容词、修饰性和非修饰性的介词短语,以及修饰性和非修饰性的名词短语等。对于手边的例子,我们可以假设一个带有方向性语义的介词选择了NP。介词是带有填充了\modvc 的PP的中心语。
%The preposition \emph{in} causes a category change: while \emph{Traumboot} `dream boat' is an O (noun), the
%combination of the preposition and the noun is an E. The example shows that \tes used the
%grammatical category to encode grammatical functions. In theories like HPSG there is a clear
%distinction: there is information about part of speech on the one hand and the function of elements as
%modifiers and predicates on the other hand. The modifier function is encoded by the selectional
%feature \textsc{mod}, which is independent of the part of speech. It is therefore possible to have modifying and
%non-modifying adjectives, modifying and non-modifying prepositional phrases, modifying and
%non-modifying noun phrases and so on. For the example at hand, one would assume a preposition with
%directional semantics that selects for an NP. The preposition is the head of a PP with a filled \modv.

使用转用的另一个方面是形态学。比如说,法语\il{French}frappant(惊人的)的派生是通过词根frapp加上后缀\suffix{ant}构成的,如图\vref{fig-transfer-frappant}所示。
%Another area in which transfer is used is morphology. For instance, the derivation of French\il{French}
%\emph{frappant} `striking' by suffixation of \suffix{ant} to the verb stem \emph{frapp} is shown in
%Figure~\vref{fig-transfer-frappant}.
\begin{figure}
\hfill
\begin{forest}
    [un exemple
      [A
        [frapp]
         [ant, dg transfer] ] ]
    \end{forest}
\hfill
\begin{forest}
dg edges
      [Adj
        [V [frapp]]
        [ant] ]
    \end{forest}
\hfill\mbox{}
\caption{\label{fig-transfer-frappant}形态学中的转用和作为正常依存的重新概念化}
%\caption{\label{fig-transfer-frappant}Transfer in morphology and its reconceptualization as
%  normal dependency}
\end{figure}%
这类转用可以被看作是普通连接关系,如果词缀被看作是中心语。在实现形态学和构式形态学领域的形态学家反对这类基于语素的分析,因为他们在会话中包括了很多空成分,如动词play到名词play的转化(请参阅图\vref{fig-transfer-play})。
%Such transfers can be subsumed under the general connection relation if the affix is treated as
%the head. Morphologists working in realizational morphology and construction morphology argue
%against such morpheme"=based analyses since they involve a lot of empty elements for conversions as
%for instance the conversion of the verb \emph{play} into the noun \emph{play} (see Figure~\vref{fig-transfer-play}). 
\begin{figure}
\hfill
\begin{forest}
      [O
        [play]
         [\_, dg transfer] ]
    \end{forest}
\hfill
\begin{forest}
dg edges
      [N
        [V [play]]
        [\trace] ]
    \end{forest}
\hfill\mbox{}
\caption{\label{fig-transfer-play}作为从I(动词)到O(实体)的转用和范畴N作为中心语的空成分的依存关系的转移}
%\caption{\label{fig-transfer-play}Conversion as transfer from I (verb) to O (substantive)
%  and as dependency with an empty element of the category N as head}
\end{figure}%
由此,词汇规则被用来表示HPSG这类理论中的派生与对话。HPSG的词汇规则基本上等同于一元分支规则(请参阅第\pageref{passiv-lr-mit-dtr}页有关\ref{passiv-lr-mit-dtr}的讨论和\ref{Abschnitt-leere-Elemente-LRs-Transformations})。词缀整合进词汇规则或者整合进区分项的形态形式的功能实现都是由词汇规则允准的。
%Consequently, lexical rules are assumed for derivations and conversions in theories like HPSG. HPSG lexical rules are basically equivalent to
%unary branching rules (see the discussion of (\ref{passiv-lr-mit-dtr}) on
%page~\pageref{passiv-lr-mit-dtr} and
%Section~\ref{Abschnitt-leere-Elemente-LRs-Transformations}). The affixes are integrated into the
%lexical rules or into realization functions that specify the morphological form of the item that is licensed by the lexical rule.

总结一下,这里所说的转用是指
%Concluding it can be said that transfer corresponds to 
\begin{itemize}
\item 如果一个词或者短语与另一个词相组合,那么就应用二叉的短语结构规则,
\item 如果一个短语没有通过任意一个额外的成分而转换到另一个范畴中,那么就应用一元的短语结构规则或者带有空中心语的二叉的短语结构规则,或者
\item 如果一个词或词根匹配到一个词或者词根上,那么就应用一个(一元)的词汇规则。
%\item binary"=branching phrase structure rules,
%if a word or phrase is combined with another word, 
%\item unary phrase structure rules or binary branching phrase structure rules together with an empty head if
%  a phrase is converted to another category without any additional element present or
%\item a (unary) lexical rule if a word or stem is mapped to a word or a stem.
\end{itemize}
关于\tes 的转用规则和短语成分规则的关系的进一步讨论请参阅 \citew[\S~4.9.1--4.9.2]{OK2015a}。Osborne \& Kahane指出,转用规则可以用来模拟离心结构,即那些没有中心语的结构。更多有关无中心语的构式内容请参阅\ref{sec-headless-constructions-dg}。
%For further discussion of the relation between \tes's transfer rules and constituency rules see
% \citew[Section~4.9.1--4.9.2]{OK2015a}. Osborne \& Kahane point out that transfer rules can be used
%to model exocentric constructions, that is, constructions in which there is no single part that
%could be identified as the head. For more on headless constructions see Section~\ref{sec-headless-constructions-dg}.

\subsection{辖域}
%\subsection{Scope}

正如 \citet[\page lix]{OK2015a}所指出的,\tes 使用了所谓的多图\isc{多图|(}\is{polygraph|(}来表示辖域关系。所以说,由于例(\mex{1})中的that you saw yesterday是指red cars,而不单是cars,这通过red和cars的联系而不是单个成分开始的线来表示\citep[\page 150, Stemma~149]{Tesniere2015a-not-crossreferenced}。
%As  \citet[\page lix]{OK2015a} point out, \tes uses so-called polygraphs\is{polygraph|(} to represent s\textsc{cop}al
%relations. So, since \emph{that you saw yesterday} in (\mex{1}) refers to \emph{red cars} rather
%than \emph{cars} alone, this is represented by a line that starts at the connection between
%\emph{red} and \emph{cars} rather than on one of the individual elements \citep[\page 150, Stemma~149]{Tesniere2015a-not-crossreferenced}.
\ea
\gll red cars that you saw yesterday\\
红色 汽车 \textsc{rel} 你 看见 昨天\\
\mytrans{你昨天看见的红色汽车}
\z
\tes 的分析如图\vref{fig-tesniere-scope}的左边表达式的描述。值得指出的是,这个表达式对应于图\ref{fig-tesniere-scope}右边的短语结构树。
%\tes's analysis is depicted in the left representation in Figure~\vref{fig-tesniere-scope}. It is worth noting that this representation
%corresponds to the phrase structure tree on the right of Figure~\ref{fig-tesniere-scope}.
\begin{figure}
\hfill
\begin{forest}
%baseline
[cars, l sep=9ex
  [red,edge label={node[midway,above,font=\small]{B~~~~~}}]
  [that you saw yesterday, no edge]]
%\draw (-2,-3) to[grid with coordinates] (3,1);
\draw (-.3,-1) -- (.6,-2)  node[font=\small,midway,above] {~~~A};
\end{forest}
\hfill
\begin{forest}
%baseline, 
word tier
[A
  [B
    [red]
    [cars] ]
  [that you saw yesterday]]
\end{forest}
\hfill\mbox{}
\caption{\label{fig-tesniere-scope}\tes 表示辖域的方式以及由 \citet[\page lix]{OK2015a}提出的基于短语结构分析的比较}
%\caption{\label{fig-tesniere-scope}\tes's way of representing scope and the comparison with phrase structure"=based analyses by  \citet[\page lix]%{OK2015a}}
\end{figure}%
%<HERE>
在red和cars之间的组合B对应于右手边图中的B结点,而且red cars和that you saw yesterday的组合A对应于A结点。所以说,在短语结构语法中被清晰表示并指派了名称的对象在\tes 的分析中是没有名字的,但是由于测谎仪的假说,可以指向这些组合。\isc{测谎仪}\is{polygraph}也请参阅图\ref{fig-small-children-are-playing-outside}的讨论,它显示出了Hudson为了模拟语义关系而提出的额外结点。
%The combination B between \emph{red} and \emph{cars} corresponds to the B node in the right-hand figure and the
%combination A of \emph{red cars} and \emph{that you saw yesterday} corresponds to the A node. So,
%what is made explicit and is assigned a name in phrase structure grammars remains nameless in \tes's
%analysis, but due to the assumption of polygraphs, it is possible to refer to the
%combinations.\is{polygraph|)} See also the discussion of Figure~\ref{fig-small-children-are-playing-outside}, which shows additional nodes
%that Hudson assumes in order to model semantic relations.
%
%% \subsection{Further issues}
%
% Da müsste man dann wohl A zu E werden lassen ...
%
%%  \citet[\page 42]{Weber92a} points out that embeddings cannot be depicted in \tes's system. The
%% example (\mex{1}a) has two adjectives that both depend on the noun. However, (\mex{1}b) is
%% different: Here \emph{alte} `old' depends on \emph{englische} `english'.
%% \eal
%% \ex 
%% \gll Pascal sammelt altes, gebrauchtes Geschirr.\\
%%      Pascal collects old used table.wear\\
%% \ex
%% \gll Pascal sammelt alte englische Kupferstiche.\\
%%      Pascal collects old English \textsc{cop}perplate.prints\\
%% \zl
%
%% \subsection{Topological Dependency Grammar}
%
%%  \citet{GK2001a}
%
%% \subsection{Unification Dependency Grammar}
%
%%  \citet{Hellwig86a-u,Hellwig2003a,Hellwig2006a}

\section{总结}
%\section{Summary and classification}
%
% Uzonyi2003a-u: Konstituenz vs. Dependenz: Zur Kompatibilität und Konvertierbarkeit der beiden Strukturprinzipien
%
% Es muss einen Valenzträger geben.

依存语法的支持者们强调的一点是依存语法比短语结构语法简单多了,因为它们有更少的结点,而且一般的概念更容易被捕捉到(比如说\citealp[\S~3.2, \S~7]{Osborne2014a-u})。这是事实:依存语法适用于导论课程中语法的讲解。但是,正如 \citet[\page 285]{SR2012a}在一篇相当具有一般性的讨论中所指出的,简单的句法的代价是复杂语义及其他成分的缺失。所以说,除了依存句法中描述的依存结构,人们还需要其他层面的信息。一个层面就是语义层,另一个是线性顺序。考虑到线性顺序,依存语法有两个选项:假设连续的成分,即可投射结构或者允许非连续成分。这些选项将在后面的章节中讨论。\ref{sec-dependency-vs-constituency}比较了依存语法和短语结构语法。它也指出了非可投射结构可以在HPSG这类理论中被模拟。语义的整合在\ref{sec-dg-daughters-mothers}进行了讨论,而且逐渐清楚的是,一旦考虑到其他层面,依存语法就不一定比短语结构语法简单了。
%Proponents of Dependency Grammar emphasize the point that Dependency Grammar is much simpler than
%phrase structure grammars, since there are fewer nodes and the general concept is more easy to
%grasp (see for instance \citealp[Section~3.2, Section~7]{Osborne2014a-u}). This is indeed true: Dependency Grammar is well-suited for teaching %grammar in
%introductory classes. However, as  \citet[\page 285]{SR2012a} point out in a rather general
%discussion, simple syntax has the price of complex semantics and vice versa. So, in addition to
%the dependency structure that is described in Dependency Syntax, one needs other levels. One level
%is the level of semantics and another one is linearization. As far as linearization is concerned,
%Dependency Grammar has two options: assuming continuous constituents, that is, projective structures,
%or allowing for discontinuous constituents. These options will be discussed in the following subsections.
%Section~\ref{sec-dependency-vs-constituency} compares dependency grammars with phrase structure
%grammars and shows that projective Dependency Grammars can be translated into phrase structure
%grammars. It also shows that non"=projective structures can be modeled in theories like HPSG.
%The integration of semantics is discussed in Section~\ref{sec-dg-daughters-mothers} and it will become clear that once other
%levels are taken into account, Dependency Grammars are not necessarily simpler than phrase structure grammars.

\subsection{次序化}
%\subsection{Linearization}
\label{sec-linearization-problems-dg}
\label{sec-dg-multiple-frontings}

我们在这一章看到了许多次序化的方法。许多人只提出了依存图和根据拓扑模型建立的一些次序化方法。正如在\ref{sec-nld-dg}所讨论的,允许中心语及其从属语的非连续排列好像打开了潘多拉的盒子。我已经讨论了 \citet{Kunze68a-u}、 \citet{Hudson97a,Hudson2000a}、 \citet*{KNR98a}和 \citet{GO2009a}提出的非局部依存的分析。
%We have seen several approaches to linearization in this chapter. Many just assume a dependency
%graph and some linearization according to the topological fields model. As has been argued in
%Section~\ref{sec-nld-dg}, allowing discontinuous serialization of a head and its dependents opens up
%Pandora's box. I have discussed the analysis of nonlocal dependencies by  \citet{Kunze68a-u},  \citet{Hudson97a,Hudson2000a},  \citet*{KNR98a},
%and  \citet{GO2009a}.
%\todostefan{S: rather Hudson 2000, Khanae et al. 1998,  \citew{DD2001a-u}, etc.} 
除了Hudson之外,其余学者都认为只有在为了避免非连续性的情况下才会假设中心语的依存成分上升到支配中心语的位置。但是,我们有理由认为前置应该按照特殊的机制来处理,即使在允许连续的序列化的情况下。比如说,例(\mex{1})中的例子是否有歧义就不能直接得到解释:
%With the exception of Hudson those authors assume that dependents of a head rise
%to a dominating head only in those cases in which a discontinuity would arise otherwise. However, there
%seems to be a reason to assume that fronting should be treated by special mechanisms even in cases
%that allow for continuous serialization. For instance, the ambiguity or lack of ambiguity of the
%examples in (\mex{1}) cannot be explained in a straightforward way:
\eal
\ex\label{ex-oft-liest-er-das-buch-nicht} 
\gll Oft liest er das Buch nicht.\\
     经常 读 他 \textsc{det}  书 不\\
\glt \quotetrans{经常的情况是他不读书。} 或者\\
     \quotetrans{他经常读书,事实不是这样的。}
%     often reads he the book not\\
%\mytrans{It is often that he does not read the book.' or\}
%     `It is not the case that he reads the book often.'
\ex
\gll dass er das Buch nicht oft liest\\
      \textsc{comp} 他 \textsc{det} 书 不 经常 读\\
\mytrans{他经常读书,事实不是这样的。}
 %    that he the book not often reads\\
%\mytrans{It is not the case that he reads the book often.}
\ex
\gll dass          er das          Buch oft nicht liest\\
     \textsc{comp} 他 \textsc{det} 书   经常  不    读\\
\mytrans{经常的情况是,他不读书。}
%     that he the book often not reads\\
%\mytrans{It is often that he does not read the book.}
\zl
这三个例子的情况是,只有(\mex{0}a)是有歧义的。即使(\mex{0}c)具有相同的语序,只要考虑到oft(经常)和nicht(不),这个句子就不是歧义的。所以,附加语的前置是歧义的原因。(\mex{0}a)的依存图如图\vref{fig-oft-liest-er-das-buch-nicht-dg}所示。
%The point about the three examples is that only (\mex{0}a) is ambiguous. Even though (\mex{0}c) has
%the same order as far as \emph{oft} `often' and \emph{nicht} `not' are concerned, the sentence is
%not ambiguous. So it is the fronting of an adjunct that is the reason for the ambiguity. The
%dependency graph for (\mex{0}a) is shown in Figure~\vref{fig-oft-liest-er-das-buch-nicht-dg}.
\begin{figure}
\centering
\begin{forest}
dg edges
[V
  [Adv, dg adjunct [oft;经常] ] 
  [liest;读] 
  [N [er;他] ]
  [N 
    [Det [das;\textsc{det}] ]
    [Buch;书] ]
  [Adv, dg adjunct [nicht;不]] ]
\end{forest}
\caption{\label{fig-oft-liest-er-das-buch-nicht-dg}\emph{Oft liest er das Buch nicht.}(他不经常读这本书)的依存图}
%\caption{\label{fig-oft-liest-er-das-buch-nicht-dg}Dependency graph for \emph{Oft liest er das Buch
%    nicht.} `He does not read the book often.'}
\end{figure}%
当然,(\mex{0}b)和(\mex{0}c)的依存关系不是不同的。图是相同的,只有语序序列是不同的。所以说,辖域的不同不能从依存关系中推导出来,而且像(\mex{1})的复杂声明是必要的:
%Of course the dependencies for (\mex{0}b) and (\mex{0}c) do not differ. The graphs would be the
%same, only differing in serialization. Therefore, differences in scope could not be derived from the
%dependencies and complicated statements like (\mex{1}) would be necessary:
\ea
如果从属语在\vfc 中次序化了,它可以跨域,而且位于它所从属的中心语的所有其他附加语下面。
%If a dependent is linearized in the \vf it can both scope over and under all other adjuncts of the
%head it is a dependent of.
\z
 \citet[\page 320]{Eroms85a}提出了否定的分析,其中否定被当作中心语来处理;也就是说,例(\mex{1})中的句子具有图\vref{dg-adv-head}中的结构。\footnote{%
但是请参阅 \citew[\S~11.2.3]{Eroms2000a}。
}
% \citet[\page 320]{Eroms85a} proposes an analysis of negation in which the negation is treated as the head;
%that is, the sentence in (\mex{1}) has the structure in Figure~\vref{dg-adv-head}.\footnote{%
%But see  \citew[Section~11.2.3]{Eroms2000a}.
%}
\ea
\gll Er kommt nicht.\\
     他 来 不\\
\mytrans{他没来。}
%     he comes not\\
%\mytrans{He does not come.}
\z
\begin{figure}
\begin{forest}
dg edges
[Adv 
  [V [N [er;他]]
     [kommt;来]]
  [nicht;不]] 
\end{forest}
\caption{\label{dg-adv-head}根据 \citet[\page 320]{Eroms85a}的否定的分析}
%\caption{\label{dg-adv-head}Analysis of negation according to  \citet[\page 320]{Eroms85a}}
\end{figure}%
% S:
%this clearly not a syntactic structure. You must use different conventions. See for instance the semantic graph of MTT.
%Moreover it is strange to propose a direct interface between semantics and word order. In mots DGs, semantics is lnked to an unordered dependency tree and this tree to the linear order.
%The necessary separation between the syntactic dependencies and the linear order is extensively discussed in the beginning of TEsnière's book.

这个分析对应于最简方案\indexmpc 中的NegP\isc{范畴!功能范畴!Neg}\is{category!functional!Neg},而且它有相同的问题:整个宾语的范畴是Adv,但是它应该是V。这是一个问题,因为更高层的谓词可以选择V,而不是Adv。\footnote{%
请参阅下面(\ref{ex-dass-er-nicht-singen-darf})中嵌套句的分析。
}
%This analysis is equivalent to analyses in the Minimalist Program\indexmp assuming a NegP\is{category!functional!Neg} and it
%has the same problem: the category of the whole object is Adv, but it should be V. This is a problem
%since higher predicates may select for a V rather than an Adv.\footnote{%
%See for instance the analysis of
%embedded sentences like (\ref{ex-dass-er-nicht-singen-darf}) below.
%}

对于成分否定或者带有成分的其他辖域来说也是一样的。比如说,例(\mex{1})的分析可以如图\vref{dg-alleged-murderer}所示。
%The same is true for constituent negation or other scope bearing elements. For example, the analysis of (\mex{1})
%would have to be the one in Figure~\vref{dg-alleged-murderer}.
\ea
\gll der angebliche Mörder\\
     \textsc{det} 被指控的 杀人犯\\
%     the alleged murderer\\
\z
\begin{figure}
\begin{forest}
dg edges
[Adj
    [Det, no edge, name=det, l+=3\baselineskip, [der;\textsc{det}] ]
  [angebliche;被指控的]
  [N, name=n [Mörder;杀人犯]]]
\draw (n.south)--(det.north);
\end{forest}
\caption{\label{dg-alleged-murderer}覆盖全域的附加语作为中心语所得到结果的分析}
%\caption{\label{dg-alleged-murderer}Analysis that would result if one considered all scope-bearing adjuncts
 % to be heads}
\end{figure}%
这个结构可能有非投射性的其他问题。\isc{可投射性}\is{projectivity} Eroms确实对限定词进行了不同的处理,所以这个非可投射性的类型对他来说不是一个问题。但是,否定的中心语分析会得到德语的所谓一致结构的非可投射性。例(\mex{1})中的句子有两种解读:在第一种解读中,否定的辖域包括singen(唱歌),而在第二种解读中包括singen darf(唱歌 允许)。
%This structure would have the additional problem of being non-projective.\is{projectivity} Eroms does treat the determiner
%differently from what is assumed here, so this type of non"=projectivity may not be a problem for
%him. However, the head analysis of negation would result in non"=projectivity in so"=called coherent
%constructions in German. The sentence in (\mex{1}) has two readings: in the first reading, the negation
%scopes over \emph{singen} `sing' and in the second one over \emph{singen darf} `sing may'.
\ea\label{ex-dass-er-nicht-singen-darf} 
\gll dass er nicht singen darf\\
     \textsc{comp} 他 不 唱歌 允许\\
\mytrans{他不被允许唱歌' 或 `他被允许不唱歌}
%     that he not sing may\\
%\mytrans{that he is not allowed to sing' or `that he is allowed not to sing}
\z
nicht(不)的辖域包括动词复杂形式的解读会得到图\vref{dg-nicht-singen-darf}中给出的非可投射结构。
%The reading in which \emph{nicht} `not' scopes over the whole verbal complex would result in the
%non-projective structure that is given in Figure~\vref{dg-nicht-singen-darf}.
\begin{figure}
\begin{forest}
dg edges
[Subjunction
  [dass;\textsc{comp}]
  [Adv
    [N, no edge, name=n, tier=n [er;他]]
    [nicht;不]
    [V,name=v 
      [V, tier=n [singen;唱歌]]
      [darf;允许]]]]
\draw (v.south)--(n.north);
\end{forest}
\caption{\label{dg-nicht-singen-darf}否定作为中心语的假说所得到结果的分析}
%\caption{\label{dg-nicht-singen-darf}Analysis that results if one assumes the negation to be a head}
\end{figure}%
Eroms还提出了一个分析,其中否定是一个词的一部分(``Wortteiläquivalent'')。但是,这里没有任何帮助,因为首先否定和动词在(\ref{ex-oft-liest-er-das-buch-nicht})中的动词二位的语境中不是相邻的,而且即使在(\ref{ex-dass-er-nicht-singen-darf})中的动词末位的语境中也是不相邻的。Eroms不得不假设,依附到否定的宾语是整个动词的复杂形式singen darf(唱歌允许),即包括两个词的复杂宾语。
%Eroms also considers an analysis in which the negation is a word part (`Wortteiläquivalent'). This
%does, however, not help here since first the negation and the verb are not adjacent in V2 contexts like
%(\ref{ex-oft-liest-er-das-buch-nicht}) and even in verb"=final contexts like
%(\ref{ex-dass-er-nicht-singen-darf}). Eroms would have to assume that the object to which the negation
%attaches is the whole verbal complex \emph{singen darf} `sing may', that is, a complex object consisting of two
%words.

这就给我们留下了图\ref{fig-oft-liest-er-das-buch-nicht-dg}中的分析的一个问题,因为我们针对不同解释的两个可能的附加语实现形式只有一个结构。这不是通过将两个可能的线性顺序简单地作为另一种语序的分析所能得到的。
%This leaves us with the analysis provided in Figure~\ref{fig-oft-liest-er-das-buch-nicht-dg} and
%hence with a problem since we have one structure with two possible adjunct realizations that
%correspond to different readings. This is not predicted by an analysis that treats the two possible
%linearizations simply as alternative orderings.
% Eroms 2000: 159 nicht is an adjunct, später dann Ergänzungskanten

Thomas Groß(p.\,c.\ 2013)提出了一个分析,其中oft并不依存于动词,而是否定。这对应于短语结构方法中的成分否定。依存图如图\vref{fig-oft-liest-er-das-buch-nicht-dg-constituent-negation}的左手边所示。
%Thomas Groß (p.\,c.\ 2013) suggested an analysis in which \emph{oft} does not depend on the verb but
%on the negation. This corresponds to constituent negation in phrase structure approaches. The
%dependency graph is shown on the left-hand side in Figure~\vref{fig-oft-liest-er-das-buch-nicht-dg-constituent-negation}.
\begin{figure}
\hfill
\begin{forest}
dg edges
[V
  [Adv, edge=dashed [oft;经常] ] 
  [liest;读] 
  [N [er;他] ]
  [N 
    [Det [das;\textsc{det}] ]
    [Buch;书] ]
  [Adv\sub{g}, dg adjunct [nicht;不]] ]
\end{forest}
\hfill
\begin{forest}
dg edges
[V, l sep+=5pt
  [N [er;他] ]
  [N 
    [Det [das;\textsc{det}] ]
    [Buch;书] ]
  [Adv, dg adjunct=4pt [nicht;不]
    [Adv [oft;经常] ] ] 
  [liest;读] ]
\end{forest}
\hfill\mbox{}
\caption{\label{fig-oft-liest-er-das-buch-nicht-dg-constituent-negation}根据Groß和动词末位变体的Oft liest er das Buch
    nicht.(他不经常读书。)的依存图}
%\caption{\label{fig-oft-liest-er-das-buch-nicht-dg-constituent-negation}Dependency graph for \emph{Oft liest er das Buch
%    nicht.} `He does not read the book often.' according to Groß and verb-final variant}
\end{figure}%
右手边的图显示了相应的动词末位句子的图。对应于成分否定的解读可以通过对比表达进行说明。但是在(\mex{1}a)中,只有oft(经常)是被否定的,oft gelesen(经常读)是在(\mex{1}b)的否定辖域中。
%The figure on the right-hand side shows the graph for the corresponding verb-final sentence. The
%reading corresponding to constituent negation can be illustrated with contrastive
%expressions. While in (\mex{1}a) it is only \emph{oft} `often' which is negated, it is \emph{oft
%  gelesen} `often read' that is in the scope of negation in (\mex{1}b).
\eal
\ex 
\gll Er hat das Buch nicht oft gelesen, sondern selten.\\
     他 \textsc{aux} \textsc{det} 书 不 经常 读     但是 很少\\
\mytrans{他不经常读书,但是偶尔读。}
%     he has the book not often read     but seldom\\
%\mytrans{He did not read the book often, but seldom.}
\ex
\gll Er hat das Buch nicht oft gelesen, sondern selten gekauft.\\
     他 \textsc{aux} \textsc{det} 书 不 经常 读     但是 很少 买\\
\mytrans{他不经常读书,但是偶尔会买书。}
%     he has the book not often read     but seldom bought\\
%\mytrans{He did not read the book often but rather bought it seldom.}
\zl
这两个解释对应于图\vref{fig-er-das-buch-nicht-oft-liest-psg}中的两个短语结构树。
%These two readings correspond to the two phrase structure trees in Figure~\vref{fig-er-das-buch-nicht-oft-liest-psg}.
\begin{figure}
\hfill
\begin{forest}
sm edges
  [V
    [N [er;他] ]
    [V
      [NP 
        [Det [das;\textsc{det}] ]
        [N [Buch;书] ] ]
      [V 
        [Adv [nicht;不]] 
        [V [Adv [oft;经常] ] 
           [V [liest;读] ] ] ] ] ]
\end{forest}
\hfill
\begin{forest}
sm edges
  [V
    [N [er;他] ]
    [V
      [NP 
        [Det [das;\textsc{det}] ]
        [N [Buch;书] ] ] 
      [V 
        [Adv 
           [Adv [nicht;不] ]
           [Adv [oft;经常] ] ]  
        [V [liest;读] ] ] ] ]
\end{forest}
\hfill\mbox{}
\caption{\label{fig-er-das-buch-nicht-oft-liest-psg}er das Buch nicht oft liest(他不经常读书)的可能句法分析}
%\caption{\label{fig-er-das-buch-nicht-oft-liest-psg}Possible syntactic analyses for \emph{er das
%    Buch  nicht oft liest} `he does not read the book often'}
\end{figure}%
需要注意的是,在HPSG的分析中,副词oft可以是短语nicht oft(不经常)的中心语。这区别于Groß提出的依存语法分析。进而,依存语法分析有两个结构:一个带依存于相同动词的所有副词的平铺结构,和一个依存于否定的oft。基于短语结构的分析有三个结构:一个结构的语序是oft在nicht之前,一个语序是nicht在oft之前,还有一个语序带有nicht和oft的直接组合。关于(\ref{ex-oft-liest-er-das-buch-nicht})中的例子的问题在于头两个结构的一个没有依存语法表示。这就可能使得它并非无法推导出语义,只是的确比基于组成成分的分析来得困难。
%Note that in an HPSG analysis, the adverb \emph{oft} would be the head of the phrase \emph{nicht oft}
%`not often'. This is different from the Dependency Grammar analysis suggested by Groß. Furthermore,
%the Dependency Grammar analysis has two structures: a flat one with all adverbs depending on the
%same verb and one in which \emph{oft} depends on the negation. The phrase structure"=based analysis
%has three structures: one with the order \emph{oft} before \emph{nicht}, one with the order
%\emph{nicht} before \emph{oft} and the one with direct combination of \emph{nicht} and
%\emph{oft}. The point about the example in (\ref{ex-oft-liest-er-das-buch-nicht}) is that one of the
%first two structures is missing in the Dependency Grammar representations. This probably does not make it
%impossible to derive the semantics, but it is more difficult than it is in constituent"=based approaches.

进而,需要注意的是,直接将依存图联系到拓扑场的模型不能解释例(\mex{1})的句子。
%Furthermore, note that models that directly relate dependency graphs to topological fields will not be able to
%account for sentences like (\mex{1}).
\ea
\gll Dem Saft eine kräftige Farbe geben Blutorangen.\footnotemark\\
     \textsc{det} 果汁 一   强烈   颜色 给 血.橙\\
%     the juice a   strong   color give blood.oranges\\
\footnotetext{%
 \citet{BC2010a}在曼海姆的德语系举办的\emph{Deutsches Referenzkorpus}(DeReKo)中发现了这个例子:\url{http://www.ids-mannheim.de/kl/projekte/korpora}
% \citet{BC2010a} found this example in the \emph{Deutsches Referenzkorpus} (DeReKo), hosted at Institut
%für Deutsche Sprache, Mannheim: \url{http://www.ids-mannheim.de/kl/projekte/korpora}
}
\mytrans{血橙给果汁一个强烈的颜色。}
%\mytrans{Blood oranges give a strong color to the juice.}
\z
这个句子的依存图如图\vref{fig-dem-saft-eine-kraeftige-farbe-dg}所示。
%The dependency graph of this sentence is given in Figure~\vref{fig-dem-saft-eine-kraeftige-farbe-dg}.
\begin{figure}[htb]
\centerline{
\begin{forest}
dg edges
[V
  [N [Det,tier=eine [dem;\textsc{det}]]
   [Saft;果汁] ]
  [N, l sep+=3pt 
      [Det,tier=eine [eine;一] ]
      [Adj, dg adjunct=4pt  [kräftige;强烈]]
      [Farbe;颜色]]
  [geben;给] 
  [N [Blutorangen;血.橙] ] ]
\end{forest}
}
\caption{\label{fig-dem-saft-eine-kraeftige-farbe-dg}Dem Saft eine kräftige Farbe geben Blutorangen.(血橙给果汁一个强烈的颜色。)的依存图}
%\caption{\label{fig-dem-saft-eine-kraeftige-farbe-dg}Dependency graph for \emph{Dem Saft eine kräftige Farbe geben Blutorangen.} `Blood oranges %give the juice a strong color.'}
\end{figure}%

这类明显的多重前置\isc{前置!显性多重前置}\is{fronting!apparent multiple}并不局限于NP。从属词的不同类型可以在\vfc 中被替换。针对数据的进一步讨论可以参考 \citew{Mueller2003b}。还有在“多重前置和信息结构”这个研究项目中收集了很多其他的数据\citep{Bildhauer2011a}。任何单独基于依存关系的理论以及不允许空成分的理论被迫放弃了动词二位(动词位于第二位)的语言分析中较为常见的限制。相较而言,像\gbc 的分析和那些假定空的动词中心语的\hpsgc
变体可以假设,这一动词中心语的一个投射占据了\vfc 的位置。这就解释了为什么\vfc 中的材料看上去像包括一个可见动词的动词性中心语:这个“前场”(Vorfelds)内部有结构区分。它们可以有一个填充的\nf,还有一个填充到句子右边界的助词。更多的数据、讨论和分析请参阅 \citew{Mueller2005d,MuellerGS}。在Gross \& Osborne的框架\citeyearpar{GO2009a}中的同等的分析可以是图\vref{fig-dem-saft-eine-kraeftige-farbe-empty-dg}中所示的图,但是需要注意的是 \citet[\page 73]{GO2009a}明确地反对空成分,而且在任何情况下就为了处理多重前置而提出一个空成分是十分特异的。\footnote{%
我在允许非连续成分的基于次序化的HPSG变体中提出了这样一个空成分\citep{Mueller2002c},但是后来进行了修改,这样只有连续成分才是可以的,动词位置被处理为中心语移位,以及动词移位分析中使用的包括相同空的动词中心语的多重前置\citep{Mueller2005d,MuellerGS}。
}
%Such apparent multiple frontings\is{fronting!apparent multiple} are not restricted to NPs. Various types of dependents can be
%placed in the \vf. An extensive discussion of the data is provided in  \citew{Mueller2003b}. Additional data
%have been collected in a research project on multiple frontings and information structure
%\citep{Bildhauer2011a}. Any theory based on dependencies alone and not allowing for
%empty elements is forced to give up the restriction commonly assumed in the analysis of V2 languages, namely that the verb is in second position. 
%In comparison, analyses like \gb and those \hpsg variants that assume an empty verbal head can
%assume that a projection of such a verbal head occupies the \vf. This explains why the material in
%the \vf behaves like a verbal projection containing a visible verb: such \emph{Vorfelds} are
%internally structured topologically. They may have a filled \nf and even a particle that fills the
%right sentence bracket. See  \citew{Mueller2005d,MuellerGS} for further data, discussion, and a
%detailed analysis. The equivalent of the analysis in Gross \& Osborne's framework \citeyearpar{GO2009a} would be something like the graph that is %shown
%in Figure~\vref{fig-dem-saft-eine-kraeftige-farbe-empty-dg}, but note that  \citet[\page 73]{GO2009a}
%explicitly reject empty elements, and in any case an empty element which is stipulated just to get the
%multiple fronting cases right would be entirely ad hoc.\footnote{%
%  I stipulated such an empty element in a linearization-based variant of HPSG allowing for
%  discontinuous constituents \citep{Mueller2002c},
%  but later modified this analysis so that only continuous constituents are allowed, verb
%  position is treated as head-movement and multiple frontings involve the same empty verbal head as
%  is used in the verb movement analysis \citep{Mueller2005d,MuellerGS}.
%}
\begin{figure}
\centerline{
\begin{forest}
dg edges
[V\sub{g}
  [V,edge=dashed [N [Det [dem;\textsc{det}]]
      [Saft;果汁] ]
     [N [Det [eine;一] ]
        [Adj, dg adjunct=4pt [kräftige;强烈]]
        [Farbe;颜色]]
     [ \trace ] ]
  [geben;给] 
  [N [Blutorangen;血.橙] ] ]
\end{forest}
}
\caption{\label{fig-dem-saft-eine-kraeftige-farbe-empty-dg}在\vfc 中带有空的动词中心语的Dem Saft eine
    kräftige Farbe geben Blutorangen.(血橙给果汁一个强烈的颜色。)的依存图}
%\caption{\label{fig-dem-saft-eine-kraeftige-farbe-empty-dg}Dependency graph for \emph{Dem Saft eine
 %   kräftige Farbe geben Blutorangen.} `Blood oranges give the juice a strong color.' with an empty
 % verbal head for the \vf}
\end{figure}%
需要指出的是,这个问题并没有通过简单地去除V2限制和允许定式动词的从属词在它的左边实现而得到解决,因为前置的成分并不必然依存于例(\mex{1})中所示的例子中的定式动词:
%It is important to note that the issue is not solved by simply dropping the V2 constraint and
%allowing dependents of the finite verb to be realized to its left, since the fronted constituents do
%not necessarily depend on the finite verb as the examples in (\mex{1}) show:
%\pagebreak
\eal
\ex
\label{ex-mehrfach-vf-adv-acc}
\gll [Gezielt] [Mitglieder] [im     Seniorenbereich]       wollen  die Kendoka~~~~~~ allerdings nicht werben.\footnotemark\\
    \spacebr{}特别地 \spacebr{}成员     \spacebr{}\textsc{prep}.\textsc{det} 高级.市民.选区 想要
    \textsc{det} 剑道家 但是    不   招募\\
\mytrans{但是,剑道家的会员招募策略并不想指向高级市民的选区。}%
%    \spacebr{}specifically \spacebr{}members     \spacebr{}in.the senior.citizens.sector want.to the Kendoka however    not   recruit\\
%\mytrans{However, the Kendoka do not intend to target the senior citizens sector with their member recruitment strategy.}%
\label{bsp-gezielt-mitglieder}
\footnotetext{%
        taz, \zhdate{1999/07/07},第18页。摘自 \citew{Mueller2002c}。
%        taz, 07.07.1999, p.\,18. Quoted from  \citew{Mueller2002c}.
      }
\ex 
\gll {}[Kurz] [die Bestzeit] hatte der Berliner Andreas Klöden [\ldots] gehalten.\footnotemark\\
	 \spacebr{}简短地 \spacebr{}\textsc{det} 最好.时光 \textsc{aux} \textsc{det} 柏林人 Andreas Klöden {} 拥有\\
%	 \spacebr{}briefly \spacebr{}the best.time had the Berliner Andreas Klöden {} held\\
\footnotetext{%
        Märkische Oderzeitung,2001年07月28日或29日,第28页。
}\label{bsp-kurz-die-bestzeit}     
\mytrans{来自柏林的Andreas Klöden拥有过简短的美好时光。}
%\mytrans{Andreas Klöden from Berlin had briefly held the record time.}
\zl
尽管相应的结构都有标记,但是这些多重前置\isc{前置!显性多重前置}\is{fronting!apparent multiple}还能够跨越小句的边界:
%And although the respective structures are marked, such multiple frontings\\is{fronting!apparent multiple} can even cross clause boundaries:
\ea 
\gll Der        Maria einen    Ring glaube  ich nicht, daß  er je   schenken wird.\footnotemark\\
     \textsc{det}.\dat{} Maria 一.\acc{} 戒指 相信 我   不    \textsc{comp} 他 曾经 给     将\\
%     the.\dat{} Maria a.\acc{} ring believe I   not    that he ever give     will\\
\footnotetext{%
 \citew[\page 67]{Fanselow93a}。
}
\mytrans{我不认为他曾经给了Maria一个戒指。}
%\mytrans{I don't think that he would ever give Maria a ring.}
\z
如果允许了这种依存关系,要限制它们就十分困难了。这里不便讨论细节问题,但是读者可以参考 \citew{Mueller2005d,MuellerGS}。
%If such dependencies are permitted it is really difficult to constrain them. The details cannot be
%discussed here, but the reader is referred to  \citew{Mueller2005d,MuellerGS}.

还需要指出的是,Engel有关德语句子的线性序列的说明\citeyearpar[\page 50]{Engel2014a},即指向定式动词前的一个成分(请参阅脚注\ref{fn-Engel-linearization})是十分不准确的。我们只能猜测词element想要表达什么意思。一个解释是它是基于成分语法的经典论断的连续成分。另一种方法是,一个中心语和一些从属语(没有必要是所有它的从属语)的连续实现。这一方法可以允许图\vref{fig-ein-junger-Kerl-stand-da}中描述的例(\mex{1})的非连续成分的外置\isc{外置}\is{extraposition}分析。
%Note also that Engel's statement regarding the linear order in German sentences \citeyearpar[\page
 % 50]{Engel2014a} referring to one element in front of the finite verb (see footnote~\ref{fn-Engel-linearization}) is very imprecise. One can only guess %what is
%intended by the word \emph{element}. One interpretation is that it is a continuous constituent in the classical sense of
%constituency"=based grammars. An alternative would be that there is a continuous realization of a
%head and some but not necessarily all of its dependents. This alternative would allow an analysis of extraposition\is{extraposition} with
%discontinuous constituents of (\mex{1}) as it is depicted in Figure~\vref{fig-ein-junger-Kerl-stand-da}.
\ea
\gll Ein junger Kerl stand da, mit langen blonden Haaren, die sein Gesicht einrahmten,   [\ldots]\footnotemark\\ 
     一 年轻 人 站 那儿 \textsc{prep} 长 金色 头发 \textsc{rel} 他的 脸 装框\\
\mytrans{一位脸旁镶满金发的年轻小伙子站在那儿} 
%     a young guy stood there with long blond hair that his face framed\\
%\mytrans{A young guy was standing there with long blond hair that framed his face} 
\footnotetext{%
	Charles Bukowski, \emph{Der Mann mit der Ledertasche}(《带皮包的男人》)。
      慕尼黑:德国平装书出版社,1994年,第201页,由Hans Hermann翻译。
%	Charles Bukowski, \emph{Der Mann mit der Ledertasche}.
%        München: Deutscher Taschenbuch Verlag, 1994, p.\,201,
%	translation by Hans Hermann.
}
\z
\begin{figure}
\centerline{
\begin{forest}
dg edges
[V
  [N,l sep+=2mm,name=n
    [Det [ein;一]]
    [Adj, dg adjunct,tier=adj [junger;年轻]]
    [Kerl;人] ]
  [stand;站]
  [Adv, dg adjunct [da;那儿]]
  [P, no edge,tier=adj,name=p [mit;\textsc{prep}]
    [N,l sep+=2mm
      [Adj, dg adjunct [langen;长] ]
      [Adj, dg adjunct [blonden;金色] ]
      [Haaren;头发] ] ] ]
\draw (n.south)--(p.north)-- +(0,6pt);
\end{forest}
}
\caption{\label{fig-ein-junger-Kerl-stand-da}在\vfc 中带有非连续成分的Ein junger Kerl stand da, mit langen blonden Haaren.(一位满头金发的小伙子站在那里。)的依存图}
%\caption{\label{fig-ein-junger-Kerl-stand-da}Dependency graph for \emph{Ein junger
%    Kerl stand da, mit langen blonden Haaren.} `A young guy was standing there with long blond hair.' with a discontinuous constituent in the \vf}
\end{figure}%
这一分析的形式化不是一个小问题,因为我们明确知道什么可以非连续地实现,以及依存关系的哪些部分一定可以连续地实现。
%A formalization of such an analysis is not trivial, since one has to be precise about what exactly can be
%realized discontinuously and which parts of a dependency must be realized
%continuously.
%% \todostefan{S: see also  \citew{GK2001a},  \citew{DD2001a-u}}\todostefan{DH: Word Grammar
%%   has both extraction and extraposition}
 \citet{KP95a}在\hpsgc 的框架下展开了外置的分析。也请参阅 \citew[Section~13.3]{Mueller99a}。我在下一节讨论了HPSG中这种次序化分析的基本机制。\LATER{Add a note on the right roof
  constraint and multiple extrapositions/or extrapositions from embedded clauses in the vorfeld}
% \citet{KP95a} developed such an analysis of extraposition in the framework of
%\hpsg. See also  \citew[Section~13.3]{Mueller99a}. I discuss the basic mechanisms for such
%linearization analyses in HPSG in the following section.\LATER{Add a note on the right roof
%  constraint and multiple extrapositions/or extrapositions from embedded clauses in the vorfeld}

\subsection{依存语法与短语结构语法}
%\subsection{Dependency Grammar vs.\ phrase structure grammar}
\label{sec-dependency-vs-constituency}
% Agel/Fischer 2010: 284 Nachteil PSG: Kopf muss ausgezeichnet werden.

本章探讨依存语法和短语结构语法之间的关系。我首先指出,可投射的依存语法可以译成短语结构语法(\ref{sec-dg-psg-translation})。然后,我将讨论带有非可投射的依存语法,并且说明他们是如何在基于次序化的HPSG理论中来表示的(\ref{sec-discontinuous-constituents-HPSG})。\ref{sec-dg-daughters-mothers}讨论了基于短语结构的理论中提出的额外结点,而且\ref{sec-headless-constructions-dg}讨论了无中心语的结构,这对于所有的依存语法都是一个问题。
%This section deals with the relation between Dependency Grammars and phrase structure grammars. I
%first show that projective Dependency Grammars can be translated into phrase structure grammars
%(Section~\ref{sec-dg-psg-translation}).
%I will then deal with non-projective DGs and show how they can be captured in linearization"=based
%HPSG (Section~\ref{sec-discontinuous-constituents-HPSG}). Section~\ref{sec-dg-daughters-mothers} argues for the additional nodes that are %assumed in phrase structure"=based
%theories and Section~\ref{sec-headless-constructions-dg} discusses headless constructions, which pose a problem for all Dependency
%Grammar accounts.

\subsubsection{将可投射性依存语法变为短语结构语法}
%\subsubsection{Translating projective Dependency Grammars into phrase structure grammars}

\label{sec-dg-psg-translation}

正如 \citet{Gaifman65a}、 \citet[\page 234]{Covington90a}、 \citet{Oliva2003a}和 \citet[\page 1093]{Hellwig2006a}\todostefan{S:  \citew{Robinson70a-u}}所指出的,某些可投射性中心语的短语结构语法可以通过将中心语移动到上一层来取代统治的结点以变成依存语法。所以在一个NP结构中,N变换到了NP的位置上,而且所有其他联系保持不变。如图\vref{fig-a-book-psg-dg}所示。
%As noted by  \citet{Gaifman65a},  \citet[\page 234]{Covington90a},  \citet{Oliva2003a} and  \citet[\page
 % 1093]{Hellwig2006a},\todostefan{S:  \citew{Robinson70a-u}} certain projective headed phrase structure grammars can be turned into
%Dependency Grammars by moving the head one level up to replace the dominating node. So in an NP
%structure, the N is shifted into the position of the NP and all other connections remain the
%same. Figure~\vref{fig-a-book-psg-dg} illustrates.
%\todostefan{S: the equivalence between flat headed
%  constituency trees (= PS) and DT has been stated by Lecerf 1960, 1961 (in French only). that's a different thing than the equivalence between grammatical formalisms}
\begin{figure}
\hfill%
\begin{forest}
sm edges
[NP
  [D [a;一] ]
  [N [book;书] ] ]
\end{forest}\hfill%
\begin{forest}
dg edges
[N
  [D [a;一] ]
  [book;书] ]
\end{forest}
\hfill\mbox{}
\caption{在短语结构语法和依存语法分析中的\label{fig-a-book-psg-dg}a book}
%\caption{\label{fig-a-book-psg-dg}\emph{a book} in a phrase structure and a
%  Dependency Grammar analysis}
\end{figure}%

当然,这个程序不能直接应用到所有的短语结构语法中,因为有些包括了更为精细的结构。比如说,规则S $\to$ NP, VP不能翻译成依存规则,因为NP和VP都属于复杂的范畴。
%Of course this procedure cannot be applied to all phrase structure grammars directly since some
%involve more elaborate structure. For instance, the rule S $\to$ NP, VP cannot be translated into a
%dependency rule, since NP and VP are both complex categories.

接下来,我想说明图\vref{fig-the-child-reads-the-book-dg}中的依存图是如何重新表示为允准了一个相似树的中心语的短语结构规则,即如图\vref{fig-the-child-reads-a-book-psg}所示。
%In what follows, I want to show how the dependency graph in Figure~\vref{fig-the-child-reads-the-book-dg} can be recast as headed phrase
%structure rules that license a similar tree, namely the one in Figure~\vref{fig-the-child-reads-a-book-psg}.
% Hellwig2006a: 1084, 1093
\begin{figure}
\centerline{%
\begin{forest}
sm edges
[V
  [N
    [D [the;\textsc{det}] ]
    [N [child;孩子] ] ]
  [V [reads;读]]
  [N
    [D [a;一] ]
    [N [book;书] ] ] ]
\end{forest}
}
\caption{\label{fig-the-child-reads-a-book-psg}带有平铺规则的短语结构的The child reads a book.的分析}
%\caption{\label{fig-the-child-reads-a-book-psg}Analysis of \emph{The child reads a book.} in a
%  phrase structure with flat rules}
\end{figure}%
我没有使用NP和VP的标签来保证两张图在最大程度上是相似的。NP和VP中的P部分是指可投射性的满足,并且经常在图中被忽视。请参阅第\ref{chap-HPSG}章有关HPSG的内容,比如说,允准了例(\mex{1})中给出的树的语法再次忽视了配价信息。
%I did not use the labels NP and VP to keep the two figures maximally similar. The P part of NP and
%VP refers to the saturation of a projection and is often ignored in figures. See Chapter~\ref{chap-HPSG} on HPSG, for example.
%The grammar that licenses the tree is given in (\mex{1}), again ignoring valence
%information.
\ea
%\label{bsp-grammatik-psg}
\begin{tabular}[t]{@{}l@{ }l}
{N} & {$\to$ D N}\\          
{V}  & {$\to$ N V N}
\end{tabular}\hspace{2cm}%
\begin{tabular}[t]{@{}l@{ }l}
{N} & {$\to$ child}\\
{N} & {$\to$ book}\\
\end{tabular}\hspace{8mm}
\begin{tabular}[t]{@{}l@{ }l}
{D}  & {$\to$ the}\\
{V} & {$\to$ reads}\\
\end{tabular}
\hspace{8mm}
\begin{tabular}[t]{@{}l@{ }l}
{D}  & {$\to$ a}\\
\end{tabular}
\z
如果我们在带有各自词汇项的例(\mex{0})中的两个最左边规则右手边替换N和V,我们就会去除允准了词语的规则,我们就会得到(\mex{1})中给出的语法的词汇化变体:
%If one replaces the N and V in the right-hand side of the two left-most rules in (\mex{0}) with the
%respective lexical items and then removes the rules that license the words, one arrives at the
%lexicalized variant of the grammar given in (\mex{1}):
\ea
%\label{bsp-grammatik-psg}
\begin{tabular}[t]{@{}l@{ }l}
{N} & {$\to$ D book}\\          
{N} & {$\to$ D child}\\          
{V}  & {$\to$ N reads N}
\end{tabular}\hspace{1.5cm}%
\begin{tabular}[t]{@{}l@{ }l}
{D}  & {$\to$ the}\\
{D}  & {$\to$ a}\\
\end{tabular}
\z
词汇化(Lexicalized)是指每一部分由语法允准的树都包括一个词汇成分。(\mex{0})中的语法允准了图\ref{fig-the-child-reads-the-book-dg}中的树。\footnote{\label{fn-flat-dg-rules}%
正如在第\pageref{page-rule-format-dg}页所提到的, \citet[\page 305]{Gaifman65a}、 \citet[\page 513]{Hays64a-u}、 \citet[\page 57]{Baumgaertner70a}和 \citet[\page 37]{Heringer96a-u}针对依存规则提出了一条普遍的规则,即它有一个特殊的标记(分别为`*'和`\textasciitilde')替代了(\mex{0})中的词汇词。Heringer的规则具有(\mex{1})中的形式:
%}
%\emph{Lexicalized}  means that every partial tree licensed by a grammar rule contains a lexical element.
%The grammar in (\mex{0}) licenses exactly the tree in
%Figure~\ref{fig-the-child-reads-the-book-dg}.\footnote{\label{fn-flat-dg-rules}%
%As mentioned on page~\pageref{page-rule-format-dg},  \citet[\page 305]{Gaifman65a},  \citet[\page
%  513]{Hays64a-u},  \citet[\page 57]{Baumgaertner70a} and  \citet[\page 37]{Heringer96a-u} suggest a
%general rule format for dependency rules that has a special marker (`*' and `\textasciitilde', respectively) in place of the lexical words in (\mex{0}). %Heringer's rules have the
%form in (\mex{1}):
\ea
X[Y1, Y2, \textasciitilde, Y3]
\z
X是中心语的范畴,Y1、Y2和Y3是中心语的从属语,而且“\textasciitilde”是中心语所插入的位置。
%X is the category of the head, Y1, Y2, and Y3 are dependents of the head and `\textasciitilde' is the position into
%which the head is inserted.
}

经典的短语结构语法和依存语法的一个重要的区别是短语结构规则给子结点强加了特定的语序。也就是说,(\mex{0})中的V规则暗示了第一个名词性投射、动词和第二个名词性投射必须按照规则表明的语序来出现。当然,这个语序限制可以松一些,正如GPSG中所作的。这基本上允准了规则右手边任意顺序的子结点。剩下的问题是附加语的整合。由于附加语也依存于中心语(请参阅图\vref{fig-the-child-often-reads-the-book-slowly}),可以提出一条规则来允准论元外的任意多的附加语。由此,(\mex{0})中的V规则应该变成(\mex{1})中的形式:\footnote{%
请参阅第\pageref{adv-metarule}页中\gpsgc 的相似规则,并且参考德语的HPSG理论的分析 \citet{Kasper94a},它提出了完全的平铺结构并且整合进了任意数量的附加语。
}
%One important difference between classical phrase structure grammars and Dependency Grammars is that the
%phrase structure rules impose a certain order on the daughters. That is, the V rule in (\mex{0})
%implies that the first nominal projection, the verb, and the second nominal projection have to
%appear in the order stated in the rule. Of course this ordering constraint can be relaxed as it is
%done in GPSG. This would basically permit any order of the daughters at the right hand side of
%rules.
%This leaves us with the integration of adjuncts. Since adjuncts depend on the head as well (see
%Figure~\vref{fig-the-child-often-reads-the-book-slowly}), a rule could be assumed that allows
%arbitrarily many adjuncts in addition to the arguments. So the V rule in (\mex{0}) would be changed
%to the one in (\mex{1}):\footnote{%
%  See page~\pageref{adv-metarule} for a similar rule in \gpsg and see  \citet{Kasper94a} for an HPSG
%analysis of German that assumes entirely flat structures and integrates an arbitrary number of adjuncts.
%}
\ea
V $\to$ N reads N Adv*
\z 

这种广义短语结构等同于可投射的依存语法。\footnote{\label{fn-dg-binary-branching}%
Sylvain Kahane(p.\,c.\, 2015)指出,二元性对于依存语法来说是非常重要的,因为主语只有一条规则,宾语只有一条规则,以及其他(比如说\citealp{Kahane2009a},这是在HPSG理论的形式框架下的依存语法的实现)。不过,我没有想到有任何原因需要反对平铺结构。比如说, \citet[\page 364]{GSag2000a-u}在HPSG理论中提出了主语助动词转换的平铺结构。在这类平铺规则中,限定语/主语和其他补足语在同一个目标下与动词组合在一起。这也适用于两个以上配价特征的语法功能范畴,如主语、直接宾语、间接宾语。也请参阅关于平铺规则的脚注\ref{fn-flat-dg-rules}。
}但是,正如我们看到的,一些研究人员对非连续的成分允许了交叉边的存在。接着,我来说明依存语法是如何在HPSG理论中被形式化的。
%Such generalized phrase structures would give us the equivalent of projective Dependency
%Grammars.\footnote{\label{fn-dg-binary-branching}%
%Sylvain Kahane (p.\,c.\, 2015) states that binarity is important for Dependency Grammars, since
%there is one rule for the subject, one for the object and so on (as for instance in
%\citealp{Kahane2009a}, which is an implementation of Dependency Grammar in the HPSG formalism). However, I do not see any reason
%to disallow for flat structures. For instance,  \citet[\page 364]{GSag2000a-u} assumed a flat rule for subject
%auxiliary inversion in HPSG. In such a flat rule the specifier/subject and the other complements are
%combined with the verb in one go. This would also work for more than two valence features that correspond
%to grammatical functions like subject, direct object, indirect object. See also
%Footnote~\ref{fn-flat-dg-rules} on flat rules.
%} However, as we have seen, some researchers allow for crossing edges, that is, for
%discontinuous constituents. In what follows, I show how such Dependency Grammars can be formalized in
%HPSG.

\subsubsection{非可投射的依存语法与带有非连续成分的短语结构语法}
%\subsubsection{Non-projective Dependency Grammars and phrase structure grammars with discontinuous constituents}
\label{sec-discontinuous-constituents-HPSG}

等同于非可投射的\isc{投射性|(}\is{projectivity|(}依存图的是短语结构语法中的非连续成分。接下来,我将给出允许非连续结构的基于短语结构的理论的一个例子。因为,正如我将展示的,非连续性也可以进行模拟,短语结构语法和依存语法之间的区别归结于词的单位是否被给予了标签(比如说NP)。
%The equivalent to non-projective\is{projectivity|(} dependency graphs are discontinuous constituents in phrase
%structure grammars. In what follows I want to provide one
%example of a phrase structure"=based theory that permits discontinuous structures. Since, as I
%will show, discontinuities can be modeled as well, the difference between phrase structure grammars
%and Dependency Grammars boils down to the question of whether units of words are given a label (for
%instance NP) or not.

在HPSG理论这类框架下来模拟非连续成分的技术最早追溯到Mike Reape针对德语所作的工作\citeyearpar{Reape91,Reape92a,Reape94a}。Reape使用了一个叫做\textsc{domain}的列表来表示按照话语表层出现顺序的符号的子结点。(\mex{1})给出了一个例子,其中,中心语短语的\domvc 从中心语的\domvc 和非中心语子结点的列表中计算出来。
%The technique that is used to model discontinuous constituents in frameworks like HPSG goes back to Mike Reape's work on German
%\citeyearpar{Reape91,Reape92a,Reape94a}. 
%Reape uses a list called \textsc{domain} to represent the daughters of a sign in the order in
%which they appear at the surface of an utterance. (\mex{1}) shows an example in which the \domv of a
%headed-phrase is computed from the \domv of the head and the list of non-head daughters.
\ea
\type{headed"=phrase} \impl
\ms{
  head-dtr$|$dom  & \ibox{1} \\
  non-head-dtrs   & \ibox{2} \\
  dom  & \ibox{1} $\bigcirc$ \ibox{2} \\
}
\z
符号“$\bigcirc$”\isc{$\bigcirc$}\is{$\bigcirc$}\isc{关系!$\bigcirc$}\is{relation!$\bigcirc$}\isrel{shuffle}\label{rel-shuffle}表示shuffle关系。shuffle连接了A、B和C三个列表,当且仅当C包括了A和B中的所有成分,而且A中成分的语序和B中成分的语序保存在C中。(\mex{1})分别显示了带有两个成分的两个集合的组合:
%The symbol `$\bigcirc$'\is{$\bigcirc$}\is{relation!$\bigcirc$}\isrel{shuffle}\label{rel-shuffle}
%stands for the \emph{shuffle} relation. \emph{shuffle} relates three lists A, B and C iff C
%contains all elements from A and B and the order of the elements in A and the order of the elements
%of B is preserved in C. (\mex{1}) shows the combination of two sets with two elements each:
\ea
$\phonliste{ a, b } \bigcirc \phonliste{ c, d } =
\begin{tabular}[t]{@{}l}
\phonliste{ a, b, c, d } $\vee$\\*[1mm]
\phonliste{ a, c, b, d } $\vee$\\*[1mm]
\phonliste{ a, c, d, b } $\vee$\\*[1mm]
\phonliste{ c, a, b, d } $\vee$\\*[1mm]
\phonliste{ c, a, d, b } $\vee$\\*[1mm]
\phonliste{ c, d, a, b }
\end{tabular}$
\z
结果是六个列表的析取。在所有这些列表中,a位于b的前面,c位于d的前面,因为在已经结合的\phonliste{ a, b }和\phonliste{ c, d }两个列表中也是这样的。但是,除了这种情况,b可以放在c和d的前面、中间和后面。每个词都带有一个域值,它是一个包括这个词本身的列表:
%The result is a disjunction of six lists. \emph{a} is ordered before \emph{b} and \emph{c} before
%\emph{d} in all of these lists, since this is also the case in the two lists \phonliste{ a, b } and
%\phonliste{ c, d } that have been combined. But apart from this, \emph{b} can be placed before, between or
%after \emph{c} and \emph{d}. Every word comes with a domain value that is a list that contains the
%word itself:
\ea
单个词的管辖范围,这里是gibt(给):\\
%Domain contribution of single words, here \emph{gibt} `gives':\\
\ibox{1} \ms{
phon & \phonliste{ gibt }\\
synsem & \ldots\\
dom  & \sliste{ \ibox{1} } \\
}
\z
(\mex{0})里的描述初看起来有些奇怪,因为它是循环的\isc{循环!特征描写中的循环}\is{cycle!in feature description},不过它理解为gibt将它自己合一到次序化域中出现的元素。
%The description in (\mex{0}) may seem strange at first glance, since it is cyclic\is{cycle!in
%  feature description}, but it can be understood as
%a statement saying that \emph{gibt} contributes itself to the items that occur in linearization domains.

(\mex{1})中的限制决定于短语的\phonvsc :
%The constraint in (\mex{1}) is responsible for the determination of the \phonvs of phrases:
\ea
\type{phrase} \impl
\ms{
 phon & \ibox{1} $\oplus$ \ldots{} $\oplus$ \ibox{n} \\ \\
     dom  & \liste{ \ms[sign]{ phon & \ibox{1} \\ }, \ldots, \ms[sign]{ phon & \ibox{n} \\ }
                  } \\
   }
\z
它证明了符号的\phonvc 是它的\textsc{domain}成分的\phonvsc 的合一。由于\textsc{domain}成分的语序对应于他们的表层语序,这是决定整个语言对象的\phonvc 的自然而然的方式。
%It states that the \phonv of a sign is the concatenation of the \phonvs of its \textsc{domain}
%elements. Since the order of the \textsc{domain} elements corresponds to their surface order, this is
%the obvious way to determine the \phonv of the whole linguistic object. 

图\vref{fig-the-child-reads-the-book-reape-binary}显示了这一机制是如何应用到带有非连续成分的二叉结构中的。
%Figure~\vref{fig-the-child-reads-the-book-reape-binary} shows how this machinery can be used to license binary
%branching structures with discontinuous constituents.
\begin{figure}
\centerline{%
\begin{forest}
sm edges
[{V[\dom \phonliste{ der Frau, ein Mann, das Buch, gibt }]}
  [{NP[\type{nom}, \dom \phonliste{ ein, Mann }]} [ein Mann;一 男人,roof]]
  [{V[\dom \phonliste{ der Frau, das Buch, gibt }]}
   [{NP[\type{dat}, \dom \phonliste{ der, Frau  }]} [der Frau;\textsc{det} 女人,roof] ]
   [{V[\dom \phonliste{ das Buch, gibt }]}
    [{~~~NP[\type{acc}, \dom \phonliste{ das, Buch }]} [das Buch;\textsc{det} 书,roof] ]
    [{V[\dom \phonliste{ gibt }]} [gibt;给] ] ] ] ]
%    [{NP[\type{nom}, \dom \phonliste{ ein, Mann }]} [ein Mann;a man,roof]]
%  [{V[\dom \phonliste{ der Frau, das Buch, gibt }]}
%   [{NP[\type{dat}, \dom \phonliste{ der, Frau  }]} [der Frau;the woman,roof] ]
%   [{V[\dom \phonliste{ das Buch, gibt }]}
%    [{~~~NP[\type{acc}, \dom \phonliste{ das, Buch }]} [das Buch;the book,roof] ]
%    [{V[\dom \phonliste{ gibt }]} [gibt;gives] ] ] ] ]
\end{forest}
\caption{\label{fig-the-child-reads-the-book-reape-binary}带有二叉结构和非连续成分的dass der Frau ein Mann das Buch gibt(一个男人给这个女人这本书)的分析}}
%roof\caption{\label{fig-the-child-reads-the-book-reape-binary}Analysis of \emph{dass der Frau ein Mann das Buch gibt} `that a man gives the woman %the book' with binary branching structures and discontinuous constituents}}
\end{figure}%%
由逗号分开的词或词的序列表示不同的域的对象,即\phonliste{ das, Buch }包括两个对象das和Buch,而且\phonliste{ das Buch, gibt }包括了两个对象das Buch和gibt。这里需要指出的很重要的是,跟中心语组合的论元按照宾语、与格、主格的语序排列,尽管短语序列域内成分是按照与格、主格和宾格的顺序排列的,而不是人们所期望的主格、与格、宾格。这是可能的,因为使用了shuffle 算子的\domvc 的计算的形成可以描写非连续成分。der Frau das Buch gibt(给这本书的这个女人)的结点是非连续的:ein Mann(一个男人)插入了der Frau(这个女人)和das Buch(这本书)的域内。这在图\vref{fig-the-child-reads-the-book-reape-binary-discont}中更为明显,它有着一个对应于他们语序的NP的次序化。
%Words or word sequences that are separated by commas stand for separate domain objects, that is,
%\phonliste{ das, Buch } contains the two objects \emph{das} and \emph{Buch} and \phonliste{ das
%  Buch, gibt } contains the two objects \emph{das Buch} and \emph{gibt}.
%The important point to note here is that the arguments are combined with the head in the order
%accusative, dative, nominative, although the elements in the constituent order domain are realized in
%the order dative, nominative, accusative rather than nominative, dative, accusative, as one would
%expect. This is possible since the formulation of the computation of the \domv using the shuffle
%operator allows for discontinuous constituents. The node for \emph{der Frau das Buch gibt} `the
%woman the book gives' is discontinuous: \emph{ein Mann} `a man' is inserted into the domain between
%\emph{der Frau} `the woman' and \emph{das Buch} `the book'.  This is more obvious in Figure~\vref{fig-the-child-reads-the-book-reape-binary-discont}, %which has a serialization of NPs that
%corresponds to their order.
\begin{figure}
\centerfit{%
\begin{forest}
sm edges
[{V[\dom \phonliste{ der Frau, ein Mann, das Buch, gibt }]}
  [{NP[\type{dat}, \dom \phonliste{ der, Frau  }]~~~~~~}, no edge, name=np-dat,tier=dat-tier, [der Frau;\textsc{det} 女人,roof] ]
  [{NP[\type{nom}, \dom \phonliste{ ein, Mann }]} [ein Mann;一 男人, roof]]
  [{V[\dom \phonliste{ der Frau, das Buch, gibt }]}, name=v
   [NP, phantom, tier=dat-tier ]
   [{V[\dom \phonliste{ das Buch, gibt }]}
    [{NP[\type{acc}, \dom \phonliste{ das, Buch }]} [das Buch;\textsc{det} 书,roof] ]
    [{V[\dom \phonliste{ gibt }]} [gibt;给] ] ] ] ]
%   [{NP[\type{dat}, \dom \phonliste{ der, Frau  }]~~~~~~}, no edge, name=np-dat,tier=dat-tier, [der Frau;the woman,roof] ]
%  [{NP[\type{nom}, \dom \phonliste{ ein, Mann }]} [ein Mann;a man, roof]]
%  [{V[\dom \phonliste{ der Frau, das Buch, gibt }]}, name=v
%   [NP, phantom, tier=dat-tier ]
%   [{V[\dom \phonliste{ das Buch, gibt }]}
%    [{NP[\type{acc}, \dom \phonliste{ das, Buch }]} [das Buch;the book,roof] ]
%    [{V[\dom \phonliste{ gibt }]} [gibt;gives] ] ] ] ]
\draw (v.south) -- (np-dat.north);
\end{forest}
}
\caption{\label{fig-the-child-reads-the-book-reape-binary-discont}具有二叉结构和显示出不一致的非连续成分的dass der Frau ein Mann das Buch gibt(一个男人给这个女人书)的分析}
% \caption{\label{fig-the-child-reads-the-book-reape-binary-discont}Analysis of \emph{dass der Frau ein Mann das Buch gibt} `that a man gives the woman the book' with binary branching structures and discontinuous constituents % showing the discontinuity}
\end{figure}% 

这种二元的分叉结构被 \citet{Kathol95a,Kathol2000a}和 \citet{Mueller95c,Babel,Mueller99a,Mueller2002b}用于对德语的分析,但是正如我们在现在这一章中看到的,依存语法提出了平铺的表示(但是请参阅第\pageref{fn-dg-binary-branching}页的脚注\ref{fn-dg-binary-branching})。模式\ref{schema-flat-prel}允准了在一个目标内实现的中心语的所有论元的结构。\footnote{我在这里提出包括在词汇中心语,而不是联系其中的\subcatlc 的所有论元成分。比如说, \citet[\page 339]{Borsley89}提出了英语\il{English}的助动词转换和威尔士语\il{Welsh}中动词位于首位的模式,它们指向主语和补足语的配价特征,并在同一平铺结构实现所有的成分。}
%Such binary branching structures were assumed for the analysis of German by  \citet{Kathol95a,Kathol2000a} and %% \citet{Mueller95c,Babel,Mueller99a,Mueller2002b}, but as we have seen throughout this chapter, Dependency Grammar assumes flat representations %(but see Footnote~\ref{fn-dg-binary-branching} on page~\pageref{fn-dg-binary-branching}). Schema~\ref{schema-flat-prel} licenses structures in which all %arguments of a head are realized in one go.\footnote{I assume here that all arguments are contained in the \subcatl of a
%lexical head, but nothing hinges on that. One could also assume several valence features and
%nevertheless get a flat structure. For instance,  \citet[\page 339]{Borsley89} suggests a schema for auxiliary inversion in English\il{English} and verb-%initial sentences in Welsh\il{Welsh} that refers to both the valence feature for subjects and for complements and realizes all elements in a flat structure.}
\begin{schema}[中心语-论元模式(平铺结构)]
%\begin{schema}[Head-Argument Schema (flat structure)]
\label{schema-flat-prel}
\type{head-argument-phrase}\istype{head"=argument"=phrase} \impl\\
\onems{
      synsem$|$loc$|$cat$|$subcat \eliste \\
      head-dtr$|$synsem$|$loc$|$cat$|$subcat \ibox{1} \\
      non-head-dtrs \ibox{1} \\
      }
\end{schema}
为了简化表达,我认为\subcatlc 包括了完整符号的描述。所以说,整个列表可以等同于非中心语子结点的列表。\footnote{%
除了这一假设,我们需要一个将类型\type{synsem}的描写的列表匹配到类型\type{sign}描写的列表的关系性限制。更多细节请参阅 \citew[\page 198]{Meurers99b}。
}
%To keep the presentation simple, I assume that the \subcatl contains descriptions of complete
%signs. Therefore the whole list can be identified with the list of non-head daughters.\footnote{%
%  Without this assumption one would need a relational constraint that maps a list with descriptions of type
%  \type{synsem} onto a list with descriptions of type \type{sign}. See  \citew[\page 198]{Meurers99b} for details.
%}
\domvc 的计算可以按照下面的方式来进行限制:
%The computation of the \domv can be constrained in the following way:
\ea
\type{headed"=phrase} \impl
\ms{
  head-dtr        & \ibox{1} \\
  non-head-dtrs   & \sliste{ \ibox{2}, \ldots, \ibox{n} } \\
  dom  & \sliste{ \ibox{1} } $\bigcirc$ \sliste{ \ibox{2} } $\bigcirc$ \ldots{} $\bigcirc$  \sliste{ \ibox{n} } \\
}
\z
这一限制是说\dom 的值是一个列表,它是变换每个包括一个子结点作为成分的单一列表的结果。这种变换操作的结果是所有子结点的所有可能的排列组合。这看起来超过了GPSG已经取得的成果,它通过将短语结构规则右手边的成分的序列抽象出来而得到。但是,需要注意的是这个机制可以用来表示更加自由的语序:通过指向子结点的\domvsc,而不是子结点本身,有可能将个别词插入\domlc 中。
%This constraint says that the value of \dom is a list which is the result of shuffling singleton
%lists each containing one daughter as elements. The result of such a shuffle operation is a
%disjunction of all possible permutations of the daughters. This seems to be overkill for something
%that GPSG already gained by abstracting away from the order of the elements on the right hand side
%of a phrase structure rule. Note, however, that this machinery can be used to reach even freer orders: by
%referring to the \domvs of the daughters rather than the daughters themselves, it is possible to insert
%individual words into the \doml.
\ea
\type{headed"=phrase} \impl
\ms{
  head-dtr$|$dom  & \ibox{1} \\
  non-head-dtrs   & \sliste{ [ dom \ibox{2} ] \ldots{} [ dom \ibox{n} ] } \\
  dom  & \sliste{ \ibox{1} } $\bigcirc$ \sliste{ \ibox{2} } $\bigcirc$ \ldots{} $\bigcirc$  \sliste{ \ibox{n} } \\
}
\z
应用这一限制,我们得到基本上按照任意排列组合的话语的所有词汇的\domvsc。我们得到的是一个不带有任何可投射限制的纯粹的依存语法。
%Using this constraint we have \domvs that basically contain all the words in an utterance in any
%permutation. What we are left with is a pure Dependency Grammar without any constraints on
%projectivity.
%\todostefan{S: ok but nobody in DG does that.
%There are plenty of works about how to relax the projectivity without suppressing it.}
根据这个语法,我们可以分析图\vref{fig-dass-die-Frauen-Tueren-oeffnen-dg}的非可投射结构以及更多的内容。针对域合并的分析如图\vref{fig-dass-die-Frauen-Tueren-oeffnen-domains}所示。
%With such a grammar we could analyze the non-projecting structure of Figure~\vref{fig-dass-die-Frauen-Tueren-oeffnen-dg} and much more. The %analysis in terms of domain union is shown in Figure~\vref{fig-dass-die-Frauen-Tueren-oeffnen-domains}. 
\begin{figure}
\centerline{%
\begin{forest}
sm edges
[{V[\dom \phonliste{ die, Frauen, Türen, öffnen }]}
  [{D[\dom \phonliste{ die }]},no edge,name=die,tier=det-n [die;\textsc{det}] ]
  % [{D[\dom \phonliste{ die }]},no edge,name=die,tier=det-n [die;the] ]
  [{NP[\dom \phonliste{ Frauen }]}
     [Frauen;女人] ]
     %    [Frauen;women] ]
  [{NP[\dom \phonliste{ die, Türen }]},name=Türen
    [Det, phantom ]
    [{N[\dom \phonliste{ Türen }]}, tier=det-n 
     [Türen;门] ] ]
     %   [Türen;doors] ] ]
  [{V[\dom \phonliste{ öffnen }]} 
    [öffnen;开] ]
 %   [öffnen;open] ]
]
\draw (Türen.south)--(die.north);
\end{forest}
}
\caption{\label{fig-dass-die-Frauen-Tueren-oeffnen-domains}使用Reape式短语成分域的dass die Frauen Türen öffnen(这个女人开门)不合适分析}
%\caption{\label{fig-dass-die-Frauen-Tueren-oeffnen-domains}Unwanted analysis of “dass die Frauen Türen %öffnen” `that the women open doors' using Reape-style constituent order domains}
\end{figure}%
很明显,我们不需要这样的不一致性。由此,我们需要有保证一致性的限制条件。一个限制条件就是要求具有投射性,这样就对应于我们上面讨论的短语结构语法。
%It is clear that such discontinuity is unwanted and hence one has to have restrictions that enforce %continuity. One possible restriction is to require projectivity and hence equivalence to phrase %structure grammars in the sense that was discussed above.
%
关于成分/""依存关系对于分析自然语言哪一个是首要/必须的这个问题是存在争议的: \citet{Hudson80a}和 \citet{Engel96a}认为依存关系是充分的,这被大部分依存语法学家所认可(根据\citealp{Engel96a}), \citet{Leiss2003a}认为不是这样的。为了解决这个问题,我们来看一些例子:
%There is some dispute going on about the question of whether constituency/""dependency is
%primary/necessary to analyze natural language: while % \citet[\page 83]{Korhonen77a}, 
% \citet{Hudson80a} and  \citet{Engel96a} claim that dependency is
%sufficient, a claim  that is shared by dependency grammarians (according to \citealp{Engel96a}),  \citet{Leiss2003a} claims
%that it is not. In order to settle the issue, let us take a look at some examples:
\ea
\gll Dass Peter kommt, klärt nicht, ob Klaus spielt.\\
     \textsc{comp} Peter 来    解决 不 是否 Klaus 玩\\
\mytrans{Peter来并不能解决Klaus是不是想玩儿这个问题。}
%     that Peter comes    resolves not whether Klaus plays\\
%\mytrans{That Peter comes does not resolve the question of whether Klaus will play.}
\z
如果我们知道话语的意义,我们可以赋予它一个依存图。让我们假设(\mex{0})的意义是像(\mex{1})这样的:
%If we know the meaning of the utterance, we can assign a dependency graph to it. Let us assume
%that
%\todostefan{S: This discussion does not make sense to me.}
%the meaning of (\mex{0}) is something like (\mex{1}):
\ea
$\neg$
\relation{resolve}(\relation{that}(\relation{come}(\relation{Peter})),\relation{whether}(\relation{play}(\relation{Klaus})))
\z
根据这个语义信息,我们当然可以给(\mex{-1})构成一个依存图。原因是依存关系在(\mex{0})的语义表达式中反应为双唯一的方式。
%With this semantic information, we can of course construct a dependency graph for (\mex{-1}). The
%reason is that the dependency relation is reflected in a bi"=unique way in the semantic representation in
%(\mex{0}). The respective graph is given in Figure~\vref{fig-dass-peter-kommt-klaert-nicht-ob}.
\begin{figure}
\centerline{%
\begin{forest}
dg edges
[V, l sep+=6pt
  [Subjunction
    [dass;\textsc{comp}]
    [V
      [N [Peter;Peter]]
      [kommt;来]]]
  [klärt;解决]
  [Adv, dg adjunct [nicht;不]]
  [Subjunction
    [ob;是否]
    [V
      [N [Klaus;Klaus]]
      [spielt;玩]]]]
\end{forest}
}
\caption{\label{fig-dass-peter-kommt-klaert-nicht-ob}能从语义表达式中推导出的Dass Peter kommt, klärt nicht, ob Klaus spielt.(Peter来并不能解决Klaus是不是想玩儿这个问题。)的依存图}
%\caption{\label{fig-dass-peter-kommt-klaert-nicht-ob}The dependency graph of \emph{Dass Peter kommt,
%    klärt nicht, ob Klaus spielt.} `That Peter comes does not resolve the question of whether Klaus
%  plays.' can be derived from the semantic representation.}
\end{figure}%
但是需要注意的是这对于普通的情况是不适用的。比如说(\mex{1})中的例子:
%But note that this does not hold in the general case. Take for instance the example in (\mex{1}):
\ea
\gll Dass Peter kommt, klärt nicht, ob Klaus kommt.\\
     \textsc{comp} Peter 来    解决 不 是否 Klaus 玩儿\\
\mytrans{Peter来并不解决Klaus是否来的问题。}
%     that Peter comes    resolves not whether Klaus plays\\
%\mytrans{That Peter comes does not resolve the question of whether Klaus comes.}
\z
这里,词kommt出现了两次。没有关于邻接、线性顺序和一致性的成分或限制的任何说明,我们不能无歧义地得到一个依存图。比如说,图\vref{fig-dass-peter-kommt-klaert-nicht-ob-non-projective}中的图完美地兼容了句子的语义:dass统治kommt,而且kommt统治Peter,而ob统治kommt,而且kommt统治Klaus。
%Here the word \emph{kommt} appears twice. Without any notion of constituency or
%restrictions regarding adjacency, linear order and continuity, we cannot assign a dependency graph
%unambiguously. For instance, the graph in Figure~\vref{fig-dass-peter-kommt-klaert-nicht-ob-non-projective} is perfectly compatible with the
%meaning of this sentence: \emph{dass} dominates \emph{kommt} and \emph{kommt} dominates
%\emph{Peter}, while \emph{ob} dominates \emph{kommt} and \emph{kommt} dominates \emph{Klaus}. 
\begin{figure}
\centerline{%
\begin{forest}
dg edges
[V, l sep+=6pt
  [Subjunction,name=subj1
    [dass;\textsc{comp}]
%      [dass;that]
    [V,name=v1,no edge
      [N,name=n1, no edge [Peter;Peter]]
      [kommt;来]]]
%           [kommt;comes]]]
  [klärt;解决]
%    [klärt;resolves]
  [Adv, dg adjunct [nicht;不]]
%    [Adv, dg adjunct [nicht;not]]
  [Subjunction,name=subj2
    [ob;是否]
  %      [ob;whether]
    [V,name=v2,no edge
      [N,name=n2, no edge [Klaus;Klaus]]
      [kommt;来]]]]
 %      [kommt;comes]]]]
\draw (subj1.south)--(v2.north);
\draw (subj2.south)--(v1.north);
\draw (v1.south)--(n2.north);
\draw (v2.south)--(n1.north);
\end{forest}
}
\caption{\label{fig-dass-peter-kommt-klaert-nicht-ob-non-projective}
Dass Peter kommt, klärt nicht, ob Klaus kommt.(Peter来并不能解决Klaus是否来的问题。)的依存图并不由语义来明确地决定。}
%  \caption{\label{fig-dass-peter-kommt-klaert-nicht-ob-non-projective}The dependency graph of
%  \emph{Dass Peter kommt, klärt nicht, ob Klaus kommt.} `That Peter comes does not resolve the
%  question of whether Klaus comes.' is not unambiguously determined by semantics.}
\end{figure}%
我在依存链中使用了错误的kommt,但是这是次序化的问题,而且不同于依存关系。一旦有人考虑到次序化的问题,图~\ref{fig-dass-peter-kommt-klaert-nicht-ob-non-projective}中的依存图就被排除了,这时因为ob(是否)并不前置于它的动词性依存成分kommt(来)。但是这个解释并不适用于图~\vref{fig-dass-die-Frauen-Tueren-oeffnen-dg}中的例子。这里,所有的依存成分都被正确地线性排列;只是die和Türen的非连续性是不合适的。如果要求die和Türen是连续的,我们基本上要求成分归位(请看第~\pageref{fn-projective-dg-vs-constituents}页的脚注~\ref{fn-projective-dg-vs-constituents})。相似地,有关连续性的没有任何限制的非投射性分析允准例(\mex{1}b)中的语词杂拌(word salad):
%I used the wrong \emph{kommt} in the dependency chains, but this is an issue of linearization and is
%independent of dependency. As soon as one takes linearization information into account, the %dependency graph in Figure~\ref{fig-dass-peter-kommt-klaert-nicht-ob-non-projective} is ruled out %since \emph{ob} `whether' does not precede its verbal dependent \emph{kommt} `comes'. But this %explanation does not work for the example in Figure~\vref{fig-dass-die-Frauen-Tueren-oeffnen-%dg}. Here, all dependents are linearized correctly; it is just the discontinuity of \emph{die} and 
%\emph{Türen} that is inappropriate. If it is required that \emph{die} and \emph{Türen} are %continuous, we have basically let constituents back in (see Footnote~\ref{fn-projective-dg-vs-%constituents} on page~\pageref{fn-projective-dg-vs-constituents}). 
%
%Similarly, non"=projective analyses without any constraints regarding continuity would permit the
%word salad in (\mex{1}b):
\eal
\ex[]{
\gll Deshalb klärt, dass Peter kommt, ob Klaus spielt.\\
     所以 解决 \textsc{comp} Peter 来 是否 Klaus 玩\\
%     therefore resolves that Peter comes whether Klaus plays\\
}
\ex[*]{
\gll Deshalb klärt dass ob Peter Klaus kommt spielt.\\
     所以 解决 \textsc{comp} 是否 Peter Klaus 来 玩\\
%     therefore resolves that whether Peter Klaus comes plays\\
}
%% Haider, das Beispiel ist vielleicht nicht so gut,
%% weil man argumentieren könnte, dass die Relativsätze in einer bestimmten Reihenfolge stehen
%% müssen, da die Bezugsnomen in einer bestimmten Reihenfolge stehen. 06.11.2014
%% Das andere Beispiel reicht ja auch.
%% \ex
%% \gll Sie hat keinem etwas gesagt, der ihr begegnete, was ihm nützte.\\
%%      she has nobody something said who her met       what him benefited\\
%% \mytrans{She not tell anybody who she met about something that benefited him.}
%% \ex
%% \gll Sie hat keinem etwas gesagt, der was ihr ihm begegnete nützte
\zl
(\mex{0}b)是(\mex{0}a)的一个变体,其中两个小句的论元的成分互相按照正确的语序排列,但是所有的小句都是非连续的,每个小句的成分都按照这个方式发生了变化。依存图如图\vref{fig-dass-ob-peter-klaus-kommt-spielt}所示。
%(\mex{0}b) is a variant of (\mex{0}a) in which the elements of the two clausal arguments are in
%correct order with respect to each other, but both clauses are discontinuous in such a way that the
%elements of each clause alternate. The dependency graph is shown in Figure~\vref{fig-dass-ob-peter-klaus-kommt-spielt}.
\begin{figure}
\centerline{%
\begin{forest}
dg edges
[V, l sep+=6pt
    [Adv, dg adjunct [deshalb;因此] ]
%        [Adv, dg adjunct [deshalb;therefore] ]
    [klärt;解决]
%        [klärt;resolves]
    [Subjunction, name=s1
        [dass;\textsc{comp}]]
%             [dass;that]]
    [Subjunction, name=s2
        [ob;是否]]
%             [ob;whether]]
    [hidden, phantom
        [hidden, phantom
            [N, name=n1
                [Peter;Peter]]]
        [hidden, phantom
            [N, name=n2[Klaus;Klaus]]]
        [V, name=v1
            [kommt;\strut 来]]]
%                   [kommt;\strut comes]]]
    [hidden, phantom
        [V, name=v2
            [spielt;玩]]]
%                      [spielt;plays]]]
]
\draw (s1.south)--(v1.north);
\draw (s2.south)--(v2.north);
\draw (v1.south)--(n1.north);
\draw (v2.south)--(n2.north);
\end{forest}
}
\caption{\label{fig-dass-ob-peter-klaus-kommt-spielt}Deshalb klärt dass ob Peter Klaus kommt spielt.(因此解决了Peter是否Klaus来玩)的语词杂拌的依存图由不限制非连续性的依存语法的非投射性所允准}
%\caption{\label{fig-dass-ob-peter-klaus-kommt-spielt}The dependency graph of the word salad
%  \emph{Deshalb klärt dass ob Peter Klaus kommt spielt.} `Therefore resolves that whether Peter
%  Klaus comes plays' which is admitted by non-projective Dependency Grammars that do not restrict discontinuity}
\end{figure}%
正如在第~\ref{sec-fcg-nld}节有关流变构式语法的非局部依存的分析中所解释的\indexfcgc,像英语和德语这类语言的语法必须按照这种方式来限制小句,除了向左前置的例外情况,它们都是连续的。我们可以在 \citew[\page 192]{Hudson80a}中找到类似的观点。Hudson也证明了,一个成分可以在英语\il{English}中前置,只要它的所有依存成分跟它一起前置(第184页)。这个“带有它的所有依存成分的成分”
就是基于组成成分的语法中的组成成分。区别在于,这个对象并没有一个明确的名字,并且不被看作是大部分依存语法中包含中心词和它的依存成分的不同实体。\footnote{%
不过,请参考 \citet{Hellwig2003a}的观点,即是有表示整个组成成分,而不仅仅是词的中心语的语言对象。 
}
%As was explained in Section~\ref{sec-fcg-nld} on the analysis of nonlocal dependencies in Fluid
%Construction Grammar\indexfcg, a grammar of languages like English and German has to %constrain the clauses in such a way that they are
%continuous with the exception of extractions to the left. A similar statement can be found in
% \citew[\page 192]{Hudson80a}. Hudson also states that an item can be fronted in English
%\il{English},
%provided all of its dependents are fronted with it (p.\,184). This ``item with all its dependents'' is the
%constituent in constituent"=based grammars. The difference is that this object is not given an
%explicit name and is not assumed to be a separate entity containing the head and its dependents %in
%most Dependency Grammars.\footnote{%
%See however  \citet{Hellwig2003a} for an explicit proposal that assumes that there is a linguistic
%object that represents the whole constituent rather than just the lexical head.
%}

现在总结下本节的主要内容,我给出了一个对应于依存语法的短语结构语法。我还展示了非连续成分是如何允准的。但是,还有未提及的问题:并不是某个短语具有的所有属性都跟词汇中心语一致,而且这些差异必须在某个地方表示。我将在下一小节讨论这个内容。
%Summing up what has been covered in this section so far, I have shown what a phrase structure
%grammar that corresponds to a certain Dependency Grammar looks like. I have also shown how %discontinuous
%constituents can be allowed for. However, there are issues that remained unaddressed so far: not %all
%properties that a certain phrase has are identical to its lexical head and the differences have to
%be represented somewhere. I will discuss this in the following subsection.
\isc{投射性|)}\is{projectivity|)} 

\subsubsection{在中心语和投射之间没有保持一致的特征}
%\subsubsection{Features that are not identical between heads and projections}
\label{sec-dg-daughters-mothers}
\label{sec-dg-is-simpler}
%
正如 \citet{Oliva2003a}指出的,依存语法和HPSG理论的相似之处仅限于\headvsc。
%As  \citet{Oliva2003a} points out, the equivalence of Dependency Grammar and HPSG only holds up as far as \headvs are concerned.
%\todostefan{S: many DG have several level of representations. You cannot compare HSG wioth %only one level of MTT. MTT has 7 level of %representations.} 
也就是,对应于HPSG理论中的\headvsc 的依存图的结点标签。但是,还有像\contc 这种表示语义的和\slaschc 这种表示非局部依存关系的额外特征。这些值通常在词汇中心语和它的短语投射之间有不同之处。为了说明这一情况,让我们看一下短语a book。词汇和完整短语的语义如(\mex{1})所示:\footnote{lambda表达式的内容请参阅\ref{sec-PSG-Semantik}。}
%That is, the node labels in dependency graphs correspond to the \headvs in
%an HPSG. There are, however, additional features like \cont for the semantics and \slasch for
%nonlocal dependencies. These values usually differ between a lexical head and its phrasal
%projections. For illustration, let us have a look at the phrase \emph{a book}. The semantics of the
%lexical material and the complete phrase is given in (\mex{1}):\footnote{%
%For lambda expressions see Section~\ref{sec-PSG-Semantik}.}
\eal
\ex \emph{a}: $\lambda P \lambda Q \exists x (P(x) \wedge Q(x))$
\ex \emph{book}: $\lambda y\;(\relation{book}(y))$
\ex \emph{a book}: $\lambda Q \exists x (\relation{book}(x) \wedge Q(x))$
\zl
现在,依存语法标记的问题是没有能够联系到a book的语义的NP结点(请参阅第\pageref{fig-a-book-psg-dg}页的图\ref{fig-a-book-psg-dg}),树中出现的唯一事物是词汇N的结点:book的结点。\footnote{\citet[\page 391--392]{Hudson2003a}\indexwgc 对此清楚地表述:“在依存分析中,从属词修饰中心语词的意义,所以后者带有整个短语的意义。例如,在\emph{long books about linguistics}中,由于从属词的修饰效应,词books表示‘关于语言学的长书’。”对于这个观点的具体实现请参阅图\vref{fig-wg-small-children-are-playing-outside}。另一种观点是在\mttc 中假定不同的表示层次\citep{Melcuk81a}。实际上,HPSG理论中的\contvc 也是一个不同的表示层。但是,这个表示层跟建立起来的其他结构是同步的。}不过,这不是一个大问题:词汇属性可以表示为作为不同特征值的最高结点的部分。那么,依存图中的N结点会有\contvc,它对应于完整短语的语义贡献以及对应于短语的词汇中心语的贡献的\textsc{lex-cont}值。所以对于a book来说,我们会得到下面的表达式:
%Now, the problem for the Dependency Grammar notation is that there is no NP node that could be
%associated with the semantics of \emph{a book} (see Figure~\ref{fig-a-book-psg-dg} on page~\pageref{fig-a-book-psg-dg}), the only thing present in the %tree is a node for the
%lexical N: the node for \emph{book}.\footnote{\citet[\page 391--392]{Hudson2003a}\indexwg is explicit about this: ``In dependency analysis, the %dependents modify the head word's meaning, so the latter carries the meaning of the whole phrase. For example, in  \emph{long books about %linguistics}, the word \emph{books} means `long books about linguistics' thanks to the modifying effect of the dependents.'' For a concrete %implementation of this idea see Figure~\vref{fig-wg-small-children-are-playing-outside}.An alternative is to assume different representational levels as in %\mtt \citep{Melcuk81a}. In fact the \contv in HPSG is also a different representational level. However, this representational level is in sync  with the other %structure that is build.} This is not a big problem, however: the lexical properties can be represented as part of the highest node as the value of a %separate feature. The N node in a dependency graph would then have a \contv that corresponds to the semantic contribution of the complete phrase %and a \textsc{lex-cont} value that corresponds to the contribution of the lexical head of the phrase. So for \emph{a book} we would get the following %representation:
\ea
\ms{
cont & $\lambda Q \exists x (\relation{book}(x) \wedge Q(x))$\\
lexical-cont & $\lambda y\;(\relation{book}(y))$
}
\z
使用这种表征方式就可以保证中心语与其依存成分的语义是其组成成分语义的函项。
%\textcolor{red}{译者注:这里英文是有问题的}。
%With this kind of representation one could maintain analyses in which the semantic contribution of a
%head together with its dependents is a function of the semantic contribution of the parts. 
%
%% 07.03.2015
%%
%% This could be done with a list with +/- REALIZED flags. The head could require everything to be
%% RELAIZED+ using a constraining equation.
%%
%% Now, there are probably further features in which lexical heads differ from their projections. For
%% instance a grammar has to distinguish between complete and incomplete linguistic objects. While
%% most heads select for complete linguistic objects there are some heads that are the result of
%% fusions and these select incomplete linguistic objects. For instance, the German preposition
%% \emph{vom} selects an NP that is still lacking a determiner:
%% \eal
%% \ex[]{
%% \gll von dem Hafen\\
%%      from the harbour\\
%% }
%% \ex[*]{
%% \gll von Hafen\\
%%      from harbour\\
%% }
%% \ex[]{
%% \gll vom Hafen\\
%%      from.the harbor\\
%% }
%% \ex[*]{
%% \gll vom dem Hafen\\
%%      from.the the harbour\\
%% }
%% \zl
%% Figure~\vref{fig-von-dem-vom-Hafen} shows the analysis of (\mex{0}a) and (\mex{0}c). If the node N
%% in the left has the same properties as the N node in the right figure, the grammar makes wrong predictions.
%% \begin{figure}
%% \hfill
%% \begin{forest}
%% dg edges
%% [P [von;from]
%%    [N 
%%      [Det [dem;the]]
%%      [Hafen;harbour]]]
%% \end{forest}
%% \hfill
%% \begin{forest}
%% dg edges
%% [P [vom;from.the]
%%    [N [Hafen;harbour]]]
%% \end{forest}
%% \hfill\mbox{}
%% \caption{\label{fig-von-dem-vom-Hafen}Dependency analysis with prepositions that are fused with a determiner}
%% \end{figure}%
%% What seems to be needed is a lexical valence feature and some indication which of the dependents are
%% realized. So \emph{Hafen} would select for a determiner and the N node in the left figure would mark
%% the determiner requirement as satisfied and the N node in the right figure would still have an
%% unsatisfied determiner requirement. Let us assume that such a feature 
%
现在,词汇中心与其投射之间还有更多的不同的特征。一个特征是\slaschc, 它用来表示HPSG理论中的非局部依存关系,并能在 \citet{GO2009a}方法中用来建立提升的元素和中心语之间的关系。当然,我们可以再次应用相同的策略。这样,我们会得到特征\textsc{lexical-slash}。但是,可以改进这种分析,而且词汇项的特征可以组织在一条路径下。这样,一般的架构如(\mex{1})所示:
%Now, there are probably further features in which lexical heads differ from their projections. 
%One such feature would be \slasch, which is used for nonlocal dependencies in HPSG and could be used to establish the relation between the risen %element and the head in an approach à la  \citet{GO2009a}. Of course we can apply the same trick again. We would then have a feature \textsc{lexical-%slash}. But this could be improved and the features of the lexical item could be grouped under one path. The general skeleton would then be (\mex{1}):
\ea
\ms{
cont & \\
slash & \\
lexical & \ms{ cont & \\
               slash & \\ }
}
\z
但是如果我们将\textsc{lexical}重新命名为\textsc{head-dtr},我们基本上会得到HPSG理论的表达式。
%But if we rename \textsc{lexical} to \textsc{head-dtr}, we basically get the HPSG
%representation. 
%
 \citet[\page 602]{Hellwig2003a}证明了他的依存语法的特殊版本叫做依存合一语法\isc{依存合一语法}\is{Dependency Unification Grammar (DUG)},他认为统治中心语选择了带有所有他们的子结点的完整结点。这些结点的属性与他们的中心语不同(第604页)。他们实际上是成分。所以这个依存语法的非常明晰和形式化的变体与HPSG理论十分相似,正如Hellwig他自己证明的(第603页)。Hudson的词语法也是明晰表示的,正如下面所示的,它与HPSG理论也是十分相似的。图\vref{fig-wg-small-children-are-playing-outside}中的表示是图\vref{fig-wg-small-children-are-playing-outside-abbreviated}表示的简缩版的详细描写。
% \citet[\page 602]{Hellwig2003a} states that his special version of Dependency Grammar,
%which he calls Dependency Unification Grammar\is{Dependency Unification Grammar (DUG)}, assumes that
%governing heads select complete nodes with all their daughters. These nodes may differ in their
%properties from their head (p.\,604). They are in fact constituents. So this very explicit and
%formalized variant of Dependency Grammar is very close to HPSG, as Hellwig states himself (p.\,603).
%
%Hudson's Word Grammar \citeyearpar{Hudson2017a} is also explicitly worked out and, as will be shown,
%it is rather similar to HPSG. The representation in
%Figure~\vref{fig-wg-small-children-are-playing-outside} is a detailed description of what the 
%abbreviated version in Figure~\vref{fig-wg-small-children-are-playing-outside-abbreviated} stands for.
\begin{figure}
\begin{forest}
  wg
  [were
    [children
      [
        [small]
        []
      ]
    ]
    [
      []
    ]
    [playing
      [
        []
        [outside]
      ]
    ]
  ]
  \draw[deparrow] (were'') to[out=west, in=north] (children'');
  \draw[deparrow] (children') to[out=west, in=north] (small);
  \draw[deparrow] (were') to[out=60, in=120] (playing'');
  \draw[deparrow] (playing'') to[out=220, in=east] (children');
  \draw[deparrow] (playing') to[out=east, in=north] (outside);
\end{forest}
\caption{\label{fig-wg-small-children-are-playing-outside}根据 \citet[\page 105]{Hudson2017a}的Small children were
    playing outside.的分析}
%\caption{\label{fig-wg-small-children-are-playing-outside}Analysis of \emph{Small children were
%    playing outside.} according to  \citet{Hudson2017a}}
\end{figure}%
\begin{figure}
\begin{forest}
  wg
  [,phantom
   [small]
   [children]
   [were]
   [playing]
   [outside]
  ]
%  \draw[deparrow] (were.north) [bend.left] to ([xshift=-10pt]children.north);(attemtion)
  \draw[deparrow] ([xshift=-3pt]were.north) to[out=north, in=north] ([xshift=5pt]children.north);
  \draw[deparrow] ([xshift=-3pt]children.north) to[out=north, in=north] (small);
  \draw[deparrow] ([xshift=3pt]were.north) to[out=north, in=north] ([xshift=-5pt]playing.north);
  \draw[deparrow] (playing.north) to[out=north, in=north] (children);
  \draw[deparrow] ([xshift=3pt]playing.north) to[out=north, in=north] (outside);
\end{forest}
\caption{\label{fig-wg-small-children-are-playing-outside-abbreviated}根据 \citet[\page 105]{Hudson2017a}的Small children were  playing outside.的简略分析}
%\caption{\label{fig-wg-small-children-are-playing-outside-abbreviated}Abbreviated analysis of \emph{Small children were  playing outside.} according to  \citet{Hudson2017a}}
\end{figure}%
在第一个图中显示的是两个结点的组合得到一个新结点。例如,playing和outside的组合得到playing$'$,small和children的组合得到children$'$,而且children$'$和playing$'$的组合得到playing$''$。were和playing$''$的组合得到were$'$,以及 children$''$和were$'$的组合得到were$''$。唯一剩下需要解释的是为什么有一个结点children,它不是两个结点的组合(即children$''$)的结果。顶点与底部的连线表示缺省的承继关系\isc{承继!缺省承继}\is{inheritance!default}。即,上层结点继承了缺省的底层结点的所有属性。缺省的可以被覆盖,即上层结点的信息会与统治结点的信息不同。这就使得按照语义的组合性的操作成为可能:两个结点组合的结果的那个结点的语义是两个组合的结点的意义的组合。再来看children,children$'$具有必须邻接到playing的属性,但是因为这个结构是一个升位结构,其中children提升到了were的主语位置上,这个属性被children的一个新的实例覆盖了,即children$''$。
%What is shown in the first diagram is that a combination of two nodes results in a new node. For
%instance, the combination of \emph{playing} and \emph{outside} yields \emph{playing}$'$, the
%combination of \emph{small} and \emph{children} yields \emph{children}$'$, and the combination of
%\emph{children}$'$ and \emph{playing}$'$ yields \emph{playing}$''$. The combination of \emph{were}
%and \emph{playing}$''$ results in \emph{were}$'$ and the combination of \emph{children}$''$ and
%\emph{were}$'$ yields \emph{were}$''$. The only thing left to explain is why there is a node for
%\emph{children} that is not the result of the combination of two nodes, namely
%\emph{children}$''$. The line with the roof at the bottom stands for default inheritance\is{inheritance!default}. That
%is, the upper node inherits all properties from the lower node by default. Defaults can be
%overridden, that is, information at the upper node may differ from information at the dominated
%node. This makes it possible to handle semantics compositionally: nodes that are the result of the
%combination of two nodes have a semantics that is the combination of the meaning of the two combined
%nodes. Turning to \emph{children} again, \emph{children}$'$ has the property that it must be adjacent to \emph{playing}, but since the
%structure is a raising structure in which \emph{children} is raised to the subject of \emph{were},
%this property is overwritten in a new instance of \emph{children}, namely \emph{children}$''$.
%
这里有趣的是我们几乎得到了一个名词性短语结构树,如果我们通过句法范畴替换图\ref{fig-wg-small-children-are-playing-outside}中的图的词的话。这个替换的结果如图\vref{fig-small-children-are-playing-outside}所示。
%The interesting point now is that we get almost a normal phrase structure tree if we replace the words in the diagram in
%Figure~\ref{fig-wg-small-children-are-playing-outside} by syntactic categories. The result of the
%replacement is shown in Figure~\vref{fig-small-children-are-playing-outside}.
\begin{figure}
\begin{forest}
  sm edges
  [V{[\emph{fin}]}$''$
    [N$''$
      [N$'$, tier=nbar, name=nbar, edge=dashed
        [Adj [small;小]]
        [N   [children;孩子们]] ] ]
    [V{[\emph{fin}]}$'$
      [V{[\emph{fin}]} [were;\textsc{aux}]]
      [V{[\emph{ing}]}$''$, name=ving, l sep+=2ex
        [V{[\emph{ing}]}$'$, tier=nbar
          [V{[\emph{ing}]} [playing;玩]]
          [Adv [outside;外面]] ] ] ]
  ]
  \draw[dashed] (ving.south)--(nbar.north);
\end{forest}
\caption{\label{fig-small-children-are-playing-outside}带有范畴符号的Small children are
   playing outside.的分析}
%\caption{\label{fig-small-children-are-playing-outside}Analysis of \emph{Small children are
%   playing outside.} with category symbols}
\end{figure}%
这张图(由虚线标记)中唯一不同的地方是N$'$与V{[ing]}$'$相组合,而且N$'$的母结点,即N$''$,与V{[fin]}$'$相组合。正如上面解释的,这依赖于词语法中升位的分析,它包括提升的项目和它的中心语之间的多重依存关系。在图\ref{fig-small-children-are-playing-outside}中有两个N结点(N$'$和N$''$),而且在图\ref{fig-wg-small-children-are-playing-outside}中有children的两个实例。除了这个,这个结构对应于HPSG语法可以允准的成分。Hudson范式中的在底部用线连接到顶端的结点与使用默认承继的子结点是相关的。这对于使用默认承继关系的HPSG的许多版本来说也是非常相似的。比如说, \citet[\page
33]{GSag2000a-u}使用了广义中心语特征原则\isc{原则!广义中心语特征原则}\is{principle!Generalized Head Feature},它默认将中心语子结点的所有特征投射到母结点上。
%The only thing unusual in this graph (marked by dashed lines) is that N$'$ is combined with V{[\emph{ing}]}$'$ and the mother
%of N$'$, namely N$''$, is combined with V{[\emph{fin}]}$'$. As explained
%above, this is due to the analysis of raising in Word Grammar,
%which involves multiple dependencies between a raised item and its heads. There are two N nodes
%(N$'$ and N$''$) in Figure~\ref{fig-small-children-are-playing-outside} and two instances of
%\emph{children} in Figure~\ref{fig-wg-small-children-are-playing-outside}. 
%Apart from this, the
%structure corresponds to what an HPSG grammar would license. The nodes in Hudson's diagram which are connected
%with lines with roofs at the bottom are related to their children using default
%inheritance. This too is rather similar to those versions of HPSG that use default inheritance. For
%instance,  \citet[\page 33]{GSag2000a-u} use a Generalized Head Feature
%Principle\is{principle!Generalized Head Feature} that projects all properties of
%the head daughter to the mother by default.
%
本节的结论是,短语结构语法和依存语法的唯一理论区别在于中间结构是如何假设的:是否存在没有主语的VP呢?附加语附加成分是否有中间结点呢?在没有涵盖了语义表示的全部可行的方案下回答这些问题是十分困难的。那些可行的方法,如Hudson和Hellwig的方法,提出了中间表达式,它使得这些方法跟基于短语结构的方法十分相似。如果我们将全部可行的依存语法的变体的结构与短语结构语法相比较的话,很清楚的是,依存语法更为简单这个观点是不受欢迎的。这个观点适用于图\ref{fig-wg-small-children-are-playing-outside-abbreviated}中的紧缩的范式的表示,但是它并不适用于全部可行的分析。
%The conclusion of this section is that the only principled difference between phrase structure grammars and Dependency Grammar is the question of %how much intermediate structure is assumed: is there a VP without the subject? Are there intermediate nodes for adjunct attachment? It is difficult to %decide these questions in the absence of fully worked out proposals that include semantic representations. Those proposals that are worked out -- like %Hudson's and Hellwig's -- assume intermediate representations, which makes these approaches rather similar to phrase structure-based approaches. If %one compares the structures of these fully worked out variants of Dependency Grammar with phrase structure grammars, it becomes clear that the %claim that Dependency Grammars are simpler is unwarranted. This claim holds for compacted schematic representations like Figure~\ref{fig-wg-small-%children-are-playing-outside-abbreviated} but it does not hold for fully worked out analyses.
%% \subsection{Extraction and coordination}
%% \label{sec-dg-coordination}
%
\subsubsection{非中心语构式}
%\subsubsection{Non-headed constructions}
\label{sec-headless-constructions-dg}
%
 \citet[\S~4.E]{Hudson80a}讨论了(\mex{1})中所示的无中心语的结构:
% \citet[Section~4.E]{Hudson80a} discusses headless constructions like those in (\mex{1}):
\eal
\ex 
\gll the rich\\
\textsc{det} 富有\\
\mytrans{富人们}
\ex 
\gll the biggest\\
\textsc{det} 最大\\
\mytrans{最大的东西}
\ex 
\gll the longer the stem\\
\textsc{det} 更长 \textsc{det} 茎\\
\mytrans{更长的茎}
\ex 
\gll (with) his hat over his eyes\\
     \hspaceThis{(}\textsc{prep} 他的 帽子 \textsc{prep} 他的 眼睛\\
\mytrans{他的帽子遮住了眼睛}
\zl
他认为,术语形容词(adjective)和名词(noun)应该属于术语实体(substantive),它包括这两个术语。然后他指出“如果一条规则需要覆盖传统上指称为带或者不带中心语的名词短语,它只称为‘名词’,而且这可以自动允许要么是实体要么是形容词作为中心语的结构。”(第195页)但是,这里需要提出的问题是,像the rich这样的实体性短语的内部依存结构是什么样的呢。连接这些项目的唯一方式看起来是假定限定词依存于形容词。但是这可以允许像the rich man的短语的两个结构:一个是限定词依存于形容词,另一个是它依存于名词。所以,词性的未分析不能解决这个问题。当然,所有无中心语的结构的所有问题都可以通过假定空成分来解决。\footnote{请参阅\ref{sec-psg-np}关于名词短语的短语结构语法中的空中心语的假设。}在\hpsgc 的关系小句的分析中是这样做的\citep[\S~5]{ps2}。英语和德语的关系小句包括一个短语,它包括一个关系词和一个句子,其中关系短语是缺失的。Pollard\& Sag提出了空关系代词,它选择关系小句以及带有空位的小句\citep[\page 216--217]{ps2}。相似的分析可以在依存语法中找到(\citealp[\page  291]{Eroms2000a})。\footnote{%
依存语法表示通常有一个\stem{d}成分作为关系小句的中心语。但是,由于关系代词也出现在小句中,而且由于\stem{d}没有被两次发音,假定一个额外的\stem{d}中心语基本上就是假定一个空的中心语。
%
另一个选择是假定具有多重功能的词汇:所以说,一个关系代词可以既是一个中心语,也同时是一个从属语(\citealp[\S 246, §8--11]{Tesniere2015a-not-crossreferenced};\citealp[\page xlvi]{OK2015a};\citealp[\page  129--130]{Kahane2009a})。至少Kahane的分析是\ref{Abschnitt-Relativsaetze-CG}讨论的范畴语法分析的一个例子,而且它具有相同的问题:如果关系代词是中心语,它选择了缺失关系代词的小句,不容易看到这个分析是如何扩展到(i)中的抽吸过程\isc{抽吸过程}\is{pied-piping}的,其中被提取的成分是一个包括关系代词,而不是代词本身的完整短语。
\ea
\gll die Frau, von deren Schwester ich ein Bild gesehen habe\\
     \textsc{det} 女人 \textsc{prep} \textsc{rel} 姐妹 我 一 图片 看见 \textsc{aux}\\
\mytrans{我在她的姐妹那里看到一张照片的那个女人}
\zlast
}
%He argues that the terms \emph{adjective} and \emph{noun} should be accompanied by the term
%\emph{substantive}, which subsumes both terms. Then he suggests that \emph{if a rule needs to cover the constructions traditionally referred to as %noun-phrases, with or without heads, it just refers to `nouns', and this will automatically allow the constructions to have either substantives or adjectives %as heads.} (p.\,195) The question that has to be asked here, however, is what the internal dependency structure of substantive phrases like \emph{the %rich} would be. The only way to connect the items seems to be to assume that the determiner is dependent on the adjective. But this would allow for %two structures of phrases like \emph{the rich man}: one in which the determiner depends on the adjective and one in which it depends on the noun. So %underspecification of part of speech does not seem to solve the problem. Of course all problems with non-headed constructions can be solved by assuming empty elements.\footnote{See Section~\ref{sec-psg-np} for the assumption of an empty head in a phrase structure grammar for noun phrases.}
%This has been done in \hpsg in the analysis of relative clauses \citep[Chapter~5]{ps2}. English and German relative clauses consist of a phrase that %contains a relative word and a sentence in which the relative phrase is missing. Pollard\& Sag assume an empty relativizer that selects for the relative %phrase and the clause with a gap \citep[\page 216--217]{ps2}. Similar analyses can be found in Dependency Grammar (\citealp[\page  291]%{Eroms2000a}).\footnote{The Dependency Grammar representations usually have a \stem{d} element as the head of the relative  clause. However, since the relative pronoun is also present in the clause and since the \stem{d}  is not pronounced twice, assuming an additional \stem{d} head is basically assuming an empty  head. 
%
%  Another option is to assume that words may have multiple functions: so, a relative pronoun may be
%  both a head and a dependent simultaneously (\citealp[Chapter 246, §8--11]{Tesniere2015a-not-crossreferenced}; \citealp[\page xlvi]{OK2015a}; %\citealp[\page  129--130]{Kahane2009a}). At least the analysis of Kahane is an instance of the Categorial  Grammar analysis that was discussed in %Section~\ref{Abschnitt-Relativsaetze-CG} and it suffers from the same problems: if the  relative pronoun is a head that selects for a clause that is %missing the relative pronoun, it is not easy to see how  this analysis extends to cases of pied-piping\is{pied-piping} like (i) in which the extracted %element is a complete phrase containing the relative pronoun rather than just the pronoun itself.
%\ea
%\gll die Frau, von deren Schwester ich ein Bild gesehen habe\\
%     the woman of whose sister I a picture seen have\\
%\mytrans{the woman of whose sister I saw a picture}
%\zlast
%}
现在,另一种分析空成分的方法是短语构式。\footnote{%
请参阅第\ref{Abschnitt-Diskussion-leere-Elemente}章关于空成分的一般分析和\ref{Abschnitt-Relativ-Interrogativsaetze}关于关系小句的特殊分析。
}\cite{Sag97a}研究了英语的关系小句,他提出了关系小句的短语分析,其中关系小句和它从新的短语中提取出的小句。 \citet{Babel}提出了一个相似的分析,并且记录在 \citew[\S~10]{Mueller99a}中。正如\ref{Abschnitt-Relativsaetze-CG}所讨论的,认为关系短语中的关系代词或者其他成分作为整个关系小句的中心语是不可行的,而且认为动词作为整个小句的中心语也是不可行的(Sag),因为关系小句修饰\nbar{}s,(定式)动词通常不会这样可投射的。
%Now, the alternative to empty elements are phrasal constructions.\footnote{%
%See Chapter~\ref{Abschnitt-Diskussion-leere-Elemente} on empty elements in general and
%Subsection~\ref{Abschnitt-Relativ-Interrogativsaetze} on relative clauses in particular.
%} \cite{Sag97a} working on relative clauses in
%English suggested a phrasal analysis of relative clauses in which the relative phrase and the clause
%from which it is extracted form a new phrase. A similar analysis was assumed by
% \citet{Babel} and is documented in  \citew[Chapter~10]{Mueller99a}. As was discussed in Section~\ref{Abschnitt-Relativsaetze-CG} it is
%neither plausible to assume the relative pronoun or some other element in the relative phrase to be the head of the entire relative clause, nor is it %plausible to assume the verb to be the head of the entire relative clause (pace Sag), since relative clauses modify \nbar{}s, something that projections
%of (finite) verbs usually do not do. 
%% Vielleicht ist das mit MRS kein Problem mehr. 27.11.2015
%% Furthermore, the semantics of verbal projections is verbal and
%% not the semantics that would be required for linguistic objects that modify nouns \citep[\page
%%   474]{Sag97a}.\footnote{%
%%    \citet[\page 474]{Sag97a} solves this problem by assuming a special Head-Adjunct Schema for
%%   relative clauses that modify nouns.%
%% }
所以说,假定一个空的中心语或者短语模式看起来是唯一的选择了。
%So assuming an empty head or a phrasal schema seems to be the only option.
%\todostefan{S: There is a third very interesting option considered by Tesnière 1959 (see our intro) and before by Sicard 1801 and maybe before and exploited in all my formalizations of extraction including the HPSG's one in Kahane 2009: the wh-word has a double position, it is both a complementizer and a pronoun.}
%
第\ref{Abschnitt-Phrasal-Lexikalisch}章致力于讨论某些语法现象是否应该分析为包括短语结构配置或者词汇分析更为合适或者更适合于模拟某些现象。我认为所有跟配价互动的现象都应该按照词汇的来处理。但是,除了配价互动现象之外还有其他现象需要用基于词汇的分析,为了分析所有语言学现象依存语法必须接受基于词汇的分析。
%Chapter~\ref{Abschnitt-Phrasal-Lexikalisch} is devoted to the discussion of whether
%certain phenomena should be analyzed as involving phrase structural configurations or whether
%lexical analyses are better suited in general or for modeling some phenomena. I argue there that all
%phenomena interacting with valence should be treated lexically. But there are other phenomena as
%well and Dependency Grammar is forced to assume lexical analyses for all linguistic
%phenomena.
%\todostefan{S: yes that's true.
%As the marker of the construction is the prep, we'll have a separate entry for each prep that
%allows this construction.} 
总有一些成分是其他成分所依存的。 \citet{Jackendoff2008a}\isc{构式!N-P-N|(}\is{construction!N-P-N|(}认为,像(\mex{1})中的N-P-N构式的一个成分作为中心语的观点是没有意义的。
%There always has to be some element on which others depend. It has been argued by
% \citet{Jackendoff2008a}\is{construction!N-P-N|(} that it does not make sense to assume that one of the elements in N-P-N constructions like those in %(\mex{1}) is the head.
\eal
\ex 
\gll day by day, paragraph by paragraph, country by country\\
天 \textsc{prep} 天 段落 \textsc{prep} 段落 国家 \textsc{prep} 国家\\
\mytrans{一天又一天,一段又一段,一个国家又一个国家}
\ex 
\gll dollar for dollar, student for student, point for point\\
美元 \textsc{prep} 美元 学生 \textsc{prep} 学生 点 \textsc{prep} 点\\
\mytrans{美元换美元,学生换学生,点换点}
\ex 
\gll face to face, bumper to bumper\\
脸 \textsc{prep} 脸 保险杠 \textsc{prep} 保险杠\\
\mytrans{脸对脸,保险杠对保险杠}
\ex 
\gll term paper after term paper, picture after picture\\
学期 试卷 \textsc{prep} 学期 试卷 图片 \textsc{prep} 图片\\
\mytrans{试卷接着试卷,图片接着图片}
\ex 
\gll book upon book, argument upon argument\\
书 \textsc{prep} 书 论点 \textsc{prep} 论点\\
\mytrans{书堆着书,论点叠着论点}
\zl
当然,有方法可以模拟所有可以在短语结构的框架下(如GPSG、CxG、HPSG或者简便句法:一个空中心语)模拟的现象。图\vref{fig-n-p-n}显示了student after student的分析。
%Of course there is a way to model all the phenomena that would be modeled by a phrasal construction in frameworks like GPSG, CxG, HPSG, or %Simpler Syntax: an empty head. Figure~\vref{fig-n-p-n} shows the analysis of \emph{student after student}.
\begin{figure}
\begin{forest}
dg edges
[N
  [\trace]
  [N [student;学生]]
  [P [after;\textsc{prep}]]
  [N [student;学生]]]
\end{forest}
\caption{\label{fig-n-p-n}带有空中心语的N-P-N构式的依存语法分析}
%\caption{\label{fig-n-p-n}Dependency Grammar analysis of the N-P-N Construction with empty head}
\end{figure}%
空N的词汇项是非常特殊的,因为没有相似的非空的词汇名词,即没有名词选了两个光杆N和一个P。
%The lexical item for the empty N would be very special, since there are no similar non-empty lexical
%nouns, that is, there is no noun that selects for two bare Ns and a P.
%% Note that the N-P-N construction is special in that Ns are combined rather than NPs. This means that
%% it must be possible that the selecting head selects for incomplete Ns. Hence, we have another
%% property that differs between words and full phrases. Again this seems to make it necessary to
%% distinguish between lexical and phrasal nodes (see Section~\ref{sec-dg-daughters-mothers}).
%
 \citet{Bargmann2015a}指出了N-P-N构式的另一使得事情更加复杂的方面。这个模式没有受限于两个名词。可以是任意数量的:
% \citet{Bargmann2015a} pointed out an additional aspect of the N-P-N construction, which makes things more complicated. The pattern is not restricted %to two nouns. There can be arbitrarily many of them:
\ea
\gll Day after day after day went by, but I never found the courage to talk to her.\\
天 \textsc{prep} 天 \textsc{prep} 天 过 \textsc{adv} 但是 我 从未 找到 \textsc{det} 勇气 \textsc{inf} 说话 \textsc{perp} 她\\
\mytrans{一天又一天过去了,但是我仍未找到勇气跟她说话。}
\z
所以说,Bargmann提出了(\mex{1})中的模式,而不是N-P-N范式,这里`+'\isc{$+$}\is{$+$}表示一个序列的至少一次重复。
%So rather than an N-P-N pattern Bargmann suggests the pattern in (\mex{1}), where `+'\is{$+$} stands for at least one repetition of a sequence.
\ea
\label{n-p-n-plus-cx}
N (P N)+
\z
现在,这种范式在基于选择的方法中是十分困难的,因为我们需要假设一个空的中心语或者选择相同介词、名词或名词性短语的任意数量对儿的名词。当然,我们可以假设P和N构成了某种成分,但是仍然有人需要确保使用了正确的介词,而且名词或者名词性投射具有正确的语音形式。另一种可能是要假设N-P-N中的第二个N可以是N-P-N,由此允许模式的循环。但是如果我们按照这个方法,核查这种限制就非常困难了,其中所包含的N应该具有相同或者至少类似的语音形式。
%Now, such patterns would be really difficult to model in selection"=based approaches, since one
%would have to assume that an empty head or a noun selects for an arbitrary number of pairs of the
%same preposition and noun or nominal phrase. Of course one could assume that P and N form some sort of constituent, but still one would have to %make sure that the right preposition is used and that the noun or nominal projection has the right phonology. Another possibility would be to assume %that the second N in N-P-N can be an N-P-N and thereby allow recursion in the pattern. But if one follows this approach it is getting really difficult to %check the constraint that the involved Ns should
%have the same or at least similar phonologies.
%
解决这些问题的一种方式当然可以是假定这里具有特殊的机制来指派一个新的范畴到一个或几个成分上。这在本质上可以是一个非中心语的短语结构规则,而且这是\tes 所提出的:转用规则(请参阅\ref{sec-transfer-dg})。但是,这当然是存粹的依存语法向混合模型的一个扩展。\isc{构式!N-P-N|)}\is{construction!N-P-N|)}
%One way out of these problems would of course be to assume that there are special combinatorial
%mechanisms that assign a new category to one or several elements. This would basically be an
%unheaded phrase structure rule and this is what \tes suggested: transfer rules (see
%Section~\ref{sec-transfer-dg}). But this is of course an extension of pure Dependency Grammar
%towards a mixed model.\is{construction!N-P-N|)}
%
请参阅\ref{sec-why-phrasal}有关深入问题的讨论,它对纯粹的基于选择的语法来说可能是有问题的。
%See Section~\ref{sec-why-phrasal} for the discussion of further cases which are probably problematic for purely selection"=based grammars.
%

%\bigskip
\exercisesfirst{
%\section*{练习题}
%\section*{Exercises}
%
请给出下面的三个句子的依存图:
%Provide the dependency graphs for the following three sentences:
\eal
\ex 
\gll Ich habe einen Mann getroffen, der blonde Haare hat.\\
     我 \textsc{aux} 一 男人 见面 \textsc{rel} 金色 头发 有\\
\mytrans{我跟一位有着金色头发的男士见面了。}
%     I have a man met who blond hair has\\
%\mytrans{I have met a man who has blond hair.}
\ex 
\gll Einen Mann getroffen, der blonde Haare hat, habe ich noch nie.\\
     一 男人 见面 \textsc{rel} 金色 头发 有 \textsc{aux} 我 还 从未\\
\mytrans{我从未见到过有着金色头发的男人。}
%     a man met who blond hair has have I yet never\\
%\mytrans{I have never met a man who has blond hair.}
\ex 
\gll Dass er morgen kommen wird, freut uns.\\
     \textsc{comp} 他 明天 来 将 愉悦 我们\\
\mytrans{他明天会来让我们很高兴。}
%     that he tomorrow come will pleases us\\
%\mytrans{That he will come tomorrow pleases us.}
\zl
你可以使用非可投射的依存关系。对于关系小句的分析,学者们通常提出一个抽象的实体,它的功能是可修饰名词的从属语,以及关系小句中的动词的中心语。
%You may use non-projective dependencies. For the analysis of relative clauses authors usually
%propose an abstract entity that functions as a dependent %of the modified noun and as a head of the
%verb in the relative clause.
}
%
%\section*{延伸阅读}
%\section*{Further reading}
\furtherreading{
%
在第\ref{chap-GB}章的延伸阅读部分,我推荐了\emph{Syntaktische Analyseperspektiven}(《句法分析的不同视角》)这本书。该书的各个章节是由不同理论的支持者所著,并且分析了相同的新闻语料。这本书还有 \citet{Engel2014a}写的一章,他提出了他的依存语法的版本,即从属动词语法(Dependent Verb Grammar)。
%In the section on further reading in Chapter~\ref{chap-GB}, I referred to the book called
%\emph{Syntaktische Analyseperspektiven} `Syntactic perspectives on analyses'. The chapters in this book have been written by proponents of various %theories and all analyze the same newspaper article. The book also contains a chapter by  \citet{Engel2014a}, assuming his version of Dependency %Grammar, namely \emph{Dependent Verb Grammar}.
%
 \citet*{AEEHHL2003a-ed-not-crossreferenced,AEEHHL2006a-ed-not-crossreferenced}出版了依存关系和配价手册,它讨论了依存语法涉及的所有方面。本章引用了其中的很多文献。对比依存语法和其他理论的文章在本书的语境下是尤为重要的: \citet{Lobin2003a}比较了依存语法和范畴语法, \citet{Oliva2003a}分析配价的表示和HPSG中的依存关系,而且 \citet*{BJR2003a-u}描述了配价和依存是如何覆盖在TAG中的。 \citet{Hellwig2006a}比较了基于规则的语法与依存语法,特别关注于计算程序的剖析。
% \citet*{AEEHHL2003a-ed-not-crossreferenced,AEEHHL2006a-ed-not-crossreferenced} published a handbook on dependency and valence that %discusses
%all aspects related to Dependency Grammar in any imaginable way. Many of the papers have been cited in this chapter. Papers comparing %Dependency Grammar with other theories are especially relevant in the context of this book:  \citet{Lobin2003a} compares Dependency Grammar and
%Categorial Grammar,  \citet{Oliva2003a} deals with the representation of valence and dependency in
%HPSG, and  \citet*{BJR2003a-u} describe how valence and dependency are covered in
%TAG.  \citet{Hellwig2006a} compares rule"=based grammars with Dependency Grammars with special consideration given to parsing by computer %programs.
%
 \citet{OG2012a-u}比较了依存语法和构式语法, \citet*{OPG2011a}认为某些最简方案的变体实际上是基于依存分析的重新发现。
% \citet{OG2012a-u} compare Dependency Grammar with Construction Grammar and  \citet*{OPG2011a} argue
%that certain variants of Minimalism are in fact reinventions of dependency"=based analyses.
%
 \citet{Tesniere59a-u}提出的依存语法的原始工作在德语\citep{Tesniere80a-u}中是部分可获得的,在英语中是全部可获得的\citep{Tesniere2015a-not-crossreferenced}。
%The original work on Dependency Grammar by  \citet{Tesniere59a-u} is also available in parts in German \citep{Tesniere80a-u} and in full in English %\citep{Tesniere2015a-not-crossreferenced}.
}
%
\if0
%
% todo: Kahane2003a-u,  BJR2003a-u
% Starosta2003a Dependency Grammar and Lexicalism -> Fernabhängigkeiten und Scrambling, Vergleich
% Minimalism & Passiv
% Hudson2003a-u bei Vererbung zitieren
% Limits: Gross2003a-u
% formal foundations: Diskontinuität, Semantics Broeker2003a-u
% Hoberg2006 63: Wortstellung
% Askedal2006 65 Infinitive
% Gross2003a: 341 Locality of Selection
% Colliander2003a:266 zur DP/NP-Diskussion Tesniere -> NP
% zitiert Lobin95a für vollständige Diskussion
%
% Lobin2003a:329  Zitiert Hudson und Pickering & Barry mit Konzepten, die mir der Catenae zu
% entsprechen scheinen
% Baumgärtner 1965, 1970
% Vennemann 77
% Vater 73
% Werner 73
% Leere Elemente für Relativsätze: Engel 1994, Eroms 2000
% S. Kahane:
%
The necessary separation between the syntactic dependencies and the linear order is extensively
discussed in the beginning of Tesniere's book.
%
Thomas Gross
Some Observations on the Hebrew Desiderative Construction – A Dependency-Based Account in Terms of Catenae
%
In the section 2.2 in the attached paper, I discuss the reasons why DG never succeeded in developing its own morphology. Section 2.3 gives a rough overview of my (and Tim's) notion of a dependency-based morphology.
%
While I believe that the catena makes a compelling contribution to the understanding of displacement, ellipsis, idioms, predicate-argument-structure, and all kinds of construction, its attraction is also in its flexibility. It allows a gradient approach to phenomena ranging from syntax to morphology. In a 2013 paper in ZfS, Tim and I make that case explicitly.
%
Liu Haitao:
%
Some possible references:
%
Baum, R. (1976) Dependenzgrammatik: Tesnières Modell der Sprachbeschreibung in wissenschaftsgeschichtlicher und kritischer Sicht. (Zeitschrift für romanische Philologie, Beiheft 151.) Tübingen: Max Niemeyer.
Groß, T. M. (1999). Theoretical Foundations of Dependency Syntax. München: iudicium.
Hudson. R. A. (2007) Language Networks: The New Word Grammar. Oxford University Press.
Hudson. R. A. (2010) An Introduction to Word Grammar. Cambridge University Press.
Liu, Haitao (2009) Dependency Grammar: from theory to practice. Beijing: Science Press. 
Melcuk, I. A. (1988) Dependency syntax: theory and practice. Albany: State University Press of New York.
Schubert, K. (1987) Metataxis: contrastive dependency syntax for machine translation. Dordrecht: Foris.
Starosta, S. (1988) The case for lexicase. London: Pinter.
Weber, H.J. (1997) Dependenzgrammatik. Ein interaktives Arbeitsbuch, Tübingen: Gunter Narr.
%
Within 60 years of the original French publication, there have been translations into German (1980), Russian (1988), Spanish (1994), Japanese (2007) and Italian (2008), referenced below.
%
Tesnière, Lucien. 1980. Grundzüge der strukturalen Syntax. Tr. Ulrich Engel. Stuttgart: Klett-Cotta. [German translation of Tesnière 1959]
Теньер, Люсьен. 1988. Основы структурного синтаксиса. Tr.  В. Г. Гака. Москва: Прогресс. [Russian translation of Tesnière 1959]
Tesnière, Lucien. 1994. Elementos de sintaxis estructural. Tr. Esther Diamante. Madrid: Editorial Gredos. [Spanish translation of Tesnière 1959]
ルシアン・テニエール. 2007. 構造統語論要説. Tr. 小泉保. 東京: 研究社. [Japanese translation of Tesnière 1959]
Tesnière, Lucien. 2008. Elementi di sintassi strutturale. Tr. Germano Proverbio & Anna Trocini Cerrina. Torino: Rosenberg & Sellier. [Italian translation of Tesnière 1959]
%
\fi
%
% lulu/wsun/sisi DONE
%      <!-- Local IspellDict: en_US-w_accents -->


