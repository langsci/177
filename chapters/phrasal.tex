%% -*- coding:utf-8 -*-
\chapter{基于短语的分析 vs.\ 基于词的分析}
\label{Abschnitt-Phrasal-Lexikalisch}\label{chap-phrasal}

\chaptersubtitle{与Stephen Wechsler合著}
当我们比较本书所述的各种理论时,会发现一个重要的问题:价和句子结构(或者更加宽泛地来说句法结构)是由词汇信息决定的;还是句法结构有独立的地位和意义,词项只是插入到句法结构中?大致说来,管约论/最简方案、词汇功能语法、范畴语法、中心语驱动的短语结构语法和依存语法等框架都是基于词汇的,而广义的短语结构语法和构式语法(\citealp{Goldberg95a,Goldberg2003b,Tomasello2003a,Tomasello2006c,Croft2001a})都是基于短语的方法。然而,这只是一种大致的分类,因为最简方案(Borer的外骨架方法,\citeyear{Borer2003a-u})和词汇功能语法(\citealp{Alsina96a};\citealp{ADT2008a,ADT2013a})阵营中也有非-词汇的方法,而构式语法阵营中也有词汇方法(基于符号的构式语法,参见~\ref{sec-SBCG})。在认知语法(\citealp{Dabrowska2001a};\citealp[\page 169]{Langacker2009a})和更简句法\citep{CJ2005a,Jackendoff2008a}等理论框架中也广泛使用基于短语的方法。本书不对这两种框架做进一步的介绍。
%This section deals with a rather crucial aspect when it comes to the comparison of the theories
%described in this book: valence and the question whether sentence structure, or rather syntactic
%structure in general, is determined by lexical information or whether syntactic structures have an
%independent existence (and meaning) and lexical items are just inserted into them. Roughly speaking,
%frameworks like GB/Minimalism, LFG, CG, HPSG, and DG are lexical, while GPSG and Construction
%Grammar (\citealp{Goldberg95a,Goldberg2003b,Tomasello2003a,Tomasello2006c,Croft2001a}) are
%phrasal approaches. This categorization reflects tendencies, but there are non-lexical 
%approaches in Minimalism (Borer's exoskeletal approach, \citeyear{Borer2003a-u}) and LFG
%(\citealp{Alsina96a}; \citealp{ADT2008a,ADT2013a}) and there are lexical approaches in Construction
%Grammar (Sign"=Based Construction Grammar, see Section~\ref{sec-SBCG}). The phrasal approach is
%wide"=spread also in frameworks like Cognitive Grammar (\citealp{Dabrowska2001a}; \citealp[\page 169]{Langacker2009a})
%and Simpler Syntax \citep{CJ2005a,Jackendoff2008a} that could not be discussed in this book.
上述问题用实例来说就是:(\mex{1}a)的语义来源于动词give(给)并且NP和动词组合在一起的结构没有贡献任何意义;或者短语模式[X Verb Y Z]贡献了某种“双及物意义”。\footnote{%
注意该结构的原型意义是领有权的转移,即Y从X那里获得Z。但是领有权也可以反向转移,例如(i.b):
%The question is whether the meafning of an utterance like (\mex{1}a) is contributed by the verb
%\emph{give} and the structure needed for the NPs occurring together with the verb does not contribute any meaning
%or whether there is a phrasal pattern [X Verb Y Z] that contributes some ``ditransitive meaning''
%whatever this may be.\footnote{%
%Note that the prototypical meaning is a transfer of possession in which Y receives Z from X, but the
%reverse holds in (i.b):
\eal
\ex 
\gll Er gibt ihr den Ball.\\
     他.\nom{} 给 她.\dat{} \textsc{det}.\acc{} 球\\
%     \gll Er gibt ihr den Ball.\\
%     he.\nom{} gives her.\dat{} the.\acc{} ball\\
\ex
\gll Er stiehlt ihr den Ball.\\
     他.\nom{} 偷  她.\dat{} \textsc{det}.\acc{} 球\\
\mytrans{他从她那儿偷了球。}
%\gll Er stiehlt ihr den Ball.\\
%     he.\nom{} steals  her.\dat{} the.\acc{} ball\\
%\mytrans{He steals the ball from her.}
\zllast
}
\eal
\ex 
\gll Peter gives Mary the book.\\
      Peter 给 Mary \textsc{det} 书\\
\mytrans{Peter给Mary这本书。}      
\ex 
\gll Peter fishes the pond empty.\\
Peter 钓鱼 \textsc{det} 池塘 空\\
\mytrans{Peter把池塘里的鱼都钓光了。}   
\zl
%\eal
%\ex Peter gives Mary the book.
%\ex Peter fishes the pond empty.
%\zl
另一个问题是:(\mex{0}b)中的成分是怎样获得允准的。这个例句非常值得我们注意,因为该句子有一种结果义,而这种结果义并不是动词fish(钓鱼)意义的一部分。整个句子的意思是:Peter钓鱼使得整个鱼塘里都没有鱼了。这种结果义也不是句子中其它成分的意义。如果采用基于词汇的方法,那么就需要假设一个词汇规则,该词汇规则允准一个词项选择Peter、the pond(池塘)和empty(空)。并且该词项贡献结果义。如果采用基于短语的方法,那么就需要假设一个短语模板[Subj V Obj Obl]。这一模板贡献结果义,而插入该模板的动词只是贡献其原型义,该意义与fish(钓鱼)在不及物结构中贡献的意义相同。我将这种基于短语的方法称作“插入方法”(plugging approaches),因为在这种方法中,词项插入到预先存在的结构中,而该结构承担大部分分析任务。
%Similarly, there is the question of how the constituents in (\mex{0}b) are licensed. This sentence is
%interesting since it has a resultative meaning that is not part of the meaning of the verb
%\emph{fish}: Peter's fishing causes the pond to become empty. Nor is this additional meaning part of
%the meaning of any other item in the sentence. On the lexical account,
%there is a lexical rule that licenses a lexical item that selects for \emph{Peter}, \emph{the pond},
%and \emph{empty}. This lexical item also contributes the resultative meaning. On the phrasal
%approach, it is assumed that there is a pattern [Subj V Obj Obl]. This pattern contributes the
%resultative meaning, while the verb that is inserted into this pattern just contributes its
%prototypical meaning, \eg the meaning that \emph{fish} would have in an
%intransitive construction. I call such phrasal approaches \emph{plugging approaches}, since lexical
%items are plugged into ready"=made structures that do most of the work.
    下面,我将更加仔细地检验这些方法,并且论证用基于词汇的方法来处理价是正确的。下面的讨论基于我更早的工作\citep{Mueller2006d,Mueller2007d,MuellerPersian}以及我与Steve Wechsler共同进行的工作\citep{MWArgSt,MWArgStReply}。 \citet{MWArgSt}中的一些章节最初翻译自 \citet{MuellerGTBuch2},但是经过与Steve Wechsler的大量讨论,我们将材料进行了重新组织和调整。所以,这里我没有使用 \citew{MuellerGTBuch2}的第11.11节而是使用了 \citew{MWArgSt}的一部分,并且增加了因空间限制而未收入该文的一些小章节(\ref{Abschnitt-Diskussion-Haugereid}和\ref{sec-neuro-linguistics})。因为曾经有过误解(例如, \citew{Boas2014a}, 参见  \citew{MWArgStReply}),所以这里需要声明一下:这一节并不是反对构式语法\indexcxgc(Construction Grammar)。正如我前文所述,基于符号的构式语法是构式的语法阵营中使用基于词汇的方法的一种理论,该理论与我所坚持的方法是兼容的。这一节也不是反对基于短语的方法,因为有些现象看起来用基于短语的方法来概括最为合适。这些现象在\ref{Abschnitt-Phrasale-Konstruktionen}中具体讨论。在下面的章节中,我反对的是基于短语方法中的一种特殊类型,即短语论元结构构式(phrasal argument structure constructions,phrasal ASCs)。我认为与价和价交替有关的现象都应该用基于词汇的方法来处理。
%In what follows I will examine these proposals in more detail and argue that the lexical approaches
%to valence are the correct ones. The discussion will be based on earlier work of mine
%\citep{Mueller2006d,Mueller2007d,MuellerPersian} and work that I did together with Steve Wechsler
%\citep{MWArgSt,MWArgStReply}. Some of the sections in  \citet{MWArgSt} started out as translations of
% \citet{MuellerGTBuch2}, but the material was reorganized and refocused due to intensive discussion
%with Steve Wechsler. So rather than using a translation of Section~11.11 of  \citew{MuellerGTBuch2},
%I use parts of  \citew{MWArgSt} here and add some subsections that had to be left out of the article
%due to space restrictions (Subsections~\ref{Abschnitt-Diskussion-Haugereid} and~\ref{sec-neuro-linguistics}).
%Because there have been misunderstandings in the past (\eg  \citew{Boas2014a}, see  \citew{MWArgStReply}), a disclaimer is necessary
%here: this section is not an argument against Construction Grammar\indexcxg. As was mentioned above
%Sign"=Based Construction Grammar is a lexical variant of Construction Grammar and hence compatible
%with what I believe to be correct. This section is also not against phrasal constructions in
%general, since there are phenomena that seem to be best captured with phrasal constructions. These are
%discussed in detail in Subsection~\ref{Abschnitt-Phrasale-Konstruktionen}. What I will argue against in
%the following subsections is a special kind of phrasal construction, namely phrasal argument
%structure constructions (phrasal ASCs). I believe that all phenomena that have to do with valence and valence
%alternations should be treated lexically.

\section{一些所谓的基于短语的模型所具有的优势}
%\section{Some putative advantages of phrasal models}
\label{Abschnitt-Stoepselei}
%
%The previous section reviewed earlier
%arguments for needing lexical representations of valence.  In this section we present more detailed
%arguments specifically directed against the claim that lexical valence representations
%(i.\,e.\ predicate argument structures) can or should be replaced by what we call a plugging proposal,
%that is, a system in which a verb or other predicator is plugged into a meaningful construction.

%As noted in Section~\ref{PAS-sec} above, we believe that grammars include meaningful phrasal
%constructions.  Our purpose is not to argue against their existence, but rather to argue that they
%cannot replace lexical valence representations.  So the existence of such phrasal constructions does not bear
%on the issue at stake here.  
    在这一节中,我们考察一些认为基于短语的构式语法比词汇规则更优的观点。在下面一节中,我将讨论支持词汇规则的论点。
%In this section we examine certain claims to purported advantages of phrasal versions of Construction Grammar over lexical rules.  
%Then in the following section, we will turn to positive arguments for lexical rules. 

\subsection{基于使用的理论}\label{usage-based-sec}
%\subsection{Usage-based theories}\label{usage-based-sec}
    对于构式语法的很多支持者来说,他们研究句法的方法都深深植根于语言是基于使用(usage-based)的理论\citep{Langacker87a-u, Goldberg95a,Croft2001a, Tomasello2003a}的本体论的局限中。支持基于使用理论的学者反对以下观点,“语言规则是将符号组合在一起的代数程序,其自身不贡献意义”\citep[\page 99]{Tomasello2003a}。所有语言学实体都是外延域中事物的符号化:“所有语言学实体都有交际功能,因为他们都直接来源于语言使用”(同上)。虽然语言的构形成分可能非常抽象,但是它们决不能与其作为一种交际工具这一最初的功能来源分离。构式的这一基于使用的观点在下面的引文中得到了很好地总结:
\begin{quotation}
最重要的一点是:构式就是使用模式,如果这些模式包含很多不同种类的具体语言符号,这些模式就会变得比较抽象。但是它们绝不是没有语义内容和交际功能的空规则。\citep[\page 100]{Tomasello2003a}\footnote{%
The most important point is that constructions are nothing more or less than patterns of usage,
which may therefore become relatively abstract if these patterns include many different kinds of
specific linguistic symbols.  But never are they empty rules devoid of semantic content or
communicative function.
}
\end{quotation}

%For many practitioners of Construction Grammar, their approach to syntax is deeply rooted in the
%ontological strictures of \emph{usage-based} theories of language \citep{Langacker87a-u, Goldberg95a,
%Croft2001a, Tomasello2003a}.  Usage-based theorists oppose the notion of ``linguistic rules conceived
%of as algebraic procedures for combining symbols that do not themselves contribute to meaning''
%\citep[\page 99]{Tomasello2003a}. All linguistic entities are symbolic of things in the realm of denotations;
%``all have communicative significance because they all derive directly from language use'' (\emph{ibid}). Although the formatives of language may be rather abstract, they can never be divorced
%from their functional origin as a tool of communication.  The usage-based view of constructions is
%summed up well in the following quote:
%\begin{quotation}
%The most important point is that constructions are nothing more or less than patterns of usage,
%which may therefore become relatively abstract if these patterns include many different kinds of
%specific linguistic symbols.  But never are they empty rules devoid of semantic content or
%communicative function. \citep[\page 100]{Tomasello2003a}
%\end{quotation}

\noindent 
因此构式与语法规则有两点不同:构式必须有意义,构式相当直接地反映了实际的“使用模式”。
%Thus constructions are said to differ from grammatical rules in two ways: they must carry meaning;
%and they reflect the actual ``patterns of usage'' fairly directly.
首先考虑第一个限制,语法的所有成分都必须有意义,我称这种观点为符号宣言(semiotic dictum)。基于词汇的理论和基于短语的理论两者之间的差异是否跟这一宣言联系最为紧密?范畴语法,一种词汇理论范式(参看第\ref{chap-CG}章),可以证明这一点。范畴语法包括有意义的词以及一些非常抽象的组合规则,例如X/Y*Y=X。根据规则-对-规则假设,组合规则指定结构整体的意义是组成部分意义的函项。虽然不清楚规则是否具有Tomasello所说的意义。
%Consider first the constraint that every element of the grammar must carry meaning, which we call
%the \emph{semiotic dictum}.  Do lexical or phrasal theories hew the most closely to this dictum?
%Categorial Grammar, the paradigm of a lexical theory (see Chapter~\ref{chap-CG}), is a strong
%contender: it consists of meaningful words, with only a few very general combinatorial rules such as
%X/Y $*$ Y = X.  Given the rule-to-rule assumption, those combinatorial rules specify the meaning of the
%whole as a function of the parts.  Whether such a rule counts as meaningful in itself in Tomasello's
%sense is not clear.
但是构式语法的组合规则,如Goldberg提出的将动词和构式组合在一起的对应原则(Correspondence Principle)\citeyearpar[\page 50]{Goldberg95a},与范畴语法中的组合规则有相同的地位。
%What does seem clear is that the combinatorial rules of Construction Grammar, such as Goldberg's
%Correspondence Principle for combining a verb with a construction \citeyearpar[\page 50]{Goldberg95a},
%have the same status as those combinatorial rules:

%\begin{quotation}
\ea
对应原则:在词汇上凸显和表达的任意一个参与者都必须与构式中一个凸显的论元融合。如果一个动词有三个凸显的参与者,那么其中一个就可能与构式中一个不凸显的论元角色融合。\citep[\page 50]{Goldberg95a}\footnote{%
The Correspondence Principle:  each participant that is lexically profiled and expressed must be
fused with a profiled argument role of the construction.  If a verb has three profiled participant
roles, then one of them may be fused with a non-profiled argument role of a construction. 
}\label{tcp}
\z
%The Correspondence Principle:  each participant that is lexically profiled and expressed must be
%fused with a profiled argument role of the construction.  If a verb has three profiled participant
%roles, then one of them may be fused with a non-profiled argument role of a construction. \citep[\page 50]{Goldberg95a}\label{tcp}
%\z
%\end{quotation}
动词和构式都指定了参与者角色,其中一些角色是凸显的(profiled)。对于动词来说,论元凸显是“由词决定的并且是高度规约化的”\citep[\page 46]{Goldberg95a}。构式中得到凸显的论元角色直接投射到语法成分,例如主语、直接宾语或间接宾语。按照对应原则,词汇上得到凸显的论元角色一定要直接投射为语法成分;除非一个动词有三个论元角色,在这种情况下,其中一个论元角色就不能直接投射为语法成分。\footnote{我们认为(\ref{tcp})中的第二句说的是第一句覆盖不了的例外情况。}就符号宣言而言,对应原则与范畴语法中的组合规则有相同的地位:即一种自身不携带意义却可以说明有意义成分的组合方式的代数规则。
%Both verbs and constructions are specified for participant roles, some of which are \emph{profiled}.
%Argument profiling for verbs is ``lexically determined and highly conventionalized''
%\citep[\page 46]{Goldberg95a}.  Profiled argument roles of a construction are mapped to direct
%grammatical functions, i.\,e., SUBJ, OBJ, or OBJ2.  By the Correspondence Principle the lexically profiled argument
%roles must be direct, unless there are three of them, in which case one may be indirect.\footnote{We
%  assume that the second sentence of (\ref{tcp}) provides for exceptions to the first sentence.}
%With respect to the semiotic dictum, the Correspondence Principle has the same status as the
%Categorial Grammar combinatorial rules: a meaningless algebraic rule that specifies the way to
%combine meaningful items.   
    下面开始论述我们赞成的词汇主义学者的句法研究方法,其中一些成分遵循符号格言,另外一些并不遵循。分析及物或不及物VP的短语结构规则(或者相应的HPSG ID程式)都不遵循符号格言。词汇价结构很明显携带意义,因为它们与特定的动词相联系。在英语双及物结构中,第一个宾语是第二个宾语所指物的“有意接受者”。所以,“他为她雕刻了一个玩具”(He carved her a toy)蕴含着他在雕刻玩具时希望她会接受。所以,向动词增加一个受益接受者论元的词汇规则会增加意义。与此相对的其他方法可能是:假设一个贡献“接受者”意义的短语性双及物构式。\footnote{%
在~\ref{coordination-sec},我们认为接受者应该借助一个词汇论元结构来增加,而不是通过一个短语构式。我们可以参考Wechsler (\citeyear[\page 111--113]{Wechsler91a-u};\citeyear[\page 88--89]{Wechsler95a-u})是如何用基于构式的和基于词汇的方法来分析英语双及物结构的。Wechsler的分析是基于Kiparsky提出的题元限制位置链接器(thematically restricted positional linker)(\citeyear{Kiparsky87a-u, Kiparsky88a-u})。我们认为哪种结构有意义不是一个理论层面上的问题。
}
%Turning now to the lexicalist syntax we favor, some elements abide by the semiotic dictum while
%others do not.  Phrase structure rules for intransitive and transitive VPs (or the respective HPSG ID schema)
%do not.  Lexical valence structures clearly carry meaning since they are associated with particular
%verbs.  In an English ditransitive, the first object expresses the role of ``intended recipient'' of
%the referent of the second object.
%Hence \emph{He carved her a toy} entails that he carved a toy with the intention that she receive
%it.  So the lexical rule that adds a benefactive recipient argument to a verb adds meaning.  Alternatively, a phrasal ditransitive construction might
%contribute that ``recipient'' meaning.\footnote{%
%In Section~\ref{coordination-sec} we argue that the
  %recipient should be added in the lexical argument structure, not through a phrasal construction.
  %See Wechsler (\citeyear[\page 111--113]{Wechsler91a-u}; \citeyear[\page 88--89]{Wechsler95a-u}) for an
  %analysis of English ditransitives with elements of both constructional and lexical approaches.  It
  %is based on Kiparsky's notion of a \emph{thematically
    %restricted positional linker} (\citeyear{Kiparsky87a-u, Kiparsky88a-u}).}  Which structures have
%meaning is an empirical question for us. 
然而,构式语法“所有构式都有意义”的看法是一个先验假设(a priori)。虽然双及物构式确实贡献意义,但是到现在为止还没有发现不及物构式或及物构式有什么真值—条件语义。很显然,构式语法学家证明“一些”像双及物构式这样的构式具有意义的证据并不能证明“所有”的短语构式都有意义。所以,就符号宣言而言,基于词汇和基于短语的方法似乎是相似的。
%In Construction Grammar, however, meaning is assumed for all constructions  \emph{a priori}.  But
%while the ditransitive construction plausibly contributes meaning, no truth-conditional meaning has
%yet been discovered for either the intransitive or bivalent transitive constructions.  Clearly the
%constructionist's evidence for the meaningfulness of \emph{certain} constructions such as the
%ditransitive does not constitute evidence that \emph{all} phrasal constructions have meaning.  So
%the lexical and phrasal approaches seem to come out the same, as far as the semiotic dictum is
%concerned.
    我们现在考虑一下基于使用理论的第二个宣言,即“语法的成分直接反映使用模式”。我们将这一宣言称作透明性宣言(the transparency dictum)。构式语法文献经常用一种非形式化的方法来呈现构式,这种呈现方式看似表征了表面成分的顺序模式:及物构式的形式是“[X VERB Y]”(Tomasello)或者“[Subj V Obj]”\citep{Goldberg95a,Goldberg2006a}\footnote{%
   \citet[\page 300]{GCS2004a}报告了一个涉及SOV模式的语言习得实现。该报告明确提到SOV语序,并将其作为构式的一部分。
};被动构式的形式是“X was VERB-ed by Y”\citep[\page 100]{Tomasello2003a}或者“Subj aux Vpp (PPby)”\citep[\page 5]{Goldberg2006a}。但是Müller (\citeyear[\S~2]{Mueller2006d})仔细考察并反对包含表面模式构式的理论。因为这种理论没有真正地反映Goldberg的实际的理论。\footnote{%
这仅仅适用于论元结构构式。在Goldberg的一些论文中,她认为非常具体的短语结构也属于构式。例如,在她论述波斯语中复杂谓词的论文\citep{Goldberg2003a}中,她就指派了\vnullc 和\vbarc 范畴。对这一分析的批评可以参看 \citew[\S~4.9]{MuellerPersian}
}如果更加细致地表征论元结构构式(argument structure constructions),就会发现论元结构构式更加抽象并且非常像词汇主义学者提出的语法成分(或者可能是LFG中的f-结构):及物构式像一条及物价结构(不包括动词本身);被动构式像被动词汇规则。
%Now consider the second usage-based dictum, that the elements of the grammar directly reflect
%patterns of usage, which we call \emph{the transparency dictum}.  The Construction Grammar
%literature often presents their constructions informally in ways that suggest that they represent
%surface constituent order patterns: the transitive construction is ``[X VERB Y]'' (Tomasello) or ``[Subj V Obj]''
%\citep{Goldberg95a,Goldberg2006a}\footnote{%
 %  \citet[\page 300]{GCS2004a} report about a language acquisition experiment that involves an SOV
 % pattern. The SOV order is mentioned explicitly and seen as part of the construction.
%}; the passive construction is ``X \emph{was} VERB\emph{ed by} Y''
%\citep[\page 100]{Tomasello2003a} or ``Subj aux Vpp (PPby)'' \citep[\page 5]{Goldberg2006a}.  But a theory
%in which constructions consist of surface patterns was considered in detail and rejected by
%Müller (\citeyear[Section~2]{Mueller2006d}), and does not accurately reflect Goldberg's actual
%theory.\footnote{%
 % This applies to argument structure constructions only. In some of her papers Goldberg assumes that
  %very specific phrase structural configurations are part of the constructions. For instance in her
  %paper on complex predicates in Persian \citep{Goldberg2003a} she assigns \vnull and \vbar categories. See
  % \citew[Section~4.9]{MuellerPersian} for a critique of that analysis.} 
%The more detailed discussions present \emph{argument structure
 % constructions}, which are more abstract and rather like the lexicalists' grammatical elements (or
%perhaps an LFG f-structure): the transitive construction resembles a transitive valence structure
%(minus the verb itself); the passive construction resembles the passive lexical rule.
    就满足基于使用理论的要求而言,我们没有发现非词汇方法与词汇方法之间的重大差异。
%With respect to fulfilling the desiderata of usage-based theorists, we do not find 
%any significant difference between the
%non-lexical and lexical approaches.  

\subsection{压制}
%\subsection{Coercion}
\label{coercion-sec}
    采用插入方法的学者经常将压制作为短语构式有用的一个证据。例如,Anatol Stefanowitsch在\emph{Algorithmen und Muster –-
  Strukturen in der Sprache}演讲集(2009)中的一篇演讲讨论了(\mex{1})中的例子:\ea
Das Tor zur Welt Hrnglb öffnete sich ohne Vorwarnung
und verschlang [sie] \ldots{} die Welt Hrnglb wird von Magiern
erschaffen, die Träume zu Realität formen können, aber
nicht in der Lage sind zu träumen. Haltet aus, Freunde.
Und ihr da draußen, bitte träumt ihnen ein Tor.\footnote{%
%\href{http://www.elbenwaldforum.de/showflat.php?Cat=&Board=Tolkiens_Werke&Number=1457418&page=3&view=collapsed&sb=5&o=&fpart=16}{\nolinkurl{http://www.elbenwaldforum.de/showflat.php?Cat=&Board=Tolkiens_Werke&}}
%\href{http://www.elbenwaldforum.de/showflat.php?Cat=&Board=Tolkiens_Werke&Number=1457418&page=3&view=collapsed&sb=5&o=&fpart=16}{\nolinkurl{Number=1457418&page=3&view=collapsed&sb=5&o=&fpart=16}}. 27.02.2010.
\url{http://www.elbenwaldforum.de/showflat.php?Cat=&Board=Tolkiens_Werke&Number=1457418&page=3&view=collapsed&sb=5&o=&fpart=16},\zhdate{2010/02/27}。
“Hrnglb世界的大门在没有警告的情况下打开了并且将他们吞入。Hrnglb世界是巫师创造的,这些巫师能将梦想变为现实,但是他们自己却不能做梦。等一下,朋友们!你们正好在那里,请为他们梦一个门吧。”
}
\z
%Researchers working with plugging proposals usually take coercion as an indication of the usefulness of phrasal
%constructions. For instance, Anatol Stefanowitsch (Lecture in the lecture series \emph{Algorithmen und Muster –-
%  Strukturen in der Sprache}, 2009) discussed the example in (\mex{1}):
%\ea
%Das Tor zur Welt Hrnglb öffnete sich ohne Vorwarnung
%und verschlang [sie] \ldots{} die Welt Hrnglb wird von Magiern
%erschaffen, die Träume zu Realität formen können, aber
%nicht in der Lage sind zu träumen. Haltet aus, Freunde.
%Und ihr da draußen, bitte träumt ihnen ein Tor.\footnote{%
%\href{http://www.elbenwaldforum.de/showflat.php?Cat=&Board=Tolkiens_Werke&Number=1457418&page=3&view=collapsed&sb=5&o=&fpart=16}{\nolinkurl{http://www.elbenwaldforum.de/showflat.php?Cat=&Board=Tolkiens_Werke&}}
%\href{http://www.elbenwaldforum.de/showflat.php?Cat=&Board=Tolkiens_Werke&Number=1457418&page=3&view=collapsed&sb=5&o=&fpart=16}{\nolinkurl{Number=1457418&page=3&view=collapsed&sb=5&o=&fpart=16}}. 27.02.2010.
%\url{http://www.elbenwaldforum.de/showflat.php?Cat=&Board=Tolkiens_Werke&Number=1457418&page=3&view=collapsed&sb=5&o=&fpart=16}, 27.02.2010.

%`The gate to the world Hrnglb opened without warning and swallowed them. The world Hrnglb is created
%by magicians that can form reality from dreams but cannot dream themselves. Hold out, friends! And
%you out there, please, dream a gate for them.'
%}
%\z
在句子中重要的部分是bitte träumt ihnen ein Tor(为他们梦一个门)。动词träumen(梦)本来是一个不及物动词,在这种艺术化语境中,被强制纳入到一个及物构式中,因而具有了及物动词的意义。构式对动词的这种强制对应着重写或者扩展该动词的属性。
%The crucial part is \emph{bitte träumt ihnen ein Tor} `Dream a gate for them'. In this fantasy
%context the word \emph{träumen}, which is intransitive, is forced into the ditransitive construction
%and therefore gets a certain meaning. This forcing of a verb corresponds to overwriting or rather extending properties
%of the verb by the phrasal construction.

%But it is possible to find other explanations for such cases. Instead of forcing an object into a
%hole in which it does not fit one could adapt the object and insert it then. The second proposal can
%be modeled by lexical rules, that is, 
    在那些插入方法认为动词信息得到重写或扩展的案例中,基于词汇的方法假设调节规则。 \citet[\S~4]{BC99a}已经详细提出一种词汇方法。\footnote{\citet{Kay2005a}在构式语法框架下展开研究,同样也提出单分支结构。}他们讨论了(\mex{1})中的例句,这些例句或者对应于典型的双及物构式(\mex{1}a)或者以多种方式从典型双及物构式衍生而来。
\eal
\ex 
\gll Mary gave Joe a present.\\
Mary 给 Joe 一 礼物\\
\mytrans{Mary给Joe了一个礼物。}  
%\ex Mary gave Joe a present.
\ex\label{paint} 
\gll Joe painted Sally a picture.\\
Joe 画画 Sally 一 画\\
\mytrans{Joe给Sally画了一幅画。}  
%\ex\label{paint} Joe painted Sally a picture.
\ex 
\gll Mary promised Joe a new car.\\
Mary 承诺 Joe 一 新 车\\
\mytrans{Mary承诺给Joe买一辆新车。}  
%\ex Mary promised Joe a new car.
\ex 
\gll He tipped Bill two pounds.\\
他 给小费 Bill 两 英镑\\
\mytrans{他给了Bill两英镑小费。}  
%\ex He tipped Bill two pounds.
\ex 
\gll The medicine brought him relief.\\
\textsc{det} 药 带来 他 缓解\\
\mytrans{这药缓解了他的痛苦。}  
%\ex The medicine brought him relief.
\ex 
\gll The music lent the party a festive air.\\
\textsc{det} 音乐 借 \textsc{det} 聚会 一 节日 气氛\\
\mytrans{音乐给聚会带来了节日气氛。}  
%\ex The music lent the party a festive air.
\ex 
\gll Jo gave Bob a punch.\\
Jo 给 Bob 一 拳\\
\mytrans{Jo 给了Bob一拳。}  
%\ex Jo gave Bob a punch.
\ex 
\gll He blew his wife a kiss.\\
他 吹 他的 妻子 一 吻\\
\mytrans{他给了他妻子一个飞吻。}  
%\ex He blew his wife a kiss.
\ex\label{ex-smiled-herself-an-upgrade} 
\gll She smiled herself an upgrade.\\
她 笑  \textsc{refl} 一 升舱\\
\mytrans{她的微笑给自己赢得了一次升舱的机会。}  
%\ex\label{ex-smiled-herself-an-upgrade} She smiled herself an upgrade.
\zl
对于这些非典型的例句,他们假设了词汇规则将及物动词paint(画)、不及物动词smile(笑)与双及物构式联系起来并且提供相应的语义信息或者相应的隐喻扩展。(\ref{ex-smiled-herself-an-upgrade})中的例子与上文讨论的träumen(做梦)相当相似,也用词汇规则分析 (第509页)。Briscoe和Copestake注意到这一词汇规则比他们提出的其它词汇规则在能产性上更加受限。因此,他们提出了一种新的表征方式,在这种表征方式中词项(包括通过词汇规则衍生出来的词项)与概率相联,所以不同模式能产性上的差异就可以表示了。
%In cases in which the plugging proposals assume that
%information is overwritten or extended, lexical approaches assume mediating lexical
%rules.  \citet[Section~4]{BC99a} have worked out a lexical approach in detail.\footnote{%
% \citet{Kay2005a}, working in the framework of CxG, also suggests unary constructions.}
%They discuss the ditransitive sentences in (\mex{1}), which either correspond to the prototypical
%ditransitive construction (\mex{1}a) or deviate from it in various ways.
%\eal
%\ex Mary gave Joe a present.
%\ex\label{paint} Joe painted Sally a picture.
%\ex Mary promised Joe a new car.
%\ex He tipped Bill two pounds.
%\ex The medicine brought him relief.
%\ex The music lent the party a festive air.
%\ex Jo gave Bob a punch.
%\ex He blew his wife a kiss.
%\ex\label{ex-smiled-herself-an-upgrade} She smiled herself an upgrade.
%\zl
%For the non-canonical examples they assume lexical rules that relate transitive (\emph{paint}) and intransitive (\emph{smile}) 
%verbs to ditransitive ones and contribute the respective semantic information or the respective
%metaphorical extension. The example in (\ref{ex-smiled-herself-an-upgrade}) is rather similar to the
%\emph{träumen} example discussed above and is also analyzed with a lexical rule (page~509). Briscoe
%and Copestake note that this lexical rule is much more restricted in its productivity than the other lexical
%rules they suggest. They take this as motivation for developing a representational
%format in which lexical items (including those that are derived by lexical rules) are
%associated with probabilities, so that differences in productivity of various patterns can be captured.
    仅仅考虑这些案例的话,我们很难找到坚实的理由从基于短语的或基于词汇的分析中作出选择。但是如果我们扩大我们的考察范围,就会发现词汇规则方法有更加广泛的应用。压制是一个非常宽泛的语用过程,可以出现在很多构式无法起作用的环境中\citep{Nunberg95a-u}。Nunberg引用了很多案例,例如餐馆服务员的问句“Who is the ham sandwich?”(谁点了火腿三明治?)\citep[\page 115]{Nunberg95a-u}。 \citet[\page 116]{CB92a}讨论了动物名称转换成物质名词的现象(也可以参看 \citet[\page 36--43]{CB95a-u})。(\mex{1})是关于一种物质而不是关于一只可爱的兔子。
\ea
\gll After several lorries had run over the body, there was rabbit splattered all over the road.\\
     在之后 几辆 货车 \textsc{aux} 行驶 \textsc{prep} \textsc{det} 身体 \expl{} \textsc{cop} 兔子 散布 全部 \textsc{prep} \textsc{det} 路\\
\mytrans{在几辆货车驶过之后,路上散布着兔子肉。}
%After several lorries had run over the body, there was rabbit splattered all over the road.
\z
作者假设了一条词汇规则,该词汇规则将可数名词投射为一个物质名词。 \citet[\page 114--115]{Fillmore99a}提出了相同的分析。这种压制可以独立于任何句法环境而存在:可以用一个词Rabbit(兔子)来回答问题“What's that stuff on the road?”(路上是什么东西?)和问题“What are you eating?”(你在吃什么东西?)。一些压制恰巧影响了动词的补语结构,但是这仅仅是一个更加概括现象的一个特例,这种概括的现象可以通过系统多义规则来分析。
%Looking narrowly at such cases, it is hard to see any rational grounds for choosing between the phrasal analysis and the lexical rule.  But if we broaden our view, the lexical rule approach can be seen to have a much wider application. 
%Coercion is a very general pragmatic process, occurring in many contexts where no construction seems
%to be responsible  \citep{Nunberg95a-u}.  Nunberg cites many cases such as the restaurant waiter
%asking \emph{Who is the ham sandwich?} \citep[\page 115]{Nunberg95a-u}.  
% \citet[\page 116]{CB92a} discuss the conversion of terms for animals to mass nouns (see also  \citet[\page 36--43]{CB95a-u}). Example (\mex{1}) is about a substance, not about a cute bunny.
%\ea
%After several lorries had run over the body, there was rabbit splattered all over the road.
%\z
%The authors suggest a lexical rule that maps a count noun onto a mass noun. This analysis is also
%assumed by  \citet[\page 114--115]{Fillmore99a}.
%Such coercion can occur without any syntactic context: one can answer the question \emph{What's that
%  stuff on the road?} or \emph{What are you eating?} with the one-word utterance \emph{Rabbit.}
%Some coercion happens to affect the complement structure of a verb, but this is simply a special
%case of a more general phenomenon that has been analyzed by rules of systematic polysemy.      

\subsection{体是一种句子层面的现象}
%\subsection{Aspect as a clause level phenomenon}
\label{sec-aspect-at-clause-level}

     \citet{Alsina96a}在词汇功能语法\indexlfgc 框架内工作,赞成一种基于句子体属性的短语方法来分析结果构式,因为体\isc{体}\is{aspect}通常被当做由句子句法决定的属性。像bark(狗叫)这样的不及物动词表示一种活动\isc{活动}\is{activity},但是包含这一动词的结果构式却代表一种完结\isc{完结}\is{accomplishment}(一种扩展的状态变化\isc{状态变化}\is{change of state})。Alsina用下面的例子来支持这一观点:
\eal
\judgewidth{(*)}
\ex[(*)]{
\gll The dog barked in five minutes.\\
\textsc{det} 狗 叫 在 五 分钟\\
\mytrans{狗叫了五分钟。}  
%The dog barked in five minutes.
}
\ex[]{
\gll The dog barked the neighbors awake in five minutes.\\
\textsc{det} 狗 叫 \textsc{det} 邻居 醒来 在 五 分钟\\
\mytrans{狗在五分钟之后将邻居叫醒了。}  
%The dog barked the neighbors awake in five minutes.
}
\zl
后一句的意思是barking(狗叫)这一事件在五分钟之后完成。(\mex{0}a)没有“狗叫”这一事件延续五分钟这一意义。如果(\mex{0}a)是合乎语法的,那么该句的意思是说明了事件开始的时间框架。
% \citet{Alsina96a}, working in the framework of LFG\indexlfg, argues for a phrasal analysis of resultative constructions based on the aspectual properties
%of sentences, since aspect\is{aspect} is normally viewed as a property that is determined by the sentence syntax. Intransitive verbs such as \emph{bark}
%refer to activities\is{activity}, a resultative construction with the same verb, however, stands for an accomplishment\is{accomplishment} (an extended
%change of state\is{change of state}).
%Alsina supports this with the following data:
%\eal
%\judgewidth{(*)}
%\ex[(*)]{
%The dog barked in five minutes.
%}
%\ex[]{
%The dog barked the neighbors awake in five minutes.
%}
%\zl
%The latter sentence means that the \emph{barking} event was completed after five minutes. A reading referring to the time span of the event
%is not available for (\mex{0}a). If (\mex{0}a) is grammatical at all, then a claim is being made about the time frame in which the event begun.
如果考虑(\mex{1}c)中的例子,会发现Alsina的论证就不再有说服力了,因为结果义已经出现在名词化中,即结果义存在于词层面上。正如(\mex{1})所示,这一对立可以出现在名词性结构中,因而是独立于句子句法存在的:
\eal
\judgewidth{\#}
\ex[]{
\gll weil sie die Nordsee in fünf Jahren leer fischten\\
	 因为 他们 \textsc{det} 北海 在 五 年 空 钓鱼\\
\mytrans{因为他们五年内就把北海里的鱼全部钓光了}
}
\ex[\#]{
\gll weil sie in fünf Jahren fischten\\
     因为 他们 在 五 年 钓鱼\\     
}
\ex[]{
\gll das Leerfischen der Nordsee in fünf Jahren\\
     \textsc{det} 空.钓鱼 \textsc{det} 北海 在 五 年\\
}
\ex[\#]{
\gll das Fischen in fünf Jahren\\
     \textsc{det} 钓鱼 在 五 年\\
}
\zl
%If we now consider examples such as (\mex{1}c), however, we see that Alsina's argumentation is not cogent since the resultative
%meaning is already  present at the word"=level in nominalizations. As the examples in (\mex{1}) show, this contrast can be observed in nominal constructions
%and is therefore independent of the sentence syntax:
%\eal
%\judgewidth{\#}
%\ex[]{
%\gll weil sie die Nordsee in fünf Jahren leer fischten\\
%	 because they the North.Sea in five years empty fished\\
%\mytrans{because they fished the North Sea (until it was) empty in five years}
%}
%\ex[\#]{
%\gll weil sie in fünf Jahren fischten\\
%     because they in five years fished\\
%}
%\ex[]{
%\gll das Leerfischen der Nordsee in fünf Jahren\\
%     the empty.fishing of.the North.Sea in five years\\
%}
%\ex[\#]{
%\gll das Fischen in fünf Jahren\\
%     the fishing in five years\\
%}
%\zl
%
    在一个基于词汇的方法中,会有一个词干选择两个NP和一个结果谓词。这一词干有合适的意义并且可以屈折或者经历派生和连续的取值。在这两个例子中,我们都可以获得包含结果义的动词,因此可以与相应副词兼容。
%In a lexical approach there is a verb stem selecting for two NPs and a resultative predicate. This
%stem has the appropriate meaning and can be inflected or undergo derivation und successive
%inflection. In both cases we get words that contain the resultative semantics and hence are
%compatible with respective adverbials. 

\subsection{简洁性和多义}
%\subsection{Simplicity and polysemy}
\label{polysemy-subsec}

    插入方法的很多直觉上的优势来源于这种方法相对于词汇规则方法有明显的简洁性。但是,认为构式语法更简洁的观点出自对词汇规则和构式语法(尤其是Goldberg\citeyearpar{Goldberg95a,Goldberg2006a}的构式语法)的误解。这一观点在错误的地方寻找差异而且忽略了两种方法真正的差异。构式语法更加简洁这一观点经常被提到,所以很有必要理解这一观点为什么是错误的。
%Much of the intuitive appeal of the plugging approach stems from its apparent simplicity relative to
%the use of lexical rules.  But the claim to greater simplicity for Construction Grammar is based on
%misunderstandings of both lexical rules and Construction Grammar (specifically of Goldberg's \citeyearpar{Goldberg95a,Goldberg2006a} version).   It draws the distinction in the wrong place and misses the real differences
%between these approaches.  This argument from simplicity is often repeated and so it is important to
%understand why it is incorrect.    
%\NOTE{Is this too combative?  maybe need to tone it down}
     \citet{Tomasello2003a}论述如下。在讨论词汇规则方法时, \citet[\page 160]{Tomasello2003a}写道:   
% \citet{Tomasello2003a} presents the argument as follows.  Discussing first the lexical rules approach,  \citet[\page 160]{Tomasello2003a} writes that 

\begin{quotation}
这一观点的隐含义是动词必须在词库中列出该动词在其所有可能出现的构式中的意义[\ldots]。例如,虽然cough(咳嗽)的原型意义只涉及一个参与者,即“咳嗽的人”;但是我们可以说He coughed her his cold(他咳嗽将感冒传染给她了),在该句中有三个核心参与者。如果采用词汇规则方法,为了产生这一表达,儿童的词库中必须有一个双及物动词词项cough(咳嗽)。\citep[\page 160]{Tomasello2003a}\footnote{%
One implication of this view is that a verb must have listed in the lexicon a different meaning for
virtually every different construction in which it participates [\ldots].  For example, while the
prototypical meaning of \emph{cough} involves only one participant, the cougher, we may say such
things as \emph{He coughed her his cold}, in which there are three core participants.  In the
lexical rules approach, in order to produce this utterance the child's lexicon must have as an entry
a ditransitive meaning for the verb \emph{cough}.
}
\end{quotation}
 \citet[\page 160]{Tomasello2003a}又引用了 \citet{FKoC88a}、 \citet{Goldberg95a}和 \citet{Croft2001a}来比较构式语法方法。他得出如下结论:
% \citet[\page 160]{Tomasello2003a} then contrasts a Construction Grammar approach, citing  \citet{FKoC88a},  \citet{Goldberg95a}, and  \citet{Croft2001a}.  He concludes as follows:

\begin{quotation}
主要观点是如果我们认为构式具有其自身的意义,并且这种意义是独立于出现于构式之中的动词的,那么我们就不需要为我们在日常生活中使用的动词在词库中列举不合理的意义。假设构式具有意义的构式语法因此比词汇规则方法更加简单并且更加合理。\citep[\page 161]{Tomasello2003a}\footnote{%
The main point is that if we grant that constructions may have meaning of their own, in relative
independence of the lexical items involved, then we do not need to populate the lexicon with all
kinds of implausible meanings for each of the verbs we use in everyday life.  The construction
grammar approach in which constructions have meanings is therefore both much simpler and much more plausible than the lexical rules approach.}
\end{quotation}

\noindent
这反映了对词汇规则的一种误解,正如它们通常被理解的那样。词库中不会有大量不合理的意义。解释He coughed her his cold(他咳嗽将感冒传染给她了)的词汇规则表示,当动词cough(咳嗽)与两个宾语组合时,整个复杂体有一定的意义(参看\citealp[\page 876]{Mueller2006d})。另外,我们区分了列举的成分(词条)和推导的成分。两者统称为词项(lexical item)。
%This reflects a misunderstanding of lexical rules, as they are normally understood.  There is no implausible sense populating the lexicon.
%The lexical rule approach to \emph{He coughed her his cold} states that when the word \emph{coughed} appears with
%two objects, the whole complex has a certain meaning (see \citealp[\page 876]{Mueller2006d}). Furthermore we explicitly distinguish between listed elements
%(lexical entries) and derived ones. The general term subsuming both is \emph{lexical item}.

%Adopting lexical rules does not mean that we `populate the
%lexicon with all kinds of implausible meanings'; quite the contrary.  
%that could not otherwise be expressed by means of forms listed in the
%lexicon.  Tomasello seems to be confusing simplicity of the grammar with simplicity of the language
%licensed by the grammar.\NOTE{St.Mü.: I do not understand this.}  (By the same flawed reasoning one could complain that every grammar is
%populated with infinitely many sentences with all kinds of implausible meanings.)   
    简洁性这一论述也是基于对于Tomasello所支持理论(即 \citet{Goldberg95a, Goldberg2006a}所提出理论)的误解。为了让他的理论成立,Tomasello必须隐含地假设动词可以与动词自由组合,即语法对于这种组合不规定任何外部限制。如果需要说明哪些动词可以出现在哪些构式中,那么更加简洁这一论述就不成立了:词汇规则方法中带有“不合理意义”的每一个词项变体都对应短语方法下一个动词—加—构式组合。
%The simplicity argument also relies on a misunderstanding of a theory Tomasello advocates, namely the
%theory due to  \citet{Goldberg95a, Goldberg2006a}.  For his argument to go through, Tomasello must tacitly assume
%that verbs can combine freely with constructions, that is, that the grammar does not place extrinsic
%constraints on such combinations.  If it is necessary to also stipulate which verbs can appear in
%which constructions, then the claim to greater simplicity collapses: each variant lexical item with
%its ``implausible meaning'' under the lexical rule approach corresponds to a verb-plus-construction
%combination under the phrasal approach. 
    下面的论述看似可以说明动词与构式可以自由组合:\footnote{这些引文的语境很清楚,动词和论元结构都被当做构式。参看 \citet[\page 21, ex.~(2)]{Goldberg2006a}。} 
%Passages such as the following may suggest that verbs and constructions are assumed to combine
%freely:\footnote{The context of these quotes makes clear that the verb and the argument structure construction are considered 
%constructions.  See  \citet[\page 21, ex.~(2)]{Goldberg2006a}.} 

%\begin{quotation}
%Constructions are combined freely to form actual expressions as long
%as they are not in conflict.  Unresolved conflicts result in judgments
%of ill-formedness.  (Goldberg 2006, p.\,10)
%\end{quotation}

\begin{quotation}
构式可以自由组合并组成真实表达,只要它们在组配(construal)时不会出现冲突(引入组配这一概念是为了允许调整和强制过程)
[\ldots]
允许构式自由组织只要它们不相互冲突,使语言具有无限生成潜力。[\ldots]也就是说,只要构式之间不存在冲突,说话者就可以自由地创造性地组合这些构式来合适地表达想要表达的信息。\citep[\page 22]{Goldberg2006a}\footnote{%
Constructions are combined freely to form actual expressions as long
as they can be construed as not being in conflict (invoking the notion
of construal is intended to allow for processes of accommodation or
coercion). 
[\ldots] 
Allowing constructions to combine freely as long as there are no
conflicts, allows for the infinitely creative potential of language.
[\ldots] That is, a speaker is free to creatively combine constructions as
long as constructions exist in the language that can be combined
suitably to categorize the target message, given that there is no
conflict among the constructions.
} 
\end{quotation}

\noindent
    但是实际上,Goldberg并不假设自由组合,而是认为动词“与构式的连接是约定俗成的” \citep[\page 50]{Goldberg95a}:动词说明了它们的参与者角色以及哪些参与者角色是强制直接论元(按照Goldberg的术语是凸显的(profiled))。其实,Goldberg自己\citeyearpar[\page 211]{Goldberg2006a}是反对Borer的假定存在的自由组合的\citeyearpar{Borer2003a-u}。理由是Borer无法解释dine(吃饭)(不及物动词)、eat(吃)(非强制及物动词)、devour(吃掉)(强制性及物动词)之间的差异。\footnote{Goldberg批评的是Borer2001年的一篇报告,该报告与 \citew{Borer2003a-u}同名。参看~\ref{sec-idiosyncratic-case-and-PP}以了解更多对该问题的讨论。就我们所知,dine / eat / devour(吃饭/吃/吃掉)这一最小对比对最初来源于 \citet[\page 89--90]{Dowty89b-u}。}Tomasello上文的论述并不能证明构式语法比词汇规则语法更加简洁。
%But in fact Goldberg does not assume free combination, but rather that a verb is ``conventionally
%associated with a construction'' \citep[\page 50]{Goldberg95a}: verbs specify their participant roles and which
%of those are obligatory direct arguments (\emph{profiled}, in Goldberg's terminology).  In fact, Goldberg herself \citeyearpar[\page 211]{Goldberg2006a}
%argues against Borer's putative assumption of free combination \citeyearpar{Borer2003a-u} on the grounds that Borer is
%unable to account for the difference between \emph{dine} (intransitive), \emph{eat} (optionally
%transitive), and \emph{devour} (obligatorily transitive).\footnote{Goldberg's critique cites a 2001
%  presentation by Borer with the same title as  \citew{Borer2003a-u}.  See
%  Section~\ref{sec-idiosyncratic-case-and-PP} for more discussion of this issue.  As far as
%we know, the \emph{dine / eat / devour} minimal triplet originally came from  \citet[\page 89--90]{Dowty89b-u}. }
%Despite Tomasello's comment above,
%Construction Grammar is no simpler than the lexical rules.   
    结果构式经常用于解释简洁性这一观点。例如, \citet[\S~7]{Goldberg95a}认为(\mex{1}a)和(\mex{1}b)中的动词sneeze(打喷嚏)是相同的,只是插入了不同的构式当中:
\eal
\ex 
\gll He sneezed.\\
他 打喷嚏\\
\mytrans{他打喷嚏。}  
%He sneezed.
\ex 
\gll He sneezed the napkin off the table.\\
他 打喷嚏 \textsc{det} 餐巾 \textsc{prep} \textsc{det} 桌子 \\
\mytrans{他打喷嚏把餐巾从桌子上吹下去了。}  
%He sneezed the napkin off the table.
\zl
(\mex{0}a)的意义大致对应于动词的意义,因为动词出现在不及物构式中。但是(\mex{0}b)中的致使"=移动构式\isc{构式!致使-移动}\is{construction!Caused"=Motion}提供了致使和移动的语义信息:他打喷嚏导致纸巾从桌子上掉下来。sneeze(打喷嚏)插入到致使"=移动构式中,该构式允准动词sneeze(打喷嚏)的主语并另外提供了两个槽:一个是主题napkin(纸巾),一个是目标off the table(从桌子上掉下来)。词汇方法与此方法基本相同,不同之处仅在于词汇规则可以进行另外的词汇过程,例如被动化The napkin was sneezed off the table(他打喷嚏使得纸巾被吹下桌子)和向名词或形容词转变(参看\ref{sec-val-morph}和\ref{sec-acquisition})。   
%The resultative construction is often used to illustrate the simplicity argument.  For example,  
% \citet[Chapter~7]{Goldberg95a} assumes that the same lexical item for the verb \emph{sneeze}
%is used in (\mex{1}a) and (\mex{1}b). It is simply inserted into different constructions:
%\eal
%\ex He sneezed.
%\ex He sneezed the napkin off the table.
%\zl
%The meaning of (\mex{0}a) corresponds more or less to the verb meaning, since the verb is used in
%the Intransitive Construction. But the Caused"=Motion Construction\is{construction!Caused"=Motion} in (\mex{0}b) contributes
%additional semantic information concerning the causation and movement: his sneezing caused the
%napkin to move off the table.  \emph{sneeze} is plugged into the Caused"=Motion Construction, which
%licenses the subject of \emph{sneeze} and additionally provides two slots: one for the theme
%(\emph{napkin}) and one for the goal (\emph{off the table}).  The lexical approach is essentially parallel,
%except that the lexical rule can feed further lexical processes like passivization (\emph{The napkin
%  was sneezed off the table}), and conversion to nouns or adjectives (see Sections
%\ref{sec-val-morph} and \ref{sec-acquisition}).   

    在另外一个有细微差异的比较中, \citet[\page 139--140]{Goldberg95a}再次考虑了Mary kicked Joe the ball (Mary将球踢给了Joe)中增加的接受者论元,在该句中kick(踢)是一个二价动词。她指出按照构式的观点“涉及动词和构式的合成结构储存在记忆中”。动词自身仍然保留其作为二价动词原来的意义,所以“我们不用为动词假设‘通过踢导致收到’这个不合理的意义”的观点似乎是说:相反地,基于词汇的方法必须承认这种不合理的动词意义,因为一个动词规则增加了第三个论元。
%In a nuanced comparison of the two approaches,  \citet[\page 139--140]{Goldberg95a}
%considers again the added recipient argument in \emph{Mary kicked Joe the ball}, where \emph{kick}
%is lexically a 2-place verb.  She notes that on the constructional view, ``the composite fused
%structure involving both verb and construction is stored in memory''.  
%The verb itself retains its original meaning as a 2-place verb, so that ``we
%avoid implausible verb senses such as `to cause to receive by kicking'.''  The idea seems to be that
%the lexical approach, in contrast, must countenance such implausible verb senses since a lexical
%rule adds a third argument.  
    但是基于词汇和基于短语的方法在这一点上实际上是没有差别的。词汇规则并不产生一个有(\mex{1}a)中那种“不合理意义”的动词。相反,词汇规则产生(\mex{1}b)中所示的意义:
\eal
\ex cause-to-receive-by-kicking(x, y, z) 
\ex cause(kick(x, y),receive(z,y))
\zl
两种方法都假设“合成结构”。就语义结构而言,意义的数量和合理性以及语义关系的多元性是相同的。两者的主要差异在于如何将这种表征与更大的句法理论相融合。这两种方法还有另外一个差异,即按照基于词汇的观点,派生的三价结构与语音串kicked(踢)联系。下面我们将证明这一点。
%But the lexical and constructional approaches are actually indistinguishable on this point.  The lexical rule does not
%produce a verb with the ``implausible sense'' in (\mex{1}a).  Instead it produces the sense in (\mex{1}b):
%\eal
%\ex cause-to-receive-by-kicking(x, y, z) 
%\ex cause(kick(x, y),receive(z,y))
%\zl
%The same sort of ``composite fused structure'' is assumed under either view.  
%With respect to the semantic structure, the number and plausibility of senses, and the polyadicity of the semantic relations, the two
%theories are identical.  They mainly differ in the way this representation fits into the larger theory of syntax.  
%They also differ in another respect: on the lexical view, the derived three"=argument valence
%structure is associated with the phonological string
%\emph{kicked}.  Next, we present evidence for this claim.
 
\section{基于词汇的方法的证据}
%\section{Evidence for lexical approaches}
\subsection{价和并列}
%\subsection{Valence and coordination}
\label{coordination-sec}
    按照基于词汇的方法,(\ref{paint})中的动词paint(粉刷)是一个二价动词,并且在单分支规则中它的直接上位节点是一个三价动词。从构式的视角来看,并不存在一个需要三个论元且只统制一个动词的谓词。并列现象为基于词汇的解释提供了证据。
%On the lexical account, the verb \emph{paint} in (\ref{paint}), for example, is lexically a
%2"=argument verb, while the unary branching node immediately dominating it is effectively a
%3"=argument verb.  On the constructional view there is no such   predicate seeking three arguments
%that dominates only the verb.  Coordination provides evidence for the lexical account.
    对于并列的概括是,如果两个成分有相容的句法属性,那么它们就可以并列,并且并列得出的成分具有组成成分的属性。这一点反应在范畴语法中就是为并列假设了一个(X\bs X)/X范畴:并列将第一个X放在左边,第二个X放在右边,结果是X。
%A generalization about coordination is that two constituents which have compatible syntactic
%properties can be coordinated and that the result of the coordination is an object that has the
%syntactic properties of each of the conjuncts. This is reflected by the
%Categorial Grammar analysis which assumes the category (X\bs X)/X for the conjunction: the
%conjunction takes an X to the right, an X to the left and the result is an X.
    例如,(\mex{1}a)就是两个动词并列的例子。并列结构know and like(知道并且喜欢)的句法表现与参与并列的简单动词一致,即该并列结构带一个主语和一个宾语。与此相似,在(\mex{1}b)中两个省略宾语的句子并列,并且结果是一个省略宾语的句子。
\eal
\ex[]{
\gll I know and like this record.\\
我 知道 并且 喜欢 \textsc{det} 唱片\\
\mytrans{我知道并且喜欢这一唱片。} 
%I know and like this record.
}
\ex[]{
\gll Bagels, I like and Ellison hates.\\
百吉饼 我 喜欢 并且 Ellison 讨厌\\
\mytrans{我喜欢但是Ellison讨厌百吉饼。}  
%Bagels, I like and Ellison hates.
}
\zl
%For example, in (\mex{1}a) we have a case of the coordination of two lexical
%verbs. The coordination \emph{know and like} behaves like the coordinated simplex verbs: it takes a
%subject and an object. Similarly, two sentences with a missing object are coordinated in (\mex{1}b)
%and the result is a sentence with a missing object. 
%\eal
%\ex[]{
%I know and like this record.
%}
%\ex[]{
%Bagels, I like and Ellison hates.
%}
%\zl
    (\mex{1})中的德语例子显示参与并列的动词必须满足格要求。在(\mex{1}b、c)中,参与并列的动词分别要求宾语是宾格和与格,但是因为例句中的名词都无法满足这一格要求,所以这两个例句都是不合法的。
\eal
\ex[]{
\gll Ich kenne und unterstütze diesen      Mann.\\
     我   知道  并且 支持     \textsc{det}.\acc{} 男人\\
\mytrans{我知道并且支持这个男人。}  
%\gll Ich kenne und unterstütze diesen      Mann.\\
%     I   know  and support     this.\acc{} man\\
}
\ex[*]{
\gll Ich kenne und helfe diesen      Mann.\\
     我   知道  并且 帮助  \textsc{det}.\acc{} 男人\\
%\gll Ich kenne und helfe diesen      Mann.\\
%     I   know  and help  this.\acc{} man\\
}
\ex[*]{
\gll Ich kenne und helfe diesem      Mann.\\
     我   知道  并且 帮助  \textsc{det}.\dat{} 男人\\
%\gll Ich kenne und helfe diesem      Mann.\\
%     I   know  and help  this.\dat{} man\\
}
\zl
%The German examples in (\mex{1}) show that the case requirement of the involved verbs has to be
%respected. In (\mex{1}b,c) the coordinated verbs require accusative and dative respectively and since
%the case requirements are incompatible with unambiguously case marked nouns both of these examples are out.
%\eal
%\ex[]{
%\gll Ich kenne und unterstütze diesen      Mann.\\
%     I   know  and support     this.\acc{} man\\
%}
%\ex[*]{
%\gll Ich kenne und helfe diesen      Mann.\\
%     I   know  and help  this.\acc{} man\\
%}
%\ex[*]{
%\gll Ich kenne und helfe diesem      Mann.\\
%     I   know  and help  this.\dat{} man\\
%}
%\zl

\noindent
    值得注意的是,基础双及物动词可以与通过词汇规则跟增加了额外论元的动词并列。(\mex{1})提供了英语的例子,(\mex{1}b)中的德语的例子引自 \citew[\page 420]{MuellerGTBuch2}:
\eal
\label{promise-make}
\ex 
\gll She then offered and made me a wonderful espresso -- nice.\\
     她  然后 提供 并 做 我 一 极好的 浓咖啡 {} 好\\
\mytrans{她为我提供并做了一杯极好的浓咖啡。}  
\footnote{%
\url{http://www.thespinroom.com.au/?p=102}, \zhdate{2012/07/07}。
}
%She then offered and made me a wonderful espresso -- nice.\footnote{%
%\url{http://www.thespinroom.com.au/?p=102} 07.07.2012}
\ex 
\label{ex-gebacken-und-gegeben}
\gll ich hab ihr jetzt diese Ladung Muffins mit den Herzchen drauf gebacken und gegeben.\footnotemark\\
     我 \textsc{aux} 她 现在 \textsc{det} 装载 松饼 \textsc{prep} \textsc{det} 小.心 在......上 烤 并且 给\\
\footnotetext{%
\url{http://www.musiker-board.de/diverses-ot/35977-die-liebe-637-print.html}, \zhdate{2012/06/08}。
}
\mytrans{我已经烤好并且把大量顶端有一个小心的松饼给她了。}
%\gll ich hab ihr jetzt diese Ladung Muffins mit den Herzchen drauf gebacken und gegeben.\footnotemark\\
%     I have her now this load Muffins with the little.heart there.on~~~~ baked and given\\
%\footnotetext{%
%\url{http://www.musiker-board.de/diverses-ot/35977-die-liebe-637-print.html}. 08.06.2012
%}
%\mytrans{I have now baked and given her this load of muffins with the little heart on top.}
\zl
\noindent
这些例子显示,两个动词都是在$V^0$层面有三个论元,因为它们参与了$V^0$层的并列:
\ea
\gll 
{}[\sub{\vnull} offered and made] [\sub{NP} me]    [\sub{NP} a wonderful espresso] \\
{}\spacebr{} 提供 并 做 \spacebr{} 我 \spacebr{} 一 极好的 浓咖啡\\
\mytrans{为我提供并做了一杯极好的浓咖啡}
\z
这一现象可以在词汇分析中得到预测而不能在非词汇分析中得到预测。\footnote{%
有人会说这些句子是否是并列VP的右节点提升现象(Right Node Raising,RNR)\citep{Bresnan74a-u, Abbott76a-u}: 
\ea \label{rnr}
\gll
She $[$ offered  \_\_\_  $]$ and $[$ made me \_\_\_ $]$  a wonderful espresso. \\
她 \spacebr{} 提供 {} \spacebr{} 并 \spacebr{} 做 我 {} \spacebr{} 一 极好的 浓咖啡\\
\mytrans{她为我提供并做了一杯极好的浓咖啡。}
\z
但这是不对的。按照这一分析,第一个动词必须没有受益格或接受者宾语。但是,me(我)应该理解为“给予”和“制作”的接受者。第二,第二个宾语可以是一个非重读的代词,即She offered and made me it(她为我提供并制作),这在RNR中是不可能出现的。注意,offered and made(提供和制作)的意义不可能是伪-并列语义offered to make(提供为了制作)。这种意义只能出现在一些动词(例如try)的词干形式中。
}  
%Interestingly, it is possible to coordinate basic ditransitive verbs with verbs that have
%additional arguments licensed by the lexical rule. (\mex{1}) provides examples in English and German
%((\mex{1}b) is quoted from  \citew[\page 420]{MuellerGTBuch2}):
%\NOTE{TL: \emph{offer} as transitive verb}

%\eal
%\label{promise-make}
%\ex She then offered and made me a wonderful espresso -- nice.\footnote{%
%\url{http://www.thespinroom.com.au/?p=102} 07.07.2012}
%\ex 
%\label{ex-gebacken-und-gegeben}
%\gll ich hab ihr jetzt diese Ladung Muffins mit den Herzchen drauf gebacken und gegeben.\footnotemark\\
%     I have her now this load Muffins with the little.heart there.on~~~~ baked and given\\
%\footnotetext{%
%\url{http://www.musiker-board.de/diverses-ot/35977-die-liebe-637-print.html}. 08.06.2012
%}
%\mytrans{I have now baked and given her this load of muffins with the little heart on top.}
%\zl
%\noindent
%These sentences show that both verbs are 3"=argument verbs at the $V^0$ level, since they involve $V^0$ coordination: 
%\ea
%{}[\sub{\vnull} offered and made] [\sub{NP} me]    [\sub{NP} a wonderful espresso] 
%\z

%\noindent
%This is expected under the lexical rule analysis but not the non-lexical constructional one.\footnote{%
%One might wonder whether these sentences could be instances of Right Node Raising (RNR) out of coordinated VPs \citep{Bresnan74a-u, Abbott76a-u}:  
%\ea \label{rnr}
%She $[$ offered  \_\_\_  $]$ and $[$ made me \_\_\_ $]$  a wonderful espresso. 
%\z
%But this cannot be correct.  
%Under such an analysis the first verb has been used without a
%benefactive or recipient object.  But \emph{me} is interpreted as the recipient of both the offering and making.
%Secondly, the second object can be an unstressed pronoun (\emph{She offered and made me it}), which is not possible in RNR.  Note that \emph{offered and made} cannot be a pseudo-coordination meaning `offered to make'.  This is possible only with stem forms of certain verbs such as \emph{try}.}  
%Also the verb \emph{offer} without the recipient (\emph{?She offered
%  a special sauce}) is somewhat more awkward than the sentences in (\ref{promise-make}).
    总结一下并列结构的论述:能并列的动词总体上来说应该有相容的句法属性,例如价属性。这意味着,例如在(\ref{promise-make}b)中,gebacken(烤)和gegeben(给)有相同的价属性。按照基于词汇的方法,制作动词gebacken(烤)与词汇规则结合可以允准双及物动词。所以可以与gegeben(给)并列。但是,按照短语方法,动词gebacken(烤)只有两个论元角色与动词gegeben(给)不兼容,因为动词gegeben(给)有三个论元角色。按照短语方法,gebacken(烤)只有出现在双及物短语构式或论元结构构式中时才能实现三个论元。但是在(\ref{promise-make})这种句子中,不是gebacken(烤)自己而是gebacken(烤)和gegeben(给)的组合出现在短语句法中。按照这一观点,就语义角色来讲这两个动词是不兼容的。
%Summarizing the coordination argument:  coordinated verbs generally must have compatible syntactic properties like valence properties.  This means that in (\ref{promise-make}b), for example,
%\emph{gebacken} `baked' and \emph{gegeben} `given' have the same valence properties. 
%On the lexical approach the creation verb
%\emph{gebacken}, together with a lexical rule, licenses a ditransitive verb.  It can therefore be coordinated with \emph{gegeben}. On the phrasal
%approach however, the verb \emph{gebacken} has two argument roles and is not compatible with the verb
%\emph{gegeben}, which has three argument roles. In the phrasal model, \emph{gebacken} can only realize three arguments when it
%enters the ditransitive phrasal construction or argument structure construction.  But in sentences like (\ref{promise-make}) it is not
%\emph{gebacken} alone that enters the phrasal syntax, but rather the combination of \emph{gebacken} and
%\emph{gegeben}. On this view, the verbs are incompatible as far as the semantic roles are concerned. 
    按照构式方法要解决这一问题,可以提出一种机制使得组成并列短语的动词具有与并列短语baked and given(被烤和被给)相同的语义角色,这样两个并列动词就兼容了。但是,这就是相当于动词baked(烤)有多个动词义项,而这一点是反对-词汇主义的学者极力避免的,这一点将在下一节论述。
%To fix this under the phrasal approach, one could posit a mechanism such that the semantic roles that are required for the coordinate phrase \emph{baked and
%  given} %percolate down to 
%  are shared by each of its conjunct verbs and that they are therefore compatible.  But this would
%  amount to saying that there are several verb senses for \emph{baked}, something that the
%  anti-lexicalists claim to avoid, as discussed in the next section.

%A reviewer of Theoretical Linguistics suggested an approach in which lexical items are
%underspecified with regard to their valence structure.  The valence information is added by the
%phrasal construction and the coordination construction has to make sure that the valence information
%on the conjuncts matches. This is an interesting suggestion but it requires the introduction of
%valence representations into the phrasal approach that are not needed for other reasons than the analysis of
%coordination. This seems to be an unwanted consequence of the phrasal analysis.
    《理论语言学》(Theoretical Linguistics)的一位审稿人正确地指出:一种(短语)论元结构构式方法可以像我们的词汇分析一样工作。我们的双及物词汇规则只是简单地被重新处理为“双及物论元结构构式”。这一构式可以与baked(烤)组合,因此可以在跟gave(给)组合之前增加第三个论元。只要这一论元结构构式方法与词汇规则方法的差异仅仅是表示方法的不同,那么就会与词汇规则以同样的方法工作。但是,论元结构构式方法的文献都将它作为与词汇规则极其不同的一种方法,在这种方法中构式通过承继层级组合而不是允许词汇规则在与其它词或短语组合之前先改变动词的论元结构。
%A reviewer of Theoretical Linguistics correctly observes that a version of the (phrasal) ASC approach could work in the exactly same way as our lexical analysis.  
%Our ditransitive lexical rule would simply be rechristened as a ``ditransitive ASC''.  This construction would combine with \emph{baked}, thus adding the
%third argument, prior to its coordination with \emph{gave}.  As long as the ASC approach is a non-distinct notational
%variant of the lexical rule approach then of course it works in exactly the same way.  But the literature on the ASC approach represents
%it as a radical alternative to lexical rules, in which constructions are combined through inheritance hierarchies, instead of allowing lexical rules 
%to alter the argument structure of a verb prior to its syntactic combination with the other words and phrases.  
    这个评论家还指出(\mex{1})中的例子显示受益论元必须在短语层面上引入。
\ea
\gll I designed and built him a house.\\
     我  设计 并且 建造 他 一 房子\\
\mytrans{我为他设计并建造了一座房子。}  
%I designed and built him a house.
\z
designed(设计)和built(建造)都是二价动词并且him(他)是让designed(设计)和built(建造)的论元扩展的受益格。但是,我们认为(\mex{0})中的句子可以看做两个词项的组合,这两个词项都由引入受益论元的词汇规则允准。即,受益论元是在并列之前引入的。
%The reviewer also remarked that examples like (\mex{1}) show that the benefactive argument has to be
%introduced on the phrasal level.
%\ea
%I designed and built him a house.
%\z
%Both \emph{designed} and \emph{built} are bivalent verbs and \emph{him} is the benefactive that
%extends both \emph{designed} and \emph{built}. However, we assume that sentences like (\mex{0}) can
%be analyzed as coordination of two verbal items that are licensed by the lexical rule that
%introduces the benefactive argument. That is, the benefactive is introduced before the coordination.
    并列现象展示了一个更为普遍的观点。正如分析(\ref{ex-gebacken-und-gegeben})中的gebacken(烤)的词汇规则那样,词汇规则的输出只是一个词(一个\xzeroc),所以与同类非派生词有相同的句法属性和价特征。这一重要的概括性解释是基于词汇方法的,但是按照基于短语的观点,这一点顶多是难以理解的。这一点可以从以下规则看出(反对-词汇主义的学者十分想取消这些词汇规则来支持短语构式)。例如,主动动词和被动动词可以并列,只要两者有相同的价属性,正如瑞典语所示:
\ea
{\raggedright
\gll Golfklubben beg\"arde och beviljade-s marklov f\"or banbygget efter en hel del f\"orhandlingar och kompromisser med L\"ansstyrelsen och 
Naturv\aa rdsverket.\footnotemark\\
高尔夫.俱乐部.\textsc{def} 要求 并且 授予-\textsc{pass} 土地.允许 \textsc{prep} 小路.建设.\textsc{def} \textsc{prep} 一 整个 部分 谈判 并且 妥协 \textsc{prep} 国家.董事会.\textsc{def} 和 自然.保护.机构.\textsc{def} \\
\footnotetext{\url{http://www.lyckselegolf.se/klubben/kort-historik/}, \zhdate{2018/04/25}。}
\par}
\mytrans{高尔夫俱乐部要求并且在跟国家机构和环境保护机构进行了大量协商和妥协之后被授予了构建场地的土地许可证。}
%\gll Golfklubben beg\"arde och beviljade-s marklov f\"or banbygget efter en hel del f\"orhandlingar och kompromisser %med L\"ansstyrelsen och 
%Naturv\aa rdsverket.\footnotemark\\
%golf.club.\textsc{def} requested and granted-\textsc{pass} ground.permit for track.build.\textsc{def} after a whole part %negotiations and compromises with county.board.\textsc{def} and nature.protection.agency.\textsc{def} \\
%\footnotetext{http://www.lyckselegolf.se/index.asp?Sida=82}
%\par}
%\mytrans{The golf club requested and was granted a ground permit for fairlane construction after a lot of negotiations and %compromises with the County Board and the Environmental Protection Agency.}
\z
\noindent
(英语中也有相同的现象,正如上述句子的英语译文所示)。双及物动词bevilja(给予)的被动形式只保留一个宾语,所以可以非常简洁地变成及物动词并且与主动及物动词beg\"ara(要求)并列。
%The coordination facts illustrate a more general point.  The output of a lexical rule such as the one that would
%apply in the analysis of \emph{gebacken} in (\ref{ex-gebacken-und-gegeben}) is just a word (an
%\xzero), so it has the same syntactic distribution as an underived word with the same category and
%valence feature.  This important generalization follows from the lexical account while on the
%phrasal view, it is mysterious at best.  The point can be shown with any of the lexical rules that
%the anti-lexicalists are so keen to eliminate in favor of phrasal constructions.  For example,
%active and passive verbs can be coordinated, as long as they have the same valence properties, as in
%this Swedish example: 

%\ea
%{\raggedright
%\gll Golfklubben beg\"arde och beviljade-s marklov f\"or banbygget efter en hel del f\"orhandlingar och kompromisser med L\"ansstyrelsen och 
%Naturv\aa rdsverket.\footnotemark\\
%golf.club.\textsc{def} requested and granted-\textsc{pass} ground.permit for track.build.\textsc{def} after a whole part negotiations and compromises with county.board.\textsc{def} and nature.protection.agency.\textsc{def} \\
%\footnotetext{http://www.lyckselegolf.se/index.asp?Sida=82}
%\par}
%\mytrans{The golf club requested and was granted a ground permit for fairlane construction after a lot of negotiations and compromises with the County Board and the Environmental Protection Agency.}
%\z
%\noindent
%(English works the same way, as shown by the grammatical translation line.)  
%The passive of the ditransitive verb \emph{bevilja} `grant' retains one object, so it is effectively
%transitive and can be coordinated with the active transitive \emph{beg\"ara} `request'. 
    另外,英语被动动词形式是一个分词,可以成为从动词派生出形容词词汇规则的输入。所有类型的英语分词都可以转换为形容词(Bresnan, \citeyear{Bresnan82a};\citeyear[\S~3]{Bresnan2001a}):
\eal
\ex 主动 现在 分词(即,叶子正在落下):the falling leaf
\ex 主动 过去 分词(即,叶子已经落下了):the fallen leaf
\ex 被动 分词(即,玩具被孩子弄坏了。):the broken toy
%\ex active present participles (cf.\,The leaf is falling): \emph{the falling leaf} 
%\ex active past participles (cf.\,The leaf has fallen): \emph{the fallen leaf} 
%\ex passive participles (cf.\,The toy is being broken (by the child).): \emph{the broken toy} 
\zl

\noindent
派生形式是形容词而不是动词,这一点可以从一系列属性得到证实。这些属性包括:添加否定前缀un后(形式为unbroken)意义是not broken(不是坏的)。而否定前缀un出现在动词之前不是表否定而是表示相反的动作,例如untie(解开)(Bresnan, \citeyear[\page 21]{Bresnan82a};\citeyear[\S~3]{Bresnan2001a})。充当谓语的形容词其主体是其主语,而充当修饰语的形容词其主体是被修饰的名词(The toy remained (un-)broken(玩具仍然完好无损);the broken toy(损坏了的玩具))。因为这些形容词是$A^0$,所以这种形式可以与另外一个$A^0$并列,如下所示:
\eal
\ex 
\gll The suspect should be considered [armed and dangerous].\\
    \textsc{det} 嫌疑犯 \textsc{aux} \textsc{aux} 认为 携带武器的 并且 危险的\\
\mytrans{我们应该认为这一嫌疑犯是携带武器并且是危险的。} 
%The suspect should be considered [armed and dangerous].
\ex 
\gll any [old, rotting, or broken] toys\\
    任意 \spacebr{}旧的 腐烂的 或者 坏的 玩具\\
\mytrans{任意旧的、腐烂或者坏的玩具} 
%any [old, rotting, or broken] toys
\zl

\noindent
在(\mex{0}b)中,三个形容词并列,其中一个是非派生的,即old(老),一个派生自现在分词,即rotting(腐烂的),一个来自被动分词,即broken(坏的)。按照词汇理论,这种现象是非常寻常的。每一个\azeroc 并列项的价特征(在HPSG中这可能是谓词的\textsc{spr}属性或者是名词前修饰语的\textsc{mod}属性)都与并列结构母节点的价属性相同。但是短语(或者是论元结构构式)理论反对词语有这种价特征。
%Moreover, the English passive verb form, being a participle, can feed a second lexical rule deriving
%adjectives from verbs.  All categories of English participles can be converted to adjectives
%(Bresnan, \citeyear{Bresnan82a}, \citeyear[Chapter~3]{Bresnan2001a}):

%\eal
%\ex active present participles (cf.\,The leaf is falling): \emph{the falling leaf} 
%\ex active past participles (cf.\,The leaf has fallen): \emph{the fallen leaf} 
%\ex passive participles (cf.\,The toy is being broken (by the child).): \emph{the broken toy} 
%\zl

%\noindent
%That the derived forms are adjectives, not verbs, is shown by a host of properties, including
%negative \emph{un-} prefixation: \emph{unbroken} means `not broken', just as \emph{unkind} means
%`not kind', while the \emph{un-} appearing on verbs indicates, not negation, but action reversal, as
%in \emph{untie} (Bresnan, \citeyear[\page 21]{Bresnan82a}, \citeyear[Chapter~3]{Bresnan2001a}).  Predicate adjectives preserve the subject of predication of the verb and for
%prenominal adjectives the rule is simply that the role that would be assigned to the subject goes to
%the modified noun instead (\emph{The toy remained (un-)broken.}; \emph{the broken toy}).  Being an
%$A^0$, such a form can be coordinated with another $A^0$, as in the following:

%\eal
%\ex The suspect should be considered [armed and dangerous].
%\ex any [old, rotting, or broken] toys
%\zl

%\noindent
%In (\mex{0}b), three adjectives are coordinated, one underived (\emph{old}), one derived from a
%present participle (\emph{rotting}), and one from a passive participle (\emph{broken}).  Such
%coordination is completely mundane on a lexical theory.  Each \azero conjunct has a valence feature
%(in HPSG it would be the \textsc{spr} feature for predicates or the \textsc{mod} feature for the prenominal
%modifiers), which is shared with the mother node of the coordinate structure.  But the point of the
%phrasal (or ASC) theory is to deny that words have such valence features.   
    价结构的词汇派生不同于短语组合得到了来自于动源名词化证据的进一步支持\citep{Wechsler2008a}。从动词派生出名词,-ing后缀可以能产地应用于所有可以屈折的动词(例如the shooting of the prisoner(囚犯的射击))。很多其它词缀,例如-(a)tion(例如*\,the shootation of the prisoner(囚犯的射击))的能产性就非常受限。所以像destruction(破坏)和distribution(分配)这种词只能储存在心理词典中,而通过附加ing构成的名词,例如,looting(抢劫)或growing(成长)都可以,(对于少数动词或者新词)只能通过词汇规则从动词或词根派生而来\citep{Zucchi93a-u}。这一差异就解释了为什么带ing的名词保留了同源动词的论元结构而其它形式则显示了一些差异。一个非常著名的例子是名词growth(种植)缺少施事论元而名词growing(种植)则保留了施事论元:试比较*\,John's growth of tomatoes(John种植马铃薯)和John's growing of tomatoes(John种植马铃薯)\citep{Chomsky70a}。\footnote{进一步的讨论可以参看~\ref{deverbal-sec}。} 
%The claim that lexical derivation of valence structure is distinct from phrasal combination is
%further supported with evidence from deverbal nominalization \citep{Wechsler2008a}.  To derive nouns
%from verbs, \emph{-ing} suffixation productively applies to all inflectable verbs (\emph{the shooting
%  of the prisoner}), while morphological productivity is severely limited for various other suffixes
%such as \emph{-(a)tion} (\emph{*\,the shootation of the prisoner}).  So forms such as \emph{destruction}
%and \emph{distribution} must be retrieved from memory while \emph{-ing} nouns such as \emph{looting} or
%\emph{growing} could be (and in the case of rare verbs or neologisms, must be) derived from the verb
%or the root through the application of a rule \citep{Zucchi93a-u}.  
%This difference explains why \emph{-ing} nominals always retain the argument structure of the cognate verb, while other forms show
%some variation.  A famous example is the lack of the agent argument for the noun \emph{growth} versus
%its retention by the noun \emph{growing}: \emph{*\,John's growth of tomatoes} versus \emph{John's growing
%  of tomatoes} \citep{Chomsky70a}.\footnote{See Section~\ref{deverbal-sec} for further discussion.} 
    但是,派生出了带ing名词的是词汇规则还是短语结构?在Marantz的\citeyearpar{Marantz97a}短语分析中,一个短语构式(表示成\vPc)负责为带ing的名词,例如growing(种植),指派施事角色。按照他的观点,这些词都没有直接通过它们的论元结构来选择施事。ing可以出现在\vPc 构式中,该构式云允准领属施事。非ing名词,像destruction(破坏)和growth(生长)不能出现在\vPc 中。这些词是否允许施事表达取决于词的语义和语用属性:destruction(破坏)涉及外部致使所以允许一个施事,而growth(种植)涉及内部致使所以不允许施事。
%But what sort of rule derives the \emph{-ing} nouns, a lexical rule or a phrasal one?  
%In Marantz's \citeyearpar{Marantz97a} phrasal analysis,  a phrasal
%construction (notated as \vP) is responsible for assigning the agent role 
%of  \emph{-ing} nouns such as \emph{growing}.  For him, none of the words directly selects an agent via its argument structure.
%The \emph{-ing} forms are
%permitted to appear in the \vP construction, which licenses the possessive agent.  
%Non-\emph{ing} nouns such as \emph{destruction} and  \emph{growth} do not appear in \vP.  Whether they allow
%expression of the agent depends on semantic and pragmatic properties of the word: \emph{destruction} involves external 
%causation so it does allow an agent, while \emph{growth} involves internal causation so it does not allow an agent.
    然而,Marantz的分析存在一个问题,那就是两种类型的名词可以并列并且共有依存成分,例(\mex{1}a)引自 \citew[\S~7]{Wechsler2008a}:
%However, a problem for Marantz is that these two types of nouns can coordinate and share dependents (example
%(\mex{1}a) is from  \citew[Section~7]{Wechsler2008a}): 

\eal
\ex 
\gll With nothing left after the soldier's [destruction and looting] of their home, they reboarded their coach and set out for the port of Calais.\\
    \textsc{prep} 没有东西 留下 \textsc{prep} \textsc{det} 士兵的 \spacebr{}破坏 和 抢劫 \textsc{prep} 他们的 家 他们 重新上 他们的 马车 和 出发 \textsc{adv} \textsc{prep} \textsc{det} 港口 \textsc{prep} 加来\\
\mytrans{在士兵破坏和抢劫之后,没有任何东西留下,他们回到他们的车上向加来港口出发。} 
%With nothing left after the soldier's [destruction and looting] of their home, they reboarded
%their coach and set out for the port of Calais.\footnote{\url{http://www.amazon.com/review/R3IG4M3Q6YYNFT}, %21.07.2012}
\ex  
\gll The [cultivation, growing or distribution] of medical marijuana within the County shall at all times occur within a secure, locked, and fully enclosed structure, including a ceiling, roof or top, and shall meet the following requirements.\\
    \textsc{det}  耕作 种植 或者 分发 \textsc{prep} 医用的 大麻 \textsc{prep} \textsc{det} 国家 \textsc{aux} 在 所有的 时间 发生 \textsc{prep} 一 安全的 封锁的 并且 全部 封闭的 结构 包括 一 天花板 屋顶 或者 顶部 并且 \textsc{aux} 达到 \textsc{det} 下面的 要求\\
\mytrans{国内医用大麻的栽培、种植或者分配都应该总是在一个安全、封锁、完全封闭的结构中进行,这一结构包括顶部等;并且需要达到以下要求。} 
%The [cultivation, growing or distribution] of medical marijuana within the County shall at all
%times occur within a secure, locked, and fully enclosed structure, including a ceiling, roof or top,
%and shall meet the following
%requirements.\footnote{%
%\href{http://www.scribd.com/doc/64013640/Tulare-County-medical-cannabis-cultivation-ordinance}{http://%www.scribd.com/doc/64013640/Tulare-County-medical-cannabis-cultivation-ordinance}, 05.03.2016}  
\zl
%\ex I believe it is time in the USA voting population to have the opportunity to vote on adding an
%amendment to the Bill of Rights to legalize the [use, growth and selling] of
%marijuana.\footnote{\url{http://signon.org/sign/constitutional-amendment-28}} 
    按照短语分析,名词looting(抢劫)和growing(种植)出现在一种句法环境(即\vPc 中),而destruction(破坏)、cultivation(栽培)和distribution(分布)出现在另外一个不同的句法环境中。这就给(\mex{0})所示的并列结构提出了矛盾的要求。 \citet{Wechsler2008a}提出的这一问题和其它问题都尚未有支持用短语理论分析论元结构的学者解决。
%On the phrasal analysis, the nouns \emph{looting} and \emph{growing} occur in one type
%of syntactic environment (namely \vP), while forms \emph{destruction}, \emph{cultivation}, 
% and \emph{distribution} occur in a different syntactic environment.  This places contradictory
%demands on the structure of coordinations like those in (\mex{0}).  As far as we know, neither this problem nor
%the others raised by  \citet{Wechsler2008a} have even been addressed by advocates of the phrasal theory of
%argument structure.    
    我们来看最后一个例子。在一个非常有影响力的短语分析中,Hale和Keyser(\citeyear{HK93a-u})通过名词合并从类似于[PUT a saddle ON x]的结构中派生出名源动词,例如,to saddle(给……装马鞍)。同样,有这种假想派生过程的动词也可以很正常地与其它类型的动词并列并且共享依存成分:
\ea
\gll Realizing the dire results of such a capture and that he was the only one to prevent it, he quickly [saddled and mounted] his trusted horse and with a grim determination began a journey that would become legendary.\\
    实现 \textsc{det} 悲惨的 结果 \textsc{prep} 这样 一 捕获 和 那 他 \textsc{cop} \textsc{det} 唯一 一个 \textsc{inf} 阻止 它 他 很快地 \spacebr{}装马鞍 并 骑上 他的 值得信任的 马 并且 \textsc{prep} 一 残忍的 决定 开始 一 旅程 \textsc{rel} \textsc{aux} 变成 传奇\\
\mytrans{意识到被捕后的悲惨下场以及他是唯一可以阻止这一结果的人,他给自己忠诚的马装上马鞍,并骑上这匹马,带着坚定的决心,开始了一场注定成为传奇的旅程。} 
\footnote{\url{http://www.jouetthouse.org/index.php?option=com_content&view=article&id=56&Itemid=63},
  \zhdate{2012/07/12}。}  
%Realizing the dire results of such a capture and that he was the only one to prevent it, he quickly
%[saddled and mounted] his trusted horse and with a grim determination began a journey that would
%become legendary.\footnote{\url{http://www.jouetthouse.org/index.php?%option=com_content&view=article&id=56&Itemid=63},
%  21.07.2012}  
\z

\noindent
正如所有这些\xnullc 并列案例所示,按照短语分析,两个动词都会给单个短语结构提出矛盾的要求。
%Consider one last example.  In an influential phrasal analysis, Hale and Keyser (\citeyear{HK93a-u})
%derived denominal verbs like \emph{to saddle} through noun incorporation out of a structure akin to
%[PUT a saddle ON x].  Again, verbs with this putative derivation routinely coordinate and share
%dependents with verbs of other types: 

%\ea
%Realizing the dire results of such a capture and that he was the only one to prevent it, he quickly
%[saddled and mounted] his trusted horse and with a grim determination began a journey that would
%become legendary.\footnote{\url{http://www.jouetthouse.org/index.php?option=com_content&view=article&id=56&Itemid=63},
%  21.07.2012}  
%\z

%\noindent
%As in all of these \xnull coordination cases, under the phrasal analysis the two verbs place
%contradictory demands on a single phrase structure.   
    一个词汇价结构是对动词能出现的多种句法环境的一种抽象和概括。诚然,价结构的一个关键用处是说明动词必须(或者可能)与什么种类的短语组合以及语义组合的结果。如果这是词汇价结构的全部作用,那么短语理论也是切实可行的。但是事实并非如此。事实证明,这一词汇价结构,一旦抽象出来,就可以与其它具有相似价结构的动词并列;或者可以作为词汇规则的输入,这种词汇规则可以说明产生的词语与输入有系统的联系。并列和词汇派生来源于基于词汇的观点,但是基于短语的理论最多把这些事实当做是难以理解的,在最坏的情况下导致短语结构不可调和的矛盾。
%A lexical valence structure is an abstraction or generalization over various occurrences of the verb
%in syntactic contexts.  To be sure, one key use of that valence structure is simply to indicate what
%sort of phrases the verb must (or can) combine with, and the result of semantic composition; if that
%were the whole story then the phrasal theory would be viable.  But it is not.  As it turns out, this
%lexical valence structure, once abstracted, can alternatively be used in other ways: among other
%possibilities, the verb (crucially including its valence structure) can be coordinated with other
%verbs that have similar valence structures; or it can serve as the input to lexical rules
%specifying a new word bearing a systematic relation to the input word.  The coordination and lexical
%derivation facts follow from the lexical view, while the phrasal theory at best leaves these facts
%as mysterious and at worst leads to irreconcilable contradictions for the phrase structure.   
%In Section~\ref{relations-sec} we consider what the phrasal analysis replace lexical rules.  

%The passive is not a syntactic construction, but rather a verbal valence pattern.  Passive verbs appear in many contexts:
%
%\eal
%\ex Fred got kicked by the mule.
%\ex Nina got Bill elected to the committee.
%\ex Sharon had the carpet cleaned.
%\ex Smith wants the picture removed from the office.
%\ex George saw his brother beaten by the soldiers.
%\ex Any boy handed a worm would scream.
%\ex Handed a worm, the boy screamed.
%\zl
%(examples a-e from Baker 1995, 259)
%
%If we posit many constructions then the fact that they all share the same valence structure, as well as the same morphological form, becomes a highly improbable coincidence.  A super-type construction for passive, with sub-types for get, have, want, etc, would be equivalent to a verb's valence structure.  
%
%passivization of expletives:  thus no fixed semantic content of the 'construction'

\subsection{价和派生形态}
%\subsection{Valence and derivational morphology}
\label{sec-val-morph}\label{sec-phrasal-LI}\label{sec-inheritance-passive-LFG}
 \citet{GJ2004a}、 \citet{Alsina96a}和 \citet*{ADT2008a,ADT2013a}提出将结果构式和(或)致使"=移动构式\isc{构式!致使-移动}\is{construction!Caused"=Motion}分析为短语构式。\footnote{%
 \citet[\S~2.3]{AT2014a}表示他们的分析不是构式的。如果说构式是一种形式—意义偶对的话,那么他们的分析是构式的,因为一个C"=结构与一个语义配对。 \citet[\S~2.2]{AT2014a}将他们的方法与他们认为是构式方法的方法进行了对比,这些方法包括构式化HPSG\citep{Sag97a}和基于符号的构式语法(参看~\ref{sec-SBCG})。这些方法与Asudeh, Dalrymple \& Toivonen提出的方法之间的唯一差异是:基于HPSG的理论是使用类型来组织的,所以构式都有名称。%
}正如我在 \citew{Mueller2006d}中所述,这一分析不符合词汇完整性(lexical integrity)假设。词汇完整性是指构词在句法之前完成并且形态结构不受句法过程影响\citep{BM95a}。\footnote{%
   \citet[\page 14]{ADT2013a} 认为瑞典语有向移动构式(Directed Motion Construction)与派生形态之间不存在互动。但是,相对应的德语的构式就与派生形态互动。瑞典语缺少这种互动可以通过瑞典语\il{瑞典语}\il{Swedish}语法的其它因素来解释。鉴于此,我认为更合适的做法是提出一种能同时适用于德语和瑞典语的分析方法。%
}我们来看一个实例,例如(\mex{1}):
% \citet{GJ2004a},  \citet{Alsina96a}, and  \citet*{ADT2008a,ADT2013a} suggest analyzing resultative
%constructions and/or caused"=motion constructions\is{construction!Caused"=Motion} as phrasal constructions.\footnote{%
% \citet[Section~2.3]{AT2014a} argue that their account is not constructional. If a construction is a
%form-meaning pair, their account is constructional, since a certain c"=structure is paired with a
%semantic contribution.  \citet[Section~2.2]{AT2014a} compare their approach with approaches in Constructional
%HPSG \citep{Sag97a} and Sign"=Based Construction Grammar (see Section~\ref{sec-SBCG}), which they term constructional. The only difference
%between these approaches and the approach by Asudeh, Dalrymple \& Toivonen is that the constructions in the HPSG"=based theories are modeled using types and
%hence have a name.%
%} As was argued in
% \citew{Mueller2006d} this is incompatible with the assumption of lexical integrity. Lexical
%integrity means that word formation happens before syntax and that the morphological structure is inaccessible to
%syntactic processes \citep{BM95a}.\footnote{%
%   \citet[\page 14]{ADT2013a} claim that the Swedish Directed Motion Construction does not interact
%  with derivational morphology. However, the parallel German construction does interact with
%  derivational morphology. The absence of this interaction in Swedish can be explained by other
%  factors of Swedish\il{Swedish} grammar and given this I believe it to be more appropriate to assume an
%  analysis that captures both the German and the Swedish data in the same way.%
%}
% \eal
% \ex
% \gll Er fährt den Wagen zu Schrott.\\
%      he drives the car to scrap.metal\\
% \mytrans{He drives the car to a wreck.}
% \ex
% \gll der zu Schrott gefahrene Wagen\\
%      the to scrap.metal driven car\\ 
% \mytrans{the car that was driven to a wreck}
% % Blood Red Shoes - der Name bezieht sich auf die blutig getanzten Schuhe Ginger Rogers
% \zl
%Let us consider a concrete example, such as (\mex{1}):
\eal
\label{ex-tanzt-schuhe-blutig}
\ex[]{
\gll Er tanzt die Schuhe blutig / in Stücke.\\
     他 跳舞 \textsc{det} 鞋子 带血的 {} \textsc{prep} 碎片\\
\mytrans{他跳舞把鞋都跳成了碎片/他跳舞把鞋里跳得都是血.}
%\gll Er tanzt die Schuhe blutig / in Stücke.\\
%     he dances the shoes bloody {} into pieces\\
}
\ex[]{
\gll die in Stücke / blutig getanzten Schuhe\\
     \textsc{det} \textsc{prep} 碎片 {} 带血的 跳舞 鞋子\\
\mytrans{跳舞跳成了碎片的鞋子/跳舞跳得都是血的鞋子.}
%\gll die in Stücke / blutig getanzten Schuhe\\
%     the into pieces {} bloody danced shoes\\
}
\ex[*]{
\gll die getanzten Schuhe\\
     \textsc{det} 跳舞的    鞋子\\
%\gll die getanzten Schuhe\\
%     the danced    shoes\\
}
\zl
    Schuhe(鞋)不是tanzt(跳舞)的语义论元 。但是,(\mex{0}a)中受格NP所指正是(\mex{0}b)中形容词分词的陈述对象。(\mex{0}b)中的这种形容词分词派生自管辖受事宾语的动词。如果受事宾语由(\mex{0}a)所示结构允准的话,就无法解释为什么分词getanzten(跳舞)在没有受格宾语实现的情况下也可以成立。结果构式和形态之间互动的更多的例子可以参看 \citew[\S~5]{Mueller2006d}。70年代末80年代初 \citet[\page 412]{Dowty78a}和 \citet[\page 21]{Bresnan82a}得出的结论是能进行形态学变化的现象应该在词的层面上处理。中心语驱动的短语结构语法、范畴语法、构式语法和词汇功能语法等框架中的解释都是假设词汇规则来允准结果构式。在这些框架中的一些基于词汇的分析可以参看 \citew{Verspoor97a}、 \citew{Wechsler97a}、 \citew{WN2001a}、Wunderlich (\citeyear[\page
  45]{Wunderlich92a-u-kopiert};\citeyear[\page 120--126]{Wunderlich97c})、 \citew{KW98a}、 \citew[\S~5]{Mueller2002b}、 \citew{Kay2005a}和 \citew{Simpson83a}。
%The shoes are not a semantic argument of \emph{tanzt}. Nevertheless the referent of the NP that is realized as
%accusative NP in (\mex{0}a) is the element the adjectival participle in (\mex{0}b) predicates
%over. Adjectival participles like the one in (\mex{0}b) are derived from a passive participle of a
%verb that governs an accusative object. If the accusative object is licensed phrasally by
%configurations like the one in (\mex{0}a), then it is not possible to explain why the participle \emph{getanzten}
%can be formed despite the absence of an accusative object in the valence specification of the verb. See  \citew[Section~5]{Mueller2006d} for
%further examples of the interaction of resultatives and morphology.
% Other valence-dependent derivations are the \bard (\suffix{able}). Resultatives appear in
% German \bards: \emph{leerfischbar} `empty.fishable' and \emph{Leerfischbarkeit}
% `empty.fishability'. The object of \emph{leer fischen} `to fish empty' is not the object of
% \emph{fischen} and hence it cannot be explained why \emph{fischbar}
%The conclusion drawn by  \citet[\page 412]{Dowty78a}
%and  \citet[\page 21]{Bresnan82a} in the late 70s and early 80s is that phenomena which feed morphology should be treated
%lexically. The natural analysis in frameworks like HPSG, CG, CxG, and LFG is therefore one that assumes
%a lexical rule for the licensing of resultative constructions. See
% \citew{Verspoor97a},  \citew{Wechsler97a},  \citew{WN2001a}, Wunderlich (\citeyear[\page
%  45]{Wunderlich92a-u-kopiert}; \citeyear[\page 120--126]{Wunderlich97c}),  \citew{KW98a},
%  \citew[Chapter~5]{Mueller2002b},  \citew{Kay2005a}, and  \citew{Simpson83a} for lexical proposals in some of
% these frameworks. 
    这一论述与\ref{sec-derivation-GPSG}所述的一种分析方法相似,这一方法与广义短语结构语法对价的表征方法有关。这种表征方式是:形态过程一定要能与发生形态变化的成分的价建立联系。这不同于在发生形态变化之后再由短语结构引入论元。
%This argument is similar to the one that was discussed in connection with the GPSG representation of
%valence in Section~\ref{sec-derivation-GPSG}: morphological processes have to be able to see the valence of the element
%they apply to. This is not the case if arguments are introduced by phrasal configurations after the
%level of morphology.

Asudeh, Dalrymple \& Toivonen的论文讲述了词汇完整性和构式。 \citet{AT2014a}回应了我们的争议论文并且(再次)指出他们的模板方法可以统一指定词和短语的功能结构。在他们的原始论文中,他们讨论了瑞典词vägen(路),这是väg(路)的有定形式。他们展示了这两个词的f"=结构与英语短语the way(路)的f"=结构是相同的。在我们的回应\citeyearpar{MWArgStReply}中,我认为我们过早地放弃了自己的立场,因为关键问题不是是否能够提供词的f"=结构,关键问题是形态,即用LFG的术语是如何通过形态分析来派生出f"=结构的。更加概括地说,学者是想派生出相应动词的所有属性,即价、语义以及语义和依存成分之间的联系。我们论述(\ref{ex-tanzt-schuhe-blutig})中例子所用的方法与 Bresnan(\citeyear[\page 21]{Bresnan82a};\citeyear[\page 31]{Bresnan2001a})使用词汇方法对被动进行分析的经典论述所用的方法是一样的。所以,要么Bresnan(和我们)的论述是不成立的,要么Bresnan(和我们)的论述都成立,而Asudeh, Dalrymple \& Toivonen的方法甚至是整个短语分析方法都是有问题的。下面我将讨论另外一个例子,该例子在 \citew[\page 869]{Mueller2006d}中讨论过,但是由于篇幅原因在 \citew{MWArgSt}中没有讨论。我将首先讨论为什么用短语分析法分析该例子是有问题的,然后解释为什么仅仅能够给词语指派f"=结构是不够的:在(\mex{1}a)中,我们将处理一个结果构式\isc{构式!结果构式|(}\is{construction!resultative|(}。按照插入方法,结果意义是由一个短语构式提供的,动词fischt(钓鱼)插入到该构式中。不用假设一个词项要求一个结果谓词作为其论元。如果没有这样的词项的话,很难说清楚如何建立(\mex{1}a)和(\mex{1}b)之间的联系:
%Asudeh, Dalrymple \& Toivonen's papers are about the concept of lexical integrity and about
%constructions.  \citet{AT2014a} replied to our target article and pointed out (again) that their
%template approach makes it possible to specify the functional structure of words and phrases
%alike. In the original paper they discussed the Swedish word \emph{vägen}, which is the definite
%form of \emph{väg} `way'. They showed that the f"=structure is parallel to the f"=structure for the
%English phrase \emph{the way}. 
%In our reply \citeyearpar{MWArgStReply}, we gave in too early, I believe. Since the point is
%not about being able to provide the f"=structure of words, the point is about morphology, that is
%-- in LFG terms -- about deriving the f"=structure by a morphological analysis. More generally
%speaking, one wants to derive all properties of the involved words, that is, their valence, their
%meaning, and the linking of this meaning to their dependents. What we used in our argument based on
%the sentences in (\ref{ex-tanzt-schuhe-blutig}) was parallel to what Bresnan (\citeyear[\page
%  21]{Bresnan82a}; \citeyear[\page 31]{Bresnan2001a}) used in her classical argument for a lexical
%treatment of the passive. So either Bresnan's argument (and ours) is invalid or both arguments are valid and there is a problem
%for Asudeh, Dalrymple \& Toivonen's approach and for phrasal approaches in general. I want to
%give another example that was already discussed in  \citew[\page 869]{Mueller2006d} but was omitted in
% \citew{MWArgSt} due to space limitations. I will first point out why this example is problematic for
%phrasal approaches and then explain why it is not sufficient to be able to assign certain
%f"=structures to words: in (\mex{1}a), we are dealing with a resultative construction\is{construction!resultative|(}.
%According to the plugging approach, the resultative meaning is contributed by a phrasal construction into which the
%verb \emph{fischt} is inserted. There is no lexical item that requires a resultative predicate as
%its argument. If no such lexical item exists, then it is unclear how the relation between (\mex{1}a)
%and (\mex{1}b) can be established: 

\eal
\ex 
\gll {}[dass] jemand die Nordsee leer fischt\\
     {}\spacebr{}\textsc{comp} 某人 \textsc{det} 北.海 空 钓鱼\\
\mytrans{某人钓鱼把北海的鱼都钓光了这件事}
%\gll {}[dass] jemand die Nordsee leer fischt\\
%     {}\spacebr{}that somebody the North.Sea empty fishes\\
%\mytrans{that somebody fishes the North Sea empty}
\ex\label{bsp-leerfischung}
\gll wegen      der \emph{Leerfischung}  der    Nordsee\footnotemark\\
     因为 \textsc{det} 空.钓鱼 \textsc{det} 北.海\\
\footnotetext{%
        taz,\zhdate{1996/06/20},第6页。%
}
\mytrans{因为导致北海鱼都没有了的钓鱼行为}
%\gll wegen      der \emph{Leerfischung}  der    Nordsee\footnotemark\\
 %    because of.the empty.fishing of.the North.Sea\\
%\footnotetext{%
%        taz, 20.06.1996, p.\,6.%
%}
%\mytrans{because of the fishing that resulted in the North Sea being empty}
\zl
    如图~\vref{Abbildung-Resultativkonstruktion-Nominalisierung}所示,中心语选择的论元和结构都是完全不同的。在(\mex{0}a)中充当主语的成分在(\mex{0}b)中并没有实现。正如常规名词化现象一样,可以借助介词durch(被)将该成分在一个PP中实现。
%As Figure~\vref{Abbildung-Resultativkonstruktion-Nominalisierung} shows, both the arguments selected by the heads and the structures are completely different.
%In (\mex{0}b), the element that is the subject of the related construction in (\mex{0}a) is not realized. As is normally the case in nominalizations,
%it is possible to realize it in a PP with the preposition \emph{durch} `by':
\ea
\gll wegen der Leerfischung der Nordsee durch die Anrainerstaaten\\
     因为 \textsc{det} 空.钓鱼 \textsc{det} 北.海 \textsc{prep} \textsc{det} 邻近的.州\\
\mytrans{因为邻近州导致北海的鱼消失的钓鱼行为}
%\gll wegen der Leerfischung der Nordsee durch die Anrainerstaaten\\
%     because of.the empty.fishing of.the North.Sea by the neighboring.states\\
%\mytrans{because of the fishing by the neighboring states that resulted in the North Sea being empty}
\z
%
\begin{figure}
%\hfill
\begin{forest}
sm edges
[S
	[NP{[\textit{nom}]}
		[jemand;某人]]
	[NP{[\textit{acc}]}
		[die Nordsee;\textsc{det} 北.海, roof]]
	[Adj
		[leer;空]]
	[V
		[fischt;钓鱼]]]
\end{forest}
\hfill
\begin{forest}
sm edges
[NP
	[Det
		[die;\textsc{det}]]
	[N$'$
		[N
			[Leerfischung;空.钓鱼]]
		[NP{[\textit{gen}]}
			[der Nordsee;\textsc{det} 北.海, roof]]]]
\end{forest}
%\hfill\mbox{}
\caption{\label{Abbildung-Resultativkonstruktion-Nominalisierung}结果构式和名词化}
\end{figure}%
%
    如果有人假设结果义来自于动词所在的一个特定的结构,那么就无法解释(\mex{-1}b),因为在这一例子的分析中并不涉及动词。当然还可以假设在(\mex{-1}a)和(\mex{-1}b)中都是动词词干插入构式。如果是这样的话,屈折语素\suffix{t}和派生语素\suffix{ung}和空名词性屈折语素都会成为独立的句法成分。但是, \citet[\page 119]{Goldberg2003a}和 \citet{ADT2013a}都认同词汇完整性,只有完整的词才能插入到构式之中所以他们不会采用这种方式来分析结果构式的名词化。
%If one assumes that the resultative meaning comes from a particular configuration in which a verb
%is realized, there would be no explanation for (\mex{-1}b) since no verb is involved in the analysis
%of this example. One could of course assume that a verb stem is inserted into a construction both in
%(\mex{-1}a) and (\mex{-1}b). The inflectional morpheme \suffix{t} and the derivational
%morpheme \suffix{ung} as well as an empty nominal inflectional morpheme would then be independent syntactic
%components of the analysis. However, since  \citet[\page 119]{Goldberg2003a} and  \citet{ADT2013a}
%assume lexical integrity, only entire words can be inserted into syntactic constructions and hence
%the analysis of the nominalization of resultative constructions sketched here is not an option for them.
    有人可能会尽力用承继\isc{承继}\is{inheritance}来解释(\mex{-1})中短语的相似之处。使用承继的方法会假设一个概括的结果构式,并且该结果构式与中心语为动词的结果构式以及名词化构式都有承继关系。我已经在 \citew[\S~5.3]{Mueller2006d}详细讨论过这种方案。这种方案并不可行,因为派生形态需要嵌套操作,而嵌套操作无法用承继层级来模拟 ( \citew{KN93a},详细讨论也可以参看 \citew{Mueller2006d})。
%One might be tempted to try and account for the similarities between the phrases in (\mex{-1}) using
%inheritance\is{inheritance}. One would specify a general resultative construction standing in an inheritance relation
%to the resultative construction with a verbal head and the nominalization construction. I have discussed this proposal in more detail in
% \citew[Section~5.3]{Mueller2006d}. It does not work as one needs embedding for derivational morphology and this cannot be modeled
%in inheritance hierarchies ( \citew{KN93a}, see also  \citew{Mueller2006d} for a detailed discussion).
    当然也可以假设(\mex{1})中的两个构式(当然也必须假设它们的结构如跟图~\ref{Abbildung-Resultativkonstruktion-Nominalisierung}一样)通过元规则相连。
%It would also be possible to assume that both constructions  in (\mex{1}), for which structures such as those in
%Figure~\ref{Abbildung-Resultativkonstruktion-Nominalisierung} would have to be assumed, are connected via metarules.
\footnote{%
  Goldberg (p.\,c.\ 2007, 2009)提出使用像广义短语结构语法元规则那样的方式将一些构式连接起来。 \citet[\page 51]{Deppermann2006a}更加赞同Croft对于构式语法的观点,他反对这种方案。他支持主动/被动\isc{被动}\is{passive}交替,并且认为被动结构有其它的信息结构\isc{信息结构}\is{information structure}属性。也要注意,广义短语结构语法元规则联系短语结构规则,即局部句法树。但是,图~\ref{Abbildung-Resultativkonstruktion-Nominalisierung-Construction}中的结构是非常复杂的。
% Goldberg (p.\,c.\ 2007, 2009) suggests connecting certain constructions using GPSG"=like metarules.
%   \citet[\page 51]{Deppermann2006a}, who has a more Croftian view of CxG, rules this out.
% He argues for active/passive\is{passive} alternations that the passive construction has other information
%structural\is{information structure} properties.  Note also that GPSG metarules relate phrase
%structure rules, that is, local trees. The structure in
%Figure~\ref{Abbildung-Resultativkonstruktion-Nominalisierung-Construction}, however, is highly complex.
}$^,$\footnote{%
(\mex{1}b)中的结构违反了词汇功能语法\indexlfgc 普遍假设的词完整性的严格定义。但是, \citet{Booij2005a,Booij2009a}在构式语法\indexcxgc 框架中展开研究,遵循一个较弱的定义。
% The structure in (\mex{1}b) violates a strict interpretation of lexical integrity as is commonly assumed in
%  LFG\indexlfg.  \citet{Booij2005a,Booij2009a}, working in Construction Grammar\indexcxg, subscribes to a somewhat
%  weaker version, however.%
}
\eal
\ex {}[ Sbj Obj Obl V ]
\ex {}[ Det [ [ Adj V -ung ] ] NP[\type{gen}] ]
\zl
(\mex{0}b)中的构式对应于图~\vref{Abbildung-Resultativkonstruktion-Nominalisierung-Construction}。
%The construction in (\mex{0}b) corresponds to
%Figure~\vref{Abbildung-Resultativkonstruktion-Nominalisierung-Construction}.
\footnote{%
  我并没有为屈折假设零词缀。在图~\ref{Abbildung-Resultativkonstruktion-Nominalisierung-Construction}中相应的词缀只是为了说明那里有一个结构。当然也可以假设一个单分支规则或构式,这在中心语驱动的短语结构语法或构式形态学中都很常见。
 % I do not assume zero affixes for inflection. The respective affix in
 % Figure~\ref{Abbildung-Resultativkonstruktion-Nominalisierung-Construction} is there to show that
  %there is structure. Alternatively one could assume a unary branching rule/construction as is
  %common in HPSG/Construction Morphology.
}
\begin{figure}
\centering
\scalebox{.98}{%
\begin{forest}
%sm edges
for tree={fit=rectangle}
[NP
	[Det]
	[N$'$
		[N 
                   [N-Stem
			[Adj]
			[V-Stem]
			[-ung] ]
                   [N-Affix [\trace] ]]
		[{NP[\textit{gen}]}] ] ]
\end{forest}
}
\caption{\label{Abbildung-Resultativkonstruktion-Nominalisierung-Construction}结果构式和名词化}
%\caption{\label{Abbildung-Resultativkonstruktion-Nominalisierung-Construction}Resultative construction and nominalization}
\end{figure}%
领属性NP是形容词的一个论元。它必须与形容词的主语位置在语义上发生联系。当然,也可以假设构式只包含[Adj V \suffix{ung}],即不包含领属性NP。还可以假设结果构式的动词性变体的形式为[OBL V],Sbj和Obj只是出现在价列表中。但是,这样一来,这几乎就是一种词汇分析。
%The genitive NP is an argument of the adjective. It has to be linked semantically to the subject slot of the adjective.
%Alternatively, one could assume that the construction only has the form [Adj V \suffix{ung}], that
%is, that it does not include the genitive NP. But then one could also assume that the verbal variant
%of the resultative construction has the form [OBL V] and that Sbj and Obj are only represented in
%the valence lists. This would almost be a lexical analysis, however.
    再回到词语完整性问题,我想指出的是Asudeh \& Toivonen可以做的是给图~\ref{Abbildung-Resultativkonstruktion-Nominalisierung-Construction}结构中的N指派某一f"=结构。但是,真正需要的是有力地解释这一f"=结构是怎样产生的以及它如何与句子层面的结果构式发生联系的。
%Turning to lexical integrity again, I want to point out that all that Asudeh \& Toivonen can do is
%assign some f"=structure to the N in
%Figure~\ref{Abbildung-Resultativkonstruktion-Nominalisierung-Construction}. What is needed, however,
%is a principled account of how this f"=structure comes about and how it is related to the
%resultative construction on the sentence level.

%% They could assume allo-constructions and make both c"=structures inherit from the same super construction.
%%
    在转入下一节有关论元结构的极端不充分赋值方法之前,我想评论一下 \citet*{AGT2014a}最近的一篇文章。该文章讨论了用短语结构的方法引入同源宾语和受益格\isc{受益格|(}\is{benefactive|(}。后者的例子见(\mex{1}a)。
%Before I turn to approaches with radical underspecification of argument structure in the next
%section, I want to comment on a more recent paper by  \citet*{AGT2014a}. The authors discuss the
%phrasal introduction of cognate objects and benefactives\is{benefactive|(}. (\mex{1}a) is an example of the latter construction. 
\eal
\ex 
\gll The performer sang the children a song.\\
     \textsc{det} 歌手 唱歌 \textsc{det} 孩子们 一 歌曲\\
\mytrans{这个歌手给孩子们唱了一首歌。}
%The performer sang the children a song.
\ex 
\gll The children were sung a song.\\
     \textsc{det} 孩子们 \textsc{aux} 唱歌 一 歌曲\\
\mytrans{这些孩子被唱了一首歌。}
%The children were sung a song. 
\zl
按照作者的观点,名词短语the children(儿童们)不是sing(唱)的论元,而是非强制性允准受益格的c"=结构规则贡献的。
%According to the authors, the noun phrase \emph{the children} is not an argument of \emph{sing} but
%contributed by the c"=structure rule that optionally licenses a benefactive.
\ea\label{c-struc-vp-benefactive}
\phraserule{V$'$}{
\rulenode{V\\* \up~=~\down\\*( @\textsc{Benefactive} )}
\rulenode{DP\\*(\up\ \lfgobj) = \down}
\rulenode{DP\\*(\up\ \objtheta) = \down}
}
\z
每当该规则被激活时,模板\textsc{Benefactive}会增加一个受益格角色并且相应语义也会与插入结构中的动词兼容。作者展示了如何利用映射来解释(\mex{-1}b)中的被动\isc{被动|(}\is{passive|(}例句,但是没有提供允准这些例句的c"=结构。为了分析这些例句,需要一个分析被动VPs的c"=结构规则并且这一规则可以允许一个受益格。所以情况会是:
%Whenever this rule is evoked, the template \textsc{Benefactive} can add a benefactive role and the
%respective semantics of this is compatible with the verb that is inserted into the structure. The
%authors show how the mappings for the passive\is{passive|(} example in (\mex{-1}b) work, but they do not provide
%the c"=structure that licenses such examples. In order to analyze these examples one would need a
%c"=structure rule for passive VPs and this rule has to license a benefactive as well. So it would
%be:
\todostefan{Is it \objtheta or \lfgobj? If it could be \lfgobj, the verb would have to be marked
  passive, since otherwise the benefactive could be introduced on intransitive verbs He laughed the children.}
\ea\label{c-struc-vp-benefactive-passive}
\phraserule{V$'$}{
\rulenode{V[pass]\\* \up~=~\down\\*( @\textsc{Benefactive} )}
\rulenode{DP\\*(\up\ \objtheta) = \down}
}
\z
注意并非任意动词都可以增加一个受益格:如(\mex{1}a)所示,向一个不及物动词增加受益格是不合乎语法的,(\mex{1}a)的被动形式也是不合乎语法的,如(\mex{1}b)所示:
%Note that a benefactive cannot be added to any verb: adding a benefactive to an intransitive verb as
%in (\mex{1}a) is out and the passive that would correspond to (\mex{1}a) is ungrammatical as well,
%as (\mex{1}b) shows:
\eal
\ex[*]{
\gll He laughed the children.\\
     他 笑 \textsc{det} 孩子们\\
%He laughed the children.
}
\ex[*]{
\gll The children were laughed.\\
     \textsc{det} 孩子们 \textsc{aux} 笑\\
%The children were laughed.
}
\zl
所以,不能仅仅假设所有c"=结构规则都非强制地引入一个受益格论元。(\ref{c-struc-vp-benefactive})和(\ref{c-struc-vp-benefactive-passive})中的两条规则都有不正常的地方。问题在于两者之间没有联系。这正是 \citet[\page 43]{Chomsky57a}在1957年所批评的,也是引入转换的原因(参看本书~\ref{Abschnitt-Transformationen})。Bresnan式LFG理论通过词汇规则后通过词汇映射理论(Lexical Mapping Theory)反映概括。但是如果在词汇表征之外增加成分,增加成分的地方的表征也应该联系起来。有人可能会说从1957年起我们关于形式工具的知识改变了。我们现在可以使用承继层级来表示概括。所以可以假设一个类型(或者一个模板),该模板是所有引入受格c"=结构的上位类型。但是,因为并非所有的规则都允许引入受格成分,这相当于说:c"=结构规则A、B、C允许引入一个受格。比较起来,基于词汇规则的方法有一个表述引入受益格。词汇规则说明什么动词适于引入一个受益格而句法规则不受影响。\isc{被动|)}\is{passive|)}
%So one could not just claim that all c"=structure rules optionally introduce a benefactive
%argument. Therefore there is something special about the two rules in (\ref{c-struc-vp-benefactive})
%and (\ref{c-struc-vp-benefactive-passive}). The problem is that there is no relation between these
%rules. They are independent statements saying that there can be a benefactive in the active and that
%there can be one in the passive. This is what  \citet[\page 43]{Chomsky57a} criticized in 1957 and
%this was the reason for the introduction of transformations (see
%Section~\ref{Abschnitt-Transformationen} of this book). Bresnan"=style LFG captured the
%generalizations by lexical rules and later by Lexical Mapping Theory. But if elements are added
%outside the lexical representations, the representations where these elements are added 
%have to be related too. One could say that our knowledge about formal tools has changed since
%1957. We now can use inheritance hierarchies to capture generalizations. So one can assume a type
%(or a template) that is the supertype of all those c"=structure rules that introduce a
%benefactive. But since not all rules allow for the introduction of a benefactive element, this
%basically amounts to saying: c"=structure rule A, B, and C allow for the introduction of a
%benefactive. In comparison, lexical rule"=based approaches have one statement introducing the
%benefactive. The lexical rule states what verbs are appropriate for adding a benefactive and
%syntactic rules are not affected.\is{passive|)}
    在 \citet{MWArgSt}中,我们指出 \citet{ADT2008a,ADT2013a}提出的分析瑞典语致使"=移动构式\isc{构式!致使-移动}\is{construction!Caused"=Motion}的方法无法用于分析德语,因为德语的致使"=移动构式与派生形态相互作用。 \citet{AT2014a}认为瑞典语不同于德语,所以适用于瑞典语的分析方法不适用于德语并不是一个问题。但是就受益格构式而言,情况则不同。虽然英语和德语在很多方面都存在差异,但是都有相似的与格构式:
%In  \citet{MWArgSt} we argued that the approach to Swedish Caused"=Motion Constructions\is{construction!Caused"=Motion} in
% \citet{ADT2008a,ADT2013a} would not carry over to German since the German construction interacts with derivational
%morphology.  \citet{AT2014a} argued that Swedish is different from German and hence there would not
%be a problem. However, the situation is different with the benefactive constructions. Although
%English and German do differ in many respects, both languages have similar dative constructions:
\eal
\ex 
\gll He baked her a cake.\\
     他 烤面包 她 一 蛋糕\\
\mytrans{他给她烤了一个蛋糕。}
%He baked her a cake.
\ex
\label{ex-er-buk-ihr-einen-kuchen} 
\gll Er buk   ihr        einen Kuchen.\\
     他 烤 她.\dat{} 一.\acc{} 蛋糕cake\\
\mytrans{他给她烤了一个蛋糕。}
%\gll Er buk   ihr        einen Kuchen.\\
%     he baked her.\dat{} a.\acc{} cake\\
\zl
现在,成分排列顺序自由这一问题可以通过假设双分支结构(在这种结构中,VP节点与它的一个论元或附接语组合)来分析(参看~\ref{Abschnitt-LFG-Umstellung})。(\mex{1})再次给出了c"=结构规则:
%Now, the analysis of the free constituent order was explained by assuming binary branching
%structures in which a VP node is combined with one of its arguments or adjuncts (see
%Section~\ref{Abschnitt-LFG-Umstellung}). The c"=structure rule is repeated in (\mex{1}):
\ea
\label{lfg-vp-regel-two}
\phraserule{VP}{
\rulenode{NP\\* (\upsp \lfgsubj|\lfgobj|\objtheta) = \down}
\rulenode{VP\\* \up~=~\down}}
\z
依存成分提供了动词的f"=结构而一致/完整性确保动词所有的论元都出现。可以向规则右边的节点增加受益格论元。但是,(\ref{ex-er-buk-ihr-einen-kuchen})的动词末位的变体会有(\mex{1})所示的结构,所以会产生伪歧义\isc{伪歧义}\is{ambiguity!spurious},因为受益格可以在任何节点引入。
%The dependent elements contribute to the f"=structure of the verb and coherence/""completeness ensure that all
%arguments of the verb are present. One could add the introduction of the benefactive argument to
%the VP node of the right-hand side of the rule. However, since the verb-final variant of
%(\ref{ex-er-buk-ihr-einen-kuchen}) would have the structure in (\mex{1}), one would get spurious
%ambiguities\is{ambiguity!spurious}, since the benefactive could be introduced at every node:
\ea
\gll weil    [\sub{VP} er [\sub{VP} ihr [\sub{VP} einen Kuchen [\sub{VP} [\sub{V} buk]]]]]\\
     因为 {}        他 {}        她 {}        一 蛋糕       {}        {}       烤\\
\mytrans{因为他为她烤了一个蛋糕。}
%\gll weil    [\sub{VP} er [\sub{VP} ihr [\sub{VP} einen Kuchen [\sub{VP} [\sub{V} buk]]]]]\\
%     because {}        he {}        her {}        a cake       {}        {}       baked\\
\z
所以唯一的选择好像是在能让嵌套继续的规则处引入受益格,即能将动词投射到VP层面上的那一规则。为了方便,在(\ref{LFG-v-vp-two})重新写下引自~\pageref{LFG-v-vp}页的(\ref{LFG-v-vp})规则:
%So the only option seems to be to introduce the benefactive at the rule that got the recursion
%going, namely the rule that projected the lexical verb to the VP level. The rule (\ref{LFG-v-vp})
%from page~\pageref{LFG-v-vp} is repeated as (\ref{LFG-v-vp-two}) for convenience.
\ea
\label{LFG-v-vp-two}
\phraserule{VP}{
\rulenode{(V)\\* \up~=~\down}}
\z
注意受益与格也可以出现在形容词环境中,如(\mex{1})所示:
%Note also that benefactive datives appear in adjectival environments as in (\mex{1}):
\eal
\ex
\gll der seiner Frau einen Kuchen backende Mann\\
     \textsc{det} 他的.\dat{} 妻子 一.\acc{} 蛋糕 烤 人\\
\mytrans{那个为她烤蛋糕的人}
%\gll der seiner Frau einen Kuchen backende Mann\\
%     the his.\dat{} wife a.\acc{} cake baking man\\
%\mytrans{the man who is baking a cake for her}
\ex
\gll der einen Kuchen seiner Frau backende Mann\\
     \textsc{det} 一.\acc{} 蛋糕  他的.\dat{} 妻子 烤 人\\
\mytrans{那个为她烤蛋糕的人}
%\gll der einen Kuchen seiner Frau backende Mann\\
%     the a.\acc{} cake  his.\dat{} wife baking man\\
%\mytrans{the man who is baking a cake for her}
\zl
为了解释这些与格,就必须假设与(\ref{LFG-v-vp-two})对应的形容词"=到"=AP规则引入与格。受益格模板的语义必须确保受益格不能加到不及物动词,如lachen(笑)或者分词形式,如lachende(笑)。虽然这可能可行,但是我认为整个方法并不好。首先,这种方法与最初的构式方案并没有任何关系,只是说明受益格可以在多个句法位置引入;第二,单分支句法规则用于一个词项,所以这一点与词汇规则十分相似;第三这种分析并没有获得构式的跨语言的共性。在基于词汇规则的方法中,如 \citet[Section~5]{BC99a}所提出的方案,受益格论元可以附加到部分动词并且词汇规则在所有有这种现象的语言中是一样的。相应语言的差异仅仅在于论元的实现方式。在有形容词分词的语言里,这些分词来源于相应的动词词干。形态规则是一致的并且独立于受益论元,另外形容词短语的句法规则不必提到受益论元。\isc{受益格|)}\is{benefactive|)}
%In order to account for these datives one would have to assume that the adjective"=to"=AP rule that
%would be parallel to (\ref{LFG-v-vp-two}) introduces the dative. The semantics of the benefactive
%template would have to somehow make sure that the benefactive argument is not added to intransitive
%verbs like \emph{lachen} `to laugh' or participles like \emph{lachende} `laughing'. While this may
%be possible, I find the overall approach unattractive. First it does not have anything to do with
%the original constructional proposal but just states that the benefactive may be introduced at
%several places in syntax, second the unary branching syntactic rule is applying to a lexical
%item and hence is very similar to a lexical rule and third the analysis does not capture cross"=linguistic commonalities of the
%construction. In a lexical rule"=based approach as the one that was suggested by  \citet[Section~5]{BC99a}, a benefactive argument is added to certain verbs
%and the lexical rule is parallel in all languages that have this phenomenon. The respective
%languages differ simply in the way the arguments are realized with respect to their heads. In languages
%that have adjectival participles, these are derived from the respective verbal stems. The
%morphological rule is the same independent of benefactive arguments and the syntactic rules for
%adjectival phrases do not have to mention benefactive arguments.\is{benefactive|)}

\section{极端不充分赋值:论元结构的终结?}
\label{radical-sec}

\subsection{新戴维森主义}

在上一节,我们检验了一些方案,这些方案假设动词有一些论元角色,并且动词插入到预先指定的能够提供额外意义的结构中。虽然我们已经指出这些方法并非没有问题,但是有更加激进的方案假设构式增加所有施事论元甚至所有论元。施事论元由动词提供这一观点是由 \citet{Marantz84a, Marantz97a}、 \citet{Kratzer96a}和 \citet{Embick2004a}等人提出的。另外有人提出所有的论元都不是由动词选择限制的。 \citet{Borer2003a-u}将这种方案称作外骨架(exoskeletal),因为句子的结构不是由谓词决定的,即动词没有投射到小句的内部“骨架”。与这种方案相对的是内骨架(endoskeletal)方法,依据这种方法句子的结构是由动词决定的,即基于词汇到方法。激进外骨架方法主要是在主流生成语法\citep{Borer94a-u,Borer2003a-u,Borer2005a-u,Schein93a-u,HK97a-u,Lohndal2012a}中提出的,但是也可以在HPSG\citep{Haugereid2009a}方法中见到。这里我们不详细讨论这种方法,但是我们会评论这种方法在分析词汇论元结构时讨论的主要问题。
%In the last section we examined proposals that assume that verbs come with certain argument roles
%and are inserted into prespecified structures that may contribute additional arguments. While we
%showed that this is not without problems, there are even more radical proposals that the
%construction adds all agent arguments, or even all arguments.  The notion that the agent argument
%should be severed from its verbs is put forth by  \citet{Marantz84a, Marantz97a},  \citet{Kratzer96a},
% \citet{Embick2004a} and others.  Others suggest that no arguments are selected by the verb.
% \citet{Borer2003a-u} calls such proposals \emph{exoskeletal} since the structure of the clause is
%not determined by the predicate, that is, the verb does not project an inner ``skeleton'' of the
%clause.  Counter to such proposals are \emph{endoskeletal} approaches, in which the structure of the
%clause is determined by the predicate, that is, lexical proposals.  The radical exoskeletal
%approaches are mainly proposed in Mainstream Generative Grammar
%\citep{Borer94a-u,Borer2003a-u,Borer2005a-u,Schein93a-u,HK97a-u,Lohndal2012a} but can also be found
%in HPSG \citep{Haugereid2009a}.  We will not discuss these proposals in detail here, but we review
%the main issues insofar as they relate to the question of lexical argument structure.
\footnote{%
对于Haugereid方法的详细讨论可以参看 \citew[\S~11.11.3]{MuellerGTBuch1}。%
%  See  \citew[Section~11.11.3]{MuellerGTBuch1} for a detailed discussion of Haugereid's approach.%
}我们的结论是现有实际现象表明词汇论元结构比其它的方法更好。 
%We conclude that the available empirical evidence favors the lexical argument structure approach over such
%alternatives.
外骨架方法通常是新戴维森主义的某种版本。 \citet{Davidson67a-u}主张在动作句中设立一个事件变量(\mex{1}a)。 \citet{Dowty89b-u}将(\mex{1}b)中的变体称作新戴维森式的(neo-Davidsonian),在这种方法中动词变成事件的一个属性,主语和补语依存成分变成次级谓词的论元,这些论元的角色是施事(agent)和主题(theme)。\footnote{%
 \citet{Dowty89b-u}称(\mex{1}a)是一种有序论元系统(ordered argument system)。
}  \citet{Kratzer96a}进一步注意到可以将两种方法融合起来,例如(\mex{1}c),其中施事(主语)论元由\relation{kill}关系提供,但是主题(宾语)论元仍然是\relation{kill}关系的论元。\footnote{%
事件变量是受限的,这一点和Davidson的最初解释是一致的。如下面所述,在Kratzer的分析中,事件变量是受lambda算子约束的。} 
%Exoskeletal approaches usually assume some version of Neo-Davidsonianism.  \citet{Davidson67a-u}
%argued for an event variable in the logical form of action sentences (\mex{1}a).
% \citet{Dowty89b-u} coined the term \emph{neo-Davidsonian} for the variant in (\mex{1}b), in which
%the verb translates to a property of events, and the subject and complement dependents are
%translated as arguments of secondary predicates such as \emph{agent} and
%\emph{theme}.\footnote{%
% \citet{Dowty89b-u} called the system in (\mex{1}a) an \emph{ordered argument system}.
%}  \citet{Kratzer96a} further noted the possibility of mixed accounts such as (\mex{1}c),
%in which the agent (subject) argument is severed from the \relation{kill} relation, but the theme (object) remains an
%argument of the \relation{kill} relation.\footnote{%
%  The event variable is shown as existentially bound, as in Davidson's original account.  
%  As discussed below, in Kratzer's version it must be bound by a lambda operator instead.} 

\eal\settowidth\jamwidth{(neo-Davidsonian)} \label{neokill1}
\ex \emph{kill}: $\lambda y\lambda x\exists e[kill(e, x, y)]$  \jambox{(Davidsonian)}
\ex \emph{kill}: $\lambda y\lambda x\exists e[kill(e) \wedge agent(e, x) \wedge theme(e, y)]$ \jambox{(neo-Davidsonian)}
\ex \emph{kill}: $\lambda y\lambda x\exists e[kill(e,y) \wedge agent(e, x)]$ \jambox{(mixed)}
\zl
 \citet{Kratzer96a}观察到戴维森、新戴维森和混合方法的差异可以“在句法层面”或者“在概念结构层面”上反映\citep[\page 110--111]{Kratzer96a}。例如,按照我们支持的基于的词汇方法,(\mex{0})中的三种方法都可以反映在动词kill(杀死)的语义内容中。按照混合模型,kill(杀死)的词项如(\ref{kill-argst-two})所示:
% \citet{Kratzer96a} observed that a distinction between Davidsonian, neo-Davidsonian and mixed can be
%made either ``in the syntax'' or ``in the conceptual structure'' \citep[\page 110--111]{Kratzer96a}.  For
%example, on a lexical approach of the sort we advocate here, any of the three alternatives in
%(\mex{0}) could be posited as the semantic content of the verb \emph{kill}.  A lexical entry for
%\emph{kill} in the mixed model is given in (\ref{kill-argst-two}). 

\ea\label{kill-argst-two}
\ms{
phon & \phonliste{ kill }\\[1mm]
%head & verb\\
arg-st & \liste{ NP$_x$, NP$_y$ }\\[2mm]
content  & kill(e, y) $\wedge$ agent(e, x)\\ 
}
\z
换句话说,词汇方法在可能事件的“概念结构”这一问题上持中立态度,正如~\ref{polysemy-subsec}一个不同连接中所说的那样。由于这一原因,很多支持新-戴维森主义方法的语义论据,例如 \citet[\S~4]{Schein93a-u}和 \citet{Lohndal2012a}提出的那些论据,就我们所见,并没有直接触及到词汇主义的核心问题。
%In other words, the lexical approach is neutral on the question of the ``conceptual structure'' of eventualities, as noted already in a different connection in 
%Section~\ref{polysemy-subsec}.  For this reason, certain semantic arguments for the neo-Davidsonian approach, such as those put forth by   \citet[Chapter~4]{Schein93a-u} 
%and  \citet{Lohndal2012a}, do not directly bear upon the issue of lexicalism, as far as we can tell.  

    但是 \citet{Kratzer96a}还有其他人,沿着这一思路更进一步并且提出了另外一种解释,新戴维森主义(或者更确切地说是混合方法)是“在句法层面上”。Kratzer认为动词仅仅指定了内部论元,如 (\mex{1}a)或(\mex{1}b)所示;而施事(外部论元)角色是由短语结构指派的。按照“新戴维斯在句法层面”上这一观点,除了事件变量,动词的词汇表征就完全没有论元,如 (\mex{1}c)所示:
%But  \citet{Kratzer96a}, among others, has gone further and argued for an account that is neo-Davidsonian (or rather, mixed) ``in the syntax''.  
%Kratzer's claim is that the verb specifies only the internal argument(s), as in (\mex{1}a) or (\mex{1}b), while the agent (external argument) role is assigned by the phrasal structure.  
%On the ``neo-Davidsonian in the syntax'' view, the lexical representation of the verb has no arguments at all, except the event variable, as shown in (\mex{1}c).

\eal
\label{neokill}\settowidth\jamwidth{(all arguments severed)}
\ex \emph{kill}: $\lambda y\lambda e[kill(e, y)]$                         \jambox{(agent is severed)}
\ex \emph{kill}: $\lambda y\lambda e[kill(e) \wedge theme(e, y)]$ \jambox{(agent is severed)}
\ex \emph{kill}: $\lambda e[kill(e))]$                                  \jambox{(all arguments severed)}
\zl
按照这些解释,动词剩余的依存成分从无声次级谓词获得语义角色,而次级谓词在短语结构中通常占据功能中心语的位置。事件识别规则可以识别动词的事件变量和无声轻动词\citep[\page 22]{Kratzer96a}:所以(\ref{neokill1})中的存在量词在(\ref{neokill})变成了lambda算子。指派施事的无生谓词称作“小动词”\isc{范畴!功能范畴!v@\textit{v}|(}\is{category!functional!v@\textit{v}|(}(参看~\ref{sec-little-v}对小动词的论述)。这些词汇外的依存成分等同于构式语法提出的构式。
%On such accounts, the remaining dependents of the verb receive their semantic roles from silent secondary predicates,
%which are usually assumed to occupy the positions of functional heads in the phrase structure.  An
%Event Identification rule identifies the event variables of the verb and the silent light verb
%\citep[\page 22]{Kratzer96a}; this is why the existential quantifiers in (\ref{neokill1}) have been
%replaced with lambda operators in  (\ref{neokill}).  A standard term for the agent-assigning silent
%predicate is ``little \emph{v}''\is{category!functional!v@\textit{v}|(} (see
%Section~\ref{sec-little-v} on \littlev).  These extra-lexical dependents are the analogs of the ones
%contributed by the constructions in Construction Grammar.   

    在下面的章节中,我重点分析支持“小动词”假设的各种论据,主要包括:习语不对称性(~\ref{idiom-asym})、动转名词(~\ref{deverbal-sec})。我们认为这些证据实际上支持词汇观点。然后我们论述外骨架方法,包括:特异句法选择(~\ref{sec-idiosyncratic-case-and-PP})和虚位 (~\ref{sec-expletives})。我们最后讨论Borer使用外骨架理论对特异性句法选择的分析 (~\ref{sec-borer}),最后是总结(~\ref{sec-underspec-summary})。
%In the following subsections we address arguments that have been put forth in favor of the \littlev
%hypothesis, from idiom asymmetries (Section~\ref{idiom-asym}) and deverbal nominals
%(Section~\ref{deverbal-sec}).  We argue that the evidence actually favors the lexical view.  Then we
%turn to problems for exoskeletal approaches, from idiosyncratic syntactic selection
%(Section~\ref{sec-idiosyncratic-case-and-PP}) and expletives (Section~\ref{sec-expletives}).  We
%conclude with a look at the treatment of idiosyncratic syntactic selection under Borer's exoskeletal theory (Section~\ref{sec-borer}), and a summary
%(Section~\ref{sec-underspec-summary}).

\subsection{小动词和习语不对称性}
\label{idiom-asym}

\mbox{} \citet{Marantz84a}和 \citet{Kratzer96a}根据假定的习语不对称性主张从(\mex{0}a)所示的论元结构提供施事。 \citet{Marantz84a}观察到虽然英语有很多习语并且动词有很多特定的意义,但是在这些习语中内部论元都是固定部分,外部论元都是可变部分,相反的情况很少见。换句话说,主语论元扮演角色的性质通常取决于宾语位置的填充项,但是相反却不成立。下面是\citep[\page 114]{Kratzer96a}举出的例子:
%\mbox{} \citet{Marantz84a} and  \citet{Kratzer96a} argued for severing the agent from the argument structure as in (\mex{0}a), on the basis of putative idiom asymmetries.
% \citet{Marantz84a} observed that while English has many idioms and specialized meanings for verbs in
%which the internal argument is the fixed part of the idiom and the external argument is free, the
%reverse situation is considerably rarer. To put it differently, the nature of the role played by the
%subject argument often depends on the filler of the object position, but not vice versa. To take
%Kratzer's examples \citep[\page 114]{Kratzer96a}: 

\eal
\ex 
\gll kill a cockroach\\
     杀死 一 蟑螂\\
\mytrans{杀死一只蟑螂}
%kill a cockroach
\ex 
\gll kill a conversation\\
    杀死 一 谈话\\
\mytrans{结束一次谈话}
%kill a conversation
\ex 
\gll kill an evening watching TV\\
     杀死 一 夜晚 看 电视\\
\mytrans{看电视来消磨一个晚上}
%kill an evening watching TV 
\ex 
\gll kill a bottle (i.e. empty it) \\
     杀死 一 瓶子\\
\mytrans{喝光一瓶饮料}
%kill a bottle (i.e. empty it) 
\ex 
\gll kill an audience (i.e., wow them) \\
     使笑死 一 观众(例如,博得......的称赞 他们)\\
\mytrans{让观众笑死了(例如,博得他们的称赞)}
%kill an audience (i.e., wow them)
\zl
另一方面,很难找到动词的特殊意义与主语的选择相关联,让宾语位置开放(下面是来自 \citew[\page 26]{Marantz84a}的例子):
%On the other hand, one does not often find special meanings of a verb associated with the choice of subject, leaving the object position open (examples from  \citew[\page 26]{Marantz84a}):

\eal
\ex 
\gll Harry killed NP.\\
     Hary 杀死 NP\\
\mytrans{Hary杀死NP。}
%Harry killed NP.
\ex 
\gll Everyone is always killing NP.\\
     每个人 \textsc{cop} 总是 杀死 NP\\
\mytrans{每个人总是在杀死NP。}
%Everyone is always killing NP. 
\ex 
\gll The drunk refused to kill NP.\\
     \textsc{det} 喝醉的 拒绝 \textsc{inf} 杀死 NP\\
\mytrans{这个醉汉拒绝杀死NP。}
%The drunk refused to kill NP. 
\ex 
\gll Silence certainly can kill NP.\\
     沉默 当然 \textsc{aux} 杀死 NP\\
\mytrans{沉默当然可以杀死NP。}
%Silence certainly can kill NP.
\zl
Kratzer观察到(\mex{1}a)所示的kill(杀死)的混合表征允许我们去指定动词随其唯一NP论元变化而变化的意义。
%Kratzer observes that a mixed representation of \emph{kill} as in (\mex{1}a) allows us to specify varying meanings that depend upon its sole NP argument.  

\eal
\ex \emph{kill}: $\lambda y\lambda e[kill(e, y)]$ 
\ex If $a$ is a time interval, then $kill(e, a) = truth$ if $e$ is an event of wasting \emph{a} \\
If $a$ is animate, then $kill(e, a) = truth$ if $e$ is an event in which $a$ dies \\
\ldots{} etc.
\zl
按照多元函数(戴维森)理论,同样可以依据施事角色的填充项来表征动词的语义。按照多元函数观点,对于(\mex{0}b)所示的那些情况“不会有技术障碍”\citep[\page 116]{Kratzer96a},除非情况相反,所以是施事角色的填充项而非主题角色决定动词的语义。但是,她写道:如果施事不是动词的一个论元,情况就不是这样。按照Kratzer的观点,提供施事的表征(例如 (\mex{0}a))不允许对取决于施事的意义提供限制,因此捕捉到了习语的不对称性。
%On the polyadic (Davidsonian) theory, the meaning could similarly be made to depend upon the filler of the agent role.  On the polyadic view, ``there is no technical obstacle'' \citep[\page 116]{Kratzer96a} to conditions like those in (\mex{0}b), except reversed, so that it is the filler of the agent role instead of the theme role that affects the meaning.  But, she writes, this could not be done if the agent is not an argument of the verb.  According to Kratzer, the agent-severed representation (such as (\mex{0}a)) disallows similar constraints on the meaning that depend upon the agent, thereby capturing the idiom asymmetry.  
    但是正如 \citet{Wechsler2005a}所指出的,即便是如Kratzer提出的那样施事从动词提供,指定依存于施事的语义也“没有技术障碍”。(\mex{0}a)确实没有施事变量。但是有一个事件变量e,而且说话者想要理解这一句子必须能够识别e的施事。所以如果在(\mex{0}b)表达中用“e的施事”来替代变量a,就可以产生违背习语不对称性的动词。
%But as noted by  \citet{Wechsler2005a}, ``there is no technical obstacle'' to specifying
%agent-dependent meanings even if the Agent has been severed from the verb as Kratzer proposes.  It
%is true that there is no variable for the agent in (\mex{0}a).  But there is an event variable
%\emph{e}, and the language user must be able to identify the agent of \emph{e} in order to interpret
%the sentence.  So one could replace the variable \emph{a} with ``the agent of \emph{e}'' in the
%expressions in (\mex{0}b), and thereby create verbs that violate the idiom asymmetry.
虽然好像看起来这只是一个技术上或者是一个无关紧要的问题,但是实际上这一点非常重要。假设我们尝试通过增加一个假设来修补Kratzer的论述:多义动词意义的改变只取决于动词表示的关系(relation)的论元,而不是事件的其它参与者。有了这一额外假设,施事是否由词项提供就不重要了。例如,我们来看下面kill(杀死)这个词项中语义内容的(混合)新-戴维森主义的表示。
%While this may seem to be a narrow technical or even pedantic point, it is nonetheless crucial.  Suppose we try to repair Kratzer's argument with an additional assumption: that modulations in the meaning of a polysemous verb can only depend upon arguments of the \emph{relation} denoted by that verb, and not on other participants in the event.  Under that additional assumption, it makes no difference whether the agent is severed from the lexical entry or not.   For example, consider the following (mixed) neo-Davidsonian representation of the semantic content in the lexical entry of \emph{kill}:    
\ea 
\emph{kill}: $\lambda y\lambda x\lambda e[kill(e,y) \wedge agent(e, x)]$ 
\z
假设意义调整只能由{kill(e,y)}关系的论元决定,即便是(\mex{0})是kill(杀死)的词项,我们也能推导出习语不对称性。所以假设我们尝试使用一个不同的假设来修改Kratzer的论述:一个多义动词的意义的调整只是取决于词汇决定的功能的一个论元。Kratzer在(\ref{neokill}a)中的“新戴维森在句法层面”的词项缺少施事论元,但是(\mex{0})中的词项却明显有一个施事论元。但是,Kratzer的词项仍然无法预测不对称性,因为正如上面所述,该词项有一个e论元,所以意义调整取决于“e的施事”。正如上面所述,那个事件论元不能删除(例如通过存现量词),因为在事件识别引入施事的无声轻动词中需要该事件论元\citep[\page 22]{Kratzer96a}。
%Assuming that sense modulations can only be affected by arguments of the \emph{kill(e,y)} relation,
%we derive the idiom asymmetry, even if (\mex{0}) is the lexical entry for \emph{kill}.  So suppose
%that we try to fix Kratzer's argument with a different assumption: that modulations in the meaning
%of a polysemous verb can only depend upon an argument of the lexically denoted function.  Kratzer's
%``neo-Davidsonian in the syntax'' lexical entry in (\ref{neokill}a) lacks the agent argument, while
%the lexical entry in (\mex{0}) clearly has one.  But Kratzer's entry still fails to predict the
%asymmetry because, as noted above, it has the \emph{e} argument and so the sense modulation can be
%conditioned on the ``agent of \emph{e}''.  As noted above, that event argument cannot be eliminated
%(for example through existential quantification) because it is needed in order to undergo event
%identification with the event argument of the silent light verb that introduces the agent
%\citep[\page 22]{Kratzer96a}.
    另外,用词汇主义的术语重新表示Kratzer的解释允许动词发生变化。这是一个重要优势,因为所谓不对称性只是一个倾向。下面的例子中主语都是习语的固定部分而非主语部分都是带有空槽的:
%Moreover, recasting Kratzer's account in lexicalist terms allows for verbs to vary.  This is an
%important advantage, because the putative asymmetry is only a tendency.  The following are examples
%in which the subject is a fixed part of the idiom and there are open slots for
%non-subjects:
\footnote{%
(\mex{1}a)引自 \citew*[\page 526]{NSW94a},(\mex{1}b)引自 \citew[\page349--350]{Bresnan82c}而(\mex{1}c)引自 \citew[\page349--350]{Bresnan82c}。
%\mex{1}a) is from  \citew*[\page 526]{NSW94a}, (\mex{1}b) from  \citew[\page349--350]{Bresnan82c}, and
%  (\mex{1}c) from  \citew[\page349--350]{Bresnan82c}.%
}
\eal
\ex\label{bird}
\gll A little bird told X that S.\\
     一 小 鸟 告诉 X \textsc{comp} S\\
\mytrans{X听到了传言S.}
 %A little bird told X that S.
%\mytrans{X heard the rumor that S.}
\ex\label{cat-tounge}
\gll The cat's got X's tongue.\\
     一 猫.\textsc{aux} 获得 X.\textsc{poss} 舌头\\
\mytrans{X无法说话.}   
%The cat's got X's tongue.
%\mytrans{X cannot speak.}   
\ex\label{what-is-eating-x}
\gll What's eating X?.\\
     什么.\textsc{aux} 吃 X\\
\mytrans{什么事让X不开心了?}
%What's eating X?
%\mytrans{Why is X so galled?}
\zl
更多有关英语和德语中主语习语的例子和讨论参看 \citew[\S~3.2.1]{MuellerLehrbuch1}。
%Further data and discussion of subject idioms in English and German can be found in  \citew[Section~3.2.1]{MuellerLehrbuch1}.
%the Appendix below.  

主语—宾语不对称性取向有着独立的解释。 \citet*{NSW94a}认为主语-宾语不对称性是由有生不对称性决定的。习语的槽倾向于是有生的,而固定的位置倾向于是非有生的。 \citet{NSW94a}从隐喻产生的习语的比喻和言语性质概括出来的。如果这种倾向可以找到独立的解释,那么词汇主义语法就可以使用混合新-戴维森词汇分解方法来编码这些模式,正如上文所解释的那样(参看 \citet{Wechsler2005a}使用这种词汇方法来分析动词buy(买)和sell(卖))。但是小动词假设严格预测所有施事动词都有这种不对称性,这种预测并不准确。\isc{范畴!功能范畴!v@\textit{v}|)}\is{category!functional!v@\textit{v}|)}
%The tendency towards a subject-object asymmetry plausibly has an independent explanation.
% \citet*{NSW94a} argue that the subject-object asymmetry is a side-effect of an animacy asymmetry.
%The open positions of idioms tend to be animate while the fixed positions tend to be inanimate.
% \citet{NSW94a} derive these animacy generalizations from the figurative and proverbial nature of the
%metaphorical transfers that give rise to idioms.  If there is an independent explanation for this
%tendency, then a lexicalist grammar successfully encodes those patterns, perhaps with a mixed
%neo-Davidsonian lexical decomposition, as explained above (see  \citet{Wechsler2005a} for such a
%lexical account of the verbs \emph{buy} and \emph{sell}).  But the \littlev hypothesis rigidly
%predicts this asymmetry for all agentive verbs, and that prediction is not borne out.
%\is{category!functional!v@\textit{v}|)}

\subsection{动转名词}
\label{deverbal-sec}

反对词汇论元结构的一个有影响的论述涉及英语中的动转名词和致使变换式。这一论述最早由 \citet{Chomsky70a}提出, \citet{Marantz97a}将该论述发展得更加详细;也可以参看 \citet{Pesetsky96a-u}和 \citet{HN2000a}。虽然该论述不断被重复,但是事实证明该论述的事实证据是不对的,并且语言事实正好相反,支持词汇论元结构\citep{Wechsler2008b, Wechsler2008a}。
%An influential argument against lexical argument structure involves English deverbal nominals and
%the causative alternation.  It originates from a mention in  \citet{Chomsky70a}, and is developed in
%detail by  \citet{Marantz97a}; see also  \citet{Pesetsky96a-u} and  \citet{HN2000a}.  The argument is
%often repeated, but it turns out that the empirical basis of the argument is incorrect, and the
%actual facts point in the opposite direction, in favor of lexical argument structure
%\citep{Wechsler2008b, Wechsler2008a}.

一些英语致使变换式动词允许施事论元的选择性省略(\ref{grow1}),但是同源名词却不允许施事论元的出现(\ref{growth1}):
%Certain English causative alternation verbs allow optional omission of the agent argument  (\ref{grow1}), while the cognate nominal disallows expression of the agent (\ref{growth1}):

\eal
\label{grow1}
\ex[]{
\gll that John grows tomatoes\\
     \textsc{comp} John 种植 西红柿\\
\mytrans{John种植西红柿这件事}
%that John grows tomatoes
}
\ex[]{
\gll that tomatoes grow\\
     \textsc{comp} 西红柿 生长\\
\mytrans{西红柿生长这件事}
%that tomatoes grow
}
\zl

\eal
\label{growth1}
\ex[*]{
\gll John's growth of tomatoes\\
     John.\textsc{poss} 种植 \textsc{prep} 西红柿\\
%John's growth of tomatoes
}
\ex[]{
\gll the tomatoes' growth, the growth of the tomatoes\\
     \textsc{det} 西红柿.\textsc{poss} 生长 \textsc{det} 生长 \textsc{prep} \textsc{det} 西红柿\\
\mytrans{西红柿的生长,西红柿的生长}
%the tomatoes' growth, the growth of the tomatoes
}
\zl
%
相反,从强制性不及物动词派生的名词,如destroy(摧毁),却允许施事论元出现,见(\ref{destruc1}a):
%In contrast, nominals derived from obligatorily transitive verbs such as \emph{destroy} allow expression of the agent, as shown in (\ref{destruc1}a):  

\eal
\label{destroy1}
\ex[]{
\gll that the army destroyed the city\\
     \textsc{comp} \textsc{det} 军队 破坏 \textsc{det} 城市\\
\mytrans{军队破坏这座城市这件事}
%that the army destroyed the city
}
\ex[*]{
\gll that the city destroyed\\
     \textsc{comp} \textsc{det} 城市 破坏\\
%that the city destroyed
}
\zl

\eal
\label{destruc1}
\ex[]{
\gll the army's destruction of the city\\
     \textsc{det} 军队.\textsc{poss} 破坏 \textsc{prep} \textsc{det} 城市\\
\mytrans{军队破坏这座城市}
%the army's destruction of the city
}
\ex[]{
\gll the city's destruction\\
     \textsc{det} 城市.\textsc{poss} 破坏\\
\mytrans{城市的破坏}
%the city's destruction
}
\zl

\noindent
沿用 \citet{Chomsky70a}的观点, \citet{Marantz97a}基于这些现象认为词项中不包含施事角色。在(\ref{grow1})和(\ref{destroy1})所示的动词性投射中,施事角色是在句法层面上由小动词(小v)指派的。(\ref{growth1})和(\ref{destruc1})所示的名词性投射缺少小动词。相反,语用因素决定哪种施事可以被领属短语所表达:领属短语可以表达“有外部致使而不是内部致使的事件的施事”,因为只有前者可以“被简单地重构”(引自 \citet[\page 218]{Marantz97a})。城市的毁坏有一个城市之外的致使因素,而西红柿生长是由西红柿内部致使的\citep{Smith70a-u}。Marantz指出,如果名词派生自一个其论元结构已经制定其施事论元的动词,这一解释就不成立,因为动转名词要从其致使变换动词承继施事论元。
%Following a suggestion by  \citet{Chomsky70a},  \citet{Marantz97a} argued on the basis of these data
%that the agent role is lacking from lexical entries. In verbal projections like (\ref{grow1}) and
%(\ref{destroy1}) the agent role is assigned in the syntax by little \emph{v}.  Nominal projections
%like (\ref{growth1}) and (\ref{destruc1}) lack little  \emph{v}.  Instead, pragmatics takes over to
%determine which agents can be expressed by the possessive phrase: the possessive can express ``the
%sort of agent implied by an event with an external rather than an internal cause'' because only the
%former can ``easily be reconstructed'' (quoted from  \citet[\page 218]{Marantz97a}).
%The destruction of a city has a cause external to the city, while the growth of tomatoes is
%internally caused by the tomatoes themselves \citep{Smith70a-u}.  Marantz points out that this
%explanation is unavailable if the noun is derived from a verb with an argument structure specifying
%its agent, since the deverbal nominal would inherit the agent of a causative alternation verb.   

这一论述的事实根据是推定的动转名词是否允许施事论元表达:动词,如grow(种植),允许施事论元表达,但是同源名词,如growth(种植),不允许。但是,事实证明,grow/growth同源对是很少见的。大多数动转名词在是否能允许施事论元出现方面与同源动词一样。另外,对于grow/growth这种例外情况,也有很好的解释\citep{Wechsler2008a}。首先考虑一下非—转换只有主题的不及物动词(非受格\isc{非受格}\is{verb!unaccusative}),如(\ref{arrive1})所示;以及非—转换及物动词,如(\ref{trans})所示。模式很清晰:如果动词没有施事论元,那么名词也没有:
%The empirical basis for this argument is the putative mismatch between the allowability of agent
%arguments, across some verb-noun cognate pairs: \eg \emph{grow} allows the agent but \emph{growth}
%does not.  But it turns out that the \emph{grow}/""\emph{growth} pattern is rare. Most deverbal nominals
%precisely parallel the cognate verb: if the verb has an agent, so does the noun.  Moreover, there is
%a ready explanation for the exceptional cases that exhibit the \emph{grow}/""\emph{growth} pattern
%\citep{Wechsler2008a}.  First consider non-alternating theme-only intransitives
%(unaccusatives\is{verb!unaccusative}), as in (\ref{arrive1}) and non-alternating transitives as in
%(\ref{trans}).  The pattern is clear: if the verb is agentless, then so is the noun:  

\begin{exe}\ex
\label{arrive1} 
arriv(al)、disappear(ance)、fall等:
%\emph{arriv(al), disappear(ance), fall} etc.:
\begin{xlist}[iv.]
\ex[]{
\gll A letter arrived.\\
     一 信 到达\\
\mytrans{一封信到达了}
%A letter arrived.
}
\ex[]{
\gll the arrival of the letter\\
     \textsc{det} 到达 \textsc{prep} \textsc{det} 信\\
\mytrans{信的到达}
%the arrival of the letter
}
\ex[*]{
\gll The mailman arrived a letter.\\
     \textsc{det} 邮递员 到达 一 信\\
%The mailman arrived a letter.
}
\ex[*]{
\gll the mailman's arrival of the letter\\
     \textsc{det} 邮递员.\textsc{poss} 到达 \textsc{prep} \textsc{det} 信\\
%the mailman's arrival of the letter
}
\zl

\begin{exe}\ex
\label{trans}
destroy/destruction、construct(ion)、creat(ion)、assign(ment)等:
%\emph{destroy/destruction, construct(ion), creat(ion), assign(ment)} etc.:
\begin{xlist}[iv.]
\ex 
\gll The army is destroying the city.\\
     \textsc{det} 军队 \textsc{aux} 破坏 \textsc{det} 城市\\
\mytrans{军队正在破坏这座城市}
%The army is destroying the city.
\ex 
\gll the army's destruction of the city\\
     \textsc{det} 军队.\textsc{poss} 破坏 \textsc{prep} \textsc{det} 城市\\
\mytrans{军队对这座城市的破坏}
%the army's destruction of the city
\zl

\noindent
这些现象支持了以下观点:名词承继了动词的词汇论元结构。对于反对词汇主义的学者来说,(\ref{arrive1}c)和(\ref{arrive1}d)的不合语法需要各自独立的解释。例如,按照Harley和Noyer\citeyear{HN2000a}的方案,词根ARRIVE的一个特征阻止它出现在小动词的环境中,而(\ref{arrive1}d)之所以不合法是因为到达这一事件的致使事件不容易从世界知识重建出来。在语言系统两个分离部分中的重复要在所有非—变换不及物和及物动词中复制,是很不合理的。
%This favors the view that the noun inherits the lexical argument structure of the verb.  For the
%anti-lexicalist, the badness of (\ref{arrive1}c) and (\ref{arrive1}d), respectively, would have to
%receive independent explanations.  For example, on Harley and Noyer's \citeyear{HN2000a} proposal,
%(\ref{arrive1}c) is disallowed because a feature of the root ARRIVE prevents it from appearing in
%the context of \emph{v}, but (\ref{arrive1}d) is instead ruled out because the cause of an event of
%arrival cannot be easily reconstructed from world knowledge.  This exact duplication in two separate
%components of the linguistic system would have to be replicated across all non-alternating
%intransitive and transitive verbs, a situation that is highly implausible.

现在开始论述致使变换动词,Marantz的论述基于没明确说明的概括:致使变换动词的同源名词(典型情况下)都缺少施事论元。但是,除了grow/growth这一案例之外,类似案例很少。除了grow(th)之外, \citet[examples (7c) and (8c)]{Chomsky70a}提到了两个体验者动词amuse(逗乐)和interest(使……产生兴趣):John amused (interested) the children with his stories(John讲故事逗乐了孩子)成立,而John's amusement (interest) of the children with his stories(John讲故事逗乐孩子这件事)不成立。 \citet{Rappaport83a-u}和 \citet{Dowty89b-u}指出这种现象可以从体方面得到独立的解释。像amusement(逗乐)和interest(使……感兴趣)这种动转体验者名词表示一种心理状态,对应的动词表示产生或导致这种状态产生的事件。这种结果名词不仅缺少施事论元而且缺少动词的所有事件论元,因为它们不指称事件。如果这些名词可以理解为表示一种事件的话,施事论元就可以表达了。
%Turning to causative alternation verbs, Marantz's argument is based on the implicit generalization
%that noun cognates of causative alternation verbs (typically) lack the agent argument.  But apart
%from the one example of \emph{grow/growth}, there do not seem to be any clear cases of this pattern.
%Besides \emph{grow(th)},  \citet[examples (7c) and (8c)]{Chomsky70a} cited two experiencer
%predicates, \emph{amuse} and \emph{interest}:  \emph{John amused (interested) the children with his
%  stories}  versus  \emph{*\,John's amusement (interest) of the children with his stories}.   But this
%was later shown by  \citet{Rappaport83a-u} and  \citet{Dowty89b-u} to have an
%independent aspectual explanation.  Deverbal experiencer nouns like \emph{amusement} and
%\emph{interest} typically denote a mental state, where the corresponding verb denotes an event in
%which such a mental state comes about or is caused.   These result nominals lack not only the agent
%but all the eventive arguments of the verb, because they do not refer to events.  Exactly to the
%extent that such nouns can be construed as representing events, expression of the agent becomes
%acceptable.   
为了回应 \citew{Chomsky70a}时,Carlota Smith(\citeyear{Smith72a-u})调查了Webster的词典,但是并没有发现支持Chomsky观点的例证,Chomsky的观点是:动转名词并不从致使变换动词承继施事论元。她列出了很多反例,包括:explode(爆炸)、divide(分开)、accelerate(加速)、expand(扩展)、neutralize(中和)、conclude(得出结论)、unify(统一)等,\citep[\page 137]{Smith72a-u}。Harley和Noyer(\citeyear{HN2000a})也注意到很多“例外”,包括explode(爆炸)、accumulate(堆积)、separate(分离)、unify(统一)、disperse(散布)、transform(转换)、dissolve/dissolution(溶解)、detach(ment)(拆开)、disengage-(ment)(分开)等。事实是,这些现象不是例外。因为并不存在一种概括,使得这些现象可以称之为例外。这一长串动词代表了一种常规,特别是通过加后缀形成的名词(后缀包括\suffix{tion}、\suffix{ment}等)。很多从变换动词派生的名词也允许施事表达,例如change、release和use。例如,my constant change of mentors from 1992--1997(从1991到1997年我频繁换顾问)、the frequent release of the prisoners by the governor(政府频繁释放囚犯)和the frequent use of sharp tools by underage children(未成年儿童频繁使用尖锐工具)(例子引自 \citet[fn.\,13]{Borer2003a-u})。\footnote{\citet[\page 79, ex.~(231)]{Pesetsky96a-u}给句子the thief's return of the money(盗贼归还了钱)标记了星号,但是对于很多人来说这句话是可以接受的。《牛津英语词典》为名词return(归还)列出了一个及物意义(义项11a)并且像her return of the spoils(她归还了赃物)这种句子在语料库中也不难找到。}   
%In a response to  \citew{Chomsky70a}, Carlota Smith (\citeyear{Smith72a-u}) surveyed
%Webster's dictionary and found no support for Chomsky's claim that deverbal nominals do not inherit
%agent arguments from causative alternation verbs.  She listed many counterexamples, including
%``\emph{explode}, \emph{divide}, \emph{accelerate}, \emph{expand}, \emph{repeat}, \emph{neutralize}, \emph{conclude}, \emph{unify}, and so on at
%length.'' \citep[\page 137]{Smith72a-u}.  Harley and Noyer (\citeyear{HN2000a}) also noted many so-called
%``exceptions'':  \emph{explode, accumulate, separate, unify, disperse, transform,
%  dissolve/dissolution, detach(ment), disengage-(ment)}, and so on.  The simple fact is that these are not
%exceptions because there is no generalization to which they can be exceptions.  These long lists of
%verbs represent the norm, especially for suffix-derived nominals (in \suffix{tion}, \suffix{ment}, etc.).
%Many zero-derived nominals from alternating verbs also allow the agent, such as  \emph{change,
%  release}, and \emph{use}: \emph{my constant change of mentors from 1992--1997}; \emph{the frequent
%  release of the prisoners by the governor};  \emph{the frequent use of sharp tools by underage children}
%(examples from \citet[fn.\,13]{Borer2003a-u}).\footnote{\citet[\page 79, ex.~(231)]{Pesetsky96a-u} assigns a star
%to \emph{the thief's return of the money}, but it is acceptable to many speakers. The \emph{Oxford
%  English Dictionary} lists a transitive sense for the noun \emph{return} (definition 11a), and
%corpus examples like \emph{her return of the spoils} are not hard to find.}   

就像上面提及的体验者名词一样,很多零派生名词都缺少事件义。一些名词拒绝同源事件动词的所有论元,而不仅仅是施事论元:例如the freeze of the water(水结冰)和the break of the window(窗子打碎)都不成立。按照Stephen Wechsler的观点,his drop of the ball(他抛下球)有一点奇怪,the drop of the ball(球掉下来)同样奇怪。a drop in temperature(温度下降)和The temperature dropped(温度下降了)相互匹配。并且动词和名词形式都不允许施事论元:The storm
  dropped the temperature(风暴使温度降低)和the storm's drop of the temperature(风暴使得温度降低)都不成立。简而言之,事实与再三提及的反对词汇价的论述正好相反。除了孤例grow/growth之外,所有表示事件的动转名词都承继动词的论元结构。
%Like the experiencer nouns mentioned above, many zero-derived nominals lack event readings.  Some
%reject all the arguments of the corresponding eventive verb, not just the agent: \emph{*\,the freeze
%  of the water}, \emph{*\,the break of the window}, and so on.  According to Stephen Wechsler,
%\emph{his drop of the ball} is slightly odd, but \emph{the drop of the ball} has exactly the same
%degree of oddness.  The locution \emph{a drop in temperature} matches the verbal one \emph{The
%  temperature dropped}, and both verbal and nominal forms disallow the agent: \emph{*\,The storm
%  dropped the temperature.} \emph{*\,the storm's drop of the temperature}.  In short, the facts seem to point
%in exactly the opposite direction from what has been assumed in this oft-repeated argument against
%lexical valence.  Apart from the one isolated case of \emph{grow/growth}, event-denoting deverbal
%nominals match their cognate verbs in their argument patterns.

现在讨论一下grow/growth本身,对于它们的非常规表现,我们找到了一个简单的解释\citep{Wechsler2008a}。当名词growth进入英语时,致使(及物)动词grow还不存在。《牛津英语词典》提供了grow和growth的实例最早出现的时间:
%Turning to \emph{grow/growth} itself, we find a simple explanation for its unusual behavior \citep{Wechsler2008a}.  When the noun \emph{growth} entered the English language,  causative (transitive)  \emph{grow} did not exist.  The OED provides these dates of the earliest attestations of \emph{grow} and \emph{growth}:	

\ea
\label{oed}
\begin{tabular}[t]{@{}l@{~}lrl@{}} 
a. & 不及物动词 grow: &  c725	& “翠绿的”\ldots{} “生长” (不及物)\\
b. & 名词 growth:   &  1587	& “生长”(不及物)\\
c. & 及物动词 grow:   &  1774	& “种植”(庄稼)\\
%a. & intransitive \emph{grow}: &  c725	& `be verdant' \ldots{} `increase' (intransitive)\\
%b. & the noun \emph{growth}:   &  1587	& `increase' (intransitive)\\
%c. & transitive \emph{grow}:   &  1774	& `cultivate (crops)'\\
\end{tabular}
\z

\noindent
因此,growth进入语言时,及物动词grow还不存在。其论元结构和语义承继自其来源动词,并且保留在现代英语中。如果正如我们主张的,动词有词汇论元结构的话,就非常有意义。因为在英语中通过加-th后缀进行名词化并不具有能产性,所以growth应该放在词库中。为了解释growth缺少施事论元这一现象,我们只需要假设一个词项的谓词性论元结构决定其是否能携带一个施事论元。所以,即便是这一现象也可以为词汇论元结构提供证据。
%Thus \emph{growth} entered the language at a time when transitive \emph{grow} did not exist. The argument structure and meaning were inherited by the noun from its source verb, and then preserved into present-day English.  This makes perfect sense if, as we claim, words have predicate argument structures.  Nominalization by \emph{-th} suffixation is not productive in English, so \emph{growth} is listed in the lexicon.  To explain why \emph{growth} lacks the agent we need only assume that a lexical entry's predicate argument structure dictates whether it takes an agent argument or not.   So even this one word provides evidence for lexical argument structure.  

\subsection{特异性句法选择}
\label{sec-idiosyncratic-case-and-PP}

%As was mentioned at the beginning of this section, proponents of so-called neo-constructivist
%approaches assume that roots are stored in the lexicon and connected to encyclopedic knowledge that
%helps to determine which arguments may be or have to be present. The arguments are licensed by
%functional projections that may contribute meaning to the core meaning contributed by the
%root or in Haugereid's proposal by binary branching ID schemata that license an argument that fills
%one of five argument roles. 

词汇价结构概念可以直接解释为什么论元实现模式与选择这些论元的特定词汇中心语密切相关。如果词项中不包含价信息,而依靠句法或者世界知识是不足以决定论元实现的,因为并非所有的关联模式都由意义决定。介宾结构的介词形式有时跟意义有松散的关系,有时候跟动词意义的关系是任意的。例如,英语动词depend(依靠)的价结构表明它选择一个带on的PP来表达它的一个语义论元:
%The notion of lexical valence structure immediately explains why the argument realization patterns
%are strongly correlated with the particular lexical heads selecting those arguments.  
%It is not sufficient to have general lexical items without valence information
%and let the syntax and world knowledge decide about argument realizations, 
%We show that such
%approaches are not sufficient for describing language in total and that 
%The concept of valence is needed, 
%because not all 
%realizational patterns are determined by the meaning. 
%The form of the preposition of a prepositional object is sometimes loosely semantically motivated but in
%other cases arbitrary.  For example, the valence structure of the English verb \emph{depend} captures the fact that it selects an \emph{on}-PP to express one of its semantic arguments: 

\eal\label{depends-on-ex}
\ex 
\gll John depends on Mary.  (\emph{counts, relies,} etc.)\\
     John 依靠 \textsc{prep} Mary(指望、仰仗等)\\
\mytrans{John依赖Mary。}
%John depends on Mary.  (\emph{counts, relies,} etc.)
\ex 
\gll John trusts (*on) Mary.\\
     John 相信 \textsc{prep} Mary\\
\mytrans{John相信Mary。} 
%John trusts (*on) Mary.  
\ex 
\ms{
phon & \phonliste{ depend }\\[1mm]
arg-st & \liste{ NP$_x$, PP[\type{on}]$_y$ }\\[2mm]
content  & depend\textrm{(}x,y\textrm{)}\\ 
}
\zl
这种特异性词汇选择在人类语言中是完全普遍存在的。动词或其它谓项经常决定是选择直接形态还是旁格形态,对于旁格来说,又可以决定是选择介词还是其它旁格。在一些语言(例如,冰岛语)中,即使是主语的格也由动词决定 \citep*{ZMT85a}。
%Such idiosyncratic lexical selection is utterly pervasive in human language.  The verb or other
%predicator often determines the choice between direct and oblique morphology, and for obliques, it
%determines the choice of adposition or oblique case.  In some languages such as Icelandic even the
%subject case can be selected by the verb \citep*{ZMT85a}.

论元选择是由特定语言决定的。英语的动词wait(等)选择for(德语的für),而德语的动词warten(等待)选择auf(即英语的on),然后再加一个受格宾语。
%Selection is language-specific.  English \emph{wait} selects \emph{for} (German \emph{für}) while German \emph{warten} selects \emph{auf} `on' with an accusative object:
\eal \label{loureed}
\ex 
\gll I am waiting for my man.\\
     我 \textsc{aux} 等待 \textsc{prep} 我的 男人\\
\mytrans{我正在等待我的男人。}
%I am waiting for my man.
\ex 
\gll Ich warte auf meinen Mann.\\
     我   等待  \textsc{prep}  我的     男人.\acc\\
\mytrans{我等待我的男人。}
%\gll Ich warte auf meinen Mann.\\
%     I   wait  on  my     man.\acc\\
\zl
%A learner has to acquire that \emph{warten}
%has to be used with \emph{auf} + accusative and not with other prepositions or other
%case. 
通常不可能找到格的语义动因。在德语中有一种倾向,用(\mex{1}b)所示的与格代替(\mex{1}a)所示的领属格,但是这种倾向没有明显的语义动因。
%It is often impossible to find semantic motivation for case.  In German there is a
%tendency to replace genitive (\mex{1}a) with dative (\mex{1}b) with no apparent semantic motivation:  
%Instead of the genitive as in
%(\mex{1}a) one also finds examples with the dative as in (\mex{1}b):
\eal
\ex 
\gll dass der Opfer gedacht werde\\
     \textsc{comp} \textsc{det} 受害者.\gen{} 铭记 \textsc{aux}\\
\mytrans{受害者将会被铭记}
%\gll dass der Opfer gedacht werde\\
%     that the victims.\gen{} remembered is\\
%\mytrans{that the victims would be remembered}
\ex 
\gll daß auch hier den Opfern des Faschismus gedacht werde [\ldots]\footnotemark\\
     \textsc{comp} 也 这里 \textsc{det} 受害者.\dat{} \textsc{det} 法西斯 铭记 \textsc{aux}\\
\mytrans{法西斯的受害者在这里也将会被铭记}
\footnotetext{%
《法兰克福评论报》(Frankfurter Rundschau),\zhdate{1997/11/07},第6页。
}
%\gll daß auch hier den Opfern des Faschismus gedacht werde [\ldots]\footnotemark\\
%     that also here the victims.\dat{} of.the fascism remembered is\\
%\mytrans{that the victims of fascism would be remembered here too}
%\footnotetext{%
%Frankfurter Rundschau, 07.11.1997, p.\,6.
%}
\zl
近义词treffen和begegnen(去见)约束不同的格(例子引自 \citet[\page 126]{ps})。
%The synonyms \emph{treffen} and \emph{begegnen} `to meet' govern different cases (example from  \citet[\page 126]{ps}).
%\eal
%\ex 
%\gll Er unterstützt ihn.\\
%     he supports him.\acc\\
%\ex 
%\gll Er hilft ihm.\\
%     he helps him.\dat{}\\
%\zl
\eal
\ex 
\gll Er traf den Mann.\\
     他.\nom{} 见 \textsc{det}.\acc{} 男人\\
\mytrans{他见了这个男人。}
%\gll Er traf den Mann.\\
%     he.\nom{} met the.\acc{} man\\
\ex 
\gll Er begegnete dem Mann.\\
     他.\nom{} 见 \textsc{det}.\dat{} 男人\\
\mytrans{他见了这个男人。}
%\gll Er begegnete dem Mann.\\
%     he.\nom{} met the.\dat{} man\\
\zl
%Similarly, \emph{helfen} `to help' governs dative while and \emph{unterstützen} `to support' governs accusative.
%In order to avoid that the verb \emph{helfen} appears in the syntactic environment that licenses (\mex{0}a)
%and that the verb \emph{unterstützen} appears in the construction that licenses (\mex{0}b), 
这就需要在动词的词项中指定相应动词的格要求。\footnote{%
  或者至少要说明在德语中,treffen宾语的格就是通常指派给宾语的缺省格,而begegnen的宾语是与格。参看 \citew{Haider85b}、 \citew{HM94a}和 \citew{Mueller2001a}对结构和词汇格的论述。.
}
%One has to specify the case that the respective verbs require in the lexical items of the verbs.\footnote{%
 % Or at least mark the fact that \emph{treffen} takes an object with the default case for
  %objects and \emph{begegnen} takes a dative object in German. See  \citew{Haider85b},  \citew{HM94a}, and
  % \citew{Mueller2001a} on structural and lexical case.
%}
% PS87: S. 126
% Wem begegneten Sie / Wen trafen Sie? -> nicht auf Semantik reduzierbar
%
%Without any semantic motivation one verb takes an accusative object and the other one takes a dative.

 \citet{Haugereid2009a}曾提出一种激进的插入方法。\footnote{%
  Haugereid方法的技术方面将在~\ref{Abschnitt-Diskussion-Haugereid}讨论。
}Haugereid(第12--13页)认为句法决定动词与五种不同的论元角色中的一种或几种任意组合。哪些论元可以与动词组合并非由动词的词项决定。\footnote{%
  Haugereid 可以对动词进行价限制,但是他声称他用这种方法只是为了让自己的计算程序运作更有效(第13页)。
}
这种观点的一个问题是一个多义动词的意义有时会取决于它的哪些论元实现了。德语动词borgen有两个意义“借入”和“借出”,两者是同一事件的两种不同叙述角度(参看 \citew{Kunze91,Kunze93}对于领属关系交换动词的更多的论述)。值得注意的是,只有“借出”意义强制要求要有一个与格宾语。\citep[\page403]{MuellerGTBuch1}:
%A radical variant of the plugging approach is suggested by  \citet{Haugereid2009a}.\footnote{%
 % Technical aspects of Haugereid's approach are discussed in Section~\ref{Abschnitt-Diskussion-Haugereid}.
%} Haugereid
%(pages\,12--13) assumes that the syntax combines a verb with an arbitrary combination of a subset of
%five different argument roles. Which arguments can be combined with a verb is not restricted by the
%lexical item of the verb.\footnote{%
 % Haugereid has the possibility to impose valence restrictions on verbs, but he claims that he uses
 % this possibility just in order to get a more efficient processing of his computer implementation (p.\,13).
%}
%\settowidth\jamwidth{(Max, 4;9)}
%A problem for such views is that the meaning of an ambiguous verb sometimes depends on which of its
%arguments are expressed.  The German verb \emph{borgen} has the two translations `borrow' and
%`lend', which basically are two different perspectives on the same event (see
% \citew{Kunze91,Kunze93} for an extensive discussion of verbs of exchange of possession).
%Interestingly, the dative object is obligatory only with the `lend' reading \citep[\page
%  403]{MuellerGTBuch1}: 
\eal
\ex 
\gll Ich borge ihm das Eichhörnchen.\\
     我   借  他 \textsc{det} 松鼠皮\\
\mytrans{我借松鼠皮给他。}
%\gll Ich borge ihm das Eichhörnchen.\\
%     I   lend  him the squirrel\\
%\mytrans{I lend the squirrel to him.}
\ex 
\gll Ich borge (mir) das Eichhörnchen.\\
     我 借 \hspaceThis{(}我 \textsc{det} 松鼠皮\\
\mytrans{我借了这张松鼠皮。}
%\gll Ich borge (mir) das Eichhörnchen.\\
%     I borrow \hspaceThis{(}me the squirrel\\
%\mytrans{I borrow the squirrel.}
\zl
如果忽略这一点,我们就只能得到“借入”义。所以语法必须为特定动词指定一个特定动词意义或者对事件的特定角度需要特定的论元。
%If we omit it, we get only the `borrow' reading. 
%So, instead of (\ref{max-lend-borrow}), Max should have
%uttered (\mex{1}a) or (\mex{1}b):
%\eal
%\ex 
%\gll Ich verspreche dir, das niemandem zu borgen.\\
%     I promise you it nobody to lend\\
%\ex 
%\gll Ich verspreche dir, das nicht zu verborgen.\\
%     I promise you it not to lend.out\\
%\zl
%It follows that all theories have to have a place where it is fixed that
%So the grammar must specify for specific verbs that certain arguments are
%necessary for a certain verb meaning or a certain perspective on an event.

具有不同价实现模式的近义词包括前面说过的最小对比三元组:dine是强制不及物动词(或者带一个on-PP),devour是一个及物动词,而eat可以用作及物动词也可以用作不及物动词\citep[\page 89--90]{Dowty89b-u}。更多例子可以参看 \citet{Levin93a-u}和 \citet{LRH2005a-u}。
%Synonyms with differing valence specifications include the minimal triplet mentioned earlier: \emph{dine} is obligatorily intransitive (or takes an \emph{on-}PP), \emph{devour} is transitive, and \emph{eat} can be used either intransitively or transitively \citep[\page 89--90]{Dowty89b-u}.  Many other examples are given in   \citet{Levin93a-u} and  \citet{LRH2005a-u}.
%Here, we have another example of different valence
%frames, without there being any possibility to reduce this to a semantic contrast with regard to the
%core meaning of the involved predicates: all three involve an eating frame.
%
%The problem of the argument realization in the triplet \emph{dine}, \emph{devour}, and \emph{eat}
%and other examples by  \citet{LRH2005a-u} that show that certain arguments are obligatory is sometimes noted in the literature, but they are simply ignored. 

按照短语构式主义方法,必须假设带有介词或格的短语模式,并允许英语动词插入其中。对于(\ref{loureed}b),模式包括一个带auf的介词宾语和一个受格NP,以及一个warten词项说明该词可以插入到这一结构中(参看 \citew[\S~5.2]{KJ85a}是如何在TAG框架中实现这一方案的)。因为就带有这种价表征的动词有很多的概括,所以需要有两个承继层级:一个是带有价属性的词项,另一个是具体短语模式,这些词项可以插入的具体构式需要这些具体短语模式。
%In a phrasal constructionist approach one would have to assume
%phrasal patterns with the preposition or case, into which the verb is inserted.  For (\ref{loureed}b), the pattern includes a prepositional object with \emph{auf} and an
%accusative NP, plus an entry for \emph{warten} specifying that it can be inserted into such a structure (see  \citew[Section~5.2]{KJ85a} for such a proposal in the framework of TAG). Since there are 
%generalizations regarding verbs with such valence representations, one would be forced
%to have two inheritance hierarchies: one for lexical entries with their valence properties and
%another one for specific phrasal patterns that are needed for the specific constructions in which
%these lexical items can be used.  

更多的时候,新-构式主义方法的支持者或者提出了很难与词汇价结构区分的方案(参看下面~\ref{sec-borer})或者简单地拒绝解决这一问题。例如, \citet{Lohndal2012a}写道:
%More often, proponents of neo-constructionist approaches either 
%make proposals that are difficult to distinguish from lexical valence structures (see Section~\ref{sec-borer} below)
%or simply decline to address the problem.  For instance,  \citet{Lohndal2012a} writes:
\begin{quotation}
这一理论中未回答的一个问题是,我们如何确保功能中心语与相关词项或词根出现在一起。对于格由功能中心语指派这一观点来说,这是一个宽泛的问题,而这里我们对此什么也不想说。\citep{Lohndal2012a}\footnote{%
An unanswered question on this story is how we ensure that the functional heads occur together with
the relevant lexical items or roots. This is a general problem for the view that Case is assigned by
functional heads, and I do not have anything to say about this issue here.
}
% p.\,18
\end{quotation}
我们认为能够在简单句中保证正确的格指派,而不产生大量过度概括或者不合法的词序,是对语言学理论的最低要求。
%We think that 
%this view is  inadequate given 
%the current state of linguistics.  
%getting case assignment
%right in simple sentences, without vast overgeneration of ill-formed word sequences, is a minimal
%requirement for a linguistic theory.  % that is asked to be taken seriously.

\subsection{虚位}
\label{sec-expletives}

最后一个用于说明价与语义之间非还原性的例子是德语中选择虚位和固有自反动词中自反论元的动词:
%A final example for the irreducibility of valence to semantics are verbs that select for expletives
%and reflexive arguments of inherently reflexive verbs in German:
\eal
\ex 
\gll weil es regnet\\
     因为 \expl{} 下雨\\
%\gll weil es regnet\\
%     because it rains\\
\ex 
\gll weil (es) mir (vor der Prüfung) graut\\
     因为 \hspaceThis{(}\expl{} 我.\dat{} \hspaceThis{(}在之前 \textsc{det} 考试 恐惧\\\\
\mytrans{因为我很害怕考试。}
%\gll weil (es) mir (vor der Prüfung) graut\\
%     because \hspaceThis{(}\textsc{expl} me.\dat{} \hspaceThis{(}before the exam dreads\\
%\mytrans{because I am dreading the exam}
\ex\label{ex-zum-Professor}
\gll weil er es bis zum Professor bringt\\
     因为 他 \textsc{expl} 直到 \textsc{prep}.\textsc{det} 教授 带来\\
\mytrans{因为他赶上了教授。}
%\gll weil er es bis zum Professor bringt\\
%     because he \textsc{expl} until to.the professor brings\\
%\mytrans{because he made it to professor}
\ex 
\gll weil es sich um den Montag handelt\\
     因为 \textsc{expl} \textsc{refl} \textsc{prep} \textsc{det} 周一 交易\\
\mytrans{因为这是关于周一的。}
%\gll weil es sich um den Montag handelt\\
%     because \textsc{expl} \textsc{refl} around the Monday trades\\
%\mytrans{because it is about the Monday}
\ex 
\gll weil ich mich (jetzt) erhole\\
     因为 我 REFL \hspaceThis{(}现在 再创造\\
\mytrans{因为我正在放松自己。}
%\gll weil ich mich (jetzt) erhole\\
%     because I myself \hspaceThis{(}now recreate\\
%\mytrans{because I am relaxing now}
\zl
(\mex{0})中的词汇中心语需要包含没有论元角色的虚位主语/宾语或反身代词。注意德语允许无主语谓词,所以虚位主语的出现并不遵循普遍原则。(\ref{ex-zum-Professor})是带有一个虚位宾语的例子。适用于主语强制出现的解释对于这种例子一定不适用。另外,必须确保erholen不能出现在[Sbj IntrVerb]结构中(该结构是最简方案为不及物动词或相应功能范畴设立的),尽管\relation{erholen}(\relation{relax})是一个一元动词,由此,erholen在语义上与这种构式兼容。
%The lexical heads in (\mex{0}) need to contain information about the expletive subjects/""objects and/""or
%reflexive pronouns that do not fill semantic roles. Note that German allows for subjectless
%predicates and hence the presence of expletive subjects cannot be claimed to follow from general
%principles. (\ref{ex-zum-Professor}) is an example with an expletive object. Explanations referring
%to the obligatory presence of a subject would fail on such examples in any case. Furthermore it has
%to be ensured that \emph{erholen} is not realized in the [Sbj IntrVerb] construction for
%intransitive verbs or respective functional categories in a Minimalist setting although the relation
%\relation{erholen} (\relation{relax}) is a one-place predicate and hence \emph{erholen} is
%semantically compatible with the construction.  

% Note also that the psychological predicate \emph{fürchten} `to dread' is semantically similar to
% \emph{grauen} `to dread'. A grammar has to account for the fact that neither verb can be used in a
% different frame:
% \eal
% \ex[]{
% Ich fürchte mich (vor der Prüfung).
% }
% \ex[*]{
% Mir fürchtet (es) (vor der Prüfung).
% }
% Das gibt es ...
% \ex[]{
% Ich graue mich (vor der Prüfung).
% }
% \zl

\subsection{一种外骨架方法}
\label{Abschnitt-Diskussion-Haugereid}

%% During the past years the phrasal analyses that were common in GPSG are reappearing in several
%% frameworks (almost all versions of Construction Grammar, some versions of LFG, Simpler Syntax). This
%% section discusses an extreme variant of HPSG, namely one that assumes that lexical items do not contain
%% valence information \citep{Haugereid2007a}. Conceptionally, such approaches are much nearer to Borer's exoskeletal approach
%% \citeyearpar{Borer2005a-u} than to HPSG, which is a strongly lexicallized theory.
%% There are many high-level

下面我将更加详细地讨论Haugereid的方案\citeyearpar{Haugereid2007a}。他的分析有上文提到的所有高层次的问题,但是因为他的方法比较详细地实现了,所以了解其假说还是很有价值的。
%In what follows I discuss Haugereid's proposal \citeyearpar{Haugereid2007a} in more detail. His analysis has all the
%high"=level problems that were mentioned in the previous subsections, but since it is worked out in
%detail it is interesting to see its predictions.

\mbox{} \citet{Haugereid2007a}在HPSG框架内,延续 \citet{Borer2005a-u}的思路提出了一种分析,在这种分析中,一个表达的意义取决于出现的论元。他认为有五种论元槽可以被指派意义角色\isc{语义角色}\is{semantic role},语义角色如下:\todostefan{Das gibt irgendwie einen Bruch, weil das so Kleinkram ist.}
%\mbox{} \citet{Haugereid2007a}, working in the framework of HPSG, suggests an analysis along the lines of  \citet{Borer2005a-u} where the meaning of an expression is defined as depending
%on the arguments that are present. He assumes that there are five argument slots that are assigned to semantic roles\is{semantic role}
%as follows:\todostefan{Das gibt irgendwie einen Bruch, weil das so Kleinkram ist.}
\begin{itemize}
\item Arg1:施事或来源
\item Arg2:受事
\item Arg3:受益者或接受者
\item Arg4:目标
\item Arg5:先行语
%\item Arg1: agent or source
%\item Arg2: patient
%\item Arg3: benefactive or recipient
%\item Arg4: goal
%\item Arg5: antecedent
\end{itemize}
在这里,先行语是一个更加概括的语义角色,代表工具、伴随格、方式和来源。Arg1--Arg3对应着主语和宾语。Arg4是路径终点的结果谓词。Arg4可以实现为PP、AP或者NP。(\mex{1})给出了Arg4实现的例子:
%Here, antecedent is a more general role that stands for instrument, comitative, manner and source.
%The roles Arg1--Arg3 correspond to subject and objects. Arg4 is a resultative predicate of the end of a path.
%Arg4 can be realized by a PP, an AP or an NP. (\mex{1})
%gives examples for the realization of Arg4:
\eal
\ex 
\gll John smashed the ball \emph{out} \emph{of} \emph{the} \emph{room}.\\
     John 猛击 \textsc{det} 球 \textsc{prep} \textsc{prep} \textsc{det} 房间\\
\mytrans{John将球扣出了房间。}
%John smashed the ball \emph{out of the room}.
\ex 
\gll John hammered the metal \emph{flat}.\\
     John 锤击 \textsc{det} 铁 平\\
\mytrans{John将铁打平了。}
%John hammered the metal \emph{flat}.
\ex 
\gll He painted the car \emph{a} \emph{brilliant} \emph{red}.\\
     他 喷漆 \textsc{det} 车 一 亮 红\\
\mytrans{他将车子喷成亮红色。}
%He painted the car \emph{a brilliant red}.
\zl
虽然Arg4在事件致使链中位于其它参与者之后,但是先行语在事件序列中先于受事。该角色实现为PP。(\mex{1})是Arg5实现的例子:
%Whereas Arg4 follows the other participants in the causal chain of events, the antecedent precedes the patient in the order of
%events. It is realized as a PP.
%(\mex{1}) is an example of the realization of Arg5:
\ea
\gll John punctured the balloon \emph{with} \emph{a} \emph{needle}.\\
     他 刺穿 \textsc{det} 气球 \textsc{prep} 一 针\\
\mytrans{他用针将这个气球刺破了。}
%John punctured the balloon \emph{with a needle}.
\z

\noindent
Haugereid认为论元框架包含这些语义角色。他给出了(\mex{1})中的例子:
%Haugereid now assumes that argument frames consist of these roles. He provides the examples in 
%(\mex{1}):
\eal
\settowidth\jamwidth{(arg12345-frame)}
\ex 
\gll John smiles.           \\
     John 笑\\\jambox{(arg1-frame)}
\mytrans{John笑。}
%John smiles.           \jambox{(arg1-frame)}
\ex 
\gll John smashed the ball. \\
     John 打碎 \textsc{det} 球\\\jambox{(arg12-frame)}
\mytrans{John打碎了这只球。}
%John smashed the ball. \jambox{(arg12-frame)}
\ex 
\gll The boat arrived.      \\
     \textsc{det} 小船 到达\\\jambox{(arg2-frame)}
\mytrans{这只小船到达了。}
%The boat arrived.      \jambox{(arg2-frame)}
\ex
\gll John gave Mary a book. \\
     John 给 Mary 一 书\\\jambox{(arg123-frame)}
\mytrans{John给Mary一本书。}
%John gave Mary a book. \jambox{(arg123-frame)}
\ex 
\gll John gave a book to Mary.\\
     John 给 一 书 \textsc{prep} Mary\\ \jambox{(arg124-frame)}
\mytrans{John把一本书给了Mary。}
%John gave a book to Mary. \jambox{(arg124-frame)}
\ex
\gll John punctured the ball with a needle. \\
     John 刺破 \textsc{det} 球 \textsc{prep} 一 针\\\jambox{(arg125-frame)}
\mytrans{John用针把这只球刺破了。} 
%John punctured the ball with a needle. \jambox{(arg125-frame)}
\zl

\noindent
Haugereid指出多个动词可以出现在多个论元框架中。他就动词drip(使滴下)提供了如(\mex{1})所示的例子:
%Haugereid points out that multiple verbs can occur in multiple argument frames. He provides the variants in (\mex{1})
%for the verb \emph{drip}:
\eal
\settowidth\jamwidth{(arg12345-frame)}
\ex
\gll The roof drips.                    \\
     \textsc{det} 屋顶 滴下\\\jambox{(arg1-frame)}
\mytrans{屋顶在滴水。} 
%The roof drips.                    \jambox{(arg1-frame)}
\ex
\gll The doctor drips into the eyes.    \\
     \textsc{det} 医生 滴下 \textsc{prep} \textsc{det} 眼睛\\\jambox{(arg14-frame)}
\mytrans{医生将水滴到眼睛里。} 
%The doctor drips into the eyes.    \jambox{(arg14-frame)}
\ex
\gll The doctor drips with water.       \\
     \textsc{det} 医生 滴下 \textsc{prep} 水\\\jambox{(arg15-frame)}
\mytrans{医生用水滴。} 
%The doctor drips with water.       \jambox{(arg15-frame)}
\ex
\gll The doctor drips into the eyes with water. \\
     \textsc{det} 医生 滴下 \textsc{prep} \textsc{det} 眼睛 \textsc{prep} 水\\\jambox{(arg145-frame)}
\mytrans{医生用水往眼睛里滴。} 
%The doctor drips into the eyes with water. \jambox{(arg145-frame)}
\ex
\gll The roof drips water.                      \\
     \textsc{det} 屋顶 滴下 水\\\jambox{(arg12-frame)}
\mytrans{屋顶往下滴水。} 
%The roof drips water.                      \jambox{(arg12-frame)}
\ex
\gll The roof drips water into the bucket.      \\
     \textsc{det} 屋顶 滴下 水 \textsc{prep} \textsc{det} 桶\\\jambox{(arg124-frame)}
\mytrans{屋顶往桶里滴水。} 
%The roof drips water into the bucket.      \jambox{(arg124-frame)}
\ex
\gll The doctor dripped the eyes with water.    \\
     \textsc{det} 医生 滴下 \textsc{det} 眼睛 \textsc{prep} 水\\\jambox{(arg125-frame)}
\mytrans{医生用水滴眼睛。}
%The doctor dripped the eyes with water.    \jambox{(arg125-frame)}
\ex
\gll The doctor dripped into the eyes with water. \\
     \textsc{det} 医生 滴下 \textsc{prep} \textsc{det} 眼睛 \textsc{prep} 水\\\jambox{(arg145-frame)}
\mytrans{医生用水往眼睛里滴。} 
%The doctor dripped into the eyes with water. \jambox{(arg145-frame)}
\ex
\gll John dripped himself two drops of water.     \\
     John 滴下 REFL 两 滴 \textsc{prep} 水\\\jambox{(arg123-frame)}
\mytrans{John给自己滴了两滴水。} 
%John dripped himself two drops of water.     \jambox{(arg123-frame)}
\ex
\gll John dripped himself two drops of water into his eyes. \\
     John 滴下 REFL 两 滴 \textsc{prep} 水 \textsc{prep} 他的 眼睛\\\jambox{(arg1234-frame)}
\mytrans{John给自己往眼睛里滴了两滴水。} 
%John dripped himself two drops of water into his eyes. \jambox{(arg1234-frame)}
\ex
\gll John dripped himself two drops of water into his eyes with a drop counter. \jambox{(arg12345-frame)}\\
     John 滴下 REFL 两 滴 \textsc{prep} 水 \textsc{prep} 他的 眼睛 \textsc{prep} 一 滴 计数器\\
\mytrans{John用滴计数器给自己往眼睛里滴了两滴水。} 
%John dripped himself two drops of water into his eyes with a drop counter. \jambox{(arg12345-frame)}
\ex
\gll Water dripped. \\
     水 滴下\\\jambox{(arg2-frame)}
\mytrans{水滴下了。} 
%Water dripped. \jambox{(arg2-frame)}
\ex
\gll It drips. \\
     \expl{} 滴下\\\jambox{(arg0-frame)}
\mytrans{在滴水。} 
%It drips. \jambox{(arg0-frame)}
\zl
他提出了图~\ref{Abbildung-Haugereid}所示的承继层级来表征所有可能的论元组合。出于空间考虑Arg5这一语义角色省略了。
%He proposes the inheritance hierarchy in Figure~\ref{Abbildung-Haugereid} in order to represent all
%possible argument combinations. The Arg5 role is omitted due to space considerations.

\begin{figure}
\oneline{%
\begin{tabular}{@{}cccccccc@{}}
\multicolumn{8}{c}{\mynode{link}{link}}\\[6ex]
\mynode{arg1p}{arg1$+$} & \mynode{arg4p}{arg4$+$} & \mynode{arg2p}{arg2$+$} & \mynode{arg3p}{arg3$+$} & \mynode{arg3m}{arg3$-$} & \mynode{arg4m}{arg4$-$} & \mynode{arg1m}{arg1$-$} & \mynode{arg2m}{arg2$-$}\\[8ex]
\mynode{arg12123124}{arg12-123-124} & \mynode{arg12124224}{arg12-124-2-24} & \mynode{arg112}{arg1-12} & \mynode{arg1223}{arg12-23} & \mynode{arg02}{arg0-2}\\[8ex]
\mynode{arg124}{arg124} & \mynode{arg123}{arg123} & \mynode{arg12}{arg12} & \mynode{arg24}{arg24} & \mynode{arg1}{arg1} & \mynode{arg2}{arg2} & \mynode{arg23}{arg23} & \mynode{arg0}{arg0}\\
\end{tabular}
% todo put south and north into a style
\begin{tikzpicture}[overlay,remember picture,shorten <=2pt,shorten >=2pt] 
\draw (link.south)--(arg1p.north)
(link.south)--(arg4p.north)
(link.south)--(arg2p.north)
(link.south)--(arg3p.north)
(link.south)--(arg3m.north)
(link.south)--(arg4m.north)
(link.south)--(arg1m.north)
(link.south)--(arg2m.north)
(arg1p.south)--(arg12123124.north)
(arg1p.south)--(arg112.north)
(arg4p.south)--(arg124.north)
(arg4p.south)--(arg24.north)
(arg2p.south)--(arg12123124.north)
(arg2p.south)--(arg12124224.north)
(arg2p.south)--(arg1223.north)
(arg3p.south)--(arg123.north)
(arg3p.south)--(arg23.north)
(arg3m.south)--(arg12124224.north)
(arg3m.south)--(arg112.north)
(arg3m.south)--(arg24.north)
(arg3m.south)--(arg02.north)
(arg4m.south)--(arg112.north)
(arg4m.south)--(arg1223.north)
(arg4m.south)--(arg02.north)
(arg1m.south)--(arg24.north)
(arg1m.south)--(arg02.north)
(arg1m.south)--(arg23.north)
(arg2m.south)--(arg1.north)
(arg2m.south)--(arg0.north)
(arg12123124.south)--(arg124.north)
(arg12123124.south)--(arg123.north)
(arg12123124.south)--(arg12.north)
(arg12124224.south)--(arg124.north)
(arg12124224.south)--(arg12.north)
(arg12124224.south)--(arg24.north)
(arg12124224.south)--(arg2.north)
(arg112.south)--(arg1.north)
(arg112.south)--(arg12.north)
(arg1223.south)--(arg12.north)
(arg1223.south)--(arg23.north)
(arg02.south)--(arg0.north)
(arg02.south)--(arg2.north);
\end{tikzpicture}

}
\caption{\label{Abbildung-Haugereid} \citet{Haugereid2007a}的论元框架层级}
%\caption{\label{Abbildung-Haugereid}Hierarchy of argument frames following  \citet{Haugereid2007a}}
\end{figure}%
Haugereid假设了二叉\isc{分支!二叉}\is{branching!binary}结构,在二叉结构中论元可以以任意顺序与中心语组合。对于每一个论元角色都有一个统制图式。实现论元角色~3的图式提供了一个连接取值\type{arg3$+$}。如果另外一个图式提供论元角色~2,那么就可以得到\type{arg23}。对于非作格不及物动词来说,可以确定它有论元框架\type{arg1}。这一框架只与\type{arg1$+$}、\type{arg2$-$}、\type{arg3$-$}和\type{arg4$-$}兼容。按照Haugereid的观点,具有可选宾语的动词被指派\type{arg1-12}。这一类型允许以下组合:\type{arg1$+$}、\type{arg2$+$}、\type{arg3$-$}和\type{arg4$-$}。
%Haugereid assumes binary"=branching\is{branching!binary} structures where arguments can be combined with a head in any order.
%There is a dominance schema for each argument role. The schema realizing the argument role~3 provides a link value \type{arg3$+$}.
%If the argument role~2 is provided by another schema, we arrive at the frame
%\type{arg23}. For unergative intransitive verbs, it is possible to determine that it has an argument frame of 
%\type{arg1}. This frame is only compatible with the types \type{arg1$+$}, \type{arg2$-$},
%\type{arg3$-$} and \type{arg4$-$}. Verbs that have an optional object are assigned to \type{arg1-12} according to Haugereid.
%This type allows for the following combinations: \type{arg1$+$}, \type{arg2$-$},
%\type{arg3$-$} and \type{arg4$-$} such as \type{arg1$+$}, \type{arg2$+$}, \type{arg3$-$} and \type{arg4$-$}.

这一方法与Goldberg的观点非常相似:动词就它们能出现的句子结构是不完全指定的,并且只有在句子中论元的实际实现才能决定哪些论元组合能实现。需要注意的是,图~\ref{Abbildung-Haugereid}中的层级对应着大量析取:它列出了论元所有组合模式。如果我们说essen(吃)有\type{arg1-12},那么这对应着析取\type{arg1}\,~$\vee$\,\type{arg12}。除了层级上的信息之外,还需要论元句法属性方面的信息(格、介词的形式、动词性补语中动词的形式)。因为这些信息一部分是由每一个动词决定的(参看~\ref{Abschnitt-Stoepselei}),这些信息无法在统制图式中表征,只能在每一个词项中列举。对于词项warten auf(等待),必须要有以下信息:主语是一个NP而介词宾语是一个带有受格的auf-PP。使用类型层级可以很简洁地表示以下事实:介词宾语是可有可无的。这种方法与(\mex{1})所示的析取指定的SUBCAT列表仅仅是形式上的差异。
%This approach comes very close to an idea by Goldberg: verbs are underspecified with regard to the sentence structures in which they occur and
%it is only the actual realization of arguments in the sentence that decides which combinations of arguments are realized.
%One should bear in mind that the hierarchy in Figure~\ref{Abbildung-Haugereid} corresponds to a considerable disjunction:
%it lists all possible realizations of arguments. If we say that \emph{essen} `to eat' has the type \type{arg1-12}, then this
%corresponds to the disjunction \type{arg1}\,~$\vee$\,\type{arg12}. In addition to the information in the hierarchy above, one also requires information about the syntactic properties of
%the arguments (case, the form of prepositions, verb forms in verbal complements). Since this information is in part specific to each verb
%(see Section~\ref{Abschnitt-Stoepselei}), it cannot be present in the dominance schemata and must instead be listed in each individual
%lexical entry. For the lexical entry for \emph{warten auf} `wait for', there must be information about the fact that the subject has to be an
%NP and that the prepositional object is an \emph{auf}-PP with accusative. The use of a type hierarchy then allows one to elegantly encode
%the fact that the prepositional object is optional. The difference to a disjunctively specified
%\subcatl with the form of (\mex{1}) is just a matter of formalization.
\ea
\subcat \sliste{ NP[\str] } $\vee$ \sliste{ NP[\str], PP[\type{auf}, \type{acc}] }
\z
%
因为Haugereid假设的结构是二叉的,所以可以派生出论元的所有序列,见(\mex{1}a--b);附接语可以附加到任意分支节点上,见(\mex{1}c--d)。
%Since Haugereid's structures are binary"=branching, it is possible to derive all permutations of arguments (\mex{1}a--b), and adjuncts can be
%attached to every branching node (\mex{1}c--d). 
\eal
\ex 
\gll dass [\sub{arg1} keiner [\sub{arg2} Pizza isst]]\\
     \textsc{comp} {} 没人 {} 披萨 吃\\
\mytrans{没人吃披萨这件事}
%\gll dass [\sub{arg1} keiner [\sub{arg2} Pizza isst]]\\
%     that {} nobody {} pizza eats\\
%\mytrans{that nobody eats pizza}
\ex 
\gll dass [\sub{arg2} Pizza [\sub{arg1} keiner isst]]\\
	 \textsc{comp} {} 披萨 {} 没人 吃\\
%\gll dass [\sub{arg2} Pizza [\sub{arg1} keiner isst]]\\
%	 that {} pizza {} nobody eats\\
\ex 
\gll dass [\sub{arg1} keiner [gerne [\sub{arg2} Pizza isst]]]\\
	 \textsc{comp} {} 没人 \spacebr{}高兴地 {} 披萨 吃\\
\mytrans{没人会高兴地吃披萨这件事}
%\gll dass [\sub{arg1} keiner [gerne [\sub{arg2} Pizza isst]]]\\
%	 that {} nobody \spacebr{}gladly {} pizza eats\\
%\mytrans{that nobody eats pizza gladly}
\ex 
\gll dass [[hier              [\sub{arg1} keiner [\sub{arg2} Pizza isst]]]\\
     \textsc{comp} \hspaceThis{[[}这里 {}          没人 {} 披萨 吃\\
\mytrans{这里没人吃披萨这件事}
%\gll dass [[hier              [\sub{arg1} keiner [\sub{arg2} Pizza isst]]]\\
%     that \hspaceThis{[[}here {}          nobody {} pizza eats\\
%\mytrans{that nobody eats pizza here}
\zl
所以,Haugereid解决了一些 \citew{Mueller2006d}指出的Goldberg分析中存在的问题。但是,他的分析中有大量其它问题,我会在下面讨论。在Haugereid的分析中,没有提到意义的组合问题。他表示遵循所谓的新"=戴维斯方法。按照这种方法,动词的论元并不直接在动词上表征。相反,动词通常有一个事件论元,并且属于这一事件的论元角色由一个独立的谓词决定。(\mex{1})展示了两种其它的方法,其中e代表事件变量。
%Haugereid has therefore found solutions for some of the problems in Goldberg's analysis that were
%pointed out in  \citew{Mueller2006d}.
%Nevertheless, there are a number of other problems, which I will discuss in what follows.
%In Haugereid's approach, nothing is said about the composition of meaning. He follows the so"=called Neo"=Davidsonian\is{Neo-Davidsonian semantics} approach.
%In this kind of semantic representation, arguments of the verb are not directly represented on the verb.
%Instead, the verb normally has an event argument and the argument roles belonging to the event in question are determined in a separate predication.
%(\mex{1}) shows two alternative representations, where \emph{e} stands for the event variable.
\eal
\ex 
\gll Der Mann isst eine Pizza.\\
	 \textsc{det} 男人 吃 一 披萨\\
\mytrans{这个男人正在吃披萨。}
%\gll Der Mann isst eine Pizza.\\
%	 the man eats a pizza\\
%\mytrans{The man is eating a pizza}
\ex \relation{eat}(e, x, y) $\wedge$ \relation{man}(x) $\wedge$ \relation{pizza}(y)
\ex \relation{eat}(e) $\wedge$ \type{agent}(e,x) $\wedge$ \type{theme}(e,y) $\wedge$ \relation{man}(x) $\wedge$ \relation{pizza}(y)
\zl
Haugereid采用了最小递归语义(MRS)\indexmrsc 作为其语义形式化表征手段(也可以参看~\ref{Abschnitt-HPSG-Semantik}和~\ref{Abschnitt-leere-Elemente-Semantik})。论元属于一个特定谓词这一事实通过相关谓词有相同句柄来表征。(\mex{0}c)的表征对应于(\mex{1}):
%Haugereid adopts Minimal Recursion Semantics (MRS)\indexmrs as his semantic formalism (see also Section~\ref{Abschnitt-HPSG-Semantik} and~\ref{Abschnitt-leere-Elemente-Semantik}). 
%The fact that arguments belong to a particular predicate is represented by the fact that the relevant predicates have the same handle. The representation in (\mex{0}c) corresponds
%to (\mex{1}):
\ea
h1:\relation{essen}(e), h1:\type{arg1}(x), h1:\type{arg2}(y), h2:\relation{mann}(x), h3:\relation{pizza}(y)
\z
这一分析概括了Goldberg的主要观点,意义来源于与一个中心语共同实现的特定成分。
%This analysis captures Goldberg's main idea: meaning arises from particular constituents being realized together with a head.

对于\isc{提升|(}\is{raising|(} (\mex{1}a)中的句子,Haugereid(2007, p.\,c.)给出如(\mex{1}b)所示的语义表征:\footnote{%
  参看 \citew[\page 165]{Haugereid2009a}来了解对(i)中挪威语例子的分析。
\ea
\gll Jon maler veggen rød.\\
     Jon 刷 墙.\defsc{} 红色\\
\mytrans{Jon把墙刷成红色。}
%\gll Jon maler veggen rød.\\
%     Jon paints wall.\defsc{} red\\
%\mytrans{Jon paints the wall red.}
\zlast
}\isc{构式!结果构式|(}\is{construction!resultative|(}
%For\is{raising|(} the sentence in (\mex{1}a), Haugereid (2007, p.\,c.) assumes the semantic representation in (\mex{1}b):\footnote{%
%  See  \citew[\page 165]{Haugereid2009a} for an analysis of the Norwegian examples in (i).
%\ea
%\gll Jon maler veggen rød.\\
%     Jon paints wall.\defsc{} red\\
%\mytrans{Jon paints the wall red.}
%\zlast
%}\is{construction!resultative|(}
\eal
\ex 
\gll der Mann den Teich leer fischt\\
	 \textsc{det} 男人 \textsc{det} 池塘 空 钓鱼\\
%\gll der Mann den Teich leer fischt\\
%	 the man the pond empty fishes\\
\ex h1:\relation{mann}(x), h2:\relation{teich}(y), h3:\relation{leer}(e),\\
    h4:\relation{fischen}(e2), h4:\type{arg1}(x), h4:\type{arg2}(y), h4:\type{arg4}(h3)
\zl
在(\mex{0}b)中,\type{arg1}、\type{arg2}和\type{arg4}关系有相同的句柄\relation{fischen}。按照Haugereid的定义,这表示\type{arg2}是事件的受事。在(\mex{0}a)中,这带来了错误的预测,因为受格成分并不是主要动词的语义论元。它是次级谓词leer(空)的语义论元并且被提升到结果构式的宾语位置。按照Haugereid的分析,受格宾语要么是动词要么是形容词的句法论元,但是它绝不可能是动词的语义论元。除了这一问题外,(\mex{0}b)中的表征并没有表示leer(空)可以支配宾语。Haugereid(2007, p.c.)表示这一点隐含在表征当中并且遵循所有\type{arg4}s支配\type{arg2}s的事实。与Haugereid的分析不同,使用词汇规则连接动词词项与带有结果义的词项的分析可以精确指定概括谓词之间关系的语义表征。另外,基于词汇规则的分析可以允准不在受事宾语和动词之间建立联系的词项(\citealp{Wechsler97a,WN2001a};\citealp[Chapter~5]{Mueller2002b})。\isc{提升|)}\is{raising|)}\isc{构式!结果构式|)}\is{construction!resultative|)}
%In (\mex{0}b), the \type{arg1}, \type{arg2} and \type{arg4} relations have the same handle as \relation{fischen}. 
%Following Haugereid's definitions, this means that \type{arg2} is the patient of the event. In the case of
%(\mex{0}a), this makes incorrect predictions since the accusative element is not a semantic argument of the main
%verb. It is a semantic argument of the secondary predicate \emph{leer} `empty' and has been raised to the object
%of the resultative construction. Depending on the exact analysis assumed, the accusative object is either a syntactic
%argument of the verb or of the adjective, however, it is never a semantic argument of the verb. In addition to this problem,
%the representation in (\mex{0}b) does not capture the fact that \emph{leer} `empty' predicates over the object. Haugereid (2007, p.c.) suggests
%that this is implicit in the representation and follows from the fact that all \type{arg4}s predicate over all \type{arg2}s.
%Unlike Haugereid's analysis, analyses using lexical rules that relate a lexical item of a verb to
%another verbal item with a resultative meaning allow for a precise specification of the semantic representation
%that then captures the semantic relation between the predicates involved. In addition, the lexical
%rule"=based analysis makes it possible to license lexical items  that do not establish a semantic relation between the accusative object and the verb
%(\citealp{Wechsler97a,WN2001a};
%\citealp[Chapter~5]{Mueller2002b}).\is{raising|)}\is{construction!resultative|)}

Haugereid大致分析了一下德语的句法并且解决了主动/被动变换问题。但是语法的某些方面并没有具体地阐述。尤其是,没有说明包含AcI动词,如sehen(看)和lassen(出租),的复杂句怎样分析。在这些结构中,嵌套和被嵌套动词的论元可以交替。 Haugereid(2007, p.\,c.)假设了特殊的规则允许渗透嵌套更深的动词,例如一条特殊规则可以让一个论元的\type{arg2}论元与动词组合。例如,为了能够在(\mex{1})中的句子中将das Nilpferd和nicht füttern helfen lässt组合起来,他假设了一条规则将双层嵌套动词的论元与另外一个动词进行组合:
%Haugereid sketches an analysis of the syntax of the German clause and tackles active/passive alternations.
%However, certain aspects of the grammar are not elaborated on. In particular, it remains unclear how complex clauses containing AcI verbs such as
%\emph{sehen} `to see' and \emph{lassen} `to let' should be analyzed. Arguments of embedded and embedding verbs can be permuted in
%these constructions. Haugereid (2007, p.\,c.) assumes special rules that allow to saturate arguments of more deeply embedded verbs, for example,
%a special rule that combines an \type{arg2} argument of an argument with a verb. In order to combine \emph{das Nilpferd} and \emph{nicht füttern helfen
%  lässt} in sentences such as (\mex{1}), he is forced to assume a special grammatical rule that combines an argument of a doubly embedded verb with another verb:
\ea
\label{ex-nilpferd-fuettern-helfen-laesst}
\gll weil    Hans Cecilia John das Nilpferd nicht füttern helfen lässt\\
     因为 Hans Cecilia John \textsc{det} 河马 \textsc{neg} 喂 帮助 让\\
\mytrans{因为Hans不让Cecilia帮助John喂养河马。}
%\gll weil    Hans Cecilia John das Nilpferd nicht füttern helfen lässt\\
%     because Hans Cecilia John the hippo not feed help let\\
%\mytrans{because Hans is not letting Cecilia help John feed the hippo.}
\z
在 \citet[\page 220]{Mueller2004b}中,我曾经指出过在复杂形式谓词中的嵌套只是受到语言运用\isc{语言运用}\is{performance}因素的约束(也可以参看~\ref{Abschnitt-Kompetenz-Performanz-TAG})。在德语中,包含四个以上动词的动词复杂体\isc{动词复杂体}\is{verbal complex}很少见。 \citet[\page 58--59]{Evers75a}指出,在荷兰语\il{荷兰语}\il{Dutch}中情况有所不同,因为荷兰语的动词复杂体的句法分支不同于德语:在荷兰语中,包含五个以上动词的动词复杂体是存在的。Evers将这种差异归结于德语动词复杂体需要更大的处理负担(也可以参看\citealp[\S~3.7]{Gibson98a})。Haugereid不得不假设荷兰语比德语有更多的规则。按照这一方式,他就会放弃语言运用与语言能力之间的差异并且将语言运用限制直接纳入到语法中。如果他要保持两者的差异,就需要假设无限的图式或者具有功能不确定性\isc{功能不确定性}\is{functional uncertainty}的规则,因为嵌套深度只由语言运用因素限制\citep{HN94a}。因为不管怎样,对于宾语提升来说需要这种提升分析(正如上面所讨论的),我们需要对这些问题予以重视。
%In  \citet[\page 220]{Mueller2004b}, I have argued that embedding under complex"=forming predicates is only constrained by performance\is{performance} factors
%(see also Section~\ref{Abschnitt-Kompetenz-Performanz-TAG}). In German, verbal complexes\is{verbal complex} with more than four verbs are barely acceptable.
% \citet[\page 58--59]{Evers75a} has pointed out, however, that the situation in Dutch\il{Dutch} is different since Dutch verbal complexes have a different branching:
%in Dutch, verbal complexes with up to five verbs are possible. Evers attributes this difference to a greater processing load for German verbal complexes
%(see also \citealp[Section~3.7]{Gibson98a}). Haugereid would have to assume that there are more rules
%for Dutch than for German. In this way, he would give up the distinction between competence and
%performance and incorporate performance restrictions directly into the grammar. If he wanted to maintain a distinction between the two, then Haugereid would
%be forced to assume an infinite number of schemata or a schema with functional uncertainty\is{functional uncertainty} since depth of embedding is only
%constrained by performance factors. Existing HPSG approaches to the analysis of verbal complexes do
%without functional uncertainty \citep{HN94a}.
%Since such raising analyses are required for object raising anyway (as discussed above), they should be given preference.

总结来说,我们必须承认Haugereid的外骨架方法解决了论元不同排序的问题,但是他没有得出正确的语义表征也不能解决论元特异性选择和虚位选择的问题。
%Summing up, it must be said that Haugereid's exoskeletal approach does account for different
%orderings of arguments, but it neither derives the correct semantic representations nor does it offer a
%solution for the problem of idiosyncratic selection of arguments and the selection of expletives.

\subsection{有没有词汇价结构的替代方法?}
\label{sec-borer}

反对价结构的理论存在的问题在于其替代理论是如何解释特异性的词汇选择的。 \citet{Borer2005a-u}在她的外骨架方法中明确反对词汇价结构。但是她设置了后-句法解释规则,而这些规则很难与词价结构区分。为了解释depend(依靠)与一个带on-PP之间的联系,她设置了以下解释规则\citep[Vol.\ II, p.\,29]{Borer2005a-u}:
%The question for theories denying the existence of valence structure is what replaces it to explain
%idiosyncratic lexical selection.  In her exoskeletal approach,  \citet{Borer2005a-u} explicitly
%rejects lexical valence structures.  But she posits post-syntactic interpretive rules that are
%difficult to distinguish from them.  To explain the correlation of \emph{depend} with an
%\emph{on}-PP, she posits the following interpretive rule \citep[Vol.\ II, p.\,29]{Borer2005a-u}:

\ea
MEANING $\Leftrightarrow$ $\pi_9 + [ \langle e^{on} \rangle ]$  
\z
Borer将所有这些特异性选择案例当做习语。在如(\mex{0})所示的规则中,“意义是相关习语的意义”\citep[Vol.\ II, p.\,27]{Borer2005a-u}。在(\mex{0})中,$\pi_9$是动词depend(依靠)的“音系索引”,而$e^{on}$“对应着一个由f-morph on指派范围的开放值”\citep[Vol.\ II, p.\,29]{Borer2005a-u},f-morphs是功能词或语素。所以这种规则跟(\ref{depends-on-ex}c)所示的词汇价结构表达了相同的信息。在讨论这种“习语”规则时,Borer写道:
%Borer refers to all such cases of idiosyncratic selection as idioms.  In a rule such as (\mex{0}),
%``MEANING is whatever the relevant idiom means'' \citep[Vol.\ II, p.\,27]{Borer2005a-u}.  In (\mex{0}),
%$\pi_9$ is the ``phonological index'' of the verb \emph{depend} and $e^{on}$ ``corresponds to an open
%value that must be assigned range by the f-morph \emph{on}'' \citep[Vol.\ II, p.\,29]{Borer2005a-u}, where f-morphs are function
%words or morphemes.  Hence this rule brings together much the same information as the lexical
%valence structure in (\ref{depends-on-ex}c).  Discussing such ``idiom'' rules, Borer writes  

\begin{quotation}
虽然按照假设一个例项不能与任何语法属性发生关联,但是本研究中的一项工具可以让我们解决这一假设施加在语法上的约束,这种工具是构成习语。 [\ldots] 这些习语性指定不仅能潜在地用于arrive(到达)和depend on(依赖),对于强制性及物动词[\ldots],对于强制带处所的动词,例如put(放置),以及带句子补语的动词都有用。

读者可能会反对说次范畴化在这里偷偷地引入进来了,通过引入所谓的“习语”(idiom)来代替词汇句法表示方法,习语实际上做了与词汇句法表示法相同的工作。当然这种反对也有一定道理,在本文的当前状态,引入习语代表着一种程度上的迁就。
  \\ \citep[Vol. II, p.\,354--355]{Borer2005a-u}\footnote{%
Although by assumption a listeme cannot be associated with any grammatical properties, one device used in this work has allowed us to get around the formidable restrictions placed on the grammar by such a constraint\,--\,the formation of idioms.  [\ldots] 
Potentially, then, within the system developed here, any syntactic or morphological property which does not reduce directly to some formal computational principle is to be captured by classifying the relevant item as an idiom\,--\,a partial representation of a phonological index with some functional value. \ldots 
Such idiomatic specification could be utilized, potentially, not just for \emph{arrive} and \emph{depend on}, but also for obligatorily transitive verbs [\ldots], for verbs such as \emph{put}, with their obligatory locative, and for verbs which require a sentential complement.

The reader may object that subcategorization, of sorts, is introduced here through the back door, with the introduction, in lieu of lexical syntactic annotation, of an articulated listed structure, called an \emph{idiom}, which accomplishes, de facto, the same task.  The objection of course has some validity, and at the present state of the art, the introduction of idioms may represent somewhat of a concession. 
\ldots  On the positive side, we note that to the extent that the existence of idioms is costly, we have attempted to put in place here a system which at least potentially extricates from the costly component of language all properties of listemes which are otherwise derivable from the structure.
}
\end{quotation}
Borer继续提出了很多问题留待以后研究,这些工作与限制可能的习语的种类有关。就这一研究模式而言,应该注意的是:词汇主义者的研究主要焦点在于缩小次范畴化的类别并且将能够推导的属性从次范畴化中提取出来。那些就是HPSG词汇层级的功能。
%Borer goes on to pose various questions for future research, related to constraining the class of
%possible idioms.   With regard to that research program it should be noted that a major focus of lexicalist research has been narrowing the class of subcategorization and extricating derivable properties from idiosyncratic subcategorization.  Those
%are the functions of HPSG lexical hierarchies, for example.  
%Whether future research within the
%exoskeletal approach can improve upon the past research in the lexical approach remains to be seen.
%\NOTE{SW:This is a bit obnoxious but I couldn't figure out how else to say it.}  
%
%The valence structure is more explicit about the linking between argument roles and complements, but this linking must be assumed on either theory.  The lexical theory of complement selection is very well-developed and well-understood, and we are unaware of any problems with it that Borer's alternative addresses.\footnote{There are some differences.  The phonological index $\pi$ is not itself a phonological representation such as a sequence of phonemes or a phonological feature matrix, but rather an abstract pointer to the phonological representation of the word.  Supposing for the sake of argument that the structure of that phonological representation is irrelevant to the rules of complement selection, then one way to capture that (hypothetical) generalization is to use the phonological index.  Another way is to posit actual phonological structure in the rule but exclude conditions relating phonological structure to complement selection from the grammar.   In any case, this issue is orthogonal to the question of lexical valence structure because either approach is consistent with the assumption of lexical valence structure, since one could posit either a phonological representation or a phonological index as the value of \textsc{phon}.}  

\subsection{小结}
\label{sec-underspec-summary}

在\ref{idiom-asym}--\ref{sec-expletives}中,我们展示了哪种论元可以在句子中实现不能归结于语义和世界知识或者主语的概括事实。结果就是价信息必须与词项联系起来。所以,要么按照\citeyearpar{Croft2003a}提出的方法或者LTAG的方法假设词项和特定短语结构之间存在关系,要么假设词汇理论。在最简方案里,设置正确的特征集合必须在词汇层面上以保证格指派正确的功能中心语。这与我们这里主张的词汇价结构相似,只不过这种方法会不必要地引入很多问题,如\ref{coordination-sec}提出的并列问题。
%In Sections~\ref{idiom-asym}--\ref{sec-expletives} we showed that the question of which
%arguments must be realized in a sentence cannot be reduced to semantics and world knowledge or to
%general facts about subjects. The consequence is that valence information has to be connected to
%lexical items. One therefore must either assume a connection between a lexical item and a certain phrasal
%configuration as in Croft's approach \citeyearpar{Croft2003a} and in LTAG or assume our lexical
%variant. In a Minimalist setting the right set of features must be specified lexically to
%ensure the presence of the right case assigning functional heads. This is basically %equivalent 
%similar to the lexical valence structures we are proposing here, except that it needlessly introduces  
%various problems discussed above, such as the problem of coordination raised in Section~\ref{coordination-sec}. 

\section{构式之间的关系}
\label{relations-sec}
按照词汇规则方法,词形式是通过词汇规则联系起来的:一个动词词干可以与一个带有定式屈折的动词和动词被动形式联系在一起;动词可以转换为形容词或名词;等等。词汇论元结构记录在词项中并且可以通过词汇规则进行调整。在这一节中,我将考察一下在短语或论元结构构式方法理论框架中什么方法可以替代词汇规则。
%On the lexical rules approach, word forms are related by lexical rules: a verb
%stem can be related to a verb with finite inflection and to a passive verb form; verbs can be converted
%to adjectives or nouns; and so on.  The lexical argument structure accompanies the word and can be manipulated by the lexical rule.  
%In Section~\ref{lex-deriv-sec} we briefly review this approach and the classic arguments for lexicalism that motivate it.  
%In this section we consider what can replace such rules within a phrasal or ASC approach.  

\subsection{构式之间的承继层级}
\label{inheritance-sec}
\label{Abschnitt-Croft}

对于词汇主义者将其与词根例项(及物,双及物等)联系的每一个价结构,短语方法都需要多个短语构式,一个短语构式代替一个词汇规则或者代替多个词汇规则的组合。以双及物结构为例,短语方法需要一个主动-双及物构式、一个被动-双及物构式等来代替每个适用于双及物动词的每条词汇规则或者词汇规则的组合。(所以 \citew[\page 169--170]{BC2005a}假设了一个主动-双及物构式和一个被动-双及物构式, \citew[\page 171--172]{KO2012a}假设了及物构式的主动和被动变体。)按照这一观点,德语的一些主动构式会是:
%For each valence structure that the lexicalist associates with a root lexeme (transitive, ditransitive, etc.), 
%the phrasal approach requires multiple phrasal constructions, one to replace each lexical rule or combination of lexical rules that can apply to the word.  
%Taking ditransitives, for example, the phrasal approach requires an active-ditransitive construction, a passive-ditransitive construction, and 
%so on, to replace the output of every lexical rule or combination of lexical rules applied to a ditransitive verb.  
%(Thus  \citew[\page 169--170]{BC2005a} assume an active-ditransitive and a
%passive-ditransitive construction and  \citew[\page 171--172]{KO2012a} assume active and passive
%variants of the transitive construction.)  On that view some of the active voice constructions for German would be:

\eal
\label{ex-active-valence}
\ex {}Nom V
\ex {}Nom Acc V
\ex {}Nom Dat V
\ex {}Nom Dat Acc V
\zl 
与(\mex{0})对应的被动构式是:
%The passive voice constructions corresponding to (\mex{0}) would be:
\eal
\label{ex-passive-valence}
\ex {}V V-Aux
\ex {}Nom V V-Aux
\ex {}Dat V V-Aux
\ex {}Dat Nom V V-Aux
\zl  

\noindent
仅仅列举所有这些构式不仅不经济而且不能表示出主动构式和被动构式之间明显的系统性关系。因为短语主义者反对词汇规则和转换,所以他们需要借助另外的方式来将构式关联起来并且借以反应主动和被动之间的常规关系。迄今为止,涉及到这一点的只有承继层级的使用,所以让我们考察一下它们。
%Merely listing all these constructions is not only uneconomical but fails to capture the obvious
%systematic relation between active and passive constructions.  Since phrasalists reject both lexical rules and transformations, they need an alternative way to relate phrasal configurations and thereby explain the regular relation between active and passive.  
%The only proposals to date involve the use of inheritance hierarchies, so let us examine them.

在不同理论框架中工作的研究者,包括支持词汇主义和短语主义的学者,都尽力发展出基于承继的方法来表示(\mex{-1})和(\mex{0})所示的价模式之间的关系(如可以参考\citealp[\page 12]{KF99a}、\citealp[\S~4]{MR2001a}、\citealp{Candito96a}、\citealp[\page 188]{CK2003a-u}、\citealp[\page 171--172]{KO2012a}、\citealp[\S~3]{Koenig99a}和\citealp{DK2000b-u,Kordoni2001b-u}在构式语法、范畴语法和HPSG理论中提出的方法)。他们的观点是单个表征(词汇或短语,到底是什么取决于具体的理论)可以从多个构式承继特征。按照短语的方法,(\mex{-1}b)中的模式可以从及物和主动构式承继特征;(\mex{0}b)中的模式可以从及物和被动构式承继特征。图~\vref{fig-passive-inheritance}展示了基于承继的词汇方法:像read(读书)和eat(吃饭)这种动词词项可以与主动或被动表征组合。主动和被动表征分别负责论元表达。
%Researchers working in various frameworks, both with lexical and phrasal orientation, have tried to develop inheritance-based analyses that could
%capture the relation between valence patterns such as those in (\mex{-1}) and (\mex{0}) (see for instance
%\citealp[\page 12]{KF99a}; \citealp[Chapter~4]{MR2001a};
%\citealp{Candito96a}; \citealp[\page 188]{CK2003a-u}; \citealp[\page 171--172]{KO2012a};
%\citealp[Chapter~3]{Koenig99a}; \citealp{DK2000b-u,Kordoni2001b-u} for proposals in CxG, TAG, and HPSG).  The idea
%is that a single representation (lexical or phrasal, depending on the theory) can inherit properties from multiple constructions.  
%So \emph{She hammered the metal flat} inherits from the resultative construction, \emph{The metal was hammered} inherits from the passive construction, and \emph{The metal was hammered flat} inherits from both constructions.  
%In a phrasal approach the description of the pattern in (\mex{-1}b) inherits from the transitive and
%the active construction and the description of (\mex{0}b) inherits from both the transitive and the
%passive constructions.  Figure~\vref{fig-passive-inheritance} illustrates the inheritance"=based
%lexical approach: a lexical entry for a verb such as \emph{read} or \emph{eat} is combined with either an active
%or passive representation. The respective representations for the active and passive are responsible
%for the expression of the arguments. 
\begin{figure}
\centering
\begin{forest}
typehierarchy
[lexeme, for descendants={l sep+=5mm}
  [passive,name=passive, [passive $\wedge$ read, name=pr]]
  [active, name=active,  [active $\wedge$  read,  name=ar]]
  [read,   name=read     [passive $\wedge$ eat,  name=pe, no edge]]
  [eat,    name=eat,     [active $\wedge$  eat,   name=ae]] ]
\draw (passive.south)--(pe.north)
      (active.south) --(ae.north)
      (read.south)   --(pr.north)
      (read.south)   --(ar.north)
      (eat.south)    --(pe.north);
\end{forest}
\caption{\label{fig-passive-inheritance}主动和被动的承继层级}
%\caption{\label{fig-passive-inheritance}Inheritance Hierarchy for active and passive}
\end{figure}%
%

正如我在\ref{sec-passive-bcg}指出的那样,基于承继的方法不能解释价的多重改变,例如被动和无人称构式,这种现象可见于立陶宛语\il{立陶宛语}\il{Lithuanian}\citep[Section~5]{Timberlake82a}、爱尔兰语\il{爱尔兰语}\il{Irish}\citep{Noonan94a}和土耳其语\il{土耳其语}\il{Turkish}\citep{Ozkaragoez86a}。Özkaragöz举出的土耳其语的例子重复写在这里,仍保留原有的标注方法,见(\ref{ex-double-passivization-two}):
%As was already discussed in Section~\ref{sec-passive-bcg}, inheritance"=based analyses cannot
%account for multiple changes in valence as for instance the combination of passive and impersonal
%construction that can be observed in languages like Lithuanian\il{Lithuanian}
%\citep[Section~5]{Timberlake82a}, Irish\il{Irish} \citep{Noonan94a}, and Turkish\il{Turkish}
%\citep{Ozkaragoez86a}. Özkaragöz's Turkish examples are repeated here with the original glossing as
%(\ref{ex-double-passivization-two}) for convenience:
\eal\label{ex-double-passivization-two}
\ex\label{ex-double-passivization-strangle-two}
\gll Bu şato-da boğ-ul-un-ur.\\
     \textsc{det} 城堡-\textsc{loc} 绞死-\textsc{pass}-\textsc{pass}-\textsc{aor}\\\hfill(Turkish)
\mytrans{某人在这座城堡中(被人)绞死了。}
%\gll Bu şato-da boğ-ul-un-ur.\\
%     this château-\textsc{loc} strangle-\textsc{pass}-\textsc{pass}-\textsc{aor}\\\hfill(Turkish)
%\mytrans{One is strangled (by one) in this château.}
\ex\label{ex-double-passivization-hit-two}
\gll Bu oda-da döv-ül-ün-ür.\\
     \textsc{det} 房间-\textsc{loc} 击打-\textsc{pass}-\textsc{pass}-\textsc{aor}\\
\mytrans{某人在这座房间例(被人)殴打。}
%\gll Bu oda-da döv-ül-ün-ür.\\
%     this room-\textsc{loc} hit-\textsc{pass}-\textsc{pass}-\textsc{aor}\\
%\mytrans{One is beaten (by one) in this room.}
\ex
\gll Harp-te vur-ul-un-ur.\\
     战争-\textsc{loc} 射杀-\textsc{pass}-\textsc{pass}-\textsc{aor}\\
\mytrans{某人在这场战争中被射杀了。}
%\gll Harp-te vur-ul-un-ur.\\
%     war-\textsc{loc} shoot-\textsc{pass}-\textsc{pass}-\textsc{aor}\\
%\mytrans{One is shot (by one) in war.}
\zl
\ref{sec-passive-bcg}中另外一个不能用承继方式解决的例子是土耳其语中的多重致使化。土耳其语允许双甚至三次致使化\citep[\page 146]{Lewis67a-u}:
%Another example from Section~\ref{sec-passive-bcg} that cannot be handled with inheritance is multiple causativization in
%Turkish. Turkish allows double and even triple causativization \citep[\page 146]{Lewis67a-u}: 
\ea
Öl-dür-t-tür-t- \hfill(Turkish)\\
`导致某人去导致某人去杀某人。' 
%`to cause somebody to cause somebody to kill somebody' 
\z 
对于这种现象,基于承继的分析不会起作用,因为多次承继同一种信息不会增加任何新的东西。 \citet{KN93a}在谈及派生形态学案例,如preprepreversion:前缀pre两次或三次地承继信息,但是并没有增加任何新信息。
%An inheritance"=based analysis would not work, since inheriting the same information several times
%does not add anything new.  \citet{KN93a} make the same point with respect to derivational morphology
%in cases like \emph{preprepreversion}: inheriting information about the prefix \prefix{pre} twice or
%more often, does not add anything.

所以,如果假设短语模型,那么唯一反映(\ref{ex-active-valence})与(\ref{ex-passive-valence})之间概括的方式就是假设类似于GPSG元规则的规则来将(\ref{ex-active-valence})和 (\ref{ex-passive-valence})中的构式联系起来。如果构式如LTAG那样在词汇层面上联系起来,那么相应的映射规则都是词汇规则。对于将LTAG与Goldberg的观点结合起来的方法,例如 \citet{KO2012a}所提出的那种,就必须要有扩展的树家族来反映有另外论元可能性并且确保正确形态形式插入相应的句法树。形态学规则会独立于派生动词性词项使用的句法结构。所以需要两种独立类型的规则:作用于句法树的GPSG式的元规则和作用于词干和词的形态学规则。我认为这是一种不必要的麻烦,而且除了麻烦之外,按照构式语法的定义这些形态学规则不会被认为是形式-意义偶对,因为形式的一方面(即需要另外的论元)没有反映在这些形态学规则中。如果这些形态学规则也可以被认为是合适的构式,那么就没有理由要求论元出现在构式中来让其可以被识别,那么词汇规则方法就可以接受了。\footnote{%
与下面(\ref{ex-tot-schiessen})中的对于Totschießen(射死)的讨论作对比。%
}
%So assuming phrasal models, the only way to capture the generalization with regard to (\ref{ex-active-valence}) and
%(\ref{ex-passive-valence}) seems to be to assume GPSG-like metarules that relate the constructions
%in (\ref{ex-active-valence}) to the ones in (\ref{ex-passive-valence}). If the constructions are
%lexically linked as in LTAG, the respective mapping rules would be lexical rules. For approaches
%that combine LTAG with the Goldbergian plugging idea such as the one by  \citet{KO2012a} one would have to
%have extended families of trees that reflect the possibility of having additional arguments and
%would have to make sure that the right morphological form is inserted into the respective trees. The
%morphological rules would be independent of the syntactic structures in which the derived verbal
%lexemes could be used. One would have to assume two independent types of rules: GPSG-like metarules
%that operate on trees and morphological rules that operate on stems and words. We believe that this
%is an unnecessary complication and apart from being complicated the morphological rules would not
%be acceptable as form-meaning pairs in the CxG sense since one aspect of the form namely that additional
%arguments are required is not captured in these morphological rules. If such morphological rules
%were accepted as proper constructions then there would not be any reason left to require that the
%arguments have to be present in a construction in order for it to be recognizable, and hence, the
%lexical approach would be accepted.\footnote{%
%Compare the discussion of \emph{Totschießen} `shoot dead' in
%example (\ref{ex-tot-schiessen}) below.%
%}

承继层级是Croft激进构式语法的主要解释工具\citep{Croft2001a}\isc{构式语法(CxG)|(}\is{Construction Grammar(CxG)|(}\isc{承继|(}\is{inheritance|(}。他也假设了短语构式并且主张将在一个分类框架(一个承继层级)中表征这些构式。他认为在这种网络中,一个语言表达的每种特异性都表征在它自己的节点中。图~\vref{Abbildung-Vererbungshierarchie-Croft}展示了他为句子假设的层级的一部分。
%Inheritance hierarchies are the main explanatory device in Croft's Radical Construction Grammar
%\citep{Croft2001a}\is{Construction Grammar(CxG)|(}\is{inheritance|(}. He also assumes phrasal constructions and suggests representing these in a taxonomic network (an inheritance hierarchy).
%He assumes that every idiosyncrasy of a linguistic expression is represented on its own node in this kind of network. Figure~\vref{Abbildung-Vererbungshierarchie-Croft} 
%shows part of the hierarchy he assumes for sentences.
\begin{figure}
\centering
\begin{forest}
for tree={draw,          % to get the boxes
          fit=rectangle, % tree layout with more space
          l+=5mm}
[Clause
  [Sbj IntrVerb
    [Sbj sleep]
    [Sbj run]]
  [Sbj TrVerb Obj
    [Sbj kick Obj
      [Sbj kick the bucket]
      [Sbj kick the habit]]
    [Sbj kiss Obj]]]
\end{forest}
\caption{\label{Abbildung-Vererbungshierarchie-Croft} \citew[\page 26]{Croft2001a}中短语模板的分类}
%\caption{\label{Abbildung-Vererbungshierarchie-Croft}Classification of phrasal patterns in  \citew[\page 26]%{Croft2001a}}
\end{figure}
这里面有包含不及物动词的句子以及包含及物动词的句子。形式为[Sbj kiss Obj]的句子是构式[Sbj TrVerb Obj]的特殊实例。[Sbj kick Obj]构式也包含次构式,即构式[Sbj kick the bucket]和构式[Subj kick the habit]。因为构式总是形式和意义的配对体,这就产生了一个问题:在包含kick(踢)的常规句子中,主语和kick(踢)的宾语之间确实存在一个“踢”的关系。但是对于kick(踢)的习语用法来说就不是这样,如(\mex{1})所示:
%There are sentences with intransitive verbs and sentences with transitive verbs. Sentences with the form
%[Sbj kiss
 % Obj] are special instances of the construction [Sbj TrVerb Obj]. The [Sbj kick Obj] construction also has further
%sub"=constructions, namely the constructions [Sbj kick the bucket] and [Subj
 % kick the habit]. 
%Since constructions are always pairs of form and meaning, this gives rise to a problem: in a normal sentence with \emph{kick}, there is a kicking relation between the subject and the
%object of\is{idiom} \emph{kick}. This is not the case for the idiomatic use of \emph{kick} in   
%(\mex{1}):
\ea
\gll He kicked the bucket.\\
     他 踢 \textsc{det} 水桶\\
\mytrans{他死了。}
%He kicked the bucket.
\z
这意味着[Sbj kick Obj]和[Sbj kick the bucket]构式之间没有常规的承继关系。相反,只有部分信息可以从[Sbj kick Obj]构式承继。其余信息由这一次构式重新定义。这种承继被称作缺省承继\isc{承继!缺省承继}\is{inheritance!default}。
%This means that there cannot be a normal inheritance relation between the [Sbj kick Obj] and the
%[Sbj kick the bucket] construction. Instead, only parts of the information may be inherited from the [Sbj kick Obj] construction. The other parts
%are redefined by the sub"=construction. This kind of inheritance is referred to as \emph{default inheritance}\is{inheritance!default}.

kick the bucket(死)是一个相对凝固的表达,即不可能在不丧失习语义的前提下将其被动化或者将其中的某些部分前置\citep*[\page 508]{NSW94a}。但是,并非所有的习语都是如此。正如 \citet*[\page 510]{NSW94a}指出的那样,有的习语可以被动化(\mex{1}a),并且有的习语的一部分可以在句子之外实现(\mex{1}b)。
%\emph{kick the bucket} is a rather fixed expression, that is, it is not possible to passivize it or front parts of it without losing
%the idiomatic reading \citep*[\page 508]{NSW94a}. However, this is not true for all idioms. As
% \citet*[\page 510]{NSW94a} have shown, there are idioms that can be passivized (\mex{1}a) as well as
%realizations of idioms where parts of idioms occur outside of the clause (\mex{1}b). 
\eal
\ex 
\gll The beans were spilled by Pat.\\
     \textsc{det} 豆 \textsc{aux} 溢出 \textsc{prep} Pat\\
\mytrans{秘密被Pat泄漏了。}
%The beans were spilled by Pat.
\ex 
\gll The strings [that Pat pulled] got Chris the job.\\
     \textsc{det} 绳子 \spacebr{}\textsc{comp} Pat 拉 获得 Chris \textsc{det} 工作\\
\mytrans{Pat走后门让Chris找到了工作。}
%The strings [that Pat pulled] got Chris the job.
\zl
%
%
现在的问题是必须在承继层级中为能够被动化的习语设置两个节点,因为在习语的主动形式和被动形式中成分的实现是存在差异的,但是意义仍然是特异的。主动形式与被动形式之间的关系就不能表示出来。 \citet{Kay2002a}曾经提出一种算法来计算能够允准主动和被动形式的层级中的对象(类似于构式的对象,简写为 CLOs)。正如我在 \citet[\S~3]{Mueller2006d}中提出的,这一算法无法得出正确的结果,并且现在很难看出怎样改进它以使其能够真正地工作。即便是接受我对承继层级的改进,仍然有现象无法使用承继层级来处理(参看本书~\ref{Abschnitt-Passiv-CxG})。
%The problem is now that one would have to assume two nodes in the inheritance hierarchy for idioms
%that can undergo passivization since the realization of the constituents is different in active and
%passive variants but the meaning is nevertheless idiosyncratic. The relation between the active and passive form would not be captured.
% \citet{Kay2002a} has proposed an algorithm for computing objects (Construction"=like objects = CLOs) from hierarchies that then license active and passive variants. As I
%have shown in  \citet[Section~3]{Mueller2006d}, this algorithm does not deliver the desired results and it is far from straightforward to improve it to the point
%that it actually works. Even if one were to adopt the changes I proposed, there are still phenomena that cannot be described using inheritance hierarchies
%(see Section~\ref{Abschnitt-Passiv-CxG} in this book).

另外一个值得注意的问题是动词必须明确在构式中列出。这就引出了另外一个问题:如果动词在构式中有不同用法,那么怎样来表征构式。如果为(\mex{1})中那样的案例在分类框架中设置一个节点,那么Goldberg对于词汇分析的批评就可以用在这里(Goldberg对词汇分析的批评是,词汇分析要为出现在不同构式中的同一个动词设置多个词项)\footnote{%
    注意我在这里使用的是术语“词条”(lexical entry),而不是“词项”(lexical item)。HPSG分析用词汇规则对应Goldberg的模板。Goldberg批评的是联系词条的词汇规则,而不是允准新词项的词汇规则,词项可以被储存也可以不储存。HPSG将后者当做词汇规则。参看~\ref{sec-hpsg-passive}。%
} :构式语法需要为每一个动词或者动词的每一个可能的用法设置一个构式。
%A further interesting point is that the verbs have to be explicitly listed in the constructions. This begs the question of how constructions should be represented where the verbs
%are used differently. If a new node in the taxonomic network is assumed for cases like (\mex{1}),
%then Goldberg's criticism of lexical analyses that assume several lexical entries for a verb that
%can appear in various constructions\footnote{%
%  Note the terminology: I used the word \emph{lexical entry} rather than \emph{lexical item}. The
%  HPSG analysis uses lexical rules that correspond to Goldberg's templates. What Goldberg criticizes
%  is lexical rules that relate lexical entries, not lexical rules that licence new lexical items,
%  which may be stored or not. HPSG takes the latter approach to lexical rules. See
%  Section~\ref{sec-hpsg-passive}.%
%} will be applicable here: one would have to
%assume constructions for every verb and every possible usage of that verb.
\ea
\gll He kicked the bucket into the corner.\\
     他 踢 \textsc{det} 桶 \textsc{prep} \textsc{det} 角落\\
\mytrans{他将桶踢到了角落里。}
%He kicked the bucket into the corner.
\z
%
%
对于否定句,Croft假设了带有多重承继的层级,如图~\vref{Abbildung-Vererbungshierarchie-mehrfach-Croft}所示。
%For sentences with negation, Croft assumes the hierarchy with multiple inheritance given in Figure~\vref{Abbildung-Vererbungshierarchie-mehrfach-Croft}. 
\begin{figure}
\centering
\begin{forest}
% we have to mention the style here, since we are overriding the global defaults for anchoring
.style={for tree={parent anchor=north, child anchor=south,grow=north,
          draw,          % to get the boxes
          fit=rectangle, % tree layout with more space
          l+=2mm}}
[I didn't sleep
%\\我 \textsc{aux}.\textsc{neg} 睡觉
  [Sbj IntrVerb]
  [Sbj Aux-n't Verb]]
\end{forest}
\caption{\label{Abbildung-Vererbungshierarchie-mehrfach-Croft} \citew[\page 26]{Croft2001a}中短语模式的互动}
%\caption{\label{Abbildung-Vererbungshierarchie-mehrfach-Croft}Interaction of phrasal patterns following  \citew[\page 26]{Croft2001a}}
\end{figure}%
这种表征方法的问题在于它仍然没有说清楚在否定句中动词语义的语义嵌套怎样表征。如果所有的构式都是形式与意义的配对体,那么必须要有[Sbj IntrVerb]结构的语义表征(\contvc\isfeat{cont}或者\textsc{sem}值\isfeat{sem})。与此相似,[Sbj Aux-n't Verb]也需要有意义。这时问题就出现了,[Sbj IntrVerb]的意义必须嵌套在否定意义之下,而这一点无法通过承继直接得出,因为X和非X是不兼容的。使用助动词特征解决这一问题有一种技术上的解决手段。因为自然语言的语法中有很多互动,那么如果假设特征是语言学对象可观察属性的直接反映的话,这种分析是不合理的。关于短语模式分类方法更详细的讨论,参看 \citew{MuellerPersian}和 \citew[\S~18.3.2.2]{MuellerLehrbuch1}。对于助动词特征在词汇基于承继的分析中的使用,参看 \citew[\S~7.5.2.2]{MuellerLehrbuch1}。\isc{承继|)}\is{inheritance|)}
%The problem with this kind of representation is that it remains unclear as to how the semantic embedding of the verb meaning under negation can
%be represented. If all constructions are pairs of form and meaning, then there would have to be a semantic representation for [Sbj IntrVerb]
%(\contv\isfeat{cont} or \textsc{sem} value\isfeat{sem}). Similarly, there would have to be a meaning for [Sbj Aux-n't Verb].
%The problem now arises that the meaning of [Sbj IntrVerb] has to be embedded under the meaning of the negation and this cannot be achieved directly
%using inheritance since X and not(X) are incompatible. There is a technical solution to this problem using auxiliary features. Since there are a number
%of interactions in grammars of natural languages, this kind of analysis is highly implausible if one claims that features are a direct reflection of
%observable properties of linguistic objects. For a more detailed discussion of approaches with classifications of phrasal patterns, see  \citew{MuellerPersian} as well as
% \citew[Section~18.3.2.2]{MuellerLehrbuch1} and for the use of auxiliary features in inheritance"=based analyses of the lexicon, see
%  \citew[Section~7.5.2.2]{MuellerLehrbuch1}.\is{inheritance|)}

\subsection{不同表征层面之间的映射}
\label{sec-mapping-between-levels}\label{sec-inheritance-passive-SimSyn}

 \citet[Chapter~6.3]{CJ2005a}认为被动应该分析为从语法功能层到表层论元实现几种可能映射之间的一个映射。指称论元实现在表层上有一定的格、一定的一致属性或者处在相应的位置上。虽然将具有不同属性的成分映射到不同的表层上的分析方式在LFG和HPSG理论\citep*{Koenig99a,BMS2001a}中都很常见,但是这种理论的总体特点是每一种互动现象都需要一层表征(Koenig方案中的\argstc、\textsc{sem-arg}、\textsc{add-arg},Bouma, Malouf \biband Sag方案中的\argstc、\textsc{deps}、\sprc、\compsc)。 \citew[\S~7.5.2.2]{MuellerLehrbuch1}充分讨论了Koenig分析方法所需要的扩展。
% \citet[Chapter~6.3]{CJ2005a} suggest that passive should be analyzed as one of several possible mappings from the
%Grammatical Function tier to the surface realization of arguments. Surface realizations of
%referential arguments can be NPs in a certain case, with certain agreement properties, or in a certain position. While such analyses that work by
%mapping elements with different properties onto different representations are common in theories
%like LFG and HPSG \citep*{Koenig99a,BMS2001a}, a general property of these analyses is that one
%needs one level of representation per interaction of phenomena (\argst, \textsc{sem-arg}, \textsc{add-arg}
%in Koenig's proposal, \argst, \textsc{deps}, \spr, \comps in Bouma, Malouf \biband Sag's proposal). This
%was discussed extensively in  \citew[Section~7.5.2.2]{MuellerLehrbuch1} with respect to extensions
%that would be needed for Koenig's analysis. 

因为Culicover和Jackendoff主张短语模型,我们这里讨论一下他们的方案。 Culicover和Jackendoff提出了一种多层模型,其中语义表征与语法功能联系在一起,语法功能与树上的位置相联系。图~\ref{fig-jackendoff-linking-active}是主动句的例子。
%Since Culicover and Jackendoff argue for a phrasal
%model, we will discuss their proposal here. Culicover and Jackendoff assume a multilayered model in
%which semantic representations are linked to grammatical functions, which are linked to tree
%positions. Figure~\ref{fig-jackendoff-linking-active} shows an example for an active sentence.
\begin{figure}
\centering
%\scalebox{.7}
{%
\begin{tabular}{ccccc}
DESIRE(&{~\mynode{b}{BILL$_2$}}, && & ~{\mynode{sw}{[SANDWICH; DEF]$_3$}})\\
\\[1ex]
       &{\mynode{gf2}{GF$_2$}}    && & {\mynode{gf3}{GF$_3$}}~\\
\\[1ex]
~~~~~~~~~\hfill{}[\sub{S} & {\mynode{np2}{NP$_2$}}  & [\sub{VP} & V$_1$ & ~~{\mynode{np3}{NP$_3$}}]] \\
\\
              & Bill           &  & desires & the sandwich.\\
              & Bill           &  & 想要 & \textsc{det} 三明治\\
\end{tabular}
\begin{tikzpicture}[overlay,remember picture] 
\draw (b)--(gf2)
      (gf2)--(np2)
      (sw)--(gf3)
      (gf3)--(np3);
\end{tikzpicture}
}
\caption{\label{fig-jackendoff-linking-active}将语法功能与树节点位置联系起来:主动}
%\caption{\label{fig-jackendoff-linking-active}Linking grammatical functions to tree positions: active}
\end{figure}%
GF代表语法功能。 \citet[\page 204]{CJ2005a}明确避免使用像主语、宾语这种名词,因为这对于他们分析被动运作非常重要。他们认为括号之后的第一个GF是括号对应的小句的主语(第195--196页),所以在英语中应该映射到合适的句法树位置上。注意语法功能和旁格的这一观点不能解释一些语言中可能有的无主语句,例如德语。\footnote{%
  当然也可以假设一个空虚位主语,正如 \citet[\page 1311]{Grewendorf93},但是空成分和特别是没有意义的空成分在构式主义文献中是尽量避免的。进一步讨论可以参看 \citew[\S~3.4, \S~11.1.1.3]{MuellerGTBuch1}。
}
%GF stands for Grammatical Function.  \citet[\page 204]{CJ2005a} explicitly avoid names like Subject and
%Object since this is crucial for their analysis of the passive to work. They assume that the first GF
%following a bracket is the subject of the clause the bracket corresponds to (p.\,195--196) and hence has to be mapped to an appropriate tree position in
%English. Note that this view of grammatical functions and obliqueness 
%is too simplistic since it cannot 
%does not account for subjectless sentences that are possible in some languages, for instance in
%German.\footnote{%
%  Of course one could assume empty expletive subjects, as was suggested by  \citet[\page
%    1311]{Grewendorf93}, but empty elements and especially those without meaning are generally
%  avoided in the constructionist literature. See  \citew[Section~3.4, Section~11.1.1.3]{MuellerGTBuch1} for further
%  discussion.
%}

关于被动,作者写道:
%Regarding the passive, the authors write:

\begin{quotation}
我们不想把被动当做一种删除或改变论元结构部分的操作,而想把被动当做独立存在的一种结构并且可以与句子的其它独立成分合一。合一的结果是允准句法和语义之间关系的另一种方法。\citep[\page 203]{CJ2005a}\footnote{%
we wish to formulate the passive not as an operation that deletes or alters part of the argument
structure, but rather as a piece of structure in its own right that can be unified with the other
independent pieces of the sentence. The result of the unification is an alternative licensing
relation between syntax and semantics.
}
\end{quotation}
他们提出了以下方式来表征被动:
%They suggest the following representation of the passive:
\ea
\label{constraint-CJ-passive}
{}\emph{[GF}$_i$ > [\emph{GF} \ldots]\emph{]}$_k$ $\Leftrightarrow$ \emph{[} \ldots V$_k$ $+$ pass \ldots (by NP$_i$) \ldots \emph{]}$_k$
\z
斜体部分是句子的常规结构而非斜体部分是常规结构的附加物,即适用于被动句的额外限制。
%The italicized parts are the normal structure of the sentence and the non-italicized parts are an
%overlay on the normal structure, that is, additional constraints that have to hold in passive
%sentences. 
图~\ref{fig-jackendoff-linking-passive}展示了上文讨论的对应于被动的例子的映射。
%Figure~\ref{fig-jackendoff-linking-passive} shows the mapping of the example discussed above that
%corresponds to the passive.

\begin{figure}
\centering
%\scalebox{.7}
{%
\begin{tabular}{ccccc}
DESIRE(&~{\mynode{b}{BILL$_2$},} & & & ~{}{\mynode{sw}{[SANDWICH; DEF]$_3$}})\\
\\[1ex]
       &{\mynode{gf2}{GF$_2$}}    &&  & {\mynode{gf3}{GF$_3$}}\\
\\[1ex]
~~~~~~~~~\hfill{}[\sub{S} & {\mynode{np3}{NP$_3$}}  & [\sub{VP} & V$_1$  & by {\mynode{np2}{NP$_2$}}]] \\
\\
              & the sandwich             & & is desired & by Bill.\\
              & \textsc{det} 三明治             & & \textsc{aux} 想要 & \textsc{prep} Bill.\\
\end{tabular}
\begin{tikzpicture}[overlay,remember picture] 
\draw (b)--(gf2)
      (gf2.south)--(np2.north)
      (sw)--(gf3)
      (gf3.south)--(np3.north);
\end{tikzpicture}
}
\caption{\label{fig-jackendoff-linking-passive}将语法功能与树节点位置连接起来:被动}
%\caption{\label{fig-jackendoff-linking-passive}Linking grammatical functions to tree positions: passive}
\end{figure}%

虽然Culicover和Jackendoff强调他们的方法与关系语法\citep{Perlmutter83a-ed}很相似,但是两者之间存在这样一个重要的差别:在关系语法中,如果需要另外的再映射,可以表示另外的层次(strata)。但是在Culicover和Jackendoff方案中,不存在另外的层次。这就导致了在分析允许多重论元交替的语言时会出现问题。来自于土耳其语的例子见(\ref{ex-double-passivization-two})。如果一种方法认为人称被动是一个概括结构和一个被动-特定结构合一的结果,那么就不能反映这一特点,因为它们过早地与一定结构组合了。主张被动的句法结构的方法存在的问题是,这种结构一旦声明就不能再被修饰了。Culicover和Jackendoff的方案就在这一层面运作,因为在(\ref{constraint-CJ-passive})中限制的右边没有很强的限制。但是还有一个不同的问题:当进行第二次被动化时,需要使用最里面的括号,即使用(\ref{constraint-CJ-passive})的结果是:
%Although Culicover and Jackendoff emphasize the similarity between their approach and Relational
%Grammar \citep{Perlmutter83a-ed}, there is an important difference: in Relational Grammar additional levels (strata) can be stipulated
%if additional remappings are needed. In Culicover and Jackendoff's proposal there is no additional
%level. This causes problems for the analysis of languages which allow for multiple argument alternations. Examples from Turkish were provided in
%(\ref{ex-double-passivization-two}). Approaches that assume that the personal passive is the unification
%of a general structure with a passive-specific structure will not be able to capture this, since they committed
%to a certain structure too early. The problem for approaches that state syntactic structure for the
%passive is that such a structure, once stated, cannot be modified. Culicover and Jackendoff's 
% proposal works in this respect since there are no strong constraints in the
%right-hand side of their constraint in (\ref{constraint-CJ-passive}). But there is a different
%problem: when passivization is applied the second time, it has to apply to the innermost bracket,
%that is, the result of applying (\ref{constraint-CJ-passive}) should be:
\ea
{}\emph{[GF}$_i$ > [\emph{GF}$_j$ \ldots]\emph{]}$_k$ $\Leftrightarrow$ \emph{[} \ldots V$_k$ $+$ pass \ldots (by NP$_i$) \ldots (by NP$_j$) \ldots\emph{]}$_k$
\z
这一点无法通过合一达到,因为合一需要检查兼容性,因为第一次使用被动是可以的,那么第二次使用被动也是可以的。表征中的点总是危险的,并且在当前的例子中,必须确保NP$_i$和NP$_j$是不同的,因为(\ref{constraint-CJ-passive})中的表述只是说句子中某处必须要是一个by-PP。真正需要的是带有GF表征的某物并且寻找最外层的括号,然后将括号放在下一个GF的左边。但是,这基本上是从一种表征映射到另外一种表征的规则,正如词汇规则所起到的作用。
%This cannot be done with unification, since unification checks for compatibility and since the first
%application of passive was possible it would be possible for the second time as well. Dots in
%representations are always dangerous and in the example at hand one would have to make sure that
%NP$_i$ and NP$_j$ are distinct, since the statement in (\ref{constraint-CJ-passive}) just says there
%has to be a \emph{by}-PP somewhere. What is needed instead of unification would be something that takes a GF representation
%and searches for the outermost bracket and then places a bracket to the left of the next GF. But
%this is basically a rule that maps one representation onto another one, just like lexical rules do.

如果Culicover和Jackendoff想要坚持映射分析,分析这种数据唯一的选择就是为无人称被动假设另外的层次,并且从该层次完成向短语结构的映射。在(\mex{1})所示的土耳其语的句子中,该句是人称被动,向这一层面的投射具有相同的功能。
%If Culicover and Jackendoff want to stick to a mapping analysis, the only option to analyze the data
%seems to be to assume an additional level for impersonal passives from which the mapping to phrase
%structure is done. In the case of Turkish sentences like (\mex{1}), which is a personal passive, the mapping
%to this level would be the identity function.   
%% \ea
%% \gll Arkada-şım bu   şato-da           boğ-ul-ur.\\
%%      friend-my this chateau-\textsc{loc} strangle-\textsc{pass}-\textsc{aor}\\
%% \mytrans{My friend is strangled (by one) in this chateau.}
%% \z
\ea
\gll Arkadaş-ım bu oda-da döv-ül-dü.\\
     朋友-我的  \textsc{det}   房间-\textsc{loc} 击打-\textsc{pass}-\textsc{aor}\\
\mytrans{我的朋友在这个房间中被人打。}%
%\gll Arkadaş-ım bu oda-da döv-ül-dü.\\
%     friend-my  this   room-\textsc{loc} hit-\textsc{pass}-\textsc{aor}\\
%\mytrans{My friend is beaten (by one) in this room.}
\z

\noindent
在被动化+无人称构式中,正确的映射将通过三个层面之间的两次映射,最后形成(\ref{ex-double-passivization-hit-two})所示的结构,为了方便在这里重复写成(\mex{1})。 
%In the case of passivization + impersonal construction, the correct mappings would be implemented by two mappings between the three levels
%that finally result in a mapping as the one that is seen in (\ref{ex-double-passivization-hit-two}),
%repeated here as (\mex{1}) for convenience.
\ea
\label{ex-double-passivization-hit-three}
\gll Bu oda-da döv-ül-ün-ür.\\
     \textsc{det} 房间-\textsc{loc} 击打-\textsc{pass}-\textsc{pass}-\textsc{aor}\\
\mytrans{某人在这间房中被人打死。}
%\gll Bu oda-da döv-ül-ün-ür.\\
%     this room-\textsc{loc} hit-\textsc{pass}-\textsc{pass}-\textsc{aor}\\
%\mytrans{One is beaten (by one) in this room.}
\z
注意被动化+无人称构式对于纯基于承继的方法都有问题。所有这些方法提出的都是他们只是记录了论元结构与短语结构之间四种不同的关系:主动、被动、无人称构式、被动+无人称构式。但是这种做法忽略了(\ref{ex-double-passivization-hit-three})是(\mex{-1})中被动的无人称形式。
%Note that passivization + impersonal construction is also problematic for purely inheritance based approaches. What
%all these approaches can suggest though is that they just stipulate four different relations between
%argument structure and phrase structure: active, passive, impersonal construction, passive + impersonal construction. But this misses the fact
%that (\ref{ex-double-passivization-hit-three}) is an impersonal variant of the passive in (\mex{-1}).

相反, \citet{Mueller2003e}提出的基于词汇规则的方法对于这种多重变换没有任何问题:被动化词汇规则的使用压制了最少的旁格论元并且提供了一个带有人称被动论元结构的词项。然后无人称词汇规则的使用压制了眼下的最少旁格论元(主动句的宾语)。结果是如(\ref{ex-double-passivization-hit-three})所示的没有任何论元的无人称构式。
%In contrast, the lexical rule-based approach suggested by
% \citet{Mueller2003e} does not have any problems with such multiple alternations:
%the application of the passivization lexical rule suppresses the least oblique argument and
%provides a lexical item with the argument structure of a personal passive. Then the impersonal
%lexical rule applies and suppresses the now least oblique argument (the object of the active
%clause). The result is impersonal constructions without any arguments as the one in (\ref{ex-double-passivization-hit-three}).

\subsection{有词汇规则的替代方法吗?}
 
在本节中,我们分析了想要用联系构式的方法来代替词汇规则所做的尝试。按照我们的评定,这些尝试都是失败的。我们相信这些方法的主要问题在于无法表示特定动词形式之间的派生特征。被动态和致使形态学如果当做词汇价结构中的操作的话,会是非常简单和规则的,词汇价结构是从词的短语环境中抽象而来。但是,在短语结构层面中非转换的规则或者系统会遇到非常严重的尚未解决的问题。
%In this section we have reviewed the attempts to replace lexical rules with methods of relating
%constructions.  These attempts have not been successful, in our assessment.  We believe that the
%essential problem with them is that they fail to capture the derivational character of the
%relationship between certain word forms.  Alternations signaled by passive voice and causative
%morphology are relatively simple and regular when formulated as operations on lexical valence
%structures that have been abstracted from their phrasal context.  But non-transformational rules or
%systems formulated on the phrasal structures encounter serious problems that have not yet been
%solved.

\section{基于短语的方法的其他问题}

 \citet{Mueller2006d}讨论了认为短语构式是邻接成分的固定形式的方案共有的问题,这种方法如 \citet{GJ2004a}。我已经展示过很多论元结构构式在组成成分的顺序方面有很大的灵活性。我讨论过结果构式与自由与格、被动和其它变价现象的互动并且展示出对于所有这些需要互动允准的构式,构式的组成部分都可以置换,动词可以出现在不同的位置,论元可以被抽取等。下面的小节将讨论小词动词现象,那些认为短语构式中动词和小词顺序固定的方法在解释这一现象时都会遇到困难。
% \citet{Mueller2006d} discussed the problems shared by proposals that assume phrasal constructions to
%be a fixed configuration of adjacent material as for instance the one by  \citet{GJ2004a}. I showed that many
%argument structure constructions allow great flexibility as far as the order of their parts is
%concerned. Back then I discussed resultative constructions in their interaction with free datives,
%passive and other valence changing phenomena and showed that for all these constructions licensed by such interactions the construction parts can be scrambled, the verb can appear in different positions,
%arguments can be extracted and so on. The following subsection discusses particle verbs, which pose
%similar problems for theories that assume a phrasal construction with fixed order of verb and particle.

\subsection{小词动词及其受到短语结构构型的约束}
\label{sec-particle-verbs-phrasal}

假设短语结构构型与语义匹配的方法存在一个普遍的问题,即构式可能出现在很多不同的环境中:构式的组成部分可能会涉及到派生形态(正如前面章节所谈到的)或者构式组成部分可能会涉及到成分分裂。后一种类型的一个例子是Booij(\citeyear[\S~2]{Booij2002a};\citeyear{Booij2012a-u})和 \citet{Blom2005a}(他们分别在构式语法\indexcxgc(Construction Grammar)和词汇功能语法\indexlfgc(LFG)框架内工作)用短语性分析来分析小词动词现象。研究丹麦语\il{丹麦语}\il{Dutch}的学者认为小词动词由短语构式(短语结构的一部分),其中第一个槽由小词占据。
%A general problem of approaches that assume phrase structure configurations paired with meaning is
%that the construction may appear in different contexts: the construction parts may be involved in
%derivational morphology (as discussed in the previous subsection) or the construction parts may be
%involved in dislocations. A clear example of the latter type is the phrasal analysis of particle
%verbs that was suggested by Booij (\citeyear[Section~2]{Booij2002a}; \citeyear{Booij2012a-u}) and   \citet{Blom2005a}, working in the
%frameworks of Construction Grammar\indexcxg and LFG\indexlfg, respectively. The authors working on Dutch\il{Dutch} and German assume that particle verbs are licensed by
%phrasal constructions (pieces of phrase structure) in which the first slot is occupied by the particle. 
\ea
{}[ X [~]\sub{V} ]\sub{V$'$} where X = P, Adv, A, or N
\z
丹麦语构式的具体例子有:
%Examples for specific Dutch constructions are:
\eal
\label{particle-konstruktionen}
\ex {}[ af   [~]\sub{V} ]\sub{V$'$}
\ex {}[ door [~]\sub{V} ]\sub{V$'$}
\ex {}[ op   [~]\sub{V} ]\sub{V$'$}
\zl 
这一分析方法基于小词不能前置这一观点。这一观点经常在文献中提及,但是这一观点是基于内省的并且对于丹麦语\il{丹麦语}\il{Dutch}和德语等语言并不适用。关于丹麦语参看 \citew[\page19]{Hoeksema91a},关于德语参看 \citew{Mueller2002b,Mueller2002d,Mueller2003a,Mueller2007c}。\footnote{%
关于小词动词内省和语料库更加根本的一些观点可以参看 \citew{Mueller2007c}和 \citew{MM2009a}。
} 
%This suggestion comes with the claim that particles cannot be fronted.  This claim is made
%frequently in the literature, but it is based on introspection and wrong for languages like Dutch\il{Dutch} and German. On Dutch see  \citew[\page19]{Hoeksema91a}, on German,
% \citew{Mueller2002b,Mueller2002d,Mueller2003a,Mueller2007c}.\footnote{%
%Some more fundamental remarks on
%introspection and corpus data with relation to particle verbs can also be found in
% \citew{Mueller2007c} and  \citew{MM2009a}.
%} 
(\mex{1})中是一个德语的例子;在引用的文献中可以看到几页经过验证的例句并且更复杂的例子也将会在第~\pageref{ex-complex-vf}页的~\ref{sec-neuro-linguistics}进行讨论。
%A German example is given in (\mex{1}); several pages of attested examples can be found in the cited references and some more complex examples will
%also be discussed in Section~\ref{sec-neuro-linguistics} on page~\pageref{ex-complex-vf}.
\ea\label{bsp-los-damit-zwei}
\gll \emph{Los} damit \emph{geht} es schon am 15. April.\footnotemark\\
      \textsc{part} \textsc{adv} 走 \expl{} 已经 \textsc{prep}.\textsc{det} 15 4月\\%
\footnotetext{%
        taz, \zhdate{2002/03/01},第8页,也可以参看 \citew[\page313]{Mueller2005d}。%
    }%
\mytrans{已经在4月15日开始了。}
%\gll \emph{Los} damit \emph{geht} es schon am 15. April.\footnotemark\\
%     \textsc{part} there.with goes it already at.the 15 April\\%
%\footnotetext{%
%        taz, 01.03.2002, p.\,8, see also  \citew[\page313]{Mueller2005d}.%
%    }%
%\mytrans{It already starts on April the 15th.}
\z
小词动词是小-习语。所以可以得出以下结论:在语序上具有一定灵活性的习语性表达不应该被表征为描述邻接成分的短语构型。对于一些习语,按照 \citew{Sag2007a}的研究思路好像是需要的。\footnote{也需要注意德语的例子最好也描述为在定式动词之前有一个复杂内部结构成分的句子,并且基于线性的方法是很令人怀疑的,如 \citew [\page244--248]{Kathol95a}或 \citew{Wetta2011a}是否可以概括这种现象。可以参看~\ref{sec-dg-multiple-frontings},与依存语法有关联的多重前置\isc{前置!多重前置}\is{fronting!apparent multiple}的讨论。
}
%Particle verbs are mini-idioms. So the conclusion is that idiomatic expressions that 
%allow for a certain flexibility in order should not be represented as phrasal configurations describing adjacent
%elements. For some idioms, a lexical analysis along the lines of  \citew{Sag2007a} seems to be
%required.\footnote{Note also that the German example is best described as a clause with a complex internally 
%  structured constituent in front of the finite verb and it is doubtful whether linearization-based
%  proposals like the ones in  \citew [\page244--248]{Kathol95a} or  \citew{Wetta2011a} can capture
%  this. See also the discussion of multiple frontings\is{fronting!apparent multiple} in connection to Dependency Grammar in Section~\ref{sec-dg-multiple-frontings}.
%}
小词动词问题会在~\ref{sec-neuro-linguistics}再次提到,那里我们将讨论来自神经科学方面的支持/反对短语分析的证据。
%The issue of particle verbs will be taken up in Section~\ref{sec-neuro-linguistics} again, where we
%discuss evidence for/against phrasal analyses from neuroscience.

\section{来自语言习得的证据} 
\label{sec-acquisition}

语言习得是基于模式的吗?如果是的话就可以作为证据来证明基于短语的方法。这一问题在\ref{Abschnitt-musterbasiert}和\ref{Abschnitt-Selektionsbasierter-Spracherwerb}已经提到过。构式在并列结构中可以不连续实现,所以必须习得的是依存的概念;习得简单的连续模式是不够的。
%The question whether language acquisition is pattern"=based and hence can be seen as evidence for
%the phrasal approach has already been touched upon in the Sections~\ref{Abschnitt-musterbasiert}
%and~\ref{Abschnitt-Selektionsbasierter-Spracherwerb}. It was argued that constructions can be
%realized discontinuously in coordinations and hence it is the notion of
%dependency that has to be acquired; acquiring simple continuous patterns is not sufficient.

因为目前关于短语方法和词汇方法的讨论跟具体的方案相关,我想增加另外两个特殊的小节:\ref{sec-recognizability-of-constructions}分析构式的可辨认性,\ref{Abschnitt-Koordination-diskont}讨论并列的具体处理方法,来说明各种理论框架是如何处理构式的非连续实现的。
%Since the present discussion about phrasal and lexical approaches deals with specific proposals, I would like
%to add two more special subsections: Section~\ref{sec-recognizability-of-constructions} deals with
%the recognizability of constructions and Section~\ref{Abschnitt-Koordination-diskont} discusses specific approaches to coordination
%in order to demonstrate how frameworks deal with the discontinuous realization of constructions.

\subsection{构式的可辨认性}
\label{sec-recognizability-of-constructions}
 
我认为纯基于模式的方法难以解释(\mex{1})所示的例子:
%I think that a purely pattern-based approach is weakened by the existence of examples like (\mex{1}):
\eal
\ex 
\gll John tried to sleep.\\
     John 努力 \textsc{inf} 睡觉\\
\mytrans{John努力去入睡。}
%John tried to sleep.
\ex
\gll John tried to be loved.\\
     John 努力 \textsc{inf} \textsc{aux} 爱\\
\mytrans{John努力被爱。} 
%John tried to be loved.
\zl
虽然在短语to sleep(去睡觉)中,sleep(睡觉)的一个论元都没有出现;在短语to be loved(被爱)中,没有出现主语和宾语,但是两个短语分别被识别为包含一个不及物动词和及物动词的短语。\footnote{%
构式主义理论不假设空成分。当然,在GB理论框架中,主语可以被实现为空成分。所以,虽然没有语音形式,但是仍然可以出现在该结构中。%
}  
%Although no argument of \emph{sleep} is present in the phrase \emph{to sleep} and neither a subject
%nor an object is realized in the phrase \emph{to be loved}, both phrases are recognized as phrases
%containing an intransitive and a transitive verb, respectively.\footnote{%
%Constructionist theories do not assume empty elements. Of course, in the GB framework the subject
%would be realized by an empty element. So it would be in the structure, although inaudible.%
%}  
这同样适用于短语构式引入/允准的论元:在(\mex{1})中,结果构式经历了被动化并且嵌套在一个控制动词下,导致在局部小句中只有结果谓词tot(死)和母句动词geschossen(射击)被明显地实现,这里用括号括起来了:
%The same applies to arguments that are supposed to be introduced/licensed by a phras\-al construction:
%in (\mex{1}) the resultative construction is passivized and then embedded under a control
%verb, resulting in a situation in which only the result predicate (\emph{tot} `dead') and the matrix verb (\emph{geschossen} `shot') are
%realized overtly within the local clause, bracketed here:
\ea
\gll Der kranke Mann wünschte sich,   [totgeschossen zu werden].\footnotemark\\
     \textsc{det} 生病的   人  希望   \self{} \spacebr{}死.射杀      \textsc{inf} \textsc{aux}\\
\footnotetext{%
 \citew[\page 387]{Mueller2007d}。
}
\mytrans{这个病人希望被射杀。}
%\gll Der kranke Mann wünschte sich,   [totgeschossen zu werden].\footnotemark\\
%     the sick   man  wished   \self{} \spacebr{}dead.shot      to be\\
%\footnotetext{%
% \citew[\page 387]{Mueller2007d}.
%}
%\mytrans{The sick man wanted to be shot dead.}
% replaced "ill" by "sick" after submission
\z
当然被动化和控制导致了这些结果,但是这里重要的一点是论元可以保持不表达或隐含,但是经常与论元明显实现形式连接的意义仍然存在\citep[Section~4]{Mueller2007d}。所以,语言学习者必须习得的是什么时候一个结果谓词和一个主要动词会同时实现,两者产生结果义。再举另外一个例子,通常在主动结果构式中实现的NP论元在如(\mex{1})所示的名词化现象中,仍然可以隐含。
%Of course passivization and control are responsible for these occurrences, but the important point
%here is that arguments can remain unexpressed or implicit and nevertheless a meaning usually
%connected to some overt realization of arguments is present \citep[Section~4]{Mueller2007d}. So,
%what has to be acquired by the language learner is that when a result predicate and a main verb are
%realized together, they contribute the resultative meaning.  
%To take another example, NP arguments that are usually realized in active resultative constructions may remain implicit
%in nominalizations like the ones in (\mex{1}):
\eal
\label{ex-tot-schiessen}
\ex 
\gll dann scheint uns das Totschießen mindestens ebensoviel Spaß zu machen\footnotemark\\
     那么 看起来 我们  \textsc{det} 死-射击 最少 一样多地 乐趣 \textsc{inf} 制造\\
\footnotetext{%
  \href{https://www.elitepartner.de/forum/wie-gehen-die-maenner-mit-den-veraenderten-anspruechen-der-frauen-um-26421-6.html}{https://www.elitepartner.de/forum/wie-gehen-die-maenner-mit-den-veraenderten-anspruechen-der-}
  \href{https://www.elitepartner.de/forum/wie-gehen-die-maenner-mit-den-veraenderten-anspruechen-der-frauen-um-26421-6.html}{frauen-um-26421-6.html}。\zhdate{2012/03/26}。
}
\mytrans{那么射杀人对于我们来说能带来最少的乐趣。}
%\gll dann scheint uns das Totschießen mindestens ebensoviel Spaß zu machen\footnotemark\\
%     then seems   us  the dead-shooting at.least as.much    fun to make\\
%\footnotetext{%
%  \href{https://www.elitepartner.de/forum/wie-gehen-die-maenner-mit-den-veraenderten-anspruechen-der-frauen-um-26421-6.html}{https://www.elitepartner.de/forum/wie-gehen-die-maenner-mit-den-veraenderten-anspruechen-der-}
%  \href{https://www.elitepartner.de/forum/wie-gehen-die-maenner-mit-den-veraenderten-anspruechen-der-frauen-um-26421-6.html}{frauen-um-26421-6.html}. 26.03.0212.
%}
%\mytrans{then the shooting dead seems to us to be as least as much fun}
% added "to us" after submission
\ex
\gll Wir lassen heut das Totgeschieße,\\                   
我们  让    今天 \textsc{det} 死.射击\\
%\gll Wir lassen heut das Totgeschieße,\\                   
%we  let    today the annoying.repeated.shooting.dead\\\\
\gll  Weil  man sowas heut nicht tut.\\
      因为 某人 这种.事情 今天 \textsc{neg} \textsc{aux}\\
%\gll  Weil  man sowas heut nicht tut.\\
%      since one such.thing today not does\\\\
\gll Und wer einen Tag sich ausruht,\\
     并且 \textsc{rel} 一 天 \textsc{self} 休息\\
%\gll Und wer einen Tag sich ausruht,\\
%     and who a day \textsc{self} rests\\\\
\gll Der schießt morgen doppelt gut.\footnotemark\\
\textsc{rel} 射击 明天 两次 好\\
\footnotetext{%
  Gedicht für den Frieden, Oliver Kalkofe, 
  \url{http://www.golyr.de/oliver-kalkofe/songtext-gedicht-fuer-den-frieden-417329.html}。\zhdate{2016/03/04}。
} 
\mytrans{今天我们不射杀任何人,这是因为今天不做这件事,而那些今天休息的人明天要射杀两次。}
%\gll Der schießt morgen doppelt gut.\footnotemark\\
%this shoots tomorrow twice good\\
%\footnotetext{%
%  Gedicht für den Frieden, Oliver Kalkofe, 
%  \url{http://www.golyr.de/oliver-kalkofe/songtext-gedicht-fuer-den-frieden-417329.html}. 04.03.2016.
%} 
%\glt `We do not shoot anybody today, since one does not do this today, and those who rest a day shoot
%twice as well tomorrow.'
\zl
对应着动词受事的论元(被射杀的人)因为名词化的句法限制,可以保持不实现。结果意义仍然可以理解,这就证明结果义并不依赖于涉及Subj、V、Obj、和Obl的结果构式的出现。
%The argument corresponding to the patient of the verb (the one who is shot) can remain unrealized,
%because of the syntax of nominalizations.  The resultative meaning is still understood, which shows
%that it does not depend upon the presence of a resultative construction involving Subj V Obj and Obl.  

\subsection{并列和不连续结构}
\label{Abschnitt-Koordination}
\label{Abschnitt-Koordination-diskont}\label{sec-coordination-cg}

本小节讨论本书论述的各种理论框架是如何分析并列结构的。这一节的目的是想展示在并列结构中简单短语模式必须分开。这一点在\ref{Abschnitt-musterbasiert}已经提及,但是为了说得更清楚,就看一下具体的方案。
%The following subsection deals with analyses of coordination in some of the frameworks that were
%introduced in this book. The purpose of the section is to show that simple phrasal patterns have to
%be broken up in coordination structures. This was already mentioned in Section~\ref{Abschnitt-musterbasiert}, but I think it
%is illuminative to have a look at concrete proposals.

范畴语法对并列有非常好的处理(参看\citealp{Steedman91a})。对称并列指的是具有相同句法属性的两个对象组成一个具有那些句法属性的对象。在第~\pageref{Seite-HPSG-Koordination}页讨论HPSG中使用特征表征方式动因时,曾经谈及相关数据。例子的英文版在这里重复表示为(\mex{1}):
%In Categorial Grammar, there is a very elegant treatment of coordination (see \citealp{Steedman91a}). 
%A generalization with regard to so"=called symmetric coordination is that two objects with the same syntactic properties are combined to an object
%with those properties. We have already encountered the relevant data in the discussion of the motivation for feature geometry in HPSG on
%page~\pageref{Seite-HPSG-Koordination}. Their English versions are repeated below as (\mex{1}):
\eal
\ex 
\gll the man and the woman\\
     \textsc{det} 男人 和 \textsc{det} 女人\\
\mytrans{男人和女人}
%the man and the woman
\ex 
\gll He knows and loves this record.\\
     他 知道 并且 喜欢 \textsc{det} 唱片\\
\mytrans{他知道并喜欢这一唱片。}
%He knows and loves this record.
\ex 
\gll He is dumb and arrogant.\\
     他 \textsc{cop} 哑 并且  傲慢\\
\mytrans{他哑巴并且傲慢。}
%He is dumb and arrogant.
\zl
 \citet{Steedman91a}用一条规则来分析(\mex{0})所示的例子:
% \citet{Steedman91a} analyzes examples such as those in (\mex{0}) with a single rule:
\ea
X conj X $\Rightarrow$ X
\z
这条规则将两个同种类的范畴用一个连接词连接起来组成一个与并列成分同类的范畴。\footnote{%
另外,我们可以为连词and(和)设立一个词项来分析所有这三个例子:and(和)是一个功能符,在其右边与任意范畴的词或短语组合。在组合之后,就需要其左边的成分与组合后的成分范畴一致。这意味着and(和)的范畴的形式是(X\bs X)/X。
这一分析不需要任何并列规则。如果假设每一个结构都有一个中心,像在\indexgbc 管辖约束理论(GB)/最简方案(MP)\indexmpc 中普遍要求的那样,那么像(\mex{0})所示的那样为并列假设一个特殊规则的无中心语的分析就会被排除。
}
%This rule combines two categories of the same kind with a conjunction in between to form a category that has the same category as the conjuncts.\footnote{%
%Alternatively, one could analyze all three examples using a single lexical entry for the conjunction
%\emph{and}: \emph{and} is a functor that takes a word or phrase
%of any category to its right and after this combination then needs to be combined with an element of the same category to its left in order to form the relevant
%category after combining with this second element. This means that the category for \emph{und} would have the form (X\bs X)/X. 
%This analysis does not require any coordination rules. If one wants to assume, as is common in\indexgb GB/MP\indexmp, that every structure has a head, then a headless
%analysis that assumes a special rule for coordination like the one in (\mex{0}) would be ruled out.
%}
图~\vref{Abb-cg-np-koordination}展示了对(\mex{-1}a)的分析,图~\vref{Abb-CG-Koordination-V}给出了跟(\mex{-1}b)相对应的英语\il{英语}\il{English}例子的分析。
%Figure~\vref{Abb-cg-np-koordination} shows the analysis of (\mex{-1}a) and
%Figure~\vref{Abb-CG-Koordination-V} gives an analysis of the corresponding English\il{English} example of
%(\mex{-1}b).
\begin{figure}
\centerline{%
\deriv{5}{
the & man & and        & the  & woman\\
\textsc{det} & \textrm{男人} & \textrm{和}        & \textsc{det}  & \textrm{女人}\\
\hr & \hr  & \hr        & \hr & \hr \\
np/n & n   & conj       & np/n & n\\
\multicolumn{2}{@{}c@{}}{\forwardapp} &  & \multicolumn{2}{c@{}}{\forwardapp} \\
\multicolumn{2}{@{}c@{}}{np}          &  & \multicolumn{2}{c@{}}{np}\\
\multicolumn{5}{@{}c@{}}{\conjapp}\\
\multicolumn{5}{@{}c@{}}{np}\\
}
}
\caption{\label{Abb-cg-np-koordination}范畴语法中两个NP并列}
%\caption{\label{Abb-cg-np-koordination}Coordination of two NPs in Categorial Grammar}
\end{figure}%

\begin{figure}
\centerline{%
\deriv{6}{
he  & knows        & and        & loves        & this  & record\\
\textrm{他}  & \textrm{知道}        & \textrm{和}        & \textrm{喜欢}        & \textrm{这个}  & \textrm{唱片}\\
\hr & \hr          & \hr        & \hr          & \hr   & \hr         \\
np  & (s\bs np)/np & conj & (s\bs np)/np & n/np  & n\\
    &              &      &              & \multicolumn{2}{c@{}}{\forwardapp}\\
    &              &      &              & \multicolumn{2}{@{}c@{}}{np}\\
    & \multicolumn{3}{c@{}}{\conjapp}\\
    & \multicolumn{3}{c@{}}{(s\bs np)/np}\\
    & \multicolumn{5}{c@{}}{\forwardapp} \\
    & \multicolumn{5}{c@{}}{s\bs np}\\
\multicolumn{6}{@{}c@{}}{\backwardapp}\\
\multicolumn{6}{@{}c@{}}{s}\\
}
}
\caption{\label{Abb-CG-Koordination-V}范畴语法中两个及物动词的并列}
%\caption{\label{Abb-CG-Koordination-V}Coordination of two transitive verbs in Categorial Grammar}
\end{figure}%

如果我们对比一下该分析与传统短语结构语法假设的分析就可以发现优势在哪里:传统的短语结构语法需要一条规则来分析NP并列,其中两个NP并列起来形成一个新的NP;对于V并列结构又需要另外一条规则。从技术的角度来看,这种做法是不好的,也没有反应对称并列的属性:两个句法范畴相同的符号互相组合。 
%If we compare this analysis to the one that would have to be assumed in traditional phrase structure grammars, it becomes apparent
%what the advantages are: one rule was required for the analysis of NP coordination where two NPs are coordinated to form an NP and another was required
%for the analysis of V coordination. This is not only undesirable from a technical point of view, neither does it capture the basic property of symmetric coordination:
%two symbols with the same syntactic category are combined with each other.

值得注意的是:这种方法还可以分析(\mex{1})所示的短语:
%It is interesting to note that it is possible to analyze phrases such as (\mex{1}) in this way:
\ea
\label{Beispiel-Gapping-Steedman}
\gll give George a book and Martha a record\\
     给 George 一 书 和 Martha 一 唱片\\
\mytrans{给George一本书给Martha一张唱片}
%give George a book and Martha a record
\z
在\ref{Abschnitt-K-Tests-Koordination},我们已经看到这种句子无法通过成分测试。但是,在范畴语法中,如果像 \citet{Dowty88a-u}和 \citet{Steedman91a}那样接受类型提升规则和组合规则,也可以很好地分析这些现象。在\ref{Abschnitt-UDC-KG},我们已经看到了向前类型提升和向后组合。为了分析(\mex{0}),就需要(\mex{1})所示的向后类型提升规则和(\mex{2})所示的向后组合规则:
%In Section~\ref{Abschnitt-K-Tests-Koordination}, we have seen that this kind of sentences is problematic for constituent tests. However, in Categorial Grammar, it is possible to
%analyze them without any problems if one adopts rules for type raising and composition as  \citet{Dowty88a-u} and  \citet{Steedman91a} do.
%In Section~\ref{Abschnitt-UDC-KG}, we have already seen forward type raising as well as forward and backward composition. In order to analyze
%(\mex{0}), one would require backward type raising repeated in (\mex{1}) and backward composition repeated in
%(\mex{2}):
\ea
向后类型提升\isc{类型提升!向后}\is{type raising!backward} (< T)\\
X $\Rightarrow$ T\bs (T/X)
\z
\ea
向后组合\isc{组合!向后}\is{composition!backward} (< B)\\
    Y\bs Z $*$ X\bs Y = X\bs Z
\z

\noindent
Dowty对(\mex{-2})的分析见图~\ref{Abb-CG-Gapping},vp代表s\bs np。
%Dowty's analysis of (\mex{-2}) is given in Figure~\ref{Abb-CG-Gapping}. vp stands for s\bs np.
% vref loopt
\begin{figure}
\oneline{%
\deriv{6}{
give       & George                  & a\;book            & and        & Martha                  & a\;record  \\
\textrm{给}         & George                  & \textrm{一书}              & \textrm{和}         & Martha                    & \textrm{一唱片}  \\
\hr        & \backwardt               & \backwardt         & \hr        & \backwardt               & \backwardt    \\
(vp/np)/np & (vp/np)\bs ((vp/np)/np) & vp\bs (vp/np)     & conj & (vp/np)\bs ((vp/np)/np) & vp\bs (vp/np)\\
           & \multicolumn{2}{c@{}}{\backwardc}        &      &  \multicolumn{2}{c@{}}{\backwardc}\\
           & \multicolumn{2}{c@{}}{vp\bs ((vp/np)/np)}&      & \multicolumn{2}{@{}c@{}}{vp\bs ((vp/np)/np)}\\
    & \multicolumn{5}{c@{}}{\conjapp}\\
    & \multicolumn{5}{c@{}}{vp\bs ((vp/np)/np)}\\
\multicolumn{6}{@{}c@{}}{\backwardapp} \\
\multicolumn{6}{@{}c@{}}{vp}\\
}
}
\caption{\label{Abb-CG-Gapping}范畴语法中的空位}
%\caption{\label{Abb-CG-Gapping}Gapping in Categorial Grammar}
\end{figure}%
这种类型提升分析经常受到批评,这是因为提升范畴会导致简单句有很多种不同的分析可能性。例如,可以首先将被类型提升的主语与动词组合然后将结果成分与宾语组合。这意味着我们除了标准的[S [V O]]分析之外,还有一个[[S V]O]分析。 \citet{Steedman91a}主张两种分析在信息结构方面存在差异,所以为正在讨论的句子假设不同的结构是合理的。
%This kind of type"=raising analysis was often criticized because raising categories leads to many different analytical possibilities
%for simple sentences. For example, one could first combine a type"=raised subject with the verb and then combine the resulting constituent
%with the object. This would mean that we would have a [[S V]
%O] in addition to the standard [S [V O]] analysis.
% \citet{Steedman91a} argues that both analyses differ in terms of information structure and it is therefore valid to assume different structures
%for the sentences in question.

这里我不想再进一步讨论这些观点。我想对比一下Steedman的词汇方法与短语性分析:所有用连续模式处理双及物模式的方法在分析上面讨论的例子时都会遇到严重问题。这一点可以通过考察 \citet{SJ96a}提出的对于并列的词汇功能语法分析\indextagc(TAG分析)得到最好的解释。如果假设[Sbj TransVerb Obj]或者[S [V O]]是一个固定的单位,那么图\vref{Abbildung-knows-loves}所示的句法树就构成分析并列的起点。
%I will not go into these points further here. However, I would like to compare Steedman's lexical approach to phrasal analyses: all approaches
%that assume that the ditransitive construction represents a continuous pattern encounter a serious problem with the examples discussed above. This
%can be best understood by considering the TAG analysis\indextag of coordination proposed by  \citet{SJ96a}.
%If one assumes that [Sbj TransVerb Obj] or [S [V O]] constitutes a fixed unit, then the trees in
%Figure~\vref{Abbildung-knows-loves} form the starting point for the analysis of coordination. 

\begin{figure}
\hfill
\begin{forest}
tag
[S
	[NP$\downarrow$]
	[VP
		[V
			[knows;知道]]
		[NP$\downarrow$]]]
\end{forest}
\hfill
\begin{forest}
tag
[S
	[NP$\downarrow$]
	[VP
		[V
			[loves;喜欢]]
		[NP$\downarrow$]]]
\end{forest}
\hfill\mbox{}
\caption{\label{Abbildung-knows-loves}“知道”和“喜欢”的基础树}
%\caption{\label{Abbildung-knows-loves}Elementary trees for \emph{knows} and \emph{loves}}
\end{figure}%

如果想要使用这些句法树/构式来分析(\mex{1}),那么原则上有两种可能性:一是假设两个完整的句子并列或者假设一些节点在并列结构中是共享的。
%If one wants to use these trees/constructions for the analysis of (\mex{1}), there are in principle
%two possibilities: one assumes that two complete sentences are coordinated or alternatively, one
%assumes that some nodes are shared in a coordinated structure.  
\ea
\label{Beispiel-he-knows-and-loves}
\gll He knows and loves this record.\\
     他 知道 并且 喜欢 \textsc{det} 唱片\\
\mytrans{他知道并且喜欢这张唱片。}
%He knows and loves this record.
\z
%
% Diese Argumentationen sind wohl allesamt gegen Old-School Transformationsanalysen.
% Wenn man nur eine Variable hat, dann kriegt man auch die falschen Lesarten nicht.
% St. Mü. 30.05.2010
%
% Ansätze zur Behandlung der Koordination, die Beispiele wie (\mex{0}) auf zwei vollständige
% koordinierte Sätze zurückführen, wären auch nicht auf alle
% Koordinationsdaten anwendbar, wie bereits  \citet[\page 102]{BV72},  \citet[\page 143]{Dowty79a},
%  \citet[\page 104--105]{denBesten83a},  \citet[\page 8--9]{Klein85} und  \citet{Eisenberg94a} festgestellt haben.
% %siehe auch deGeest70a:40
% Das Problem ist, dass die beiden folgenden Sätze nicht bedeutungsgleich sind:
% \eal
% \ex[]{
% Ein Mädchen stand an der Ecke und winkte mir über die Straße.
% }
% \ex[]{
% Ein Mädchen stand an der Ecke und ein Mädchen winkte mir über die Straße.
% }
% \zl
% Der erste Satz ist wahr, wenn es ein Mädchen gibt, dass sowohl an der Ecke steht als auch winkt,
% der zweite Satz wäre auch wahr, wenn es zwei Mädchen gäbe, von denen das eine an der Ecke steht und
% das andere winkt.
%
 \citet{Abeille2006a}已经证明如果假设(\mex{0})所示的那些并列例子总是涉及两个完整句子的并列,那么不可能概括所有的现象。也需要允许Steedman分析(也可以参看\ref{Abschnitt-Spezfikatoren-MP})中使用的词汇并列。 \citet{SJ96a}提出了一种词汇功能语法分析\indextagc(TAG analysis),其中节点可以在并列结构中共享。(\ref{Beispiel-he-knows-and-loves})的分析可以见图~\vref{Abbildung-He-knows-and-loves-this-record-TAG}。
% \citet{Abeille2006a} has shown that it is not possible to capture all the data if one assumes that cases of coordination such as  those in
%(\mex{0}) always involve the coordination of two complete clauses. It is also necessary to allow for lexical coordination of the kind we saw
%in Steedman's analysis (see also Section~\ref{Abschnitt-Spezfikatoren-MP}).
% \citet{SJ96a} develop a TAG analysis\indextag in which nodes are shared in coordinate structures.
%The analysis of (\ref{Beispiel-he-knows-and-loves}) can be seen in Figure~\vref{Abbildung-He-knows-and-loves-this-record-TAG}.
%vref loopt hier
\begin{figure}
\centering
\begin{forest}
sm edges
[\phantom{S}
  [S, no edge
	[NP,name=np11
		[he;他]]
	[VP, name=vp1
          [V,name=v1    [knows;知道]]]]
  [S, no edge, name=s2
        [V, name=vcoord, no edge [and, name=and, no edge]]
        [VP
           [V,name=v2 [loves;喜欢]]
           [NP, name=np22 [this record;这 唱片, roof]]]]]
\draw (s2.south)--(np11.north)
      (vp1.south)--(np22.north);
\draw[thick] (vcoord.south)--(v1.north)
             (vcoord.south)--(v2.north)
             (vcoord.south)--(and.north);
\end{forest}
\caption{\label{Abbildung-He-knows-and-loves-this-record-TAG}TAG对于“He knows and
    loves this record.”的分析}
%\caption{\label{Abbildung-He-knows-and-loves-this-record-TAG}TAG analysis of \emph{He knows and
%    loves this record.}}
\end{figure}%
在这个图中,主语和宾语节点都只出现一次。两个基础句法树的S节点都统制he(他)NP。同样,宾语NP节点属于两个VP。连接词连接用粗线标志的两个动词。Sarkar和Joshi提供了一种算法来决定哪些节点可以共享。这一结构初看起来可能有点奇怪,但是对于树邻接语法而言,这并不是派生结果树,相对而言派生过程树是重要的,因为要用派生过程树来计算语义。作者展示了为当前讨论的例子以及更加复杂的例子都可以正确地构建派生过程树。
%The subject and object nodes are only present once in this figure. The S nodes of both elementary trees both dominate the \emph{he} NP.
%In the same way, the object NP node belongs to both VPs. The conjunction connects two verbs indicated by the thick lines. Sarkar and Joshi provide an
%algorithm that determines which nodes are to be shared. The structure may look strange at first, but for TAG purposes, it is not the derived tree but rather the derivation tree that is important, since this is the one that is used to compute the semantic interpretation. The authors show that the derivation trees
%for the example under discussion and even more complex examples can be constructed correctly.

在中心语驱动的短语结构语法\indexhpsgc(HPSG)和词汇功能语法\indexlfgc(LFG)中,如在范畴语法中一样,结构构建是由价驱动的,分析上述例句没有问题;两个动词结合,然后结合之后的成分像简单动词一样。这一分析见图~\vref{Abbildung-He-knows-and-loves-this-record-HPSG}。这一分析与图~\ref{Abb-CG-Koordination-V}所示的范畴语法的分析相似。\footnote{%
在依存语法\indexdgc(Dependency Grammar)中也可以有相对应的分析。虽然\tes 最初的分析并不相同。参看\ref{sec-dg-coordination}的讨论。
}
%In theories such as HPSG\indexhpsg and LFG\indexlfg where structure building is, as in Categorial Grammar, driven by valence, the above sentence is unproblematic:
%both verbs are conjoined and then the combination behaves like a simple verb. The analysis of this is given in Figure~\vref{Abbildung-He-knows-and-loves-this-record-HPSG}. 
%This analysis is similar to the Categorial Grammar analysis in
%Figure~\ref{Abb-CG-Koordination-V}.\footnote{%
%  A parallel analysis in Dependency Grammar\indexdg is possible as well. \tes's original analysis
%  was different though. See Section~\ref{sec-dg-coordination} for discussion.
%}
\begin{figure}
\centering
\begin{forest}
sm edges
[S
	[NP
		[he;他]]
	[VP
		[V
			[V
				[knows;知道]]
			[and;和]
			[V
				[loves;喜欢]]]
		[NP
			[this record;这 唱片,roof]]]]
\end{forest}
\caption{\label{Abbildung-He-knows-and-loves-this-record-HPSG}基于选择的方式对于“He knows and
    loves this record.”的分析}
%\caption{\label{Abbildung-He-knows-and-loves-this-record-HPSG}Selection"=based analysis of \emph{He knows and
%    loves this record.} in tree notation}
\end{figure}%
按照Goldberg的插入分析,也可以接受这种方法来分析并列。knows(知道)和loves(喜欢)可以首先被插入到并列构式中,然后结果再被插入到及物构式中。具体来说,knows and loves(知道并且喜欢)的语义如何跟及物构式的意义组合还不清楚,因为这一短语的意义有点像\relation{and}(\relation{know}(x, y), \relation{love}(x, y)),即一个复杂事件带有至少两个开放论元槽x和y(并且有可能另外一个事件和一个取决于所用语义理论的世界变量)。Goldberg可能会接受图~\ref{Abbildung-He-knows-and-loves-this-record-TAG}所示的分析,来确保插入分析。
%With Goldberg's plugging analysis one could also adopt this approach to coordination: here, \emph{knows}
%and \emph{loves} would first be plugged into a coordination construction and the result would then be plugged into the transitive construction.
%Exactly how the semantics of \emph{knows and loves} is combined with that of the transitive construction is unclear since the meaning of this phrase
%is something like \relation{and}(\relation{know}(x, y), \relation{love}(x, y)), that is, a complex event with at least two open argument slots x and y 
%(and possibly additionally an event and a world variable depending on the semantic theory that is used). Goldberg would probably have to adopt an analysis such as the one in 
%Figure~\ref{Abbildung-He-knows-and-loves-this-record-TAG} in order to maintain the plugging analysis.

Croft一定必须采用TAG分析,因为动词已经出现在构式当中。对于(\ref{Beispiel-Gapping-Steedman})中的例子,Goldberg和Croft可能会采用图~\vref{Abbildung-He-gave-george-a-book-and-martha-a-record-TAG}所示的TAG分析:
%Croft would definitely have to adopt the TAG analysis since the verb is already present in his constructions. For the example in (\ref{Beispiel-Gapping-Steedman}),
%both Goldberg and Croft would have to draw from the TAG analysis in Figure~\vref{Abbildung-He-gave-george-a-book-and-martha-a-record-TAG}.
\begin{figure}
\centering
\begin{forest}
sm edges
[\phantom{S}
  [S, no edge
	[NP, name=np11
		[he;他]]
	[VP, name=vp1
		[V, name=v1 [gave;给]]
		[NP [George;George]]
	        [NP [a book;一 书,roof]]]]
  [VP,name=vpcoord, no edge [and, name=and, no edge]]
  [S, name=s2, no edge
    [VP, name=vp2
      [NP [Martha;Martha]]
      [NP [a record;一 唱片,roof]]]]]
\draw (s2.south)--(np11.north)
      (vp2.south)--(v1.north);
\draw[thick] (vpcoord.south)--(vp1.north)
             (vpcoord.south)--(vp2.north)
             (vpcoord.south)--(and.north);
\end{forest}
\caption{\label{Abbildung-He-gave-george-a-book-and-martha-a-record-TAG}TAG对于“He
    gave George a book and Martha a record.”的分析}
%\caption{\label{Abbildung-He-gave-george-a-book-and-martha-a-record-TAG}TAG analysis of \emph{He
%    gave George a book and Martha a record.}}
\end{figure}%

\noindent
这一分析的结果是需要引入不连续\isc{成分!不连续}\is{constituent!discontinuous}成分。因为并列允许很多变体,所以构式的所有论元之间都可以有空位。一个包含及物动词的例子见(\mex{1}):
%The consequence of this is that one requires discontinuous\is{constituent!discontinuous} constituents. Since coordination allows a considerable number
%of variants, there can be gaps between all arguments of constructions. An example with a ditransitive verb is given in (\mex{1}):
\ea
\gll He gave George and sent Martha a record.\\
     他 给 George 并且 寄给 Marhta 一 唱片\\
\mytrans{他给George一张唱片并且寄给Martha一张唱片。}
%He gave George and sent Martha a record.
\z
参看 \citew{Crysmann2003c}和 \citew{BS2004a}为特定并列结构假设不连续成分的中心语驱动的短语结构语法分析\indexhpsgc。
%See  \citew{Crysmann2003c} and  \citew{BS2004a} for HPSG analyses\indexhpsg that assume discontinuous constituents for particular
%coordination structures.

这些分析得出的结论是,特定成分紧邻出现并且这种紧邻出现与特定的意义相联系这一论述被很大程度上削弱了。说话者真正习得的能力是中心语必须与处在句子中某处的论元共现并且中心语涉及的所有要求都必须满足($\theta$"=准则, 一致/完整性,空\subcatlc)。中心语自身并不一定与它们的论元紧邻出现。参看\ref{Abschnitt-musterbasiert}关于语言习得基于模式模型的讨论。
%The result of these considerations is that the argument that particular elements occur next to each
%other and that this occurrence is associated with a particular meaning is considerably
%weakened. What competent speakers do acquire is the knowledge that heads must occur with their
%arguments somewhere in the utterance and that all the requirements of the heads involved have to
%somehow be satisfied ($\theta$"=Criterion, coherence/completeness, empty \subcatl).  The heads
%themselves need not necessarily occur directly adjacent to their arguments. See the discussion in
%Section~\ref{Abschnitt-musterbasiert} about pattern"=based models of language acquisition.

计算图~\ref{Abbildung-He-gave-george-a-book-and-martha-a-record-TAG}所示复杂结构的语义贡献不可能非常简单。在树邻接语法\indextagc(TAG)中,除了派生过程树之外还有派生结果树用于计算语言对象的语义。构式语法没有这种分层表征。这里讨论的句子的意义怎样从它们的组成部分派生出来对于短语方法来说仍然是开放的。
%The computation of the semantic contribution of complex structures such as those in
%Figure~\ref{Abbildung-He-gave-george-a-book-and-martha-a-record-TAG} is by no means trivial. In
%TAG\indextag, there is the derivation tree in addition to the derived tree that can then be used to
%compute the semantic contribution of a linguistic object. Construction Grammar does not have this
%separate level of representation. The question of how the meaning of the sentences discussed here is
%derived from their component parts still remains open for phrasal approaches.

总结关于语言习得的这一节,我们认为价表征是语言习得的结果,因为在话语中各种可能的构型对于构建依存关系是必须的。也可以参看 \citew[\page 439]{Behrens2009a}相似的结论。
%Concluding the section on language acquisition, we assume that a valence representation is the
%result of language acquisition, since this is necessary for establishing the dependency relations in
%various possible configurations in an utterance. See also  \citew[\page 439]{Behrens2009a} for a similar conclusion. 

\section{来自心理语言学和神经语言学的证据}

这一节包括三个部分:第一个部分比较用词汇规则、不完全赋值、用短语方法析取来分析价交替。在\ref{sec-psycho-lv}中,我们讨论了解释轻动词构式的方法,\ref{sec-neuro-linguistics}讨论神经语言学的发现。
%This section has three parts: in the first part we compare approaches which assume that valence
%alternations are modeled by lexical rules, underspecification, or disjunctions with phrasal
%approaches. In Subsection~\ref{sec-psycho-lv} part we discuss approaches to light verb constructions and Subsection~\ref{sec-neuro-linguistics}
%is devoted to neurolinguistic findings.

\subsection{词汇规则 vs.\ 短语结构}
\label{sec-lr-phrasal-psycho}
\mbox{} \citet[\S~1.4.5]{Goldberg95a}使用来自心理语言学实验的证据来反对词汇方法使用词汇规则来说明论元结构的交替: \citet{CT88a}展示(\mex{1})所示的带有真正词汇歧义的句子与包含具有相同核心语义的动词的句子具有不同的处理时间。
%\mbox{} \citet[Section~1.4.5]{Goldberg95a} uses evidence from psycholinguistic experiments to argue against lexical
%approaches that use lexical rules to account for argument structure alternations:  \citet{CT88a}
%showed that sentences with true lexical ambiguity like those in (\mex{1}) and sentences with two
%verbs with the same core meaning have different processing times.
\eal
\ex
\gll Bill set the alarm clock onto the shelf.\\
     Bill 放置 \textsc{det} 闹铃 钟表 \textsc{prep} \textsc{det} 书架\\
\mytrans{Bill将闹钟放置在书架上。} 
%Bill set the alarm clock onto the shelf.
\ex
\gll Bill set the alarm clock for six.\\
     BIll 设置 \textsc{det} 闹铃 钟表 \textsc{prep} 六\\
\mytrans{Bill将闹铃设为六点。} 
%Bill set the alarm clock for six.
\zl
\eal
\ex
\gll Bill loaded the truck onto the ship.\\
     Bill 装车 \textsc{det} 卡车 \textsc{prep} \textsc{det} 轮船\\
\mytrans{Bill将卡车装在轮船上。} 
%Bill loaded the truck onto the ship.
\ex
\gll Bill loaded the truck with bricks.\\
     Bill 装车 \textsc{det} 卡车 \textsc{prep} 砖\\
\mytrans{Bill将卡车装满了砖。} 
%Bill loaded the truck with bricks.
\zl
词汇歧义造成的错误比使用同一个动词所犯的错误会使得处理时间增加得更多。实验证明,(\mex{-1})中两个句子处理时间差异比(\mex{0})中两个句子处理时间差异更大。(\mex{0}a)和(\mex{0}b)中句子处理时间差异可以通过使用不同的短语构式来解释。在基于词库的方法中,要解释这种差异,就需要假设其中一个词项更加基础,即储存在心理词典中,而另外一个是从储存的词项派生而来的。使用词汇规则会花费时间,但是因为词项是相互联系的,所以花费的总时间比处理两个没有联系的词项更短\citep[\page 405]{Mueller2002b}。
%Errors due to lexical ambiguity cause a bigger increase in processing time than errors in the use of
%the same verb. Experiments showed that there was a bigger difference in processing times for the
%sentences in (\mex{-1}) than for the sentences in (\mex{0}). The difference in processing times
%between (\mex{0}a) and (\mex{0}b) would be explained by different preferences for phrasal
%constructions. In a lexicon-based approach one could explain the difference by assuming that one
%lexical item is more basic, that is, stored in the mental dictionary and the other is derived from
%the stored one. The application of lexical rules would be time consuming, but since the lexical
%items are related, the overall time consumption is smaller than the time needed to process
%two unrelated items \citep[\page 405]{Mueller2002b}.

换一种方法,可以假设这两种价模式的词项都是词汇规则作用的结果。与短语构式一样,词汇规则也会有不同的偏向。这就显示,词汇方法也可以很好地解释实验结果,所以这并不能促使我们选择短语方法。
%Alternatively one could assume that the lexical items for both valence patterns are the result of
%lexical rule applications. As with the phrasal constructions, the lexical rules would have different
%preferences. This shows that the lexical approach can explain the experimental results as well, so
%that they do not force us to prefer phrasal approaches.

 \citet[\page 18]{Goldberg95a}认为词汇方法必须假设load(装车)有两个不同的意义,所以load(装车)所在的变换式会与具有完全不同意义的两个动词所在的句子表现相同。上面讨论的实验显示这种预测是不对的,所以词汇分析就证明是错的。但是,正如 \citew[\S~11.11.8.2]{MuellerGTBuch1}所指出的那样,这一论述有两个漏洞:我们假设允准(\mex{0}a)构式的意义是 C$_1$,而允准(\mex{0}b)构式的意义是 C$_2$。按照这一假设,按照词汇分析方法,两个词项的语义应该如(\mex{1})所示。load(\ldots)是短语分析方法假设的动词的语义贡献。
% \citet[\page 18]{Goldberg95a} claims that lexical approaches have to assume two variants of \emph{load}
%with different meaning and that this would predict that \emph{load} alternations would behave like
%two verbs that really have absolutely different meanings. The experiments discussed above show that
%such predictions are wrong and hence lexical analyses would be falsified. However, as was shown in
% \citew[Section~11.11.8.2]{MuellerGTBuch1}, the argumentation contains two flaws: let's assume that the construction
%meaning of the construction that licenses (\mex{0}a) is C$_1$ and the construction meaning of the
%construction that licenses (\mex{0}b) is C$_2$. Under such assumptions the semantic contribution of
%the two lexical items in the lexical analysis would be (\mex{1}). load(\ldots) is the contribution
%of the verb that would be assumed in phrasal analyses.
\ea
\begin{tabular}[t]{@{}l@{~}l@{~}l@{}}
a. & load (onto): & C$_1$ $\wedge$ load(\ldots)\\
b. & load (with): & C$_2$ $\wedge$ load(\ldots)\\
\end{tabular}
\z
(\mex{0})显示了,词项的部分意义相同。所以,我们预测load非偏向论元实现的处理会比set非偏向语义的处理更加简单:在后一个例子中,需要激活一个全新的动词;而在第一个案例中,部分意义已经被激活了。\footnote{%
   也可以参看 \citew[\page 64--65]{Croft2003a}针对Goldberg对这里对实验结果的解读的简短反驳。%
}
%(\mex{0}) shows that the lexical items partly share their semantic contribution. We hence predict
%that the processing of the dispreferred argument realization of \emph{load} is simpler than the
%dispreferred meaning of \emph{set}: in the latter case a completely new verb has to be activated
%while in the first case parts of the meaning are activated already.\footnote{%
%   See also  \citew[\page 64--65]{Croft2003a} for a brief rejection of Goldberg's interpretation of the
%   experiment that corresponds to what is said here.%
%}
 \citet[\page 107]{Goldberg95a}对于(\mex{1})所示的处所变换式\isc{处所变换式}\is{locative alternation}反对使用基于词汇规则的方法,因为按照她的观点,基于词汇规则的方法必须假设其中一个词汇形式更加基础。
% \citet[\page 107]{Goldberg95a} argues against lexical rule-based approaches for locative
%alternations\is{locative alternation} like (\mex{1}), since according to her such approaches have to assume that one of the verb forms has to be the more
%basic form.
\eal
\ex
\gll He loaded hay onto the wagon.\\
     他 装车 干草 \textsc{prep} \textsc{det} 货车\\
\mytrans{他往货车中装干草。} 
%He loaded hay onto the wagon.
\ex
\gll He loaded the wagon with hay.\\
     他 装车 \textsc{det} 卡车 \textsc{prep} 干草\\
\mytrans{他将卡车装满了}
%He loaded the wagon with hay.
\zl
她认为这是有问题的,因为我们对于哪一个是基础的哪一个是派生的并没有明显的语感。她认为短语方法的优势在于很多构式可以不用假设哪一个更加基础而联系在一起。存在两个短语模式,动词可以在其中一个构式中出现。这一批评可以通过两种方式来解决:第一可以向类型层级中引入两个词汇类型(\type{onto-verb}和\type{with-verb}) 。这两个类型对应着用于分析(\mex{0}a)和(\mex{0}b)的价框架。这些类型可以有一个共同的与所有spray(喷洒)/load(装车)动词相关的上位类型 (\type{onto-with-verb})。其中一个下位类型或者动词的相应词项就是偏向的成员。这对应着词库中的析取,不同之处在于短语方法在短语构式集合中假设了析取。
%She remarks that this is problematic since we do not have clear intuitions about what the basic and
%what the derived forms are. She argues that the advantage of phrasal approaches is that various
%constructions can be related to each other without requiring the assumption that one of the
%constructions is more basic than the other. There are two phrasal patterns and the verb is used in
%one of the two patterns. This criticism can be addressed in two ways: first one could introduce two
%lexical types (for instance \type{onto-verb} and \type{with-verb}) into a type hierarchy. The two
%types correspond to two valence frames that are needed for the analysis of (\mex{0}a) and
%(\mex{0}b). These types can have a common supertype (\type{onto-with-verb}) which is relevant for all
%\emph{spray}/\emph{load} verbs. One of the subtypes or the respective lexical item of the verb is
%the preferred one. This corresponds to a disjunction in the lexicon, while the phrasal approach
%assumes a disjunction in the set of phrasal constructions. 
%BCPW2005 nehmen eine Familie von Lexikoneinträgen an.

这种方法的一种变体是假设load(装车)的词汇描述只是包含描述所有spray(喷洒)/load(装车)动词的上位类型。因为模板理论方法假设是话语模板的所有结构只包含最为具体的类型(参看 \citew{King99a-u}和 \citew[\page 21]{ps2}),所以说明具有\type{onto-with-verb}类型的像load(装车)一样的动词就足够了。因为这一类型有两个下位类型,所以在真实模型中,load必须是\type{onto-verb}或\type{with-verb}。\footnote{%
	这一分析并不允许对于其中一种模板实现指定具体偏向动词,因为词库只包括概括的类型。
}
%A variant of this approach is to assume that the lexical description of \emph{load} just contains
%the supertype describing all \emph{spray}/\emph{load} verbs. Since model theoretic approaches
%assume that all structures that are models of utterances contain only maximally specific types (see
%for instance  \citew{King99a-u} and  \citew[\page 21]{ps2}), it is sufficient to say about verbs like
%\emph{load} that they are of type \type{onto-with-verb}. As this type has exactly two subtypes,
%\emph{load} has to be either \type{onto-verb} or \type{with-verb} in an actual model.\footnote{%
%  This analysis does not allow the specification of verb specific preferences for one of the realization
%  patterns since the lexicon contains the general type only.
%}

第二种方法是坚持词汇规则并且为列举在词库中的动词的词干假设一个单一的表征。另外,假设两个词汇规则将基本词项映射到另外的经过屈折之后可以在句法中使用的另外的词项。这两个词汇规则可以通过一个类型层级中的类型描述,从而有一个共同的上位类型。这就可以描述词汇规则之间的共同点。因此,我们就可以达到与短语构式相同的效果(两个词汇规则vs.\ 两个短语构式)。唯一的差异是在词汇方法中,动作在一个更深的层级上,即词库\citep[\page 405--406]{Mueller2002b}。
%A second option is to stick with lexical rules and to assume a single representation for the root of
%a verb that is listed in the lexicon. In addition, one assumes two lexical rules that map this basic
%lexical item onto other items that can be used in syntax after being inflected. The two lexical
%rules can be described by types that are part of a type hierarchy and that have a common
%supertype. This would capture commonalities between the lexical rules. We therefore have the same
%situation as with phrasal constructions (two lexical rules vs.\ two phrasal constructions). The only
%difference is that the action is one level deeper in the lexical approach, namely in the lexicon \citep[\page 405--406]{Mueller2002b}. 

关于(\mex{1}c)所示结果构式的处理是平行的:
%The argumentation with regard to the processing of resultative constructions like (\mex{1}c) is parallel:
\eal
\ex
\gll He drinks.\\
     他 喝酒\\
\mytrans{他喝酒。} 
%He drinks.
\ex
\gll He drinks the milk.\\
     他 喝 \textsc{det} 牛奶\\
\mytrans{他喝牛奶。} 
%He drinks the milk.
\ex
\gll He drinks the pub empty.\\
     他 喝酒 \textsc{det} 酒馆 空\\
\mytrans{他把酒馆里的酒都喝光了。} 
%He drinks the pub empty.
\zl
当人类分析一个句子时,他们逐步构建结构。如果听到与当前假设不兼容的词,句法分析过程就会终止或者当前假设就会修改。在(\mex{0}c)中,the pub并不对应着drink的常规及物用法,所以相应的假设就会需要修改。在短语方法中,就会使用结果构式而非及物构式。在词汇分析中,就会使用结果词汇规则允准的词项而不是二价的词项。一般而言,建立句法结构和词库会为我们的处理能力提出不同要求。但是,当分析(\mex{0}c)时,drink(喝)的词项已经被激活了,我们只能使用另外一个。对于我们来说,还不清楚心理语言学实验是否可以区分这两种方法,但是好像不太可能。
%When humans parse a sentence they build up structure incrementally. If one hears a word that is
%incompatible with the current hypothesis, the parsing process breaks down or the current hypothesis
%is revised. In (\mex{0}c) \emph{the pub} does not correspond to the normal transitive use of
%\emph{drink}, so the respective hypothesis has to be revised. In the phrasal approach the resultative
%construction would have to be used instead of the transitive construction. In the lexical analysis
%the lexical item that is licensed by the resultative lexical rule would have to be used rather than
%the bivalent one. Building syntactic structure and lexicon access in general place different
%demands on our processing capacities. However, when (\mex{0}c) is parsed, the lexical items for
%\emph{drink} are active already, we only have to use a different one. It is currently unclear to us
%whether psycholinguistic experiments can differentiate between the two approaches, but it seems to
%be unlikely.

\subsection{轻动词}
\label{sec-psycho-lv}

 \citet*{WJKP2014a}报告了很多验证处理轻动词构式不同方法假设的实验。(\mex{1}a)展示了一个典型的轻动词构式:take是一个与名词组合的轻动词,该名词充当主要的谓项。
% \citet*{WJKP2014a} report on a number of experiments that test predictions made by
%various approaches to light verb constructions. (\mex{1}a) shows a typical light verb construction:
%\emph{take} is a light verb that is combined with the nominal that provides the main
%predication. 
\eal
\ex 
\gll take a walk to the park\\
     带走 一 散步 \textsc{prep} \textsc{det} 公园\\
\mytrans{散步到公园}
%take a walk to the park
\ex
\gll walk to the park\\
     散步 \textsc{prep} \textsc{det} 公园\\
\mytrans{散步到公园} 
%walk to the park
\zl
%%  \citet{HK93a-u} assume that (\mex{1}b) is derived from (\mex{1}a).
%% 
%% The structure they assume for the light verb construction is more complex than the one for the non-light verb in (\mex{0}b). This is due
%% to the fact that it is assumed that the noun \emph{walk} incorporates into a v node by
%% head-movement. As  \citet{WJKP2014a} point out, this makes wrong predictions as far as processing is concerned. This
%% approach predicts that light verb constructions should be easier to process, which is not borne
%% out. Furthermore, head-movement approaches assume that the light verb constructions differ in their
%% underlying and surface structure from the non-light verb constructions. However, to the extent that
%% structural priming reflects syntactic structure, data from priming experiments suggest that the
%% light verb constructions and non-light verb constructions share the same kind of syntax
%% \citep{Wittenberg:2012mz}.  

%% Therefore there remain two classes of approaches as psycholinguistically plausible candidates
%% \citep{WP2011a}
 \citet{WP2011a}考察了两个在心理学上合理的轻动词构式分析理论。短语方法认为轻动词构式是与语义联系的需要储存的对象\citep{Goldberg2003a}。另外一种组合的观点认为结构的语义是事件名词的语义和轻动词的语义的组合\citep{Grimshaw97a-u,Butt2003a-u,Jackendoff2002a-u,CJ2005a,MuellerPersian,BPW2008a-u}。因为轻动词构式非常常见(\citealp*{Pinango:2006qy};\citealp[\page 399]{WP2011a}),所以认为轻动词构式是需要存储的单位并且动词和宾语都是固定的短语方法推测轻动词构式会比(\mex{1})所示的非轻动词构式获取得更快\citep[\page 396]{WP2011a}。
% \citet{WP2011a} examined two psychologically plausible theories of light verb constructions.  The phrasal approach 
% assumes that light verb constructions are stored objects associated with semantics \citep{Goldberg2003a}.
%The alternative compositional view assumes that the semantics is computed as a fusion of the
%semantics of the event noun and the semantics of the
%light verb \citep{Grimshaw97a-u,Butt2003a-u,Jackendoff2002a-u,CJ2005a,MuellerPersian,BPW2008a-u}.  
%Since light verb constructions are extremely frequent (\citealp*{Pinango:2006qy};
%\citealp[\page 399]{WP2011a}), the phrasal approaches assuming that
%light verb constructions are stored items with the object and verb fixed predict that light verb
%constructions should be retrievable faster than non-light verb constructions like (\mex{1}) \citep[\page
%  396]{WP2011a}. 
\ea
\gll take a frisbee to the park\\
     带走 一 飞盘 \textsc{prep} \textsc{det} 公园\\
\mytrans{带个飞盘去公园}
%take a frisbee to the park
\z
但是不是这样的。正如Wittenberg和Piñango发现的那样,在一定的允准条件下,处理方面没有差异(像英语一样的VO语言中的名词,以及像德语一样OV语言中的动词)。
%This is not the case. As Wittenberg and Piñango found, there is no difference in processing at the licensing
%condition (the noun in VO languages like English and the verb in OV languages like German). 

但是, \citet{WP2011a}发现在轻动词构式处理“之后”有一个增加的处理负担300ms。他们认为在句法组合之后名词与动词发生语义组合。虽然句法组合很快,但是语义计算需要另外的资源,这些资源大约需要300ms。动词提供体信息并且整合名词成分的语义。语义角色融合了。如果说整个轻动词构式是一个需要储存的与整体意义联系的单位的话,资源花费效应就不会出现(第404页)。我们可以得出结论,Wittenberg和Piñango的结论与词汇方案兼容,而与短语观点不兼容。
%However,  \citet{WP2011a} found an increased processing load 300ms \emph{after} the light verb construction is
%processed. The authors explain this by assuming that semantic integration of the noun with the
%verbal meaning takes place after the syntactic
%combination. While the syntactic combination is rather fast, the semantic computation takes
%additional resources and this is measurable at 300ms. The verb contributes aspectual information and integrates
%the meaning of the nominal element. The semantic roles are fused. The resource consumption effect
%would not be expected if the complete light verb construction were a stored item that is
%retrieved together with the complete meaning (p.\,404). We can conclude that Wittenberg and
%Piñango's results are compatible with the lexical proposal, but are
%incompatible with the phrasal view. % suggested by  \citet{Goldberg2003a}. 

\subsection{来自神经语言学的证据}
\label{sec-neuro-linguistics}

%\subsection{Particle Verbs}
\mbox{} \citet*{PCShandbookCxG}讨论了神经语言学事实并将其与CxG语法理论联系在一起。一个重要的发现是错误的词(词项)导致的大脑反应与不正确的词串(即句法组合)导致的大脑反应存在差异。这表明存在一个经验性的基础来决定这一问题。
%\mbox{} \citet*{PCShandbookCxG} discuss neurolinguistic facts and relate them to the CxG view of grammar
%theory. One important finding is that deviant words (lexical items) cause brain responses that differ in polarity
%from brain responses on incorrect strings of words, that is, syntactic combinations. This suggests
%that there is indeed an empirical basis for deciding the issue.

就(\mex{1})所示的致使移动构式\isc{构式!致使移动构式}\is{construction!Caused"=Motion}的标准例子,作者写道:
%Concerning the standard example of the Caused"=Motion Construction\is{construction!Caused"=Motion} in (\mex{1}) the authors write the
%following:
\ea
\gll She sneezed the foam off the cappuccino.\\
     她 打喷嚏 \textsc{det} 泡沫 \textsc{prep} \textsc{det} 卡布奇诺\\
\mytrans{她打喷嚏将泡沫从卡布奇诺咖啡上吹下来。}
%She sneezed the foam off the cappuccino.
\footnote{%
 \citew[\page 42]{Goldberg2006a}。
}
\z
\begin{quotation}
  这一连串大脑活动可能最初会导致动词sneeze(打喷嚏)与blow(吹风)的DCNA同时激活,并且因此与提到的句子共同激活。最终,一个一价动词和与另外动词联系的DCNA可能会导致前一个一价动词被归入到三价动词的范畴和DCNA集合中,在laugh NP off the stage(将NP笑下台阶)中的动词laugh也完成了相同的过程。\citep*{PCShandbookCxG}\footnote{%
  this constellation of brain activities may initially lead to the co"=activation of the verb \emph{sneeze}
  with the DCNAs for \emph{blow} and thus to the sentence mentioned. Ultimately, such co-activation of a
  one-place verb and DCNAs associated with other verbs may result in the former one-place verb being
  subsumed into a three-place verb category and DCNA set, a process which arguably has been
  accomplished for the verb \emph{laugh} as used in the sequence \emph{laugh NP off the stage}. 
  }
\end{quotation}
一个DCNA是一个分离的组合神经集合。关于DCNA的特征,作者写道:
%A DCNA is a discrete combinatorial neuronal assembly. Regarding the specifics of DCNAs the authors write that 
\begin{quotation}
除了将范畴连接起来,典型的DCNA还建立了范畴成分之间的时间顺序。不对时间顺序提出要求的DCNA(所以,原则上,相当于两个成分的AND单位)会组合顺序自由或允许置换的成分。\citep*[\page 404]{PCShandbookCxG}\footnote{%
Apart from linking categories together, typical DCNAs establish a temporal order between the
category members they bind to. DCNAs that do not impose temporal order (thus acting, in principle,
as AND units for two constituents) are thought to join together constituents whose sequential order
is free or allow for scrambling.
}
% allow for scrambling steht wirklich so im quote
\end{quotation}
我认为这一观点与上面所述的词汇观点完全兼容:词项或者DCNA需要特定论元出现。存在一条词汇规则,允准某个不及物动词在blow(吹风)价框架的允准条件下进入致使移动构式\isc{构式!致使移动构式}\is{construction!Caused"=Motion}中。
%I believe that this view is entirely compatible with the lexical view outlined above: the lexical
%item or DCNA requires certain arguments to be present. A lexical rule that relates an intransitive verb
%to one that can be used in the Caused"=Motion Construction\is{construction!Caused"=Motion} is an
%explicit representation of what it means to activate the valence frame of \emph{blow}.

作者引用了他们早期的成果\citep*{CSP2010a}并且主张小词动词是词汇单位,允许不连续实现(第21页)。他们将论述限定在经常出现的小词动词上。这一论述当然与我们这里的假设兼容,但是当涉及到小词动词完全能产的用法时大脑活动方面的差异非常值得注意。例如,在德语中任何语义合适的单价动词可以与体小词los(开始)组合:lostanzen(开始跳舞)、loslachen(开始笑)、lossingen(开始唱歌)\ldots。与此相似,单价动词与an组合在一起表示“方向朝向”(directed-towards)也是能产的:anfahren(驶向)、laugh in the direction of(朝着某个方向笑)、ansegeln(航向)\ldots{}(参看 \citew{Stiebels96a}对多种能产模式的讨论)
%The authors cite earlier work \citep*{CSP2010a} and argue that particle verbs are lexical objects,
%admitting for a discontinuous realization despite their lexical status
%(p.\,21). They restrict their claim to frequently occurring particle verbs. This claim is of course
%compatible with our assumptions here, but the differences in brain behavior are interesting when it
%comes to fully productive uses of particle verbs. For instance any semantically appropriate monovalent verb in German can
%be combined with the aspectual particle \emph{los}: \emph{lostanzen} `start to dance',
%\emph{loslachen} `start to laugh', \emph{lossingen} `start to sing', \ldots. Similarly, the
%combination of monovalent verbs with the particle \emph{an} with the reading \emph{directed-towards} is
%also productive: \emph{anfahren} `drive towards', \emph{anlachen} `laugh in the direction of',
%\emph{ansegeln} `sail towards', \ldots{} (see  \citew{Stiebels96a} on various productive
%patterns). 
%As was argued in Section~\label{sec-bar-derivation}, this pattern of particle verb formation
%interacts with derivational morphology.
值得注意的问题是遵循这种模式但是出现频率比较低的小词动词表现如何。就实验证据而言,这仍然是一个开放的问题,但是正如我下面要进行论证的, \citet{Mueller2003a}提出的对于小词动词的词汇方案跟两种结果都兼容。
%The interesting question is how particle verbs behave that follow these patterns but occur with low
%frequency. This is still an open question as far as the experimental evidence is concerned, but as
%I argue below lexical proposals to particle verbs as the one suggested by  \citet{Mueller2003a} are
%compatible with both possible outcomes.

总结一下迄今为止的讨论,词汇方法与搜集到的神经生物学的证据都兼容并且就小词动词而言词汇方法似乎比 \citet[\S~2]{Booij2002a}和 \citet{Blom2005a}提出的短语方法(参看\ref{sec-particle-verbs-phrasal}的讨论)更加适合。但是,总体上来说,一个不连续词项到底意味着什么仍然是一个开放的问题。不连续词这个概念的历史很悠久\citep{Wells47a},但是就该观点并没有很多的形式框架。 \citet*{NSW94a}在基于线性化的框架内提出了一种表征方式,这种表征方式由 \citet{Reape94a}和 \citet*[\page 244--248]{Kathol95a}指出, \citet{Crysmann2002a}将这种分析详细地呈现出来。Kathol为aufwachen(醒来)设置的词项如(\mex{1})所示:
%Summarizing the discussion so far, lexical approaches are compatible with the accumulated neurobiological evidence 
%and as far as particle verbs are concerned they seem to be better suited than the phrasal proposals
%by  \citet[Section~2]{Booij2002a} and  \citet{Blom2005a} (See Section~\ref{sec-particle-verbs-phrasal}
%for discussion). However, in general, it remains an open question what 
%it means to be a discontinuous lexical item. The idea of discontinuous words is pretty old
%\citep{Wells47a}, but there have not been many formal accounts of this idea.  \citet*{NSW94a} suggest
%a representation in a linearization-based framework of the kind that was proposed by
% \citet{Reape94a} and  \citet*[\page 244--248]{Kathol95a} and  \citet{Crysmann2002a} worked out such
%analyses in detail. Kathol's lexical item for \emph{aufwachen} `to wake up' is given in (\mex{1}):
\eas
\label{le-aufwachen-Kathol}
\mbox{aufwachen (根据\citealp[\page 246]{Kathol95a}):}\\
%\mbox{\emph{aufwachen} (following \citealp[\page 246]{Kathol95a}):}\\
\begin{tabular}{@{}l@{}}
\onems{
\ldots$|$head   \ibox{1} \type{verb}\\
\ldots$|$vcomp  \eliste\\
dom \liste{ \onems{ \phonliste{ wachen }\\
                      \ldots$|$head  \ibox{1}\\
                      \ldots$|$vcomp \sliste{ \ibox{2} }\\
                    }} $\bigcirc$
    \liste{ \onems[vc]{ \phonliste{ auf\/ }\\
                      synsem \ibox{2} \ms{ \ldots$|$head \onems[sepref~]{flip $-$\\
                                                                     }\\
                                         }\\
                    }
            }\\
}
\end{tabular}
\zs
词汇表征包含取值为列表的特征\textsc{dom},该特征包含对于主要动词和小词的描述(参见\ref{sec-discontinuous-constituents-HPSG}的具体论述)。\domlc 是一个包含中心语依存成分的列表。只要不违背线性化规则,依存成分可以以任何顺序排列\citep{Reape94a}。小词和主要动词之间的依存是通过特征\vcompc 的取值来描写的,该特征是一个论元选择的价特征,该论元可以与其中心语组成一个复杂谓项。shuffle算子$\bigcirc$会将两个列表组合起来而不限定两个列表中元素的顺序,即“wachen,auf”和“auf,wachen”都是可能的。小标记\type{vc}是在句子中指派一个拓扑场。
%The lexical representation contains the list-valued feature \textsc{dom} that contains a description of the
%main verb and the particle (see Section~\ref{sec-discontinuous-constituents-HPSG} for details). The \doml is a list that contains the dependents of a head. The
%dependents can be ordered in any order provided no linearization rule is violated
%\citep{Reape94a}. The dependency between the particle and the main verb was characterized 
%by the value of the \vcomp feature, which is a valence feature for the selection of arguments that
%form a complex predicate with their head. The shuffle operator $\bigcirc$ concatenates two lists
%without specifying an order between the elements of the two lists, that is, both \emph{wachen},
%\emph{auf} and \emph{auf}, \emph{wachen} are possible. The little marking \type{vc} is an assignment
%to a topological field in the clause.

我批评这种基于线性化的方案,因为这种方案没有说清,声称小词只是在其动词域内线性化的分析怎样解释(\mex{1})所示的涉及复杂句法结构的句子\citep{Mueller2007d}。德语是一种V2语言并且一个成分前置到限定动词的前面位置通常被描述为一种非局部依存现象:即,即便是赞成基于线性化分析的学者也不认为句首位置是由成分的简单排序来填充的\citep{Kathol2000a,Mueller99a,Mueller2002b,TBjerre2006a}。\footnote{%
  在中心语驱动的短语结构语法\indexhpsgc(HPSG)框架内工作的 \citet[\S~6.3]{Kathol95a}曾经为简单句提出过这种分析,但是后来放弃了这一观点。同样在HPSG框架内工作的 \citet{Wetta2011a}提出了一个纯粹的基于线性化的方法。与此相似,在依存语法\indexdgc(Dependency Grammar)中工作的 \citet{GO2009a}也认为在简单句中有一个简单依存结构,而需要一个特殊机制来解释嵌套句的提取。在 \citew{MuellerGS}中,我反驳了该方案的\isc{辖域}\is{s\textsc{cop}e}问题、简单句与复杂句并列的问题、跨界抽取\isc{跨界抽取}\is{Across the Board Extraction}以及明显的多重前置\isc{前置!明显多重前置}\is{fronting!apparent multiple}。也可以参见\ref{sec-linearization-problems-dg}。
}

%I criticized such linearization-based proposals since it is unclear how
%analyses that claim that the particle is just linearized in the domain of its verb can account for
%sentences like (\mex{1}), in which complex syntactic structures are involved \citep{Mueller2007d}. German is a V2 language
%and the fronting of a constituent into the position before the finite verb is usually described as
%some sort of nonlocal dependency; that is, even authors who favor linearization-based analyses do
%not assume that the initial position is filled by simple reordering of material
%\citep{Kathol2000a,Mueller99a,Mueller2002b,TBjerre2006a}.\footnote{%
%   \citet[Section~6.3]{Kathol95a} working in HPSG\indexhpsg suggested such an analysis for simple sentences, but later
%  changed his view.  \citet{Wetta2011a} also working in HPSG assumes a purely linearization-based
%  approach. Similarly  \citet{GO2009a} working in  Dependency Grammar\indexdg assume that there is a
%  simple dependency structure in simple sentences while there are special mechanisms to account for
%  extraction out of embedded clauses. I argue against such proposals in  \citew{MuellerGS} referring
%  to the s\textsc{cop}e\is{s\textsc{cop}e} of adjuncts, coordination of simple with complex sentences and Across the Board
%  Extraction\is{Across the Board Extraction} and apparent multiple frontings\is{fronting!apparent
%    multiple}. See also Section~\ref{sec-linearization-problems-dg}.
%}
\eal
\label{ex-complex-vf}
\ex
\gll {}[\sub{vf} [\sub{mf} Den Atem]  [\sub{vc} an]] hielt die ganze Judenheit.\footnotemark\\
       {}        {}        \textsc{det} 呼吸 {}    \partic{}  屏住  \textsc{det} 所有 犹太.社团\\
%       {}        {}        the breath {}    \partic{}  held  the whole Jewish.community\\
\mytrans{所有的犹太人都屏住了呼吸。}
%\mytrans{The whole Jewish community held their breath.}
\footnotetext{%
Lion Feuchtwanger, \emph{Jud Süß},第276页,摘自 \citew[\page 56]{Grubacic65a}。
}
\ex\label{bsp-wieder-an-tritt-zwei}
\gll {}[\sub{vf} [\sub{mf} Wieder] [\sub{vc} an]] treten auch die beiden Sozialdemokraten.\footnotemark\\
      {}         {}        再   {}        \partic{} 踢 也 \textsc{det} 两 社会.民主党\\
%      {}         {}        again   {}        \partic{} kick also the two Social.Democrats\\
\footnotetext{%
  taz, bremen, \zhdate{2004/05/24},第21页。
}
\mytrans{两个社会民主党的成员也再次参选公职了。} % check
%\mytrans{The two Social Democrats are also running for office again.} % check

\ex
\gll {}[\sub{vf} [\sub{vc} Los]        [\sub{nf} damit]]    geht es schon   am 15. April.\footnotemark\\
       {}        {}        \textsc{part}  {}        \textsc{adv} 开始 \expl{} 已经 \textsc{prep}.\textsc{det} 15 四月\\%
%       {}        {}        \textsc{part}  {}        there.with went it already at.the 15 April\\%
\footnotetext{%
        taz, \zhdate{2002/03/01},第8页。%
    }%
\mytrans{在四月十五日就已经开始了。}
%\mytrans{It already starts on April the 15th.}
\zl
从(\mex{0})所示的例子可以得出的结论是小词以复杂的方式与句子句法进行交互。这一现象可以通过 \citew[Chapter~6]{Mueller2002b}和 \citew{Mueller2003a}介绍的词汇方法来描写:主要动词选择动词性小词。通过假设wachen选择auf,就可以表征动词和小词之间的紧密联系。\footnote{%
   \citet[\page 197]{CSP2010a}写道:“结果提供了神经语言学的证据,证明短语动词是词项。实际上,我们发现的相对于不恰当的组合,存在的短语动词有更高的激活性,表明一个动词与其小词共同组成一个词汇表征,即一个单独词位,并且该词位对应着一个统一的皮层记忆电路,这一点与编码一个单独单词相近。”我认为我的分析与这一观点兼容。
}这一词汇分析提供了一种容易的方式来解释完全不透明的小词动词,例如an-fangen(开始)。但是,我们也主张用词汇的方法来解释透明小词动词,例如losfahren(开始驾驶)和jemanden/etwas anfahren(向什么人或物体开去)。这一分析涉及一个允准动词词项选择附接语小词的词汇规则。小词an和los可以修饰动词并且提供论元(例如an)和小词语义。这一分析显示可以与神经机制发现兼容:如果即便是低频透明的小词动词组合都作为一个单位储存,那么我在上面提到的著作中的相对概括的词汇规则就概括了大量词汇小词动词和它们相应的主要动词之间的关系。单个的小词动词将是一种特殊实现,其形式与不透明小词动词(例如anfangen)一致。如果真的发现低频的带有小词动词的组合引起大脑中的句法反应,这也可以解释:词汇规则允准一个可以选择一个副词性成分的词项。这种选择关系与NP“der Mut”(勇气)中限定词与名词之间的关系平行, \citet[\page 191]{CSP2010a}将这种组合当做句法组合来讨论。注意这一分析也与 \citet*{SPP2005a-u}所做的观察兼容:形态学词缀也会导致词汇反应。在我的分析中,主要动词的词干与另外一个选择小词的词干关联。这一词干与导致大脑中词汇激活模式的(派生或屈折)的形态学词缀组合。在这一组合之后,动词与小词组合,并且这种依存可以是词汇的或句法的,具体是词汇还是句法的取决于将来实验的结果。这一分析与两种结果都兼容。
%The conclusion that has to be drawn from examples like (\mex{0}) is that particles interact in
%complex ways with the syntax of sentences. This is captured by the lexical treatment that was
%suggested in  \citew[Chapter~6]{Mueller2002b} and  \citew{Mueller2003a}: the main verb selects for the verbal
%particle. By assuming that \emph{wachen} selects for \emph{auf}, the tight connection between verb
%and particle is represented.\footnote{%
% \citet[\page 197]{CSP2010a} write: ``the results provide neurophysiological evidence that
%  phrasal verbs are lexical items. Indeed, the increased activation that we found for existing
%  phrasal verbs, as compared to infelicitous combinations, suggests that a verb and its particle
%  together form one single lexical representation, i.\,e.\ a single lexeme, and that a unified
%  cortical memory circuit exists for it, similar to that encoding a single word''.
%I believe that my analysis is compatible with this statement.
%} Such a lexical analysis provides an easy way to account for fully
%nontransparent particle verbs like \emph{an-fangen} `to begin'. However, I also argued for a
%lexical treatment of transparent particle verbs like \emph{losfahren} `to start to drive' and
%\emph{jemanden/etwas anfahren} `drive directed towards somebody/something'. The
%analysis involves a lexical rule that licenses a verbal item selecting for an adjunct
%particle. The particles \emph{an} and \emph{los} can modify verbs and contribute arguments (in the
%case of \emph{an}) and the particle semantics. This analysis can be shown to be compatible with the
%neuro"=mechanical findings: if it is the case that even transparent particle verb combinations with
%low frequency are stored, then the rather general lexical rule that I suggested in the works cited above is the
%generalization of the relation between a large amount of lexical particle verb items and their respective main
%verb. The individual particle verbs would be special instantiations that have the form of the
%particle specified as it is also the case for non-transparent particle verbs like \emph{anfangen}.
%If it should turn out that productive combinations with particle verbs of low
%frequency cause syntactic reflexes in the brain, this could be explained as well: the lexical rule
%licenses an item that selects for an adverbial element. This selection would then be seen as
%parallel to the relation between the determiner and the noun in the NP \emph{der Mut} `the
%courage', which  \citet[\page 191]{CSP2010a} discuss as an example of a syntactic combination. Note
%that this analysis is also compatible with another observation made by  \citet*{SPP2005a-u}:
%morphological affixes also cause the lexical reflexes. In my analysis the stem of the main
%verb is related to another stem that selects for a particle. This stem can be combined with
%(derivational and inflectional) morphological affixes causing the lexical activation pattern in the
%brain. After this combination the verb is combined with the particle and the dependency can be
%either a lexical or a syntactic one, depending on the results of the experiments to be carried
%out. The analysis is compatible with both results.
% CSP2010:198 say they have to check less frequent items.

注意我的分析可以确保词汇完整性原则。所以我不同意 \citet*[\page 198]{CSP2010a}的观点,他认为他们“提供了证据证明潜在可分解的多词项自己可以像词一样,所以可以反对一个已被广泛接受的语言学原则,即词汇完整性原则。”我同意非透明的小词动词是多词词位,但是多次词位的存在并不能证明句法可以达到词内部形态学结构。 \citew{Mueller2002b,Mueller2002d}中曾经论及小词动词与明显短语习语具有相似性;并且得出结论,习语地位与词的地位并不相关。如(\mex{1})中的例子所示,因为明显短语习语与派生形态学并不强制语法学家放弃词汇完整性,所以可以得出结论的是小词动词并非是让人放弃词汇完整性原则的有说服力的证据。\footnote{%
  但是,参看 \citew{Booij2009a}针对词汇完整性提出的一些挑战。
}
%Note that my analysis allows the principle of lexical integrity to be maintained. I therefore do
%not follow  \citet*[\page 198]{CSP2010a}, who claim that they ``provide proof that potentially
%  separable multi-word items can nonetheless be word-like themselves, and thus against the validity
%  of a once well-established linguistic principle, the Lexical Integrity Principle''. I agree that
%non-transparent particle verbs are multi-word lexemes, but the existence of multi-word lexemes does
%not show that syntax has access to the word"=internal morphological structure. The parallel between
%particle verbs and clearly phrasal idioms was discussed in  \citew{Mueller2002b,Mueller2002d} and it
%was concluded that idiom"=status is irrelevant for the question of wordhood. Since the interaction of
%clearly phrasal idioms with derivational morphology as evidenced by examples like (\mex{1}) did not
%force grammarians to give up on lexical integrity, it can be argued that particle verbs are not
%convincing evidence for giving up the Lexical Integrity Principle either.\footnote{%
%  However, see  \citew{Booij2009a} for some challenges to lexical integrity.
%}
\eal
\ex
\gll Er hat ins Gras gebissen.\\
     他 \textsc{aux} \textsc{prep}.\textsc{det} 草地 一点\\
\mytrans{他战死沙场。}
%\gll Er hat ins Gras gebissen.\\
%     he has in.the grass bit\\
%\mytrans{He bit the dust.}
\ex 
\gll "`Heath Ledger"' kann ich nicht einmal schreiben, ohne dass mir sein ins Gras-Gebeiße wieder so
wahnsinnig leid tut% -- Den hatte ich so gerne.
\footnotemark\\
    \spacebr{}Heath Ledger \textsc{aux} 我 \textsc{neg} 甚至 写 \textsc{prep} \textsc{comp} 我 他的 \textsc{prep}.\textsc{det} 草地.一点 再次 这样
    疯狂 悲伤 做\\
\footnotetext{%
\url{http://www.coffee2watch.at/egala}. \zhdate{2012/03/23} 
}
\mytrans{写下“Heath Ledger”时,连我也无法不再次为其战死沙场而悲伤。}
%\gll "`Heath Ledger"' kann ich nicht einmal schreiben, ohne dass mir sein ins Gras-Gebeiße wieder so
%wahnsinnig leid tut% -- Den hatte ich so gerne.
%\footnotemark\\
%    \spacebr{}Heath Ledger can I not even write without that me his in.the grass.biting again so
%    crazy sorrow does\\
%\footnotetext{%
%\url{http://www.coffee2watch.at/egala}. 23.03.2012 
%}
%\mytrans{I cannot even write ``Heath Ledger'' without being sad again about his biting the dust.}
\zl
(\mex{0}b)中的例子涉及到环缀\gee 的不连续派生(\citealp[\S~3.4.3]{Luedeling2001a};\citealp[\page 324--327, 372--377]{Mueller2002b};\citealp[\S~2.2.1, \S~5.2.1]{Mueller2003a})。习语ins Gras beiß-(战死沙场)存在并且有习语义。参看 \citew{Sag2007a}可以解释(\mex{0})所示例子的词汇方法。
%The example in (\mex{0}b) involves the discontinuous derivation with the circumfix \gee
%(\citealp[Section~3.4.3]{Luedeling2001a}; \citealp[\page 324--327, 372--377]{Mueller2002b};
%\citealp[Section~2.2.1, Section~5.2.1]{Mueller2003a}). Still the parts of the idiom \emph{ins Gras
%  beiß-} `bite the dust' are present and with them the idiomatic reading. See  \citew{Sag2007a} for a lexical analysis of idioms that can
%explain examples like (\mex{0}).

% As was shown in  \citew{Mueller2002b,Mueller2002d,Mueller2003a,Mueller2006c} (German) particle verbs
% pattern in many respects with verbal complexes, resultative predicates, and other predicative
% constructions like \textsc{cop}ula constructions and \emph{consider} type predications.

% In our model the verb selects for a particle and then the two parts can be realized discontinuously.

% We think that further experiments are needed in order to establish what exactly has been
% measured. The authors mention on page 21--22 that morphologically complex structures also cause a
% lexical response in the brain. This is interesting since clearly syntactic constructions as (\ref)
% repeated here as (\mex{1}) interact with morphology. So it would be interesting to see the
% neuro-imaging results and respective explanations.
% \ea
% \gll die in Stücke / blutig getanzten Schuhe\\
%      the into pieces {} bloody danced shoes\\
% \mytrans{the shoes that were danced bloody / into pieces}
% \z

%% Eva Witttenberg noted several problems with this paper.
%% \subsection{Light Verb Constructions}

%% The last subsection showed that certain results from neuro linguistics are compatible with both the
%% lexicalist and the phrasal constructionalist view. This subsection briefly discusses results of
%% research on light verb constructions.  \citet{BBRBWA2009a} examined lightverb constructions
%% entertaining the hypotheis of  \citet{Butt2010a} that the lightverb is underspecified with regard to
%% its meaning and that the argument roles are (partly) contributed by the non-verbal element (see also
%% Section~\ref{sec-psycho-lv}). This is in contrast with a view by  \citet{Goldberg2003a} in which the
%% lightverb construction is a phrasal construction into which normal verbs are inserted and which
%% contributes the meaning.

%% The authors of the study examined MGG data of verbs that could be used in both light/non-light
%% constructions and verbs that are unambiguously non-light verbs. They examined the verbs in
%% isolation, in minimal context and in sentence context. They found that there are different
%% activations for light verbs and heavy verbs without any context (p.\,177). They interpreted their
%% findings as support for the lexical analysis suggested by  \citet{Butt2010a}. This is also the
%% analysis that we assume \citep{MuellerPersian}.

所以,虽然我认为中心语用不同价模式现象无法分出词汇方法和短语方法孰优孰劣(\ref{sec-lr-phrasal-psycho}),但是好像有方法可以验证,高频使用并且强组合的模式是应该分析为有一个固定形式和意义的固定块,还是应该被分析为组合性的。
%So, while I think that it is impossible to distinguish phrasal and lexical approaches for phenomena
%where heads are used with different valence patterns (Section~\ref{sec-lr-phrasal-psycho}), there seem to be ways to test whether patterns
%with high frequency and strong collocations should be analyzed as one fixed chunk of material with a
%fixed form and a fixed meaning or whether they should be analyzed compositionally.

\section{来自统计分析的证据}
\label{stat-sec}

在这一节,我们想来看一下声称支持短语观点的来自统计学的证据。我们首先看一下面向数据的分析技术, \citet{Bod2009a}曾用这种技术来为语言习得构建模型,然后讨论一下 \citet{SG2009a}提出的搭配构式分析。最后,我们认为这些分布分析不能够解决论元结构构式是用短语方法还是词汇方法进行分析的问题。
%In this section, we want to look at arguments from statistics that have been claimed to support a phrasal
%view.  We first look at data-oriented parsing, a technique that was successfully used by
% \citet{Bod2009a} to model language acquisition and then we turn to the collostructional analysis by
% \citet{SG2009a}.  Lastly, we argue that these distributional analyses cannot decide the question
%whether argument structure constructions are phrasal or lexical.

\subsection{无监督面向数据分析技术}
\label{Abschnitt-U-Dop-phrasal}

在\isc{统计学|(}\is{statistics|(}\isc{无监督的面向数据的句法分析(U-DOP)|(}\is{Unsupervised Data-Oriented Parsing (U-DOP)|(} 第~\ref{Abschnitt-UDOP}节,我们论述了Bod针对自然语言话语的结构化所采用的方法\citeyearpar{Bod2009a}。如果假设语言是从语言输入而来的,并不借助天赋语言知识,Bob从词语分布提取出来的结构,儿童也需要学习(词类、语义以及包括的语境)。这些结构也需要包含在语言学理论当中。因为Bob没有足够的数据,他的实验基于二叉树假设,并且因为这一原因,就无法从其结论中得出规则是否可以允准平铺或二叉结构。将来,这一问题很有可能有合适的答案。在基于分布的分析中不能确定的是具体在句法树的哪一个节点上引入意义。 \citet[\page 132]{Bod2009b}表明他的方法在Goldberg的意义层面上构建了“一个可验证的构式语法实现”,但是他构建的句法树不能帮助我们决定短语、词汇还是带有空成分\isc{空成分}\is{empty element}的分析更好。这些分析方法如图~\vref{Abbildung-DOP-Resultatives}所示。
%In\is{statistics|(}\is{Unsupervised Data-Oriented Parsing (U-DOP)|(} Section~\ref{Abschnitt-UDOP},
%we saw Bod's approach to the structuring of natural language utterances \citeyearpar{Bod2009a}.
%If one assumes that language is acquired from the input without innate knowledge, the structures
%that Bod extracts from the distribution of words would have to be the ones that children also learn
%(parts of speech, meaning, and context would also have to be included). These structures would then
%also have to be the ones assumed in linguistic theories. Since Bod does not have enough data, he
%carried out experiments under the assumption of binary"=branching trees and, for this reason, it is
%not possible to draw any conclusions from his work about whether rules license flat or
%binary"=branching structures. There will almost certainly be interesting answers to this question in
%the future. What can certainly not be determined in a distribution"=based analysis is the exact node
%in the tree where meaning is introduced.  \citet[\page 132]{Bod2009b} claims that his approach
%constitutes ``a testable realization of CxG'' in the Goldbergian sense, but the trees that he can
%construct do not help us to decide between phrasal or lexical analyses or analyses with empty
%heads\is{empty element}. These alternative analyses are represented in
%Figure~\vref{Abbildung-DOP-Resultatives}.
\footnote{%
如果假设平铺结构的话,讨论起来会更加容易。
%The discussion is perhaps easier to follow if one assumes flat structures rather than binary"=branching ones.\\

\raisebox{2\baselineskip}{\begin{forest}
[X
       [X [er;他] ]
       [X [ihn;他] ]
       [X [leer;空] ]
       [X [fischt;钓] ]
]
\end{forest}}\hfill
\begin{forest}
[X
       [X [er;他] ]
       [X [ihn;他] ]
       [X [leer;空] ]
       [X [X [fischt;钓] ] ]
]
\end{forest}
\hfill
\begin{forest}
[X
       [X [er;他] ]
       [X [ihn;他] ]
       [X [leer;空] ]
       [X [X [fischt;钓] ] 
           [X [\trace{}] ]]
]
\end{forest}

\noindent
第一个图对应着Goldberg所说的短语构式,其中动词插入到构式中并且意义是最顶端的节点上。在第二个图中,有一个词汇规则提供结果义和相应的价信息。在第三个分析中,存在一个空中心语与动词组合,其效果与词汇规则的效果一致。
%The first figure corresponds to the Goldbergian view of phrasal constructions where the verb is inserted into the construction
%and the meaning is present at the topmost node. In the second figure, there is a lexical rule that provides the resultative semantics
%and the corresponding valence information. In the third analysis, there is an empty head that combines with the verb and has ultimately
%the same effect as the lexical rule.
}
\begin{figure}
\hfill
\begin{forest}
sm edges
[X
	[X
		[er;他]]
	[X
		[X
			[ihn;他]]
		[X
			[X
				[leer;空]]
			[X
				[fischt;钓]]]]]
\end{forest}
\hfill
\begin{forest}
sm edges
[X
	[X
		[er;他]]
	[X
		[X
			[ihn;他]]
		[X
			[X
				[leer;空]]
			[X
				[X
					[fischt;钓]]]]]]
\end{forest}
\hfill
\begin{forest}
sm edges
[X
	[X
		[er;他]]
	[X
		[X
			[ihn;他]]
		[X
			[X
				[leer;空]]
			[X
				[X
					[fischt;钓]]
				[X
					[\trace]]]]]]
\end{forest}
%
\hfill\mbox{}
\caption{\label{Abbildung-DOP-Resultatives}结果构式三种可能的分析方式:完全基于构式、词汇规则和空中心语}
%\caption{\label{Abbildung-DOP-Resultatives}Three possible analyses for resultative construction: holistic construction,
%lexical rule, empty head}
\end{figure}%
第一个图代表一个复杂构式贡献整个结构的意义。第二个图对应词汇规则分析方法,第三个图对应着带有空中心语的分析。一个分布分析不能决定这些方案中的哪一个更好。分布是参照词进行计算的:并没有考虑词语的意义。所以,只能说词fischt(钓鱼)出现在一个特定的话语中,但是不可能知道该词是否包含结果义。相似的,一个分布分析无法帮助区分包含以及不包含词汇中心的理论分析。空中心语在这种分析中是不可见的。这是一种理论建构,正如我们在\ref{Abschnitt-leere-Elemente-LRs-Transformations}所示,可以将使用空中心语的分析转换成词汇规则方法。对于当前的例子,对于某一特定分析的所有论证都完全是理论内部的。
%The first figure stands for a complex construction that contributes the meaning as a whole. The second figure corresponds to the analysis
%with a lexical rule and the third corresponds to the analysis with an empty head. A distributional analysis cannot decide between these theoretical
%proposals.
%Distribution is computed with reference to words; what the words actually mean is not taken into account. As such, it is only possible to say
%that the word \emph{fischt} `fishes' occurs in a particular utterance, however it is not possible to see if this word contains resultative semantics or not. 
%Similarly, a distributional analysis does not help to distinguish between theoretical analyses with or without a lexical head.
%The empty head is not perceptible in the signal. It is a theoretical construct and, as we have seen in Section~\ref{Abschnitt-leere-Elemente-LRs-Transformations},
%it is possible to translate an analysis using an empty head into one with a lexical rule. For the present example, any argumentation for a particular analysis will 
%be purely theory"=internal.

虽然无监督面向数据技术分析不能帮助我们决定使用哪种分析方式,但是仍然有语法的某些方面,这些结构可以提供信息:在二分结构假设之下,也可以有不同的分支的可能性,这一点取决于是否假设带有动词移位的分析。这意味着虽然没有在输入中假设一个空成分,但是在统计上的派生结果树中仍然有反映。图~\vref{Abbildung-DOP-Verbbewegung}中左边的树展示了一个遵循 \citew[\page 159]{Steedman2000a-u}观点的分析,参看\ref{sec-Verbstellung-CG-Steedman}。右边的树展示了来自GB类型动词移位的分析(参看\ref{Abschnitt-Verbstellung-GB})。
%Although Unsupervised Data-Oriented Parsing (U"=DOP) cannot help us to decide between analyses, there are areas of grammar for which these structures are of interest: under the assumption of
%binary"=branching structures, there are different branching possibilities depending on whether one assumes an analysis with verb movement or not. This means that
%although one does not see an empty element in the input, there is a reflex in statistically"=derived trees. The left tree in
%Figure~\vref{Abbildung-DOP-Verbbewegung} shows a structure that one would expect from an analysis following
% \citew[\page 159]{Steedman2000a-u}, see Section~\ref{sec-Verbstellung-CG-Steedman}. The tree on the right shows a structure that would be expected
%from a GB"=type verb movement analysis (see Section~\ref{Abschnitt-Verbstellung-GB}). 
\begin{figure}
\hfill%
\begin{forest}
sm edges
[X
	[X
		[X
			[kennt;认识]]
		[X
			[er;他]]]
	[X
		[ihn;他]]]
\end{forest}
\hfill
\begin{forest}
sm edges
[X
	[X
		[kennt;认识]]
	[X
		[X
			[er;他]]
		[X
			[ihn;他]]]]
\end{forest}
\hfill\mbox{}
\caption{\label{Abbildung-DOP-Verbbewegung}带有动词移位和不带动词移位的结构}
%\caption{\label{Abbildung-DOP-Verbbewegung}Structures corresponding to analyses with or without verb movement}
\end{figure}% 
但是,现在就这一问题没有清晰的发现(Bod, p.\,c.\ 2009)。\todostefan{ask again}在U"=DOP树中有很多变化。分配给一个话语的结构取决于动词(Bod,参考《华尔街日报》)。这里,看一下这一点是否会因为更多的数据而发生变化。在任何情况下,看一下所有动词或者某些类别的动词表现如何都是非常有价值的。U"=DOP流程适用于包含至少一个词的树。如果另外考虑词类的话,就会产生我们在前面章节看到的结构。例如,次句法树就不会有X作为其女儿节点,而是会以NP和V作为其女儿节点。也可以用这种次句法树进行统计工作并且在计算中使用词的词类信息(非终结符)而不是使用词自身。例如,可以获得符号V的句法树而不是特定动词的很多的句法树。所以不是为动词küssen(吻)、kennen(知道)和sehen(看)分析出不同的句法树,而是应该为对应于及物动词所需句法树的“动词”这一词性提供三个相同的句法树。V句法树的频率因此会高于具体动词句法树的频率。因此,需要一个更好的数据去计算图~\ref{Abbildung-DOP-Verbbewegung}所示的话语结构。我想在未来会有更多的这一方面的研究成果。\isc{无监督的面向数据的句法分析(U-DOP)|)}\is{Unsupervised Data-Oriented Parsing (U-DOP)|)}
%But at present, there is no clear finding in this regard (Bod, p.\,c.\ 2009).\todostefan{ask again} There is a great deal of variance in the U"=DOP trees.
%The structure assigned to an utterance depends on the verb (Bod, referring to the Wall Street Journal). 
%Here, it would be interesting to see if this changes with a larger data sample.
%In any case, it would be interesting to look at how all verbs as well as particular verb classes behave. The U"=DOP procedure
%applies to trees containing at least one word each. If one makes use of parts of speech in addition, this results in structures that correspond to
%the ones we have seen in the preceding chapters.
%Sub"=trees would then not have two Xs as their daughters but rather NP and V, for example. It is
%also possible to do statistic work with this kind of subtrees and use the part of speech symbols of
%words (the preterminal symbols) rather than the words themselves in the computation. For example, one would get trees for the symbol V instead of many trees for
%specific verbs. So instead of having three different trees for \emph{küssen} `kiss', \emph{kennen} `know' and
%\emph{sehen} `see', one would have three identical trees for the part of speech ``verb'' that corresponds to the
%trees that are needed for transitive verbs. The probability of the V tree is therefore higher than
%the probabilities of the trees for the individual verbs. Hence one would have a better set of data
%to compute structures for utterances such as those in Figure~\ref{Abbildung-DOP-Verbbewegung}. 
%I believe that there are further results in this area to be found in the years to come.\is{Unsupervised Data-Oriented Parsing (U-DOP)|)}

总结一下本节,我们认为Bod的文章在刺激贫乏论这一问题上是一个里程碑式的研究,但是它还能不能证明构式主义理论中的即基于短语的方法这一特定版本是对的。
%Concluding this subsection, we contend that Bod's paper is a milestone in the Poverty of the
%Stimulus debate, but it does not and cannot show that a particular version of constructionist
%theories, namely the phrasal one, is correct.

\subsection{搭配构式}

\mbox{} \citet[\S~5]{SG2009a}提出了一种插入分析:“如果词的意义与构式匹配,则词语出现在一个给定构式(提供的槽)中”。作者声称他们的“搭配构式分析已经从多个角度验证了插入分析”。 Stefanowitsch和Gries可以展示特定动词多半出现在特定构式中,而其它动词从不出现在相应构式中。例如,give(给)、tell(告诉)、send(寄送)、offer(提供)、show(展示)可以出现在双及物构式中,而make(制作)和do(做)则不能出现在该构式中,即就动词在语料库中出现的总频率来讲,它们比预期出现的频率低得多。就这一分布,作者写道:
%\mbox{} \citet[Section~5]{SG2009a} assume a plugging analysis: ``words
%  occur in (slots provided by) a given   construction if their meaning matches that of the
%  construction''. The authors claim that their \emph{collostructional analysis has confirmed}
%  [\emph{the plugging analysis}] \emph{from various perspectives}. Stefanowitsch and Gries are able to show that certain verbs occur more often
%than not in particular constructions, while other verbs never occur in the respective
%constructions. For instance, \emph{give}, \emph{tell}, \emph{send}, \emph{offer} and \emph{show} are
%attracted by the Ditransitive Construction, while \emph{make} and \emph{do} are repelled by this
%construction, that is they occur significantly less often in this construction than what would be
%expected given the overall frequency of verbs in the corpus. Regarding this distribution the authors write:
\begin{quotation}
这些结构对于搭配词位分析很典型,因为它们说明了两件事。首先,实际上词项和语法结构之间具有重要的联系。第二,这些联系为语义一致提供了清晰的证据:被强吸引的组合词位都涉及“转移”概念,或者是字面上的或者是隐喻的,对于双及物结构来说这都很典型。这种结果足够典型来支持一个概括观点,搭配构式分析实际上可以用于首先识别语法构式的意义。\citep[\page 943]{SG2009a}\footnote{%
  These results are typical for collexeme analysis in that they show two things. First, there are
  indeed significant associations between lexical items and grammatical structures. Second, these
  associations provide clear evidence for semantic coherence: the strongly attracted collexemes all
  involve a notion of `transfer', either literally or metaphorically, which is the meaning typically
  posited for the ditransitive. This kind of result is typical enough to warrant a general claim
  that collostructional analysis can in fact be used to identify the meaning of a grammatical
  construction in the first place.
  }
\end{quotation}

% SW: many 'latinate' verbs of transfer like donate, distribute, contribute, etc.-- do not occur in the double object construction.  Stefanowitsch and Gries miss this fact since they only look at the verbs that DO occur. (this is true even when they compare NP-NP to NP-PPto (Table 43.12): they only look at verbs that occur in both).  So their data do not support CG as they define it:
% Stefanowitsch and Gries, p. 941:  "According to Construction Grammar, this relationship 
% [between lexis and grammatical structure] is determined by semantic compatibility: words occur in 
% (slots provided by) a given construction if their meaning matches that of the construction."  
% This 'strong version' of CG is a myth.  Adele knows it doesn't work, as she noted already in
% 1995.  You have to stipulate which words go in which constructions.  
% p. 946:  "...clear evidence for the associations between words and constructions and for semantic compatibility as the main principle governing these associations." This is vaguer: "the main principle" rather than the conditional implication stated above.

\noindent
我们希望前面的讨论已经清楚说明词在语料库中的分布无法看做支持短语分析的证据。语料库的研究表明give(给)经常出现在一个容纳三个论元的模式中(Subject Verb Object1 Object2),并且该动词与其它动词构成一个聚合体并且具有转移义。但是这一语料库库数据不能说明这一意义是短语模式还是词汇词项提供的。
%We hope that the preceding discussion has made clear that the distribution of words in a corpus cannot
%be seen as evidence for a phrasal analysis. The corpus study shows that \emph{give} usually is used
%with three arguments in a certain pattern that is typical for English (Subject Verb Object1 Object2)
%and that this verb forms a cluster with other verbs that have a transfer component in their meaning.
%The corpus data do not show whether this meaning is contributed by a phrasal pattern or by lexical
%entries that are used in a certain configuration.

\section{结论}

%We have shown in this paper that there are no compelling arguments for assuming phrasal argument structure
%constructions, but that there are several arguments against them. Assuming a lexical or\,--\,in the
%terminology of  \citet{Goldberg2013a}\,--\,template-based view solves all the problems that arise for
%phrasal approaches.
%
%Furthermore we showed that radically underspecified approaches in the sense of  \citet{Borer2005a-u},
% \citet{Haugereid2007a}, and  \citet{Lohndal2012a} are not restricted enough. The only way to establish
%the necessary restrictions is a lexical representation, since the information that has to be
%captured is in part lexeme dependent.

词汇观点的核心是一个动词要储存其价结构,价结构记录了动词如何与其依存成分在句法和语义上组合。重要的是,价结构是从动词词例真实句法环境中抽象出来的。一旦提出出来之后,价结构可以满足除了允准该动词最直接编码的短语结构之外的其它方面:可以作为词汇规则的输入,这些词汇规则可以以一种系统的方式来支配该动词;可以与另一谓词的价结构组合;可以与相似动词并列;等。这种抽象可以非常简单地解释大量鲁棒的复杂的语言学现象。我们考察了很多反对词汇规则方法赞成短语模式方式的证据。我们发现用短语方法来表征论元结构都没有说服力:没有强有力的证据来支持这些方法,并且引出了大量问题:
%The essence of the lexical view is that a verb is stored with a valence structure indicating how it
%combines semantically and syntactically with its dependents.  Crucially, that structure is
%abstracted from the actual syntactic context of particular tokens of the verb.  Once abstracted,
%that valence structure can meet other fates besides licensing the phrasal structure that it most
%directly encodes: it can undergo lexical rules that manipulate that structure in systematic ways; it
%can be composed with the valence structure of another predicate; it can be coordinated with similar
%verbs; and so on.  Such an abstraction allows for simple explanations of a wide range of
%robust, complex linguistic phenomena.  We have surveyed the arguments against the lexical valence
%approach and in favor of a phrasal representation instead.  We find the case for a phrasal
%representation of argument structure to be unconvincing: there are no compelling arguments in favor
%of such approaches, and they introduce a number of problems:
\begin{itemize}
\item 无法解释价改变过程与派生形态学的互动。
\item 无法解释价改变过程的重复发生。
\item 会过度生成,除非假设词项与短语构式之间具有联系。
\item 无法解释部分前置例子中的论元分布。
%\item They offer no account for the interaction of valence changing processes and derivational morphology.
%\item They offer no account for the interaction of valence changing processes and coordination of words.
%\item They offer no account for the iteration of valence changing processes.
%\item They overgenerate, unless a link between lexical items and phrasal constructions is assumed.
%\item They offer no account for distribution of arguments in partial fronting examples.
\end{itemize}
假设词汇价结构可以解决所有短语方法遇到的问题。
%Assuming a lexical valence structure
%allows us to solve all the problems that arise with phrasal approaches.

\section{为什么要选择(短语)构式?}
\label{Abschnitt-Phrasale-Konstruktionen}\label{sec-why-phrasal}

在前面的章节中,我们反对在语法描写中假设太多的短语结构。如果想要避免使用用于从单个基本结构派生交替模式的转换并且仍然坚持词汇完整性,那么在分析价变化与派生形态互动的现象时,短语分析就会是不成立的。但是,在很多方面这两者之间并不发生互动关系。在这些案例中,就需要在空中心语和短语结构之间进行选择。在这一节中,我们会讨论一些这样的案例。
%In previous sections, I have argued against assuming too much phrasality in grammatical descriptions.
%If one wishes to avoid transformations in order to derive alternative patterns from a single base structure, while still maintaining lexical integrity,
%then phrasal analyses become untenable for analyzing all those phenomena where changes in valence and derivational morphology interact. There are, however, some areas
%in which these two do not interact. In these cases, there is mostly a choice between analyses with silent heads and those with phrasal constructions. In this section,
%I will discuss some of these cases.

\subsection{无动词指令语}
\label{Abschnitt-Phrasale-Konstruktionen-Jacobs}

\exewidth{(135)}
 \citet{Jacobs2008a}展示了有一些语言学现象,在一组单词中假设一个中心语是没有意义的。这种结构最好描述为短语结构,在这种结构中特定成分连接形成一种无法从其组成成分推导出的完整的意义。Jacobs认为是短语模板的例子见(\mex{1}),无动词指定语\isc{指定语}\is{directive}见(\ref{Beispiel-Direktiva}):
% \citet{Jacobs2008a} showed that there are linguistic phenomena where it does not make sense to assume that there is a head
%in a particular group of words. These configurations are best described as phrasal constructions, in which the adjacency of particular
%constituents leads to a complete meaning that goes beyond the sum of its parts. Examples of the
%phenomena that are discussed by Jacobs are phrasal templates such as those in (\mex{1})
%and verbless directives\is{directive} as in (\ref{Beispiel-Direktiva}):
\begin{exe}
%\ex Pro\sub{1/2pers} N                          \jambox{Ich Idiot!, Du Armer!, \ldots}
%% \begin{tabular}[t]{@{}l@{~}ll@{}}
%% a. & Pro\sub{+w,kaus/fin} NP      & Wozu Konstruktionen?, Warum ich?, \ldots\\
%%    &                              & `Why constructions?, Why me?'\\
%% b. & NP\sub{akk} Y\sub{PP/A/Adv}  & Den Hut in der Hand (kam er ins Zimmer).\\
%%    &                              & `(he came into the room) hat in hand'\\
%% \end{tabular}
\ex Pro\sub{+w,caus/purp} NP
\begin{xlist}
\ex
\gll  Wozu Konstruktionen?\\
      为什么 构式\\
\mytrans{为什么是构式?}
%\gll  Wozu Konstruktionen?\\
%      why constructions\\
%\mytrans{Why constructions?}
\ex 
\gll Warum ich?\\
     为什么 我\\
\mytrans{为什么是我?}
%\gll Warum ich?\\
%     why I\\
%\mytrans{Why me?}
\end{xlist}
\end{exe}
\ea
NP\sub{acc} Y\sub{PP/A/Adv}\\
\gll Den Hut in der Hand (kam er ins Zimmer).\\
     \textsc{det} 帽子 \textsc{prep} \textsc{det} 手 \hspaceThis{(}来 他 \textsc{prep}.\textsc{det} 房间\\
\mytrans{(他进入房间)拿着一顶帽子。}
%\gll Den Hut in der Hand (kam er ins Zimmer).\\
%     the hat in the hand \hspaceThis{(}came he into.the room\\
%\mytrans{(He came into the room) hat in hand.}

\z
在(\mex{-1})中,我们分析简缩问句:
%In (\mex{-1}), we are dealing with abbreviated questions:
\eal
\ex 
\gll Wozu braucht man Konstruktionen? / Wozu sollte man Konstruktionen annehmen?\\
     为什么 需要 某人 构式 {} 为什么 应该 某人 构式 假设\\
\mytrans{我们为什么需要构式?' / `我们为什么应该假设构式?}
%\gll Wozu braucht man Konstruktionen? / Wozu sollte man Konstruktionen annehmen?\\
%     to.what needs one constructions {} to.what should one constructions assume\\
%\mytrans{Why do we need constructions?' / `Why should we assume constructions?}
\ex 
\gll Warum soll ich das machen? / Warum wurde ich ausgewählt? / Warum passiert mir sowas?\\
	 为什么 应该 我 这 做 {} 为什么 \textsc{aux} 我 选择 {} 为什么 发生 我 这种.事情\\
\mytrans{我为什么应该做那件事?' / `为什么我被选中了?' / `为什么那种事情会发生在我身上?}
%\gll Warum soll ich das machen? / Warum wurde ich ausgewählt? / Warum passiert mir sowas?\\
%	 why should I that do {} why was I chosen {} why happens me something.like.that\\
%\mytrans{Why should I do that?' / `Why was I chosen?' / `Why do things like that happen to me?}
\zl
例(\mex{-1})省略了一个分词:
%In (\mex{-1}), a participle has been omitted:
\ea
\gll Den Hut in der Hand haltend kam er ins Zimmer.\\
	 \textsc{det} 帽子.\acc{} \textsc{prep} \textsc{det} 手 抓着 来 他 \textsc{prep}.\textsc{det} 房间\\
\mytrans{他拿着一顶帽子进入房间。}
%\gll Den Hut in der Hand haltend kam er ins Zimmer.\\
%	 the hat.\acc{} in the hand holding came he in.the room\\
%\mytrans{He came into the room hat in hand.}
\z
(\mex{-2})所示案例可以使用一个空中心语\isc{空中心语}\is{empty head}来分析,该空中心语对应haltend(持握)。对于(\mex{-3})可以或者假设一个带有多个空成分的句法结构,或者一个能够选择构式两个部分并且提供(\mex{-1})所示意义的空中心语。如果采用第一种方法,假设一些空中心语,那么就需要解释为什么这些空中心语不能出现在其它构式中。例如,需要假设一个与man(一个/你)对应的空成分。但是,这样一个空成分决不能出现在嵌套句中,因为在嵌套句中主语不能简单地省略:
%Cases such as (\mex{-2}) can be analyzed with an empty head\is{empty head} that corresponds to \emph{haltend} `holding'.
%For (\mex{-3}), on the other hand, one would require either a syntactic structure with multiple empty elements, or an empty head that
%selects both parts of the construction and contributes the components of meaning that are present in (\mex{-1}).
%If one adopts the first approach with multiple silent elements, then one would have to explain why these elements cannot occur in other
%constructions. For example, it would be necessary to assume an empty element corresponding to
%\emph{man} `one'/""`you'. But such an empty element could never occur in embedded clauses since subjects cannot simply be omitted there:
\ea[*]{
\gll weil dieses Buch gerne liest\\
	 因为 这 书 高兴地 阅读\\
\glt 想要表达的意思:\quotetrans{因为他/她/它喜欢读这本书。}
%\gll weil dieses Buch gerne liest\\
%	 because this book gladly reads\\
%\glt Intended: `because he/she/it likes to read this book'
}
\z
如果想要采用第二种方法,就必须假设一个意义非常怪异的空中心语。
%If one were to follow the second approach, one would be forced to assume an empty head with particularly odd semantics.

(\mex{1})和(\mex{2})中的指令语存在相似的问题(英语\il{英语}\il{English}中对应的例子可以参看 \citew[\page 220]{JP2005a-u}):
%The directives in (\mex{1}) and (\mex{2}) are similarly problematic (see also  \citew[\page
%  220]{JP2005a-u} for parallel examples in English\il{English}):
\eal
\label{Beispiel-Direktiva}
\ex 
\gll Her  mit  dem Geld   / dem gestohlenen Geld!\\
     到这里来 \textsc{prep} \textsc{det} 钱 {} \textsc{det} 被偷的 钱\\
\mytrans{将偷的钱拿过来!}
%\gll Her  mit  dem Geld   / dem gestohlenen Geld!\\
%     here with the money {} the stolen money\\
%\mytrans{Hand over the (stolen) money!}
\ex 
\gll Weg  mit  dem Krempel / dem alten Krempel!\\
     丢掉 \textsc{prep} \textsc{det} 废品   {} \textsc{det} 旧 废品\\
\mytrans{丢掉这件(旧)废品!}
%\gll Weg  mit  dem Krempel / dem alten Krempel!\\
%     away with the junk   {} the old junk\\
%\mytrans{Get rid of this (old) junk!}
\ex 
\gll Nieder mit den Studiengebühren / den sozialfeindlichen Studiengebühren!\\
     降低 \textsc{prep} \textsc{det} 学费.费用  {} \textsc{det} 反社会的 学费.费用\\
\mytrans{降低(反社会)学费!}
%\gll Nieder mit den Studiengebühren / den sozialfeindlichen Studiengebühren!\\
%     down with the tuition.fees  {} the antisocial tuition.fees\\
%\mytrans{Down with (antisocial) tuition fees!}
\zl
\eal
\ex 
\gll In den Müll mit diesen Klamotten!\\
     \textsc{prep} \textsc{det} 垃圾 \textsc{prep} 这 衣服\\
\mytrans{将这些衣服扔到垃圾里!}
%\gll In den Müll mit diesen Klamotten!\\
%     in the trash with these clothes\\
%\mytrans{Throw these clothes into the trash!}
\ex 
\gll Zur Hölle mit dieser Regierung!\\
	 \textsc{prep}.\textsc{det} 地狱 \textsc{prep} 这 政府\\
\mytrans{这个政府见鬼去吧!}
%\gll Zur Hölle mit dieser Regierung!\\
%	 to.the hell with this government\\
%\mytrans{To hell with this government!}
\zl
这里也无法简单地识别出一个省略\isc{省略}\is{ellipsis}动词。当然,可以假设一个空中心语,该空中心语选择一个副词或者一个mit-PP,但是这是“特设的”。
%Here, it is also not possible to simply identify an
%elided\is{ellipsis} verb. It is, of course, possible to assume an empty head that selects an adverb or a 
%\emph{mit}-PP, but this would be \emph{ad hoc}.
% Dann ist es ja nicht schlimm:
% (und ansonsten äquivalent zur phrasalen Analyse, siehe Abschnitt~\ref). 
另外,还可以假设(\mex{-1})中的副词选择mit-PP。如果是这样的话,就必须忽略一个事实:副词通常是不带论元的。(\mex{0})中Jacob的例子也是如此。对于这些例子,必须假设in和zur(\textsc{prep}.\textsc{det})分别充当中心语。每一个介词都必须选择一个名词短语和一个mit-PP。虽然这在技术上是可行的,但是这种方案是不好的,正如范畴语法要为随迁(pied"=piping)构式假设多个词项一样(参看\ref{Abschnitt-Relativsaetze-CG})。
%Alternatively, it would be possible to assume that adverbs in (\mex{-1}) select the \emph{mit}-PP. Here, one would have to disregard the fact that adverbs
%do not normally take any arguments. The same is true of Jacobs's examples in (\mex{0}). For these,
%one would have to assume that \emph{in} and \emph{zur} `to the' are the respective heads. Each of
%the prepositions would then have to select a noun phrase and a \emph{mit}-PP. While this is technically possible, it is as unattractive
%as the multiple lexical entries that Categorial Grammar has to assume for pied"=piping constructions (see Section~\ref{Abschnitt-Relativsaetze-CG}). 

G.\  \citet{GMueller2009a}曾经提出过一个更加复杂的分析。Müller 将无动词指令语处理为逆被动构式\isc{逆被动|(}\is{antipassive|(}。逆被动构式或者涉及指令宾语的完全压缩或者实现为旁格成分(PP)。动词上也可能有形态标记。虽然通常主语不会受到逆被动的影响,但是会因为宾语实现上的变化而在作格系统中的得到一个不同的格。根据G.\ Müller\aimention{Gereon M{\"u}ller}的观点,(\mex{1}a)和(\mex{1}b)之间的关系与主动-被动之间的关系类似:
%A considerably more complicated analysis has been proposed by G.\  \citet{GMueller2009a}. Müller treats verbless directives as antipassive constructions\is{antipassive|(}. 
%Antipassive constructions involve either the complete suppression of the direct object or its realization as an oblique element (PP). There
%can also be morphological marking on the verb. The subject is normally not affected by the antipassive but can, however, receive a different case
%in ergative case systems due to changes in the realization of the object. According to
%G.\ Müller\aimention{Gereon M{\"u}ller}, there is a relation between (\mex{1}a) and (\mex{1}b) that is similar to active"=passive pairs:

\eal
\ex 
\gll {}[dass] jemand diese Klamotten in den Müll schmeißt\\
     {}\spacebr{}\textsc{comp} 某人 这 衣服 \textsc{prep} \textsc{det} 垃圾 扔\\
\mytrans{某人将这些衣服扔到垃圾里这件事}     
%\gll {}[dass] jemand diese Klamotten in den Müll schmeißt\\
%     {}\spacebr{}that somebody these clothes in the trash throws\\
%\mytrans{that somebody throws these clothes into the thrash}     
\ex\label{in-den-Muell-mit} 
\gll In den Müll mit diesen Klamotten!\\
     \textsc{prep} \textsc{det} 垃圾 \textsc{prep} 这 衣服\\
\mytrans{将这些衣服扔到垃圾中!}
%\gll In den Müll mit diesen Klamotten!\\
%     in the trash with these clothes\\
%\mytrans{Throw these clothes into the trash!}
\zl
一个空被动语素吸收了动词指派受格的能力(也可以参看\ref{Abschnitt-GB-Passiv}管辖约束理论对于被动的分析)。因此,宾语必须实现为PP或者完全不实现。这遵循Burzio概说\isc{Burzio概说}\is{Burzio's Generalization}:当受格宾语被压缩时,就不会存在外部论元。像很多分布形态学\isc{分布形态学}\is{Distributed Morphology}的支持者(\egc \citealp{Marantz97a})\todostefan{Halle Marantz 93/94}那样,G.\,Müller\aimention{Gereon M{\"u}ller}假设词项在句法之后插入到完全句法树中。逆被动语素在相关句法树节点产生一个特征结构束,该句法树节点与德语的一些动词不兼容,如schmeißen(扔),这就是为什么只有带有对应指定的空动词才能插入。能够触动方向PP移位的机制在这里不能进一步展开论述。逆被动语素强制要求动词重新排序到句首位置(到C,参看\ref{Abschnitt-Verbstellung-GB}和\ref{sec-verb-position-MP})。按照规定,前场的填充只在C位置被可见动词填充的句子才是可能的,这也是为什么G.\,Müller的分析只能派生出V1小句。这些句子被分析为祈使句或极性问句。图~\vref{abb-in-den-Muell-Gereon}给出了对(\mex{0}b)的分析。
%An empty passive morpheme absorbs the capability of the verb to assign accusative (see also Section~\ref{Abschnitt-GB-Passiv} 
%on the analysis of the passive in \gbt). The object therefore has to be realized as a PP or not at all. It follows from Burzio's
%Generalization\is{Burzio's Generalization} that as the accusative object has been suppressed, there cannot be an external argument.
%G.\,Müller\aimention{Gereon M{\"u}ller} assumes, like proponents of Distributed Morphology\is{Distributed Morphology} (\eg \citealp{Marantz97a})\todostefan{Halle
%  Marantz 93/94}, that lexical entries are inserted into complete trees post"=syntactically. The antipassive morpheme creates a feature bundle in the relevant
%  tree node that is not compatible with German verbs such as \emph{schmeißen} `throw' and this is why only a null verb with the corresponding specifications can be
%  inserted. Movement of the directional PP is triggered by mechanisms that cannot be discussed further here. The antipassive morpheme forces an obligatory
%reordering of the verb\is{verb position} in initial position (to C, see
%Section~\ref{Abschnitt-Verbstellung-GB} and Section~\ref{sec-verb-position-MP}). By stipulation, filling the prefield is only possible in sentences where the C position is filled by a visible verb and this is why
%G.\,Müller's analysis does only derive V1 clauses. These are interpreted as imperatives or polar questions. Figure~\vref{abb-in-den-Muell-Gereon}
%gives the analysis of (\mex{0}b).
\begin{figure}
\centering
\begin{forest}
[CP
	[C
		[v $+$ APASS
			[V, name=v1]
			[v $+$ APASS
				[$\varnothing$, name=zero, tier=word]]]
		[C]]
	[vP
		[PP$_2$
			[in den Müll;\textsc{prep} \textsc{det} 垃圾,tier=word,roof]]
		[v$'$
			[VP
				[DP$_1$
					[(mit) diesen Klamotten;(\textsc{prep})这 衣服,roof]]
				[V$'$
					[t$_2$]
					[t$_V$, tier=word]]]
			[v
				[t$_v$, tier=word]]]]]
\draw (v1.south)--(zero.north);
\end{forest}
\caption{G.\, \citet{GMueller2009a}}\label{abb-in-den-Muell-Gereon}将In den Müll mit diesen Klamotten(在有这些衣服的垃圾箱中)分析为一个逆被动构式
%\caption{\emph{In den Müll mit diesen Klamotten} `in the trash with these clothes' as an antipassive following G.\, \citet{GMueller2009a}}\label{abb-in-den-Muell-Gereon}
\end{figure}%
%\noindent
 \citet{Budde2010a}和 \citet{Mache2010a}注意到上述数据的讨论忽略了这种构式也可以出现在疑问句中这一事实:
% \citet{Budde2010a} and  \citet{Mache2010a} note that the discussion of the data has neglected the fact that there are also interrogative variants
%of the construction:
\eal
\ex 
\gll Wohin mit den Klamotten?\\
	 到哪儿 \textsc{prep} \textsc{det} 衣服\\
\mytrans{这些衣服应该去哪儿?}
%\gll Wohin mit den Klamotten?\\
%	 where.to with the clothes\\
%\mytrans{Where should the clothes go?}
\ex 
\gll Wohin mit dem ganzen Geld?\\
	 到哪儿 \textsc{prep} \textsc{det} 整个 钱\\
\mytrans{所有这些钱应该去哪里?}
%\gll Wohin mit dem ganzen Geld?\\
%	 where.to with the entire money\\
%\mytrans{Where should all this money go?}
\zl
因为这些问题对应着V2句子,所以不能要求如果C位置被填充了,前场只能被填充。
%Since these questions correspond to V2 sentences, one does not require the constraint that the prefield can only be filled if the C position
%is filled. 
% Das sind eigene Konstruktionen
%% Such a constraint would be problematic in any case since
%% there are sentences without \textsc{cop}ula\is{\textsc{cop}ula} (Müller:
%% \citeyear[\page 73--74]{Mueller2002b}; \citeyear{Mueller2004e}).

这一分析的一个主要优势是它能够派生出与这一类构式有关的不同的句子类型:带有V1的变体对应着极性问句和祈使句,而带有疑问词的V2变体对应着wh-问句。G.\,Müller\aimention{Gereon M{\"u}ller}提出的方法的一个进一步的结果是不需要另外解释与语法的其他互动。例如,构式与副词互动的方式遵循这一分析:
%One major advantage of this analysis is that it derives the different sentence types that are possible with this kind of construction:
%the V1"=variants correspond to polar questions and imperatives, and the V2"=variants with a question word correspond to \emph{wh}"=questions.
%A further consequence of the approach pointed out by G.\,Müller\aimention{Gereon M{\"u}ller} is that no further explanation is required for
%other interactions with the grammar. For example, the way in which the constructions interact with adverbs follows from the analysis:
{\judgewidth{?*}
\eal
\ex[]{
\gll Schmeiß den Krempel weg!\\
	 扔 \textsc{det} 垃圾 丢掉\\
%\gll Schmeiß den Krempel weg!\\
%	 throw the junk away\\
}
\ex[]{
\gll Schmeiß den Krempel schnell weg!\\
	 扔 \textsc{det} 垃圾 快 丢掉\\
%\gll Schmeiß den Krempel schnell weg!\\
%	 throw the junk quickly away\\
}
\ex[?*]{
\gll Schmeiß den Krempel sorgfältig weg!\\
	  扔 \textsc{det} 垃圾 仔细地 丢掉\\
%\gll Schmeiß den Krempel sorgfältig weg!\\
%	 throw the junk carefully away\\
}
\zl
\eal
\ex[]{
\gll Weg mit dem Krempel!\\
	丢掉 \textsc{prep} \textsc{det} 垃圾\\
%\gll Weg mit dem Krempel!\\
%	 away with the junk\\
}
\ex[]{
\gll Schnell weg mit dem Krempel!\\
	 快 丢掉 \textsc{prep} \textsc{det} 垃圾\\
%\gll Schnell weg mit dem Krempel!\\
%	 quickly away with the junk\\
}
\ex[?*]{
\gll Sorgfältig weg mit dem Krempel!\\
	 仔细地 丢掉 \textsc{prep} \textsc{det} 垃圾\\
%\gll Sorgfältig weg mit dem Krempel!\\
%	 carefully away with the junk\\
}
\zl}

\noindent
但是仍然应该记住该分析的代价:它假设了一个德语中其它现象都不需要的空逆被动语素。该语素只能用于这里讨论的这种构式。这一语素与任意动词都不兼容并且会激发强制的动词移位,这是其它构成动词要素的语素所不具有的。
%Nevertheless one should still bear the price of this analysis in mind: it assumes an empty antipassive morpheme that is otherwise not
%needed in German. It would only be used in constructions of the kind discussed here. This morpheme is not compatible with
%any verb and it also triggers obligatory verb movement, which is something that is not known from any other morpheme used
%to form verb diatheses.
% Imperativ?

如果假设人类已经有了这一逆被动语素,即这一语素是天赋普遍语法\indexugc(Universal Grammar)的一部分,那么这一分析的代价就会降低。但是,如果遵循这一章前面小节的论述,那么只有在别的解释都行不通的情况下才会假设天赋语言学知识。
%The costs of this analysis are, of course, less severe if one assumes that humans already have this antipassive morpheme anyway, that is, this morpheme
%is part of our innate Universal Grammar\indexug. But if one follows the argumentation from the earlier sections of this chapter, then one should only assume
%innate linguistic knowledge if there is no alternative explanation\aimention{Gereon M{\"u}ller}.

G.\ Müller的分析可以被转换为HPSG的表示,如(\mex{1})所示:
%G.\ Müller's analysis can be translated into HPSG. The result is given in (\mex{1}):
\ea
\oneline{%
\onems[verb-initial-lr]{
%synsem$|$loc$|$cat$|$subcat \sliste{ [ nonloc$|$inher$|$slash \eliste ] }\\
rels \relliste{ \ms[imperative-or-interrogative]{
                  event & \ibox{2}\\
                  } } $\oplus$ \etag\\[2mm]
lex-dtr  \onems{
          phon \eliste\\
          ss$|$loc \onems{ cat   \onems{ head$|$mod \type{none}\\
                                           subcat     \sliste{ XP[\textsc{mod} \ldots{}  ind \ibox{1}], (PP[\type{mit}]\ind{1}) }\\
                                         }\\
                            cont  \ms{
                                   ind & \ibox{2}\\
                                   rels & \liste{ \ms[directive]{
                                                  event       & \ibox{2}\\
                                                  patient    & \ibox{1}\\
                                                  }\\
                                                 }\\
                                      }\\
                          }\\
}\\
}}
\z
(\mex{0})包含了出现在动词居首\isc{动词居首}\is{verb position}位置空动词的一个词项。\relation{directive}是一个更概括关系的占位符,该关系应该被视作这一构式所有可能意义的一个上位类型。这些语义包括schmeißen(去扔)和Monika Budde向我指出的案例,例如(\mex{1}):
%(\mex{0}) contains a lexical entry for an empty verb in verb"=initial position.\is{verb position} \relation{directive}
%is a placeholder for a more general relation that should be viewed as a supertype of all possible meanings of this
%construction. These subsume both \emph{schmeißen} `to throw' and cases such as (\mex{1}) that were pointed out to me by Monika Budde:
\ea
\label{Klavier-durch-die-Tuer}
\gll Und mit dem Klavier ganz langsam durch die Tür!\\
	 并且 \textsc{prep} \textsc{det} 钢琴 非常 慢 \textsc{prep} \textsc{det} 门\\
\mytrans{很慢地搬着钢琴通过门!}
%\gll Und mit dem Klavier ganz langsam durch die Tür!\\
%	 and with the piano very slowly through the door\\
%\mytrans{Carry the piano very slowly through the door!}
\z
因为在这一构式中,只允许动词首位和动词二位,所以为动词居首位置(参看第~\pageref{lr-verb-movement}页)使用词汇规则是强制的。这一点可以通过以下途径实现:将词汇使用的结果写在词库中,而不将词汇规则作用的对象出现在词库中。 \citet[\S~3.4.2, 5.3]{Koenig99a}举出了英语\il{英语}\il{English}中类似的rumored(谣言)“it is rumored that \ldots”(谣言说……)和aggressive(有攻击性的)。动词rumored(谣言)没有主动形式,这一事实可以通过以下假设来解释:只有使用被动词汇规则的结果才会出现在词库中。派生出分词的实际的动词或者动词词干只能充当词汇规则的子节点而不是独立的语言学单位。相似的,动词*aggress只能充当能允准aggressive(有攻击性的)的(没有能产性的)形容词规则以及允准agreession(攻击性)的名词化规则的子节点。
%Since only verb"=initial and verb"=second orders are possible in this construction, the application of the lexical rule for verb"=initial position
%(see page~\pageref{lr-verb-movement}) is obligatory. This can be achieved by writing the result of the application of this lexical rule into the lexicon, without
%having the object to which the rule should have applied actually being present in the lexicon itself. 
% \citet[Section~3.4.2, 5.3]{Koenig99a} proposed something similar for English\il{English} \emph{rumored} `it is rumored that \ldots' and \emph{aggressive}. 
%There is no active variant of the verb \emph{rumored}, a fact that can be captured by the assumption that only the result of applying a passive lexical rule
%is present in the lexicon. The actual verb or verb stem from which the participle form has been
%derived exists only as the daughter of a lexical rule but not as an independent linguistic
%object. Similarly, the verb \noword{aggress} only exists as the daughter of a (non"=productive)
%adjective rule that licenses \emph{aggressive} and a nominalization rule licensing \emph{aggression}.

mit-PP的可选择性用(\mex{-1})中的括号来表示。如果加上从SYNSEM下\type{verb-initial-lr}中的承继的信息,那么结果见(\mex{1})。
%The optionality of the \emph{mit}-PP is signaled by the brackets in (\mex{-1}). If one adds the information inherited from the type \type{verb-initial-lr}
%under \synsem, then the result is (\mex{1}).
%\vpageref{in-den-muell-lexical}.
%\begin{figure}[hbp]
\ea
\label{in-den-muell-lexical}
\oneline{%
\onems[verb-initial-lr]{
synsem$|$loc \ms{ head & \ms[verb]{vform & fin\\
                                          initial & $+$\\
                                          dsl     & none\\
                                 }\\
                           subcat & \sliste{ \onems{ loc$|$cat \onems{ head  \ms[verb]{
                                                               dsl & \ibox{3}\\
                                                               }\\
                                                         subcat \eliste\\
                                                       }
                                              }}\\
                         }\\
rels \relliste{ \ms[imperative-or-interrogative]{
                  event & \ibox{2}\\
                  } } $\oplus$ \ibox{4}\\[5mm]
lex-dtr  \onems{
          phon \eliste\\
          ss$|$loc \ibox{3} \onems{ cat   \onems{ head$|$mod \type{none}\\
                                           subcat     \sliste{ XP[\textsc{mod} \ldots{}  ind \ibox{1}], (PP[\type{mit}]\ind{1}) }\\
                                         }\\
                            cont  \ms{
                                   ind & \ibox{2}\\
                                   rels & \ibox{4} \liste{ \ms[directive]{
                                                  event       & \ibox{2}\\
                                                  patient    & \ibox{1}\\
                                                  }\\
                                                 }\\
                                      }\\
                          }\\
}\\
}}
\z
%\vspace{-\baselineskip}
%\end{figure}%
%
(\mex{0})中空动词的价属性很大程度上取决于限定动词居首顺序的词汇规则:V1"=LR允准一个动词性中心语,该中心语需要一个VP在其右边,并且该VP丢失一个带有局部属性\textsc{lex-dtr} \iboxb{3}的动词。
%The valence properties of the empty verb in (\mex{0}) are to a large extent determined by the lexical rule for verb"=initial order: the V1"=LR licenses a verbal head
%that requires a VP to its right that is missing a verb with the local properties of the \textsc{lex-dtr} \iboxb{3}.
% Außerdem wird
% von der selegierten VP verlangt, dass sie eine leere \slashl hat. Daraus ergibt sich, dass kein
% Element aus der VP extrahiert werden darf, weshalb mit dem Eintrag in (\mex{0}) ausschließlich
% Verberstsätze abgeleitet werden können. 

取决于句子类型(陈述、其实或疑问)的语义信息是在V1"=LR中由动词的形态组成和被选择VP的\slashv 取值所决定的(参看 Müller\citeyear[\S~10.3]{MuellerLehrbuch1};\citeyear{MuellerSatztypen};\citeyear{MuellerGS})。为\type{祈使-或-疑问}赋予语义排除了“陈述”,如V2"=小句所示。这一类型是朝“祈使”或“疑问”方向分解最终取决于话语的其他属性,例如语调或疑问代词。
%Semantic information dependent on sentence type (assertion, imperative or question) is determined inside the V1"=LR depending on the morphological
%make"=up of the verb and the \slashv of the selected VP (see Müller
%\citeyear[Section~10.3]{MuellerLehrbuch1}; \citeyear{MuellerSatztypen}; \citeyear{MuellerGS}).
%Setting the semantics to \type{imperative-or-interrogative}  rules out \emph{assertion} as it occurs in V2"=clauses.
%Whether this type is resolved in the direction of \type{imperative} or
%\type{interrogative} is ultimately decided by further properties of the utterance such as intonation or the use of interrogative pronouns.

(\mex{0})中词汇子节点的价以及与语义角色的连接(与受事角色的连接)被很简单地标注出来。每一种方法都必须标注动词的论元实现为mit-PP。因为德语中没有逆被动\isc{逆被动|)}\is{antipassive|)},应该由(\mex{0})中逆被动词汇规则达到的效应被简单地写成了动词移位规则的\textsc{lex-dtr}。
%The valence of the lexical daughters in (\mex{0}) as well as the connection to the semantic role (the linking to the patient role) are simply stipulated.
%Every approach has to stipulate that an argument of the verb has to be expressed as a \emph{mit}-PP. Since there is no antipassive\is{antipassive|)} in German,
%the effect that could be otherwise achieved by an antipassive lexical rule in (\mex{0}) is simply written into the \textsc{lex-dtr} of the verb movement rule.

\textsc{lex-dtr}的\subcatlc 包含一个修饰语(副词或方向PP)和一个mit-PP。这一mit-PP与\relation{指令}是同指的并且修饰语指称mit-PP的指称对象。\relation{指令}的施事是尚未指定的,因为它取决于语境(说话者,听话者,第三者)。
%The \subcatl of \textsc{lex-dtr} contains a modifier (adverb, directional PP) and the 
%\emph{mit}-PP. This \emph{mit}-PP is co"=indexed with the patient of \relation{directive} and the modifier refers to the referent of the \emph{mit}-PP. The agent
%of \relation{directive} is unspecified since it depends on the context (speaker, hearer, third person).

该分析见图~\vref{verb-movement-muell}。
%This analysis is shown in Figure~\vref{verb-movement-muell}.
\begin{figure}
\oneline{%
\begin{forest}
sm edges
[V{[\subcat \sliste{}]}
	[V{[\subcat \sliste{ \ibox{1} [\textsc{head$|$dsl} \ibox{2}] }]}
		[V{[\textsc{loc} \ibox{2} ]}, tier=pp, edge label={node[midway,right]{V1-LR}}
			[\trace]]]
	[\ibox{1} V\feattab{
                        \textsc{head$|$dsl} \ibox{2},\\
                        \subcat \sliste{} }
		[\ibox{3} PP, tier=pp
			[in den Müll;\textsc{prep} \textsc{det} 车库,roof]]
			%[in den Müll;in the garbage,roof]]
		[V\feattab{
                         \textsc{head$|$dsl} \ibox{2},\\
                         \subcat \sliste{ \ibox{3} }}
			[\ibox{4} PP{[\type{mit}]}
				[mit diesen Klamotten;\textsc{prep} 这些 衣服,roof]]
				%[mit diesen Klamotten;with these clothes,roof]]
			[V\ibox{2}\feattab{ \textsc{head$|$dsl} \ibox{2},\\
                                            \subcat \sliste{ \ibox{3}, \ibox{4} }}
				[\trace]]]]]
\end{forest}
}
\caption{\label{verb-movement-muell}利用HPSG分析“In den Müll mit diesen Klamotten!/?”}
%\caption{\label{verb-movement-muell}HPSG variant of the analysis of \emph{In den Müll mit diesen Klamotten!/?}}
\end{figure}%
在这里,V[\textsc{loc} \ibox{2}]对应着(\mex{0})中的\textsc{lex-dtr}。V1-LR允准一个需要带有\dslvc \ibox{2}的最大动词投射的成分。因为\dslc 是一个中心语特征,所以信息在中心语路径上出现。\dslvc 与动词移位语迹上的\localvc(图~\ref{verb-movement-muell}中\iboxt{2})同指(请看第~\pageref{le-verbspur}页)。这就确保在句子末尾的空成分与(\mex{0})中\textsc{lex-dtr}所具有的局部属性是完全相同的。因此,正确的句法和语义信息都出现在动词语迹上并且涉及动词语迹的结构构建遵循通常的原则。这一结构对应着我们在第~\ref{Kapitel-HPSG}章为德语假设的结构。因此,对于融合附接语有通常的可能性。语义的正确派生,特别是在祈使和疑问语义下的嵌套,随后自动进行(对于与动词位置连接的附接语的语义,可以参看 \citew[\S~9.4]{MuellerLehrbuch1})。带有方向(\ref{Klavier-durch-die-Tuer})前的mit-PP以及在mit-PP(\ref{in-den-Muell-mit})之前的方向的序列变体就遵循常规机制。
%Here, V[\textsc{loc} \ibox{2}] corresponds to the \textsc{lex-dtr} in (\mex{0}). The V1-LR licenses an element that requires a maximal verb projection
%with that exact \dslv \ibox{2}. Since \dsl is a head feature, the information is present along the head path. The \dslv is identified with the \localv
%(\iboxt{2} in Figure~\ref{verb-movement-muell}) in the verb movement trace (see page~\pageref{le-verbspur}). 
%This ensures that the empty element at the end of sentence has exactly the same local properties that the \textsc{lex-dtr} in (\mex{0}) has.
%Thus, both the correct syntactic and semantic information is present on the verb trace and structure
%building involving the verb trace follows the usual principles.
%The structures correspond to the structures that were assumed for German sentences in Chapter~\ref{Kapitel-HPSG}.
%Therefore, there are the usual possibilities for integrating adjuncts. The correct derivation of the semantics, in particular embedding under
%imperative or interrogative semantics, follows automatically (for the semantics of adjuncts in conjunction with verb position, see   \citew[Section~9.4]{MuellerLehrbuch1}). 
%Also, the ordering variants with the \emph{mit}-PP preceding the direction (\ref{Klavier-durch-die-Tuer}) and direction preceding the 
%\emph{mit}-PP (\ref{in-den-Muell-mit}) follow from the usual mechanisms.

如果反对当前的分析,剩余的唯一的方法就是短语结构或者连接构式组成部分并提供相应语义的统制图式。确实,怎样以一种非标记方式来讲附接语融入到短语构式中仍然是一个悬而未决的问题,但是Jakob  \citet{Mache2010a}已经做出了初步成果,即只要假设一个合适的短语图式,指令仍然可以插入到整个语法当中。
%If one rejects the analyses discussed up to this point, then one is only really left with phrasal constructions or dominance schemata that connect parts
%of the construction and contribute the relevant semantics. Exactly how one can integrate adjuncts into the phrasal construction in a non"=stipulative way
%remains an open question; however, there are already some initial results by Jakob  \citet{Mache2010a} suggesting that directives can still be sensibly integrated into
%the entire grammar provided an appropriate phrasal schema is assumed.

\subsection{连动式}
有些语言\isc{动词!连动式|(}\is{verb!serial|(}\il{现代汉语|(}\il{Mandarin Chinese|(}有所谓的连动式。例如,在现代汉语中可以构成只有一个主语和几个动词短语的句子\citep[\S~21]{LT81a}。连动式有多种意义,具体意义是什么取决于VP当中体标记的分布:\footnote{%
  详细论述和更多的参考文献可以参看 \citew{ML2009a}。
} 
如果第一个VP包含一个完成体标记,那么连动式的意义是“VP1为了达到VP2”(\mex{1}a)。如果第二个VP包含一个完成体标记,那么整个结构的意义是“因为VP1而VP2”,如果第一个VP包含一个延续体标记和动词hold(持)或use(使用),那么整个结构的意义是“VP2使用VP1”(\mex{1}c)。
%There\is{verb!serial|(}\il{Mandarin Chinese|(} are languages with so"=called serial verbs. For example, it is possible to form sentences in Mandarin Chinese where there is only one subject
%and several verb phrases \citep[Chapter~21]{LT81a}. There are multiple readings depending on the
%distribution of aspect marking inside the VP:\footnote{%
%  See  \citew{ML2009a} for a detailed discussion and further references.
%} if the first VP contains a perfect marker, then we have
%the meaning `VP1 in order to do/achieve VP2' (\mex{1}a). If the second VP contains a perfect marker, then the entire construction means `VP2 because VP1' (\mex{1}b), and if the
%first VP contains a durative marker and the verb \emph{hold} or \emph{use}, then the entire construction means `VP2 using VP1' (\mex{1}c). 
\eal
\ex
     他取了钱去逛街 \\
%\mytrans{他取了钱去商店。}
%\gll Ta1 qu3 le qian2 qu4 guang1jie1. \\
%     he withdraw \textsc{prf} money go shop \\
%\mytrans{He withdrew money to go shopping.}

%\ex
   %  他住中国学了中文 \\
%\mytrans{他学习汉语因为他住在中国。}
%\gll Ta1 zhu4 Zhong1guo2 xue2 le Han4yu3. \\
%     he  live China learn \textsc{prf} Chinese \\
%\mytrans{He learned Chinese because he lived in China.}

\ex
%\gll Ta1 na2 zhe kuai4zi chi1 fan4.\\
     他拿这筷子吃饭 \\
%\mytrans{他用筷子吃饭。}
%\gll Ta1 na2 zhe kuai4zi chi1 fan4.\\
%     he  take \textsc{dur} chopsticks eat food \\
%\mytrans{He eats with chopsticks.}
\zl
如果我们分析这些句子,我们只能看到两个毗邻的VP。但是,整个句子的意义不能从其组成成分完全推导得出。基于不同种类的体标记,我们可以得到不同的意义。正如我们在译文中所见,英语中有时会用连词来表示两个小句或动词短语之间的关系。
%If we consider the sentences, we only see two adjacent VPs. The meanings of the entire sentences, however, contain parts of meaning that go beyond the meaning
%of the verb phrases. Depending on the kind of aspect marking, we arrive at different interpretations with regard to the semantic combination of verb phrases.
%As can be seen in the translations, English sometimes uses conjunctions in order to express relations between clauses or verb phrases.

总共有三种可能的方式来概括这些例子:
%There are three possible ways to capture these data:
\begin{enumerate}
\item 可以认为汉语母语者简单地从语境中推导出两个VP之间的关系,
%\item One could claim that speakers of Chinese simply deduce the relation between the VPs from the context,
\item 可以认为汉语中有对应于because(因为)或to(为了)的空中心语,
%\item one could assume that there are empty heads in Chinese corresponding to \emph{because} or \emph{to}, or
\item 为连动式假设一个短语结构,来解释整个构式的意义因VP中的体标记不同而不同。
%\item one could assume a phrasal construction for serial verbs that contributes the correct semantics for the complete
%meaning depending on the aspect marking inside the VPs.
\end{enumerate}
第一种方法不能令人满意,因为构式的意义并非是任意的。确实存在一个语法理论应该概括的语法规约性。第二种方法有一种标记特征,所以如果要避免空成分,只能选取第三种方法。 \citet{ML2009a}就提供了一种对应的分析。\isc{动词!连动式|)}\is{verb!serial|)}\il{现代汉语|)}\il{Mandarin Chinese|)}
%The first approach is unsatisfactory because the meaning does not vary arbitrarily. There are grammaticalized conventions that
%should be captured by a theory. The second solution has a stipulative character and thus, if one wishes to avoid empty elements, only
%the third solution remains.  \citet{ML2009a} have presented a corresponding analysis.\is{verb!serial|)}\il{Mandarin Chinese|)}

\subsection{关系小句和疑问小句}
\label{Abschnitt-Relativ-Interrogativsaetze}
\mbox{} \citet{Sag97a}\isc{关系小句|(}\is{relative clause|(}\isc{疑问小句|(}\is{interrogative clause|(}为\il{英语|(}\il{English|(}关系小句提出了一种短语分析, \citet{GSag2000a-u}也为疑问小句提供了一种短语分析。关系小句和疑问小句包含一个前置的短语和一个丢失前置短语的小句或动词短语。前置的短语包含一个关系代词或疑问代词。
%\mbox{} \citet{Sag97a}\is{relative clause|(}\is{interrogative clause|(}
%develops a phrasal analysis of English\il{English|(} relative clauses as have  \citet{GSag2000a-u} for interrogative clauses.
%Relative and interrogative clauses consist of a fronted phrase and a clause or a verb phrase missing the fronted phrase.
%The fronted phrase contains a relative or interrogative pronoun.
\eal
\ex
\gll the man [who] sleeps\\
     \textsc{det} 男人 \spacebr{}\textsc{rel} 睡觉\\
\mytrans{睡觉的男人} 
%the man [who] sleeps
\ex
\gll the man [who] we know\\
     \textsc{det} 男人 \spacebr{}\textsc{rel} 我们 知道\\
\mytrans{我们知道的那个男人} 
%the man [who] we know
\ex
\gll the man [whose mother] visited Kim\\
     \textsc{det} 男人 \spacebr{}\textsc{rel} 母亲 探望 Kim\\
\mytrans{他母亲探望Kim的那个男人} 
%the man [whose mother] visited Kim
\ex
\gll a house [in which] to live\\
     一 房子 \spacebr{}\textsc{prep} \textsc{rel} \textsc{prep} 住\\
\mytrans{将要有人入住的房子} 
%a house [in which] to live
\zl
\eal
\ex
\gll I wonder [who] you know.\\
     我 猜想 \spacebr{}谁 你 知道\\
\mytrans{我猜想你知道谁} 
%I wonder [who] you know.
\ex
\gll  I want to know [why] you did this.\\
     我 想 \textsc{inf} 知道 \spacebr{}为什么 你 做 这\\
\mytrans{我想知道你为什么做这件事。} 
%I want to know [why] you did this.
\zl
对于关系小句的GB分析已经在图~\ref{Abbildung-GB-Relativsatz}中给出了。在该分析中,C位置上有一个空成分\isc{空成分}\is{empty element}并且一个成分从IP移位到指定语。%
%The GB analysis of relative clauses is given in Figure~\ref{Abbildung-GB-Relativsatz}.
%In this analysis, an empty head\is{empty element} is in the C position and an element from the IP is moved
%to the specifier position.%
\begin{figure}
\centering
\begin{forest}
sm edges, for tree={fit=rectangle}
[CP{[\type{rel}]}
	[NP
		[whose remarks;谁的 意见,roof]]
	[\cbar{[\type{rel}]}
		[\cnull{[\type{rel}]}
			[\trace]]
		[IP,l sep+=\baselineskip
			[they seemed to want to object to;他们 看起来 \textsc{inf} 想要 \textsc{inf} 反对 \textsc{inf},roof]]]]
\end{forest}
\caption{\label{Abbildung-GB-Relativsatz}对\gbc 理论中结果小句的分析 }
%\caption{\label{Abbildung-GB-Relativsatz}Analysis of relative clauses in \gbt }
\end{figure}%

%\noindent
与此相对的分析可以参看图~\vref{Abbildung-HPSG-Relativsatz},该分析将各个成分直接组合起来形成一个关系小句。
%The alternative analysis shown in Figure~\vref{Abbildung-HPSG-Relativsatz} involves combining the subparts directly
%in order to form a relative clause.
\begin{figure}
\begin{forest}
sm edges, for tree={fit=rectangle}
[S{[\type{rel}]}
	[NP
		[whose remarks;谁的 意见,roof]]
	[S,l sep+=\baselineskip
		[they seemed to want to object to;他们 看起来 \textsc{inf} 想要 \textsc{inf} 反对 \textsc{inf},roof]]]
\end{forest}
\caption{\label{Abbildung-HPSG-Relativsatz} \citew{Sag97a}中使用HPSG对关系小句的分析}
%\caption{\label{Abbildung-HPSG-Relativsatz}Analysis of relative clauses in HPSG following  \citew{Sag97a}}
\end{figure}%
 \citet{Borsley2006a}指出如果想要用词汇的方法处理英语中不同类型的关系小句,就必须假设六个空中心语。这些空中心语可以用对应的图式来规避和替代(参看\ref{chap-empty}对于空成分的论述)。在 \citet{Webelhuth2011a}中也可以发现类似的对德语的论述:德语语法也需要为相关类型的关系小句假设六个空中心语。%
% \citet{Borsley2006a} has shown that one would require six empty heads in order to capture the
%various types of relative clauses possible in English if one wanted to analyze them lexically. These heads can be avoided and replaced by corresponding schemata
%(see Chapter~\ref{chap-empty} on empty elements). A parallel argument can also be found in  \citet{Webelhuth2011a}
%for German: grammars of German would also have to assume six empty heads for the relevant types of relative clause.%
\nocite{Borsley2007a}
% Generalisierungen über verschieden Relativsatzkonstruktionen kann man in Vererbungshierarchien
% erfassen. Natürlich könnte man genauso die Generalisierungen in Bezug auf die Eigenschaften der
% leeren Köpfe in Vererbungshierarchien erfassen.

与我们已经讨论过的结果构式不同,疑问小句和关系小句组成成分的顺序不会改变。不存在价改变\isc{价!价改变}\is{valence!change}和与派生形态\isc{形态}\is{morphology}的互动。因此,没有证据反对短语分析。
%Unlike the resultative constructions that were already discussed, there is no variability among interrogative and relative clauses with regard to the order of
%their parts. There are no changes in valence\is{valence!change} and no interaction with derivational morphology\is{morphology}. Thus, nothing speaks against a phrasal
%analysis.
% Aber spricht auch etwas dafür? Sag weißt auf folgende Daten aus
% dem Koreanischen\il{Koreanisch} hin. Im Beispiel (\mex{1}b) kommt das Verb \emph{legen} in einem
% Relativsatz vor. Es ist besonders flektiert, \dash, es ist für die Verwendung in Relativsätzen
% ausgezeichnet. 
% \eal
% \ex 
% \gll John-i chayk-ul ku sangca-ey neh-ess-ta.\\
%      John-nom Buch-acc die Schachtel-loc legen-past-decl\\
% \mytrans{John legte das Buch in die Schachtel.}
% \ex 
% \gll {}[[John-i chayk-ul neh-un] sangca-ka] khu-ta.\\
%        \hspaceThis{[[}John-nom Buch-acc legen-rel Schachtel-nom groß-decl\\
% \mytrans{Die Schachtel, in die John das Buch gelegt hat, ist groß.}
% \zl
% % Wenn man das komplett parallel machen wollte, müsste man für das Englische eine disjunktive
% % Spezifikation von MOD-Werten und entsprechendem semantischen Beitrag annehmen. In einem Fall
% % handelt es sich um das normale Verb und im anderen Fall um des Relativsatz-Verb mit nominaler Semantik.
% Allgemein gilt, dass immer das höchste Verb des Relativsatzes flektiert wird. Das kann man gut
% erklären, wenn man annimmt, dass dieses Verb der Kopf des Relativsatzes ist.
如果想要避免假设空中心语,那么就应该选择Sag提出的对于关系小句的分析,或者Müller(\citeyear[Chapter~10]{Mueller99a};\citeyear[Chapter~11]{MuellerLehrbuch1})提出的相似的分析。后一种分析方法不用为名词-关系小句组合提供一个特殊的图式,因为关系小句的语义内容是由关系小句图式提供的。%
%If one wishes to avoid the assumption of empty heads, then one should opt for the analysis of relative clauses 
%by Sag, or the variant in Müller (\citeyear[Chapter~10]{Mueller99a}; \citeyear[Chapter~11]{MuellerLehrbuch1}). The latter analysis does without a special schema
%for noun"=relative clause combinations since the semantic content of the relative clause is provided by the relative clause
%schema.%

 \citet{Sag2010b}讨论了英语\il{英语}\il{English}中的长距离依存现象,这些现象在管辖约束理论和最简方案\indexmpc(Minimalist Program)中被归入wh移位。他展示了这并非是一种统一的现象。他研究了wh疑问句(\mex{1})、wh感叹句\isc{wh-感叹句@\emph{wh}-感叹句}\is{wh-exclamative@\emph{wh}"=exclamative}(\mex{2})、话题化\isc{话题化}\is{topicalization}(\mex{3})、wh关系化小句\isc{关系小句}\is{relative clause}(\mex{4})和the小句\isc{the-小句@\emph{the}-小句}\is{the-clause@\emph{the}-clause}(\mex{5}):
% \citet{Sag2010b} discusses long"=distance dependencies in English\il{English} that are subsumed
%under the term \emph{wh}"=movement in \gbt and the Minimalist Program\indexmp. He shows that this is by no means a
%uniform phenomenon.  He investigates \emph{wh}"=questions (\mex{1}),
%\emph{wh}"=exclamatives\is{wh-exclamative@\emph{wh}"=exclamative} (\mex{2}),
%topicalization\is{topicalization} (\mex{3}), \emph{wh}"=relative clauses\is{relative clause}
%(\mex{4}) and \emph{the}"=clauses\is{the-clause@\emph{the}-clause} (\mex{5}):
\eal
\ex
\gll How foolish is he?\\
     \textsc{adv} 傻 \textsc{cop} 他\\
\mytrans{他有多么傻?} 
%How foolish is he?
\ex
\gll I wonder \emph{how} \emph{foolish} \emph{he} \emph{is}.\\
     我 猜想 \textsc{adv} 傻 他 \textsc{cop}\\
\mytrans{我猜想他有多么傻。}
%I wonder \emph{how foolish he is}.
\zl

\eal
\ex
\gll What a fool he is!\\
     \textsc{adv} 一 傻 他 \textsc{cop}\\
\mytrans{他这个傻瓜!}
%What a fool he is!
\ex
\gll It's amazing \emph{how} \emph{odd} \emph{it} \emph{is}.\\
     \textsc{expl}.\textsc{cop} 惊奇 \textsc{adv} 奇怪 它 \textsc{cop}\\
\mytrans{它那么奇怪真是令人吃惊!} 
%It's amazing \emph{how odd it is}.
\zl
\ea
\gll The bagels, I like.\\
     \textsc{det} 百吉饼 我 喜欢\\
\mytrans{百吉饼,我喜欢。}
%The bagels, I like.
\z
\eal
\ex
\gll I met the person \emph{who} \emph{they} \emph{nominated}.\\
     我 见 \textsc{det} 人 \textsc{rel} 他们 任命\\
\mytrans{我会见了他们任命的人。} 
%I met the person \emph{who they nominated}.
\ex
\gll I'm looking for a bank \emph{in} \emph{which} \emph{to} \emph{place} \emph{my} \emph{trust}.\\
     我.\textsc{cop} 找 \textsc{prep} 一 银行 \textsc{prep} \textsc{rel} \textsc{inf} 放置 我的 基金\\
\mytrans{我正在寻找一个银行来储存我的基金。} 
%I'm looking for a bank \emph{in which to place my trust}.
\zl
\eal
\ex
\gll The more people I met, \emph{the} \emph{happier} \emph{I} \emph{became}.\\
     \textsc{det} 更多 人 我 见 \textsc{det} 更高兴 我 变得\\
\mytrans{我见的人越多我越高兴。} 
%The more people I met, \emph{the happier I became}.
\ex 
\gll \emph{The} \emph{more} \emph{people} \emph{I} \emph{met}, the happier I became.\\
     \textsc{det} 更多 人 我 见 \textsc{det} 更高兴 我 变得\\
\mytrans{我见的人越多我越高兴。}
%\emph{The more people I met}, the happier I became.
\zl
这些构式在很多方面都有差异。Sag列出了每一个构式都需要回答的问题:
%These individual constructions vary in many respects. Sag lists the following questions that have to be answered
%for each construction:
\begin{itemize}
\item 在填充项子节点上有没有一特殊的wh"=成分,如果有的话,是什么类型的成分?
%\item Is there a special \emph{wh}"=element in the filler daughter and, if so, what kind of element is it?
\item 填充项子节点可以哪种句法范畴?
%\item Which syntactic categories can the filler daughters have?
%\item Welche syntaktischen Kategorien kann die Kopf"|tochter haben?
\item 中心语子节点可以倒置吗?是限定的吗?是强制的吗?
%\item Can the head"=daughter be inverted or finite? Is this obligatory?
\item 父节点的语义和/或句法范畴是什么?
%\item What is the semantic and/or syntactic category of the mother node?
\item 中心语子节点的语义和/或句法范畴是什么?
%\item What is the semantic and/or syntactic category of the head"=daughter?
\item 句子是一个岛吗?一定是一个独立小句吗?
%\item Is the sentence an island? Does it have to be an independent clause?
\end{itemize}
在这些方面存在的差异需要在语法理论中反映出来。Sag用多种图式的分析方式确保父节点的范畴和语义与两个子节点的属性对应。两类构式所有限制都反映在一个承继\isc{承继}\is{inheritance}层级中,所以构式之间的相似点可以得到解释。这一分析当然也可以在GB理论中用空中心语\isc{空中心语}\is{empty head}来实现。还必须找到某种方法来反映这些构式之间的相同点。可以将对于空中心语的限制放在承继层级中。那么,不同分析方法只是表示方法的不同。如果想要在语法中避免空成分,那么短语方法是很好的。\isc{关系小句|)}\is{relative clause|)}\isc{疑问小句|)}\is{interrogative clause|)}\il{英语|)}\il{English|)}
%The variation that exists in this domain has to be captured somehow by a theory of grammar. Sag develops an analysis with multiple schemata
%that ensure that the category and semantic contribution of the mother node correspond to the properties of both daughters. The constraints for both classes of
%constructions and specific constructions are represented in an inheritance hierarchy\is{inheritance} so that the similarities between the constructions can be
%accounted for. The analysis can of course also be formulated in a GB"=style using empty heads\is{empty head}. One would then have to find some way of capturing
%the generalizations pertaining to the construction. This is possible if one represents the constraints on empty heads in an inheritance hierarchy. Then, the approaches
%would simply be notational variants of one another. If one wishes to avoid empty elements in the grammar, then the phrasal approach would be preferable.
%\is{relative clause|)}\is{interrogative clause|)}\il{English|)}

\subsection{N-P-N构式}
\label{Abschnitt-NPN-Konstruktion}

\mbox{} \citet{Jackendoff2008a}\isc{构式!N-P-N|(}\is{construction!N-P-N|(}讨论了英语的N-P-N构式。这种构式的例子见(\mex{1}):
%\mbox{} \citet{Jackendoff2008a}\is{construction!N-P-N|(} discusses the English N-P-N construction. Examples of this construction are given in (\mex{1}):
\eal
\ex 
\gll day by day, paragraph by paragraph, country by country\\
     天 \textsc{prep} 天 段 \textsc{prep} 段 国 \textsc{prep} 国\\
\mytrans{一天天,一段段,一国国}
%day by day, paragraph by paragraph, country by country
\ex
\gll dollar for dollar, student for student, point for point\\
     美元 \textsc{prep} 美元 学生 \textsc{prep} 学生 点 \textsc{prep} 点\\
\mytrans{一美元一美元,一个学生一个学生,一点一点} 
%dollar for dollar, student for student, point for point
\ex 
\gll face to face, bumper to bumper\\
     脸 \textsc{prep} 脸 满杯 \textsc{prep} 满杯\\
\mytrans{面对面,杯对杯}
%face to face, bumper to bumper
\ex
\gll term paper after term paper, picture after picture\\
     学期 论文 \textsc{prep} 学期 论文 图 \textsc{prep} 图\\
\mytrans{一篇学期论文接一篇学期论文,一幅画接一幅画} 
%term paper after term paper, picture after picture
\ex 
\gll book upon book, argument upon argument\\
     书籍 \textsc{prep} 书籍 论据 \textsc{prep} 论据\\
\mytrans{一本书接着一本书,一段评论接着一段评论}
%book upon book, argument upon argument
\zl
该构式相当受限:带冠词名词或者复数名词都不能出现在该构式中。在该构式中第一个名词和第二个名词的语音形式必须相同。德语中也有相似的构式:
%This construction is relatively restricted: articles and plural nouns are not allowed. The phonological content of the first noun has to correspond to that
%of the second. There are also similar constructions in German:
\eal
% Und du läufst Rüssel an Schwanz hinterher
%\ex Sie lagen Gesicht an Gesicht im Bett.
\ex 
\gll Er hat Buch um Buch verschlungen.\\
	 他 \textsc{aux} 书 \textsc{prep} 书 吞咽\\
\mytrans{他疯狂地一本接一本地读书。}
%\gll Er hat Buch um Buch verschlungen.\\
%	 he has book around book swallowed\\
%\mytrans{He binge-read book after book.}
\ex 
\gll Zeile für Zeile\footnotemark\\
	 行 \textsc{prep} 行\\
\mytrans{一行接一行}
%\gll Zeile für Zeile\footnotemark\\
%	 line for line\\
%\mytrans{line by line}
\footnotetext{%
  \emph{Zwölf Städte}. Einstürzende Neubauten. Fünf auf der nach oben offenen Richterskala, 1987.
}
\zl
确定这种N-P-N构式的意义并不简单。Jackendoff大致总结出了这一构式的意义,即“很多X连续地出现”。
%Determining the meaning contribution of this kind of N-P-N construction is by no means trivial. Jackendoff suggests the meaning
%\emph{many Xs in succession} as an approximation.

Jackendoff从句法的角度来解释该构式会存在问题,因为很难简单地确定哪个成分是中心语。另外,如果遵循\xbarc 理论的假设,也无法确定剩余部分的结构是什么。如果假设介词um是中心语,那么该构式应该与NP组合,但是事实并非如此:
%Jackendoff points out that this construction is problematic from a syntactic perspective since it is not possible
%to determine a head in a straightforward way. It is also not clear what the structure of the remaining material is if one is working under assumptions of
%\xbart. If the preposition \emph{um} were the head in (\mex{0}a), then one would expect that it is combined with an NP, however this is not possible:
\eal
\ex[*]{
\gll Er hat dieses Buch um jenes Buch verschlungen.\\
	 他 \textsc{aux} 那 书 \textsc{prep} \textsc{det} 书 吞咽\\
%\gll Er hat dieses Buch um jenes Buch verschlungen.\\
%	 he has this book around this book swallowed\\
} 
\ex[*]{
\gll Er hat ein Buch um ein Buch verschlungen.\\
	 他 \textsc{aux} 一 书 \textsc{prep} \textsc{det} 书 吞咽\\
%\gll Er hat ein Buch um ein Buch verschlungen.\\
%	 he has a book around a book swallowed\\
}
\zl
% Sie lagen sein Gesicht an ihrem Gesicht im Bett.
对于这种结构,需要假设一个介词在其右边选择一个名词,并且该介词如果真的在其右边选择一个名词的话,就必须在其左边选择一个同形的名词。对于N-um-N和N-für-N来说,完全不清楚整个构式与单个介词有什么关系。当然也可以为这种现象提出一种词汇分析,但是情况与结果构式不同,在结果构式中简单动词的语义扮演重要角色。另外,与结果构式不同,构式N-P-N中各成分的顺序是固定不变的。不可能提取一个名词或者将介词放在两个名词之前。从句法上来讲,一些N-P-N组合很像一个NP\citep[\page 9]{Jackendoff2008a}:
%For this kind of structures, it would be necessary to assume that a preposition selects a noun to its right and, if it find this, it then requires
%a second noun of this exact form to its left. For N-\emph{um}-N and N-\emph{für}-N, it is not entirely clear  what the entire construction has to do with
%the individual prepositions. One could also try to develop a lexical analysis for this phenomenon, but the facts are different to those for resultative constructions:
%in resultative constructions, the semantics of simplex verbs clearly plays a role. Furthermore, unlike with the resultative construction, the order of the component
%parts of the construction is fixed in the N-P-N construction. It is not possible to extract a noun or place the preposition in front of both nouns. Syntactically,
%the N-P-N combination with some prepositions behaves like an NP \citep[\page 9]{Jackendoff2008a}:
\ea
\gll Student after/upon/*by student flunked.\\
	 学生 \textsc{prep} {} \textsc{prep} {} \hspaceThis{*}\textsc{prep} 学生 退学\\
\mytrans{学生一个接一个地退学}
%Student after/upon/*by student flunked.
\z
如果将介词看作是构式的中心语,这一点也非常奇怪。
%This is also strange if one wishes to view the preposition as the head of the construction.

与词汇分析相反,Jackendoff为N-after-N组合提出了下面的短语构式方法:
%Instead of a lexical analysis, Jackendoff proposes the following phrasal construction for N-\emph{after}-N combinations:
\ea
\begin{tabular}[t]{@{}ll@{}}
意义:& 许多 X$_i$s 相连 [或者不管它是如何编码的]\\
句法:& [\sub{NP} N$_i$ P$_j$ N$_i$]\\
语音:& Wd$_i$ after$_j$ Wd$_i$\\
%Meaning: & MANY X$_i$s IN SUCCESSION [or however it is encoded]\\
%Syntax:  & [\sub{NP} N$_i$ P$_j$ N$_i$]\\
%Phonology: & Wd$_i$ after$_j$ Wd$_i$\\
\end{tabular}
\z
% Nimmt man die Merkmalsgeometrie von  \citet{ps2} an, so kann man eine lexikalische Analyse nicht
% formulieren, da Köpfe die phonologischen Eigenschaften ihrer Argumente nicht selegieren
% können.\footnote{%
%   Eine lexikalische Analyse wird doch möglich, wenn m
%
整个结构的意义以及N-P-N构式具有NP的句法属性都反映在构式层面上。
%The entire meaning as well as the fact that the N-P-N has the syntactic properties of an NP would be captured on the construction level.

我已经在\ref{sec-headless-constructions-dg}讨论了 \citet{Bargmann2015a}提出的例子,这些例子显示N-P-N构式后面还可以附加P-N组合:
%I already discussed examples by  \citet{Bargmann2015a} in Section~\ref{sec-headless-constructions-dg}
%that show that N-P-N constructions may be extended by further P-N combinations:
\ea
\gll Day after day after day went by, but I never found the courage to talk to her.\\
	 天 \textsc{prep} 天 \textsc{prep} 天 走 \textsc{prep} 但是 我 从未 找到 \textsc{det} 勇气 \textsc{inf} 说话 \textsc{prep} 她\\
\mytrans{日子一天一天一天地过去了,但是我始终没有鼓起勇气跟她说话。}
%Day after day after day went by, but I never found the courage to talk to her.
\z
所以,并非假设N-P-N模式,Bargmann假设了(\mex{1})所示的模式,其中`+'\is{$+$}代表某个序列至少重复一次。
%So rather than an N-P-N pattern Bargmann suggests the pattern in (\mex{1}), where `+'\is{$+$} stands for at
%least one repetition of a sequence.
\ea
N (P N)+
\z
正如我在第~\pageref{n-p-n-plus-cx}页指出的那样,该模式在基于选择的方法中很难实现。虽然可以假设一个N可以携带任意数量的P-N结构,但是这对于中心语来说是不常见的。与此相反,可以假设一种嵌套,那么N就可以与一个P组合然后再跟一个N-P-N组合,最终得到N-P-N-P-N组合。但是这种方法很难确保整个构式中的名词形式一致。为了限制这种一致性,与N-P-N组合的N必须能够限制深层嵌套在N-P-N对象中的名词(也可以参看\ref{sec-locality})。
%As was pointed out on page~\pageref{n-p-n-plus-cx} this pattern is not easy to cover in
%selection"=based approaches. One could assume that an N takes arbitrarily many P-N combinations,
%which would be very unusual for heads. Alternatively, one could assume recursion, so N would be
%combined with a P and with an N-P-N to yield N-P-N-P-N. But such an analysis would make it
%really difficult to enforce the restrictions regarding the identity of the nouns in the complete
%construction. In order to enforce such an identity the N that is combined with N-P-N would have to
%impose constraints regarding deeply embedded nouns inside the embedded N-P-N object (see also Section~\ref{sec-locality}).

G.\  \citet{GMueller2011a}为N-P-N构式提出了一种词汇分析。他假设介词可以有一个特征\textsc{redup}。在Buch um Buch(一本书接一本书)分析中,介词与右边名词组合um Buch。就语音形式而言,“Buch”(书)的重叠由\textsc{redup}特征激发,因此产生了Buch um Buch(一本书接一本书)。这一分析也有Jackendoff指出的问题:为了推导出构式的语义,语义必须储存在重复介词的词项中(或者在解释句法的相关后续部分中)。另外,也不清楚这一重复分析如何处理Bargmann提出的例子。
%G.\  \citet{GMueller2011a} proposes a lexical analysis of the N-P-N construction. He assumes that prepositions can have a feature \textsc{redup}.
%In the analysis of \emph{Buch um Buch} `book after book', the preposition is combined with the right noun \emph{um Buch}. In the phonological component, reduplication of \emph{Buch} is triggered
%  by the \textsc{redup} feature, thereby yielding \emph{Buch um Buch}.
%This analysis also suffers from the problems pointed out by Jackendoff: in order to derive the semantics of the construction, the semantics would have to be present
%in the lexical entry of the reduplicating preposition (or in a relevant subsequent component that
%interprets the syntax).  Furthermore it is unclear how a reduplication analysis would deal with the
%Bargmann data.
% Zwar ist es richtig, dass
% Präpositionen im Deutschen auch ohne Artikel verwendet werden können, aber 
\isc{构式!N-P-N|)}\is{construction!N-P-N|)}
\isc{构式语法(CxG)|)}\is{Construction Grammar (CxG)|)}


\if 0

Draußen, auf der regennassen Friedrichstraße, quietschten die Straßenbahnen um die Kurve am
Oranienburger Tor.

Olaf Schwarzbach Forelle Grau: Die Geschichte von OL. Berlin: Berlin Verlag, 2015, S.287.

\fi
%      <!-- Local IspellDict: en_US-w_accents -->
