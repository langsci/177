\section*{致谢}
%\section*{Acknowledgments}

我要感谢跟我讨论本书早期德语版的
%I would like to thank 
David Adger\aimention{David Adger}、Jason Baldridge\aimention{Jason Baldridge}、 Felix Bildhauer\aimention{Felix Bildhauer}、
Emily M.\ Bender\aimention{Emily M. Bender}、Stefan Evert\aimention{Stefan Evert}、Gisbert Fanselow\aimention{Gisbert Fanselow}, 
Sandiway Fong\aimention{Sandiway Fong}、Hans"=Martin Gärtner\aimention{Hans"=Martin G"artner}、Kim Gerdes\aimention{Kim Gerdes}、Adele Goldberg\aimention{Adele Goldberg}、Bob Levine\aimention{Robert D. Levine}、Paul Kay\aimention{Paul Kay}、
Jakob Maché\aimention{Jakob Mach{\'e}}、Guido Mensching\aimention{Guido Mensching}、Laura Michaelis\aimention{Laura Michaelis}、Geoffrey Pullum\aimention{Geoffrey K. Pullum}、Uli Sauerland\aimention{Uli Sauerland}、Roland Schäfer\aimention{Roland Sch"afer},
Jan Strunk\aimention{Jan Strunk}、Remi van Trijp\aimention{Remi van Trijp}、Shravan Vasishth\aimention{Shravan Vasishth}、Tom Wasow\aimention{Tom Wasow}和
%and
Stephen Wechsler\aimention{Stephen Mark Wechsler}
%for discussion and 
%
以及对本书早期德语版本提出评论的
Monika Budde\aimention{Monika Budde}、Philippa Cook\aimention{Philippa Helen Cook}、Laura Kallmeyer\aimention{Laura Kallmeyer}、Tibor Kiss\aimention{Tibor Kiss}、Gisela Klann"=Delius\aimention{Gisela Klann"=Delius}、 Jonas Kuhn\aimention{Jonas Kuhn},
Timm Lichte\aimention{Timm Lichte}、% für Kommentare zum TAG"=Kapitel 
Anke Lüdeling\aimention{Anke L"udeling}、Jens Michaelis\aimention{Jens Michaelis}、Bjarne Ørsnes\aimention{Bjarne {\O}rsnes}、Andreas Pankau\aimention{Andreas Pankau}、    % Chomsky 2013
Christian Pietsch\aimention{Christian Pietsch}、Frank Richter\aimention{Frank Richter}、Ivan Sag\aimention{Ivan A. Sag}
和
%and
Eva Wittenberg\aimention{Eva Wittenberg}。
%for comments on earlier versions of the German edition of this book and
%
%
我还要感谢在本书的早期版本中提出评论的Thomas Groß\aimention{Thomas M. Gro{\ss}}、Dick Hudson\aimention{Richard Hudson}、
Sylvain Kahane\aimention{Sylvain Kahane}、Paul Kay\aimention{Paul Kay}、Haitao Liu (刘 海涛)\aimention{Haitao Liu}、Andrew McIntyre\aimention{Andrew McIntyre}、Sebastian Nordhoff\aimention{Sebastian Nordhoff}、Tim Osborne\aimention{Timothy Osborne}、
Andreas Pankau\aimention{Andreas Pankau}和Christoph Schwarze\aimention{Christoph Schwarze}。
%for comments on earlier versions of this book. 
感谢Leonardo Boiko和Sven Verdoolaege挑出了错别字。特别感谢Martin Haspelmath\aimention{Martin Haspelmath}对本书英文版的早期版本提出的详细评论。
%Thanks to Leonardo Boiko and Sven Verdoolaege for pointing out typos.
%Special thanks go to Martin Haspelmath\aimention{Martin Haspelmath} for very detailed comments on an
%earlier version of the English book. 

%我还要感谢在此书的中文译本中给予我帮助的学者与学生们,他们是曹晓玉、刘海涛、刘晓、卢达威、詹卫东。
%I would also like to thank to 
%刘海涛、刘晓
%for comments on the Chinese version of this book.

本书是语言科学出版社出版的通过公开评审的第一本书(详见下面)。我感谢Dick Hudson、Paul Kay、Antonio Machicao y Priemer、Andrew McIntyre、Sebastian Nordhoff和一位匿名评论者对本书提出的评论。这些评论记录在\href{\lsURL}{本书的下载页面}中。除此之外,本书还经过了公开校对的阶段(也请看下面)。有些校对者不仅做了校对的工作,还提出了具有高度价值的评论。我决定将这些评论作为附加的公开评论发布出来。在这里需要特别感谢的有Armin Buch、Leonel de Alencar、Andreas Hölzl、Gianina Iordăchioaia、Timm Lichte、Antonio Machicao y Priemer和Neal Whitman。
%This book was the first Language Science Press book that had an open review phase (see below). I
%thank Dick Hudson, Paul Kay, Antonio Machicao y Priemer, Andrew McIntyre, Sebastian Nordhoff, and one anonymous open
%reviewer for their comments. Theses comments are documented at the \href{\lsURL}{download page of
%  this book}. In addition the book went through a stage of community proofreading (see also
%below). Some of the proofreaders did much more than proofreading, their comments are highly
%appreciated and I decided to publish these comments as additional open reviews.
%%Armin Buch, 
%Leonel de Alencar,
%Andreas Hölzl,
%Gianina Iordăchioaia,
%Timm Lichte,
%Antonio Machicao y Priemer, and
%Neal Whitman
%deserve special mention here.

我感谢Wolfgang Sternefeld和Frank Richter,他们对本书的德语版做了详尽的评论。他们指出了一些错误和疏漏之处,我们在德语的第二版中进行了改正,这些都在英语版中有所显示。
%I thank Wolfgang Sternefeld and Frank Richter, who wrote a detailed review of the German version of
%this book \citep{SR2012a}. They pointed out some mistakes and omissions that were corrected in the second edition
%of the German book and which are of course not present in the English version.

感谢所有对本书进行评论和提出改进意见的学生们。特别是Lisa DeringerAleksandra Gabryszak、Simon Lohmiller、Theresa Kallenbach、Steffen Neu\-schulz、Reka Meszaros-Segner、Lena Terhart和Elodie Winckel。
%Thanks to all the students who commented on the book and whose questions lead to improvements. 
%Lisa Deringer,
%Aleksandra Gabryszak, % Student SS 2010 gute Fragen, GB-Verbbewegung und LMT
%Simon Lohmiller, %Student, Typos und allgemeine Anregung zu Einführungskapitel
%Theresa Kallenbach, %Studentin, GPSG
%Steffen Neu\-schulz,  % Student SS 2010 gute Fragen
%Reka Meszaros-Segner,
%Lena Terhart and
%Elodie Winckel deserve special mention.

由于本书是基于我在语言理论领域中的所有经验,我想感谢那些在会议、工作坊、暑期学校期间以及通过邮件跟我讨论过语言学的学者们。
%Since this book is built upon all my experience in the area of grammatical theory, I want to thank
%all those with whom I ever discussed linguistics during and after talks at conferences, workshops,
%summer schools or via email.
特别值得列出的有Werner Abraham、John Bateman、
Dorothee Beermann、
Rens Bod、
Miriam Butt、
Manfred Bierwisch、
Ann Copestake、
Holger Diessel、
Kerstin Fischer、
Dan Flickinger、
Peter Gallmann、
%Adele Goldberg,  included above
Petter Haugereid、
Lars Hellan、
% Paul Kay, included above
Tibor Kiss、
Wolfgang Klein、 
Hans"=Ulrich Krieger、
%Emily M. Bender, included above
Andrew McIntyre、
Detmar Meurers、
%Laura Michaelis,  included above
Gereon Müller、
Martin Neef、
Manfred Sailer、 
Anatol Stefanowitsch、
Peter Svenonius、
Michael Tomasello、 
Hans Uszkoreit、
Gert Webelhuth、
% Stephen Wechsler included above
Daniel Wiechmann 
和
%and 
Arne Zeschel。
%deserve special mention.

我感谢Sebastian Nordhoff针对递归(recursion\is{recursion})这一术语的评论。
%I thank Sebastian Nordhoff for a comment regarding the completion of the subject index entry for \emph{recursion}\is{recursion}.

Andrew Murphy翻译了英文版中的第一章到第三章,第五章到第十章,以及第十二章到第二十三章的内容。特别感谢!
%Andrew Murphy translated part of Chapter~1 and the Chapters~2--3, 5--10, and 12--23. Many thanks for this!

我还要感谢27位校对着,他们的工作对本书的改进提供了极大的帮助。我从他们每个人那里获得的意见比从出版商那里获得的意见还要多。有些意见是针对内容的,而不是错别字和格式的。没有一位受雇于出版商的校对人员能够发现这些错误与不一致的地方,因为出版商的员工没有人懂得本书覆盖的所有语法理论。
%I also want to thank the 27 community proofreaders (\makeatletter\@proofreader\makeatother) that each worked on one or more chapters and
%really improved this book. I got more comments from every one of them than I ever got for a book
%done with a commercial publisher. Some comments were on content rather than on typos and layout
%issues. No proofreader employed by a commercial publisher would have spotted these mistakes and
%inconsistencies since commercial publishers do not have staff that knows all the grammatical
%theories that are covered in this book. 

过去的几年中,学界举办了几场理论比较的工作坊。我受邀参加了其中的三个工作坊。感谢Helge Dyvik\aimention{Helge Dyvik}和Torbjørn Nordgård\aimention{Torbj{\o}rn
  Nordg{\r{a}}rd}邀请我参加2005年在卑尔根举办的挪威博士生秋季学校“对比中的语言与理论”(\emph{Languages and Theories in Contrast})。Guido Mensching\aimention{Guido Mensching}和Elisabeth
Stark\aimention{Elisabeth Stark}邀请我参加了2007年在柏林自由大学举办的“比较语言与比较理论:生成语法与构式语法”(\emph{Comparing Languages and Comparing Theories:
  Generative Grammar and Construction Grammar})工作坊。Andreas Pankau\aimention{Andreas Pankau} 邀请我参加2009年在乌得勒支举办的“比较框架”(\emph{Comparing
  Frameworks})工作坊。我在跟参加这些活动的学者们的讨论中受益良多,本书也受益于这些交流。
%During the past years, a number of workshops on theory comparison have taken place. I was invited to three of them.
%I thank Helge Dyvik\aimention{Helge Dyvik} and Torbjørn Nordgård\aimention{Torbj{\o}rn
%  Nordg{\r{a}}rd}\todostefan{Indexeinträge für Torbjorn and Bjarne do not work} for inviting me to the fall school for Norwegian PhD
%students  \emph{Languages and Theories in Contrast}, which took place 2005 in Bergen. Guido Mensching\aimention{Guido Mensching} and Elisabeth
%Stark\aimention{Elisabeth Stark} invited me to the workshop \emph{Comparing Languages and Comparing Theories:
%  Generative Grammar and Construction Grammar}, which took place in 2007 at the Freie Universität
%Berlin and Andreas Pankau\aimention{Andreas Pankau} invited me to the workshop \emph{Comparing
%  Frameworks} in 2009 in Utrecht. I really enjoyed the discussion with all participants of these
%events and this book benefited enormously from the interchange.

感谢Peter Gallmann\aimention{Peter Gallmann},我在耶拿期间跟他讨论了他课件中\gb 理论的部分。第~\ref{Abschnitt-T-Modell}--\ref{Abschnitt-GB-Passiv}节与他的版本很像,并参考了其中很多内容。感谢David Reitte提供的组合性范畴语法的\LaTeX{}宏包,Mary Dalrymple和Jonas Kuhn提供的LFG宏包和示例结构,以及Laura Kallmeyer提供的大部分TAG分析中的\LaTeX{}资源。由于与\XeLaTeX 的兼容性问题,大部分树都调整为\texttt{forest}包的格式,但是原始的树和文本都给予了我很多灵感,没有他们,相应章节中的图绝不会像现在这样好看。
%I thank Peter Gallmann\aimention{Peter Gallmann} for the discussion of his lecture notes on \gb
%during my time in Jena. The Sections~\ref{Abschnitt-T-Modell}--\ref{Abschnitt-GB-Passiv} have a
%structure that is similar to the one of his script and take over a lot. Thanks to David Reitter for
%the \LaTeX{} macros for Combinatorial Categorial Grammar, to Mary Dalrymple and Jonas Kuhn for the LFG
%macros and example structures, and to Laura Kallmeyer for the \LaTeX{} sources of most of the TAG
%analyses. Most of the trees have been adapted to the \texttt{forest} package because of compatibility issues
%with \XeLaTeX, but the original trees and texts were a great source of inspiration and without them
%the figures in the respective chapters would not be half as pretty as they are now.

我感谢Sašo Živanović实现了\LaTeX{}的宏包\texttt{forest}。这个宏包简化了树、依存图和类型层级的格式。
我还要感谢他在邮件和 \href{http://www.stackexchange.com}{stackexchange} 上的具体帮助。当然,对于那些
在stackexchange上活跃的人所提供的帮助仅仅表示感谢是不够的:大部分有关本书格式的细节问题或者现在由
\lsp 使用的\LaTeX{}类型的应用都在几分钟内得到解答。感谢你们!因为本书是一本CC-BY版权下的公开图书,它
也是一本公开资源的书。感兴趣的读者可以在 \url{https://github.com/langsci/25} 拷贝这些资源。通过将本书的资源公开,我将\LaTeX{}大师们提供的资源传递下去,并希望其他人能够从中获益,并且学会按照更好看和更高效的方式来编写他们的语言学论文。
%I thank Sašo Živanović for implementing the \LaTeX{} package \texttt{forest}. It really simplifies
%typesetting of trees, dependency graphs, and type hierarchies. I also thank him for individual help
%via email and on \href{http://www.stackexchange.com}{stackexchange}. In general, those active on stackexchange could not be thanked
%enough: most of my questions regarding specific details of the typesetting of this book or the
%implementation of the \LaTeX{} classes that are used by \lsp now have been answered within several
%minutes. Thank you! Since this book is a true open access book under the CC-BY license, it can also
%be an open source book. The interested reader finds a \textsc{cop}y of the source code at \url{https://github.com/langsci/25}. By making the book open source I pass on the knowledge provided by the \LaTeX{} gurus and
%hope that others benefit from this and learn to typeset their linguistics papers in nicer and/or
%more efficient ways.

我还要感谢Viola Auermann、Antje Bahlke、Sarah Dietzfelbinger、Lea Helmers和Chiara Jancke所做的大量复印工作。Viola还在英译本定稿前的最后阶段帮我校对。我还要感谢我的(前任)实验室成员们Felix Bildhauer、Philippa Cook、Janna Lipenkova、Jakob Maché、Bjarne Ørsnes和Roland Schäfer\aimention{Roland Sch{\"a}fer}。他们在教学中以及其他方面都给予我很多帮助。从2007年直到本书第一版德语版的教材出版的这些年中,德语语言学的三个终身教职中有两个职位都是空缺的,如果没有他们的帮助,我是无法完成教学任务并完成这本书的。
%Viola Auermann and Antje Bahlke, Sarah Dietzfelbinger, Lea Helmers, and Chiara Jancke cannot be thanked enough for their work at the \textsc{cop}y machines. Viola
%also helped a lot with proof reading prefinal stages of the translation.
%I also want to thank my (former) lab members Felix Bildhauer, Philippa Cook, Janna Lipenkova, Jakob Maché,
%Bjarne Ørsnes and Roland Schäfer\aimention{Roland Sch{\"a}fer}, which were mentioned above already
%for other reasons, for their help with teaching. During the years from 2007 until the publication of
%the first German edition of this book two of the three tenured positions in German Linguistics were
%unfilled and I would have not been able to maintain the teaching requirements without their help and
%would have never finished the \emph{Grammatiktheorie} book.

%我要感谢Tibor Kiss针对提问技巧的建议。他的外交式的辞令给我树立了很好的榜样,我希望这点也体现在本书中。
%I thank Tibor Kiss for advice in questions of style. His diplomatic way always was a shining
%example for me and I hope that this is also reflected in this book.

%      <!-- Local IspellDict: en_US-w_accents -->
