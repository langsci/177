\backmatter
\bookmarksetup{startatroot}

% for PoD with two volumes
%\ifoot{}\ofoot{}

% authorindex needs bib-file
\printbibliography[heading=chinese] 
%\bibliography{gt}
%\input grammatical-theory-bib-final
\cleardoublepage

\small
   


\phantomsection%this allows hyperlink in ToC to work
\addcontentsline{toc}{chapter}{索引} 
%\addcontentsline{toc}{chapter}{Index} 
\addcontentsline{toc}{section}{人名索引}
\ohead{人名索引}
\renewcommand{\lsNameIndexTitle}{人名索引}
%\addcontentsline{toc}{section}{Name index}
%\ohead{Name index}
%with biblatex
%\printindex
%without it 
\sloppy
\printindex 
  
\phantomsection%this allows hyperlink in ToC to work
\addcontentsline{toc}{section}{语言索引(汉英对照)}
\ohead{语言索引} 
\renewcommand{\lsLanguageIndexTitle}{语言索引(汉英对照)}
%\addcontentsline{toc}{section}{Language index}
%\ohead{Language index} 
\printindex[lan] 

\phantomsection%this allows hyperlink in ToC to work
\addcontentsline{toc}{section}{语言索引(英汉对照)}
\ohead{语言索引} 
\newcommand{\lsLanguageIndexTitleEng}{语言索引(英汉对照)}
%\addcontentsline{toc}{section}{Language index}
%\ohead{Language index} 
\printindex[lane] 


  
\phantomsection%this allows hyperlink in ToC to work
\addcontentsline{toc}{section}{术语索引(汉英对照)}
%subject index (Chinese to English)
%\addcontentsline{toc}{section}{术语索引}
\ohead{术语索引}
\renewcommand{\lsSubjectIndexTitle}{术语索引(汉英对照)}
%\renewcommand{\lsSubjectIndexTitle}{术语索引}
%\addcontentsline{toc}{section}{Subject index}
%\ohead{Subject index}
\printindex[sbj]


\phantomsection%this allows hyperlink in ToC to work
\addcontentsline{toc}{section}{术语索引(英汉对照)}
%subject index (English to Chinese)
%\addcontentsline{toc}{section}{中文术语索引}
\ohead{中文术语索引}
%\addcontentsline{toc}{section}{Subject index Chinese}
%\ohead{Subject index Chinese}
\printindex[sbc]
