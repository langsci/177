%% -*- coding:utf-8 -*-
\documentclass[ number=1
                ,series=tbls,
	        %,blackandwhite
	        ,smallfont
                ,bibtex
	        %,draftmode  
		  ]{langsci/langscibook}                          

\usepackage{langsci-forest-setup,makros.2e,dalrymple}

%\usepackage{forest,lsp-forest-setup-texlive-2013,makros.2e,dalrymple}

%\usetikzlibrary{tikzmark}

% http://tex.stackexchange.com/questions/218417/replacing-tree-dvips-connect-nodes-in-a-tabular-environment/218458#218458
% Instead of using the package tikzmark, you can define your own \tikzmark being a regular node. There's no need to use tcolorbox package.

\newcommand{\mynode}[2]%
           {\tikz[baseline=(#1.base), remember picture]\node[outer sep=0pt, inner sep=0pt] (#1) {#2};}

\newcommand{\mysubnode}[2]%
           {\tikz[baseline=(#1.base), remember picture]\node[outer sep=0pt, inner sep=0pt] (#1) {#2};}

           
\newcommand{\centtab}[1]{\begin{tabular}[t]{@{}c@{}}#1\end{tabular}}

\begin{document}
\thispagestyle{empty}

\begin{forest}
sm edges
[\subnode{cp}{CP}
  [\centtab{(\up \textsc{df})= (\up \textsc{comp}* \textsc{gf})\\
            (\up \textsc{df})=\down \\
            \mysubnode{npobj}{NP}}
        [den Apfel;the apple, roof]]
  [\centtab{\up~=~\down\\
    \mysubnode{cbar}{\cbar}}
    [\centtab{\up~=~\down\\
     \mysubnode{c}{C}} [verschlingt;devours]]
    [\centtab{\up~=~\down\\
     \mysubnode{vp}{VP}}
      [\centtab{(\upsp \textsc{subj}) = \down\\
                \mysubnode{npsubj}{NP}}
        [David;David]]]]]
\end{forest}\hfill
\raisebox{13em}{%
\mynode{all}{\lfgms{ pred & `VERSCHLINGEN\sliste{\subj,\obj}'~~\\
         subj & \mynode{fdavid}{\lfgms{ pred &  `DAVID' \\
                                      case & nom\\
                   }}\\
         tense & PRES\\
         topic & \mynode{fapple}{\lfgms{ pred & `APFEL'\\
                                    case & acc\\
                   }}\\
         obj & \mynode{obj}{}
       }}}
\begin{tikzpicture}[overlay,remember picture] 
\draw[->] (cp)     to[out= 0,in=180] (all.west);
\draw[->] (cbar)   to[out=10,in=180] (all.west);
\draw[->] (c)      to[out=10,in=180] (all.west);
\draw[->] (vp)     to[out= 0,in=180] (all.west);
\draw[->] (npsubj) to[out=15,in=200] (fdavid.west);
\draw[->] (npobj)  to[out= 0,in=215] (fapple.west);
%
% the arc in the f-structure
%\draw     (obj)    .. controls +(right:10em) and +(up:4em)
%                   .. (fapple.east);

\draw      (obj)   to[out=0,in=0,distance=6em] (fapple.east);
\end{tikzpicture}
\end{document}



