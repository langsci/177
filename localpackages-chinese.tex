%% -*- coding:utf-8 -*-

% general stuff for grammar theory is in local packages.tex
% everything that is specific to Chinese is in here.

%% below are added by Lulu for the Chinese version of the book
\usepackage{zhnumber}
% to type­set Chi­nese rep­re­sen­ta­tions of num­bers.  

\usepackage{titlesec}  
% for Chinese style of titles of chapters and sections.
\renewcommand{\partname}{第\thepart{}部分}  
\def\thepart{\zhdig{part}}
%\pagenumbering {zhnum}
\renewcommand{\chaptername}{第\zhnum{chapter}章}  

% The following would also change the chapter and section number in the heading. We do not want
% this. 27.10.2016
%\def\thechapter{\zhnum{chapter}}
%\def\thesection{\zhnum{section}}


%re\newcommand{\sectionname}{节}  
\renewcommand{\figurename}{图}  
\renewcommand{\tablename}{表}  
\renewcommand{\bibname}{参考文献}  
\renewcommand{\contentsname}{目~~~~录}  
\renewcommand{\listfigurename}{图~目~录}  
\renewcommand{\listtablename}{表~目~录}  
\renewcommand{\indexname}{索~引}  
\renewcommand{\abstractname}{\Large{摘~要}}  
\newcommand{\keywords}[1]{\\ \\ \textbf{关~键~词}:#1}  
\titleformat{\chapter}[block]{\center\Large\bf}{\chaptername}{20pt}{}  
\titleformat{\section}[block]{\large\bf}{\thesection}{10pt}{}  
\titleformat{\tableofcontents}[block]{\center\Large\bf}{\contensname}{20pt}{}  
\titleformat{\part}[block]{\center\Large\bf}{\partname}{20pt}{}  
\titleformat{\index}[block]{\center\Large\bf}{\indexname}{20pt}{}  

\usepackage{titlesec}  
% for Chinese style of table of contents
%\titlecontents{\chapter}

\usepackage{indentfirst}
% In Chinese, each paragraph should be indented.
\setlength{\parindent}{2em}
% the indented space should be two characters.

% quotes should be in simkai(楷体)
\newCJKfontfamily[\fontpath simkai]\simkai{KaiTi}[Mapping=tex-text,Ligatures=Common,Scale=MatchUppercase]
\AtBeginEnvironment{quotation}{\simkai}


%\usepackage{newunicodechar}
%\newunicodechar{(}{(}
