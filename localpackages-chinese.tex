%% -*- coding:utf-8 -*-

% general stuff for grammar theory is in localpackages.tex
% everything that is specific to Chinese is in here.

%% below are added by Lulu for the Chinese version of the book
\usepackage{zhnumber}
% to type­set Chinese representations of numbers.  

%\usepackage[UTF8, heading = true]{ctex}

%\usepackage{titlesec}  
\AtBeginDocument{

% % for Chinese style of titles of chapters and sections.
% \renewcommand{\partname}{第\thepart{}部分}  
\def\thepart{\zhdig{part}}
% %\pagenumbering {zhnum}
% \renewcommand{\chaptername}{第\zhnum{chapter}章}  
% \renewcommand{\appendixname}{附录}
% \appto\appendix{
% \renewcommand{\chaptername}{\appendixname\thechapter} 
% }
%\renewcommand{\chapterautorefname}{第\zhnum{chapter}章}%

% for Chinese style of pages. 15.4.2019
%   \def\reftextfaceafter{在\reftextvario{下}{后} 一页}%
%    \def\reftextfacebefore{在\reftextvario{上}{前} 一页}%
%    \def\reftextafter{在前一页}%
%    \def\reftextbefore{在后一页}%
%    \def\reftextcurrent{在这一页}%
%    \def\reftextfaraway#1{在第~\pageref{#1}页}%
%    \def\reftextpagerange#1#2{在第~\pageref{#1}--\pageref{#2}页}%
%    \def\reftextlabelrange#1#2{从第\ref{#1}页到第~\ref{#2}页}%

% The following would also change the chapter and section number in the heading. We do not want
% this. 27.10.2016
%\def\thechapter{\zhnum{chapter}}
%\def\thesection{\zhnum{section}}

%\lsBookLanguageChinese
%\renewcommand{\lsBookLanguage}{chinese}

%re\newcommand{\sectionname}{节}  
\renewcommand{\figurename}{图}  
\renewcommand{\tablename}{表}  
%\renewcommand{\bibname}{参考文献}  
\renewcommand{\contentsname}{目~~~~录}  
\renewcommand{\listfigurename}{图~目~录}  
\renewcommand{\listtablename}{表~目~录}  
\renewcommand{\indexname}{索~引}  
\renewcommand{\abstractname}{\Large{摘~要}}  
\newcommand{\keywords}[1]{\\ \\ \textbf{关~键~词}:#1}  
% \titleformat{\chapter}[block]{\center\Large\bf}{\chaptername}{20pt}{}  
% \titleformat{\appendix}[block]{\center\Large\bf}{\appendixname}{20pt}{}  
% \titleformat{\section}[block]{\large\bf}{\thesection}{10pt}{}  
% \titleformat{\tableofcontents}[block]{\center\Large\bf}{\contensname}{20pt}{}  
% \titleformat{\part}[block]{\center\Large\bf}{\partname}{20pt}{}  
% \titleformat{\index}[block]{\center\Large\bf}{\indexname}{20pt}{}  
}
% for Chinese style of table of contents
%\titlecontents{\chapter}


% scrbook does the trick
% \renewcaptionname{USenglish}{\contentsname}{目~~~~录}
% % for Chinese style of titles of chapters and sections.
% \renewcaptionname{USenglish}{\partname}{第\thepart{}部分}  
% \def\thepart{\zhdig{part}}
% %\pagenumbering {zhnum}
% \renewcaptionname{USenglish}{\chaptername}{第\zhnum{chapter}章}  
% \renewcaptionname{USenglish}{\appendixname}{附录}
% \appto\appendix{
% \renewcaptionname{USenglish}{\chaptername}{\appendixname\thechapter} 


\renewcaptionname{USenglish}{\contentsname}{目~~~~录}
\renewcaptionname{USenglish}{\appendixname}{附录}

% https://tex.stackexchange.com/a/542570/18561

\renewcommand*{\partformat}{第\zhdig{part}部分\hspace{20pt}}
\renewcommand*{\partheadmidvskip}{}
\renewcommand*{\chapterformat}{第\zhnum{chapter}章\hspace{20pt}}
\renewcommand*{\raggedchapter}{\centering}
\renewcommand*{\sectionformat}{\thesection\hspace{10pt}}
%\renewcommand*{\appendixformat}{\appendixname\thechapter\hspace{10pt}}

\appto\appendix{
% \renewcommand{\chaptername}{\appendixname\thechapter}
\def\thechapter{\Alph{chapter}} 
}


\addtokomafont{disposition}{\rmfamily}
\setkomafont{partnumber}{\Large}
\setkomafont{part}{\Large}
\setkomafont{chapter}{\Large}
\setkomafont{section}{\large}


\usepackage{indentfirst}
% In Chinese, each paragraph should be indented.
\setlength{\parindent}{2em}
% the indented space should be two characters.

\newcommand{\fontpath}{./fonts/}

% quotes should be in simkai(楷体)
% simkai is MicroSoft
%\newCJKfontfamily[\fontpath/halfish simkai]\simkai{KaiTi}[Mapping=tex-text,Ligatures=Common,Scale=MatchUppercase]

% quotes should be in FZKTJW
% Lulu Wang, 15.05.2020
% FZKTJW.TTF by the FounderType, the url is as follows and the font is in the attached file.
% https://www.foundertype.com/index.php/FindFont/searchFont?keyword=楷体
%
% The FounderType is created by a large group originated from Peking University. There are four types that are free for commercial publishing and personal designing. FZKTJW.TFF is one of these free types.
\newCJKfontfamily[\fontpath/]\simkai{FZKai-Z03S}[Mapping=tex-text,Scale=MatchUppercase]

% indentation of the whole quotation is 2em like the paragraph indentation
% St. Mü. 22.04.2018
\makeatletter
\renewenvironment{quotation}
			   {\list{}{\leftmargin2em\listparindent 2em%
						%\itemindent    \listparindent
						%\rightmargin   \leftmargin
						\parsep        \z@ \@plus\p@}%
				\item\relax\hspace{2em}}% 22.07.2019 added hspace. I guess this is a
                                  % bad hack but it seems to work. St. Mü.
			   {\endlist}
\makeatother
\AtBeginEnvironment{quotation}{\simkai}

% this influences enumerate and itemize
\RequirePackage{enumitem}
\setlist{leftmargin=3em}


%\usepackage{newunicodechar}
%\newunicodechar{(}{(}



% use Chinese parenthesis
% https://tex.stackexchange.com/a/428832/18561
\makeatletter
\renewcommand*{\bibleftparen}{\unspace\blx@postpunct(}
%\renewcommand*{\bibleftparen}{\blx@postpunct(}
\renewcommand*{\bibrightparen}{\blx@postpunct)\unspace\midsentence}
%\renewcommand*{\bibrightparen}{\blx@postpunct)\midsentence}

% The new font seems to require the space after the closing bracket.
\renewcommand*{\bibrightparen}{\blx@postpunct)\midsentence}


\AtBeginDocument{
\newindex{sbc}{scx}{scd}{中文术语索引}
}

\newcommand{\footnoteindex@sbc}[2]{\index[sbc]{#2|infn{#1}}}
\newcommand{\isc}[1]{%
  \if@noftnote%
    \index[sbc]{#1}%
    \else%
    \edef\tempnumber{\thefootnote}%
    \expandafter\footnoteindex@sbc\expandafter{\tempnumber}{#1}%
    %\indexftn{#1}{\value{footnotemark}}%
  \fi%
}

\renewcommand{\lsImpressumCitationText}{
  \onlyAuthor
  \renewcommand{\newlineCover}{}
  \renewcommand{\newlineSpine}{}
  {\lsImpressionCitationAuthor}\if\lsEditorSuffix\empty\else\ \lsEditorSuffix\fi. %
  {\lsYear}. %
  \textit{\@title}\if\@subtitle\empty\else: \textit{\@subtitle}\fi\ %
  (\lsSeriesTitle). %
  柏林:语言科学出版社
%  Berlin: Language Science Press.
}


\renewcommand{\lsImpressum}{
\raggedright

\lsImpressumCitationText

\vfill

本书可在以下网址获取: \\
%This title can be downloaded at:\\
\url{\lsURL}

© \the\year, \iflsCollection the authors\else\@author\fi

\newcommand{\ccby}{CC-BY} 
\ifx\lsCopyright\ccby 
本书的发表采用知识共享署名4.0许可协议(CC BY 4.0)授权:
%Published under the Creative Commons Attribution 4.0 Licence (CC BY 4.0):
http://creativecommons.org/licenses/by/4.0/ 
\else
本书的发表采用知识共享署名—禁止演绎4.0许可协议(CC BY-ND 4.0)授权:
%Published under the Creative Commons Attribution-NoDerivatives 4.0 Licence (CC BY-ND 4.0):
http://creativecommons.org/licenses/by-nd/4.0/ 
\fi

\noindent 
\begin{tabular}{@{}llr@{}}
ISBN: & 完整电子版:& \lsISBNdigital \\
      &	精装:& \lsISBNhardcover\\
      &	简装:& \lsISBNsoftcover\\
      &	美版简装:& \lsISBNsoftcoverus \\
%ISBN: & \multicolumn{2}{r}{Digital, complete work:} & \lsISBNdigital \\
%      &	Hardcover: & vol1:        & \lsISBNhardcoverOne & vol. 2 & \lsISBNhardcoverTwo\\
%      &	Softcover: & vol1:        & \lsISBNsoftcoverOne & vol. 2 & \lsISBNsoftcoverTwo\\
 %     &	Softcover US: & vol1:        & \lsISBNsoftcoverusOne & vol. 2 & \lsISBNsoftcoverusTwo \\
\end{tabular}

\IfStrEq{\lsISSN}{??}		% \IfStrEq from xstring
	{}
	{ISSN: \lsISSN}

\IfStrEq{\lsBookDOI}{??}		% \IfStrEq from xstring
	{}
	{\doi{\lsBookDOI}}


\bigskip

封面设计: 
%Cover and concept of design:
Ulrike Harbort \\
\if\@translator\empty\else
译者:
%Translators:
\@translator \\
\fi
\if\@typesetter\empty\else
排版:
%Typesetting:
\@typesetter \\
\fi
\if\@illustrator\empty\else
插图:
%Illustration:
\@illustrator \\
\fi
\if\@proofreader\empty\else
校对:
%Proofreading:
\@proofreader \\
\fi
\if\@openreviewer\empty\else
公开评审:
%Open reviewing:
\@openreviewer \\
\fi
字体:Linux Libertine、Arimo、DejaVu Sans Mono\lsAdditionalFontsImprint\\
%Fonts: Linux Libertine, Arimo, DejaVu Sans Mono\lsAdditionalFontsImprint\\
排版软件:\XeLaTeX
%Typesetting software: \XeLaTeX


\bigskip

%语言科学出版社\\
Language Science Press\\
Unter den Linden 6\\
10099 Berlin, Germany\\
\href{http://langsci-press.org}{langsci-press.org}

\vfill

本书由柏林自由大学存储和编目 \\[3ex]
%Storage and cataloguing done by FU Berlin \\[3ex]  

\includestoragelogo\\[3ex]

\vfill

\noindent
语言科学出版社无法保证本书中URL链接地址的准确性和可访问性,也无法保证这些网站的内容准确与适当。本书中的商品价格、旅行时间表和其他事实性信息,在初次出版的时候是准确的,但无法保证后继准确。
%\lsp has no responsibility for the persistence or accuracy of URLs for
%external or third-party Internet websites referred to in this
%publication, and does not guarantee that any content on such websites
%is, or will remain, accurate or appropriate. 
%Information regarding prices, travel timetables and other factual information given in this work are correct at the time of first publication but \lsp does not guarantee the accuracy of such information thereafter. 
}


\makeatother



\newcommand{\mytrans}[1]{\trans\quotetrans{#1}}

% use Chinese single quotes and adjust the kerning
\newcommand{\quotetrans}[1]{‘#1’}
\xeCJKsetkern{。}{’}{0.2em}

% The following does not work since we have trnaslations ending in question marks or without punctuation
%\newcommand{\quotetrans}[1]{`#1\hspace{-0.2em}'}



% use a less heavy variant of Libertine: Display font
  %% \setmainfont[
  %%         Ligatures={TeX,Common},
  %%         Path=\fontpath,
  %%         PunctuationSpace=0,
  %%         Numbers={Proportional},
  %%         BoldFont = LinLibertine_RZ.otf ,
  %%         ItalicFont = LinLibertine_RI.otf ,
  %%         BoldItalicFont = LinLibertine_RZI.otf,
  %%         BoldSlantedFont = LinLibertine_RZ.otf,
  %%         SlantedFont    = LinLibertine_R.otf,
  %%         SlantedFeatures = {FakeSlant=0.25},
  %%         BoldSlantedFeatures = {FakeSlant=0.25},
  %%         SmallCapsFeatures = {FakeSlant=0,Numbers=OldStyle},
  %%         ]{LinLibertine_DR.otf} 



%\setCJKmainfont{LinLibertine_R.otf}


\makeatletter
\newlength\fake@f
\newlength\fake@c
\def\fakesc#1{%
  \begingroup%
  \xdef\fake@name{\csname\curr@fontshape/\f@size\endcsname}%
  \fontsize{\fontdimen8\fake@name}{\baselineskip}\selectfont%
  \uppercase{#1}%
  \endgroup%
}
\makeatother



%\setmainfont{SourceHanSerifSC-ExtraLight}\setCJKmainfont{SourceHanSerifSC-ExtraLight}  %\let\textsc\fakesc
%\setmainfont{SourceHanSerifSC-Light}\setCJKmainfont{SourceHanSerifSC-Light}  %\let\textsc\fakesc
%\setmainfont{SourceHanSerifSC-Medium}\setCJKmainfont{SourceHanSerifSC-Medium}  %\let\textsc\fakesc


%\setmainfont{cmunrm.otf}

%\setmainfont{SimSun}\setCJKmainfont{SimSun}

%\setCJKmainfont{SourceHanSerifSC-ExtraLight} zu dünn

%\setCJKmainfont{SourceHanSerifSC-Light} zu dünn

%\setCJKmainfont{SourceHanSerifSC-Medium}  nicht schön ???


%\setCJKmainfont[Scale=MatchUppercase]{SourceHanSerifSC-Regular}


%\setmainfont{SourceSerifPro-Regular}\setCJKmainfont[Scale=MatchUppercase]{SourceHanSerifSC-Regular}

% Download fonts at: https://github.com/adobe-fonts/source-han-serif
% https://github.com/adobe-fonts/source-serif-pro

\setmainfont[BoldFont={SourceSerifPro-Bold},
             Path=\fontpath, 
             ItalicFont={SourceSerifPro-LightIt},
             BoldItalicFont={SourceSerifPro-BoldIt}
             ]{SourceSerifPro-Light}
\setCJKmainfont[Scale=MatchUppercase,
                Path=\fontpath]{SourceHanSerifSC-Regular}

\newcommand{\passive}{被}
\newcommand{\passivepst}{\passive.\textsc{pst}}
\newcommand{\passiveprs}{\passive.\textsc{prs}}
\newcommand{\passiveimp}{\passive.\textsc{imp}}

