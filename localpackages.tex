%\usepackage{bigfoot}


% http://tex.stackexchange.com/questions/38607/no-room-for-a-new-dimen
\usepackage{etex}\reserveinserts{28}

% http://tex.stackexchange.com/questions/229500/tikzmark-and-xelatex
% temporary fix, remove later
%\newcount\pdftexversion \pdftexversion140 \def\pgfsysdriver{pgfsys-dvipdfm.def} \usepackage{tikz} \usetikzlibrary{tikzmark}

%\usepackage[section]{placeins}

\usepackage{eurosym} % should go once Berthold fixes unicode


% http://tex.stackexchange.com/questions/284097/subscript-like-math-but-without-the-minus-sign?noredirect=1#comment685345_284097
% for subscripts
\usepackage{amsmath}
%\usepackage{unicode-math} breaks the CCG derivations, the horizontal lines are too high


% \justify to switch of \raggedright in translations
%\usepackage{ragged2e}

% Haitao Liu
\usepackage{xeCJK}
\setCJKmainfont{SimSun}



\hypersetup{bookmarksopenlevel=0}

\iftoggle{draft}{
\usepackage{todonotes}
}{
\usepackage[disable]{todonotes}
}



\usepackage{metalogo} % xelatex

\usepackage{multicol}

\usepackage{langsci/styles/langsci-forest-setup}

\usepackage{bookmark}

\forestset{
      terminus/.style={tier=word, for children={tier=tabular}, for tree={fit=band}, for descendants={no path, align=left, l sep=0pt}},
      sn edges original/.style={for tree={parent anchor=south, child anchor=north,align=center,base=top}},
      no path/.style={edge path={}},
      set me left/.style={calign with current edge, child anchor=north west, for parent={parent anchor=south west}},
}

 
\usepackage{styles/my-ccg-ohne-colortbl}


\usepackage{langsci/styles/jambox}

\usepackage{langsci/styles/langsci-optional}



\usepackage{german}\selectlanguage{USenglish}


\usepackage[final]{epsfig}
\usepackage{graphicx}


\usepackage{styles/makros.2e,styles/article-ex,styles/additional-langsci-index-shortcuts,
styles/eng-date,styles/my-theorems}



\usepackage{lastpage,float,comment,soul,tabularx}


% loaded in macros.2e \usepackage[english]{varioref}
% do not stop and warn! This will be tested in the final version
%\vrefwarning


\usepackage{ogonek}        % For Ewa Dabrowska


\usepackage{mycommands}% \dash


% still needed
\usepackage{tikz-qtree}

\usepackage{langsci/styles/langsci-gb4e}


\usepackage{subfig}

%\renewcommand{\xbar}{X̅\xspace}


\usepackage{forest}
\useforestlibrary{linguistics}
\forestapplylibrarydefaults{linguistics}
% above are for forest version2

\usepackage{pstricks,pst-node}

%\nodemargin5pt%\treelinewidth2pt\arrowwidth6pt\arrowlength10pt
\psset{nodesep=5pt} %,linewidth=0.8pt,arrowscale=2}
\psset{linewidth=0.5pt}
\setcounter{secnumdepth}{4}


\usepackage{dgmacros,pst-tree,trees,dalrymple} % Mary Dalrymples macros


%%% trick for using adjustbox
\let\pstricksclipbox\clipbox
\let\clipbox\relax

% http://tex.stackexchange.com/questions/206728/aligning-several-forest-trees-in-centered-way/206731#206731
% for aligning TAG trees
\usepackage[export]{adjustbox}

% draw a grid for getting the coordinates
\usepackage{tikz-grid}

% for offsets in trees
\newlength{\offset}
\newlength{\offsetup}

\ifxetex
\usepackage{styles/eng-hyp-utf8}
\else
\usepackage{styles/eng-hyp}
\fi

\usepackage{appendix}


% adds lines to both the odd and even page.
% bloddy hell! This is really an alpha package! Do not use the draft option! 07.03.2016
%\usepackage{addlines}

% please ignore all calls to \addlines as for now, they are left over from the English text
\newcommand{\addlines}[1][1]{}


%% \let\addlinesold=\addlines
%% % there is one optional argument. Second element in brackets is the default 
%% \renewcommand{\addlines}[1][1]{
%% \todosatz{addlines}
%% \addlinesold[#1]
%% }


% for addlines to work
%\strictpagecheck         reinsert later when addlines is really used.



% http://tex.stackexchange.com/questions/3223/subscripts-for-primed-variables
%
% to get 
% {}[ af   [~]\sub{V} ]\sub{V$'$}
%
% typeset properly. Thanks, Sebastian.
%
\usepackage{subdepth}


%\usepackage{caption}

%% below are added by Lulu for the Chinese version of the book
\usepackage{zhnumber}
% to type­set Chi­nese rep­re­sen­ta­tions of num­bers.  

\usepackage{titlesec}  
% for Chinese style of titles of chapters and sections.
\renewcommand{\partname}{第\zhnumber{\thepart}部分}  
%\pagenumbering {zhnum}
\renewcommand{\chaptername}{第\zhnumber{\thechapter}章}  
%re\newcommand{\sectionname}{节}  
\renewcommand{\figurename}{图}  
\renewcommand{\tablename}{表}  
\renewcommand{\bibname}{参考文献}  
\renewcommand{\contentsname}{目~~~~录}  
\renewcommand{\listfigurename}{图~目~录}  
\renewcommand{\listtablename}{表~目~录}  
\renewcommand{\indexname}{索~引}  
\renewcommand{\abstractname}{\Large{摘~要}}  
\newcommand{\keywords}[1]{\\ \\ \textbf{关~键~词}:#1}  
\titleformat{\chapter}[block]{\center\Large\bf}{\chaptername}{20pt}{}  
\titleformat{\section}[block]{\large\bf}{\thesection}{10pt}{}  
\titleformat{\tableofcontents}[block]{\center\Large\bf}{\contensname}{20pt}{}  
\titleformat{\part}[block]{\center\Large\bf}{\partname}{20pt}{}  

\usepackage{titlesec}  
% for Chinese style of table of contents
%\titlecontents{\chapter}

\usepackage{indentfirst}
% In Chinese, each paragraph should be indented.
\setlength{\parindent}{2em}
% the indented space should be two characters.












